\chapter{Dynamic response of quark-gluon plasma to electromagnetic fields}\label{chap:QCD}

 In this chapter, we discuss the application of the ultrarelativistic limit of the polarization tensor in Chapter \ref{chap:PlasmaSF} to the electromagnetic properties of quark-gluon plasma (QGP), as found in \cite{Grayson:2022asf}. QGP is an extreme state of matter composed of free quarks and gluons, which occurs in the aftermath of colliding nuclei in particle accelerators and existed a few microseconds after the big bang \cite{Letessier:2002ony}. 

The electromagnetic fields generated by colliding relativistic heavy ions in particle colliders are some of the largest in the known Universe, on the order of $ec|B| \approx m_\pi^2$, but exist for very short times $t_{\text{coll}}= 2 R/\gamma \sim 10^{-25}\,\textrm{s}$ due to the Lorentz contraction of the colliding nuclei. The magnetic field generated in these collisions is interesting due to its role in separating electric charge in the QGP through the chiral magnetic effect (CME) \cite{Kharzeev:2007jp}. The electric current generated by the CME could lead to a charge separation along magnetic field lines. If a magnetic field survives in QGP until the time of hadronization of the QGP, which we will refer to as the freeze-out time $t_f$, it could also lead to a difference in the global polarization of $\Lambda$ hyperons and antihyperons \cite{Muller:2018ibh}. Charge separation in the hadron was recently studied in \cite{PhysRevX.14.011028}. 

\begin{figure}[h!]
    \centering
    \includegraphics[width=0.85\linewidth]{plots/chap02QCD/Bfield.png}
    \caption{The vacuum magnetic field for two colliding lead Pb nuclei is shown for impact parameter $b=3R$ and $\gamma =37$. (At larger Lorentz factors, a graphical representation is difficult to visualize without scaling the fields with $\gamma$). The vector potential is plotted in the collision plane, and red arrows indicate the direction of the moving nuclei. This plot mainly shows the magnetic field distribution, which is Lorentz contracted along the direction of motion. The magnetic field lines circulate out of the collision plane perpendicular to the velocity, adding together at the collision center.  }
    \label{fig:vacmag}
\end{figure}

The distribution of the vacuum magnetic field given by the Li\'enard-Wiechert fields is plotted in \reff{fig:vacmag}. This is the same magnetic field found by Lorentz boosting the Coulomb field of a nucleus at rest. We neglect the portion of the field that depends on acceleration since it is small for vacuum scattering of heavy nuclei, compared to the field that depends on velocity.

This magnetic field is treated as an external perturbation on the quark-gluon plasma, filling the overlap region between the two nuclei after they collide. For simplicity, the QGP is modeled as an infinite medium so that complications do not arise at the boundary. The temperature of QGP depends strongly on the collision energy of the nuclei. In \cite{Grayson:2022asf} we study Au+Au collisions at $\sqrt{s_{\text{NN}}}=200\,$GeV with QGP temperature $T=300$\,MeV.  After Heavy Ions collide, the conducting QGP medium generates long-range decaying tails or wakefields in the magnetic field that extend far beyond the collision time \cite{Tuchin:2010vs}. The conductivity of QGP determines the strength of these wakefields. We aim to model these fields in QGP using the formulation discussed in Chapter \ref{chap:PlasmaSF}.

\section{EM conductivity of quark-gluon plasma}

Past analytic calculations \cite{Tuchin:2010vs,Deng:2012pc,McLerran:2013hla,Tuchin:2013apa,Gursoy:2014aka,Li:2016tel,Roy:2015kma} solve Maxwell's equations in the presence of static electric conductivity 
\begin{equation}
   \sigma_0 = \frac{m_D^2}{3\kappa}\,,
\end{equation} 
in a  hydrodynamically evolving QGP. For a collisionless plasma $\kappa\rightarrow0$, the conductivity is infinite, and the medium behaves as a perfect conductor. This work introduces the frequency and wavevector dependence of the QGP analytically using the polarization tensor previously obtained in \cite{Formanek:2021blc}.

Numerical calculations \cite{Inghirami:2016iru, Inghirami:2019mkc} have incorporated the dynamical response of QGP by numerically solving the coupled magneto-hydrodynamic equations for a conducting quark-gluon plasma in the presence of the colliding nuclear charges. More recent calculations \cite{Yan:2021zjc,Wang:2021oqq} also incorporate the frequency and wave-vector dependence of QGP response to electromagnetic fields by solving the coupled Vlasov-Boltzmann--Maxwell equations numerically.



\subsection{The Ultrarelativistic EM polarization tensor in QGP}\label{sec:linresp}

In this Section, we review the ultrarelativistic polarization tensor, including damping, for the idealized case where the QGP is infinite, homogeneous, and stationary. This calculation differs from \cite{Formanek:2021blc} only in that we consider three quark species: up, down, and strange. We start with the Vlasov-Boltzmann equation for each quark flavor \req{eq:boltzmanncov} where we assume all quarks collide on a momentum-averaged time scale $\tau_{\text{rel}} = \kappa^{-1}$. The induced current $ j_{\mathrm{ind}}^\mu$ can be written in terms of the phase-space distribution of quarks and anti-quarks as
\begin{equation}\label{eq:current}
   j_{\mathrm{ind}}^\mu(x) = 2 N_c \int (dp)p^\mu \\ \times \sum_{u,d,s} q_f (f_{f}(x,p) - f_{\bar{f}}(x,p))=  4 N_Q e^2 \int (dp)p^\mu \delta f(x,p)\,,
\end{equation}
where  $N_c$ is the number of colors. We sum over the quark flavors with charges $q_f$, and in the final result we replace $q_f \delta f = \delta f_f$. The result \req{eq:current} differs from that found in the case of an electron-positron plasma by the factor
\begin{equation}
N_Q \equiv N_c\sum_f (q_f/e)^2 = 2\,,
\end{equation}
for three light quark flavors ($u,d,s$).

In the ultrarelativistic limit, neglecting quark masses, one finds the polarization functions \cite{Formanek:2021blc}:
\begin{align}\label{eq:polfuncsUltra}
&\Pi_{\parallel}(\omega,|\boldsymbol{k}|) = m_D^2\frac{\omega^2}{\boldsymbol{k}^2}\left(1 - \frac{\omega \Lambda}{2|\boldsymbol{k}|-i\kappa \Lambda}\right)\,,\\
&\Pi_{\perp}(\omega,|\boldsymbol{k}|) = \frac{m_D^2\,\omega}{4 |\boldsymbol{k}|}\left( \Lambda \left(\frac{\omega'^2}{\boldsymbol{k}^2} - 1\right) - \frac{2\omega'}{ |\boldsymbol{k}|}\right)\,,
\end{align}
where $\Lambda(\omega,\boldsymbol{k})$ is defined as
\begin{align}\label{eq:definitions}
 \Lambda \equiv \ln \frac{\omega'+  |\boldsymbol{k}|}{\omega'- |\boldsymbol{k}|}\,, \quad \text{with} \quad \omega' = \omega+i\kappa\,.
\end{align}
The parallel and transverse polarization functions have the same form as in \cite{Formanek:2021blc} except for an overall factor $N_Q$  as found in \cite{Kapusta:1992fm, Grayson:2022asf}:
\begin{equation}\label{eq:DebyemQCD}
    {m_D}^2_{(\text{EM})} = \sum_{u,d,s} q^2_f T^2 \frac{N_c}{3} = N_Q\frac{e^2T^2}{3} \equiv C_{\text{em}}T^2\,,
\end{equation}
where $C_{\text{em}} =  2e^2/3$. In the following, we will use $m_D$ as short-hand notation for the electromagnetic screening mass since we do not discuss color screening here.
The transverse conductivity $\sigma_{\perp}$, which controls the response of the plasma to magnetic fields, is related to the imaginary part of the transverse polarization function as in \req{eq:sigmaperp}



\subsection{QCD Damping rate in QGP}

The strength of the plasma response to an external magnetic field depends on the quark damping rate $\kappa$ and the electromagnetic screening mass $m_D$. The scale of the collisional quark damping $\kappa$ is much larger than the electromagnetic Debye mass $m_D$ and electromagnetic damping $\kappa_{\text{EM}}$, because it depends on the strong coupling constant $\alpha_s$, not the electromagnetic coupling $\alpha$.

In \cite{Grayson:2022asf}, we use the first-order electromagnetic Debye mass \req{eq:DebyemQCD} to estimate the electromagnetic screening mass $m_D$. The collision rate $\kappa$ is related to the inverse of the mean-free time of quarks in QGP. We adopt a value for $\kappa$ from \cite{Mrowczynski:1988xu} where the mean-free time is given by the product of the parton density in the QGP and the quark-parton transport cross-section, leading to the expression 
\begin{equation}\label{eq:kappadef}
    \kappa(T) = \frac{10}{17\pi} (9 N_f +16) \zeta(3) \alpha_s^2 \ln\left(\frac{1}{\alpha_s}\right) T\,,
\end{equation}

\begin{figure}[h!]
    \centering
    \includegraphics[width=0.85\linewidth]{plots/chap02QCD/kappaDEBYE.png}
    \caption{\textit{From \cite{Grayson:2022asf}.} Plot of the electromagnetic Debye mass and the QCD dampening rate $\kappa$ as a function of temperature. At temperature $T=300\,$MeV used here, $\kappa = 4.86\, m_D$.\label{fig:kappaDebye}}
\end{figure}

where $N_f$ is the number of flavors, $\zeta(x)$ denotes the Riemann zeta function, and $\alpha_s(T)$ is the running QCD coupling.  We model the running of the QCD coupling constant as a function of temperature in the range $T<5T_c$ using a fit provided in \cite{Letessier:2002ony}:
\begin{equation}\label{eq:alphas}
    \alpha_s(T) \approx \frac{\alpha_s(T_c)}{1+C \ln(T/T_c)}\,,
\end{equation}
where $C=0.760 \pm 0.002$. For the QCD (pseudo-)critical temperature we use $T_c = 160\,$MeV. The QED Debye mass is compared to $\kappa(T)$ in Fig.~\ref{fig:kappaDebye}. 
This is plotted along with the electromagnetic Debye mass in \reff{fig:kappaDebye}. We can expect the electromagnetic response of QGP response to be over-damped since $\kappa> \frac{2}{\sqrt{3} m_D}$ giving a plasma frequency \req{eq:plasmafreq} which is imaginary over the range of temperatures relevant for QGP.


\begin{figure}[h!]
    \centering
    \includegraphics[width=0.85\linewidth]{plots/chap02QCD/condcomp.png}
    \caption{The black line shows the static conductivity $\sigma_0$ as a function of temperature predicted by \req{eq:condstat}, which is compared to lattice results adapted from \cite{Aarts:2020dda} for $T>T_c$. The factor of $C_{\text{em}}$, defined in \req{eq:DebyemQCD}, normalizes the conductivity by the charge of the plasma constituents, such that results using different numbers of dynamical quark flavors can be compared. We indicate each set of points by its arXiv reference: blue diamonds \cite{Amato:2013naa, Aarts:2014nba}, green circles \cite{Brandt:2015aqk}, and red triangles \cite{Astrakhantsev:2019zkr}.}
    \label{fig:lattice comp}
\end{figure}


We can then use the Debye mass \req{eq:DebyemQCD} and the damping rate \req{eq:kappadef} to calculate the static conductivity \req{eq:condstat}, shown as a black line in \reff{fig:lattice comp}, which we then compare to Lattice calculations of the conductivity in QGP.


These lattice-QCD results \cite{Amato:2013naa, Aarts:2014nba,Brandt:2015aqk,Astrakhantsev:2019zkr} are scaled with temperature $T$ to remove the linear temperature dependence. We also scale the conductivity with $C_{\text{em}}$, as defined in \req{eq:DebyemQCD}, such that computations with different numbers of flavors can be compared. One can see that the conductivity value predicted by \req{eq:kappadef}, plotted in Fig.~\ref{fig:lattice comp} as a black line, lies well within the lattice-QCD results. We will use the value predicted by \reff{fig:lattice comp}, $\sigma = 5.01\,$MeV at $T=300\,$GeV, in the next section to compute the screened heavy ion fields in QGP.

% Here, we focus on the damping effect of the collision rate $\kappa$ on the induced magnetic fields in nuclear collisions. The collision term generates a nonvanishing conductivity via the imaginary part of the polarization tensor. This conductivity manifests itself in poles of the resummed propagator in the lower complex $\omega$-plane that generate long-range tails or wake fields that extend far beyond the collision time. In Table~\ref{tab:timescales} we collect the relevant timescales of the problem in ascending order.  The collision time $t_{\text{coll}}$ is much shorter than all other relevant time scales. The collective oscillations of the plasma are highly damped by the large value of the relaxation time giving rise to over-damped behavior.

% \begin{table}[h!]
% \caption{\label{tab:timescales}
% Approximate time scales relevant to the electromagnetic response of QGP for an Au+Au collisions at $\sqrt{s_{\text{NN}}}=200\,$GeV with QGP temperature $T=300$\,MeV. Time scales are shown in ascending order.}
% \begin{ruledtabular}
% \begin{tabular}{ccc}
% \textrm{Time Scale}&
% \textrm{Formula}&
% \multicolumn{1}{c}{\textrm{Time (fm/c)}}\\
% \colrule
%  \textrm{Collision Time} & $t_\text{coll} = 2 R/\gamma$ & 0.086\footnote{Calculated using the Gaussian radius $R = 4.33$\,fm defined in \req{eq:radius}.} \\
% \textrm{Relaxation Time} & $\tau_\text{rel} = 1/\kappa$ & 0.36 \\

% \textrm{Freeze-out time} & $t_f$ & $5$\footnote{Estimated using 2+1 dimensional hydrodynamic evolution \cite{Song:2007ux}.}\\
% \textrm{ Decay Time} & $t_{\sigma} = 1/\sigma_0 = \kappa/\omega_p^2$ & 59\footnote{The decay time is the large damping $\kappa/\omega_p$ expansion of the plasma oscillation frequency \req{eq:plasmafreq}. } \\
% \end{tabular}
% \end{ruledtabular}
% \end{table}



%%%%%%%%%%%%%%%%%%%%%%%%%%%%%%%%%%%%%%%%%%%%%%%%%%%%%%%%%%%%%%%%%%%%%%%%%%%%%%%%%%%%%%%%%%%%%%%%

\section{Magnetic field in QGP during a nuclear collision}\label{sec:Maxwell2}
Assuming that the QGP is an infinite homogeneous and stationary medium near equilibrium, we can solve Maxwell's equations for the self-consistent fields as in Section~\ref{sec:Maxwell}. Then the magnetic field is given by Fourier transforming the momentum space expressions given in \reqs{eq:aperp}{eq:ftfields} to position space
\begin{equation}\label{eq:magorgin}
   \boldsymbol{B}(t, z) = \int \frac{d^4k}{(2\pi)^4}  e^{-i\omega t+ik_z z}
 \frac{\mu_0 i \boldsymbol{k} \times\ft{j}_{\perp \text{ext}}(\omega, \boldsymbol{k})}{\boldsymbol{k}^2 - \omega^2 - \mu_0 \Pi_{\perp}(\omega, \boldsymbol{k})}\,.
\end{equation}
We choose the collision center as the origin of our spatial coordinate system and align the spatial $z$-axis with the beam direction. Due to the symmetry of the colliding ions, the only nonzero component of the magnetic field along the $z$-axis points out of the collision plane ($x-y$ plane). In our coordinate system used in \cite{Grayson:2022asf}, this corresponds to the $y$-component of the magnetic field. 

For ease of calculation, we specify the external 4-current using two colliding Gaussians charge distributions normalized to the nuclear rms radius $R$ and charge $Z$:
\begin{equation}\label{eq:rhoext}
\rho_{\text{ext}\pm }(t,\boldsymbol{x}) = \frac{Zq\gamma}{\pi^{3/2}R^3}e^{-\frac{1}{R^2}(x\mp b/2)^2}e^{-\frac{1}{R^2}y^2}
\times e^{-\frac{\gamma^2}{R^2}(z\mp \beta t)^2}\,,
\end{equation}
where $\gamma$ is the Lorentz factor, $\beta$ is the ratio of the ion speed to the speed of light, respectively, and $b$ is the impact parameter of the collision. The plus and minus signs indicate motion in the $\pm \hat{z}$-direction (beam-axis). This charge distribution corresponds to the vector current
\begin{equation}\label{eq:jext}
\boldsymbol{j}_{\text{ext}\pm}(t, \boldsymbol{x}) = \pm \beta \hatv{z} \rho_{\text{ext}\pm}(t, \boldsymbol{x})\,.
\end{equation}
Further details of the external charge distribution for two colliding nuclei are presented in Appendix B. of \cite{Grayson:2022asf}.

The numerical result for the position-space magnetic field found by Fourier transforming \req{eq:magorgin} using the full transverse polarization function \req{eq:polfuncsUltra} is shown as a red dashed line in Fig.~\ref{fig:bfcomp} and compared with various models of conductivity. These other models and their connections to published works are discussed in detail in \cite{Grayson:2022asf}.

\phantom{Phantom text}
\begin{figure}[h]
\centering              
\includegraphics[width=0.46\linewidth]{plots/chap02QCD/bf100.png}
%}
\hspace{0.05\linewidth}
\includegraphics[width=0.44\linewidth]{plots/chap02QCD/bf100lin.png}
%}
\caption{\textit{From \cite{Grayson:2022asf}.} The magnetic field at the collision center as a function of time, with $T = 300$\,MeV for Au-Au collisions ($Z=79$) at $\sqrt{s_\text{NN}} = 200$\,GeV and impact parameter $b = 6.4\,$fm. The left panel shows the magnetic field on a semi-logarithmic scale up to $ct = 5$\,fm. The right panel shows the early-time magnetic field on a linear scale. At the chosen temperature, the electromagnetic Debye mass is $m_D = 74\,$MeV, and the quark damping rate is $\kappa = 4.86\,m_D$. This gives a static conductivity of $\sigma_0 = 5.01\,$MeV. Comparing the different approximations, we see they all have similar asymptotic behavior. Only the Drude conductivity, the light-cone limit of the conductivity, and the full conductivity $\sigma_\perp(\omega,\boldsymbol{k})$ describe the field at early times. Note that the plasma is considered homogeneous and stationary here. In a more realistic situation, the field would become screened only after the QGP is formed in the collision.\label{fig:bfcomp}}
\end{figure}

One of the important results of this paper was that the fields of the ions, traveling near the speed of light, probe the polarization tensor along the light cone. The transverse conductivity on the light cone is
\begin{equation}\label{eq:lightcone}
    \sigma_\perp (\omega = |\boldsymbol{k}|)  =  i \frac{m_D^2}{4 \omega}\left( \frac{\kappa^2}{\omega^2} \xi \ln\xi +\frac{i\kappa}{\omega}\left(\xi+1\right)\right)\,,
\end{equation}
where $\xi$ is defined as
\begin{equation}\label{eq:xidef}
    \xi \equiv 1- 2i \frac{\omega}{\kappa}\,.
\end{equation}
The light-cone conductivity simplifies the calculation of plasma response since it only depends on a single variable ($\omega = |\boldsymbol{k}|$). One can see that \req{eq:lightcone} shown as an opaque grey line traces out the full numerical solution \req{eq:magorgin} shown as a dashed red line. The light-cone conductivity accurately models the magnetic field in QGP since the ions traveling near the light's speed only sample the polarization tensor on the light-cone. One subject of future research is to use the light-cone conductivity to attain analytical formulas for electromagnetic fields in position space in light-cone coordinates.


The simplest method to calculate the late-time magnetic field of colliding nuclei is to assume a static conductivity \cite{Tuchin:2013apa}. In this case, the magnetic field in Fourier space has the form
\begin{equation}\label{eq:bstat}
    \ft{B}(\omega,\boldsymbol{k}) = \frac{ \mu_0 i\boldsymbol{k} \times \ft{j}_{\perp \text{ext}}}{\boldsymbol{k}^2 - \omega^2 - i\omega\sigma_0}\,,
\end{equation}
which is Fourier transformed using contour integration in the appendix of \cite{Grayson:2022asf} to
\begin{equation}\label{eq:banalyticapp}
   B_y(t) = -\mu_0 \frac{ Zq \beta }{(2\pi)} \frac{ b\sigma_0}{4t^2} e^{\frac{-b^2 \sigma_0}{16 t}}\,.
\end{equation}
Looking at the left panel of Fig.~\ref{fig:bfcomp}, the static conductivity initially overestimates the magnetic field after the external field begins to disappear since the effect of dynamic screening is not captured. Every model of the response function predicts similar values for the magnetic field approaching the freeze-out time $t_f \approx 5\,$fm/c \cite{Song:2007ux}. This is because the static conductivity determines the dependence of the magnetic field at times later than $t>1/\sigma \approx 59$\,fm/c after which damping of the initial magnetic field pulse is irrelevant. 

Alternatively, by assuming a point-like charge distribution $R\rightarrow0$ and approximating the magnetic field for $ 1/\sigma_0 > t\gg 1/\kappa$ one can derive the late-time magnetic field using the Drude conductivity \req{eq:drude}
\begin{equation}\label{eq:latetimeB}
   B_y(t) \approx  \mu_0 \frac{ Ze \beta b \kappa \omega_p }{8\pi}\bigg[ \frac{1- e^{-\kappa t}}{\kappa t} - e^{-\kappa t} \text{Ei}\left(t\kappa\right)\bigg]\,.
\end{equation}
This result, derived in Appendix \ref{sec:magf}, has the advantage of accurately describing the late-time magnetic field $t>t_f$  at large $\gamma$ as shown in \reff{fig:bcolcomp}.

Both these results illustrate that the late-time magnetic field has a finite limit when $\gamma\rightarrow\infty$ as it depends only on $\beta$, but not on $\gamma$.
\begin{figure}
\centering
\includegraphics[width=0.85\linewidth]{plots/chap02QCD/bfgaamacomp.png}
    \caption{\textit{Adapted from \cite{Grayson:2022asf}.} Plot of the freeze-out magnetic field for $T= 150$\,MeV. We expect that around this temperature QGP will hadronize into a mixed phase \cite{Letessier:1992xd}. The approximate late time solution \req{eq:banalyticapp} shown as an orange dashed line is compared to numerical calculations using the full polarization tensor \req{eq:magorgin} and to the late time analytic expansion \req{eq:latetimeB}. The approximate solution does not fully match the ultrarelativistic limit until times $t > t_{\sigma} \approx 59$\,fm/c. The magnetic field is independent of the beam energy over a wide range of $\gamma$ but begins to diverge slowly from the ultrarelativistic case at around $\gamma \leq 15$ for the time window shown in the figure. Lower beam energies result in a somewhat larger field at late times.\label{fig:bcolcomp}}
\end{figure}
The approximation used to derive this solution holds for $\gamma\beta \gg \sqrt{ \kappa/\sigma_0} \approx 12$. In Fig.~\ref{fig:bcolcomp} we compare \req{eq:banalyticapp} to the full numerical result to explore its dependence on $\gamma$.  One can see that the static case \req{eq:banalyticapp} (black solid line) begins to diverge from the numerical solution, shown as dashed colored lines at around $\gamma \approx 15$.  In Fig.~\ref{fig:bcolcomp} one can see that the late-time magnetic field has a very soft dependence on collision energy. The time at which hadronization occurs $t_f$, which varies with collision energy, has a much stronger effect on the magnitude of the freeze-out field. Since the remnant magnetic field at hadronization does not depend strongly on the collision energy, an experimental measurement of the magnetic field at different collision energies could permit a determination of the electrical conductivity of the QGP or a determination of the freeze-out time of QGP if the conductivity is assumed to be known. 

As the QGP begins to hadronize at time $t_f$, one may expect hadrons to be statistically polarized with respect to the magnetic field. In \cite{Muller:2018ibh} the measured difference in global polarization of hyperons and antihyperons is used to give an upper bound on the magnetic field at QGP freeze-out, $B \sim 2.7\times 10^{-3}\,m_{\pi}^2$ for Au+Au collisions at $\sqrt{s_\text{NN}} = 200$\,GeV. Looking at Fig.~\ref{fig:bcolcomp} the magnetic field for $\gamma = 100$ at QGP freeze-out $t_f \approx 5 $\,fm/c is predicted to be $B \approx 1.2\times 10^{-3}\,m_{\pi}^2$, somewhat below this upper bound. Note that this assumes the polarization rapidly equilibrates in the plasma. It also neglects any interactions during the hadron gas phase of the collision. 
%%%%%%%%%%%%%%%%%%%%%%%%%%%%%%%%%%%%%%%%%%%%%%%%%%%%%%%%%%%%%%%%%%%%%%%%%%%%%%%%%%%%%%%%%%%%%%%%%%%%%%%%%%%%%%%%%%%%

% \section{The QGP Polarization tensor}\label{sec:linresp}
% \subsection{Derivation of the Polarization tensor}

% In this Section we derive the polarization tensor, including damping, for the idealized case where the QGP is homogeneous and stationary. We follow the derivation presented in \cite{Formanek:2021blc} for the damped polarization tensor of an electron-positron plasma. The calculation differs slightly from \cite{Formanek:2021blc}, since in QGP we consider three quark species: up, down, and strange. We start from the Vlasov-Boltzmann equation for each quark flavor:
% \begin{equation}\label{eq:VBE}
% (p \cdot \partial) f_f(x,p) + q_f F^{\mu\nu} p_\nu \frac{\partial f_f(x,p)}{\partial p^\mu} = (p\cdot u)C_f(x,p)\,,
% \end{equation}
% The collision term $C_f(x,p)$ in the BGK form is given by
% \begin{equation}\label{eq:collision}
%     C_f(x,p) =\kappa_f\left(\eq{f}_f (p)\frac{n_f(x)}{{\eq{n}_f}} - f_f(x,p)\right)\,,
% \end{equation}
% where plasma constituents collide on a momentum-averaged time scale $\tau_{\text{rel}} = \kappa^{-1}$. The collision term is constructed such that \req{eq:VBE} retains current conservation \cite{Bhatnagar:1954zz}. We show in Sect.~\ref{sec:energymomcons} that energy is also conserved for the case of a neutral particle-antiparticle plasma at linear order in the external field.

% The induced current $ j_{\mathrm{ind}}^\mu$ can be written in terms of the phase-space distribution of quarks and anti-quarks as
% \begin{multline}\label{eq:current}
%    j_{\mathrm{ind}}^\mu(x) = 2 N_c \int (dp)p^\mu \\ \times \sum_{u,d,s} q_f (f_{f}(x,p) - f_{\bar{f}}(x,p))\,,
% \end{multline}
% where  $N_c$ is the number of colors, and we sum over the quark flavors with charges $q_f$. One can calculate the induced current for small perturbations away from equilibrium for each quark flavor
% \begin{equation}\label{eq:perturbation}
% f_f(x,p) = {\eq{f}_f}(p) + \delta f_f(x,p)\,,
% \end{equation}
% Note that the equilibrium contributions ${\eq{f}_f}(p)$ do not contribute to \req{eq:current} because of the opposite sign of the charges of particles and antiparticles, but the perturbations $\delta f$ add up due the change in sign of the external force $qF^{\mu\nu}p_\nu$:
% \begin{multline}\label{eq:current2}
%     j_{\mathrm{ind}}^\mu(x) = 2 N_c \int (dp)p^\mu \sum_{u,d,s} q_f (\delta f_{f}(x,p) - \delta f_{\bar{f}}(x,p))\\
%  = 4 N_c \int (dp)p^\mu \sum_{u,d,s} q_f^2 \delta f(x,p)\\
%   =  4 N_Q e^2 \int (dp)p^\mu \delta f(x,p)\,.
% \end{multline}
% In the second line we pulled out a factor of electric charge $\delta f_{f} = q_f \delta f$. The perturbations $\delta f$ are identical for all quark species in the ultrarelativistic limit. The result \req{eq:current2} differs from that found in the case of an electron-positron plasma by the factor
% \begin{equation}
% N_Q \equiv N_c\sum_f (q_f/e)^2 = 2\,,
% \end{equation}
% where the numerical value holds for three light quarks flavors ($u,d,s$). We refer to \cite{Formanek:2021blc} for the derivation of the polarization tensor in terms of integrals over the phase-space distribution $\delta f$, because the only difference is the overall factor $N_Q$.

% As noted in the previous Section, the polarization tensor in \req{eq:poltensgen} may be written in terms of two independent components: the longitudinal polarization function $\Pi_{\parallel}$, which describes response parallel to wave-vector $\boldsymbol{k}$, and the transverse polarization function $\Pi_{\perp}$, which describes response in the plane perpendicular to wave-vector $\boldsymbol{k}$. When the $\mu=3$ ($z$) axis is chosen along the wave-vector $\boldsymbol{k}$, the longitudinal and transverse polarization functions relate to the components of the polarization tensor \req{eq:poltenmat} along the coordinate axes as
% \begin{equation}\label{eq:piLT}
%     \Pi_{\parallel} =\Pi^3_3, \quad \Pi_{\perp} =\Pi^1_1=\Pi^2_2\,.
% \end{equation}
% In the ultrarelativistic limit, neglecting quark masses, one finds \cite{Formanek:2021blc}:
% \begin{align}\label{eq:polfuncs}
% &\Pi_{\parallel}(\omega,|\boldsymbol{k}|) = m_D^2\frac{\omega^2}{\boldsymbol{k}^2}\left(1 - \frac{\omega \Lambda}{2|\boldsymbol{k}|-i\kappa \Lambda}\right)\,,\\
% &\Pi_{\perp}(\omega,|\boldsymbol{k}|) = \frac{m_D^2\,\omega}{4 |\boldsymbol{k}|}\left( \Lambda \left(\frac{\omega'^2}{\boldsymbol{k}^2} - 1\right) - \frac{2\omega'}{ |\boldsymbol{k}|}\right)\,,
% \end{align}
% where $\Lambda(\omega,\boldsymbol{k})$ is defined as
% \begin{align}\label{eq:definitions}
%  \Lambda \equiv \ln \frac{\omega'+  |\boldsymbol{k}|}{\omega'- |\boldsymbol{k}|}\,, \quad \text{with} \quad \omega' = \omega+i\kappa.
% \end{align}
% The natural logarithm leads to branch cut in the complex $\omega$ plane running from $-|\boldsymbol{k}|-i\kappa$ to  $|\boldsymbol{k}|-i\kappa$ as noted in \cite{Romatschke:2015gic}. The parallel and transverse polarization functions have the same form as in \cite{Formanek:2021blc} except for an overall factor $N_Q$ that is contained in the leading order electromagnetic Debye mass for the QGP plasma \cite{Kapusta:1992fm}:
% \begin{equation}\label{eq:DebyemQCD}
%     {m_D}^2_{(\text{EM})} = \sum_{u,d,s} q^2_f T^2 \frac{N_c}{3} = N_Q\frac{e^2T^2}{3} \equiv C_{\text{em}}T^2\,,
% \end{equation}
% where $C_{\text{em}} =  2e^2/3$. In the following we will use $m_D$ as short-hand notation for the electromagnetic screening mass since we do not discuss color screening here.

% The polarization tensor may be written in any general frame by using \req{eq:poltensgen}, but for our purposes it will be simpler to carry out calculations in the coordinate system where $\boldsymbol{k}$ aligns with the $z$-axis so that the polarization tensor takes the form shown in \req{eq:poltenmat}.


%  \subsection{QGP parameters}
 
% The strength of the plasma response to an external magnetic field depends on the values of two physical parameters: the quark damping rate $\kappa$, and the electromagnetic screening mass $m_D$. In this Section we provide estimates for these parameters. 

% We adopt the perturbative result \req{eq:DebyemQCD} to estimate $m_D$. Higher-order corrections to this expression can been derived from higher-order calculation of the vector spectral function in thermal perturbation theory (see \cite{Jackson:2019mop} and references cited therein).

% The scale of the collisional quark damping $\kappa$ is much larger than the electromagnetic Debye mass $m_D$ because it depends on the strong coupling constant $\alpha_s$, not the electromagnetic coupling $\alpha$. Solving the dispersion relation
% \begin{equation}
%     \frac{1}{(k\cdot u)^2}(k^2+ \mu_0\Pi_\parallel(\omega, k))(k^2 + \mu_0 \Pi_\perp(\omega, k))^2=0 \,,
% \end{equation}
% see \cite{melrose2008quantum}, in the limit $\boldsymbol{k}\rightarrow 0$ one finds for the plasma oscillation frequency \cite{Formanek:2021blc}
% \begin{equation}\label{eq:plasmafreq}
%     \omega_{p}^\pm = -\frac{i\kappa}{2} \pm \sqrt{\frac{m_D^2}{3} - \frac{\kappa}{4}^2}\,.
% \end{equation}
% We see that if $\kappa > \tfrac{2}{\sqrt{3}}m_D $, the plasma oscillations are over-damped.

% \begin{figure}[h!]
%     \centering
%     \includegraphics[width=0.85\linewidth]{plots/chap02QCD/kappaDEBYE.png}
%     \caption{Plot of the QED Debye mass and the QCD dampening rate $\kappa$ as a function of temperature. At temperature $T=300\,$MeV used in the plots below, $\kappa = 4.86\, m_D$.\label{fig:kappaDebye}}
% \end{figure}

% The collision rate $\kappa$ is related to the inverse of the mean-free time of quarks in QGP. In kinetic theory the mean-free time is given by the product of the parton density in the QGP and the quark-parton transport cross section, leading to the expression \cite{Mrowczynski:1988xu}
% \begin{equation}\label{eq:kappadef}
%     \kappa(T) = \frac{10}{17\pi} (9 N_f +16) \zeta(3) \alpha_s^2 \ln\left(\frac{1}{\alpha_s}\right) T\,,
% \end{equation}
% where $N_f$ is the number of flavors, $\zeta(x)$ denotes the Riemann zeta function, and $\alpha_s(T)$ is the running QCD coupling.  We model the running of the QCD coupling constant as a function of temperature in the range $T<5T_c$ using a fit provided in \cite{Letessier:2002ony}:
% \begin{equation}\label{eq:alphas}
%     \alpha_s(T) \approx \frac{\alpha_s(T_c)}{1+C \ln(T/T_c)}\,,
% \end{equation}
% where $C=0.760 \pm 0.002$. For the QCD (pseudo-)critical temperature we use $T_c = 160\,$MeV. The QED Debye mass is compared to $\kappa(T)$ in Fig.~\ref{fig:kappaDebye}. 

% From $\kappa(T)$ in \req{eq:kappadef} and the running of the coupling in \req{eq:alphas}, we calculate the static conductivity using the leading order electromagnetic Debye mass $m_D$. The momentum dependent transverse conductivity $\sigma_{\perp}$, which controls the response of the plasma to magnetic fields, is related to the imaginary part of the transverse polarization function $\Pi_{\perp}$ as follows \cite{melrose2008quantum}:
% \begin{equation}\label{eq:conddef}
%     \sigma_{\perp}(\omega,\boldsymbol{k}) = -i \frac{\Pi_{\perp}(\omega,\boldsymbol{k})}{\omega}\,.
% \end{equation}
% In the long wavelength limit $\boldsymbol{k}\rightarrow0$, the branch cut in \req{eq:definitions} shrinks to a single pole at $\omega = -i \kappa$, and the conductivity has the simple form
% \begin{equation}\label{eq:conddrude}
%     \sigma_{\perp}(\omega, 0 ) = \sigma_{\parallel}(\omega, 0 ) = \frac{\sigma_0}{1-i\omega/\kappa}\,.
% \end{equation}
% We will refer to $\sigma_{\perp}(\omega, 0 )$ as the Drude model \cite{Drude:1900}. In the static limit $\omega\rightarrow0$ the parallel and perpendicular conductivities are the same, and the static conductivity $\sigma_0$ is given by
% \begin{equation}\label{eq:condstat}
%    \sigma_0 = \frac{m_D^2}{3\kappa}\,.
% \end{equation} 
%  The static conductivity determines the late time behavior of the magnetic field.

% %%%%%%%%%%%%%%%%%%%%%%%%%%%%%%%%%%%%%%%%%%%%%%%%%%%%%%%%%%%%%%%%%%%%%%%%%%%%%%%%%%%%%%%%%

% \subsection{Energy-momentum conservation}\label{sec:energymomcons}

% In general, the modified BGK collision term \req{eq:collision} violates energy and momentum conservation. Rocha {\it et al.} \cite{Rocha:2021zcw} recently showed how energy-momentum conservation can be restored by introducing a linearized collision operator that is projected on eigenfunctions of the conserved quantities with eigenvalue zero. Here we show explicitly that for a symmetric particle-antiparticle plasma the energy momentum violations cancel at linear order in the external field. 

% Recall that the energy momentum tensor $T^{\mu\nu}$ of the plasma is given by
% \begin{equation}
%     T^{\mu \nu} = 2 \int (dp) p^{\mu} p^{\nu} (f_-(x,p) + f_+(x,p) )\,,
% \end{equation}
% where the factor of two accounts for spin and $f_\pm(x,p)$ represent the distributions of particles ($+$) and antiparticles ($-$), respectively. We recall that for $T^{\mu\nu}$ to be conserved the covariant divergence 
% \begin{equation}
%     \partial_{\mu} T^{\mu \nu} = 2\,\partial_{\mu} \int (dp)p^{\mu} p^{\nu}\left(f_-(x,p) + f_+(x,p)\right)
% \end{equation}
% must vanish. In linear response the distribution functions $ f_{\pm} (x,p)$ are given by
% \begin{equation}
%     f_{\pm} (x,p) = \delta  f_{\pm} (x,p) + \eq{f}(p)\,.
% \end{equation}
% Equation \req{eq:current2} indicates that the perturbation $ \delta  f_{\pm}$ is linear in the quark charge
% \begin{equation}
%     \delta f_\pm = \pm q\, \delta f\,.
% \end{equation}
% This leads to a cancellation of the particle and antiparticle perturbations in the energy-momentum tensor at linear order:
% \begin{equation}
%     \partial_{\mu} T^{\mu \nu} = 4\,\partial_{\mu}\left( \int (dp)p^{\mu} p^{\nu}\eq{f}(p)\right) = 0\,.
% \end{equation}
% Thus for a symmetric particle-antiparticle plasma corrections to the energy-momentum tensor appear only at second order in external field. This is a general consequence of CPT symmetry of the medium. 


% %%%%%%%%%%%%%%%%%%%%%%%%%%%%%%%%%%%%%%%%%%%%%%%%%%%%%%%%%%%%%%%%%%%%%%%%%%%%%%%%%%%%%%%%%%%%%%%%%%%


% \section{Magnetic field in a nuclear collision}\label{sec:results}

% In this Section we calculate the magnetic field at the center of the heavy ion collision by Fourier transforming the momentum space magnetic field \req{eq:ftfields} to position space. We calculate the self-consistent magnetic field using the potentials given in \reqs{eq:phi}{eq:aperp} and model the response of QGP using the idealized case of a homogeneous, stationary plasma detailed in Sects.~\ref{sec:Maxwell2} and \ref{sec:linresp}. The external fields are specified by the moving Gaussian charge distributions defined in \reqs{eq:rhoext}{eq:jext}.

% The magnetic field is of particular interest due to its role in the separation of electric charge in the QGP through the chiral magnetic effect (CME) \cite{Kharzeev:2007jp}. In the large magnetic fields that occur in heavy ion collisions the electric current generated by the CME could lead to a charge separation along the direction of the magnetic field. Whether this effect is observable depends strongly on the size of the magnetic field. If a magnetic field of meaningful strength survives until the time of hadronization of the QGP, it could also lead to a difference in the global polarization of $\Lambda$ hyperons and antihyperons \cite{Muller:2018ibh}.

% We chose the collision center as the origin of our spatial coordinate system and align the spatial $z$-axis with the beam direction. We calculate the magnetic field along the $z$-axis by Fourier transforming the momentum space expressions given in \reqs{eq:aperp}{eq:ftfields}:
% \begin{multline}\label{eq:magorgin}
%    \boldsymbol{B}(t, z) = \int \frac{d^4k}{(2\pi)^4}  e^{-i\omega t+ik_z z}
%  \frac{\mu_0 i \boldsymbol{k} \times\ft{j}_{\perp \text{ext}}(\omega, \boldsymbol{k})}{\boldsymbol{k}^2 - \omega^2 - \mu_0 \Pi_{\perp}(\omega, \boldsymbol{k})}
% \end{multline}
% to position space. It is convenient to perform the Fourier integrals in cylindrical coordinates $(\boldsymbol{x}_\perp,z)$. The angular integral $d\theta$ and the integral over momentum along the beam axis $d k_z $ can be performed exactly. The $d k_z $ integral is trivial due to the delta function in the external charge distribution \req{eq:extchgfreq}. The frequency integral $d\omega$ and the transverse momentum integral $dk_\rho$ must, in general, be done numerically. We present the details of this calculation in Appendix \ref{sec:magf}. Due to the symmetry of the colliding ions, the only nonzero component of the magnetic field along the $z$-axis points out of the collision plane ($x-y$ plane). In our coordinate system, described in Appendix \ref{sec:freechg}, this corresponds to the $y$-component of the magnetic field. The numerical results for the position-space magnetic field are shown in Fig.~\ref{fig:bfcomp} and compared with earlier results.

% To connect to these previous studies, we compute the magnetic field in position space at the origin in various levels of approximation defined in \reqs{eq:conddef}{eq:condstat} and \req{eq:lightcone}.
% \begingroup
% \renewcommand{\arraystretch}{1.5} % Default value: 1
% \begin{table}[b]
% \caption{\label{tab:cond}
% Conductivity models used to calculate the resulting magnetic field. Each conductivity represents the response of QGP with a different spacetime dependence. }
% \begin{ruledtabular}
% \begin{tabular}{ccc}
% \textrm{Conductivity}&
% \textrm{Dependence}&
% \multicolumn{1}{c}{\textrm{Formula}}\\
% \colrule
%  \textrm{Full} & $\sigma_{\perp}(\omega,\boldsymbol{k})$ & $-i \Pi_{\perp}(\omega,\boldsymbol{k})/\omega$  \\
%  \textrm{Light-cone} & $\sigma_{\perp}(\omega = |\boldsymbol{k}|)$ & \req{eq:lightcone}  \\
% \textrm{Drude} & $\sigma_{\perp}(\omega, 0 )$ & $\sigma_0/(1-i\omega/\kappa)$ \\
% \textrm{Static} & $\sigma_0$ & $m_D^2/(3\kappa)$ \\
% \end{tabular}
% \end{ruledtabular}
% \end{table}
% \endgroup
% These conductivities, collected in Table\,\ref{tab:cond}, refer to different treatments of the frequency $\omega$ and wave-vector $\boldsymbol{k}$ dependence of the conductivity $\sigma_\perp(\omega, \boldsymbol{k})$. For instance, solving for the magnetic field in the limit $\boldsymbol{k}\rightarrow0$ assumes that the spatial dependence of the external field can be neglected, not superficially a good approximation because at any given time $t$ the field varies rapidly with $z$.  The levels of approximation we consider include: the full space- and time-dependence of the conductivity $\sigma_\perp(\omega, \boldsymbol{k})$, the Drude model \req{eq:conddrude}, and the static response $\sigma_\perp(0,0)$. We list these limits in \reqs{eq:conddef}{eq:condstat}, respectively. 

% The fourth limit we are considering is the conductivity along the light-cone $\sigma_\perp(\omega,k_z=\pm\omega,k_\rho=0)$. We now show that the light-cone limit closely resembles the Drude model. We first recall that the frequency dependence of the free charge distribution in cylindrical coordinates \req{eq:extchgfreq} has the form
% \begin{multline}
% \wt{\rho}_{\text{ext}\pm}(\omega,\boldsymbol{k}) = 2\pi Zq\, e^{-(k_{\rho}^2 + k_z^2/\gamma^2)\frac{R^2}{4}} \\
% \times e^{\mp \frac{ik_{\rho} b \cos\theta }{2}} \delta(\omega \mp k_z \beta)\,.
% \end{multline} 
% After performing the Fourier transform over the parallel component of the wave-vector $k_z$ using the delta function, the magnitude of the wave-vector $|\boldsymbol{k}|$ is effectively set to the light-cone $\omega \approx |\boldsymbol{k}|$, with a small deviation due to the transverse dependence of the field,
% \begin{equation}
%     |\boldsymbol{k}|^2 = k_z^2+ k_{\rho}^2 \rightarrow  (\omega/\beta)^2+ k_{\rho}^2\,.
% \end{equation}
% Inspecting the external charge distribution after this replacement
% \begin{multline}
% \wt{\rho}_{\text{ext}\pm}(\omega,k_{\rho}) = 2\pi Zq\, e^{-(k_{\rho}^2 + \omega^2/(\beta \gamma)^2)\frac{R^2}{4}} \\
% \times e^{\mp \frac{ik_{\rho} b \cos\theta }{2}}\,,
% \end{multline}
% we can see that the size of the deviation from the light-cone due to $k_{\rho}$ is or order $O(1/R)$, while the width of the current distribution in frequency space is of order $O(\beta\gamma/R)$.
% \begin{figure}[h!]
%     \centering
%     \includegraphics[width=0.85\linewidth]{plots/chap02QCD/lightcone2.png}
%     \caption{The magnitude of the polarization tensor is plotted in momentum space showing deviations in $k_\rho$ from the light-cone $\omega = |\boldsymbol{k}|$ on the horizontal axis. The contours show lines of constant magnitude of $\Pi_\perp(\omega, |\boldsymbol{k}|)$; lighter shading indicates increasing magnitude. The dashed line encapsulates the $2\sigma$ support of the external charge distribution. The width of the external charge distribution in momentum space is $\sqrt{2}/R$ in the transverse direction and  $\beta \gamma \sqrt{2}/R$ along the light-cone. One can see that in the region sampled by the external charge distribution the polarization tensor is effectively constant as a function of $k_\rho$. \label{fig:lightcone}}
% \end{figure} 
% The region of two-sigma support of the Gaussian charge distribution is shown as the region enclosed by the dashed line in Fig.~\ref{fig:lightcone}. The polarization tensor is approximately constant as a function of $k_\rho$ in this region. This implies that one can approximate the integral in \req{eq:magorgin} using the polarization function at $k_\rho = 0$, i.~e.\, on the light-cone. This means that the fields of the ions, traveling near the speed of light, probe the polarization tensor along the light-cone. In this limit, the transverse conductivity near the light-cone is
% \begin{equation}\label{eq:lightcone}
%     \sigma_\perp (\omega = |\boldsymbol{k}|)  =  i \frac{m_D^2}{4 \omega}\left( \frac{\kappa^2}{\omega^2} \xi \ln\xi +\frac{i\kappa}{\omega}\left(\xi+1\right)\right)\,,
% \end{equation}
% where $\xi$ is defined as
% \begin{equation}\label{eq:xidef}
%     \xi \equiv 1- 2i \frac{\omega}{\kappa}\,.
% \end{equation}
% Since the light-cone conductivity only depends on a single variable ($\omega = |\boldsymbol{k}|$) it simplifies integrals involved in the Fourier transform of fields back into position space.

% Our results for the magnetic field at the collision center $B_y(t,0)$ are shown in Fig.~\ref{fig:bfcomp}. The right panel of the figure shows the field at early times ($|t| < 0.25~\text{fm}/c$) on a linear scale, the left panel shows the field over a wider time range on a logarithmic scale. The most general case $\sigma_\perp(\omega, \boldsymbol{k})$, shown as the dashed red curve in Fig.~\ref{fig:bfcomp}, includes the full time- and space-dependent response of the medium to the fields of the colliding ions. The blue dashed curve shows the magnetic field in the  Drude model approximation \req{eq:conddrude}, where the response depends only on time. The magnetic field using the light-cone conductivity is seen as the gray line overlapping the red dashed line in Fig.~\ref{fig:bfcomp},  where $\sigma_0$ is defined in \req{eq:condstat}. The result of Fourier transforming this expression is shown as the brown dotted curve in Fig.~\ref{fig:bfcomp}. Our results differ slightly from those of \cite{Tuchin:2013apa} because here we account for the finite size of the ions and use a slightly different conductivity value. 

% \phantom{Phantom text}
% \begin{figure}[htb]
% \centering
% \includegraphics[width=0.46\linewidth]{plots/chap02QCD/bf100.png}
% %}
% \hspace{0.05\linewidth}
% \includegraphics[width=0.44\linewidth]{plots/chap02QCD/bf100lin.png}
% %}
% \caption{The magnetic field at the collision center as a function of time, with $T = 300$\,MeV for a Au-Au collisions ($Z=79$) at $\sqrt{s_\text{NN}} = 200$\,GeV and impact parameter $b = 6.4\,$fm. The left panel shows the magnetic field on a semi-logarithmic scale up to $ct = 5$\,fm. The right panel shows the early-time magnetic field on a linear scale. At the chosen temperature the electromagnetic Debye mass is $m_D = 74\,$MeV and the quark damping rate is $\kappa = 4.86\,m_D$. This gives a static conductivity of $\sigma_0 = 5.01\,$MeV. Comparing the different approximations we see that all of them have similar asymptotic behavior. Only the Drude conductivity, the light-cone limit of the conductivity, and the full conductivity $\sigma_\perp(\omega,\boldsymbol{k})$ describe the field at early times. Note that here that the plasma is considered homogeneous and stationary. In a more realistic situation the field would become screened only after the QGP is formed in the collision.\label{fig:bfcomp}}
% \end{figure}

% The magnetic field in the presence of a QGP was previously calculated using a static conductivity in \cite{Tuchin:2013apa}. In this case, the magnetic field in Fourier space has the form
% \begin{equation}\label{eq:bstat}
%     \ft{B}(\omega,\boldsymbol{k}) = \frac{ \mu_0 i\boldsymbol{k} \times \ft{j}_{\perp \text{ext}}}{\boldsymbol{k}^2 - \omega^2 - i\omega\sigma_0}\,,
% \end{equation}

% Looking at the left panel of Fig.~\ref{fig:bfcomp}, one can see that every model of the response function predicts similar values for the magnetic field approaching the freeze-out time $t_f$. This is because the static conductivity determines the late-time dependence of the magnetic field. As we discuss in Appendix \ref{sec:magf}, we can expect the static solution to match the full solution when $t > 1/\kappa$. The static conductivity initially overestimates the magnetic field after the external field begins to fall, since the effect of dynamic screening is not captured. This matches the qualitative picture given by the detailed numerical transport calculation done in \cite{Wang:2021oqq}. The full space-time dependent model and the Drude model model behave similarly for most times, and are almost identical for $t>1/\kappa \approx 0.36$\,fm/c. The magnetic field calculated using the polarization tensor evaluated on the light cone tracks the general solution at all times.

% We can use the light-cone conductivity in \req{eq:lightcone} to understand why the Drude model $\sigma_\perp(\omega, 0)$ matches the full solution for times $t>1/\kappa$. Late times probe the small frequency limit of the conductivity. An expansion of \req{eq:lightcone} in $\omega/\kappa$ yields
% \begin{multline}
% \sigma_\perp (\omega = |\boldsymbol{k}|)   = \sigma_0\left(1+i\omega/\kappa\right)\\
% - \frac{6 \sigma_0}{5 }\frac{\omega^2}{\kappa^2}+O\left(\frac{\omega^3}{\kappa^3}\right)\,.
% \end{multline}
% We then compare to the same expansion for the Drude conductivity
% \begin{multline}
% \sigma_\perp (\omega,0) = \frac{\sigma_0}{1- i\omega/\kappa}  \approx  \sigma_0\left(1+i\omega/\kappa\right) \\
% - \sigma_0 \frac{\omega^2}{\kappa^2}+O\left(\frac{\omega^3}{\kappa^3}\right)\,.
% \end{multline}
% The lowest-order term, which coincides with the expression for the Drude model, closely approximates the full solution when $\kappa \gg \omega$ as shown in Fig~\ref{fig:condlightcomp}. Since $\kappa t_f \gg 1$ for the QGP, the series converges rapidly for times of the order of the freeze-out time $t_f$. 
% \begin{figure}[h!]
%     \centering
%     \includegraphics[width=0.85\linewidth]{plots/chap02QCD/condlightcomp.png}
%     \caption{Comparison of the conductivity on the light-cone to $ \sigma_{\perp}(\omega,\boldsymbol{k}\rightarrow 0) $, scaled with the static conductivity. We see that at small $\omega/\kappa$, i.e. times much larger than $1/\kappa$, both approximations converge to the static case, while they diverge $\omega/\kappa > 1$. This predicts that the Drude model will underestimate screening at small times, which is exactly what we observe in Fig.~\protect\ref{fig:bfcomp}. \label{fig:condlightcomp}}
% \end{figure} 

% The simple form of the Drude approximation \req{eq:conddrude} allows one to find the poles of the denominator in \req{eq:magorgin}, analytically. The frequency integral can then be done using the residue theorem, allowing for an approximate analytical expression for the late-time magnetic field. This is done in Appendix \ref{sec:magf}. 
% In the ultrarelativistic limit $\gamma\gg 1$ and large times $t\gg 1/\kappa$ gives
% \begin{equation}\label{eq:banalyticapp}
%    B_y(t) = -\mu_0 \frac{ Zq \beta }{(2\pi)} \frac{ b\sigma_0}{4t^2} e^{\frac{-b^2 \sigma_0}{16 t}}\,,
% \end{equation}
% This result differs from the ``diffusive'' solution of Tuchin \cite{Tuchin:2013apa} by a factor 1/4 in the exponent, due to their convention for impact parameter $b\rightarrow2b$. The reason why $\kappa$ does not appear in the expression \req{eq:banalyticapp} for the late-time magnetic field lies in the hierarchy of time scales $t_\text{coll} \ll 1/\kappa \ll t_f$, which makes plasma damping irrelevant during the spike of the external field as well as at freeze-out.

% Interestingly, this solution has a finite limit when $\gamma\rightarrow\infty$ as it depends only on $\beta$, but not on $\gamma$. This property, which was first observed by Tuchin \cite{Tuchin:2013apa}, can be understood as follows: For late times the Fourier integral of \req{eq:extchgfreq} is dominated by contributions from small frequencies $\omega$, and it is sufficient to consider the $\omega \rightarrow 0$ limit of the Fourier spectrum of the external charge distributions $\wt{\rho}_{f\pm}$ given in \req{eq:By}. In this limit \req{eq:extchgfreq} takes the form
% \begin{equation}
% \wt{\rho}_{\text{ext}\pm}(0,\boldsymbol{k}) \rightarrow 2\pi Ze\, e^{-k_\rho^2R^2/4} e^{\mp \frac{i k_\rho b \cos \theta }{2}} \delta(k_z \beta)\,,
% \end{equation} 
% which is independent of $\gamma$. This occurs because
% \begin{equation}
% \wt{\rho}_{\text{ext}\pm}(0,\boldsymbol{k}) = \int dt \int d^3x e^{-\boldsymbol{k}\cdot\boldsymbol{x}} \rho_{\text{ext}\pm}(0,\boldsymbol{x})
% \end{equation}
% integrates over the passage of the entire nucleus at a given location $\boldsymbol{x}$ and thus is independent of $\gamma$ as the total charge is Lorentz invariant. We conclude that, quite generally, for high collision energies the remnant magnetic field at late times is determined by the time-integrated action of the external electromagnetic pulse on the QGP. In a more realistic calculation, where the QGP is not present for the entire duration of the pulse, because it is created during the collision, the remnant magnetic field will be diminished as only a fraction of the pulse acts on the QGP. We therefore expect our result to represent an upper bound to the late-time magnetic field in a realistic collision scenario.

% \begin{figure}
%     \centering
% \includegraphics[width=0.95\linewidth]{plots/chap02QCD/bfgaamacomp.png}
%     \caption{Plot of the freeze-out magnetic field for $T= 150$\,MeV. We expect that around this temperature QGP will hadronize into a mixed phase \cite{Letessier:1992xd}. The approximate late time solution \req{eq:banalyticapp} shown as an orange dashed line is compared to numerical calculations using the full polarization tensor \req{eq:magorgin}. the approximate solution does not fully match the ultrarelativistic limit until times $t > t_{\sigma} \approx 59$\,fm/c.  The The magnetic field is independent of the beam energy over a wide range of $\gamma$ but begins to diverge slowly from the ultrarelativistic case at around $\gamma \leq 15$ for the time window shown in the figure. Lower beam energies result in a somewhat larger field at late time.\label{fig:bcolcomp}}
% \end{figure}

% The approximation used to derive this solution holds for $\gamma\beta \gg \sqrt{ \kappa/\sigma_0} \approx 12$. In Fig.~\ref{fig:bcolcomp} we compare \req{eq:banastat} to the full numerical result to explore its dependence on $\gamma$.  One can see that the ultrarelativistic case (black solid line) begins to diverge from the numerical solution at around $\gamma \approx 15$ for the times shown. The early time magnetic field is not shown because the initial temperature of QGP will depend strongly on the collision energy. The times are chosen such that they cover the range of freeze-out times predicted for QGP for the range of experimental collision energies used \cite{Bass:2000ib}. We do not show curves for $\gamma<10$ because we expect the effects of chemical potential will become important, yet here chemical potential $\mu$ is set to zero. In Fig.~\ref{fig:bcolcomp} one can see that the late-time magnetic field has a very soft dependence on collision energy. The time at which the magnetic field freezes out, which varies with collision energy, has a much stronger effect on the magnitude of the freeze-out field.

% As the QGP begins to hadronize at time $t_f$, one may expect hadrons to be statistically polarized with respect to the magnetic field. In \cite{Muller:2018ibh} the measured difference in global polarization of hyperons and antihyperons is used to give an upper bound on the magnetic field at QGP freeze-out, $B \sim 2.7\times 10^{-3}\,m_{\pi}^2$ for Au+Au collisions at $\sqrt{s_\text{NN}} = 200$\,GeV. Looking at Fig.~\ref{fig:bcolcomp} the magnetic field for $\gamma = 100$ at QGP freeze-out $t_f \approx 5 $\,fm/c is predicted to be $B \approx 1.2\times 10^{-3}\,m_{\pi}^2$, somewhat below this upper bound. Note that this assumes the polarization  rapidly equilibriates in the plasma. It also neglects any interactions during the hadron gas phase of the collision. 

% \begin{figure}
% \vskip 16pt
%     \centering
% \includegraphics[width=0.85\linewidth]{plots/chap02QCD/kappacomp.png}
%     \caption{Comparison of the magnetic field for different values of quark damping rate or, equivalently, electric conductivity. Larger values of  the damping rate $\kappa$ represent smaller conductivities and vice versa as indicated by \req{eq:condstat}. The black dashed line and the solid black line represent the limits of zero and infinite conductivity, respectively. One can see that as $\kappa$ increases the asymptotic value of the magnetic field decreases.\label{fig:kappacomp}}
    
% \end{figure}

% In Fig.~\ref{fig:kappacomp} we look at the magnetic field at the origin for different values of $\kappa$. Increasing $\kappa$ reduces the static conductivity $\sigma_0$ which decreases the asymptotic value of the magnetic field as indicated by \req{eq:banalyticapp}. As $\kappa$ goes to zero the results converge to the case of ideal conductivity  $\sigma_0 \rightarrow \infty$ where the magnetic field quickly approaches a constant value. This case was studied in \cite{Deng:2012pc} where the authors considered a magnetic field that falls to a constant value and then decreases with $1/t$ due to Bjorken flow. More recent calculations \cite{Yan:2021zjc,Wang:2021oqq} solve the Vlasov-Boltzmann equation numerically with parton-parton scattering. The magnetic field predicted by \cite{Yan:2021zjc} is around $\sim 10^{-4} m_\pi^2$ after $t\approx 2$\,fm/c, which is two orders of magnitude lower than the value found here (see Fig.~\ref{fig:bcolcomp}). However, the magnetic field predicted by \cite{Wang:2021oqq} is around $\sim 10^{-2} m_\pi^2$ after $t\approx 2$\,fm/c, which is in agreement with our model.

% In Fig.~\ref{fig:lighfield} we show a space-time contour plot of the magnetic field. The field at the higher collision energy (on the left) has a higher peak magnetic field. For lower collision energy (on the right) the field is less Lorentz contracted, and leads to a magnetic field at late times that is a factor of $\sim1.1$ larger. The freeze-out magnetic field will increase at lower collision energy mainly due to the decreasing freeze-out time.


% \phantom{Phantom text}

% \begin{figure}[H]
% \centering
% \includegraphics[width=0.45\linewidth]{plots/chap02QCD/lightBfplot.png}
% %}
% \hspace{0.05\linewidth}
% \includegraphics[width=0.45\linewidth]{plots/chap02QCD/lightBfplot10.png}
% %}
% \caption{Space-time plot of the magnetic field on the beam axis ($x=y=0$) in eternal (pre-existent) QGP with $T = 300$\,MeV for a Au-Au collision at impact parameter $b = 6.4\,$fm. Left panel: collision energy $\sqrt{s_\text{NN}} = 200$\,GeV; right panel: collision energy $\sqrt{s_\text{NN}} = 17$\,GeV. The same value of $\kappa$ is used as in Fig.~\ref{fig:bfcomp}. In a more realistic scenario, where the QGP is formed during the collision, the field would only create induced currents in the upper light-cone.\label{fig:lighfield}}
% \end{figure}


%%%%%%%%%%%%%%%%%%%%%%%%%%%%%%%%%%%%%%%%%%%%%%%%%%%%%%%%%%%%%%%%%%%%%%%%%%%%%%%%%%%%%%%%%%%%%%%%%%%%%%%%%%%%%%%%%%%%%%%%

\section{Towards a more realistic QGP: discussion and outlook }\label{sec:ConclusionsQGP}
The work reviewed here and presented in Appendix \ref{appendixB} calculates the magnetic field of two colliding nuclei in a stationary, homogeneous QGP using relativistic kinetic theory with collisional damping. Our first main finding in \cite{Grayson:2022asf} was that the response to the external magnetic field is controlled by the polarization function along the light-cone, $\Pi^\mu_\nu(\omega ,|\boldsymbol{k}|\approx\omega)$. This allowed us to derive an approximate analytic solution for the magnetic field that considers the dynamics of the medium's response. We also discussed how the late-time magnetic field at hadronization does not depend strongly on the collision energy. This gives the possibility that an experimental measurement of the magnetic field at different collision energies could permit a determination of the electrical conductivity of the QGP \cite{PhysRevX.14.011028}. We must also know how the freeze-out time depends on collision energy to make this measurement.

\subsection{The QGP medium}
This calculation can be improved in numerous ways. One of our main interests is to incorporate a finite size and a time-dependent onset in the QGP medium, which we describe here as infinite and homogenous. Boundary effects at the QGP surface are likely crucial for many collisions since the Debye sphere is not much smaller than the size of QGP, or similarly, the skin depth is probably large in comparison to the radius of QGP. Plasma skin effects could lead to novel electromagnetic phenomena at the QGP surface. We have begun some work on implementing an initial onset and formation time for QGP in the Vlasov-Boltzmann equation, effectively creating a boundary in time. This work should be extendable to studying plasma with a finite boundary in space which could be interesting with respect to the study of surface plasmons.

QGP is also not stationary; peripheral heavy ion collisions are one of the most highly rotational systems ever observed \cite{PhysRevC.87.034906, PhysRevC.93.064907,PhysRevC.94.044910, doi:10.1146/annurev-nucl-021920-095245}. This is due to the huge angular momentum of the colliding system. This rotation can be incorporated into the equilibrium distribution \cite{Hakim2011}, which creates a temperature that depends on radius \cite{chernikov1964equilibrium} changing our description of the magnetic field.

In \cite{Grayson:2022asf} it would have been simple to use the adiabatic expansion of a relativistic ideal gas \cite{Bjorken:1982qr} to parameterize the temperature dependence as a function of time. To reduce the number of free parameters, we found the magnetic field at large times by simply assuming the plasma temperature was the freeze-out temperature \reff{fig:bcolcomp}.

Many enhancements can be made that require numerical solutions of the linear response equations, Such improvements would include a realistic space-time dependence of the medium (formation and hydrodynamical evolution), nonzero net baryon density, quark thermal mass corrections \cite{PhysRevD.26.2789}, and viscous corrections to the unperturbed phase-space distribution used to calculate the polarization tensor.

\subsection{Electric field in QGP}

\phantom{Phantom text}
\begin{figure}[h!]
\centering
\includegraphics[width=0.45\linewidth]{plots/chap02QCD/Eyy.png}
%}
\hspace{0.05\linewidth}
\includegraphics[width=0.45\linewidth]{plots/chap02QCD/Ezz.png}
%}
\caption{Plots comparing the electric field in vacuum, shown as a black dashed line, to the electric field in QGP shown as the red points. The left plot shows the transverse electric field screened by the plasma. The plot on the right shows the electric field in the direction of motion enhanced by the plasma. We choose $T = 300$\,MeV and $Z=79$, for Au-AU collisions at $\sqrt{s} = 200$\,GeV at an impact parameter of half nuclear overlap $b = 1 R = 6.4\,$fm. The vertical line in the left plot indicates $ y = R$, approximately the transverse size of QGP. \label{fig:efcomp}}
\end{figure}



Of course, we could have also studied electric fields in QGP which are in the same order as the magnetic fields $e|E| \approx m_\pi^2$. These fields are of interest in strong field QED since they are far beyond the Schwinger limit $e|E| \approx m_e^2$. Preliminary QGP electric field calculations are shown in 
\reff{fig:efcomp}. In QGP, the transverse electric field $E_y$ is screened while the eclectic field is enhanced in the direction of motion. The electric field is also interesting since it could do a significant amount of work on the QGP possibly reheating it after its formation through ohmic heating. 

Additionally, we were interested in studying the distribution of electric charge around relativistic heavy nuclei in QGP. This can be found by Fourier transforming \req{eq:indch} for the external charge distribution \req{eq:rhoext}. The induced charge density for a single traveling nucleus at low $\gamma$ is shown in \reff{fig:efcomp}. The external charge distribution increases with the Lorentz factor $\gamma$, but the total induced charge, which is the integral of the red dashed line, remains constant but trails behind further at larger velocities.

\phantom{Phantom text}
\begin{figure}[h!]
\centering
\includegraphics[width=0.45\linewidth]{plots/chap02QCD/indchg12.png}
%}
\hspace{0.05\linewidth}
\includegraphics[width=0.45\linewidth]{plots/chap02QCD/indchg5.png}
%}
\caption{The external (black), induced (red dashed), and total charge density (blue dashed) for a single nucleus traveling in the $+\boldsymbol{\hat{z}}$ direction at $\gamma = 1.2$ on the left and $\gamma = 5$ on the right. The induced charge distribution trails behind the nuclei. The external charge density increases with $\gamma$. The induced charge distribution trails behind the nuclei more for larger $\gamma$. \label{fig:potcomp}}
\end{figure}

As seen in \reff{fig:potcomp}, a wakefield of induced charge forms behind the traveling nucleus in QGP. In \reff{fig:chgwake}, we show a two-dimensional contour plot of the charged wake. The wakefield depicted in \reff{fig:chgwake} is damped at traverse distances instead of conical as in the collisionless case.
\begin{figure}[h!]
\centering
\includegraphics[width=0.85\linewidth]{plots/chap02QCD/chwake.png}
%}
\caption{2D plot of the wake field of a single traveling gold nucleus $\gamma = 5$ in QGP. The blue arrow indicates the direction of motion and the grey disk represents the Lorentz contracted nucleus. Lines of constant charge density are shown as contours. \label{fig:chgwake}}
\end{figure}


The Electromagnetic polarization tensor in QGP also has applicability in cosmology, where a QGP existed during the first $10~\mu$s of the early Universe. In the next chapter, we will study somewhat later times a few seconds after the Big Bang, when the universe was filled with electron-positron plasma. In these situations, the assumption of homogeneity and stationary of the medium on the scale of the relevant parameters, $m_D$, and $\kappa$, is well justified.





% %===================================================================
% %===================================APPENDICES======================

% %===================================================================
% \section{Electric current of two colliding nuclei}\label{sec:freechg}

% Here we define the free charge and current density used to describe heavy ion collisions. We wish to model two nuclei moving at constant velocity $\pm \beta$ along the collision axis ($\hat{z}$ direction) that are offset by $\pm b/2$ within the collision plane ($\hat{x}$ direction).  For simplicity we model the charge distribution as a gaussian in all directions
% \begin{multline}
% \rho_{\text{ext}\pm }(t,\boldsymbol{x}) = \frac{Ze\gamma}{\pi^{3/2}R^3}e^{-\frac{1}{R^2}(x\mp b/2)^2}e^{-\frac{1}{R^2}y^2}\\
% \times e^{-\frac{\gamma^2}{R^2}(z\mp \beta t)^2}\,,
% \end{multline}
% where the normalization is chosen in such a way that 
% \begin{equation}
% \int \rho_{\text{ext}\pm}(t,\boldsymbol{x}) d^3\boldsymbol{x} = Ze\,,
% \end{equation}
% is the total charge of the heavy ion nucleus and $\gamma$ is the usual relativistic factor. The Gaussian radius parameter $R$ is related to the mean squared radius of the nucleus, $\langle r^2\rangle$ at rest, ($\gamma = 1$) by
% \begin{equation}\label{eq:radius}
% \langle r^2 \rangle = \frac{1}{Ze}\int r^2 \rho_{\text{ext}\pm}(\boldsymbol{x}) d^3\boldsymbol{x} = \frac{3}{2}R^2\,,
% \end{equation}
% which is measured experimentally for a gold nucleus to be $\sqrt{\langle r^2 \rangle} \approx 5.30 \,$fm \cite{DeVries:1987atn} .

% At time $t = 0$ both nuclei are localized at the $z = 0$ plane and we assume that before and after the collision they continue moving on a straight line along the $z$-axis. The gaussian form of the charge distributions allows us to evaluate the Fourier transformations easily. The transforms in the transverse directions are
% \begin{align}
% \int_{-\infty}^\infty dy \ e^{-ik_y y}e^{-y^2/R^2} &= R \sqrt{\pi} e^{-k_y^2 R^2/4}\,, \\ 
% \int_{-\infty}^\infty \ dx e^{-ik_x x}e^{-(x\mp b/2)^2/R^2} &=  R \sqrt{\pi}e^{-k_x^2 R^2/4}e^{\pm i k_x b /2}\,.
% \end{align}
% The last two integrals are a bit more complicated because they are coupled
% \begin{multline}
% \int_{-\infty}^\infty e^{i\omega t} \left(\int_{-\infty}^\infty e^{-ik_z z}e^{- \frac{\gamma^2}{R^2}(z^2 \pm 2z\beta t)} dz \right)\\
% \times e^{-\frac{\gamma^2}{R^2}\beta^2t^2}dct = \frac{R \sqrt{\pi}}{\gamma}e^{-\frac{k_z^2R^2}{4\gamma^2}} \int_{-\infty}^\infty e^{i(\omega \pm k_z \beta)t}dt\\
% = \frac{2 R\pi^{3/2}}{\gamma}e^{-\frac{k_z^2R^2}{4\gamma^2}} \delta(\omega \pm k_z \beta)\,,
% \end{multline}
% where delta function appears because both nuclei move at a constant velocity. Altogether the Fourier transformed charge distributions are 
% \begin{multline}
% \wt{\rho}_{\text{ext}\pm}(\omega,\boldsymbol{k}) = 2\pi Ze\, e^{-(k_x^2 + k_y^2 + k_z^2/\gamma^2)\frac{R^2}{4}} \\
% \times e^{\mp \frac{i k_x b}{2}} \delta(\omega \mp k_z \beta)\,,
% \end{multline} 
% which may be written in cylindrical coordinates
% \begin{multline}\label{eq:extchgfreq}
% \wt{\rho}_{\text{ext}\pm}(\omega,\boldsymbol{k}) = 2\pi Ze\, e^{-(k_\rho^2 + k_z^2/\gamma^2)\frac{R^2}{4}} \\
% \times e^{\mp \frac{i k_\rho b \cos \theta }{2}} \delta(\omega \mp k_z \beta)\,.
% \end{multline} 
% The current densities are obtained from
% \begin{equation}\label{eq:extcurrent}
% \ft{j}_{\text{ext}\pm}(\omega, \boldsymbol{k}) = \pm \beta \hatv{z} \wt{\rho}_{\text{ext}\pm}(\omega, \boldsymbol{k})\,.
% \end{equation}
% The transverse component of the current given by
% \begin{multline}\label{eq:jperpext}
% \ft{j}_{\perp,\text{ext}} = \ft{j}_\text{ext} - (\hatv{k} \cdot \ft{j}_\text{ext}) \hatv{k} \\= (\hatv{z} - \hat{k}_z \hatv{k})\beta(\wt{\rho}_{\text{ext}+} - \wt{\rho}_{\text{ext}-})\,.
% \end{multline}

% %%%%%%%%%%%%%%%%%%%%%%%%%%%%%%%%%%%%%%%%%%%%%%%%%%%%%%%%%%%%%%%%%%%

% \section{Magnetic field at the collision center}\label{sec:magf}

% The magnetic field inside the plasma is given in Fourier space by
% \begin{equation}\label{eq:magapp}
%    \ft{B} = i \boldsymbol{k} \times \wt{\boldsymbol{A}} = i \boldsymbol{k} \times \wt{\boldsymbol{A}}_{\perp}\,,
% \end{equation}
% where the potential $\wt{\boldsymbol{A}}$ has been projected into components transverse and longitudinal to $\boldsymbol{k}$. In the following we represent the wave-vector in cylindrical coordinates $\boldsymbol{k} = (k_{\rho} \cos \theta,k_{\rho}\sin \theta, k_{z}) $. Using the expression for the self-consistent vector potential \req{eq:aperp} we find for the magnetic field,
% \begin{equation}
%    \ft{B} =
%  \frac{\mu_0 i \boldsymbol{k} \times\ft{j}_{\perp \text{ext}}}{\boldsymbol{k}^2 - \omega^2 - \mu_0 \Pi_{\perp}}\,.
% \end{equation}
% Given the definition of $\ft{j}_{\perp,\text{ext}}$ in \req{eq:jperpext} we can replace the perpendicular component of the current by its full form, adding the $\pm$ components of \req{eq:extcurrent}:
% \begin{equation}
%    \ft{B} =
%  \frac{\mu_0 i \boldsymbol{k} \times\ft{j}_{\text{ext}}}{\boldsymbol{k}^2 - \omega^2 - \mu_0 \Pi_{\perp}}\,.
% \end{equation}
% We now Fourier transform this quantity back to position space in order to calculate the magnetic field at the collision center as a function of time. Due to symmetry the only nonvanishing component of the magnetic field at this location will be the $y$-component:
% \begin{equation}\label{eq:By}
%    \wt{B}_y = 
%  \mu_0 \frac{ i k_x \beta (\wt{\rho}_{\text{ext}-}-\wt{\rho}_{\text{ext}+})}{\boldsymbol{k}^2 - \omega^2 - \mu_0 \Pi_{\perp}}\,.
% \end{equation}
% The Fourier transform at any point along the collision axis ($x=y=0$) is given by
% \begin{equation}
%   B_y(z,t) =  \int \frac{d^4k}{(2\pi)^4}  e^{-i\omega t+ ik_z z} \wt{B}_y(\omega, \boldsymbol{k})\,.
% \end{equation}
% In cylindrical coordinates the integral can be written as
% \begin{multline}
%   B_y(z,t) =  \frac{1}{(2\pi)^4}\int k_{\rho}dk_{\rho} d\omega dk_z d\theta  e^{-i\omega t+ ik_z z}\\
%  \mu_0 \frac{ i \beta k_{\rho} \cos \theta (\wt{\rho}_{f-}-\wt{\rho}_{f+})}{\boldsymbol{k}^2 - \omega^2 - \mu_0 \Pi_{\perp}(\omega, |\boldsymbol{k}|)}\,.
% \end{multline}
% We can use the the delta function in the Fourier transformed current \req{eq:extchgfreq} to trivially perform the $k_z$ integral:
% \begin{multline}
%    B_y(z,t) = -\mu_0 \frac{ Ze \beta }{(2\pi)^3}  \int dk_{\rho} d\omega d\theta \  e^{-i\omega t} \\ \frac{ 2 k_{\rho}^2  \cos \theta \sin \left(  k_{\rho} \cos \theta \frac{b}{2} - \frac{\omega z}{\beta} \right)
%   e^{-(k_{\rho}^2+ \omega^2/(\beta\gamma)^2)\frac{R^2}{4}}}{\omega^2/(\gamma\beta)^2 + k_{\rho}^2  - \mu_0 \Pi_{\perp}\left(\omega, \sqrt{k_{\rho}^2 + \omega^2/\beta^2} \right)}\,.
% \end{multline}
% We next perform the angular integration
%  \begin{multline}\label{eq:intnum}
%    B_y(z,t) = -\mu_0 \frac{ Ze \beta }{(2\pi)^2}  \int dk_{\rho} d\omega \  e^{-i\omega t} \\ \frac{ 2 k_{\rho}^2 J_1 \left(\frac{k_{\rho} b}{2} \right) \cos \left( \frac{\omega z}{\beta}\right)
%   e^{-(k_{\rho}^2+ \omega^2/(\beta\gamma)^2)\frac{R^2}{4}}}{\omega^2/(\gamma\beta)^2 + k_{\rho}^2  - \mu_0 \Pi_{\perp}\left(\omega, \sqrt{k_{\rho}^2 + \omega^2/\beta^2} \right)}\,,
% \end{multline}
% where $J_1$ is a Bessel function of the first kind. The remaining integrals have to be performed numerically. From here forward we pull the factor of $\mu_0$ into the Debye mass $m_D^2$, such that the factor of $e^2$ goes to $4 \pi \alpha$ in \req{eq:DebyemQCD}.

% We can obtain an analytical expression for the Drude approximation \req{eq:conddrude} in the limit $\gamma\beta \gg \sqrt{ \kappa/\sigma_0}$, which is valid when $\gamma \gg 12$ for the values of $\sigma_0$ and $\kappa$ adopted here.  In this limit we can neglect the first term in the denominator of \req{eq:intnum}, which now takes the simple form
% \begin{equation}
%       k_\rho^2 - i \omega \frac{\omega_p^2}{\kappa - i\omega}\,.
% \end{equation}
% Note that using \req{eq:condstat} we can see $\omega_p^2 = m_D^2/3 = \kappa \sigma_0 $. The integrand of \req{eq:intnum} then has a single pole at 
% \begin{equation}
% \omega = -i\frac{k_\rho^2\kappa}{k_\rho^2+\omega_p^2}\,,
% \end{equation}
% and the frequency integral can be performed by contour integration in the lower complex plane. Consistently neglecting the term proportional to $\omega^2/(\beta\gamma)^2$ in the exponent, the integration yields:
% \begin{multline}\label{eq:bintfull}
%    B_y(z,t) \approx - \mu_0 \frac{ Ze \beta }{2\pi}  \int dk_{\rho}\, 2\kappa k_{\rho}^2\\
%     \frac{\omega_p^2}{(k_\rho^2+\omega_p^2)^2} 
%     J_1 \left(\frac{k_{\rho} b}{2} \right) e^{-k_{\rho}^2R^2/4} \\
%    \cosh \left( \frac{k_\rho^2\kappa}{k_\rho^2+\omega_p^2} \frac{z}{\beta}\right)
%    \exp \left( - \frac{k_\rho^2\kappa t}{k_\rho^2+\omega_p^2} \right)\,.
% \end{multline}
% For late times $t$ the exponential factor only samples the small $k_\rho$ region of the integrand ($k_\rho^2 < \sigma_0/t$). We can then neglect $k_\rho$ with respect to $ \omega_p$ in the integrand provided that $\sigma_0/t \ll \omega_p^2$, which is satisfied when $t \gg 1/\kappa = t_{\text{rel}} \approx 1\,\text{fm}/c$. The expression then takes the simplified form:
% yielding
% \begin{multline}\label{eq:bappint}
%    B_y(z,t) \approx - \mu_0 \frac{ Ze \beta }{2\pi}  \int dk_{\rho}\, \frac{2k_{\rho}^2}{\sigma_0}
%    J_1 \left(\frac{k_{\rho} b}{2} \right) \\
%    e^{-k_{\rho}^2R^2/4} \cosh \left( \frac{k_\rho^2}{\sigma_0} \frac{z}{\beta} \right) e^{-k_\rho^2 t/\sigma_0}\,.
% \end{multline}
% The integral over $k_\rho$ can now be performed analytically resulting in:
% \begin{equation}
%    B_y(z,t) \approx - \mu_0 \frac{ Ze \beta }{2\pi}  \frac{b}{8\sigma_0}
%    \left[ \frac{e^{-\frac{b^2}{16L_+}}}{L_+^2} + \frac{e^{-\frac{b^2}{16L_-}}}{L_-^2} \right]
% \end{equation}
% with 
% \begin{equation}
% L_\pm = \frac{R^2}{4} + \frac{t\pm z/\beta}{\sigma_0} \,.
% \end{equation}
% At the collision center $(z=0)$ and for $t \gg \sigma_0R^2/4$ our result simplifies to
% \begin{equation}\label{eq:banastat}
%    B_y(0,t) \approx - \mu_0 \frac{ Ze \beta }{2\pi}  \frac{b\sigma_0}{4t^2} e^{-\frac{\sigma_0b^2}{16t}}\,.
% \end{equation}

% This result differs from Tuchin's \cite{Tuchin:2013apa} by a factor 1/4 in the exponent due to their convention for impact parameter $b\rightarrow2b$.




 

%%%%%%%%%%%%%%%%%%%%%%%%%%%%%%%%%%%%%%%%%%%%%%%%
%%%%%%%%%%%%%%%%%%%%%%%%%%%%%%%%%%%%%%%%%%%%%%%%

% \begin{thebibliography}{99}

% %%%%%%%%%%%%%%%%%  Analytic - Constant Conductivity 

%     \bibitem{Tuchin:2010vs}
%     K.~Tuchin,
%     %``Synchrotron radiation by fast fermions in heavy-ion collisions,''
%     Phys. Rev. C \textbf{82}, 034904 (2010)
%     [erratum: Phys. Rev. C \textbf{83} (2011), 039903]
%     % doi:10.1103/PhysRevC.83.039903
%     [arXiv:1006.3051 [nucl-th]].
    
%     \bibitem{Deng:2012pc}
%     W.~T.~Deng and X.~G.~Huang,
%     %``Event-by-event generation of electromagnetic fields in heavy-ion collisions,''
%     Phys. Rev. C \textbf{85}, 044907 (2012)
%     % doi:10.1103/PhysRevC.85.044907
%     [arXiv:1201.5108 [nucl-th]].
    
%     \bibitem{Tuchin:2013apa}
%     K.~Tuchin,
%     %``Time and space dependence of the electromagnetic field in relativistic heavy-ion collisions,''
%     Phys. Rev. C \textbf{88}, 024911 (2013) 
%     % doi:10.1103/PhysRevC.88.024911
%      [arXiv:1305.5806 [hep-ph]].
    
%     \bibitem{McLerran:2013hla}
%     L.~McLerran and V.~Skokov,
%     %``Comments About the Electromagnetic Field in Heavy-Ion Collisions,''
%     Nucl. Phys. A \textbf{929}, 184 (2014)
%     % doi:10.1016/j.nuclphysa.2014.05.008
%     [arXiv:1305.0774 [hep-ph]].
    
%     \bibitem{Gursoy:2014aka}
%     U.~Gursoy, D.~Kharzeev and K.~Rajagopal,
%     %``Magnetohydrodynamics, charged currents and directed flow in heavy ion collisions,''
%     Phys. Rev. C \textbf{89}, 054905 (2014) 
%     % doi:10.1103/PhysRevC.89.054905
%     [arXiv:1401.3805 [hep-ph]].

%     \bibitem{Roy:2015kma}
%     V.~Roy, S.~Pu, L.~Rezzolla and D.~Rischke,
%     %``Analytic Bjorken flow in one-dimensional relativistic magnetohydrodynamics,''
%     Phys. Lett. B \textbf{750}, 45 (2015)
%     % doi:10.1016/j.physletb.2015.08.046
%     [arXiv:1506.06620 [nucl-th]].
    
%     \bibitem{Li:2016tel}
%     H.~Li, X.~l.~Sheng and Q.~Wang,
%     %``Electromagnetic fields with electric and chiral magnetic conductivities in heavy ion collisions,''
%     Phys. Rev. C \textbf{94}, 044903 (2016)
%     % doi:10.1103/PhysRevC.94.044903
%     [arXiv:1602.02223 [nucl-th]].

% %%%%%%%%%%%%%%%%%  Numerical (MDH) - Constant Conductivity 

%     \bibitem{Inghirami:2016iru}
%     G.~Inghirami, L.~Del Zanna, A.~Beraudo, M.~H.~Moghaddam, F.~Becattini and M.~Bleicher,
%     %``Numerical magneto-hydrodynamics for relativistic nuclear collisions,''
%     Eur. Phys. J. C \textbf{76}, 659 (2016)
%     % doi:10.1140/epjc/s10052-016-4516-8
%     [arXiv:1609.03042 [hep-ph]].
    
%     \bibitem{Inghirami:2019mkc}
%     G.~Inghirami, M.~Mace, Y.~Hirono, L.~Del Zanna, D.~E.~Kharzeev and M.~Bleicher,
%     %``Magnetic fields in heavy ion collisions: flow and charge transport,''
%     Eur. Phys. J. C \textbf{80}, 293 (2020)
%     % doi:10.1140/epjc/s10052-020-7847-4
%     [arXiv:1908.07605 [hep-ph]].
    
% %%%%%%%%%%%%%%%%%  Numerical (MDH) - Dynamic Conductivity 
 
%     \bibitem{Yan:2021zjc}
%     L.~Yan and X.~G.~Huang,
%     %``Dynamical evolution of magnetic field in the pre-equilibrium quark-gluon plasma,''
%     [arXiv:2104.00831 [nucl-th]].
    
%     \bibitem{Wang:2021oqq}
%     Z.~Wang, J.~Zhao, C.~Greiner, Z.~Xu and P.~Zhuang,
%     %``Incomplete electromagnetic response of hot QCD matter,''
%     [arXiv:2110.14302 [hep-ph]].
    
% %%%%%%%%%%%%%%%%%%%%%%%%%%%%%%%%%%%%%%%%%%%%%%%%%%%%%%%%%%%%%%%%%%%%%%%%%%%

%     \bibitem{Formanek:2021blc}
%     M.~Formanek, C.~Grayson, J.~Rafelski and B.~M\"uller,
%     %``Current-conserving relativistic linear response for collisional plasmas,''
%     Annals Phys. \textbf{434}, 168605 (2021)
%     % doi:10.1016/j.aop.2021.168605
%      [arXiv:2105.07897 [physics.plasm-ph]].
    
%     \bibitem{Florkowski:2017olj}
%     W.~Florkowski, M.~P.~Heller and M.~Spalinski,
%     %``New theories of relativistic hydrodynamics in the LHC era,''
%     Rept. Prog. Phys. \textbf{81}, 046001 (2018)
%     % doi:10.1088/1361-6633/aaa091
%     [arXiv:1707.02282 [hep-ph]].
    
%     \bibitem{Rocha:2021zcw}
%     G.~S.~Rocha, G.~S.~Denicol and J.~Noronha,
%     %``Novel Relaxation Time Approximation to the Relativistic Boltzmann Equation,''
%     Phys. Rev. Lett. \textbf{127}, 042301 (2021)
%     % doi:10.1103/PhysRevLett.127.042301
%     [arXiv:2103.07489 [nucl-th]].

%     \bibitem{Bhatnagar:1954zz}
% 	P.~L.~Bhatnagar, E.~P.~Gross and M.~Krook,
% 	%``A Model for Collision Processes in Gases. 1. Small Amplitude Processes in Charged and Neutral One-Component Systems,''
% 	Phys. Rev. \textbf{94}, 511 (1954).
% 	%doi:10.1103/PhysRev.94.511
	
%     \bibitem{Song:2007ux}
%     H.~Song and U.~W.~Heinz,
%     %``Causal viscous hydrodynamics in 2+1 dimensions for relativistic heavy-ion collisions,''
%     Phys. Rev. C \textbf{77}, 064901 (2008)
%     % doi:10.1103/PhysRevC.77.064901
%     [arXiv:0712.3715 [nucl-th]].
    
%     \bibitem{Starke:2014tfa}
% 	R.~Starke and G.~A.~H.~Schober,
% 	%\lq\lq Relativistic covariance of Ohm's law,\rq\rq
% 	Int. J. Mod. Phys. D \textbf{25}, 1640010 (2016)
% 	%doi:10.1142/S0218271816400101
% 	[arXiv:1409.3723 [math-ph]].
	
%     \bibitem{Weldon:1982aq}
% 	H.~A.~Weldon,
% 	%\lq\lq Covariant Calculations at Finite Temperature: The Relativistic Plasma,\rq\rq
% 	Phys. Rev. D \textbf{26}, 1394 (1982).
% 	%doi:10.1103/PhysRevD.26.1394
	
% 	\bibitem{melrose2008quantum}
% 	D.~Melrose,
% 	{\it Quantum Plasmadynamics: Unmagnetized Plasmas},
% 	Lect. Notes Phys. \textbf{735} 
% 	(Springer, New York, 2008).
% 	%doi:10.1007/978-0-387-73903-8 

%     \bibitem{Romatschke:2015gic}
%     P.~Romatschke,
%     %``Retarded correlators in kinetic theory: branch cuts, poles and hydrodynamic onset transitions,''
%     Eur. Phys. J. C \textbf{76}, 352 (2016)
%     % doi:10.1140/epjc/s10052-016-4169-7
%     [arXiv:1512.02641 [hep-th]].
    
%    \bibitem{Kapusta:1992fm}
%     J.~I.~Kapusta,
%     %``Screening of static QED electric fields in hot QCD,''
%     Phys. Rev. D \textbf{46}, 4749 (1992)
%     % doi:10.1103/PhysRevD.46.4749
    
%     \bibitem{Jackson:2019mop}
%     G.~Jackson,
%     %``Two-loop thermal spectral functions with general kinematics,''
%     Phys. Rev. D \textbf{100}, 116019 (2019)
%     %doi:10.1103/PhysRevD.100.116019
%     [arXiv:1910.07552 [hep-ph]].
    
%     \bibitem{Mrowczynski:1988xu}
%     S.~Mr\'owczy\'nski,
%     %``On the Transport Coefficients of a Quark Plasma,''
%     Acta Phys. Polon. B \textbf{19}, 91 (1988).
	
%     \bibitem{Rafelski:2002}
%     J.~Letessier, and  J.~Rafelski,
%     {\it Hadrons and Quark-Gluon Plasma} 
%     % (Cambridge Monographs on Particle Physics, Nuclear Physics and Cosmology). 
%    ( Cambridge University Press, 2002)
%     %doi:10.1017/CBO9780511534997
    
%     \bibitem{Drude:1900}
% 	P.~Drude,
% 	Ann.\ Phys.\ (Leipzig) \textbf{1}, 566 (1900).
	
% 	\bibitem{Aarts:2020dda}
%     G.~Aarts and A.~Nikolaev,
%     %``Electrical conductivity of the quark-gluon plasma: perspective from lattice QCD,''
%     Eur. Phys. J. A \textbf{57}, 118 (2021)
%     % doi:10.1140/epja/s10050-021-00436-5
%      [arXiv:2008.12326 [hep-lat]].
    
%     %\cite{Brandt:2012jc}
%     \bibitem{Brandt:2012jc}
%     B.~B.~Brandt, A.~Francis, H.~B.~Meyer and H.~Wittig,
%     %``Thermal Correlators in the \textbackslash{}rho\textbackslash{} channel of two-flavor QCD,''
%     JHEP \textbf{03} (2013), 100
%     % doi:10.1007/JHEP03(2013)100
%     % [arXiv:1212.4200 [hep-lat]].
%     %103 citations counted in INSPIRE as of 10 Jan 2022
    
%     %\cite{Amato:2013naa}
%     \bibitem{Amato:2013naa}
%     A.~Amato, G.~Aarts, C.~Allton, P.~Giudice, S.~Hands and J.~I.~Skullerud,
%     %``Electrical conductivity of the quark-gluon plasma across the deconfinement transition,''
%     Phys. Rev. Lett. \textbf{111}, 172001 (2013)
%     % doi:10.1103/PhysRevLett.111.172001
%      [arXiv:1307.6763 [hep-lat]].
%     %182 citations counted in INSPIRE as of 10 Jan 2022
    
%     %\cite{Aarts:2014nba}
%     \bibitem{Aarts:2014nba}
%     G.~Aarts, C.~Allton, A.~Amato, P.~Giudice, S.~Hands and J.~I.~Skullerud,
%     %``Electrical conductivity and charge diffusion in thermal QCD from the lattice,''
%     JHEP \textbf{02}, 186 (2015)
%     % doi:10.1007/JHEP02(2015)186
%      [arXiv:1412.6411 [hep-lat]].
%     %168 citations counted in INSPIRE as of 10 Jan 2022
    
%     %\cite{Brandt:2015aqk}
%     \bibitem{Brandt:2015aqk}
%     B.~B.~Brandt, A.~Francis, B.~J\"ager and H.~B.~Meyer,
%     %``Charge transport and vector meson dissociation across the thermal phase transition in lattice QCD with two light quark flavors,''
%     Phys. Rev. D \textbf{93}, 054510 (2016) 
%     % doi:10.1103/PhysRevD.93.054510
%     [arXiv:1512.07249 [hep-lat]].
%     %53 citations counted in INSPIRE as of 10 Jan 2022
    
%     %\cite{Astrakhantsev:2019zkr}
%     \bibitem{Astrakhantsev:2019zkr}
%     N.~Astrakhantsev, V.~V.~Braguta, M.~D'Elia, A.~Y.~Kotov, A.~A.~Nikolaev and F.~Sanfilippo,
%     %``Lattice study of the electromagnetic conductivity of the quark-gluon plasma in an external magnetic field,''
%     Phys. Rev. D \textbf{102}, 054516 (2020) 
%     % doi:10.1103/PhysRevD.102.054516
%      [arXiv:1910.08516 [hep-lat]].
%     %32 citations counted in INSPIRE as of 10 Jan 2022
    
%     %\cite{Greif:2014oia}
%     \bibitem{Greif:2014oia}
%     M.~Greif, I.~Bouras, C.~Greiner and Z.~Xu,
%     %``Electric conductivity of the quark-gluon plasma investigated using a perturbative QCD based parton cascade,''
%     Phys. Rev. D \textbf{90}, 094014 (2014) 
%     % doi:10.1103/PhysRevD.90.094014
%     [arXiv:1408.7049 [nucl-th]].
%     %89 citations counted in INSPIRE as of 10 Jan 2022
    
    
%     \bibitem{Satow:2014lia}
% 	D.~Satow,
% 	%\lq\lq Nonlinear electromagnetic response in quark-gluon plasma,\rq\rq
% 	Phys. Rev. D \textbf{90}, 034018 (2014)
% 	%doi:10.1103/PhysRevD.90.034018
% 	[arXiv:1406.7032 [hep-ph]].
    
%     \bibitem{Kharzeev:2007jp}
%     D.~E.~Kharzeev, L.~D.~McLerran and H.~J.~Warringa,
%     %``The Effects of topological charge change in heavy ion collisions: 'Event by event P and CP violation',''
%     Nucl. Phys. A \textbf{803}, 227 (2008)
%     % doi:10.1016/j.nuclphysa.2008.02.298
%     [arXiv:0711.0950 [hep-ph]].
    
%     \bibitem{Bass:2000ib}
%     S.~A.~Bass and A.~Dumitru,
%     %``Dynamics of hot bulk QCD matter: From the quark gluon plasma to hadronic freezeout,''
%     Phys. Rev. C \textbf{61}, 064909 (2000)
%     % doi:10.1103/PhysRevC.61.064909
%     [arXiv:nucl-th/0001033 [nucl-th]].
    
%     \bibitem{Muller:2018ibh}
%     B.~M\"uller and A.~Sch\"afer,
%     %``Chiral magnetic effect and an experimental bound on the late time magnetic field strength,''
%     Phys. Rev. D \textbf{98}, 071902 (2018) 
%     % doi:10.1103/PhysRevD.98.071902
%     [arXiv:1806.10907 [hep-ph]].
    
%     \bibitem{Letessier:1992xd}
%     J.~Letessier, A.~Tounsi, U.~W.~Heinz, J.~Sollfrank and J.~Rafelski,
%     %``Evidence for a high entropy phase in nuclear collisions,''
%     Phys. Rev. Lett. \textbf{70}, 3530 (1993)
%     % doi:10.1103/PhysRevLett.70.3530
%     [arXiv:hep-ph/9711349 [hep-ph]].



%   %\cite{Bjorken:1982qr}
%     \bibitem{Bjorken:1982qr}
%     J.~D.~Bjorken,
%     %``Highly Relativistic Nucleus-Nucleus Collisions: The Central Rapidity Region,''
%     Phys. Rev. D \textbf{27} (1983), 140-151
%     doi:10.1103/PhysRevD.27.140
%     %3331 citations counted in INSPIRE as of 03 Feb 2022
    
%     %\cite{Song:2010mg}
%     \bibitem{Song:2010mg}
%     H.~Song, S.~A.~Bass, U.~Heinz, T.~Hirano and C.~Shen,
%     %``200 A GeV Au+Au collisions serve a nearly perfect quark-gluon liquid,''
%     Phys. Rev. Lett. \textbf{106} (2011), 192301
%     [erratum: Phys. Rev. Lett. \textbf{109} (2012), 139904]
%     doi:10.1103/PhysRevLett.106.192301
%     [arXiv:1011.2783 [nucl-th]].
%     %453 citations counted in INSPIRE as of 17 Feb 2022
    
%     %\cite{Stewart:2021mjz}
%     \bibitem{Stewart:2021mjz}
%     E.~Stewart and K.~Tuchin,
%     %``Continuous evolution of electromagnetic field in heavy-ion collisions,''
%     Nucl. Phys. A \textbf{1016}, 122308 (2021)
%     % doi:10.1016/j.nuclphysa.2021.122308
%     [arXiv:2106.09124 [nucl-th]].
%     %3 citations counted in INSPIRE as of 12 Jan 2022
    
%     \bibitem{DeVries:1987atn}
%     H.~De Vries, C.~W.~De Jager and C.~De Vries,
%     %``Nuclear charge and magnetization density distribution parameters from elastic electron scattering,''
%     Atom. Data Nucl. Data Tabl. \textbf{36}, 495 (1987)
%     % doi:10.1016/0092-640X(87)90013-1
    
%     \end{thebibliography}
    
    
% % %%%%%%%%%%%%%%%

%%%%%%%%%%%%%%%%%



%%%%%%%%%%%%%%%%%%%%%%% 
