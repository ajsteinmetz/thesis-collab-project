In this chapter, we will introduce the fundamental concepts in cosmology for us to explore the properties of the Universe during the `first hour'. I will first present the standard cosmological Friedmann-Lemaitre-Robertson-Walker (FLRW) model, then introduce the general Fermi/Bose distribution with and its application in the early Universe. Finally I present an overview of Universe evolution from $300\,\mathrm{MeV}>T>0.02\,\mathrm{MeV}$.
The Natural unit $c=\hbar=k_{B}=1$ is used throughout the thesis for discussion.
\section{Textbook review: the standard FLRW-Universe model}
%In this section we will focus on the following:
%\begin{itemize}
%    \item The Robertson –Walker Universe
%    \item The Friedmann equation (Hubble %    \item The composition of the universe
%\end{itemize}
The Friedmann-Lemaitre-Robertson-Walker (FLRW) Universe is a theoretical model used widely to describe the cosmological evolution of the Universe. It is based on the cosmological principles which assumes homogeneity and isotropy of the Universe on large scales. In general, the FLRW metric can be written as
\begin{align}\label{metric}
ds^2=c^2dt^2-a^2(t)\left[ \frac{dr^2}{1-kr^2}+r^2(d\theta^2+\sin^2\theta\,d\phi^2)\right].
\end{align}
The metric is characterized by the scale factor $a(t)$ which measures the size of the Universe as a function of time $t$. The geometric parameter $k$ identifies the Gaussian geometry of the spatial hyper-surfaces defined by co-moving observers. The metrics are qualitatively different depending on the value of $k$. We have $k=1$ which correspond to the closed Universe,  $k=0$ correspond to flat Universe, and $k=-1$ for open geometries of the Universe. Current observation of cosmic microwave background (CMB) anisotropy preferred value $k=0$~[\cite{Planck:2013pxb,Planck:2015fie,Planck:2018vyg}].