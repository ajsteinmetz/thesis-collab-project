\title{Particles and Plasma in Cosmological Evolution}

\author{
Jeremiah Birrell\inst{1}\fnmsep\thanks{\email{jeremey.birrell@gmail.com}},
Christopher Grayson\inst{1}\fnmsep\thanks{\email{chrisgray1044@arizona.edu}},
Johann Rafelski\inst{1}\fnmsep\thanks{\email{johannr@arizona.edu}}
Andrew Steinmetz\inst{1,2}\fnmsep\thanks{\email{ajsteinmetz@arizona.edu}},
Cheng Tao Yang\inst{1}\fnmsep\thanks{\email{chengtaoyang@arizona.edu}},
}

\institute{Department of Physics, The University of Arizona, Tucson, AZ, 85721, USA
\and Arizona College of Technology, Hebei University of Technology, Tianjin 300130, China
}

\abstract{
% Cheng Tao's Abstract
This work aims to deepen the understanding of the primordial composition of the Universe in the temperature range $300\,\mathrm{MeV}>T>0.02\,\mathrm{MeV}$. In the following I exploit known properties of elementary particles and apply methods of kinetic theory and statistical physics to advance the understanding of  the  cosmic plasma. Within the Big Bang model the Universe began as a highly energetic fireball with an immensely high temperature and energy density. Consequently, an ultra-relativistic plasma was generated, exhibiting distinct properties as the Universe expanded and cooled.  When the Universe is hot and dense, fundamental particles (such as quarks, leptons, and gauge bosons) play a crucial role in understanding the early Universe. These elementary particles were abundantly present once the temperature dropped below $T=130$\,GeV. Their interactions governed the dynamics of the early Universe. Our research focuses on investigation of these fundamental particles during the epoch which transits from primordial quark-gluon degrees of freedom to the era of normal matter plasma (H$^+$, He$^{+}$, $e^-$). Our findings will offer valuable insights into the properties of the early Universe governing the properties of matter surrounding us today.
% Jeremey's Abstract
% Chris's Abstract
% Andrew's Abstract
An interesting application of these theoretical developments is to study primordial magnetization in the early universe during the hot dense electron-positron plasma epoch. We propose a model of magnetic thermal matter-antimatter plasmas. We analyze the paramagnetic characteristics of electron-positron plasma when exposed to an external primordial field. We determine the magnitude of a small polarization asymmetry sufficient to generate field strengths in agreement with those measured today in deep intergalactic space.
} 
\maketitle

