%%%%%%%%%%%%%%%%%%%%%%%% Springer-Verlag
\documentclass[epjST,usenames,dvipsnames,vecphys]{svjour}

\usepackage{amsmath,amssymb}
\usepackage{float}
\usepackage{hyperref}
\usepackage[framemethod=TikZ]{mdframed}
\usepackage{footnote}
%\usepackage{natbib}
\usepackage{comment}
\usepackage{doi}
\usepackage{enumitem}
\usepackage{slashed}
\usepackage{xcolor}
\usepackage{graphicx}
%\usepackage{deluxetable} % Allows use of AASTEX deluxe tables
\usepackage{aastex_hack} % Allows other AASTEX functionality.
\usepackage{appendix}
\makesavenoteenv{tabular}
\makesavenoteenv{table}
\makesavenoteenv{figure}
%\makesavenoteenv{mdframed}
%\usepackage[usenames,dvipsnames]{xcolor}


% Useful macros for equations and units
\newcommand*{\TeV}{\text{ TeV}}
\newcommand*{\GeV}{\text{ GeV}}
\newcommand*{\MeV}{\text{ MeV}}
\newcommand*{\keV}{\text{ keV}}
\newcommand*{\eV}{\text{ eV}}
\newcommand*{\meV}{\text{ meV}}
\newcommand*{\Msun}{\mathrm{M}_{\odot}}
\newcommand*{\bb}{\boldsymbol}
\newcommand*{\beqn}{\begin{equation}}
\newcommand*{\eeqn}{\end{equation}}
\newcommand{\req}[1]{Eq.~(\ref{#1})}
\newcommand{\rf}[1]{Fig.~{\ref{#1}}}
\newcommand{\rt}[1]{Table~{\ref{#1}}}
\newcommand{\rsec}[1]{Sec.~{\ref{#1}}}
\newcommand{\rchap}[1]{Ch.~{\ref{#1}}}
\newcommand{\rapp}[1]{Appendix~{\ref{#1}}}
\newcommand{\mydoi}[2]{\href{http://dx.doi.org/#2}{#1}}
\newcommand{\E}{\mathrm{e}}
\newcommand{\ie}{{\em i.e.}} % i.e.
\newcommand{\ms}[1]{\rule[-#1mm]{0mm}{0mm}}
\newcommand{\grad}{\operatorname{grad}}
\newcommand{\diag}{\mathrm{diag}}
\newcommand{\cccite}[1]{\textit{Published in \cite{#1} under Creative Commons}}
% Write a command which takes in two references and says something like: "Published in [#] under CC-BY-4.0 and adapted from [#].
% Chris's commands
% Extra brackets to be removed.
% Question marks in section 8.
% Change figure captions and message Jan.
\newcommand{\reff}[1]{Figure\,({\ref{#1}})}
\newcommand{\wt}[1]{\widetilde{#1}}
\newcommand{\eq}[1]{#1^{(\mathrm{eq})}}
\newcommand{\peq}[1]{#1^{'(\mathrm{eq})}}
\newcommand{\reqs}[2]{Eqs.\,({\ref{#1}}-{\ref{#2}})}
\newcommand{\mk}{|\boldsymbol{k}|}
\newcommand{\ft}[1]{\widetilde{\boldsymbol{#1}}}
\newcommand{\hatv}[1]{\hat{\boldsymbol{#1}}}
\newcommand{\bom}[1]{{\color{blue}#1}}
\newcommand{\jr}[1]{{\color{red}#1}}
\newcommand{\cg}[1]{{\color{magenta}#1}}
\newcommand{\mf}[1]{{\color{cyan}#1}}

%\renewcommand*\contentsname{\thesubsection\ Contents}

% Struts for tables 
\newcommand\Tstrut{\rule{0pt}{2.6ex}}
\newcommand\Bstrut{\rule[-0.9ex]{0pt}{0pt}}
\newcommand{\TBstrut}{\Tstrut\Bstrut}

\numberwithin{equation}{section}

\begin{document}
%%%%%%%%%%%%%%%%%%%%%%%%
\title{Quarks to Cosmos: Particles and Plasma in\\ 
 Cosmological Evolution}

\author{
Jeremiah Birrell\inst{1}, %\fnmsep\thanks{\email{jeremey.birrell@gmail.com}},
%Shelbi J. Foster\inst{1}, %\fnmsep\thanks{\email{shelbijfoster@arizona.edu}}, 
Christopher Grayson\inst{1}, %\fnmsep\thanks{\email{chrisgray1044@arizona.edu}},
Johann Rafelski\inst{1}%
\fnmsep\thanks{Corresponding author%\email{rafelski@gmail.com}}, 
\email{johannr@arizona.edu}}, 
%\fnmsep\thanks{\email{johannr@arizona.edu}}
\newline Andrew Steinmetz\inst{1}, %\fnmsep\thanks{\email{ajsteinmetz@arizona.edu}}
Cheng Tao Yang\inst{1}
%\fnmsep\thanks{\email{chengtaoyang@arizona.edu}}
}

\institute{Department of Physics, The University of Arizona, Tucson, AZ, 85721, USA
}

\abstract{We describe in the context of the particle physics (PP) standard model (SM) `PP-SM' the understanding of the primordial properties and composition of the Universe in the temperature range $130\,\mathrm{GeV}>T>20\,\mathrm{keV}$. The Universe evolution is described using FLWR cosmology.
%model combined with $\Lambda\mathrm{CDM}$ ($\Lambda$=Einstein cosmological term dark energy + Cold Dark Matter) time evolution, most if not all of the here presented results will remain valid should this framework undergo further refinement. This is so since 
In the considered temperature range the unknown cold dark matter and dark energy content of $\Lambda\mathrm{CDM}$ have a negligible influence allowing a reliable understanding of physical properties of the Universe based on PP-SM energy-momentum alone.  We follow the arrow of time in the expanding and cooling Universe: After the PP-SM heavies $(t, H, W, Z)$ diminish in abundance below $T\simeq 50$\,GeV, the PP-SM plasma in the Universe is governed by the strongly interacting Quark-Gluon content. Once the temperature drops below $T\simeq 150$\,MeV, quarks and gluons hadronize into strongly interacting matter particles comprising a dense baryon-antibaryon content. Rapid disappearance of baryonic antimatter ensues, which adopting the present day photon-to-baryon ratio completes at $T_\mathrm{B}=38.2$\,MeV. We study the ensuing disappearance of strangeness and mesons in general. We show that the different eras defined by particle populations are barely separated from each other with abundance of muons fading out just prior to $T=\mathcal{O}(2.5)$\,MeV, the era of emergence of the free-streaming neutrinos. We develop methods allowing the study of the ensuing speed of the Universe expansion as a function of fundamental coupling parameters in the primordial epoch. We discuss the two relevant fundamental constants controlling the decoupling of neutrinos. We subsequently follow the early Universe as it passes through the hot dense electron-positron plasma epoch. The high density of positron antimatter disappears near $T=20.3$\,keV, well after the Big Bang nucleosynthesis era: Nuclear reactions occur in the presence of a highly mobile and relatively strongly interacting electron-positron plasma phase.  We apply plasma theory methods to describe the strong screening effects between heavy dust particle (nucleons). We analyze the paramagnetic characteristics of the electron-positron plasma when exposed to an external primordial magnetic field.}
\maketitle
%%%%%%%%%%%%%%%%%%%%%%%%%%%%%%%%%%%%%%%
%\setcounter{\thesubsection}{1}
%\stepcounter{\thesubsection}
\newpage
%\setcounter{secnumdepth}{3}
\setcounter{tocdepth}{2}
\tableofcontents
%%%%%%%%%%%%%%%%%%%%%%%%%%%%%%%%%%%%%%%

%%%%%%%%%%%%%%%%%%%%%%%%%%%%%%%%%%%%%%% 
\section{Introduction}
%%%%%%%%%%%%%%%%%%%%%%%%%%%%%
\subsection{Universe evolution time arrow}\label{ssec:UniLab}
\subsubsection{Theoretical models of the primordial Universe}
\paragraph{Connecting prior works:}
In this report we explore the connection between particle, nuclear, and plasma physics and the evolution of the Universe in the domain described by laws of physics we learned about in laboratory experiments. However, this report is not a traditional review: We collect here in heavily edited and re-sequenced manner selected material from the contents of four Ph.D. Thesis completed at the Department of Physics, The University of Arizona by:
\begin{enumerate}
\item Jeremiah Birrell~\cite{Birrell:2014ona},
\item Christopher M. Grayson~\cite{Grayson:2024okq},
\item Andrew J. Steinmetz~\cite{Steinmetz:2023ucp}, and
\item Cheng Tao Yang~\cite{Yang:2024ret}.
\end{enumerate}
Due to graduation time constraints some of this presented material is only found in follow-up publications which are cited. Also included is:
\begin{enumerate}
\item[5.] Material from the prior work of one of us (JR) which describes our understanding about QGP and hadronic matter in the context of the early Universe.
\end{enumerate}
It is our hope that this collection of material allows the reader to obtain a smooth connection in the entire spplicable temperature domain $130\,\mathrm{GeV}>T>20\,\mathrm{keV}$.

\paragraph{Dominance of visible matter:}
In this report, we aim to connect various eras of cosmological evolution which can be addressed with some confidence in view of the already known particle and nuclear properties as measured experimentally. By analyzing the early Universe as a function of time we are exploring the role of particle physics standard model (PP-SM) in the Universe evolution.  We snapshot in this report specific epochs in primordial Universe, or/and on specific physical conditions such as primordial magnetic fields.

In the cosmic epoch considered here with temperature above $kT=20$\,keV the present day dominant dark matter and dark energy played a negligible role in the cosmos. There are two unknown dark component as one is able to disentangle their behavior given two inputs, pressure and energy density which are related by equations of state.  

Dark energy in conventional definition is akin to $\Lambda$=Einstein's cosmological term. $\Lambda$ is a fixed property of the Universe and does not scale with temperature. In comparison radiation energy content scales with $T^4$ and is vastly dominant in the temperature range we explore. Cold {\it i.e.\/} dark matter (CDM) content scales with $T^{3/2}$ for $m/T\gg 1$. In the temperature regime of interest to us CDM complements the invisible normal baryonic matter. The further back we look at the hot Universe, the more irrelevant become all form of matter including  ``dark'' matter component. 

There are considerable tensions in the present day speed of expansion (Hubble parameter)~\cite{DiValentino:2024spr,DiValentino:2021izs}: Extrapolation from the distant past are smaller than the Universe properties observed and studied in the current epoch a result stated often asking the question "67 or 75?". However, this unresolved issue arises from the epochs when  the Universe was in its atomic, molecular, stellar forms which is in principle irrelevant to our particle and plasma study.  But this separation of scales maybe not complete. Depending on details of PP-SM in the neutrino sector there may be some impact on what exactly is observed in contemporary Universe. One could argue that the effort to study the "Unknown" darkness (dark matter, dark energy) in cosmology  suffers from the lack of understanding of the "Known" in the primordial cosmos.   This is one of many  motivations for the research effort we pursue. 
 
%%%%%%%%%%%%%%%%%%%%%%%%%%%%%%%%%
\paragraph{Towards experimental study of primordial particle Universe:} Just before quarks and gluons were recognized as pivotal degrees of freedom in the primordial Universe, the so-called `Lee-Wick' model of dense  primordial matter prompted a high level meeting which had pivotal impact: The Bear Mountain November 1974 meeting was not open to all interested researchers: Only a few dozen  were invited to join the participant club, see last page of the meeting report \url{https://www.osti.gov/servlets/purl/4061527}, for further discussion see Ref.\,\cite{Rafelski:2019twp}.  

It is noteworthy that our report appears in essence on the 50th year anniversary of this November 29-December 1, 1974 meeting. Within just half a century our understanding and insight evolved decisively rendering all but one insight of the 1974 meeting obsolete. The particle and nuclear physics elite  of the epoch accepted the novel opportunity to experimentally explore hot and dense hadron (strongly interacting) matter by colliding high energy nuclei (heavy ions). Indeed one of the participants, Alfred Goldhaber, planted in the Nature magazine~\cite{Goldhaber:1978qp} the seed which grew into the RHIC collider at BNL-New York. It is hard to ascribe to this report any more relevance.

%%%%%%%%%%%%%%%%%%%%%%%%%%%%%%%%%
\paragraph{Extending the domain of primordial Universe:}Thanks to tireless effort of Rolf Hagedorn~\cite{Rafelski:2016hnq} the European laboratory CERN was intellectually well positioned for the rapid development of related physics ideas. A major physics motivation that soon emerged was the understanding of the primordial composition of the hot Universe. The pre-1970 idea advanced by Hagedorn, and by Huang and Weinberg~\cite{Huang:1970iq} was that the Universe was bound to a maximum Hagedorn temperature of $kT\le kT_H=150-180$\,MeV at which the energy content diverged. In the following years and indeed by the time of the Bear Mountain meeting the idea that a symmetry restoring change in phase structure would develop at finite temperature was taking hold~\cite{Weinberg:1974hy,Harrington:1974fc}.

Today we understand Hagedorn temperature $T_H$ to be the phase transition to the deconfined phase of matter where quarks and gluons can exist. The first clear statement about the existence of a phase boundary connecting Hagedorn hadron gas with constituent quarks invoking deconfinement at high temperature was the 1975 work of Cabibbo and Parisi~\cite{Cabibbo:1975ig}, which was followed by a more quantitative characterization within the realm of the bag model first without hadrons~\cite{Chin:1978gj} and soon after by Rafelski and Hagedorn, see Ref.\,\cite{Rafelski:2015cxa} and appendices A and B therein, implementing the Cabibbo-Parisi proposal in all detail.

Could QGP really exist beyond Hagedorn temperature? A broad acceptance of this new insight took decades to take hold. Many works in 1970's epoch missed the need to smoothly connect the hot quark-matter to hadron matter allowing for the melting of hadrons into their constituents. Yet other large body of work in this epoch addressed the dissolution of hadrons at zero temperature but high density of particles into quark constituents of astrophysical interest, without relevance to the understanding of both the early Universe and the relativistic heavy ion collisions; these two fields relate to each other and require understanding of the high temperature properties of elementary strongly interacting matter we study in this work connecting quarks to cosmos.
 
 When one of us (JR) first arrived at CERN in 1977, he found himself immersed into ardent discussions about the structure of the hot primordial Universe: Was it perhaps a dense baryon-antibaryon universe? Or was indeed the confinement condition not really retained at high temperature~\cite{Weinberg:1974hy,Harrington:1974fc,Cabibbo:1975ig}? These questions were sorted out in the following decade. However, the question, can we really tell apart in an experiment two different phases of matter haunted this field of research for decades to come~\cite{Rafelski:2015cxa,Harris:2024aov}, a topic which is not addressed in this work.


%%%%%%%%%%%%%%%%%%%%%%%%%%%%%%%%%%%%%%%
\subsubsection{Cosmology Primer}
\label{sec:flrw}
%%%%%%%%%%%%%%%%%%%%%%%%%%%%%%%%%%%%%%%
Our journey in time through expanding Universe has as objective the understanding of how different evolution eras impact each other. We are seeking to gain deeper insights into the fundamental processes that shaped our cosmos, providing a clearer picture of the universe's origin and its ongoing expansion. Therefore it is appropriate to begin with a short review of Universe dynamics.

We begin by introducing some of the necessary cosmology which will be useful throughout this work. As noted above we describe Universe evolution within the context of $\Lambda\mathrm{CDM}$ model of cosmology where the contemporary universe is approximately 69\% dark energy, 26\% dark matter, 5\% baryons, and $<1$\% photons and neutrinos in energy density~\cite{Davis:2003ad,Planck:2018vyg}. For most part our results will remain valid if one day this model evolves to account for tensions in modeling current Universe Hubble expansion. This is so since our work applies to the early Universe period where neither dark energy nor dark matter is relevant, expansion of the Universe is driven nearly solely by radiation and matter-antimatter content. 

%%%%%%%%%%%%%%%%%%%%%%%%%%%%%%%%%%%%%%%
\paragraph{Conventions in cosmology:}
There are several sign conventions in use in general relativity. As discussed by Hobson, Efstathiou and Lasenby~\cite{Hobson:2006se}, these conventions differ by three sign factors $S1$, $S2$, $S3$, which appear in the following objects:
\vspace{3mm}

Metric Signature: 
\beql{conv:metric}\eta^{\mu\nu}=(S1)\text{Diag}(1,-1,-1,-1)
\eeqn
\vspace{3mm}

Riemann Tensor: 
\beql{conv:Riemann}
R^\mu_{\alpha\beta\gamma}=(S2)(\partial_{\beta}\Gamma^\mu_{\alpha\gamma}-\partial_{\gamma}\Gamma^\mu_{\alpha\beta}+\Gamma^\mu_{\sigma\beta}\Gamma^\sigma_{\gamma\alpha}-\Gamma^\mu_{\sigma\gamma}\Gamma^\sigma_{\beta\alpha})
\eeqn
\vspace{3mm}

Einstein Equation: 
\beql{conv:EinstEq}
G_{\mu\nu}=(S3)8\pi G_NT_{\mu\nu}
\eeqn
\vspace{3mm}

Ricci Tensor:
\beql{conv:RicciT}
R_{\mu\nu}=(S2)(S3)R^\alpha_{\mu\alpha\nu}
\eeqn
\vspace{3mm}

\noindent The sign $S3$ comes from the choice of what index is contracted in forming the Ricci tensor. Since that sign factor appears in both $R_{\mu\nu}$ and $R$ it affects the overall sign of $G_{\mu\nu}$ and therefore Einstein's equation as shown above (here the cosmological constant is considered part of $T_{\mu\nu}$). In this work we will use the 
\beql{eq:3S}
\{(S_1), (S_2),(S_3)\}=(+,+,+)
\eeqn
convention.
%%%%%%%%%%%%%%%%%%%%%%%%%%%%%%%%%%
\paragraph{FLRW Cosmology:} The Friedmann-Lema{\^i}tre-Robertson-Walker (FLRW) line element and metric~\cite{Hartle:2003yu,Hobson:2006se,Misner:1973prb,Weinberg:1972kfs} in spherical coordinates is
\begin{gather}
 \label{FLRW} ds^2=dt^2-a^2(t)\left[\frac{dr^2}{1-kr^{2}}+r^{2}d\theta^2+r^{2}\sin\theta^{2}d\phi^2\right]\,,\\[0.3cm]
 g_{\alpha\beta}=
 \begin{pmatrix}
 1&0&0&0\\
 0&-\frac{a^{2}(t)}{1-kr^{2}}&0&0\\
 0&0&-a^{2}(t)r^{2}&0\\
 0&0&0&-a^{2}(t)r^{2}\sin\theta^{2}
 \end{pmatrix}\,.
\end{gather}
The Gaussian curvature $k$ informs the spatial hyper-surfaces defined by comoving observers. The spatial shape of the universe has the following possibilities~\cite{Planck:2018vyg}: infinite flat Euclidean $(k=0)$, finite spherical but unbounded $(k=+1)$, or infinite hyperbolic saddle-shaped $(k=-1)$. Observation indicates our universe is flat or nearly so. Current observation of cosmic microwave background (CMB) anisotropy imply the preferred value $k=0$~\cite{Planck:2018vyg,Planck:2015fie,Planck:2013pxb}.

In an expanding (or contracting) universe which is both homogeneous and isotropic, the scale factor $a(t)$ denotes the change of proper distances $L(t)$ over time as
\begin{gather}
 L(t)=L_{0}\frac{a_{0}}{a(t)}\rightarrow L(z)=L_{0}(1+z)\,,
\end{gather}
where $z$ is the redshift and $L_{0}$ the comoving length. This implies volumes change with $V(t)=V_{0}/a^{3}(t)$ where $V_{0}=L_{0}^{3}$ is the comoving Cartesian volume. In terms of temperature, we can consider the expansion to be an adiabatic process~\cite{Abdalla:2022yfr} which results in a smooth shifting of the relevant dynamical quantities. As the universe undergoes isotropic expansion, the temperature decreases as 
\begin{gather}
 \label{tscale}
 T(t)=T_{0}\frac{a_{0}}{a(t)}\rightarrow T(z)=T_{0}(1+z)\,,
\end{gather}
where $z$ is the redshift. The entropy within a comoving volume is kept constant until gravitational collapse effects become relevant. The comoving temperature $T_{0}$ is given by the the present CMB temperature $T_{0}=2.726{\rm\ K}\simeq 2.349\times10^{-4}\eV$~\cite{Planck:2018vyg}, with contemporary scale factor $a_{0}=1$.

The cosmological dynamical equations describing the evolution of the Universe follow from the Einstein equations. In general, the Einstein equation with cosmological constant $\Lambda$ can be written as:
\beqn\label{Einstine}
G^{\mu\nu} -\Lambda g^{\mu\nu}=\frac{\hbar c}{c^4M_p^2} T^{\mu\nu}, \quad G^{\mu\nu}=R^{\mu\nu}-\frac{R}{2} g^{\mu\nu}\,,
\quad R= g_{\mu\nu}R^{\mu\nu}
\eeqn
where the Planck mass $M_p$ is defined in terms of $G_N$, the Newtonian constant of gravitation
\beql{eq:GN}
\frac{1}{c^4}8\pi G_N\equiv  \frac{\hbar c}{c^4M_p^2}\,, \qquad 
M_p c^2=2.4353\, 10^{18}\,\mathrm{GeV}\,.
\eeqn
Our definition of $M_p$, while more convenient in cosmology, differs by the factor $1/\sqrt{8\pi}$ from the particle physics convention introduced by particle data group (PDG)~\cite{ParticleDataGroup:2022pth}
\beql{eq:MplPDG}
 \sqrt{8\pi} M_p c^2 \equiv M_p^\mathrm{PDG} c^2\equiv =1.2209\, 10^{19}\,\mathrm{GeV}\,,
\eeqn
Above the space curvature has dimension 1/length$^2$ and the energy momentum tensor energy/length$^3$, all units are maintained by factors $\hbar$ and $c$. However, from now on we will often omit to state explicitly factors $\hbar$ or $c$.

Recall that the Einstein tensor $G^{\mu\nu}$ is divergence free and so is the stress energy tensor, $T^{\mu\nu}$. In a homogeneous isotropic spacetime, the matter content is necessarily characterized by two quantities, the energy density $\rho$ and isotropic pressure~$P$
\begin{equation}
 T^\mu_\nu =\mathrm{diag}(\rho, -P, -P, -P).
\end{equation}
 It is common to absorb the Einstein cosmological constant $\Lambda$ into $\rho$ and $P$ by defining
\beqn\label{EpsLam}
\rho_\Lambda=M_p^2\Lambda, \qquad P_\Lambda=-M_p^2 \Lambda.
\eeqn
We implicitly consider this done from now on. 

As the universe expands, redshift reduces the momenta of particles lowering their contribution to the energy content of the universe. This cosmic redshift is written as
\begin{alignat}{1}
 \label{Redshift} p_{i}(t) = p_{i,0}\frac{a_{0}}{a(t)}\,.
\end{alignat}
Momentum (and the energy of massless particles $E=pc$) scales with the same factor as temperature. Since mass does not evolve in time,the energy of massive free particles in the universe  scales differently based on their momentum (and thus temperature). When hot and relativistic, particle energy decreases inversely with scale factor like radiation. As the particles transition to non-relativistic (NR) energies, they decrease with the inverse square of the scale factor
\begin{alignat}{1}
 \label{EScale} E(t) = E_{0}\frac{a_{0}}{a(t)}\xrightarrow{\mathrm{NR}}\ E_{0}\frac{a_{0}^{2}}{a(t)^{2}}\,.
\end{alignat}
This occurs because of the functional dependence of energy on momentum in the relativistic $E\sim p$ versus non-relativistic $E\sim p^{2}$ cases.

%%%%%%%%%%%%%%%%%%%%%%%%%%%%%%%%%%%%%%%%%%%%%%%%%%%%%
\paragraph{Hubble parameter and deceleration parameter:}
The global Universe dynamics can be characterized by two quantities, the Hubble parameter $H$, a strongly time dependent quantity on cosmological time scales, and the deceleration parameter $q$,
\beqn\label{dynamic}
\boxed{H(t)\equiv\frac{\dot a }{a} }\,,\qquad 
q\equiv -\frac{a\ddot a}{\dot a^2}\,.
\eeqn
We note the relations
\beqn
 \frac{\ddot a}{a}=-qH^2,\qquad\boxed{ \dot H=-H^2(1+q)}\,. 
\eeqn

Two dynamically independent equations arise using the metric \req{FLRW} in the Einstein equation \req{Einstine}
\beqn\label{hubble}
\frac{8\pi G_N}{3} \rho = \frac{\dot a^2+k}{a^2}
=H^2\left( 1+\frac { k }{\dot a^2}\right),
\qquad
\frac{4\pi G_N}{3} (\rho+3P) =-\frac{\ddot a}{a}=qH^2.
\eeqn
These are also known as the Friedmann equations. The total density $\rho=\rho_\mathrm{total}$ is the sum of all contributions from any form of matter, radiation or field. This includes but is not limited to: dark energy $(\Lambda)$, dark matter (DM), baryons (B), leptons $(\ell,\nu)$ and photons $(\gamma)$. Depending on the age of the universe, the relative importance of each group changes as each dilutes differently under expansion with dark energy infamously remaining constant in density and accelerating the universe today.

There is a simple way to detemine  dependence of $q$ on Universe structure and dynamics: We can eliminate the strength of the interaction, $G_N$, by solving the equations \eqref{hubble} for ${8\pi G_N}/{3}$ and equating the result to find a relatively simple constraint for the deceleration parameter
\beqn\label{qparam}
q=\frac 1 2 \left(1+3\frac{P}{\rho}\right)\left(1+\frac{k}{\dot a^2}\right).
\eeqn
 
% Add "arrow of time"
% Add QGP regime
% Extend lepton regime
% Matter regime
% Dark energy regime

%%%%%%%%%%%%%%%%%%%%%%%%%%%%%%%%%%%%%%%%%%%%%%%
%\subsubsection{Eras of the Universe}
%\label{ssection:ErasUniv}
\paragraph{Eras of the Universe:}
From this point on, we work within the flat cosmological model with $k=0$ and so $q$ is determined entirely within the FLRW cosmological model by the matter content of the Universe
\begin{equation}\label{qparam}
\boxed{q=\frac 1 2 \left(1+3\frac{P}{\rho}\right)}\,.
\end{equation} 
The early universe was~\cite{Rafelski:2013yka}:
\begin{itemize}
\item radiation dominated $P=\rho/3$ resulting in value $q\to q_r = 1$
\item subsequently matter dominated $P\ll \rho$ {\it i.e.\/} $q \to q_m= 1/2 $, and 
\item lastly, the contemporary universe is undergoing a transition from matter to dark energy dominated $P=-\rho$ approaching the asymptotic value of $q\to q_d = -1$. 
\end{itemize} 
The value of the deceleration parameter is thus according to \req{qparam}  an indicator of the transition between different eras of the Universe's history: radiation dominated, matter dominated and dark energy dominated. 

The cosmic deceleration  parameter $q$ is for historical reasons  positive under deceleration $q>0$. Conversely, accelerating Universe has $q<0$. This convention was chosen under the tacit assumption that the universe should be decelerating, before the discovery of dark energy. The value of $q$ depends on energy content: 

\begin{itemize}
\item Radiation dominated Universe: \beql{Eq:radU}
P=\rho/3 \implies q=1\,.
\eeqn
\item (Non-relativistic) Matter dominated Universe: 
\beql{Eq:nonrmU}
P\ll\rho \implies q=1/2\,.
\eeqn
\item Dark energy ($\Lambda$) dominated Universe: 
\beql{Eq:darkU} 
P=-\rho \implies q=-1\,.
\eeqn
\end{itemize}
We use $q$ first to characterize the era from today back to the end of neutrino freeze-out and then from freeze-out until the end of the hadron era.

As must be the case for any solution of Einstein's equations, \req{hubble} implies that the energy momentum tensor of matter is divergence free
\beqn\label{divTmn}
\nabla_\nu T^{\mu\nu} =0 \Rightarrow -\frac{\dot\rho}{\rho+P}=3\frac{\dot a}{a}=3H.
\eeqn
 Given an equation of state $P(\rho)$, solution of \req{divTmn} describes the dynamical evolution of matter in the Universe. Combined with the Hubble equation
\begin{equation}\label{Hubble_eq}
H^2=\frac{\rho}{3M_p^2}
\end{equation}
this allows us to understand the large-scale dynamics and evolution of the Universe, including the Hubble expansion and the behavior of matter and energy over cosmic time.

%%%%%%%%%%%%%%%%%%%%%%%%%%%%%%%%%%%%%%
\subsection{Composition of the Universe}
\subsubsection{Particle content of the Universe}\label{ssec:ParticleU}
%%%%%%%%%%%%%%%%%%%%%%%%%%%%%%%%%%%%%%
Our detailed understanding of the primordial Universe arises from half a century of research in the fields of cosmology, heavy ion collisions, particle, nuclear and plasma physics. Considering present day knowledge, it is not possible to imagine a different way in which the early Universe might have evolved. 

We believe today that the primordial deconfined matter we call quark-gluon plasma (QGP) filled the entire Universe and lasted for about first $20-30\ \mathrm{\mu s}$ after the Big Bang~\cite{Letessier:2002ony}. The deconfined condition allows free motion of quarks and gluons along with all other elementary particles. This hot particle soup contained the building blocks of the usual matter that today surrounds us, and all other elementary matter:
\begin{itemize}
\item The up $u$ and down $d$ quarks now hidden in protons and neutrons;
\item Electrons, three types (flavors) of neutrinos;
\item[] There were also unstable particle present which can decay but are reformed in hot universe:
\item Heavy unstable leptons muon $\mu$ and tauon $\tau$;
\item Unstable when bound in present day matter strange $s$, and heavy charm $c$ and bottom $b$ quarks;
\item[] At yet higher temperatures unreachable today in laboratory experiments we encounter all the remaining much heavier standard model particles: 
\item Electroweak theory gauge Bosons W$^\pm$ and Z$^0$, the top $t$ quark, and the Higgs particle H.
\item The QGP phase of matter contains also the gluons, particles mediating the strong interaction of deconfined quarks.
\end{itemize}

The question we address in this report is how this very hot soup of elementary matter evolves and connects to the normal matter in the era of the Big-Bang nucleosynthesis. We present here the theoretical insights gained over the past dozen years in an effort to create a backdrop of knowledge allowing to seek any remnant observables accessible today.

The one yet not fully clarified epoch concerns emergence of particles with properties as we observed them in the laboratory today, this happend at highest temperature we consider in this work: We presume that the quark-gluon dominated cosmic plasma (QGP) emerged at about $T\simeq 130$\,GeV and evolves in thermal and chemical equilibrium towards hadronization at about $T\simeq150$\,MeV, in later part of this evolution dominated by far by strong interactions. 

%%%%%%%%%%%%%%%%%%%%%%%%%%%%%%%%%%%%%%%%%%
\paragraph{Cosmic plasma in the early Universe $300\,\mathrm{MeV}>T>0.02\,\mathrm{MeV}$:}
\label{ssec:plasmas}
%In this section we will focus on the following:
%\begin{itemize}
% \item Five different plasma epoch from $0.3\mathrm{GeV}>T>20$keV
%\end{itemize}

The primordial hot Universe fireball underwent several nearly adiabatic phase changes that dramatically evolved its bulk properties as it expanded and cooled~\cite{Rafelski:2023emw}. 
We present an overview of the Universe evolution as a function of temperature from $300\,\mathrm{MeV}>T>0.02\,\mathrm{MeV}$ and main events constituting the history of the early Universe in \rf{Fig:Overview}. After the electroweak symmetry breaking epoch and presumably inflation, the comic plasma in the early Universe evolves in the following sequence:

%%%%%%%%%%%%%%%%%%%%%%%%%%%%%%%%%%%%%%%
\begin{figure}[ht]
 \centerline{\includegraphics[width=\textwidth,width=\linewidth]{./plots/CosmicTimeTemperature_Project}}
 \caption{Adapted from the thesis of C.T. Yang \cite{Yang:2024ret}:
 The relation between time and temperature in the first hour of the Universe beginning shortly before QGP hadronization $300\,\mathrm{MeV}>T>0.02\,\mathrm{MeV} $ and ending with antimatter disappearance.Temperature/time range for several key events is indicated.}
 \label{Fig:Overview}
\end{figure}
%%%%%%%%%%%%%%%%%%%%%%%%%%%%%%%%%%%%%%%
 
We focus our study on the first hour of the Universe evolution which takes us down to temperature of about $T\simeq 20\keV$.
\begin{enumerate}
\item \textbf{Primordial quark-gluon plasma}: 
At early times when the temperature was between $130\,\mathrm{GeV}>T>0.15\,\mathrm{GeV}$ we have the building blocks of the Universe as we know them today, including the leptons, vector bosons, and all three families of deconfined quarks and gluons which propagated freely in plasma. As all hadrons are dissolved into their constituents during this time, strongly interacting particles $u,d,s,t,b,c,g$ controlled the fate of the Universe. When temperature is near to the QGP phase transition $300\, \mathrm{MeV}>T>150$ MeV, the bottom quark breaks the detail balance and disappearance from particle inventory provides the arrow in time (see Chapter~\ref{Bottom} for detail).
 
\item \textbf{Hadronic epoch}: Around the hadronization temperature $T_H\approx150\,\mathrm{MeV}$, a phase transformation occurred, forcing the free quarks and gluons become confined within baryon and mesons \cite{Letessier:2005qe}. In the temperature range $ 150\,\mathrm{MeV}>T>20\,\mathrm{MeV}$, the Universe is rich in physics phenomena involving strange mesons and (anti)baryons including (anti)hyperon abundances~\cite{Fromerth:2012fe,Yang:2021bko}. The antibaryons disappear from the Universe at temperature $T=38.2$ MeV, and strangeness can be produced by the inverse decay reactions that are in equilibrium via weak, electromagnetic, and strong interactions in the early Universe until $T\approx13$ MeV (see Chapter~\ref{Strangeness} for detailed discussion).

\item \textbf{Lepton-photon epoch}: For temperature $10\,\mathrm{MeV}>T>2\,\mathrm{MeV}$, the Universe contained relativistic electrons, positrons, photons, and three species of (anti)neutrinos. During this epoch massless leptons and photons controlled the fate of the Universe. Massive $\tau^\pm$ disappear from the plasma at high temperature via decay processes. However $\mu^\pm$ leptons can persist in the early Universe until temperature $T=4.2$ MeV, and positron $e^+$ can persist until the temperature $T=0.02$ MeV (See Chapter~\ref{Electron} for discussion).

Neutrinos were still coupled to the charged leptons via the weak interaction~\cite{Birrell:2012gg,Birrell:2014ona} and freeze-out at temperature range $3\,\mathrm{MeV}>T>2\,\mathrm{MeV}$ which depends on the neutrino's flavors and the magnitude of the Standard Model parameters (See Chapter~\ref{Neutrino} for details). After neutrino freeze-out, they still play a important role in the Universe expansion via the effective number of neutrinos $N_{\nu}^{\mathrm{eff}}$ and affects the Hubble parameter significantly. 
\item \textbf{Electron-positron epoch}: After neutrinos freeze-out at $T=3\sim2\,\mathrm{MeV}$ and become free-streaming in the early Universe, the cosmic plasma was dominated by electrons, positrons, and photons. The $e^\pm$ plasma existed until $T\approx 0.02\,\mathrm{MeV}$. 
\item \textbf{BBN in midst of ${\mathbf e^+e^-}$ plasms:} Contrary to what was the prevailing context only a few years ago, it is today understood that BBN occurred within a rich electron-positron plasma environment. There are 1000's of${  e^+e^-}$-pairs for each nucleon undergoing nuclear BBN fusion reaction.  
\item \textbf{Primordial magnetism}\index{magnetism!primordial}: $e^{+}e^{-}$-plasma (electron-positrons pairs) at temperatures reaching well below BBN epoch in the primordial universe could be a source of the present day intergalactic magnetic fields. See Chapter~\ref{Electron} for detailed discussion. We explore Landau diamagnetic and magnetic dipole moment paramagnetic properties. A relatively small magnitude of the $e^{+}e^{-}$ magnetic moment polarization asymmetry suffices to produce a self-magnetization in the universe consistent with present day observations.
\end{enumerate}

After $e^\pm$ annihilation finishes at a temperature near 20keV, the Universe was still opaque to photons due to large scattering photon-electron Thompson cross section. Observational cosmology study ofthe Cosmic Microwave Background (CMB)~\cite{Planck:2018vyg} reaches to the epoch of free electron binding into atoms -- a process referred to as recombination -- at $T\approx 0.25\,\mathrm{eV}$. We focus on the temperature $300\,\mathrm{MeV}>T>0.02\,\mathrm{MeV}$ which corresponds to 

 We will address the cosmic plasma as follow: In Chapter~\ref{Bottom}, we discuss the heavy quarks (bottom/charm) abundance near to the QGP hadronization and show the nonequilibrium of bottom quark. In Chapter~\ref{Strangeness} we study the strangeness abundance after hadronization and show the long lasting strangeness in the early Universe. In Chapter~\ref{Neutrino} we focus on the neutrino-matter interactions and the evolution of cosmic neutrino in early universe before/after freeze-out. In Chapter~\ref{Electron} we study the abundance of charged leptons $\mu^\pm$ and $e^\pm$ and show that the present of $e^\pm$ plasma plays an important role in early Universe.
%%%%%%%%%%%%%%%%%%%%%%%%%%%%%%%%%%%%%%

%%%%%%%%%%%%%%%%%%%%%%%%%%%
\paragraph{Quark-Gluon Plasma Epoch:}
\label{sec:qgpOverview}

We will illustrate some deviations from equilibrium near to the hadronization matter formation condition. Specifically, we demonstrate the presence of chemical non-equilibrium that continues to the present day. We focus on the distinction between chemical and kinetic equilibrium and freeze-out.

We explore the following evolution of heavy strongly interacting hadrons and the annihilation of antimatter. This leads us to the leptonic dominated era in history of the Universe. Using nonquilibrium concepts, we derive those properties of neutrino freeze-out that depend only on conservation laws and are independent of the details of the microscopic scattering processes. 


...More to follow ... Include references to chapters.... jr
 

%%%%%%%%%%%%%%%%%%%%%%%%%%%%%%%%%%%%%%%%%
\paragraph{Neutrino Freeze-out:}
\label{ssec:nuoverview}
We characterize the dependence of both $N_\nu$ and the deviation of the neutrino distribution from chemical equilibrium on the neutrino kinetic freeze-out temperature. We detail a spectral method was designed extend the regime of applicability to systems far from chemical equilibrium and/or that undergo significant reheating.

%%%%%%%%%%%%%%%%%%%%%%%%%%%%%%%%%%%%%%%%%
\paragraph{Positron Antimatter:}
\label{ssec:PositronView}
In the early universe during the hot dense electron-positron plasma epoch (after neutrino freeze-out), we analyze the paramagnetic characteristics of electron-positron plasma when exposed to an external primordial field. As temperature decreases nuclear reactions between remaining tiny baryon imbalance produce light nuclei and our interest in this well explored era of big-bang nucleosynthesis (BBN) is focused on the presence of a rich electron-positron plasma and its effect on nuclear reactions, a topic that has escaped wide attention.

%%%%%%%%%%%%%%%%%%%%%%%%%%%%%%%%
\paragraph{Relic Neutrino Background}\label{ch:intro}
%{\color{blue}\textbf{CTYang:} I think this paragraph can be the introduction for cosmic neutrino background section}
At a temperature of $5$ MeV the Universe consisted of a plasma of $e^\pm$, photons, and neutrinos. At around $1$ MeV neutrinos stop interacting, or freeze-out, and begin to free-stream through the Universe. Today they comprise the relic neutrino background. Photons freeze-out around $0.25$ eV and today they make up the Cosmic Microwave Background (CMB), currently at $T_{\gamma,0}=0.235$ meV. Relic neutrinos have not been directly measured, but their impact on the speed of expansion of the Universe is imprinted on the CMB. Indirect measurements of the relic neutrino background, such as by the Planck satellite~\cite{Planck:2018vyg,Planck:2015fie,Planck:2013pxb}, constrain neutrino properties such as mass and number of massless degrees of freedom.

In later chapters, we will study the details of the neutrino freeze-out process and their impact on observables in more detail. Here we present an overview of the era just prior to neutrino freeze-out through current epoch, putting the relic neutrinos in their proper context. Much of this material, including most Figures, was adapted from~\cite{Rafelski:2013yka}.


%%%%%%%%%%%%%%%%%%%%%%%%%%%%%%%%%%%%%%%%%%%%
\subsubsection{Reheating History of the Universe}\label{Eralink}

At times where dimensional scales are irrelevant, entropy conservation means that temperature scales inversely with the scale factor $a(t)$. This follows from \req{divTmn} when $ \rho\simeq 3P \propto T^4$. However, as the temperature drops and at their respective $m\simeq T$ scales, successively less massive particles annihilate and disappear from the thermal Universe. Their entropy reheats the other degrees of freedom and thus in the process, the entropy originating in a massive degree of freedom is shifted into the effectively massless degrees of freedom that still remain. This causes the $T\propto 1/a(t)$ scaling to break down; during each of these `reorganization' periods the drop in temperature is slowed by the concentration of entropy in fewer degrees of freedom, leading to a change in the reheating ratio, $R$, defined as
\begin{equation}\label{redshiftratio}
R\equiv \frac{1+z}{ T_\gamma/T_{\gamma,0}}, \qquad 1+z\equiv \frac{a_{0}}{a(t)}.
\end{equation}
The reheating ratio connects the photon temperature redshift to the geometric redshift, where $a_0$ is the scale factor today (often normalized to $1$) and quantifies the deviation from the scaling relation between $a(t)$ and $T$.

As we will see, the change in $R$ can be computed by the drop in the number of degrees of freedom. At a temperature on the order of the top quark mass, when all standard model particles were in thermal equilibrium, the Universe was pushed apart by 28 bosonic and 90 fermionic degrees of freedom. The total number of degrees of freedom can be computed as follows. 

For bosons we have the following: the doublet of charged Higgs particles has $4=2\times2=1+3$ degrees of freedom -- three will migrate to the longitudinal components of $W^\pm, Z$ when the electro-weak vacuum freezes and the EW symmetry breaking arises, while one is retained in the one single dynamical charge neutral Higgs component. In the massless stage, the SU(2)$\times$U(1) theory has 4$\times$2=8 gauge degrees of freedom where the first coefficient is the number of particles $(\gamma, Z, W^\pm)$ and each massless gauge boson has two transverse polarizations. Adding in $8_c\times2_s=16$ gluonic degrees of freedom we obtain 4+8+16=28 bosonic degrees of freedom. 

The count of fermionic degrees of freedom includes three $f$ families, two spins $s$, another factor two for particle-antiparticle duality. We have in each family of flavors a doublet of $2\times 3_c$ quarks, 1-lepton and 1/2 neutrinos (due left-handedness which was not implemented counting spin). Thus we find that a total $3_f\times 2_p\times 2_s\times(2\times 3_c+1_l+1/2_\nu)=90$ fermionic degrees of freedom. We further recall that massless fermions contribute 7/8 of that of bosons in both pressure and energy density. Thus the total number of massless Standard Model particles at a temperature above the top quark mass scale, referring by convention to bosonic degrees of freedom, is $g_{\rm SM}=28+90\times 7/8=106.75$ 



In \rf{fig:dof} we show the cube of the reheating ratio \req{redshiftratio} as a function of photon temperature $T_\gamma$ from the primordial high temperature early Universe on the right to the present on the left, where $R$ must be by definition unity. The periods of change seen in Figure \ref{fig:dof} come when the temperature crosses the mass of a particle species that is in equilibrium. One can see drops corresponding to the disappearance of particles as indicated. After $e^+e^-$ annihilation on the left, there are no significant degrees of freedom remaining to annihilate and feed entropy into photons, and so $R$ remains constant until today. We show the result using a Fermi gas model with a very rough model for the QGP phase transition and hadronization period near $O(100\MeV)$. The fermi gas model is a poor approximation above the QGP phase transition; a more precise model using lattice QCD, see e.g.~\cite{Borsanyi:2013bia}, together with a high temperature perturbative QCD expansion, see e.g.~\cite{Letessier:2002ony}, would be needed to improve on this situation but the details do not impact the neutrino freeze-out period near $1\MeV$ which is our primary concern, and so we do not consider these issues further here.

%%%%%%%%%%%%%%%%%%%%%%%%%%%%%%%%%%%%%%%
\begin{figure} 
\centerline{\includegraphics[height=5.2cm]{01-introduction/Figures/degrees_of_freedom.PNG}}
\caption{\cccite{Rafelski:2023emw}, adapted from Ref.~\cite{Rafelski:2013yka} and thesis of J.Birrell \cite{Birrell:2014ona}. Disappearance of degrees of freedom throughout the history of the Universe. The Universe volume inflated approximately by a factor of 27 above the thermal red shift scale as massive particles disappeared successively from the inventory. The dashed portion is a qualitative description
of QGP hadronization.\label{fig:dof}}
 \end{figure}
%%%%%%%%%%%%%%%%%%%%%%%%%%%%%%%%%%%%%%%



As long as the dynamics are at least approximately entropy conserving, the total drop in $R$ is entirely determined by entropy conservation. Namely, the magnitude of the drop in $R$ Figure~\ref{fig:dof} is a measure of the number of degrees of freedom that have disappeared from the Universe. Consider two times $t_1$ and $t_2$ at which all particle species that have not yet annihilated are effectively massless. By conservation of comoving entropy and scaling $T\propto 1/a$ we have
\begin{equation}\label{r_ratio}
1=\frac{a_1^3S_{1}}{a_2^3 S_2}=\frac{a_1^3\sum_ig_i T_{i,1}^3}{a_2^3\sum_j g_j T_{j,2}^3},\qquad \left(\frac{R_1}{R_2}\right)^3=\frac{\sum_ig_i (T_{i,1}/T_{\gamma,1})^3}{\sum_j g_j (T_{j,2}/T_{\gamma,2})^3}
\end{equation}
where the sums are over the total number of degrees of freedom present at the indicated time and the degeneracy factors $g_i$ contain the $7/8$ factor for fermions. In the second form we divided the numerator and denominator by $a_{0}T_{\gamma,0}$. We distinguish between the temperature of each particle species and our reference temperature, the photon temperature. This is important since today neutrinos are colder than photons, due to photon reheating from $e^\pm$ annihilation occurring after neutrinos decoupled (this is only an approximation, a point we will study in detail in subsequent chapters). By conservation of entropy one obtains the neutrino to photon temperature ratio of
\begin{equation}\label{T_nu_T_gamma}
T_\nu/T_\gamma=({4}/{11})^{1/3}.
\end{equation}
We will call this the reheating ratio in the decoupled limit. For details on the derivation of this standard result, see, e.g., Section \ref{Tnugam} where it is obtained as a special case of a more general analysis.

Using \req{r_ratio} we compute the total drop in $R^3$ shown in Figure \ref{fig:dof}. At $T=T_\gamma=\mathcal{O}(100\GeV)$ the number of active degrees of freedom is slightly below $g_{\rm SM}=106.75$ due to the partial disappearance of top quarks, but this approximation will be good enough for our purposes. At this time, all the species are in thermal equilibrium with photons and so $T_{i,1}/T_{\gamma,1}=1$ for all $i$. Today we have $2$ photon and $7/8\times 6$ neutrino degrees of freedom and a neutrino to photon temperature ratio \req{T_nu_T_gamma}. Therefore we have
\begin{equation}
\left(\frac{R_{100GeV}}{R_{now}}\right)^3= \frac{g_{SM}}{g_{\rm now}}=\frac{106.75}{2+\frac{7}{8}\times 6\times \frac{4}{11}}\approx 27.3
\end{equation}
which is the fractional change we see in the fermi gas model curve in Figure \ref{fig:dof} (as mentioned above, the QCD model is reduced due to interactions). The meaning of this factor is that the Universe approximately inflated by a factor 27 above the thermal red shift scale as massive particles disappeared successively from the inventory. 

From the perspective of reheating, the history of the Universe from the end of $e^\pm$ annihilation until today has been uneventful. 

We can shed additional light on this period and others by looking at the composition of the Universe as a function of temperature

%%%%%%%%%%%%%%%%%%%%%%%%%%%%%%%%%%%%
\begin{figure}
\centerline{\includegraphics[height=8.2cm]{01-introduction/Figures/energy_densities_total.eps}}\label{fig:energy_frac}
\caption{\cccite{Rafelski:2023emw}, expanded from results of the thesis of J.Birrell \cite{Birrell:2014ona}. Current era: $69\%$ dark energy, $26\%$ dark matter, $5\%$ baryons, $<1\%$ photons and neutrinos, $1$ massless and $2\times .1$ eV neutrinos (Neutrino mass choice is just for illustration. Other values are possible). }
 \end{figure}
%%%%%%%%%%%%%%%%%%%%%%%%%%%%%%%%%%
In Figure \ref{fig:energy_frac} we begin on the right at the end of the hadron era with the disappearance of muons and pions. This constitutes a reheating period, with energy and entropy from these particles being transferred to the remaining $e^\pm$, photon, neutrino plasma. Continuing to $T=O(1)$ MeV, we come to the annihilation of $e^\pm$ and the photon reheating period. Notice that only the photon energy density fraction increases here. As discussed above, a common simplifying assumption is that neutrinos are already decoupled at this time and hence do not share in the reheating process, leading to a difference in photon and neutrino temperatures \req{T_nu_T_gamma}.

After passing through a long period, from $T=O(1)$ MeV until $T=O(1)$ eV, where the energy density is dominated by photons and free-streaming neutrinos, we then come to the beginning of the matter dominated regime, where the energy density is dominated by dark matter and baryonic matter. This transition is the result of the redshifting of the photon and neutrino energy, $\rho\propto T^4$, whereas for non-relativistic matter $\rho\propto a^{-3}\propto T^3$. Note that our inclusion of neutrino mass causes the leveling out of the neutrino energy density fraction during this period, as compared to the continued redshifting of the photon energy.

Finally, as we move towards the present day CMB temperature of $T_{\gamma,0}=0.235$ meV on the left hand side, we have entered the dark energy dominated regime. For the present day values, we have used the fits from the Planck data~\cite{Planck:2018vyg,Planck:2015fie,Planck:2013pxb} of $69\%$ dark energy, $26\%$ dark matter and $5\%$ baryons (and zero spatial curvature). The photon energy density is fixed by the CMB temperature $T_{\gamma,0}$ and the neutrino energy density is fixed by $T_{\gamma,0}$ along with the photon to neutrino temperature ratio. Both constitute $<1\%$ of the current energy budget.

%%%%%%%%%%%%%%%%%%%%%%%%%%%%%%%%%%%%%%%%%%%%%%
%\subsubsection\label{recomb}
\paragraph{Looking back in time:}
In the following we use the mix of matter (31\%) and dark energy (69\%) with photon and neutrino backgrounds favored by the latest Planck results~\cite{Planck:2018vyg,Planck:2015fie,Planck:2013pxb}, where we gave two neutrino species mass of $m_\nu=30\meV$ and a third neutrino remains massless. This is a different mass value than used above and again, it is only for illustration-- other mass choices are possible within present day constraints and will impact to some degree where exactly matter dominance emerges from the radiative Universe. We presume that neutrino kinetic freeze-out completed before the onset of $e^\pm$-annihilation into photons, leading to the neutrino to photon temperature ratio \req{T_nu_T_gamma}. Again, this is a common simplifying assumption. Much of the remainder of this work will involve improving on this approximation, but for the purposes of this overview it is sufficient.

Figure \ref{fig:today} shows in the left frame the temperature (left axis) and deceleration parameter (right axis) from shortly after the completion neutrino freeze-out until today. The horizontal dot-dashed lines show the pure radiation-dominated value of $q=1$ and the matter-dominated value of $q=1/2$. The expansion in this era starts off as radiation-dominated, but transitions to matter-dominated starting around $T=\mathcal{O}(10\eV)$ and begins to transition to a dark energy dominated era at $T=\mathcal{O}(1\meV)$. We are still in the midst of this transition today. The vertical dot-dashed lines show the time of recombination at $T\simeq0.25\eV$, when the Universe became transparent to photons, and reionization at $T\simeq {\cal O}(1\meV)$, when hydrogen in the Universe was again ionized due to light from the first galaxies~\cite{Zaroubi:2012in}. 

On the right in Figure \ref{fig:today} we show the Hubble parameter $H$ and redshift $z+1\equiv a_0/a(t)$. We can see in Figure \ref{fig:today} a visible deviation from power law behavior due to the transitions from radiation to matter dominated and from matter to dark energy dominated expansion. These transitions are accentuated and more easily visualized in the form of the deceleration parameter $q$. The time span covered by the Figure \ref{fig:today} is in essence the entire lifespan of the Universe, but of course on a logarithmic time scale there is a lot of room for interesting physics in the tiny blip that happened beforehand.


%%%%%%%%%%%%%%%%%%%%%%%%%%%%%%%%%%%%%%%
\begin{figure}
\begin{minipage}{\linewidth}
\makebox[0.5\linewidth]%
{\includegraphics[keepaspectratio=true,scale=0.52]{01-introduction/Figures/T_q_today.eps}}
\makebox[0.5\linewidth]%
{\includegraphics[keepaspectratio=true,scale=0.52]{01-introduction/Figures/H_z_today.eps}}
\caption{\cccite{Rafelski:2013yka}: Transition periods in the composition of the Universe: on left -- evolution of temperature $T$ and deceleration parameter $q$; on right -- evolution of the Hubble parameter $H$ and redshift $z$. 
\label{fig:today} }
\end{minipage}
\end{figure}
%%%%%%%%%%%%%%%%%%%%%%%%%%%%%%%%%%%%%%%

%%%%%%%%%%%%%%%%%%%%%%%%%%%%%%%%%%%%%%%%%
\subsubsection{Neutrino freeze-out era} \label{nudecoup}
%\paragraph{Neutrino freeze-out era:} 
The era separating the photon-neutrino-matter-dark energy Universe we just described from the end of the hadron Universe is quite complex in its evolution. We begin when the number of $e^\pm$-pairs has decayed to the same abundance as the number of baryons in the Universe at the temperature $T=\mathcal{O}(10\keV)$ and reach back to $T={\cal O}(30\MeV)$ where muons and pions are disappearing from the Universe.

%%%%%%%%%%%%%%%%%%%%%%%%%%%%%%%%%%%%%%%
\begin{figure}
\begin{minipage}{\linewidth}
\makebox[0.5\linewidth]%
{\includegraphics[keepaspectratio=true,scale=0.54]{01-introduction/Figures/T_q_BBN.eps}} 
\makebox[0.5\linewidth]%
{\includegraphics[keepaspectratio=true,scale=0.54]{01-introduction/Figures/H_z_BBN.eps}} 
\caption{\cccite{Rafelski:2013yka}: From the end of baryon antimatter annihilation through BBN, see Figure \ref{fig:today}. % on left -- evolution of temperature $T$ and deceleration parameter $q$; on right -- evolution of the Hubble parameter $H$ and redshift $z$.
\label{fig:BBN} }
\end{minipage}
\end{figure}
%%%%%%%%%%%%%%%%%%%%%%%%%%%%%%%%%%%%%%%

 In Figure~\ref{fig:BBN} the horizontal dot-dashed line for $q=1$ shows the pure radiation dominated value with two exceptions. First, the presence of massive pions and muons reduce the value of $q$ near to the maximal temperature shown. Second, when the temperature is near the value of the electron mass, the $e^\pm$-pairs are not yet fully depleted but already sufficiently non-relativistic to cause another dip in $q$. These are not large drops; the expansion is still predominately radiation dominated. But $q$ provides a sensitive measure of when various mass scales become relevant and is a good indicator of the presence of a reheating period.

 The dashed line shows the neutrino temperature, which decouples from the $e^\pm$ and photon temperature at $T={\cal O}(1\MeV)$ when neutrinos freeze-out and begin free streaming. In Figure~\ref{fig:BBN} the unit of time is seconds and the range spans the domain from fractions of a millisecond to a few hours. After neutrino freeze-out we come to Big Bang Nucleosynthesis, the period when the lighter elements were synthesized in a hot but relatively dilute plasma~\cite{Iocco:2008va}. We left some time gap between this and the domain shown in Figure \ref{fig:today} describing the current era -- there is an uneventful evolution between the two domains. 
 


%%%%%%%%%%%%%%%%%%%%%%%%%%%%%%%%%%%%%%%%%%%%%%%%%
\paragraph{Properties of Neutrino Freeze-out in the Early Universe:}
 Neutrino freeze-out is, as far as we know, the unique era in the history of the Universe when a significant matter fraction froze out at the same time that a reheating period was beginning, namely the start of $e^\pm$ annihilation. It is this coincidence that makes neutrino freeze-out a rich and complicated period to study as compared to the many other reheating periods in the history of the Universe. 

The properties of the neutrino background are influenced by the details of the freeze-out or decoupling process at a temperature $T=\mathcal{O}(2\mathrm{MeV})$. In the literature one finds estimates of freeze-out temperatures based on a comparison of Hubble expansion with neutrino scattering length and considering only number changing (i.e. chemical) processes. In the paper~\cite{Birrell:2014uka}, we employ a similar definition of freeze-out temperature in the context of the Boltzmann equation and refine the results by noting that there are three different freeze-out processes for neutrinos:

%In general, the particle freezeout process include both chemical and kinetic which lead to particle become free-streaming in the early Universe (for detail discussion see Chapter~\ref{Introduction}), we have:

1. Neutrino chemical freeze-out: the temperature at which neutrino number changing processes such as $e^-e^+\to\nu\overline\nu$ effectively cease. After chemical freeze-out, there are no reactions that, in a noteworthy fashion, can
change the neutrino abundance and so particle number is conserved. %Prior to the chemical freezeout temperature, number changing processes are significant and keep the particle in chemical (and thermal) equilibrium, implying the distribution function of neutrino has the Fermi-Dirac form:
%\begin{equation}\label{equilibrium}
%f_{c}(t,E)=\frac{1}{\exp(E/T)+1}, \qquad\text{ for } T> T_{ch}.
%\end{equation}

2. Neutrino kinetic freeze-out: the temperature at which the neutrino momentum exchanging interactions such as $e^\pm\nu\to e^\pm\nu$ are no longer occurring rapidly enough to maintain an equilibrium momentum distribution. %When $T_k<T(t)<T_{ch}$, the number-changing process no longer occurs rapidly enough to keep the distribution in chemical equilibrium but there is still sufficient momentum exchange to keep the distribution in thermal equilibrium. The distribution function is therefore obtained by maximizing entropy, with fixed energy, particle number, and antiparticle number separately. This implies that the distribution function has the form
%\begin{equation}\label{kinetic_equilib}
%f_k(t,E)=\frac{1}{\Upsilon^{-1}\exp(E/T)+1},\qquad \text{ for }T_k< T< T_{ch}.
%\end{equation}
%The time dependent generalized fugacity $\Upsilon(t)$ controls the occupancy of phase space and is necessary once $T(t)<T_{ch}$ in order to conserve particle number.

3. Collisions between neutrinos:
Those collisions $\nu\nu\to\nu\nu$ are capable of reequilibrating energy within and between neutrino flavor families. These processes end at a yet lower temperature and the neutrinos will be truly free-streaming from that point on.
 
%3. Free streaming: for $T<T_k$ there are no longer any significant interactions that couple the particle species of interest and so they begin to free-stream through the Universe. The Einstein-Vlasov equation can be solved~\cite{Choquet-Bruhat:2009xil,Birrell:2012gg}, to yield the free-streaming momentum distribution
%\begin{equation}\label{free_stream_dist}
%f(t,E)=\frac{1}{\Upsilon^{-1}e^{\sqrt{p^2/T_{fs}^2+m_\nu^2 /T_k^2}}+ 1},\qquad T_{fs}=\frac{T_ka(t_k)}{a(t)}
%\end{equation}
%where the free-streaming effective temperature $T_{fs}$ is obtained by redshifting the temperature at kinetic freeze-out, and $m_\nu$ is the mass of neutrino.





To estimate the freeze-out temperature, we need to solve the Boltzmann equation with different types of collision terms with the transition matrices from Table.~\ref{T005} and Table.~\ref{T006}. The paper~\cite{Birrell:2014uka} developed a new method for analytically simplifying the collision integrals and showing that the neutrino freeze-out temperature is controlled by fundamental coupling constants and particle masses. The freeze-out temperature depends only on the magnitude of the Weinberg angle in the form $\sin^2(\theta_W)$, and a dimensionless relative interaction strength parameter $\eta$,
\begin{align}
\eta\equiv M_p m_e^3 G_F^2, \qquad M_p^2\equiv \frac{1}{8\pi G_N}, \end{align}
a combination of the electron mass $m_e$, Newton constant $G_N$ (expressed above in terms of Planck mass $M_p$), and the Fermi constant $G_F$. The dimensionless interaction strength parameter $\eta$ in the present-day vacuum has the value
\begin{align}
\eta_0\equiv \left.M_p m_e^3 G_F^2\right|_0 = 0.04421 .
\end{align}

The magnitude of $\sin^2(\theta_W)$ is not fixed by known phenomena. It could be subject to variation as a function of time or temperature. In \rf{fig:freezeoutT} we show the dependence of neutrino freeze-out temperatures for $\nu_e$ and $\nu_{\mu,\tau}$ on SM model parameters $\sin^2(\theta_W)$ and $\eta$ in detail. We note the difference in freeze-out between electron neutrino and the heavier flavors. However:

 
%~~~~~~~~~~~~~~~~~~~~~~~~~~~~~~~~~~~~~~~~~~~~~~~~~
\begin{figure}[ht]
\centerline{\includegraphics[width=0.47\columnwidth]{./plots/nu_e_freezeout.pdf}
\hspace{1mm}\includegraphics[width=0.47\columnwidth]{./plots/nu_mu_freezeout.pdf}}
\centerline{\includegraphics[width=0.47\columnwidth]{./plots/nu_e_freezeout_GF.pdf}
\hspace{1mm}\includegraphics[width=0.47\columnwidth]{./plots/nu_mu_freezeout_GF.pdf}}

\caption{\cccite{Birrell:2014uka}, adapted from Ref.~\cite{Birrell:2014uka} and thesis of J.Birrell \cite{Birrell:2014ona}. Freeze-out temperatures for electron neutrinos (left) and $\mu$, $\tau$ neutrinos (right) for the three types of freeze-out processes. Top panels print temperature curves as a function of $\sin^2(\theta_W)$ for $\eta=\eta_0$, the vertical dashed line is $\sin^2(\theta_W)=0.23$; bottom panels are printed as a function of relative change in interaction strength $\eta/\eta_0$ obtained for $\sin^2(\theta_W)=0.23$.}
\label{fig:freezeoutT}
 \end{figure}
%~~~~~~~~~~~~~~~~~~~~~~~~~~~~~~~~~~~~~~~~~~~~~

Neutrinos are produced in elementary processes in flavor eigenstates. These are not mass eigenstates. Due to the difference in neutrino mass the coherent propagating components move at different velocity. Their very, very low interaction probability creates in laboratory the effect one refers to as `flavor oscillation'. How does this impact neurino freeze-out? Near to freeze-out temperature the electron-neutrino can still `annihilate' on electrons while absence of muons and taus in the cosmic plasma at a temperature of a few MeV makes these two neutrino flavors less interactive. 

Oscillation thus provide a mechanism in which the heavier flavors remain reactive in matter as they develop the more interactive electron-neutrino component. Conversely, electron neutrino freeze-out is occurring sooner than presented in \rf{fig:freezeoutT} since only a part of this flavor wave remains available to interact. Overall the difference in freeze-out of the three different flavors diminishes compared to results seen in \rf{fig:freezeoutT}. However, the effect was found negligible in work of Mangano et. al.~\cite{Mangano:2005cc}.

A discussion of the implications and connections of the results on neutrino freezeout to other areas of physics, including Big Bang Nucleosynthesis and dark radiation is described in more detail in~\cite{Dreiner:2011fp,Boehm:2012gr,Blennow:2012de,Birrell:2014uka}. A comprehensive investigation of neutrino freezeout, and a novel approach to analytically simplify the collision integrals for the Boltzmann equation can be found in Jeremiah Birrell‘s PhD thesis~\cite{Birrell:2014ona}. 

%%%%%%%%%%%%%%%%%%%%%%%%%%%%%%%%%%%%%%%%%%%%%%%
\subsection{Electron-positron plasma during BBN}\label{sec:density}
In this section, we will review the presence of electron-positron plasma during BBN. Before BBN, temperatures $T>86.7\,$keV were high enough that any nuclei formed would be disassociated by the vast number of high energy photons present \cite{Pitrou:2018cgg}. Once the temperature cooled to around $T<50\,$keV most of the nuclear reactions forming nuclei had already occurred. In this study, we can neglect the universe's expansion since, at this period, the time scale of expansion $H^{-1}$ is orders of magnitude larger than the time scale of BBN. We also note that at this point neutrinos have become free streaming \cite{Birrell:2012gg}.

In \cite{Grayson:2022asf}, C.T. Yang used the present-day baryon-to-photon ratio: $B/N_\gamma =n_B/n_\gamma= 6.05\times10^{-10}$ from Cosmic Microwave Background (CMB)~\cite{ParticleDataGroup:2022pth} and the charge neutrality of the universe to find the electron-positron chemical potential and density during BBN.

\begin{figure}[ht]
\begin{center}
%\includegraphics[width=0.95\linewidth]{Chemical_Plasma}
%\includegraphics[width=0.95\linewidth]{Density_Plasma002}
\includegraphics[width=0.95\linewidth]{plots/chap03BBN/May152023_EPDensity_Chemical}
\caption{\cccite{Grayson:2023flr}, adapted from Ref.~\cite{Grayson:2023flr} and thesis of C.T.Yang \cite{Yang:2024ret}. Left axis: The chemical potential of electrons as a function of temperature. Right axis: the ratio of electron (positron) number density to baryon density as a function of temperature. The solid blue line is the electron density, the red dashed line is the positron density, and the green dotted line is the number density with $\mu_e=0$. When $T=20.3\,\mathrm{keV}$ (the purple vertical line) positron density decreases rapidly because of the annihilation. The vertical black dotted lines represent BBN temperature range $86\,\mathrm{keV}>\mathrm{T_{BBN}}>50\,\mathrm{keV}$.}
\label{BBN_Electron}
\end{center}
\end{figure}

In Fig.~\ref{BBN_Electron} (left axis), we plot the electron chemical potential as a function of temperature. We can see the value of chemical potential is comparatively small $\mu_e/T\approx10^{-6}\sim10^{-7}$ during the BBN temperature range, implying an equal number of electrons and positrons in plasma. Thus, for the proceeding calculations, we will set $\mu =0$. When the temperature is around $T=70\,\mathrm{keV}$, the density of electrons and positrons is comparatively large $n_{e^\pm}\approx10^7\,n_B$. This indicates that we can assume the universe is filled mainly with electrons and positrons with light nuclei being a very small component of the plasma. Later when the temperature is around $T=20.3\,\mathrm{keV}$, the positron density decreases, transforming the pair-plasma to an electron-baryon plasma.


% In this section, we will derive the dependence of electron chemical potential, and hence $e^+e^-$ density, as a function of the photon background temperature $T$ by employing the following physical principles:
% \begin{enumerate}
% \item Charge neutrality of the Universe:
% \begin{align}\label{neutrality}
% n_{e^-}-n_{{e^+}}=n_p-n_{\overline{p}}\approx\,n_p,
% \end{align}
% where $n_\ell$ denotes the number density of particle type $\ell$.
% \item Neutrinos decouple (freeze out) at a temperature $T_f\simeq 2$ MeV, after which they free stream through the Universe with an effective temperature~\cite{Birrell:2012gg}
% \begin{align}
% T_\nu(t)=T_f\,\frac{a(t_f)}{a(t)},
% \end{align}
% where $a(t)$ is the--Friedmann--Lema\^{i}tre--Robertson--Walker (FLRW) Universe scale factor which is a function of cosmic time $t$, and $t_f$ represents the cosmic time when neutrino freezes out.
% \item Total comoving entropy is conserved. At $T\leq T_f$, the dominant contributors to entropy are photons, $e^+e^-$, and neutrinos. In addition, after neutrino freeze out, neutrino comoving entropy is independently conserved~\cite{Birrell:2012gg}. This implies that the combined comoving entropy in $e^+e^-\gamma$ is also conserved for $T\leq T_f$.
% \end{enumerate}

% Motivated by the fact that comoving entropy in $\gamma$, $e^+e^-$ is conserved after neutrino freezeout, we rewrite the charge neutrality condition, Eq.~(\ref{neutrality}), in the form
% \begin{align}\label{charge_neutral_cond2}
% n_{e^-}-n_{{e^+}}=X_p\frac{n_B}{s_{\gamma,e^\pm}} s_{\gamma,e^\pm},\qquad X_p\equiv\frac{n_p}{n_B},
% \end{align}
% where $n_B$ is the number density of baryons, $s_{\gamma,e^\pm}$ is the combined entropy density in photons, electrons, and positrons. During the Universe expansion, the comoving entropy and baryon number are conserved quantities; hence the ratio $n_B/s_{\gamma,e^\pm}$ is conserved. We have
% \begin{align}
% \frac{n_B}{s_{\gamma,e^\pm,}}=\left(\frac{n_B}{s_{\gamma,e^\pm}}\right)_{t_0}\!\!\!\!=\left(\frac{n_B}{s_{\gamma}}\right)_{t_0}\!\!\!\!=\left(\frac{n_B}{n_\gamma}\right)_{t_0}\left(\frac{n_\gamma}{s_{\gamma}}\right)_{t_0},
% \end{align}
% where the subscript $t_0$ denotes the present day value, and the second equality is obtained by observing that the present day $e^+e^-$-entropy density is negligible compared to the photon entropy density. We can evaluate the ratio by giving the present day baryon-to-photon ratio: $B/N_\gamma =n_B/n_\gamma= 0.605\times10^{-9}$ from Cosmic Microwave Background (CMB)~\cite{ParticleDataGroup:2022pth} and the entropy per particle for a massless boson: $(s/n)_{\mathrm{boson}}\approx 3.602$.

% The total entropy density of photons, electrons, and positrons can be written as
% \begin{align}\label{entropy_per_baryon}
% s_{\gamma,e^\pm}=\frac{2\pi^2}{45}g_\gamma\,T^3+\frac{\rho_{e^\pm}+P_{e^\pm}}{T}-\frac{\mu_e}{T}(n_{e^-}-n_{{e^+}}),
% \end{align}
% where $ \rho_{e^\pm}=\rho_{e^-}+\rho_{e^+}$ and $P_{e^\pm}=P_{e^-}+P_{{e^+}}$ are the total energy density and pressure of electrons and positron respectively.

% By incorporating Eq.~(\ref{charge_neutral_cond2}) and Eq.~(\ref{entropy_per_baryon}), the charge neutrality condition can be expressed as
% \begin{align}\label{charge_neutral_cond3}
% &\left[1+X_p\left(\frac{n_B}{n_\gamma}\right)_{t_0}\left(\frac{n_\gamma}{s_{\gamma}}\right)_{t_0}\frac{\mu_e}{T}\right]\frac{n_{e^-}-n_{{e^+}}}{T^3}\notag\\
% &\qquad\qquad\qquad=X_p\left(\frac{n_B}{n_\gamma}\right)_{t_0}\left(\frac{n_\gamma}{s_{\gamma}}\right)_{t_0} \left(\frac{2\pi^2}{45}g_\gamma+\frac{\rho_{e^\pm}+P_{e^\pm}}{T^4}\right).
% \end{align}

% Using Fermi distribution, the number density of electrons over positrons in the early Universe is given by
% \begin{align}\label{ee_density}
% n_{e^-}-n_{{e^+}}&=\frac{g_e}{2\pi^2}\left[\int_0^\infty\frac{p^2dp}{\exp{\left((E-\mu_e)\right)/T}+1}\right.\left.-\int_0^\infty\frac{p^2dp}{\exp{\left((E+\mu_e)/T\right)}+1}\right]\notag\\
% &=\frac{g_e}{2\pi^2}\,{T^3}\,\tanh(b_e)M_e^3\int_{1}^\infty \!\!\!\!\frac{ \eta \sqrt{\eta^2-1} d\eta}{1+\cosh(M_e\eta)/\cosh(b_e)},
% \end{align}
% where we have introduced the dimensionless variables as follows: 
% \begin{align}\label{Variables}
% \eta=\frac{E}{m_e},\qquad M_e=\frac{m_e}{T},\qquad b_e=\frac{\mu_e}{T}.
% \end{align}
% Substituting Eq.~(\ref{ee_density}) into Eq.~(\ref{charge_neutral_cond3}) and giving the value of $X_p$, then the charge neutrality condition can be solved to determine $\mu_e/T$ as a function of $M_e$ and $T$. 
% %Fig~~~~~~~~~~~~~~~~~~~~~~~~~~~~~~~~~~~~~~~~~~~~~~~~~~~~~
% \begin{figure}[ht]
% \begin{center}
% %\includegraphics[width=0.95\linewidth]{Chemical_Plasma}
% %\includegraphics[width=0.95\linewidth]{Density_Plasma002}
% \includegraphics[width=0.95\linewidth]{plots/chap03BBN/May152023_EPDensity_Chemical}
% \caption{Left axis: The chemical potential of electrons as a function of temperature by numerically solving Eq. (\ref{charge_neutral_cond3}) with $n_p/n_B=0.878$ and $n_B/n_\gamma=6.05\times10^{-10}$. Right axis: the ratio of electron (positron) number density to baryon density as a function of temperature. The solid blue line is the electron density, the red dashed line is the positron density, and the green dotted line is the number density with $\mu_e=0$. When $T=20.3\,\mathrm{keV}$ (the purple vertical line) positron density decreases rapidly because of the annihilation. The vertical black dotted lines represent BBN temperature range $86\,\mathrm{keV}>\mathrm{T_{BBN}}>50\,\mathrm{keV}$.}
% \label{BBN_Electron}
% \end{center}
% \end{figure}
%~~~~~~~~~~~~~~~~~~~~~~~~~~~~~~~~~~~~~~~~~~~~~~~~~~~~~

% In Fig.~\ref{BBN_Electron} (left axis), we solve Eq.~(\ref{charge_neutral_cond3}) numerically and plot the electron chemical potential as a function of temperature with the following parameters: proton concentration $X_p=0.878$ from observation~\cite{ParticleDataGroup:2022pth} and $n_B/n_\gamma=6.05\times10^{-10}$ from CMB. We can see the value of chemical potential is comparatively small $\mu_e/T\approx10^{-6}\sim10^{-7}$ during the BBN temperature range, implying an equal number of electrons and positrons in plasma. The ratio of electron (positron) number density to baryon density shows that the Universe was filled with an electron-positron rich plasma during the accepted BBN temperature range. For example, when the temperature is around $T=70\,\mathrm{keV}$, the density of electrons and positrons is comparatively large $n_{e^\pm}\approx10^7\,n_B$. At $90$\,keV, the electron and positron density is near the solar core density, see Fig.~19 in Ref.~\cite{Rafelski:2023emw}. Later when the temperature is around $T=20.3\,\mathrm{keV}$, the positron density decreases, transforming the pair-plasma to an electron-baryon plasma.


%%%%%%%%%%%%%%%%%%%%%%%%%%%%%%%%%%%%%%%%%%%%%%%%%%%%%
%\subsubsection\label{sec:relax}
\paragraph{The damping rate in electron-positron plasma:}
To find the damping rate in the BBN electron-positron plasma, we considered the major scatterings between photons and $e^+e^-$ pairs: inverse Compton scattering, M{\o}ller scattering, and Bhabha scattering
\begin{align}
&e^\pm+\gamma\longrightarrow e^\pm+\gamma,\qquad e^\pm+e^\pm\longrightarrow e^\pm+e^\pm,\qquad e^\pm+e^\mp\longrightarrow e^\pm+e^\mp\,.
\end{align}
% In general, to evaluate the reaction rate in two-body reaction $1+2\rightarrow3+4$ in the Boltzmann approximation we can use reaction cross-section $\sigma(s)$ and the relation from Ref.~\cite{Letessier:2002ony}:
% \begin{align}\label{GeneralRate}
% R_{12\rightarrow34}=\frac{g_1g_2}{32\pi^4}\frac{T}{1+I_{12}}\!\!\int^\infty_{s_{th}}\!\!\!\!ds\,\sigma(s)\frac{\lambda_2(s)}{\sqrt{s}}\!K_1\!\!\left({\sqrt{s}}/{T}\right),
% \end{align}
% where $K_1$ is the first-order Bessel function and the K\"{a}ll\'{e}n function $\lambda_2(s)$ is defined as
% \begin{align}
% \lambda_2(s)=\left[s-(m_1+m_2)^2\right]\left[s-(m_1-m_2)^2\right],
% \end{align}
% with $m_{1/2}$ and $g_{1/2}$ are the masses and degeneracies of initial interacting particles. The factor $1/(1+I_{12})$ is introduced to avoid double counting of indistinguishable pairs of particles, we have $I_{12}=1$ for identical particles and $I_{12}=0$ otherwise. 

% The two-body cross-section can be obtained by averaging the matrix element over the Mandelstam variable $t$. We have
% \begin{align}
% &\sigma_{e^\pm\gamma}=\frac{1}{16\pi(s-m_e^2)^2}\int^0_{-(s-m_e^2)^2/s}\!\!\!\!\!\!\!\!\!\!\!\!\!\!\!\!\! dt\, |M_{e^\pm\gamma}|^2,\\
% &\sigma_{e^-e^-}=\frac{1}{16\pi s(s-4m_e^2)}\int^0_{-(s-4m_e^2)}\!\!\!\!\!\!\!\!\!\!\!\!\!\!\!\!\!dt\, |M_{e^\pm e^\pm}|^2,\\
% &\sigma_{e^-e^+}=\frac{1}{16\pi s(s-4m_e^2)}\int^0_{-(s-4m_e^2)}\!\!\!\!\!\!\!\!\!\!\!\!\!\!\!\!\!dt\, |M_{e^\pm e^\mp}|^2.
% \end{align}
% The matrix element associated with inverse Compton scattering is given by~\cite{Kuznetsova:2009bq, Kuznetsova:2011wt}
% \begin{align}
% |M_{e^\pm\gamma}|^2\!&=32 \pi^2\alpha^2\bigg[4\left(\frac{m_e^2}{m_e^2-s}+\frac{m_e^2}{m_e^2-u}\right)^2\notag\\
% & -\frac{4m_e^2}{m_e^2-s}-\frac{4m_e^2}{m_e^2-u} -
% \frac{m_e^2-u}{m_e^2-s} -\frac{m_e^2-s}{m_e^2-u}\bigg],
% \end{align}
% and for M{\o}ller and Bhabha scattering we have respectively
% \begin{align}
% &|M_{e^{-}e^{-}}|^{2}\!=64\pi^{2}\alpha^{2}\bigg[
% \frac{s^{2}+u^{2}+8m_e^{2}(t-m_e^{2})}{2(t-m^2_{\gamma})^{2}} \notag\\ 
% &+\frac{{s^{2}+t^{2}}+8m_e^{2}
% (u-m_e^{2})}{2(u-m_{\gamma}^2)^{2}} + \frac{\left( {s}-2m_e^{2}\right)\left({s}-6m_e^{2}\right)}
% {(t-m_{\gamma}^2)(u-m_{\gamma}^2)} \bigg],
% \end{align}
% and
% \begin{align}
% &|M_{e^- e^+}|^{2}=64\pi^{2}\alpha^{2}
% \bigg[\frac{s^{2}+u^{2}+8m_e^{2}(t-m_e^{2})}{2(t-m^2_{\gamma})^{2}}\notag\\
% &+\frac{u^{2}+t^{2}+8m_e^{2}
% (s-m_e^{2})}{2(s-m^2_{\gamma})^{2}} + \frac{\left({u}-2m_e^{2}\right)\left({u}-6m_e^{2}\right)}
% {(t-m^2_{\gamma})(s-m^2_{\gamma})} \bigg],
% \label{M_fi_b}
% \end{align}
% where $s, t, u$ are the Mandelstam variables and we included the photon mass $m_\gamma$ to avoid singularity in reaction matrix elements. The photon mass in plasma is equal to the plasma frequency, in our case we have~\cite{Kislinger:1975uy}
% \begin{equation}
% m^2_\gamma=\omega^2_{p}=8\pi\alpha\int\frac{d^3p_e}{(2\pi)^3}\left(1-\frac{p_e^2}{3E_e^2}\right)\frac{f_- +f_+}{E_e},
% \end{equation}
% where $f_-$ and $f_+$ are the single-particle distribution functions for electrons and positrons respectively, and $E_e=\sqrt{p_e^2+m^2_e}$ is the electron energy. In the BBN temperature range $m_e\gg T$ and if we consider the non-relativistic electron and positron plasma
% \begin{align}
% m^2_\gamma=\frac{4\pi\alpha}{2m_e}\left(\frac{2m_eT}{\pi}\right)^{3/2}e^{-m_e/T}\cosh\left(\frac{\mu_e}{T}\right).
% \end{align}

% Substituting the cross-sections into Eq.~(\ref{GeneralRate}), the thermal reaction rate per volume for inverse Compton scattering can be written as
% \begin{align}
% R_{e^\pm\gamma}=\frac{g_eg_\gamma}{16\left(2\pi\right)^5}T\int_{m_e^2}^\infty\!\!\!\!ds\frac{K_1(\sqrt{s}/T)}{\sqrt{s}}\int^0_{-(s-m_e^2)^2/s}\!\!\!\!\!\!\!\!\!\!\!\!\!\!\!\!\!\!\!\!\!\!dt\, |M_{e^\pm\gamma}|^2,
% \end{align} 
% for M{\o}ller and Bhabha scattering we have
% \begin{align}
% &R_{e^\pm e^\pm}=\frac{g_eg_e}{16\left(2\pi\right)^5}T\!\!\int_{4m_e^2}^\infty\!\!\!\!ds\frac{K_1(\sqrt{s}/T)}{\sqrt{s}}\int^0_{-(s-4m_e^2)}\!\!\!\!\!\!\!\!\!\!\!\!\!\!\!\!\!\!\!\!\!\!dt\,|M_{e^\pm e^\pm}|^2,\\
% &R_{e^\pm e^\mp}=\frac{g_eg_e}{16\left(2\pi\right)^5}T\!\!\int_{4m_e^2}^\infty\!\!\!\!ds\frac{K_1(\sqrt{s}/T)}{\sqrt{s}}\int^0_{-(s-4m_e^2)}\!\!\!\!\!\!\!\!\!\!\!\!\!\!\!\!\!\!\!\!\!\!dt\,|M_{e^\pm e^\mp}|^2.
% \end{align}
\begin{figure}[h!]
\begin{center}
\includegraphics[width=0.95\linewidth]{plots/chap03BBN/May152023Kappa_EPPlasma002}
%\includegraphics[width=0.95\linewidth]{May152023Kappa_EPPlasma}
\caption{\cccite{Grayson:2023flr}, adapted from Ref.~\cite{Grayson:2023flr} and thesis of C.T.Yang \cite{Yang:2024ret}. The relaxation rate $\kappa$ as a function of temperature in non-relativistic electron-positron plasma. We present reaction rates for M{\o}ller scattering $R_{e^\pm e^\pm}$ (blue dashed line), Bhabha scattering $R_{e^\pm e^\mp}$ (red dashed line), and inverse Compton scattering $R_{e^\pm \gamma}$ (green dashed line). The average relaxation rate from Eq.~(\ref{Kappa}) is shown as a black solid line. The vertical black dotted lines represent BBN temperature range $86\,\mathrm{keV}>\mathrm{T_{BBN}}>50\,\mathrm{keV}$ during which the average relaxation rate is $\kappa=10\sim12$ keV. The dominant reactions during BBN are the M{\o}ller and Bhabha scatterings. The purple solid line represents the Debye mass given by Eq.~(\ref{eq:mL})}.
\label{RelaxationRate_fig}
\end{center}
\end{figure}
C. T. Yang calculated these reaction rates by considering in-plasma tree-level QED scatterings using an infrared cutoff at $\omega_p$, which is the temperature-dependent effective photon mass in a plasma \cite{Yang:2024ret}.
In Fig.~\ref{RelaxationRate_fig}, we show the reaction rates $R$ for M{\o}ller, Bhabha, and inverse Compton scattering as a function of temperature. For temperatures $T>12.0$ keV, the dominant reactions in plasma are M{\o}ller and Bhabha scatterings between electrons and positrons. Thus, we can neglect the inverse Compton scattering in the BBN temperature range.
We then use these scattering rates in plasma to find the average damping rate:
\begin{align}\label{Kappa}
\kappa=\frac{R_{e^\pm e^\pm}+R_{e^\pm e^\mp}+R_{e^\pm\gamma}}{\sqrt{n_{e^-}n_{e^+}}}\approx\frac{R_{e^\pm e^\pm}+R_{e^\pm e^\mp}}{\sqrt{n_{e^-}n_{e^+}}},
\end{align}
where we neglect the inverse Compton scattering during BBN as discussed. The density function ${\sqrt{n_{e^-}n_{e^+}}}$ in the Boltzmann limit is given by
\begin{align}
{\sqrt{n_{e^-}n_{e^+}}}=\frac{g_e}{2\pi^3}T^3\left(\frac{m_e}{T}\right)^2K_2(m_e/T).
\end{align}
In Fig.~\ref{RelaxationRate_fig} that the total damping rate $\kappa$ given by Eq.~(\ref{Kappa}) is approximately constant $\kappa=10-12$ keV during the BBN. This value is much larger than the Debye mass defined below in Eq.~(\ref{eq:mL}). Below $T<20.3$ keV, the relaxation rate $\kappa$ decreases rapidly because the positrons disappear. At $T=12$\,keV, the inverse Compton scattering of remaining electrons becomes the dominant scattering process. 

\paragraph{Electron positron plasma screening in BBN:}\label{sec:Discussion}

In this chapter, we review \cite{Grayson:2023flr}, which applies the non-relativistic longitudinal polarization function to study the dynamics of the electron-positron plasma in the early Universe. In particular, we discussed the damping rate, the electron-positron to baryon density ratio, and their potential implications for Big Bang Nucleosynthesis (BBN) through screening within linear response theory. We derived an approximate analytic formula for the potential of a moving heavy charge in a collisional plasma in \req{eq:pos_point_DDS} describing screening effects previously found only numerically \cite{Hwang:2021kno}. Our analytic formula can be readily used to estimate the effect of screening on thermonuclear reactions using \req{eq:DDSenhance}. The correction to thermonuclear reactions due to damped-dynamic screening is small due to the low velocity of nuclei and a large amount of collisional scattering. This is in line with the findings of \cite{Hwang:2021kno}, who conclude that even though the densities are large, they are not enough to modify the potential at short distances related to screening. The analytic expression we find for the nuclear reaction rate enhancement \req{eq:DDSenhance} in a collisional plasma could be useful in other fusion environments such as stellar fusion and laboratory fusion experiments, such as those discussed in ~\cite{Labaune:2013dla,Margarone:2022mdpi}.

Overall we were very surprised to find that the screening effects in BBN were so small even in the static case, considering that the number densities present during BBN are $\sim 10^4$ times normal matter. If we compare this to screening effects on Earth, we can see that although plasmas occur at lower densities, they also occur in much colder environments. The strength of the screening effect is related to the Debye mass
\begin{equation}
m_D^2 \sim \frac{n_\text{eq} }{T}\,,
\end{equation}
which is on the order of a few keV during BBN. On earth, $n_\text{eq}$ is decreased by $\sim 10^4$, but T is decreased by $\sim 10^6$. Thus, we would expect to see similar, if not larger, screening effects on Earth. For instance, the Debye screening length in extracellular fluid in the body is 8 \AA ngstrom \cite{Wennerstrom:2020}, only a factor of $\sim 20$ times larger than the Debye length during BBN. We can have these large densities at low temperatures on earth due to gravity's agglomeration of matter in the universe.

\subsubsection{The short-range screening potential}
In \cite{Grayson:2023flr}, a proposal is made to study the short-range potential relevant to quantum tunneling in thermonuclear reactions. Since the Gamow energy at which nuclei are most likely to tunnel is above the thermal energy, the portion of the screening potential relevant for tunneling does not satisfy the "weak-field" limit where the electromagnetic energy is small compared to the thermal energy
\begin{equation}
 \frac{q \phi(x)}{T} \ll 1\,.
\end{equation}
When this condition is not satisfied one must consider the full equilibrium distribution when calculating the short-range potential \cite{Hakim:1967prd,DeGroot:1980dk}
\begin{equation}\label{eq:Boltz}
 f_B^\pm(x,p) = e^{-(p_0\pm e\phi(x))/T}\,.
\end{equation}
The $e\phi$ term in the exponential accounts for the change in energy of a charge in the plasma due to its presence in an external field. For this equilibrium distribution, a linear response is no longer possible since the equilibrium distribution depends on the external electromagnetic field. In equilibrium one can find the static screening potential for strong electromagnetic fields using the nonlinear Poisson-Boltzmann equation,
\begin{equation}\label{eq:Poisson-Boltz}
 -\nabla^2 e\phi_{(\text{eq})}(x)/T +m_D^2\sinh\left[e\phi_{(\text{eq})}(x)/T\right] =e\rho_\mathrm{ext}(x)/T\,.
 \end{equation}
This equation has a well-known solution for an infinite sheet which we used to argue the importance of strong screening in BBN. 
In a future publication, we will solve the Poisson-Boltzmann equation with strong screening to calculate the short-range screening potential in BBN. We note that the toy model in \cite{Grayson:2023flr} overestimates strong screening effects for two reasons: an infinite sheet has a constant electric field requiring more polarizing charge density to screen the field, and the Boltzmann distribution in \req{eq:Boltz} does not account for the stacking of electron-positron states when the density of electrons and positrons becomes very large near the nucleus. Both of these effects significantly reduce the effect of strong screening on reaction rates, but at the time of writing, it seems that strong screening will create a larger effect on nuclear reaction rates than damped-dynamic screening. Predicting enhanced screening may be relevant for the anomalous screening observed in the measurements of astrophysical S(E) factors \cite{Zhang:2020nuc}.

%%%%%%%%%%%%%%%%%%%%%%%%%%%%%%%%%%%%%%%
\paragraph{Dusty plasmas}
In \cite{Grayson:2023flr} were very interested in finding BBN plasma had similar properties to planetary and space dusty plasma theory~\cite{Montgomery:1970jpp,Stenflo:1973,Shukla:2002ppcf,Lampe:2000pop}. The large distance behavior of the damped-dynamic BBN screened potential in \req{eq:pos_point_DDS} matched that of slowly moving dust particles in plasma \cite{Stenflo:1973}. Dusty plasma theory studies many effects not currently included in BBN plasma studies, including dust charging, dust acoustic waves, dust instabilities, and structure formation (dust crystals)~\cite{Shukla:2002ppcf}. We expect that these results can be ported to the nuclear light element dust dynamics in the primordial $e^-e^+\gamma$ QED plasma. This interdisciplinary connection, which has not been recognized previously, could have substantial implications for our understanding of the evolution of matter in the early Universe.

%%%%%%%%%%%%%%%%%%%%%%%%%%%%%%
\paragraph{Structure formation in the early universe}
One may note that the portion of the screening potential in the direction of motion is slightly binding \reff{fig:dynamiclinear}.
Given recent observations from JWST \cite{Ferreira:2023jwst}, one could speculate that this polarization binding effect could lead to the early seeding of matter structure in the Universe. In dusty plasmas, this happens due to an attractive portion of the internuclear (charged dust) potential, which favors a specific average separation between charged dust particles. From the plasma parameters calculated in \cite{Grayson:2023flr} the depth of the binding potential is on the order of 10's of electron volts. While this is more energy than the usual molecular bond, the BBN temperature is likely too large for these binding energies to play a role in dynamics. Given the potential in \req{eq:pos_point_DDS}, calculating if a molecular bound state exists is the subject of future work.
%%%%%%%%%%%%%%%%%%%%%%%%%%%%%%%%


\subsection{REMOVE: Summary and Outlook}
\label{sec:SummOut}
THIS IS FOR LAST SECTION COLLECTION OF MATERIAL NOT FIT TO USE IN INTRO

\subsubsection{QGP in the early universe}
 QGP filled the universe  after the Big Bang \cite{Rafelski:2013yka}. One could study the QCD potential of heavy quarks propagating through QGP in heavy ion collisions or the early universe. In dynamic screening of heavy quarks, it is possible that a long-range attractive potential also arises, leading to the formation of inhomogeneities in the early universe during the QGP epoch. In QCD, the transport equations are complicated by their non-abelian nature. Since gluons can couple to the external field of heavy quarks, they would contribute non-trivially to the transport equations. We expect we would also have to use strong-field kinetic theory due to the large QCD coupling $\alpha_s$. Kinetic theory in QGP is discussed in detail in \cite{Mrowczynski:2016etf}.
%%%%%%%%%%%%%%%%%%%%%%%%%%%%%%%%%%%%%%%%%%%%%%%%%%%%%%%%%%%%%%%%%%%
% {Composition
%%%%%%%%%%%%%%%%%%%%%%%%%%%%%%%%%%%%%%%%%%%%%%%%%%%%%%%%%%%%%%%%%%%%%%
%%%%%%%%%%%%%%%%%%%%%%%%
% Chapters from Cheng Tao Yang's dissertation
\section{Particles and plasma in the Universe} \label{Introduction}
%{Introduction to cosmology and overview}
In this chapter, we will introduce the fundamental concepts in cosmology for us to explore the properties of the Universe during the `first hour'. I will first present the standard cosmological Friedmann-Lemaitre-Robertson-Walker (FLRW) model, then introduce the general Fermi/Bose distribution with and its application in the early Universe. Finally I present an overview of Universe evolution from $300\,\mathrm{MeV}>T>0.02\,\mathrm{MeV}$.
The Natural unit $c=\hbar=k_{B}=1$ is used throughout the thesis for discussion.
%{Introduction\daggerfootnote{This chapter has been published previously as \citet{Gottbrath1999}.}}

%~~~~~~~~~~~~~~~~~~~~~~~~~~~~~~~~~~~~~~~~~~~~~~~~~
%The modern observation in cosmology has demonstrated that the observed universe is highly symmetric in its large-scale structure and 

%~~~~~~~~~~~~~~~~~~~~~~~~~~~~~~~~~~~~~~~~~~~~~~~~~

\subsection{Approaching equilibrium: Fermi/Bose distributions}

In the early Universe, the reaction rates of particles in the cosmic plasma were much greater than the Universe expansion rate $H$. Therefore, the local thermal equilibrium has been maintained. Assuming the particles are in thermal equilibrium, the dynamical information can be obtained from the single-particle distribution function. The general relativistic covariant Fermi/Bose momentum distribution can be written as
\begin{align}
f_{F/B}(\Upsilon_i,p_i)=\frac{1}{\Upsilon^{-1}_i\exp{\left[(u\cdot p_i-\mu_i)/T\right]}\pm1}
\end{align}
where the plus sign applies for fermions, and the minus sign for bosons. The Lorentz scalar $(u_i\cdot p_i)$ is a scalar product of the particle four momentum $p^\mu_i$ with the local four vector of velocity $u^\mu$. In the absence of local matter flow, the local rest frame is the laboratory frame 
\begin{align}
u^\mu=\left(1,\vec{0}\right),\,\,\,\,\,\,\,\,\, p^\mu_i=\left(E_i,\vec{p}_i\right).
\end{align}  
The parameter $\Upsilon_i$ is the fugacity of a given particle which describes the pair density and it is the same for both particles and antiparticles. For $\Upsilon_i=1$ the distribution maximizes the entropy content at a fixed particle energy. The parameter $\mu_i$ is the chemical potential for a given particle which is associated to the density difference between particles and antiparticles. 

In general there are two types of chemical equilibriums associated with the chemical parameters $\Upsilon$ and $\mu$. We have:
\begin{itemize}
\item Absolute chemical equilibrium:\\
The absolute chemical equilibrium is the level to which energy is shared into accessible degrees of freedom, e.g. the particles can be made as energy is converted into matter.
The absolute equilibrium is reached when the phase space occupancy approaches unity $\Upsilon\to1$. 
 \item Relative chemical equilibrium:\\
 The relative chemical equilibrium is associated with the chemical potential $\mu$ which involves reactions that distribute a certain already existent element/property among different accessible compounds. 
 \end{itemize}
The dynamics of absolute chemical equilibrium, in which energy can be converted to and from particles and antiparticles, is especially important. The consequences for the energy conversion to from particles/antiparticle can be seen in the first law of thermodynamics by introducing the general chemical potential $\mu_N$ for particle and $\mu_{\bar{N}}$ for antiparticle as follows:
\begin{align}
\mu_N\equiv\mu+T\ln\Upsilon,\qquad{\mu_{\bar{N}}}\equiv{-\mu}+T\ln\Upsilon.
\end{align}
Then the first law of thermodynamics can be written as
\begin{align}
dE&=-PdV+TdS+{\mu_N}dN+{\mu_{\bar{N}}}d{\bar{N}}
\\&=-PdV+TdS+{\mu}(dN-d{\bar{N}})+T\ln{\Upsilon}(dN+d{\bar{N}}).
\end{align}
It shows that the chemical potential $\mu$ is the energy required to change the difference between particles and antiparticles, and the $T\ln\Upsilon$ is the energy required to change the total number of particle and antiparticle, and the fugacity $\Upsilon$ is the parameter to adjust the energy.

\subsubsection{Boltzmann equation and particle freeze-out}
The Boltzmann equation describes the evolution of distribution function f in phase space. The Boltzmann equation in the FLRW universe can be written as
\begin{align}
\frac{\partial f}{\partial t}-\frac{\left(E^2-m^2\right)}{E}H\frac{\partial f}{\partial E}=\frac{1}{E}\sum_{q}\mathcal{C}_q[f],
\end{align}
where $H=\dot{a}/a$ is the Hubble parameter. Due to homogeneity and isotropy of the Universe, the distribution function depends on time $t$ and energy $E=\sqrt{p^2+m^2}$ only. The collision term $\sum_qC_q$ represents all elastic and inelastic interactions and $q$ labels the corresponding physical process. In general, the collision term is proportional to the relaxation time for given collision as follows \cite{ANDERSON1974466}
\begin{align}
\frac{1}{E}\mathcal{C}_q[f]\propto\frac{1}{\tau_{rel}}
\end{align}
where $\tau_{rel}$ is the relaxation time for the reaction, which is on the order of magnitude of time for the reaction to reach chemical equilibrium. 


As the Universe expands, the collision term in the Boltzmann equation competes with the Hubble term. In general, a given particle freeze-out from the cosmic plasma when its interaction rate $\tau_{rel}^{-1}$ becomes smaller than the Hubble expansion rate
\begin{align}
H\geqslant\tau_{rel}^{-1}.
\end{align}
When this happens, the particle's interactions are not rapid enough to maintain thermal distribution, either because the density of particles becomes so low that the chances of any two particles meeting each other becomes negligible, or because the particle energy becomes too low to interact. The freeze-out process can be categorized into three distinct stages based on the type of freeze-out interactions, we have~\cite{Birrell:2012gg,Rafelski:2023emw}:

\begin{itemize}
\item Chemical freeze-out :\\
As the Universe expands and the temperature drops, the rate of the inelastic scattering (e.g. production and annihilation reaction) that maintain the equilibrium density becomes smaller than the expansion rate. At this point, the inelastic scattering ceases, and a relic population of particles remain. Prior to the chemical freeze-out temperature, number changing processes are significant and keep the particle in thermal equilibrium, implying that the distribution function has the usual Fermi-Dirac form 
\begin{equation}\label{equilibrium}
f_{th}(t,E)=\frac{1}{\exp[(E-\mu)/T]+1},\qquad \text{ for } T(t)> T_{ch}.
\end{equation}
where $T_{ch}$ represents the chemical freeze-out temperature.


\item Kinetic freeze-out:\\
After chemical freeze-out, particles  still scatter elastically from other particles and keep thermal equilibrium in the primordial plasma. As the temperature drops, the rate of elastic scattering reaction that maintain the thermal equilibrium become smaller than the expansion rate. At that time, elastic scattering processes cease, and the relic particles do not interact with other particles in the primordial plasma anymore. Before the kinetic freeze-out, the distribution function has the form
\begin{equation}\label{kinetic_equilib}
f_k(t,E)=\frac{1}{\Upsilon^{-1}\exp[(E-\mu)/T]+1},\qquad \text{ for }T_f< T(t)< T_{ch},
\end{equation}
where $T_f$ represents the kinetic freeze-out temperature. The generalized fugacity $\Upsilon(t)$ controls the occupancy of phase space and is necessary once $T(t)<T_{ch}$ in order to conserve particle number.


\item{Free streaming:}\\
After kinetic freeze-out, the particles have fully decoupled from the primordial plasma, and thereby ceased influencing the dynamics of the Universe and become free-streaming. The Einstein-Vlasov equation can be solved \cite{choquet2008general} and the free-streaming momentum distribution can be written as \cite{Birrell:2012gg}
\begin{equation}\label{free_stream_dist}
f_{fs}(t,E)=\frac{1}{\Upsilon^{-1}\exp{\left[\sqrt{\frac{E^2-m^2}{T_{fs}^2}+\frac{m^2}{T^2_f}}-\frac{\mu}{T_f}\right]+1}},\quad T_{fs}(t)=\frac{T_fa(t_k)}{a(t)},
\end{equation}
where the free-streaming effective temperature $T_{fs}$ is obtained by redshifting the temperature at kinetic freeze-out. If a massive particle (e.g. dark matter) freeze-out from cosmic plasma in the nonrelativistic regime, $m\gg T_f$. We can use the
Boltzmann approximation, and the free-streaming distribution for nonrelativistic particle becomes
\begin{align}
&f^B_{fs}(t,p)=\Upsilon\,e^{-(m+\mu)/T_f}\exp\left[-\frac{1}{ T_{eff}}\frac{p^2}{2m}\right],\quad T_{eff}=\left(\frac{a(t_f)}{a(t)}\right)^2T_f,
\end{align}
where we define the effective temperature $T_{eff}$ for massive free-streaming particle. In this scenario, the effective temperature for massive particles decreases faster than the Universe temperature cools. It's worth emphasizing the different temperatures between cold free-streaming particles and hot cosmic plasma would affect the evolution of the early Universe and require more detailed study. 
\end{itemize}

The division of the freeze-out process into these three regimes is a simplification. However, it is a very useful approximation in the study of cosmology~\cite{Mangano:2005cc,Birrell:2014gea}. For detailed discussion, see \cite{Birrell:2012gg,Rafelski:2023emw}.




\subsection{Thermodynamics of the Early Universe}
In the case of local thermal equilibrium, the laws of thermodynamics can provide a framework for understanding the behavior of particle's energy density, pressure, number density and entropy in the early Universe.

Using the relativistic covariant Fermi/Bose momentum distribution, the corresponding energy density, pressure, and number densities for particle species $i$ are given by
\begin{align}
\rho_i&=g_i\int\!\!\frac{d^3p}{(2\pi)^3}Ef_{F/B}=\frac{g_i}{2\pi^2}\!\int_{m_i}^\infty\!\!\!dE\,\frac{E^2\left(E^2-m_i^2\right)^{1/2}}{\Upsilon_i^{-1}e^{(E-\mu_i)/T}\pm 1},\label{energy_density}\\[0.2cm]
P_i&=g_i\int\!\!\frac{d^3p}{(2\pi)^3}\frac{p^2}{3E}f_{F/B}=\frac{g_i}{6\pi^2}\!\int_{m_i}^\infty\!\!\!dE\,\frac{\left(E^2-m_i^2\right)^{3/2}}{\Upsilon_i^{-1} e^{(E-\mu_i)/T}\pm 1},\label{Pressure_density}\\[0.2cm]
n_i&=g_i\int\!\!\frac{d^3p}{(2\pi)^3}f_{F/B}=\frac{g_i}{2\pi^2}\!\int_{m_i}^\infty\!\!\!dE\,\frac{E(E^2-m_i^2)^{1/2} }{\Upsilon_i^{-1}e^{(E-\mu_i)/T}\pm 1}
\label{number_density}
\end{align}
where $g_i$ is the degeneracy of the particle species. By including the fugacity parameter $\Upsilon_i$ allows us to characterize particle properties in nonchemical equilibrium situations.
On the other hand, the corresponding free-streaming energy density, pressure, and number densities can be written as
\begin{align}
\rho_i&=g_i\int\!\!\frac{d^3p}{(2\pi)^3}Ef_{fs}=\frac{g_i}{2\pi^2}\!\int_{m_i}^\infty\!\!\!dE\,\frac{E^2\left(E^2-m_i^2\right)^{1/2}}{\Upsilon_i^{-1}e^{\sqrt{p^2/T_{fs}^2+m_i^2 /T_f^2}-\mu_i/T_f}\pm 1},\label{free_energy_density}\\[0.2cm]
P_i&=g_i\int\!\!\frac{d^3p}{(2\pi)^3}\frac{p^2}{3E}f_{fs}=\frac{g_i}{6\pi^2}\!\int_{m_i}^\infty\!\!\!dE\,\frac{\left(E^2-m_i^2\right)^{3/2}}{\Upsilon_i^{-1}e^{\sqrt{p^2/T_{fs}^2+m_i^2 /T_f^2}-\mu_i/T_f}\pm1},\label{free_Pressure_density}\\[0.2cm]
n_i&=g_i\int\!\!\frac{d^3p}{(2\pi)^3}f_{fs}=\frac{g_i}{2\pi^2}\!\int_{m_i}^\infty\!\!\!dE\,\frac{E(E^2-m_i^2)^{1/2} }{\Upsilon_i^{-1}e^{\sqrt{p^2/T_{fs}^2+m_i^2 /T_f^2}-\mu_i/T_f}\pm1},
\label{free_number_density}
\end{align} 
which are different from the thermal equilibrium Eq.~(\ref{energy_density}), Eq.~(\ref{Pressure_density}), and Eq.~(\ref{number_density}), by replacing the mass by a time dependant effective mass $m\,T_{fs}(t)/T_f$ in the exponential.





Given the energy density, pressure, and number densities, the entropy density for particle species $i$ can be written as
\begin{align}\label{entropy}
\sigma_i=\frac{S_i}{V}=\left(\frac{\rho_i+P_i}{T}-\frac{\mu_i}{T}\,n_i\right).
\end{align}
In general the chemical potential is associated with the baryon number. Since the net baryon number density relative to the photon number density is of order $10^{-9}$. In this case, we can neglect the small chemical potential when calculating the total entropy density in the Universe. The total entropy density in the early Universe can be written as
\begin{align}
&\sigma=\sum_i\,\sigma_i=\frac{2\pi^2}{45}g^s_\ast\,T^3,\\
&g^s_\ast=\sum_{i=\mathrm{bosons}}g_i\left({\frac{T_i}{T_\gamma}}\right)^3B\left(\frac{m_i}{T_i}\right)+\frac{7}{8}\sum_{i=\mathrm{fermions}}g_i\left({\frac{T_i}{T_\gamma}}\right)^3F\left(\frac{m_i}{T_i}\right),
\end{align}
where $g^s_\ast$ counts the effective number of `entropy' degrees of freedom. The functions $B(m_i/T)$ and $F(m_i/T)$ are defined as
\begin{align}
&B\left(\frac{m_i}{T}\right)=\frac{45}{12\pi^4}\int^\infty_{m_i/T}\,dx\sqrt{x^2-\left(\frac{m_i}{T}\right)^2}\left[4x^2-\left(\frac{m_i}{T}\right)^2\right]\frac{1}{\Upsilon^{-1}_ie^x-1},\\
&F\left(\frac{m_i}{T}\right)=\frac{45}{12\pi^4}\frac{8}{7}\int^\infty_{m_i/T}\,dx\sqrt{x^2-\left(\frac{m_i}{T}\right)^2}\left[4x^2-\left(\frac{m_i}{T}\right)^2\right]\frac{1}{\Upsilon^{-1}_ie^x+1}.
\end{align}
In Fig.~\ref{EntropyDOF_Fig} we plot the $g^s_\ast$ as a function of temperature, the effect of particle mass threshold~\cite{Coc:2006rt} is considered in the calculation for all involved particles. When $T$ decreases below the mass of particle $T\ll m_i$, this particle species becomes nonrelativistic and the contribution to $g^s_\ast$ becomes negligible, creating the dependence on $T$ seen in Fig.\,\ref{EntropyDOF_Fig}.
%~~~~~~~~~~~~~~~~~~~~~~~~~~~~~~~~~~~~~~~~~~~~~~~~~~~~~~~~~~~~~~~~~~~~~~~~~~~~~~~~
\begin{figure}[t]
%\begin{center}
\centering
\includegraphics[width=\linewidth]
%{./plots/DOF_entropy.jpg}
{./plots/g_entropy.jpg}
\caption{The entropy degrees of freedom as a function of $T$ in the early Universe epoch after hadronization $10^{-2}\,\mathrm{MeV} \leqslant  T \leqslant 150 $\,MeV. When particle species becomes nonrelativistic $T\ll m_i$, the contribution to $g^s_\ast$ becomes negligible, as a result creating the dependence $g^s_\ast(T)$.The vertical lines represents the mass of particles: $m_e=0.511$ MeV, $m_\mu=105.6$ MeV, and pion average mass $m_\pi\approx138$ MeV. Adapted from thesis of C.T.Yang \cite{Yang:2024ret}.}
\label{EntropyDOF_Fig}  
\end{figure}
%~~~~~~~~~~~~~~~~~~~~~~~~~~~~~~~~~~~~~~~~~~~~~~~~~~~~~~~~~~~~~~~~~~~~~~~~~~~~~~~~


\subsubsection{Relation between time and temperature}
Considering the comoving entropy conservation, we have
\begin{align}
S=\sigma V\propto g^s_\ast T^3a^3=\mathrm{constant},
\end{align}
where $g^s_\ast$ is the entropy degree of freedom and $a$ is the scale factor. Differentiating the entropy with respect to time $t$ we obtain
\begin{align}
\left[\frac{\dot{T}}{g^s_\ast}\frac{dg^s_\ast}{dT}+3\frac{\dot{T}}{T}+3\frac{\dot{a}}{a}\right]g^s_\ast T^3a^3=0,\qquad \dot{T}=\frac{dT}{dt}.
\end{align}
Solving $dT/dt$ and taking the integral, the relation between time and temperature in early universe can be written as
\begin{align}\label{time}
t(T)=t_0-\int^T_{T_0}dT\frac{1}{HT}\left[1+\frac{T}{3g^s_\ast}\frac{dg^s_\ast}{dT}\right],\qquad H^2=\frac{8\pi G_N}{3}\rho_{tot}
\end{align}
where $T_0$ and $t_0$ represent the initial temperature and time respectively, $H$ is the Hubble parameter and $\rho_{tot}$ is the total energy density in early Universe. From Eq. (\ref{time}) we see that the cosmic time depends on the entropy degrees of freedom $g^\ast_s$, which are characterized by the relativistic components in the early Universe. In the temperature range we consider $300\,\mathrm{MeV}>T>0.02\,\mathrm{MeV}$ the Universe is radiation-dominated and $\Lambda$CDM model is not used in this epoch.

%~~~~~~~~~~~~~~~~~~~~~~~~~~~~~~~~~~~~~~~~~~~~~~~~~~~~~~~~~~~~~~~$

\subsubsection{The baryon-per-entropy density ratio}
An important assumption allowing us to explore the early Universe evolution is that following on the era of matter genesis both baryon and entropy content is conserved in the comoving volume. Both baryon and entropy density scale with the third power of the expansion parameter $a(t)$. Therefore the ratio of baryon number density to visible matter entropy density remains constant throughout the evolution of universe. We have
\begin{align}
\frac{n_B-n_{\overline{B}}}{\sigma}= \left.\frac{n_B-n_{\overline{B}}}{ \sigma}\right|_{t_0}=\mathrm{Const.}\;
\end{align}
The subscript $t_0$ denotes the present day condition, and $\sigma$ is the total entropy density.
The observation gives the present baryon-to-photon ratio ~\cite{ParticleDataGroup:2022pth} $5.8 \times 10^{-10} \leqslant(n_B-n_{\overline{B}})/n_\gamma\leqslant6.5\times10^{-10}$. This small value quantifies the matter-antimatter asymmetry in the present day Universe, and allows the determination of the present value of baryon per entropy ratio~\cite{Rafelski:2019twp,Fromerth:2002wb,Fromerth:2012fe}:
\begin{align}\label{BaryonEntropyRatio}
\left.\frac{n_B-n_{\overline{B}}}{ \sigma}\right|_{t_0}=\eta\left(\frac{n_\gamma}{\sigma_\gamma+\sigma_\nu}\right)_{\!t_0}\!\!\!\!=(8.69\pm0.05)\!\!\times\!\!10^{-11},\qquad \eta=\frac{(n_B-n_{\overline{B}})}{n_\gamma},
\end{align}
where the $\eta=(6.12\pm0.04)\times10^{-10}$~\cite{ParticleDataGroup:2022pth} is used in calculation. To obtain the ratio, we consider that the Universe today is dominated by photons and free-streaming massless neutrinos~\cite{Birrell:2012gg}, and $\sigma_\gamma$ and $\sigma_\nu$ are the entropy densities for photon and neutrino respectively. We have
\begin{align}
    \frac{\sigma_\nu}{\sigma_\gamma}=\frac{7}{8}\,\frac{g_\nu}{g_\gamma}\left(\frac{T_\nu}{T_\gamma}\right)^3\,\qquad\frac{T_\nu}{T_\gamma}=\left(\frac{4}{11}\right)^{1/3}
\end{align}
and the entropy-per-particle for massless bosons and fermions are given by~\cite{Fromerth:2012fe}
\begin{align}
s/n|_\mathrm{boson}\approx 3.60\,,\qquad
s/n|_\mathrm{fermion}\approx 4.20\,.
\end{align}
However, from the neutrino oscillation experiment, we know that the the neutrinos are not massless particles. 
The mass differences between neutrino mass eigenstates are~\cite{ParticleDataGroup:2022pth}:
\begin{align}
&\Delta{m}_{21}^2=7.39^{+0.21}_{-0.20}\times10^{-5}\,\mathrm{eV}^2,\\
&\Delta{m}_{32}^2=2.45^{+0.03}_{-0.03}\times10^{-3}\,\mathrm{eV}^2.
\end{align}
Neutrino mass eigenvalues can be ordered in the normal mass hierarchy ($m_1\ll m_2<m_3$) or inverted mass hierarchy ($m_3\ll m_1<m_2$). All three mass states remained relativistic until the temperature dropped below their rest mass. These results allow for the possibility that one mass eigenstate or two mass eigenstates of neutrinos may become non-relativistic today, which can affect the baryon-per-entropy ratio.




%~~~~~~~~~~~~~~~~~~~~~~~~~~~~~~~~~~~~~~~~~~~~~~~~~~~



\subsubsection{Nonequilibrium: departure from detailed balance}
Thermal equilibrium implies both chemical equilibrium (particles abundances are balanced) and kinetic equilibrium (energy is evenly distributed). In chemical equilibrium, the rates of the forward and reverse reactions are equal, resulting in a balance between production and annihilation/decay rates, which is called detailed balance. The chemical non-equilibrium can be achieved by breaking this detailed balance and leading to change in particle abundance over time. On the other hand, kinetic equilibrium is usually established much quicker and has less impact on the actual particle abundances.
The chemical nonequilibrium condition is more important than the kinetic equilibrium because it relates to the arrow of time for the particle reactions. 

The chemical nonequilibrium conditions in the early Universe are of general interest: they are understood to be prerequisite for the arrow of time dependent processes to take hold in the Hubble expanding Universe. The arrow of time plays an important role in the evolution of the early Universe, for example:
 1.) The Big Bang Nucleosynthesis (BBN)~\cite{Pitrou:2018cgg,Kolb:1990vq,Dodelson:2003ft,Mukhanov:2005sc}  the synthesis of light elements of  e.g. D, $^3$He, $^4$He, and $^7$Li are produced at temperatures around $86\,\mathrm{keV}>T_{BBN}>50\,\mathrm{keV}$. 
 2.) Baryogenesis is believed to occur at or before the Universe underwent electroweak phase transition~\cite{Kolb:1990vq} at a temperature $T\simeq 130$\, GeV, which generates the excess of baryon number compared to anti-baryon number in order to create the observed baryon number today.

When Universe expands and temperature cools down, the chemical non-equilibrium can be achieved by breaking the detailed balance between particle production reaction and annihilation/decay as follows:

1.) The particle production rate becomes slower than the rate of Universe expansion and the production reaction freezeout. Once the production reactions freezeout from the cosmic plasma, the corresponding detailed balance is broken and particle abundance decrease via the decay/annihilation reactions.
 

2.) The non-equilibrium can also be achieved when the production reaction slows down and is not able to keep up with decay/annihilation reaction. In this case, the Hubble expansion rate is much longer than the decay and production rate and is not relevant to the nonequilibrium process. The key factor is competition between production and decay/annihilation  which can result in chemical nonequilibrium in the early Universe.

\noindent We will investigate the nonequilibrium situation for bottom quarks and strang quarks in early universe and their application in Chapter~\ref{Bottom} and Chapter~\ref{Strangeness}, respectively.  

\subsection{Heavy quarks in cosmic plasma} \label{Bottom}

The primordial quark-gluon plasma (QGP) refers to the state of matter that existed in the early Universe, specifically for time $t\approx10\, \mathrm{\mu s}$ after the Big Bang. At that time the Universe was controlled by the strongly interacting particles: quarks and gluons. In this chapter, I study the heavy bottom and charm flavor quarks near to the QGP hadronization temperature $0.3\,\mathrm{GeV}>T>0.15\,\mathrm{GeV}$ and examine the relaxation time for the production and decay of bottom/charm quarks then show that the bottom quark nonequliibrium occur near to QGP –hadronization and create the arrow in time in the early Universe.

%~~~~~~~~~~~~~~~~~~~~~~~~~~~~~~~~~~~~~~~~~~~~~~~~~

\subsubsection{Overview of heavy quarks in primordial QGP}
%In this section we will focus on the following:
%\begin{itemize}
%    \item Review of primordial Quark-Gluon Plasma
%    \item Briefly remark the top quark
%\end{itemize}




In the QGP epoch, up and down $(u,d)$ (anti)quarks are effectively massless and remain in equilibrium via quark-gluon fusion. Strange $(s)$ (anti)quarks are in equilibrium via weak, electromagnetic, and strong interactions until $T\approx13$ MeV~\cite{Yang:2021bko}. The massive top $(t)$ (anti)quarks decay via the channel $t\to W+b$, with $\Gamma_t=1.4\pm0.2$\;GeV \cite{ParticleDataGroup:2018ovx} which implies that no bound state of top quarks have time to form. Given the large value of $\Gamma_t$ we realize that top quarks in hot QGP can be produced by the $ W+b\to t$ fusion process -- given the strength of this process there is no freeze-out of top quarks until $W$ itself freezes out. To address the top quarks in QGP, a dynamic theory for $W$ abundance is needed, a topic we will consider in the future. Finally, the bottom $(b)$ and charm $(c)$ quarks can be produced from strong interactions via quark-gluon pair fusion processes and disappear via weak interaction decays, and their abundance depends on the competition between the strong fusion and weak decay reaction rates.

The properties of QGP can be studied by experimental observations from high-energy heavy-ion collision experiments, such as  the Relativistic Heavy Ion Collider (RHIC) and the Large Hadron Collider (LHC). However, the conditions in the early Universe and those created in relativistic collisions are different. For example, the primordial QGP survives for about $10\mu$s in the cosmological Big Bang. On the other hand, the QGP formed in micro-bangs resulting from ultra-relativistic nuclear collisions has a lifespan of around  $10^{-23}$ s~\cite{Rafelski:2001hp}. Due to the considerably slower expansion rate of the Universe compared to quark production reactions and decays, in practicality, quark remained in equilibrium, and the quark fugacity is $\Upsilon=1$ during the QGP epoch.


However near the hadronization temperature, the heavy quarks abundance and deviations from chemical equilibrium have not yet been studied in great detail. In following we will focus on bottom and charm quarks. We will show that the bottom quarks can deviate from chemical equilibrium $\Upsilon\neq1$ by breaking the detailed balance between production and decay reactions of the quarks.


%~~~~~~~~~~~~~~~~~~~~~~~~~~~~~~~~~~~~~~~~~~~~~~~~~

\subsubsection{Bottom and Charm quark near QGP hadronization}
%In this section we will focus on the following:
%\begin{itemize}
%    \item Bottom/charm quarks production and decay in primordial QGP 
%    \item Dynamic equation for  bottom abundance (Stationary and non-stationary fugacity)
%    \item Nonstationary departure from detailed balance
%\end{itemize}
In the following we consider the temperature near QGP hadronization $0.3\,\mathrm{GeV}>T>0.15\,\mathrm{GeV}$, and study the bottom and charm abundance by examining the relevant reaction rates of their production and decay.
In thermal equilibrium the number density of light quarks can be evaluated in the massless limit, and we have\index{number density of quark}
\begin{align}\label{FermiN}
n_q=\frac{g_{q}}{2\pi^2}\,T^3 F(\Upsilon_q)\;, \quad F=\int_0^\infty \frac{x^2dx}{1+\Upsilon_q^{-1}e^x}\;,
\end{align}
where $\Upsilon_q$ is the quark fugacity. We have $ F(\Upsilon_q=1)=3\,\zeta(3)/2$ with the Riemann zeta function $\zeta(3)\approx1.202$.
The thermal equilibrium number density of heavy quarks with mass $m\gg T$ can be well described by the Boltzmann expansion of the Fermi distribution function, giving
\begin{align}\label{BoltzN}
n_{q}\!=\!\frac{g_{q}T^3}{2\pi^2}\sum_{n=1}^{\infty}\frac{(-1)^{n+1}\Upsilon_q^n}{n^4}\left(\frac{n\,m_{q}}{T}\right)^{\!2}\!K_2\left(\frac{n\,m_{q}}{T}\right),
\end{align} 
where $K_2$ is the modified Bessel functions of integer order $2$. In the case of interest, when $m\gg T$, it suffices to consider the Boltzmann limit and  keep the first term $n=1$ in the expansion. The first term  $n=1$ also suffices for both charmed $c$-quarks and bottom $b$-quarks, giving
\begin{align}
&n_{b,c}={\Upsilon_{b,c}\,}n^{th}_{b,c},\qquad n^{th}_{b,c}=\frac{g_{b,c}}{2\pi^2}\,T^3\left(\frac{m_{b,c}}{T}\right)^2\,K_2(m_{b,c}/T).
\end{align}
However, for strange $s$ quarks, several terms are needed. 
%~~~~~~~Figure1~~~~~~~~~~~~~~~~~~~~~~~~~~~~~~~~~~~~~~~~~~~~~~~~~~~~~~~~~~~~~~~~~~~~~~~~~~~~~~~~~
\begin{figure}[t]
\begin{center}
\includegraphics[width=0.9\textwidth]{./plots/bcQuarkDensity_new}
\caption{
The quark number density normalized by entropy density, as a function of temperature in the early Universe with $\Upsilon=1$. The $b$-quark mass parameters shown are $m_b=4.2\,\mathrm{GeV}$ (blue) dotted line, $m_b=4.7\,\mathrm{GeV}$ (black) solid line, and $m_b=5.2\,\mathrm{GeV}$ (red) dashed line. For  $c$-quark  $m_c=0.93\,\mathrm{GeV}$  (blue) dotted line, $m_c=1.04\,\mathrm{GeV}$ (black) solid line, and $m_c=1.15\,\mathrm{GeV}$ (red) dashed line. Adapted from thesis of C.T.Yang \cite{Yang:2024ret}.}
\label{number_entropy_b002}
\end{center}
\end{figure}
%~~~~~~~~~~~~~~~~~~~~~~~~~~~~~~~~~~~~~~~~~~~~~~~~~~~~~~~~~~~~~~~~~~~~~~~~~~~~~~~~~~~~~~~~~~~~~


In Fig.~\ref{number_entropy_b002} we show the equilibrium ($\Upsilon=1$) number density per entropy density  ratio as a function of temperature $T$ of quarks. The entropy density is given by Eq.~(\ref{entropy}) and only light particles contribute to the entropy density; thus the result we consider is independent of actual abundance of $c$, $b$ and other heavy particles. We evaluated the density-per-entropy ratio for  $m_b=4.2,\;4.7,\;5.2$\,GeV and $m_c=1.04\pm0.11$\,GeV. The $m_b\simeq 5.2\,\mathrm{GeV}$ is  a typical potential model mass used in modeling bound states of bottom, and $m_b=4.2,\,4.7\,\mathrm{GeV}$ is the current quark mass at low and high energy scales. In Fig.~\ref{number_entropy_b002} we see that the charm abundance in the domain of interest $0.3\,\mathrm{GeV}>T>0.15\,\mathrm{GeV}$ is about $10^4\sim\!\!10^{9}$ times greater than the bottom quarks. This implies that the small $b$,$\bar b$ quark abundance is embedded in a large background comprising all lighter $u,d,s,c$ quarks and antiquarks, as well as gluons $g$.

\subsubsection{Reaction rate for quarks production and decay}
In primordial QGP, the bottom and charm quarks can be produced from strong interactions via quark-gluon pair fusion processes and disappear via weak interaction decays. For production, we have the following processes
\begin{align}
 q+\bar{q}&\longrightarrow b+\bar b,\qquad q+\bar{q}\longrightarrow c+\bar c,\\
 g+g&\longrightarrow b+\bar b,\qquad g+g\longrightarrow c+\bar c,
\end{align}
for bottom and charm we have
\begin{align}
 &b\longrightarrow c+l+\overline{\nu_l}, \qquad b\longrightarrow c+q+\bar{q},\\
&c\longrightarrow s+l+\overline{\nu_l},\qquad c\longrightarrow s+q+\bar{q},
\end{align}
for their decay. In following we will calculate the production rate and decay rate for bottom and charm quarks and compare to the Universe expansion rate. We will show that in the epoch of interest to us the characteristic Universe expansion time $1/H$ is much longer than the lifespan and production time of the bottom/charm quark. In this case, the dilution of bottom/charm quark due to the Universe expansion is slow compare to the the strong interaction production, and the weak interaction decay of the bottom/charm. 
 



%~~~~~~~~~~~~~~~~~~~~~~~~~~~~~~~~~~~~~~~~~~~~~~~~~~~~~~~~~~

\paragraph{Quark production rate via strong interaction}
For the quark-gluon pair fusion processes\index{bottom quark!production rate}
%\begin{align}
% q+q&\longrightarrow b+\bar b,\qquad q+q\longrightarrow c+\bar c,\\
% g+g&\longrightarrow b+\bar b,\qquad g+g\longrightarrow c+\bar c,
%\end{align}
the evaluation of the lowest-order Feynman diagrams yields the cross sections~\cite{Letessier:2002ony}:
\begin{align}
&\sigma_{q\bar{q}\rightarrow b\bar{b},c\bar{c}}=\frac{8\pi\alpha_s^2}{27s}\left(1+\frac{2m_{b,c}^2}{s}\right)w(s),\,\qquad w(s)=\sqrt{1-{4m^2_{b,c}}/{s}},\\
&\sigma_{gg\rightarrow b\bar{b},c\bar{c}}=\!\frac{\pi\alpha_s^2}{3s}\bigg[\left(1\!+\!\frac{4m^2_{b,c}}{s}\!+\!\frac{m^4_{b,c}}{s^2}\right)\ln{\left(\frac{1+w(s)}{1-w(s)}\right)}\!-\!\left(\frac{7}{4}\!+\!\frac{31m^2_{b,c}}{4s}\right)w(s)\bigg],
\end{align} 
where $m_{b,c}$ represents the mass of bottom or charm quark, $s$ is the Mandelstam variable, and $\alpha_s$ is the QCD coupling constant. Considering the perturbation expansion of the coupling constant $\alpha_s$ for the two-loop approximation~\cite{Letessier:2002ony}, we have:
\begin{align}
\alpha_s(\mu^2)=\frac{4\pi}{\beta_0\ln({\mu^2/\Lambda^2})}\bigg[1-\frac{\beta_1}{\beta_0}\frac{\ln(\ln{(\mu^2/\Lambda^2)})}{\ln(\mu^2/\Lambda^2)}\bigg],
\end{align}
where $\mu$ is the renormalization energy scale and $\Lambda^2$ is a parameter that determines the strength of the interaction at a given energy scale in QCD. The energy scale we consider is based on required gluon/quark collisions above $b\bar b$ energy threshold, so we have $\mu=2m_b+T$. For the energy scale $\mu>2m_b$ we have $\Lambda=180\sim230$ MeV( $\Lambda\approx205$ MeV in our calculation), and the parameters $\beta_0=11-2n_f/3$, $\beta_1=102-38n_f/3$ with the number of active fermions $n_f=4$. 

In general the thermal reaction rate per unit time and volume $R$ can be written in terms of the scattering cross section as follows~\cite{Letessier:2002ony}:
\begin{align}
R\equiv\sum_i\int_{s_{th}}^\infty\!ds\,\frac{dR_i}{ds}=\sum_i\int_{s_{th}}^\infty\!ds\,\sigma_i(s)\,P_i(s),
\end{align}
where $\sigma_i(s)$ is the cross section of the reaction channel $i$, and $P_i(s)$ is the number of collisions per unit time and volume. Considering the quantum nature of the colliding particles (i.e., Fermi and Bose distribution) with the massless limit and chemical equilibrium condition ($\Upsilon=1$), we obtain~\cite{Letessier:2002ony}
\begin{align}
&P_i(s)=\frac{g_1g_2}{32\pi^4}\,\frac{T}{1+I_{12}}\frac{\lambda_2}{\sqrt{s}}\!\sum_{l,n=1}^{\infty}\!(\pm)^{l+n}\frac{K_1(\sqrt{lns}/T)}{\sqrt{ln}},\\
&\lambda_2\equiv\left[s-\left(m_1+m_2\right)^2\right]\,\left[s-\left(m_1-m_2\right)^2\right],
\end{align}
where $+$ is for boson and $-$ is for fermions, and the factor $1/(1+I_{12})$ is introduced to avoid double counting of indistinguishable pairs of particles. $I_{12}=1$ for identical pair of particles, otherwise $I_{12}=0$. Hence the total thermal reaction rate per volume for bottom quark production can be written as
\begin{align}
\label{Bquark_Source}
R^{\mathrm{Source}}_{b,c}=\int^\infty_{s_{th}}ds\,\bigg[\sigma_{q\bar{q}\rightarrow b\bar{b},c\bar{c}}\,P_q+\sigma_{gg\rightarrow b\bar{b},c\bar{c}}\,P_g\bigg]%=R_{q\bar{q}\rightarrow b\bar{b},c\bar{c}}+R_{gg\rightarrow b\bar{b},c\bar{c}}.
\end{align}
We introduce the bottom/charm quark relaxation time for the quark-gluon pair fusion as follows:
\begin{align}
\label{relaxation_time}
&{\tau_{b,c}^{\mathrm{Source}}}\equiv\frac{dn_{b,c}/d\Upsilon_{b,c}}{R^{\mathrm{Source}}_{b,c}}\;,\quad
\end{align}
where $dn_{b,c}/d\Upsilon_{b,c}=n^{th}_{b,c}$ in the Boltzmann approximation. The relaxation time is on the order of magnitude of time needed to reach chemical equilibrium. In Fig.~\ref{BCreaction_fig} we show the characteristic time for bottom/charm  quark strong interaction production in the domain of interest, $ 0.3\,\mathrm{GeV}>T> 0.15\,\mathrm{GeV}$. 

\paragraph{Quark decay rate via weak interaction}
The bottom/charm quark decay via the weak interaction.\index{bottom quark!decay rate}
%\begin{align}
% &b\longrightarrow c+l+\nu_l, \qquad %b\longrightarrow c+q+\bar{q},\\
%&c\longrightarrow s+l+\nu_l,\qquad c\longrightarrow s+q+\bar{q},
%\end{align}
The vacuum decay rate for $1\to2+3+4$ in vacuum can be evaluated via the weak interaction:
\begin{align}
\frac{1}{\tau_1}=&\frac{64G^2_F\,V^2_{12}\,V^2_{34}}{(4\pi)^3g_1}\,m^5_1\times\left[\frac{1}{2}{\left(1-\frac{m^2_2}{m^2_1}-\frac{m^2_3}{m^2_1}+\frac{m^2_4}{m^2_1}\right)}\mathcal{J}_1-\frac{2}{3}\mathcal{J}_2\right],
\end{align}
where the Fermi constant is $G_F=1.166\times10^{-5}\,\mathrm{GeV}^{-2}$, $V_{ij}$ is the element of the Cabibbo-Kobayashi-Maskawa (CKM) matrix~\cite{Czarnecki:2004cw} for quark channel and $V_{l\nu_l}=1$ for lepton channel. The functions $\mathcal{J}_1$ and $\mathcal{J}_2$ are given by
\begin{align}
&\mathcal{J}_1\!=\!\!\!\int_0^{(1-m^2_2/m^2_1)/2}\!\!\!\!\!\!\!\!dx\left(1\!-\!2x\!-\!\frac{m^2_2}{m_1^2}\right)^{\!\!2}\left[\frac{1}{(1-2x)^2}-1\right]\\
&\mathcal{J}_2\!=\!\!\!\int_0^{(1-m^2_2/m^2_1)/2}\!\!\!\!\!\!\!\!dx\left(1\!-\!2x\!-\!\frac{m^2_2}{m_1^2}\right)^{\!\!3}\left[\frac{1}{(1-2x)^3}-1\right]
\end{align}
The modification due to the heat bath(plasma) is small because the bottom and charm  mass $m_{b,c}\gg T$~\cite{Kuznetsova:2008jt}. In the temperature range we are interested in, the decay rate in the vacuum is a good approximation for our calculation. We show the lifespan for bottom/charm quark in Fig.~\ref{BCreaction_fig}. 

%there is no modification due to the phase space blocking because the bottom/charm mass is too heavier $m_{b,c}\gg T$~\cite{Kuznetsova:2008jt}. We did not include above either base enhancement nor fermi blocking factors since process of bottom decay involve energies beyond those available for $ 150< T< 300\,\mathrm{MeV}$.



%%%%%%%%%%%%%%%%%%%%%%%%%%%%%%%%%%%%%%%
\begin{figure} %[ht]
    \centering
    \includegraphics[width=0.85\textwidth]{./plots/CharmQuark_QGP.jpg}
    \includegraphics[width=0.85\textwidth]{./plots/BQuarkReactionTime_bottom.jpg}
    \caption{\cccite{Rafelski:2023emw}, adapted from thesis of C.T.Yang \cite{Yang:2024ret}. Comparison of Hubble time $1/H$, quark lifespan $\tau_{q}$, and characteristic time for production via quark-gluon pair fusion for (top figure) charm and (bottom figure) bottom quarks as a function of temperature. Both figures end at approximately the hadronization temperature of $T_{H}\approx150$ MeV. Three different masses $m_{b}=4.2$ GeV (blue short dashes), $4.7$ GeV, (solid black), $5.2$ GeV (red long dashes) for bottom quarks are plotted to account for its decay width.}
\label{BCreaction_fig}
\end{figure}
%%%%%%%%%%%%%%%%%%%%%%%%%%%%%%%%%%%%%%%



%~~~~~~~~~~~~~~~~~~~~~~~~~~~~~~~~~~~~~~~~~~~~~~~~~~~~~~~~~~~~~~~~~~~~~~~~~~~~
\paragraph{Hubble expansion rate}
In the early Universe, within a temperature range $130\, \mathrm{GeV}>T>0.15\,\mathrm{GeV}$,  we have the following particles:  photons, $8$ color charge gluons, $W^\pm$, $Z^0$, three generations of $3$ color charge quarks and leptons in the primordial QGP.  The Hubble parameter can be written in terms of particle energy density $\rho_i$
\begin{align}
H^2=\frac{8\pi G_N}{3}\left(\rho_\gamma+\rho_{\mathrm{lepton}}+\rho_{\mathrm{quark}}+\rho_{g,{W^\pm},{Z^0}}\right),
\end{align}
where $G_N$ is the Newtonian constant of gravitation. The effectively massless particles and radiation dominate the speed of expansion of the Universe. The characteristic Universe expansion time constant $1/H$ is seen in Fig.~\ref{BCreaction_fig}. In the epoch of interest to us $0.3\,\mathrm{GeV}>T>0.15\,\mathrm{GeV}$, the Hubble time $1/H\approx10^{-5}$ sec which is much longer than the lifespan and production time of the bottom and charm quarks. 
%~~~~~~~~~~~~~~~~~~~~~~~~~~~~~~~~~~~~~~~~~~~~~~~~~~~~~~~~~~~~~~~~~~~~~~~~~~~~~~~~~~~~~~~

\paragraph{Rate Comparison: Strong fusion, Weak decay, and Hubble expansion}
In Fig.~\ref{BCreaction_fig} (top), we plot the relaxation time of the production/decay for charm quarks and Hubble time $1/H$ as a function of temperature. Throughout the entire duration of QGP, the Hubble time is larger than the lifespan and production times of the charm quark. %Therefore, the heavy charm quark remains in equilibrium as its processes occur faster than the expansion of the Universe. 
Additionally, the charm quark production time is faster than the decay. The faster quark-gluon pair fusion keeps the charm in chemical equilibrium up until hadronization. After hadronization, charm quarks form heavy mesons that decay into multi-particles quickly in plasma. The daughter particles from charm meson decay can interact and reequilibrate with
the plasma quickly. In this case the energy required for the inverse decay reaction to produce
charm meson is difficult to overcome and causing the charm quark to vanish from the inventory of particles via decay in the Universe.

In Fig.~\ref{BCreaction_fig} (bottom) we present the relaxation times for production and decay of the bottom quark with different masses as a function of temperature. It shows that both production and decay are faster than the Hubble time $1/H$ for the duration of QGP. However, unlike charm quarks, the relaxation time for bottom quark production intersects with bottom quark decay at different temperatures which depends on the mass of the bottom. The intersection implies that the bottom quark freeze-out from the primordial plasma before hadronization as the production process slows down at low temperatures and the subsequent weak interaction decay leads to a dilution of the bottom quark content within the QGP plasma. All of this occurs with rates faster than Hubble expansion and thus as the Universe expands, the system departs from a detailed chemical balance because of the competition between decay and production reactions in QGP. In this scenario, the dynamic equation on bottom abundance is required and causes the distribution to deviate from equilibrium with $\Upsilon\neq1$ in the temperature range below the crossing point but before the hadronization. 



\subsubsection{Bottom quark abundance nonequilibrium}\index{bottom quark!nonequilibrium abundance}

The competition between weak interaction decay and strong interaction production rates leads the dynamic bottom abundance in QGP. The dynamic equation for bottom quark abundance in QGP can be written as \index{bottom quark!population equation}
\begin{align}
\label{Bquark_eq}
\frac{1}{V}\frac{dN_b}{dt}=\big(\,1-\Upsilon^2_{b}\,\big)\,R^{\mathrm{Source}}_{b}-\Upsilon_b\,R^{\mathrm{Decay}}_{b}\;,
\end{align}
where $R^{\mathrm{Source}}_{b}$ and $R^{\mathrm{Decay}}_{b}$ are the thermal reaction rates per volume of production and decay of bottom quark, respectively. The bottom source rates are the gluon and quark fusion rates Eq.~(\ref{Bquark_Source}). The decay rate depends on whether the bottom quarks are freely present in the plasma or are bounded within mesons. We consider two extreme scenarios for the bottom quark population: 1.) all bottom flavor is free, and 2.) all bottom flavor is bounded into mesons in QGP. In Fig.~\ref{ReactionTime}  we show the characteristic interaction times relevant to the abundance of bottom quarks, as well as the Hubble time $1/H$ for the temperature range of interest, $0.3\,\mathrm{GeV}> T> 0.15\,\mathrm{GeV}$.

%~~~~~~~~~~~~~~~~~~~~~~~~~~~~~~~~~~~~~~~~~~~~~~~~~~~~~~~~~~~~~
\begin{figure}[ht]
\begin{center}
\includegraphics[width=0.85\textwidth]{./plots/BQuarkReactionTime003}
\caption{Production and decay characteristic times of bottom quark and the Hubble time $1/H$ within the temperature range of interest  $ 0.3\,\mathrm{GeV}>T> 0.15\,\mathrm{MeV}$. Near the top of figure  $1/H$ (brown solid line) and $\tau_T$ (brown dashed line); other horizontal lines are bottom-quark (in QGP) weak interaction lifetimes $\tau_b$ for the three different masses: $m_b=4.2\,\mathrm{GeV}$ (blue dotted line), $m_b=4.7\,\mathrm{GeV}$ (black solid  line), $m_b=5.2\,\mathrm{GeV}$ (red dashed line), and the vacuum lifespan $\tau_B$ of the  B$_c$ meson (green solid  line). The relaxation time for strong interaction bottom production $g+g, q+\bar q\rightarrow b+\bar{b}$ is shown with three different bottom masses and same type-color coding as weak interaction decay rate. At bottom of figure the in plasma formation process (dashed lines, purple) $b+c\rightarrow \mathrm{B}_c+g$ with cross section range $\sigma=0.1,10\,\mathrm{mb}$. Adapted from thesis of C.T.Yang \cite{Yang:2024ret}}
\label{ReactionTime}
\end{center}
\end{figure}
%~~~~~~~~~~~~~~~~~~~~~~~~~~~~~~~~~~~~~~~~~~~~~~~~~~~~~~~~~~~~~
Considering all bottom flavor is free in QGP, the bottom decay rate per volume is the bottom lifespan weighted with density of particles Eq.~(\ref{BoltzN})~~\cite{Kuznetsova:2008jt}. We have
\begin{align}\hspace{0.5cm}
R^{\mathrm{Decay}}_b=\frac{dn_b/d\Upsilon_b}{\tau_b},\,\,\,\,\, \tau_b\approx0.57\times10^{-11} \mathrm{sec}.
\end{align}
On the other hand, $b$,$\bar b$ quark abundance is embedded in a large background comprising all lighter quarks and antiquarks (see Fig.~\ref{number_entropy_b002}). After formation the heavy $b, \bar b$ quark can bind with any of the available lighter quarks, with the most likely outcome being a chain of reactions 
\begin{align}
&b+q\longrightarrow\mathrm{B}+g\;,\\
&\mathrm{B}+s\longrightarrow\mathrm{B}_s+q\;,\\
&\mathrm{B}_s+c\longrightarrow\mathrm{B}_c+s\;,
\end{align}
with each step providing a gain in binding energy and reduced speed due to the diminishing abundance of heavier quarks $s, c$. To capture the lower limit of the rate of $\mathrm{B}_c$ production we show in Fig.~\ref{ReactionTime} the expected formation rate by considering the direct process $b+\overline c\rightarrow \mathrm{B}_c+g$, considering the range of cross section $\sigma=0.1\sim10\,\mathrm{mb}$ ~\cite{Schroedter:2000ek}. The rapid formation rate of B$_c(b\bar c)$ states in primordial plasma is shown by purple dashed lines at bottom in Fig.~\ref{ReactionTime}, we have
\begin{align}
\tau (b+\overline c\rightarrow \mathrm{B}_c+g)\approx(10^{-16}\sim10^{-14})\times\frac{1}{H} \;.
\end{align}

Despite the low abundance of charm, the rate of $\mathrm{B}_c$ formation is relatively fast, and that of lighter flavored B-mesons is substantially higher. Note that as long as we have bottom quarks made in gluon/quark fusion bound practically immediately with any quarks $u, d, s$ into B-mesons, we can use the production rate of $b, \bar b$ pairs as the rate of B-meson formation in the primordial-QGP, which all decay with lifespan of pico-seconds. We believe that this process is fast enough to allow consideration of bottom decay from the B$_c(b\bar c)$, $\overline{\mathrm{B}}_c(\bar b c)$ states~\cite{Yang:2020nne}.  
 Based on the hypothesis that all bottom flavor is bound rapidly into $\mathrm{B}_c^\pm$ mesons, we have 
\begin{align}\label{Bc_source}
g+g, q+q \longleftrightarrow &b+\bar b\;[b(\bar{b})+\bar{c}(c)]\longrightarrow \mathrm{B}_c^\pm\longrightarrow\mathrm{anything}.
\end{align}
In this case, the decay rate per volume can be written as
\begin{align}\hspace{0.5cm}
 R^{\mathrm{Decay}}_b=\frac{dn_b/d\Upsilon_b}{\tau_{\mathrm{B}_c}},\,\,\,\,\, \tau_{\mathrm{B}_c}\approx0.51\times10^{-12} \mathrm{sec}.
 \end{align}



To investigate the nonequilibrium phenomena of bottom quarks, we aim to replace the variation of particle abundance seen on LHS in Eq.~(\ref{Bquark_eq}) by the time variation of abundance fugacity $\Upsilon$.
This substitution allows us to derive the dynamic equation for the fugacity parameter and enables us to study the fugacity as a function of time . Considering the expansion of the Universe we have
\begin{align}\label{number_dilution}
\frac{1}{V}\frac{dN_b}{dt}=\frac{dn_b}{d\Upsilon_b}\frac{d\Upsilon_b}{dt}+\frac{dn_b}{dT}\frac{dT}{dt}+3Hn_b,\;
\end{align}
where we use $d\ln(V)/dt=3H$ for the Universe expansion. Substituting Eq.~(\ref{number_dilution}) into Eq.~(\ref{Bquark_eq}) and dividing both sides of equation by $dn_b/{d\Upsilon_b}=n^{th}_b$, the fugacity equation becomes
\begin{align}
\frac{d\Upsilon_b}{dt}+&3H\Upsilon_b+\Upsilon_b\frac{dn^{th}_b/dT}{n^{th}_b}\frac{dT}{dt}=\left(1-\Upsilon_b^2\right)\frac{1}{\tau_{b}^{\mathrm{Source}}}-\Upsilon_b\frac{1}{\tau^{\mathrm{Decay}}_b}\;,
\end{align}
where relaxation time for bottom production is obtained using Eq.~(\ref{relaxation_time}). It is convenient to introduce the relaxation time $1/\tau_T$ as follows,
\begin{align}
\frac{1}{\tau_T}\equiv-\frac{dn^{th}_b/dT}{n^{th}_b}\frac{dT}{dt},
\end{align}
where we put '$-$' sign in the definition to have $\tau_T>0$. The relaxation time $\tau_T$ represents how the bottom density changes due to the Universe temperature cooling. In this case, the fugacity equation can be written as
\begin{align}\label{Fugacity_Eq0}
\frac{d\Upsilon_b}{dt}\!\!=&(1-\Upsilon_{b}^2)\frac{1}{\tau_{b}^{\mathrm{Source}}}
\!-\!\Upsilon_{b}\left(\frac{1}{\tau^{\mathrm{Decay}}_b}+3H\!-\!\frac{1}{\tau_T}\right).
\end{align}
In following sections we will solve the fugacity differential equation in two different scenarios: stationary and nonstationary Universe.

\paragraph{First solution: stationary Universe}
In Fig.~\ref{BCreaction_fig} (bottom) we show that the relaxation time for both production and decay are faster than the Hubble time $1/H$ for the duration of QGP, which implies that $H,1/\tau_T\ll1/\tau_{b}^{\mathrm{Source}},1/\tau^{\mathrm{Decay}}_b$. In this scenario, we can solve the fugacity equation by considering the stationary Universe first, i.e., the Universe is not expanding and we have
\begin{align}\label{stationary}
H=0,\qquad1/\tau_T=0.
\end{align} 
In the stationary Universe at each given temperature we consider the dynamic equilibrium condition (detailed balance) between production and decay reactions that keep
\begin{align}
\frac{d\Upsilon_b}{dt}=0.
\end{align}
Neglecting the time dependence of the fugacity $d\Upsilon_b/dt$ and substituting the condition Eq.~(\ref{stationary}) into the fugacity equation Eq.~(\ref{Fugacity_Eq0}), then we can solve the quadratic equation to obtain the stationary fugacity as follows: \index{bottom quark!stationary fugacity}
\begin{align}
\label{Fugacity_Sol}
\Upsilon_{\mathrm{st}}&=\sqrt{1+\left(\frac{\tau_{source}}{2\tau_{decay}}\right)^2}-\left(\frac{\tau_{source}}{2\tau_{decay}}\right).
\end{align}
%~~~~~~~Figure~~~~~~~~~~~~~~~~~~~~~~~~~~~~~~~~~~~~~~~~~~~~~~~~~~~~~~~~~~~~~~~~~~~~~~~~~~~~~~~~~
\begin{figure}[ht]
\begin{center}
\includegraphics[width=\textwidth]{./plots/BquarkFugacity_tot}
\caption{\cccite{Rafelski:2023emw}, adapted from thesis of C.T.Yang \cite{Yang:2024ret}.The fugacity of free/ bounded bottom quark as a function of temperature in the early Universe for $m_b=4.2\,\mathrm{GeV}$ (blue), $m_b=4.7\,\mathrm{GeV}$ (black), and $m_b=5.2\,\mathrm{GeV}$ (red). The solid lines represent the case bottom quark bound into $B_c$ mesons, and the dashed lines label the case of  free bottom quark.}
\label{fugacity_bc}
\end{center}
\end{figure}
%~~~~~~~~~~~~~~~~~~~~~~~~~~~~~~~~~~~~~~~~~~~~~~~~~~~~~~~~~~~~~~~~~~~~~~~~~~~~~~~~~~~~~~~~~~~~~~~
\,
In Fig.~\ref{fugacity_bc} the fugacity of bottom quark $\Upsilon_{\mathrm{st}}$ as a function of temperature, Eq.~(\ref{Fugacity_Sol}) is shown around the temperature $T=0.3\,\mathrm{GeV}>T>0.15\,\mathrm{GeV}$ for different masses of bottom quarks. In all cases we see prolonged non-equilibrium, this happens since the decay and reformation rates of bottom quarks are comparable to each other as we have noted in Fig.~\ref{ReactionTime} where both lines cross. One of the key results shown in Fig.~\ref{fugacity_bc} is that the smaller mass of bottom quark slows the strong interaction formation rate to the value of weak interaction decays just near the phase transformation of QGP to HG phase. Finally, the stationary fugacity corresponds to the reversible reactions in the stationary Universe. In this case, there is no arrow in time for bottom quark because of the detailed balance.



\paragraph{Non-stationary correction in expanding Universe}

The Universe is expanding and temperature is a function of time. In this section we now consider the fugacity as a function of time and study the correction in fugacity due to the expanding Universe. In general, the fugacity of bottom quark can be written as 
\begin{align}\label{Nonstationary_sol}
&\Upsilon_b=\Upsilon_{\mathrm{st}}+\Upsilon^{\mathrm{non}}_{\mathrm{st}}=\Upsilon_\mathrm{st}\left(1+x\right),\quad x\equiv{\Upsilon_\mathrm{st}^{\mathrm{non}}}/{\Upsilon_\mathrm{st}},
\end{align}
where the variable $x$ corresponds to the correction due to non-stationary Universe. Substituting the general solution Eq.(\ref{Nonstationary_sol}) into differential equation Eq.(\ref{Fugacity_Eq0}), we obtain
\begin{align}\label{Nonstationary_eq}
\frac{dx}{dt}=-x^2\frac{\Upsilon_\mathrm{st}}{\tau_{source}}&-x\left[\frac{1}{\tau_{eff}}+3H-\frac{1}{\tau_T}\right]-\left[\frac{d\ln\Upsilon_\mathrm{st}}{dt}+3H-\frac{1}{\tau_T}\right],
\end{align}
where the effective relaxation time $1/\tau_{eff}$ is defined as
\begin{align}
\frac{1}{\tau_{eff}}\equiv\left[\frac{2\Upsilon_\mathrm{st}}{\tau_{source}}+\frac{1}{\tau_{decay}}+\frac{d\ln\Upsilon_\mathrm{st}}{dt}\right].
\end{align}
%~~~~~~~Figure~~~~~~~~~~~~~~~~~~~~~~~~~~~~~~~~~~~~~~~~~~~~~~~~~~~~~~~~~~~~~~~~~~~~~~~~~~~~~~~~~
\begin{figure}[t]
\begin{center}
\includegraphics[width=\textwidth]{./plots/Tau_RelaxationTime002}
\caption{The effective relaxation time $\tau_{eff}$ as a function of temperature in the early Universe for bottom mass $m_b=4.7$GeV.  For comparison, we also plot the vacuum lifespan of $B_c$ meson $\tau_{B_c}^{decay}$(red dashed-line), the relaxation time for bottom production $\tau^b_{source}$ (blue dashed-line), Hubble expansion time $1/H$(brown solid line) and relaxation time for temperature cooling $\tau_T$(brown dashed-line). Adapted from thesis of C.T.Yang \cite{Yang:2024ret}.}
\label{RelaxationTime_eff}
\end{center}
\end{figure}
%~~~~~~~~~~~~~~~~~~~~~~~~~~~~~~~~~~~~~~~~~~~~~~~~~~~~~~~~~~~~~~~~~~~~~~~~~~~~~~~~~~~~~~~~~~~~~~~
In Fig.~\ref{RelaxationTime_eff} we see that when temperature is near to $T=0.2$ GeV, we have $1/\tau_{eff}\approx10^{7}H$, and $1/\tau_{eff}\approx10^5/\tau_T$. In this case, the last two terms in Eq.~(\ref{Nonstationary_eq}) compare to $1/\tau_{eff}$ can be neglected, and the differential equation becomes
\begin{align}\label{nonstationary_eq}
\frac{dx}{dt}=-\frac{x^2\,\Upsilon_\mathrm{st}}{\tau_{source}}&-\frac{x}{\tau_{eff}}-\left[\frac{d\ln\Upsilon_\mathrm{st}}{dt}+3H-\frac{1}{\tau_T}\right],
\end{align}


To solve the variable $x$ we consider the case $dx/dt,x^2\ll1$ first, we neglect the terms $dx/dt$ and $x^2$ in Eq.~(\ref{nonstationary_eq})  then solve the linear fugacity equation.  We will establish that these approximations are justified by checking the magnitude of the solution. Neglecting terms $dx/dt$ and $x^2$ in Eq.~(\ref{nonstationary_eq}) we obtain
\begin{align}
x\approx\tau_{eff}\left[\frac{d\ln\Upsilon_\mathrm{st}}{dt}+3H-\frac{1}{\tau_T}\right].
\end{align}
It is convenient to change the variable from time to temperature. For an isentropically-expanding universe, we have
\begin{align}\label{tau_H}
\frac{dt}{dT}=-\frac{\tau^\ast_H}{T},\qquad \tau^\ast_H=\frac{1}{H}\left(1+\frac{T}{3g^s_\ast}\frac{dg^s_\ast}{dT}\right).
\end{align}
In this case, we have
\begin{align}
x=\tau_{eff}\left[\frac{1}{\Upsilon_\mathrm{st}}\frac{d\Upsilon_\mathrm{st}}{dT}\frac{T}{\tau^\ast_H}+3H-\frac{1}{\tau_T}\right].
\end{align}
Finally, we can obtain the nonstationary fugacity by multiplying the fugacity ratio $x$ with $\Upsilon_\mathrm{st}$, giving \index{bottom quark! nonstationary fugacity}
\begin{align}
\Upsilon_{\mathrm{st}}^{\mathrm{non}}
&\approx\left(\frac{\tau_{eff}}{\tau^\ast_H}\right)\left[\frac{d\Upsilon_\mathrm{st}}{dT}T-\Upsilon_{\mathrm{st}}\left(3H\tau^\ast_H-\frac{\tau^\ast_H}{\tau_T}\right)\right].
\end{align}

In Fig.~\ref{NonFugacity} we plot the nonstationary $\Upsilon^{\mathrm{non}}_\mathrm{st}$ as a function of temperature. The nonstationary fugacity $\Upsilon^{\mathrm{non}}_\mathrm{st}$ follows the behavior of $d\Upsilon_{\mathrm{st}}/dT$, which corresponds to the irreversible process in expanding Universe. In this case, the irreversible nonequilibrium process creates the arrow in time for bottom quark in the Universe. The large value of Hubble time compares to the effective relaxation time suppressing the value of nonstationary fugacity to $\mathcal{O}\sim10^{-7}$, which shows that the neglecting $dx/dt,x^2\ll1$ is a good approximation for solving the non-stationary fugacity in the early Universe.
%~~~~~~~Figure7~~~~~~~~~~~~~~~~~~~~~~~~~~~~~~~~~~~~~~~~~~~~~~~~~~~~~~~~~~~~~~~~~~~~~~~~~~~~~~~~~
\begin{figure}[t]
\begin{center}
\includegraphics[width=\textwidth]{./plots/NonstationaryFugacity}
\caption{The non-stationary fugacity $\Upsilon_\mathrm{st}^{\mathrm{non}}$ as a function of temperature in the Universe for different bottom mass $m_b=4.2\,\mathrm{GeV}$ (blue), $m_b=4.7\,\mathrm{GeV}$ (black), and $m_b=5.2\,\mathrm{GeV}$ (red) for the case bottom  quarks bound into $B_c$ mesons. Adapted from thesis of C.T. Yang \cite{Yang:2024ret}.}
\label{NonFugacity}
\end{center}
\end{figure}
%~~~~~~~~~~~~~~~~~~~~~~~~~~~~~~~~~~~~~~~~~~~~~~~~~~~~~~~~~~~~~~~~~~~~~~~~~~~~~~~~~~~~~~~~~~~~~~~

To conclude this chapter, we have demonstrated that the bottom quark nonequliibrium occurs near the QGP phase transition around the temperature $T=0.3\sim0.15$ GeV in Fig.~\ref{fugacity_bc} and Fig.~\ref{NonFugacity}. We show the competition between weak interaction decay and the strong interaction $g+g\to b+\bar b$, $q+\bar q \to b+\bar b$ fusion processes drive the bottom quark departure from the equilibrium and create the arrow of time in the early Universe at relatively low QGP temperature. The results provide a strong motive for exploring the physics of baryon nonconservation involving the bottomnium mesons or/and bottom quarks in a thermal environment.

%~~~~~~~~~~~~~~~~~~~~~~~~~~~~~~~~~~~~~~~~~~~~~~~~~
%~~~~~~~~~~~~~~~~~~~~~~~~~~~~~~~~~~~~~~~~~~~~~~~~~
%{Introduction\daggerfootnote{This chapter has been published previously as \citet{Gottbrath1999}.}}

%\begin{figure}
%\centering
%\includegraphics[angle=0,width=\columnwidth]{fig1.pdf}
%\caption[]{}
%\label{fig1}
%\end{figure}
\section{Strangeness abundance in cosmic plasma}
\label{Strangeness}
%{Heavy-quark in primordial QGP: after hadronization}
As the Universe expanded and cooled down to the hadronization temperature $T_H\approx150$ MeV, the primordial QGP underwent a phase transformation called hadronization. This transition resulted in the confinement of the strong force, causing quarks and gluons to combine and form matter 
and antimatter. After hadronization, one may think the relatively short lived massive hadrons decay rapidly and disappear from the Universe. However, the most abundant hadrons, pions $\pi(q\bar q)$, can be produced via their inverse decay process $\gamma\gamma\rightarrow\pi^0$ and retain their chemical equilibrium until temperature $T=3\sim5$ MeV~\cite{Kuznetsova:2008jt}. 

Following the idea and the framework presented by~\cite{Kuznetsova:2008jt}, we investigate the strange particle composition of the expanding early Universe in the epoch $150\,\mathrm{MeV}\ge T\ge 10$\,MeV, and examine the freeze-out temperature for strangeness-producing  by comparing the relevant reaction rates to the Hubble expansion rate. We show that strangeness is kept in equilibrium via weak, electromagnetic, and strong interactions in the early Universe until $T\approx13$ MeV.


 
%~~~~~~~~~~~~~~~~~~~~~~~~~~~~~~~~~~~~~~~~~~~~~~~~~



%~~~~~~~~~~~~~~~~~~~~~~~~~~~~~~~~~~~~~~~~~~~~~~~~~

\subsection{Chemical equilibrium in the hadronic Universe}
%In this section we will focus on the following:
%\begin{itemize}
%    \item Chemical potentials of $\mu_B$ and $\mu_s$
%    \item Composition of universe (strangeness abundance)
%\end{itemize}

In this section, we explore the Universe composition assuming both kinetic and particle abundance equilibrium (chemical equilibrium) by considering the charge neutrality and prescribed conserved baryon-per-entropy-ratio ${(n_B-n_{\overline{B}})}/{\sigma}$ to determine the baryon chemical potential $\mu_B$~\cite{Fromerth:2012fe,Rafelski:2013yka}. With the chemical potential as a function of temperature, we can obtain the particle number densities for different species and study their composition in the early Universe.

We improve the prior work~\cite{Fromerth:2012fe} by considering the conserved entropy per baryon ratio with conservation of strangeness in the early Universe. To study the baryon and strange quark chemical potential, it is convenient to introduce the chemical fugacity for strangeness $\lambda_s$ and quark $\lambda_q$ as follows:
\begin{align}
\lambda_s=\exp(\mu_s/T)\,\quad \lambda_q=\exp(\mu_B/3T),
\end{align}
where $\mu_s$ and $\mu_B$ are the chemical potential of strangeness and baryon, respectively. For the quark fagucity $\lambda_q$, we divide the chemical potential of baryons by 3 as an approximation for quark chemical potential. Imposing the conservation of strangeness  
$\langle s-\bar s \rangle=0$, we have, when the baryon chemical potential does not vanish the chemical potential of strangeness in the early Universe satisfying (see Section 11.5 in \,\cite{Letessier:2002ony})
\begin{align}\label{museq}
\lambda_s=\lambda_q\sqrt{\frac{F_K+\lambda^{-3}_q\,F_Y}{F_K+\lambda^3_q\,F_Y}}.
\end{align}
where we employ the phase-space function $F_i$ for sets of nucleon $N$, kaon $K$, and hyperon $Y$ particles defined as (see \cite{Letessier:2002ony}, Section 11.4):
\begin{align}
&F_N=\sum_{N_i}\,g_{N_i}W(m_{N_i}/T)\;, \quad N_i=n, p, \Delta(1232),\\
&F_K=\sum_{K_i}\,g_{K_i}W(m_{K_i}/T)\;, \quad K_i=K^0, \overline{K^0}, K^\pm, K^\ast(892),\\
&F_Y=\sum_{Y_i}\,g_{Y_i}W(m_{Y_i}/T)\;, \quad Y_i=\Lambda, \Sigma^0,\Sigma^\pm, \Sigma(1385),
\end{align}
where $g_{N_i,K_i,Y_i}$ are the degenerate factors, $W(x)=x^2K_2(x)$ with $K_2$ is the modified Bessel functions of integer order "$2$".  

Considering the Boltzmann approximation for the massive particle number density we have
\begin{align}
\label{Density_N}
&n_N=\frac{T^3}{2\pi^2}\lambda_q^3F_N,\quad\qquad\qquad n_{\overline N}=\frac{T^3}{2\pi^2}\lambda^{-3}_qF_N,\\
\label{Density_K}
&n_K=\frac{T^3}{2\pi^2}\left(\lambda_s\lambda_q^{-1}\right)F_K,\,\qquad n_{\overline{K}}=\frac{T^3}{2\pi^2}\left(\lambda_s^{-1}\lambda_q\right)F_K,\\
\label{Density_Y}
&n_Y=\frac{T^3}{2\pi^2}\left(\lambda_q^2\lambda_s\right)F_Y,\quad\qquad n_{\overline Y}=\frac{T^3}{2\pi^2}\left(\lambda^{-2}_q\lambda_s^{-1}\right)F_Y.
\end{align}
In this case, the net baryon density in the early Universe with temperature range $150\,\mathrm{MeV}> T>10$\,MeV can be written as 
\begin{align}
\frac{\left(n_B-n_{\overline{B}}\right)}{\sigma}&=\frac{1}{\sigma}\left[\left(n_p-n_{\overline{p}}\right)+\left(n_n-n_{\overline{n}}\right)+\left(n_Y-n_{\overline{Y}}\right)\right]\notag\\
&=\frac{T^3}{2\pi^2\,\sigma}\left[\left(\lambda_q^3-\lambda^{-3}_q\right)F_N+\left(\lambda_q^2\lambda_s-\lambda^{-2}_q\lambda_s^{-1}\right)F_Y\right]\notag\\
&=\frac{T^3}{2\pi^2\sigma}\left(\lambda_q^3-\lambda_q^{-3}\right)F_N\left[1+\frac{\lambda_s}{\lambda_q}\left(\frac{\lambda_q^3-\lambda^{-1}_q\lambda_s^{-2}}{\lambda^3_q-\lambda^{-3}_q}\right)\,\frac{F_Y}{F_N}\right]\notag\\
&\approx\frac{T^3}{2\pi^2\sigma}\left(\lambda_q^3-\lambda_q^{-3}\right)F_N\left[1+\frac{\lambda_s}{\lambda_q}\,\frac{F_Y}{F_N}\right],
\end{align}
where we can neglect the term $F_Y/F_K$ in the expansion of Eq.(\ref{museq}) in our temperature range. Introducing the strangeness $\langle s-\bar s\rangle=0$ constraint and using the entropy density in early universe, the explicit relation for baryon to entropy ratio becomes
\begin{align}\label{muBeq}
\frac{n_B-n_{\overline{B}}}{\sigma}&=\frac{45}{2\pi^4g^s_\ast}\sinh\left[\frac{\mu_B}{T}\right]F_N\times\left[1+\frac{F_Y}{F_N}\sqrt{\frac{1+e^{-\mu_B/T}\,F_Y/F_K}{1+e^{\mu_B/T}\,F_Y/F_K}}\right].
\end{align}
Governing Eq.\,(\ref{muBeq}) is the present-day baryon-per-entropy-ratio, and we obtain the value 
\begin{align}\label{BdS}
\frac{n_B-n_{\overline{B}}}{\sigma}= \left.\frac{n_B-n_{\overline{B}}}{ \sigma}\right|_{t_0}=(0.865\pm0.008)\times10^{-10} \;.
\end{align}
For a detailed evaluation method we refer to this earlier work now using a baryon-to-photon ratio~\cite{ParticleDataGroup:2018ovx}: $\left(n_B-n_{\overline{B}}\right)/n_\gamma= (0.609\pm0.06)\times10^{-9}$, as well as the entropy per particle for a massless boson $\sigma/n|_\mathrm{boson}\approx 3.60$ and a massless fermion $\sigma/n|_\mathrm{fermion}\approx 4.20$. 

%~~~~~~~~~~~~~~~~~~~~~~~~~~~~~~~~~~~~~~~~~~~~~~~~~~~~~~~~~~~~~~~~~~~~~~~~~~~~~~~~
\begin{figure}[t]
%\begin{center}
\centering
\includegraphics[width=\textwidth]{./plots/New_Chemical_Potential_C.jpg}
\caption{\cccite{Yang:2021bko}, adapted from thesis of C.T.Yang \cite{Yang:2024ret}. The chemical potential of baryon $\mu_B/T$ and strangeness $\mu_s/T$ as a function of temperature $150\,\mathrm{MeV}> T>10\,\mathrm{MeV}$ in the early Universe; for comparison we show $m_N/T $ with $m_N=938.92$\,MeV, the average nucleon mass.}
\label{ChemPotFig}
%\end{center}
\end{figure}
%~~~~~~~~~~~~~~~~~~~~~~~~~~~~~~~~~~~~~~~~~~~~~~~~~~~~~~~~~~~~~~~~~~~~~~~~~~~~~~
%%%%%%%%%%%%%%%%%%%%%%%%%%%%%%%%%%%%%%%
\begin{figure}[h]
\centering
\includegraphics[width=\textwidth]{./plots/Baryon_Antibaryon_cm.jpg}
\caption{\cccite{universe9070309}, adapted from thesis of C.T.Yang \cite{Yang:2024ret}. The baryon (blue solid line) and antibaryon (red solid line) number density as a function of temperature in the range $150\,\mathrm{MeV}>T>5\,\mathrm{MeV}$. The blue dotted line is the extrapolated value for baryon density. The temperature $T=38.2\,\mathrm{MeV}$ is denoted when the ratio $n_{\overline B}/(n_B-n_{\overline B})=1$ which defines the condition where antibaryons disappear from the Universe.}
\label{Baryon_fig}
\end{figure}
%%%%%%%%%%%%%%%%%%%%%%%%%%%%%%%%%%%%%%%


We solve Eq.~(\,\ref{museq}) and Eq~(\ref{muBeq}) numerically to obtain baryon and strangeness chemical potentials as a function of temperature in Fig.~\ref{ChemPotFig}. The chemical potential changes dramatically in the temperature window $50\,\mathrm{MeV}\le T\le 30$\,MeV, its behavior describing the process of antibaryon disappearance. Substituting the chemical potential $\lambda_q$ and $\lambda_s$ into particle density Eq.~(\ref{Density_N}), Eq.~(\ref{Density_K}), and Eq.~(\ref{Density_Y}), we can obtain the particle number densities for different species as a function of temperature.

In Fig.~\ref{Baryon_fig} we plot the number density of baryon and antibaryon as a function of temperature. We consider that when the  $n_{\overline B}\ll(n_B-n_{\overline B})$ the anitbaryons density is sufficient low and disappear from the Universe inventory quickly. To determine the temperature where antibaryons is sufficient law in the Universe inventory we defined the condition when the ratio $n_{\overline B}/(n_B-n_{\overline B})=1$. This condition is reached in an expanding Universe at $T=38.2$\,MeV, which is in agreement with the qualitative result in \cite{kolb1990early}. After this temperature, the net baryon density dilutes with a residual co-moving conserved quantity determined by the observed baryon asymmetry.



%~~~~~~~~~~~~~~~~~~~~~~~~~~~~~~~~~~~~~~~~~~~~~~~~~~~~~~~~~~~~~~~~~~~~~~~~~~~~~~~~
\begin{figure}[bt]
%\begin{center}
\centering
\includegraphics[width=\textwidth]{./plots/Meson_Baryon_density_ratio_C.jpg}
\caption{\cccite{universe9070309}, adapted from Ref.~\cite{Yang:2021bko} and thesis of C.T.Yang \cite{Yang:2024ret}. Ratios of hadronic particle number densities as a function of temperature $150\,\mathrm{MeV}> T>10\,\mathrm{MeV}$ in the early Universe, with baryon $B$ yields: pions $\pi$ (brown line), kaons $K( q\bar s)$ (blue), antibaryon $\overline B$ (black), hyperon $Y$ (red) and anti-hyperons $\overline Y$ (dashed red). Also shown $\overline K/Y$(purple).}
\label{EquilibPartRatiosFig}
%\end{center}
\end{figure}
%~~~~~~~~~~~~~~~~~~~~~~~~~~~~~~~~~~~~~~~~~~~~~~~~~~~~~~~~~~~~~~~~~~~~~~~~~~~~~~

In Fig.~\ref{EquilibPartRatiosFig} we show examples of particle abundance ratios of interest. %Considering $n_Y/n_B$ we see that hyperons $Y(sqq)$ remain a noticeable 1\% component in baryon yield through this domain of antibaryon decoupling.
Pions $\pi(q\bar q)$ are the most abundant hadrons $n_\pi/n_B\gg1$, because of their low mass and the reaction $\gamma\gamma\rightarrow\pi^0$, which assures chemical yield equilibrium~\cite{Kuznetsova:2008jt}. For $150\,\mathrm{MeV}>T>20.8\,\mathrm{MeV}$, we see the ratio $n_{{\overline K}(\bar q s)}/n_B\gg1$, which implies pair abundance of strangeness is more abundant than baryons, and is dominantly present in mesons, since $n_{\overline K}/n_Y\gg1$. 
For $20.8\,\mathrm{MeV}>T$, the baryon becomes dominant $n_{\overline K}/n_B<1$, which implies that the strange meson is embedded in a large background of baryons, and the exchange reaction $\overline{K}+N\rightarrow \Lambda+\pi$ can re-equilibrate kaons and hyperons in the temperature range; therefore strangeness symmetry $s=\bar s$ is maintained. For $12.9\,\mathrm{MeV}>T$ we have $n_Y/n_B>n_{\overline K}/n_B$, now the still existent tiny abundance of strangeness is found predominantly in hyperons.


%~~~~~~~~~~~~~~~~~~~~~~~~~~~~~~~~~~~~~~~~~~~~~~~~~

%~~~~~~~~~~~~~~~~~~~~~~~~~~~~~~~~~~~~~~~~~~~~~~~~~

\subsection{Seeking strangeness freeze-out chemical nonequilibrium
}
%In this section we will focus on the following:
%\begin{itemize}
%    \item Relevant strangeness reactions
%    \item Strangeness creation/annihilation in mesons
%    \item Strangeness production/ exchange in hyperons
%    \item Strangeness epochs in  the Universe 
%\end{itemize}

%In this section, we focus on investigating the strangeness abundance and decoupling following the formation of normal matter during the hadronization process of the quark-gluon plasma (QGP). Nonequilibrium conditions in the early Universe are of general interest: they are understood to be prerequisite for the arrow of time dependent processes to take hold in the Hubble expanding Universe.
%~~~~~~~~~~~~~~~~~~~~~~~~~~~~~~~~~~~~~~~~~~~~~~~~~~~~~~~~~~~~

%\subsection{Relevant strangeness reactions}
This section considers an unstable strange particle $S$ decaying into two particles $1$ and $2$, which themselves have no strangeness content. In a dense and high-temperature plasma with particles $1$ and $2$ in thermal equilibrium, the inverse reaction populates the system with particle $S$. This is written schematically as
\begin{align}
 S\Longleftrightarrow1+2,\qquad \mathrm{Example}: K^0\Longleftrightarrow\pi+\pi\,.
\end{align}
The natural decay of the daughter particles provides the intrinsic strength of the inverse strangeness production reaction rate. As long as both decay and production reactions are possible, particle $S$ abundance remains in thermal equilibrium. This balance between production and decay rates is called a detailed balance.

Once the primordial Universe expansion rate $1/H$ overwhelms the strongly temperature dependent back-reaction and the back reaction freeze-out, then the decay $S\rightarrow 1+2$ occurs out of balance and particle $S$ disappears from the inventory. The two-on-two strangeness producing reactions have a significantly higher strangeness production reaction threshold, thus especially near to strangeness decoupling their influence is negligible. Such reactions are more important near the QGP hadronization temperature $T_H\simeq 150$\,MeV, and they characterize strangeness exchange reactions such as $\mathrm{K}+N\leftrightarrow \Lambda+\pi$, (see Chapter 18 in \cite{Letessier:2002ony}).


In Fig.~\ref{Strangeness_map2} we show reactions relevant to strangeness evolution in the considered Universe evolution epoch $150\,\mathrm{MeV}\ge T\ge 10$\,MeV  and their pertinent reaction strength. As shown:
\begin{itemize}
\item
We study strange quark abundance in baryons and mesons, considering both open and hidden strangeness (hidden: $s\bar s$-content). Important source reactions are $l^-+l^+\rightarrow\phi$, $\rho+\pi\rightarrow\phi$, $\pi+\pi\rightarrow K_\mathrm{S}$, $\Lambda \leftrightarrow \pi+ N$, and $\mu^\pm+\nu\rightarrow K^\pm$. 
\item
Muons and pions are coupled through electromagnetic reactions $\mu^++\mu^-\leftrightarrow\gamma+\gamma$ and $\pi\leftrightarrow\gamma+\gamma$ to the photon background and retain their chemical equilibrium until the temperature $T =4$\, MeV and $T=5$\,MeV, respectively~\cite{Rafelski:2021aey,Kuznetsova:2008jt}. The large $\phi\leftrightarrow K+K$ rate assures $\phi$ and $K$ are in relative chemical equilibrium.
\end{itemize}
In order to determine where exactly strangeness disappears from the Universe inventory, we explore the magnitudes of different rates of production and decay processes in mesons and hyperons.
%~~~~~~~Figure~~~~~~~~~~~~~~~~~~~~~~~~~~~~~~~~~~~~~~~~~~~~~~~~~~~~~~~~~~~~~~~~~~~~
\begin{figure} %[h]
%\begin{center}
\centering
\includegraphics[width=0.75\linewidth]{./plots/Strangeness002_newJ.jpg}
\caption{\cccite{universe9070309}, adapted from Ref.~\cite{Yang:2021bko} and thesis of C.T.Yang \cite{Yang:2024ret}. The strangeness abundance changing reactions in the primordial Universe. The red circles show strangeness carrying hadronic particles; red thick lines denote effectively instantaneous reactions. Black thick lines show relatively strong hadronic reactions. The reaction rates required to describe  strangeness time evolution are shown in \cite{Rafelski:2020ajx}.
}
\label{Strangeness_map2}
%\end{center}
\end{figure}
%~~~~~~~~~~~~~~~~~~~~~~~~~~~~~~~~~~~~~~~~~~~~~~~~~~~~~~~~~~~~~~~~~~~~~~~~~~



\subsection{Strangeness creation/annihilation rate in mesons}
From Fig.~\ref{Strangeness_map2} in the meson domain, the relevant interaction rates competing with Hubble time are the reactions
\begin{align}
 &\pi+\pi\leftrightarrow K\,,\quad\mu^\pm+\nu\leftrightarrow K^\pm\,,\quad l^++l^-\leftrightarrow\phi\,,\\
 &\rho+\pi\leftrightarrow\phi\,,\quad \pi+\pi\leftrightarrow\rho\,.
\end{align}
The thermal reaction rate per time and volume for two body-to-one particle reactions $1+2\rightarrow 3$ has been presented before~\cite{Koch:1986ud,Kuznetsova:2008jt,Kuznetsova:2010pi}. In full kinetic and chemical equilibrium, the reaction rate per time per volume can be written as~\cite{Kuznetsova:2010pi} :
\begin{align}
&R_{12\to 3}=\frac{g_3}{(2\pi)^2}\,\frac{m_3}{\tau^0_3}\,\int^\infty_0\frac{p^2_3dp_3}{E_3}\frac{e^{E_3/T}}{e^{E_3/T}\pm1}\Phi(p_3)\;,
\end{align}
where $\tau^0_3$ is the vacuum lifetime of particle $3$. The positive sign $``+"$ is for the case when particle $3$ is a boson, and negative sign $``-"$ for fermion. The function $\Phi(p_3)$ for the non-relativistic limit $m_3\gg p_3,T$ can be written as 
\begin{align}
\Phi(p_3\to0)=2\frac{1}{(e^{E_1/T}\pm1)(e^{E_2/T}\pm1)}.
\end{align}


Considering the Boltzmann limit, the thermal reaction rate per unit time and volume becomes
\begin{align}
\label{Thermal_Rate}
R_{12\rightarrow3}=\frac{g_3}{2\pi^2}\left(\frac{T^3}{\tau^0_3}\right)\left(\frac{m_3}{T}\right)^2\,K_1(m_3/T),
\end{align}
where $K_1$ is the modified Bessel functions of integer order "$1$". In order to compare the reaction time with Hubble time $1/H$, it is convenient to define the relaxation time for the process $1+2\rightarrow 3$ as follows:
\begin{align}
\label{Reaction_Time}
\tau_{12\rightarrow 3}\equiv\frac{n^{eq}_{1}}{R_{12\rightarrow n}}\,,\quad
n^{eq}_1=\frac{g_1}{2\pi^2}\int_{m_1}^\infty\!\!\!\!dE\,\frac{E\,\sqrt{E^2-m_1^2}}{\exp{\left(E/T\right)}\pm1}\;, 
\end{align}
where $n^{eq}_1$\,is the thermal equilibrium number density of particle\,$1$ with the `heavy' mass $m_1>T$.  Combining Eq.\,(\ref{Thermal_Rate}) with  Eq.\,(\ref{Reaction_Time}) we obtain
\begin{align}\label{RelaxationTime}
&\frac{\tau_{12\rightarrow3}}{ \tau^0_3}=  
\frac{2\pi^2 n^{eq}_1/T^3}{g_3(m_3/T)^2\,K_1(m_3/T)}\,, \quad 
n^{eq}_1\simeq g_1\left(\frac{m_1 T}{2\pi}\right)^{3/2}e^{-m_1/T},
\end{align}
where, conveniently, the relaxation time does not depend on the abundant and often relativistic heat bath component $2$, {\it e.g.\/} $l^\pm,\pi,\nu,\gamma$. The density of heavy particles\,$1$\,and\,$3$ can in general be well approximated using the leading and usually nonrelativistic Boltzmann term as shown above.

In general, the reaction rates for inelastic collision process capable of changing particle number, for example $\pi\pi\to K^0$, is suppressed by the factor $\exp{(-m_{K^0}/T)}$. On the other hand, there is no suppression for the elastic momentum and energy exchanging particle collisions in plasma. We conclude that for the case $m\gg T$, the dominant collision term in the relativistic Boltzmann equation is the elastic collision term, keeping all heavy particles in kinetic energy equilibrium with the plasma. This allows us to study the particle abundance in plasma presuming the energy-momentum statistical distribution equilibrium exists. This insight was discussed in detail in the preparatory phase of laboratory exploration of hot hadron and quark matter, see~\cite{Koch:1986ud}. In order to study the particle abundance in the Universe when $m\gg T$, instead of solving the exact Boltzmann equation, we can separate the fast energy-momentum equilibrating collisions from the slow particle number changing inelastic collisions. In the following we explore the rates of inelastic collision and compare the relaxation times of particle production in all relevant reactions with the Universe expansion rate.



It is common to refer to particle freeze-out as the epoch where a given type of particle ceases to interact with other particles. In this situation the particle abundance decouples from the cosmic plasma, a chemical nonequilibrium and even complete abundance disappearance of this particle can happen; the condition for the given reaction $1+2\rightarrow 3$ to decouple is
\begin{align}
\tau_{12\rightarrow 3}(T_f)=1/H(T_f),
\end{align}
where $T_f$ is the freeze-out temperature.
In the epoch of interest, $150\,\mathrm{MeV}>T>10\,\mathrm{MeV}$, the Universe is dominated by radiation and effectively massless matter behaving like radiation. The Hubble parameter can be written as~\cite{Kolb:1990vq}
\begin{align}\label{H2g}
H^2=H^2_{rad}\left(1+\frac{\rho_{\pi,\,\mu,\,\rho}}{\rho_\mathrm{rad}}+\frac{\rho_\mathrm{strange}}{\rho_\mathrm{rad}}\right)=\frac{8\pi^3G_\mathrm{N}}{90}g^e_\ast T^4,\qquad H^2_\mathrm{rad}=\frac{8\pi G_\mathrm{N}\,\rho_\mathrm{rad}}{3},
\end{align}
where: $g^e_\ast$ is the total number of effective relativistic `energy' degrees of freedom; $G_\mathrm{N}$ is the Newtonian constant of gravitation; the `radiation' energy density includes $\rho_\mathrm{rad}=\rho_\gamma+\rho_\nu+\rho_{e^\pm}$ for photons, neutrinos, and massless electrons(positrons). The massive-particle correction is $\rho_{\pi,\,\mu,\,\rho}=\rho_\pi+\rho_\mu+\rho_\rho$; and at highest $T$ of interest, also of (minor) relevance, $\rho_\mathrm{strange}=\rho_{K^0}+\rho_{K^\pm}+\rho_{K^\ast}+\rho_{\eta}+\rho_{\eta^\prime}$.
%~~~~~~~Figure~~~~~~~~~~~~~~~~~~~~~~~~~~~~~~~~~~~~~~~~~~~~~~~~~~~~~~~~~~~~~~~~~~~~~~~~~~
\begin{figure}[ht]
%\begin{center}
\centering
%\includegraphics[width=0.95\linewidth]{./plots/Strangeness_Hubble_C.jpg}
\includegraphics[width=1\linewidth]{./plots/Strangeness_Hubble002.jpg}
\includegraphics[width=1\linewidth]{./plots/Strangeness_Hubble003.jpg}
\caption{\cccite{universe9070309}, adapted from Ref.~\cite{Yang:2021bko} and thesis of C.T.Yang \cite{Yang:2024ret}. Hadronic relaxation reaction times, see Eq.\,(\ref{Reaction_Time}), as a function of temperature $T$, are compared to Hubble time $1/H$ (black solid line). At bottom the horizontal black-dashed line is the natural (vacuum) lifespan of $\rho$.}
\label{reaction_time_tot}
%\end{center}
\end{figure}
%~~~~~~~~~~~~~~~~~~~~~~~~~~~~~~~~~~~~~~~~~~~~~~~~~~~~~~~~~~~~~~~~~~~~~~~~~~~~~~~~~~~~~


When presenting the reaction rates and quoting decoupling as a function of temperature $T$, we must remember that for a temperature range $50\,\mathrm{MeV}>T>5$\,MeV, we have $10^{-1}<dT/dt<10^{-4}$\,MeV/$\mu$s. We estimate the width of freezeout temperature interval $\Delta T_f$  as follows:
\begin{align}
\frac{1}{\Delta T_f}\equiv \left[\frac{1}{(\Gamma_{12\to3}/H)}\frac{d(\Gamma_{12\to3}/H)}{dT}\right]_{T_f},\quad \Gamma_{12\to3}\equiv\frac{1}{\tau_{12\to3}}.
\end{align}
Using Eq.(\ref{H2g}) and Eq.(\ref{RelaxationTime}) and considering the temperature range $50\,\mathrm{MeV}>T>5$\,MeV with $g^e_\ast\approx\mathrm{constant}$ we obtain using the Boltzmann approximation to describe the  massive particles\,$1$\,and\,$3$
\begin{align}\label{DeltaFreezeout}
 \frac{\Delta T_f}{ T_f} \approx\frac{T_f  }{ m_3 - m_1 -2T_f}\,,\quad m_3 - m_1>> T_f\,.
\end{align}
The width of freeze-out is shown in the right column in Table~\ref{FreezeoutTemperature_table}. We see a range of $2$-$10\%$. Therefore it is justified to consider as a decoupling condition in time the value of temperature at which the pertinent rate cross the Hubble expansion rate, see Fig.~\ref{reaction_time_tot}.
 
In Fig.~\ref{reaction_time_tot} we plot the  hadronic reaction relaxation times $\tau_{i}$ in the meson sector as a function of temperature compared to Hubble time $1/H$.
It shows that the weak interaction reaction $\mu^\pm+\nu_{\mu}\rightarrow K^\pm$ becomes slower compared to the Universe expansion near temperature $T_f^{K^\pm}=33.8\,\mathrm{MeV}$, signaling the onset of abundance nonequilibrium for $K^\pm$. For $T<T_f^{K^\pm}$, the reactions $\mu^\pm+\nu_{\mu}\rightarrow K^\pm$ decouples from the cosmic plasma; the corresponding detailed balance can be broken and the decay reactions $K^\pm\rightarrow\mu^\pm+\nu_{\mu}$ are acting like a (small) ``hole'' in the strangeness abundance ``pot''. If other strangeness production reactions did not exist, strangeness would disappear as the Universe cools below $T_f^{K^\pm}$. However, we have other reactions: $l^++l^-\leftrightarrow\phi$, $\pi+\pi\leftrightarrow K$, and $\rho+\pi\leftrightarrow\phi$ can still produce the strangeness in cosmic plasma and the rate is very large compared to the weak interaction decay.
%~~~~~~~~~~~~~~~~~~~~~~~~~~~~~~~~~~~~~~~~~~~~~~~~~~~~~~~~~~~~~~~~~~~~~~~~~~~~~~~~~~~~~~~~~
\begin{table}%[h]
\centering
\begin{tabular}{c| c| c}
\hline\hline
Reactions &Freeze-out Temperature (MeV) & {$\Delta T_f$\,(MeV)} \\
\hline
$\mu^\pm\nu\rightarrow K^\pm$ & $T_f=33.8$\,MeV & {$3.5$ \,MeV}\\ 
\hline
$e^+e^-\rightarrow \phi$ & $T_f=24.9$\,MeV &{$0.6$\,MeV}\\
$\mu^+\mu^-\rightarrow\phi$ & $T_f=23.5$\,MeV &{$0.6$\,MeV}\\
\hline
 $\pi\pi\rightarrow K$ & $T_f=19.8$\,MeV&{$1.2$\,MeV}\\
\hline
$\pi\pi\rightarrow\rho$ & $T_f=12.3$\,MeV&{$0.2$\,MeV}\\
\hline\hline
\end{tabular}
\caption{The characteristic strangeness reaction and their freeze-out temperature and temperature width in early Universe.}
\label{FreezeoutTemperature_table} 
\end{table}

%~~~~~~~~~~~~~~~~~~~~~~~~~~~~~~~~~~~~~~~~~~~~~~~~~~~~~~~~~~~~~~~~~~~~~~~~~ 

In Table~\ref{FreezeoutTemperature_table} we show the characteristic strangeness reactions and their freeze-out temperatures in the early Universe. The intersection of strangeness reaction times with $1/H$ occurs for $l^-+l^+\rightarrow\phi$ at $T_f^\phi=25\sim23\,\mathrm{MeV}$, and for $\pi+\pi\rightarrow K$ at $T_f^K=19.8\,\mathrm{MeV}$, for $\pi+\pi\rightarrow\rho$ at $T_f^\rho=12.3\,\mathrm{MeV}$. The reactions $\gamma+\gamma\rightarrow\pi$ and $\rho+\pi\leftrightarrow\phi$ are faster compared to $1/H$. However, the $\rho\to\pi+\pi$ lifetime (black dashed line in Fig.~\ref{reaction_time_tot}) is smaller than the reaction $\rho+\pi\leftrightarrow\phi$; in this case, most of $\rho$-meson decays faster, thus are absent and cannot contribute to the strangeness creation in the meson sector. Below the temperature $T<20$\,MeV, all the detail balances in the strange meson reactions are broken and the strangeness in the meson sector should disappear rapidly, were it not for the small number of baryons present in the Universe.



%~~~~~~~~~~~~~~~~~~~~~~~~~~~~~~~~~~~~~~~~~~~~~~~~~~~~~~~~



\subsection{Strangeness production/ exchange rate in hyperons}
In order to understand strangeness in hyperons in the baryonic domain, we now consider the strangeness production reaction $\pi +N\rightarrow K+\Lambda$, the strangeness exchange reaction $\overline{K}+N\rightarrow \Lambda+\pi$; and the strangeness decay $\Lambda\rightarrow N+\pi$. The competition between different strangeness reactions allows strange hyperons and antihyperons to influence the dynamic nonequilibrium condition, including development of $\langle s-\bar s\rangle \ne 0$. %The cross sections $\sigma_{\overline{K}N\rightarrow \Lambda\pi}$ and $\sigma_{\pi N\rightarrow K\Lambda}$ are obtained from experiment.

To evaluate the reaction rate in two-body reaction $1+2\rightarrow3+4$ in the Boltzmann approximation we can use the reaction cross section $\sigma(s)$ and the relation~\cite{Letessier:2002ony}:
\begin{align}
R_{12\rightarrow34}=\frac{g_1g_2}{32\pi^4}\frac{T}{1+I_{12}}\!\!\int^\infty_{s_{th}}\!\!\!\!ds\,\sigma(s)\frac{\lambda_2(s)}{\sqrt{s}}\!K_1\!\!\left({\sqrt{s}}/{T}\right),
\end{align}
where $K_1$ is the Bessel function of order $1$ and the function $\lambda_2(s)$ is defined as
\begin{align}
\lambda_2(s)=\left[s-(m_1+m_2)^2\right]\left[s-(m_1-m_2)^2\right],
\end{align}
with $m_1$ and $m_2$, $g_1$ and $g_2$ as the masses and degeneracy of the initial interacting particle. The factor $1/(1+I_{12})$ is introduced to avoid double counting of indistinguishable pairs of particles; we have $I_{12}=1$ for identical particles and $I_{12}=0$ for others. 

The thermal averaged cross sections for the strangeness production and exchange processes are about $\sigma_{\pi N\rightarrow K\Lambda}\sim0.1\,\mathrm{mb}$ and $\sigma_{\overline{K}N\rightarrow \Lambda\pi}=1\sim3\,\mathrm{mb}$ in the energy range in which we are interested~\cite{Koch:1986ud}. The cross section can be parameterized as follows:\\
1) For the cross section $\sigma_{\overline{K}N\rightarrow \Lambda\pi}$ we use~\cite{Koch:1986ud}
 \begin{align}
 \sigma_{\overline{K}N\rightarrow \Lambda\pi}=\frac{1}{2}\left(\sigma_{K^-p\rightarrow \Lambda\pi^0}+\sigma_{K^-n\rightarrow \Lambda\pi^-}\right)\,.
\end{align}
Here the experimental cross sections can be parameterized as 
\begin{align}
&\sigma_{K^-p\rightarrow \Lambda\pi^0}\!\!=\!\!\left(\begin{array}{l}\!\!1479.53\mathrm{mb}\!\cdot\!\exp{\left(\frac{-3.377\sqrt{s}}{\mathrm{GeV}}\right)},\; \mathrm{for}\,\sqrt{s_m}\!\!<\!\!\sqrt{s}\!<\!3.2\mathrm{GeV} \\ \\0.3\mathrm{mb}\!\cdot\!\exp{\left(\frac{-0.72\sqrt{s}}{\mathrm{GeV}}\right)},\; \mathrm{for}\sqrt{s}>3.2\mathrm{GeV}\end{array}\right.\\
&\sigma_{K^-n\rightarrow \Lambda\pi^-}\!\!=\!\!1132.27\mathrm{mb}\!\cdot\!\exp{\left(\frac{-3.063\sqrt{s}}{\mathrm{GeV}}\right)},\; \mathrm{for}\sqrt{s}>1.699\mathrm{GeV},
\end{align}
where $\sqrt{s_m}=1.473$ GeV.\\
2) For the cross section $\sigma_{\pi N\rightarrow K\Lambda}$ we use~\cite{Cugnon:1984pm}
\begin{align}
&\sigma_{\pi N\rightarrow K\Lambda}=\frac{1}{4}\times\sigma_{\pi p\rightarrow K^0\Lambda}\,.
\end{align}
The experimental $\sigma_{\pi p\rightarrow K^0\Lambda}$  can be approximated as follows
\begin{align}
\sigma_{\pi p\rightarrow K^0\Lambda}=\left(\begin{array}{l}\frac{0.9\mathrm{mb}\cdot\left(\sqrt{s}-\sqrt{s_0}\right)}{0.091\mathrm{GeV}},\; \mathrm{for} \sqrt{s_0}<\sqrt{s}<1.7\mathrm{GeV} \\ \\ \frac{90\mathrm{MeV\cdot mb}}{\sqrt{s}-1.6\mathrm{GeV}},\; \mathrm{for}\sqrt{s}>1.7\mathrm{GeV},\end{array}\right.
 \end{align}
 with $ \sqrt{s_0}=m_\Lambda+m_K$. 

Given the cross sections, we obtain the thermal reaction rate per volume for strangeness exchange reaction seen in Fig.~\ref{Lambda_Rate_volume.fig}. We see that around $T=20$\,MeV, the dominant reactions for the hyperon $\Lambda$ production is $\overline{K}+N\leftrightarrow\Lambda+\pi$. At the same time, the $\pi+\pi\to K$ reaction becomes slower than Hubble time and kaon $K$ decay rapidly in the early Universe. However, the anti-kaons $\overline K$ produce the hyperon $\Lambda$ because of the strangeness exchange reaction $\overline{K}+N\rightarrow\Lambda+\pi$ in the baryon-dominated Universe. We have strangeness in $\Lambda$ and it disappears from the Universe via the decay $\Lambda\rightarrow N+\pi$. Both strangeness and anti-strangeness disappear because of the $K\rightarrow\pi+\pi$ and $\Lambda\rightarrow N+\pi$, while the strangeness abundance $s = \bar{s}$ in the early Universe remains.

%~~~~~~~~~~~~~~~~~~~~~~~~~~~~~~~~~~~~~~~~~~~~~~~~~~~~~~~~~~~~~~~~~~~~~~~~~~~~~~~~
\begin{figure}[ht]
%\begin{center}
\centering
\includegraphics[width=0.9\linewidth]{./plots/NewHyperonRate_C.jpg}
\caption{\cccite{universe9070309}, adapted from Ref.~\cite{Yang:2021bko} and thesis of C.T.Yang \cite{Yang:2024ret}. Thermal reaction rate $R$ per volume and time for important hadronic strangeness production and exchange processes as a function of temperature $150\,\mathrm{MeV}> T>10\,\mathrm{MeV}$ in the early Universe.}
\label{Lambda_Rate_volume.fig}
%\end{center}
\end{figure}
%~~~~~~~~~~~~~~~~~~~~~~~~~~~~~~~~~~~~~~~~~~~~~~~~~~~~~~~~~~~~~~~~~~~~~~~~~~~~~~

Around $T=12.9$\,MeV, the reaction $\Lambda+\pi\rightarrow\overline{K}+N$ becomes slower than the strangeness decay $\Lambda\leftrightarrow N+\pi$ and shows that at the low temperature the $\Lambda$ particles are still in equilibrium via the reaction $\Lambda\leftrightarrow N+\pi$ and little strangeness remains in the $\Lambda$. Then strangeness abundance becomes asymmetric $s\gg \bar{s}$, which implies that the assumption for strangeness conservation can only be valid until the temperature $T\sim13$\,MeV. Below this temperature a new regime opens up in which the tiny residual strangeness abundance is governed by weak decays with no re-equilibration with mesons. Also, in view of baron asymmetry, $\langle s-\bar s\rangle \ne 0$.

The primary conclusion of this first study of strangeness production and content in the early Universe, following on QGP hadronization, is that the relevant temperature domains indicate a complex interplay between baryon and meson (strange and non-strange) abundances and non-trivial decoupling from equilibrium for strange and non-strange mesons. We believe that this work contributes to the opening of a new and rich domain in the study of the Universe evolution in the future. 

%~~~~~~~~~~~~~~~~~~~~~~~~~~~~~~~~~~~~~~~~~~~~~~~~~

%{Introduction\daggerfootnote{This chapter has been published previously as \citet{Gottbrath1999}.}}



\section{Neutrinos in cosmic plasma - CTY}\label{Neutrino}
Neutrinos are fundamental particles and play an important role in the evolution of the Universe. In the early Universe the neutrinos are kept in equilibrium with cosmic plasma via the weak interaction. The neutrino-matter interactions play a crucial role in  understanding of neutrinos evolution in the early Universe (such as the neutrino freezeout) and the later Universe (the property of today's neutrino background). In this chapter, I will examine the neutrino coherent and incoherent scattering with matter and their application in cosmology. The investigation of the relation between the effective number of neutrinos $N^{\mathrm{eff}}_\nu$ and lepton asymmetry $L$ after neutrino freezeout and its impact on Universe expansion is also discussed in this chapter. 

%{Introduction\daggerfootnote{This chapter has been published previously as \citet{Gottbrath1999}.}}

%~~~~~~~~~~~~~~~~~~~~~~~~~~~~~~~~~~~~~~~~~~~~~~~~~

%\section{Neutrino in particle physics}
%From the standard neutrino oscillation theory, the neutrino flavor eigenstate $\nu^{\alpha}$ can be described as a superposition of mass eigenstates $\nu_{k}$ as follow:
%\begin{align}\label{NuFlavors}
%	\nu_{\alpha}=\sum_k^nU^\ast_{\alpha k}\nu_{k}, \qquad\alpha=e,\mu,\tau,\qquad k=1,2,3,\dots,n
%\end{align}
%where $U$ is the the Pontecorvo-Maki-Nakagawa-Sakata (PMNS) mixing matrix~[\cite{King:2013eh,FernandezMartinez:2016lgt}] which are both in general complex and unitary. Evaluating freely propagating plane waves in the relativistic limit yields the vacuum oscillation probability between flavors $\nu_\alpha$ and $\nu_\beta$ written as~[\cite{ParticleDataGroup:2022pth}]
%\begin{align}\label{NuOscillation}
%  P_{\alpha\rightarrow\beta}
% =&\delta_{\alpha\beta}-4\sum_{i<j}^n \mathrm{Re}\left[U_{\alpha i}U^\ast_{\beta i}U^\ast_{\alpha j}U_{\beta j}\right]\sin^2\!\!\left(\frac{\Delta m^2_{ij}L}{4E}\right)\notag\\
% &+2\sum_{i<j}^n \mathrm{Im}\left[U_{\alpha i}U^\ast_{\beta i}U^\ast_{\alpha j}U_{\beta j}\right]\sin\!\!\left(\frac{\Delta m^2_{ij}L}{2E}\right)
% ,\qquad\Delta m^2_{ij}\equiv{m^2_i-m^2_j}
%\end{align}
%where $L$ is the distance traveled by the neutrino between production and detection. The square mass difference $\Delta m^2_{ij}$ has been experimentally measured, from the neutrino oscillation experiment~[\cite{ParticleDataGroup:2022pth}]:
%\begin{align}
%&\Delta{m}_{21}^2=7.39^{+0.21}_{-0.20}\times10^{-5}\,\mathrm{eV}^2,\\
%&\Delta{m}_{32}^2=2.45^{+0.03}_{-0.03}\times10^{-3}\,\mathrm{eV}^2.
%\end{align}
%and neutrino mass eigenvalue can be ordered in the normal mass hierarchy ($m_1\ll m_2<m_3$) or inverted mass hierarchy ($m_3\ll m_1<m_2$). All three mass states remained relativistic until the temperature dropped below their rest mass. These results allow for the possibility that one mass eigenstate or two mass eigenstates of neutrinos may become non-relativistic today. 

%In this chapter, we discuss the neutrino interaction with matter coherent/incoherent scattering, an overview of neutrino freezeout and effective number of neutrino in early Universe.


%~~~~~~~~~~~~~~~~~~~~~~~~~~~~~~~~~~~~~~~~~~~~~~~~~

\section{Matrix element for neutrino coherent/ incoherent scattering}
 According to the standard model, neutrinos interact with other particles via the Charged-Current(CC) and Neutral-Current(NC) interactions. Their Lagrangian can be written as~[\cite{Giunti:2007ry}]
\begin{align}
&\mathcal{L}^{CC}=\frac{g}{2\sqrt{2}}\left(j^\mu_W\,W_\mu+{j^\mu_W}^\dagger\,W^\dagger_\mu\right),\qquad\mathcal{L}^{NC}=-\frac{g}{2\cos{\theta_w}}\,j^\mu_Z\,Z_\mu,
\end{align}
where $g=e\sin\theta_w$, $W^\mu$ and $Z^\mu$ are W and Z boson gauge fields, and $j^\mu_W$ and $j^\mu_Z$ are the charged-current and neutral-current separately. In the limit of energies lower than the $W(m_w=80\,\mathrm{GeV})$ and $Z(m_z=91\,\mathrm{GeV})$ gauge bosons, the effective Lagrangians are given by
\begin{align}\label{L_low}
\mathcal{L}^{CC}_{eff}=-\frac{G_F}{\sqrt{2}}\,j^\dagger_{W\,\mu}\,j^\mu_W,\qquad
\mathcal{L}^{NC}_{eff}=-\frac{G_F}{\sqrt{2}}\,j^\dagger_{Z\,\mu}\,j^\mu_Z,\qquad \frac{G_F}{\sqrt{2}}=\frac{g^2}{8m^2_W},
\end{align}
where $G_F=1.1664\times10^{-5}\,\mathrm{GeV}^{-2}$ is the Fermi constant, which is one of the important parameters that determine the strength of the weak interaction rate. When neutrinos interact with matter, based on the neutrino's wavelength, they can undergo two types of scattering processes: coherent scattering and incoherent scattering with the particles in the medium. 

With coherent scattering, neutrinos interact with the entire  composite system rather than individual particles within the system. The coherent scattering is particularly relevant for low-energy neutrinos when the wavelength of neutrino is much larger than the size of system. In $1978$, Lincoln Wolfenstein pointed out that the coherent forward scattering of neutrinos off matter could be very important in studying the behavior of neutrino flavor oscillation in a dense medium~[\cite{PhysRevD.17.2369}]. The fact that neutrinos propagating in matter may interact with the background particles can be described by the picture of free neutrinos traveling in an effective potential.

For incoherent scattering, neutrinos interact with particles in the medium individually. Incoherent scattering is typically more prominent for high-energy neutrinos, where the wavelength of neutrino is smaller compared to the spacing between particles. Study of incoherent scattering of high-energy neutrinos is important for understanding the physics in various astrophysical systems (e.g. supernova, stellar formation) and the evolution of the early Universe.

In this section, we discuss the coherent scattering between long wavelength neutrinos and atoms, and study the effective potential for neutrino coherent interaction. Then we present the matrix elements that describe the incoherent interaction between high energy neutrinos and other fundamental particles in the early Universe. Understanding these matrix elements is crucial for comprehending the process of neutrino freeze-out in the early Universe.

%~~~~~~~~~~~~~~~~~~~~~~~~~~~~~~~~~~~~~~~~~~~~~~~~~~~~~~~~~~~~~~~~~~~~~~~

\subsection{Long wavelength limit of neutrino-atom coherent scattering}\label{LongWavelength}
According to the standard cosmological model, the Universe today is filled with the cosmic neutrinos with temperature $T_{\nu}^0=1.9 \,\mathrm{K}=1.7\times10^{-4}\,\mathrm{eV}$.
The average momentum of present-day relic neutrinos is given by $\langle p_\nu^0\rangle\approx3.15\,T_\nu^0$ and the typical wavelength $\lambda_\nu^0={2\pi}/{\langle p_\nu^0\rangle}\approx2.3\times10^5\,\mathrm{\AA}$, which is much larger than the radius at the atomic scale, such as the Bohr radius $R_{\mathrm{atom}}=0.529\,\mathrm{\AA}$. In this case we have the long wavelength condition $\lambda_\nu\gg\,R_{\mathrm{atom}}$ for cosmic neutrino background today.  

Under the condition $\lambda_\nu\gg\,R_{\mathrm{atom}}$, when the neutrino is scattering off an atom, the interaction can be coherent scattering [\cite{PhysRevD.38.32,PhysRevD.21.663,Papavassiliou:2005cs}]. According to the principles of quantum mechanics, with neutrino scattering it is impossible to identify which scatters the neutrino interacts with and thus it is necessary to sum over all possible contributions. In such circumstances, it is appropriate to view the scattering reaction as taking place on the atom as a whole, i.e.,
\begin{align}
\nu+\mathrm{Atom}\longrightarrow\nu+\mathrm{Atom}.
\end{align}

Considering a neutrino elastic scattering off an atom which is composed of $Z$ protons, $N$ neutrons and $Z$ electrons. For the elastic neutrino atom scattering, the low-energy neutrinos scatter off both atomic electrons and nucleus. For nucleus parts, we consider that the neutrinos interact via the $Z^0$ boson with a nucleus as
\begin{align}
\nu+A^{Z}_N\longrightarrow\nu+A^{Z}_N.
\end{align}
In this process a neutrino of any flavor scatters off a nucleus with the same strength. Therefore, the scattering will be insensitive to neutrino flavor. On the other hand, the neutrons can also interact via the $W^\pm$ with nucleus as 
\begin{align}
\nu_l+A^{Z}_N\longrightarrow\,l^-+A^{{Z}+1}_N,
\end{align}
which is a quasi-elastic process for neutrino scattering with the nucleus; we have $A^{Z_e}_N\rightarrow\,A^{{Z_e}+1}_N$. Since this process will change the nucleus state into an excited one, we will not consider its effect here. For detail discussion pf quasi-elastic scattering see ~[\cite{SajjadAthar:2022pjt}].

For atomic electrons, the neutrinos can interact via the $Z^0$ and $W^\pm$ bosons with electrons for different flavors, we have
\begin{align}
&\nu_e+e^-\longrightarrow\nu_e+e^-\,\,\,(\mathrm{Z^0,\,W^\pm\,exchange}),\\
&\nu_{\mu,\tau}+e^-\longrightarrow\nu_{\mu,\tau}+e^-\,\,\,(\mathrm{Z^0\,exchange}).
\end{align}
Because of the fact that the coupling of $\nu_e$ to electrons is quite different from that of $\nu_{\mu,\tau}$, one may expect large differences in the behavior of $\nu_e$ scattering compared to the other neutrino types.


%~~~~~~~~~~~~~~~~~~~~~~~~~~~~~~~~~~~~~~~~~~~~~~~~~~
\subsubsection{Neutrino-atom coherent scattering amplitude/matrix element} 
This section considers how a neutrino scatters from a composite system, assumed to consist of $N$ individual constituents at positions $x_i,\,i=1,2,....N$. Due to the superposition principle, the scattering amplitude $\mathcal{M}_\mathrm{sys}(\bold{p}^\prime,\bold{p})$ for scattering from an incoming momentum $\mathbf{p}$ to an outgoing momentum $\bold{p}^\prime$ is given as the sum of the contributions from each constituent [\cite{Freedman:1977xn,Papavassiliou:2005cs}]:
\begin{align}
\mathcal{M}_\mathrm{sys}(\bold{p}^\prime,\bold{p})=\sum_i^N\,\mathcal{M}_i(\bold{p}^\prime,\bold{p})\,e^{i\bold{q}\cdot\bold{x}_i},
\end{align}
where $\bold{q}=\bold{p}^\prime-\bold{p}$ is the momentum transfer and the individual amplitudes $\mathcal{M}_i(\bold{p}^\prime,\bold{p})$ are added with a relative phase factor determined by the corresponding wave function. %The transition probability is then given by
%\begin{align}
%\mathcal{P}_{\mathrm{sys}}(\bold{p}^\prime,\bold{p})&=|\mathcal{M}_\mathrm{sys}(\bold{p}^\prime,\bold{p})|^2\notag\\
%&=\sum_i|\mathcal{M}_i(\bold{p}^\prime,\bold{p})|^2+\sum_{i,j}^{i\neq\,j}\mathcal{M}_i(\bold{p}^\prime,\bold{p})\mathcal{M}_j^\dagger(\bold{p}^\prime,\bold{p})\,e^{i\bold{q}\cdot\left(\bold{x}_j-\bold{x}_i\right)}.
%\end{align} 
In principle, due to the presence of the phase factors, major cancellation may take place among the terms for the condition $|\bold{q}|R\gg1$, where $R$ is the size of the composite system, and the scattering would be incoherent. However, for the momentum small compared to the inverse target size, i.e., $|\bold{q}|R\ll1$, then all phase factors may be approximated by unity and contributions from individual scatters add coherently. 


In the case of neutrino coherent scattering with an atom: If we consider sufficiently small momentum transfer to an atom from a neutrino which satisfies the coherence condition, i.e., $|\bold{q}|R_{\mathrm{atom}}\ll1$, then the relevant phase factors have little effect, allowing us to write the transition amplitude as [\cite{Nicolescu:2013rxa}]
\begin{align}
\label{M_atom}
\mathcal{M}_\mathrm{atom}=\sum_t\frac{G_F}{\sqrt{2}}\left[\overline{u}(p^\prime_\nu)\gamma_\mu\left(1-\gamma_5\right)u(p_\nu)\right]\left[\overline{u}(p^\prime_t)\gamma^\mu\left(c^t_V-c^t_A\gamma^5\right)u(p_t)\right],
\end{align}
where $t$ is all the target constituents (Z protons, N neutrons and Z electrons). The transition amplitude includes contributions from both charged and neutral currents, with
\begin{align}\label{CC_int}
&\mathrm{Charged\,\,Current}: c^t_V=c^t_A=1\\
\label{NC_int}
&\mathrm{Neutral\,\, Current}: c^t_V=I_3-2\mathcal{Q}\sin^2\theta_w,\qquad c^t_A=I_3
\end{align}
where $I_3$ is the weak isospin, $\theta_w$ is the Weinberg angle, and $\mathcal{Q}$ is the particle electric charge. 

Considering the target can be regarded as an equal mixture of spin states $s_z=\pm1/2$, and we can simplify the transition amplitude by summing the coupling constants of the constituents [\cite{PhysRevD.21.663,Sehgal:1986gn}]. We have
\begin{align}
\label{Transition}
\mathcal{M}_\mathrm{atom}=&\frac{G_F}{2\sqrt{2}}\left[\overline{u}(p^\prime_\nu)\gamma_\mu\left(1-\gamma_5\right)u(p_\nu)\right]\notag\\&\bigg[\overline{u}(p^\prime_{a})\sum_t\left(C_L+C_R\right)_t\gamma^\mu\,u(p_{a})-\overline{u}(p^\prime_{a})\sum_t\left(C_L-C_R\right)_t\gamma^\mu\gamma^5u(p_{a})\bigg],
\end{align}
where the $u(p_\nu)$, $u(p^\prime_\nu)$ are the initial and final neutrino states and $u(p_a)$, $u_(p^\prime_a)$ are the initial and final states of the target atom. 
The coupling coefficients $C_L$ and $C_V$ are defined as
\begin{align}
&C_L=c_V+c_A,\,\,\,\,\,C_R=c_V-c_A,
\end{align}
where the coupling constants for neutrino scattering with proton, neutron, and electron are given by Table.~\ref{Table_coupling}. The coupling constants for  $\nu_{\mu,\tau}$ are the same as for the $\nu_e$, excepting  the absence  of a charged current in neutrino-electron scattering.
%%%%%%%%%%%%%%%%%%%%%%%%%%%%%%%%%%%%%%%%%%%%%%%%%%%%%%
\begin{table}[h]
\begin{tabular}[c]{c|c|c|c|c}
\hline\hline
& Electron ($Z^0$ boson) & Electron ($W^\pm$ boson) & Proton (uud) & Neutron (udd)\\
\hline
$C_L$ & $-1+2\sin^2\theta_w$ & $2$ & $1-2\sin^2\theta_w$ & $-1$ \\
\hline
$C_R$ & $2\sin^2\theta_w$ & $0$ &$-2\sin^2\theta_w$ & $0$ \\
\hline\hline
\end{tabular}
\caption{The coupling constants for neutrino scattering with proton, neutron, and electron.}
\label{Table_coupling}  
\end{table}
%%%%%%%%%%%%%%%%%%%%%%%%%%%%%%%%%%%%%%%%%%%%%%%%%%%%%

Given the neutrino-atom coherent scattering amplitude Eq.(\ref{Transition}), the transition matrix element can be written as
\begin{align}
\label{scattering_matrix}
|\mathcal{M}_{\mathrm{atom}}|^2=\frac{G^2_F}{8}L_{\alpha\beta}^{\mathrm{neutrino}}\,\Gamma^{\alpha\beta}_{\mathrm{atom}},
\end{align}
where the neutrino tensor $L_{\alpha\beta}^{\mathrm{neutrino}}$ is given by
\begin{align}
\label{neutrino_tensor}
L_{\alpha\beta}^{\mathrm{neutrino}}
&=\mathrm{Tr}\left[\gamma_\alpha\left(1-\gamma_5\right)(\slashed{p}_\nu+m_\nu)\gamma_\beta\left(1-\gamma_5\right)(\slashed{p}^\prime_\nu+m_\nu)\right]\notag\\
&=8\left[(p_\nu)_\alpha\,(p^\prime_{\nu})_\beta+(p_\nu)^\prime_\alpha\,(p_\nu)_\beta-g_{\alpha\beta}(p_\nu\cdot\,p_\nu^\prime)+i\epsilon_{\alpha\sigma\beta\lambda}(p_\nu)^\sigma(p^\prime_\nu)^\lambda\right],
\end{align}
and the atomic tensor $\Gamma^{\alpha\beta}_\mathrm{atom}$ can be written as
\begin{align}
\label{atomic_tensor}
\Gamma^{\alpha\beta}_\mathrm{atom}
&=\mathrm{Tr}\bigg[(C_{LR}\gamma^\alpha-C^\prime_{LR}\gamma^\alpha\gamma^5)(\slashed{p}_a+M_a)(C_{LR}\gamma^\beta-C^\prime_{LR}\gamma^\beta\gamma^5)(\slashed{p}^\prime_a+M_a)\bigg]\notag\\
&=4\bigg\{(C^2_{LR}+C^{\prime2}_{LR})\left[(p_a)^\alpha\,(p^\prime_a)^\beta+(p_a)^{\prime\alpha}\,(p_a)^\beta\right]\notag\\
&\qquad-g^{\alpha\beta}\bigg[(C^2_{LR}-C^{\prime2}_{LR})(p_a\cdot\,p_a^\prime)-(C^2_{LR}-C^{\prime2}_{LR})M^2_a\bigg]\notag\\&\qquad\qquad+2iC_{LR}C^\prime_{LR}\epsilon^{\alpha\sigma^\prime\beta\lambda^\prime}(p_a)_{\sigma^\prime}(p^\prime_a)^{\lambda^\prime}\bigg\},
\end{align}
where $M_a$ is the target atom's mass $(M_a = AM_\mathrm{nucleon}, A=Z+N)$, the coupling constants $C_{LR}$ and $C^\prime_{LR}$ are defined by
\begin{align}
C_{LR}=\sum_t(C_L+C_R)_t,\,\,\,\,\,\,C^\prime_{LR}=\sum_t(C_L-C_R)_t.
\end{align}
Substituting Eq.(\ref{neutrino_tensor}) and Eq.(\ref{atomic_tensor}) into Eq.(\ref{scattering_matrix}), then the transition matrix element for coherent elastic neutrino atom scattering is given by:
\begin{align}
|\mathcal{M}_{\mathrm{atom}}|^2&=\frac{G^2_F}{8}L_{\alpha\beta}^{\mathrm{neutrino}}\,\Gamma^{\alpha\beta}_{\mathrm{atom}}\notag\\
&=8G^2_F\bigg[(C_{LR}+C^\prime_{LR})^2\,(p_\nu\cdot\,p_a)(p^\prime_\nu\cdot\,p^\prime_a)+(C_{LR}-C^\prime_{LR})^2\,(p_\nu\cdot\,p^\prime_a)(p^\prime_\nu\cdot\,p_a)\notag\\&
\,\,\,\,\,\,-(C^2_{LR}-C^{\prime2}_{LR})M^2_a(p_\nu\cdot\,p_\nu^\prime)\bigg].
\end{align}
Taking the atom at rest in the laboratory frame, and considering small momentum transfer to an atom from a neutrino, i.e., $q^2=(p_\nu-p^\prime_\nu)^2=(p_a^\prime-p_a)^2\ll\,M^2_a$, we have
\begin{align}
&p_\nu\cdot\,p_a=E_\nu\,M_a,\\
&p_\nu^\prime\cdot\,p_a=E_\nu^\prime\,M_a\approx\,E_\nu\,M_a,\\
&p^\prime_\nu\cdot\,p^\prime_a=p^\prime_\nu\cdot(p_a+q)=E^\prime_\nu\,M_a\left[\left(1+\frac{q_0}{M_a}\right)-\frac{|p^\prime_\nu||q|}{M_a}\cos\theta\right]\approx\,E_\nu\,M_a,\\
&p_\nu\cdot\,p^\prime_a=p_\nu\cdot(p_a+q)=E_\nu\,M_a\left[\left(1+\frac{q_0}{M_a}\right)-\frac{|p^\prime_\nu||q|}{M_a}\cos\theta\right]\approx\,E_\nu\,M_a.
\end{align}
Then the transition matrix element for neutrino coherent elastic scattering off a rest atom can be written as
\begin{align}\label{M_general}
|\mathcal{M}_{\mathrm{atom}}|^2&=8\,G^2_F\,M_a\,E_\nu^2\left[C^2_{LR}\left(1+\frac{|p_\nu|^2}{E^2_\nu}\cos\theta\right)+3C^{\prime2}_{LR}\left(1-\frac{|p_\nu|^2}{3E_\nu^2}\cos\theta\right)\right],
\end{align}
which is consistent with the results in papers [\cite{PhysRevD.38.32,PhysRevD.21.663,Papavassiliou:2005cs,Smith:1985mta}].
From the above formula we found that the scattering matrix neatly divides into two distinct components: a vector-like component (first term) and an axial-vector like component (second term). They have different angular dependencies: the vector part has a $\left({|p_\nu|^2}/{E^2_\nu}\cos\theta\right)$ dependence, while the axial part has a $\left(-{|p_\nu|^2}/{3E_\nu^2}\cos\theta\right)$ behavior. However, in the case of the nonrelativistic neutrino, both angular dependencies can be neglected because of the limit $p_\nu\ll\,m_\nu$. 


Next, we consider the nonrelativistic electron neutrino $\nu_e$ scattering off an general atom with $Z$ protons, $N$ neutrons and $Z$ electrons. Then from Eq.~(\ref{M_general}), the matrix element can be written as
\begin{align}
\label{Probability_e}
|\mathcal{M}_{\mathrm{atom}}|^2&=8\,G^2_F\,M_a\,E_\nu^2\left[\left(3Z-A\right)^2\left(1+\frac{|p_\nu|^2}{E^2_\nu}\cos\theta\right)+3\left(3Z-A\right)^2\left(1-\frac{|p_\nu|^2}{3E_\nu^2}\cos\theta\right)\right]\notag\\
&\approx32\,G^2_F\,M_a\,E_\nu^2\left(3Z-A\right)^2,
\end{align}
where we neglect the angular dependence because of the nonrelativistic limit, and the coefficient $\left(3Z-A\right)^2$ for different target atoms are given in Table.(\ref{Table001}). On the other hand, for nonrelativistic $\nu_{\mu,\tau}$, the scattering matrix is given by
\begin{align}
\label{Probability_mt}
|\mathcal{M}_{\mathrm{atom}}|^2&=8\,G^2_F\,M_a\,E_\nu^2\left[\left(A-Z\right)^2\left(1+\frac{|p_\nu|^2}{E^2_\nu}\cos\theta\right)+3\left(A-Z\right)^2\left(1-\frac{|p_\nu|^2}{3E_\nu^2}\cos\theta\right)\right]\notag\\
&\approx32\,G^2_F\,M_a\,E_\nu^2\left(Z-A\right)^2,
\end{align}
where the coefficient $\left(Z-A\right)^2$ different target atoms are given in Table.(\ref{Table001}). The transition matrix for $\nu_e$ differs from that of $\nu_{\mu,\tau}$; this is due to the charged current reaction with the atomic electrons. Furthermore, the neutral current interaction for the electron and proton will cancel each other because of the opposite weak isospin $I_3$ and charge $\mathcal{Q}$. As a result, the coherent neutrino scattering from an atom is sensitive to the method of the neutrino-electron coupling.
%%%%%%%%%%%%%%%%%%%%%%%%%%%%%%%%%%%%%%%%%%
%%%%%%%%%%%%%%%%%%%%%%%%%%%%%%%%%%%%%%%%%%%%%%%%%%%%%%%%%%%%%%
\begin{table}[h]
\centering
\begin{tabular}{c|c|c}
\hline\hline
 Neutrino Flavor:&$\nu_e$ &$\nu_{\mu,\tau}$\\
\hline\hline
Target Atom & $(3Z-A)^2$  & $(Z-A)^2$\\
\hline
$H_2(A=2, Z=2)$ & $16$ & $0$\\
\hline
${}^{3}H_e(A=3, Z=2)$  & $9$ & $1$\\
\hline
$HD(A=3, Z=2)$ & $9$   & $1$\\
\hline
${}^{4}_2H_e(A=4, Z=2)$  &$4$ & $4$\\
\hline
$DD(A=4, Z=2)$  & $4$ & $4$\\
\hline
${}^{12}_{{}6}C(A=12, Z=6)$  & $36$& $36$\\
\hline\hline
\end{tabular}
\caption{The coefficients for transition amplitude and scattering probability of $\nu_e$ and $\nu_{\mu,\tau}$  coherent elastic scattering off different target atoms. The definition of atomic mass is $A=Z+N$, where $Z$ and $N$ are the number of protons and neutron respectively.}
\label{Table001}  
\end{table}%
%%%%%%%%%%%%%%%%%%%%%%%%%%%%%%%%%%%%%%%%%%%%%%%%%%%%%%%%%%%%%%

%%%%%%%%%%%%%%%%%%%%%%%%%%%%%%%%%%%%%%%%%%%%%%%%%%%%%%%%%%%%%%%%%%%%%%%%
%%%%%%%%%%%%%%%%%%%%%%%%%%%%%%%%%%%%%%%%%%%%%%%%%%%%%%%%%%%%%%%%%%%%%%%%
\subsubsection{Mean field potential for neutrino coherent scattering}
When neutrinos are propagating in matter and interacting with the background particles, they can be described by the picture of free neutrinos traveling in an effective potential~[\cite{PhysRevD.17.2369}]. In the following we describe the effective potential between neutrinos and the target atom, and generalize the potential to the case of neutrino coherent scattering with a multi-atom system.


Let us consider a neutrino elastic scattering off an atom which is composed of Z protons, N neutrons and Z electrons. For the elastic neutrino atom scattering, the low-energy neutrinos are scattering off both atomic electrons and the nucleus. Considering the effective low-energy CC and NC interactions, the effective Hamiltonian in current-current interaction form can be written as 
\begin{align}
\label{H_atom}
\mathcal{H}_I^{\mathrm{atom}}&=\mathcal{H}^\mathrm{electron}_I+\mathcal{H}^\mathrm{nucleon}_I=\frac{G_F}{\sqrt{2}}\,\left(j_\mu\,\mathcal{J}^\mu_{\mathrm{electron}}+j_\mu\,\mathcal{J}^\mu_\mathrm{nucleon}\right),
\end{align}
where $\mathcal{J}^\mu_{\mathrm{nucleon}}$ denote the hadronic current for nucleus, $j^\mu$ and $\mathcal{J}^\mu_{\mathrm{electron}}$ are the lepton currents for neutrino and electron respectively. According to the weak interaction theory, the lepton current for neutrino and  electron can be written as
\begin{align}
&j_\mu=\overline{\psi_{\nu}}\,\gamma_\mu\,\left(1-\gamma_5\right)\,\psi_\nu,\\
\label{Current_e}
&\mathcal{J}^\mu_{\mathrm{electron}}=\overline{\psi_{e}}\,\gamma_\mu\,\left(1-\gamma_5\right)\,\psi_e\,\,\,\,\,(\mathrm{W^\pm\,exchange}),\\
&\mathcal{J}^\mu_{\mathrm{electron}}=\overline{\psi_{e}}\,\gamma_\mu\,\left(c_V^e-c_A^e\gamma_5\right)\,\psi_e\,\,\,\,\,(\mathrm{Z^0\,exchange}),
\end{align}
where  $\psi_\nu$ and $\psi_e$ represent the spinor for the neutrino and electron, respectively. From Eq.~(\ref{NC_int}) the coupling coefficient for electrons are $c^e_V=-1/2+2\sin^2\theta_w$ and $c^e_A=-1/2$. The hadronic current for is given by the expression~[\cite{Giunti:2007ry}]
\begin{align}
\label{Current_h}
\mathcal{J}^\mu_\mathrm{nucleon}\equiv\overline{\psi_t}\,\gamma^\mu\left(c^t_V-c^t_A\gamma^5\right)\psi_t,
\end{align}
where subscript $t$ means the target constituents (protons and neutrons). From Eq.~(\ref{NC_int}) the coupling constants for proton(uud) and neutron(udd) are given by
\begin{align}
&c^p_V=\frac{1}{2}-2\sin^2\theta_w,\,\,\,\,c^p_A=\frac{1}{2},\,\,\,\,\,\mathrm{proton}\\
&c^n_V=-\frac{1}{2}\,\,\,\,c^n_A=-\frac{1}{2},\,\,\,\,\,\mathrm{neutron}.
\end{align}


To obtain the effective potential for atom, we need to average the effective Hamiltonian over the electron and nucleon background. For the neutrino-nucleon (proton,neutron) interaction, we only have the neutral current interaction via $Z^0$ boson. However, for the neutrino-electron interaction, we can have charged-current or neutral current interaction depending on the flavor or neutrino. In following, we consider interaction between $\nu_e$ and electrons first which includes both charged and neutral-currents interaction for general discussion.

Considering atomic electrons as a gas of unpolarized electrons with a statistical distribution function $f(E_e)$, the effective potential for neutrino-electron interaction can be obtained by averaging the effective Hamiltonian over the electron background~[\cite{Giunti:2007ry}]
\begin{align}
\langle{\mathcal{H}^\mathrm{electron}_{I}}\rangle&=\frac{G_F}{\sqrt{2}}\int\,\frac{d^3p_e}{(2\pi)^32E_e}\,f(E_e,T)\left[\overline{\psi_\nu}(x)\,\gamma_\mu\left(1-\gamma_5\right)\,\psi_\nu(x)\right]\notag\\&\times\frac{1}{2}\!\sum_{h_e=\pm1}\!\!\langle\,e^-(p_e,h_e)|\overline{\psi_e}\,\gamma^\mu\big((1+c^e_V)\!-\!(1+c^e_A)\gamma_5\big)\,\psi_e|e^-(p_e,h_e)\rangle,
\end{align}
where $h_e$ denotes the helicity of the electron. The average over helicity of the electron matrix element can be calculated with Dirac spinor and gamma matrix traces ~[\cite{Giunti:2007ry}]. Then the average effective Lagrangian can be written as
\begin{align}
\langle{\mathcal{H}^\mathrm{electron}_{I}}\rangle&=\frac{G_F}{\sqrt{2}}(1+c^e_V)\int\,\frac{d^3p_e}{(2\pi)^3}f(E_e)\left[\overline{\psi_\nu}(x)\,\frac{\gamma^\mu{p_e}_\mu}{E_e}\left(1-\gamma_5\right)\,\psi_\nu(x)\right]\notag\\
&=\frac{G_F}{\sqrt{2}}\,(1+c^e_V)\,\left[\int\,\frac{d^3p_e}{(2\pi)^3}f(E_e)\left(\gamma^0-\frac{\vec{\gamma}\cdot\vec{{p}}_e}{E_e}\right)\right]\overline{\psi_\nu}(x)\left(1-\gamma_5\right)\psi_\nu(x)\notag\\
&=\left[\frac{G_F}{\sqrt{2}}\left(1+c^e_V\right)n_{e}\right]\,\overline{\psi_\nu}(x)\gamma^0\left(1-\gamma_5\right)\psi_\nu(x),
\end{align}
where $n_e$ is the number density of the electron. In this case, the effective potential for neutrino-atomic electron interaction can be written as
\begin{align}
V^{\mathrm{electron}}_{I}=\frac{G_F}{\sqrt{2}}\left(1+c^e_V\right)n_{e}=\frac{G_F}{\sqrt{2}}\left(4\sin^2\theta_w+1\right)n_{e}.
\end{align}
The same method can be applied to the neutrino-nuclear interactions. Following the same approach and averaging the effective neutrino-nuclear Hamiltonian over the nuclear background, the effective potential experienced by a neutrino in a background of neutron/proton is given by~[\cite{Giunti:2007ry}] 
\begin{align}
&V_{I}^{\mathrm{proton}}=\frac{G_F}{\sqrt{2}}\left(1-4\sin^2\theta_w\right)n_{p},\qquad V_{I}^{\mathrm{neutron}}=-\frac{G_F}{\sqrt{2}}\,n_{n},
\end{align}
where $n_p$ and $n_n$ represent the number density of proton and neutron.
Combining the neutron and proton potential together, we define the effective nucleon potential experienced by neutrino as 
\begin{align}
V_I^{\mathrm{nucleon}}\equiv-\frac{G_F}{\sqrt{2}}\bigg[1-\left(1-4\sin^2\theta_w\right)\xi\bigg]n_{n},\qquad\xi=n_{p}/n_{n},
\end{align}
where $\xi$ is the ratio between proton and neutron number density.

In our study, we generalize the effective potential to the case of neutrino coherent scattering with multi-atom system, we consider a neutrino coherent forward scatters from a spherical symmetric system which is composed by atoms. In this case, the neutrino scatters off every atom, and it is impossible to identify which scatterer the neutrino interacts with and thus it is necessary to sum over all possible contributions from each atom. In such circumstances, it is appropriate to assume that the number density of electrons and neutrons can be written as
\begin{align}
&n_e=Z_e\,\left(\frac{N_\mathrm{atom}}{V}\right),\,\,\,\,\mathrm{and}\,\,\,\,n_n=N\,\,\left(\frac{N_\mathrm{atom}}{V}\right),
\end{align}
where $N_\mathrm{atom}$ is the number of atoms inside the system, $V$ is the volume of system, $Z$ is the number of electrons, and $N$ is the number of neutrons.
%%%%%%%%%%%%%%%%%%%%%%%%%%%%%%%%%%%%%%%%%%%%%%%%%%%%%%%%%%%%%%%%%%%%%
Then the effective potential is given by
\begin{align}
\label{Potential}
V_{I}&=V_I^{\mathrm{electron}}+V_I^{\mathrm{nucleon}}\notag\\&=\frac{G_F}{\sqrt{2}}\left(\frac{N_\mathrm{atom}}{V}\right)\bigg\{\left(4\sin^2\theta_w\pm1\right)\,Z_e-\bigg[1-\left(1-4\sin^2\theta_w\right)\xi\bigg]\,N\bigg\},
\end{align}
where the $+$ sign is for electron neutrinos $\nu_e$ and the $-$ sign is for muon(tau) neutrinos $\nu_{\mu,\tau}$, separately. 
From Eq.~(\ref{Potential}), it shows that the effective potential depends on the number density of electrons and nucleons contained within the wavelength. 
Thus by increasing the  atoms contained in the wavelength or selecting different atoms as targets, we can enhance the effective potential and may be able to provide a sensitive way to detect the cosmic neutrino background. Beside the detection of cosmic neutrino background, the effective potential for multi-atom can also provide new approaches for studying other aspects of neutrino physics in the future.

%%%%%%%%%%%%%%%%%%%%
%%%%%%%%%%%%%%%%%%%%%%%%%%%%%%%%%%%%%%%%%%%%%%%%%%%%%%%%%%%%%%%%%%%%%%%%
\subsection{Matrix elements of incoherent neutrino scattering}

To determine the freeze-out temperature (chemical/kinetic freeze-out) for a given flavor of neutrinos, we need to know all the elastic and inelastic interaction processes in the early Universe and compare their interaction rate with Hubble expansion rate. In this section we summarize the matrix elements for the neutrino annihilation/production processes and elastic scattering processes which are relevant for investigating neutrino freezeout. These matrix elements serve as one of the fundamental ingredients for solving the Boltzmann equation ~[\cite{Birrell:2014uka}].

%If we assume the neutrinos are so light that they decoupled from the primordial cosmic plasma at temperature $T>m_\nu$ and 
Considering the Universe with temperature $T\approx\mathcal{O}$(MeV), the   particle species in comisc plasma are given by:
\begin{align}
\mathrm{Particle\,\,species\,\, in \,\,plasma:}
\left\{\gamma,\, l^-,\,l^+,\, \nu_e,\, \nu_\mu,\, \nu_\tau,\, \bar{\nu}_e,\, \bar{\nu}_\mu,\, \bar{\nu}_\tau\right\},
\end{align}
 where $l^\pm$ represents the charged leptons. In this case, neutrinos can interact with all these particles via weak interactions and remain in equilibrium. In Table.~\ref{T005} and Table.~\ref{T006} we present the matrix elements $|M|^2$ for different weak interaction processes in the early Universe.

In the calculation of transition amplitude, we use the low energy approximation for $W^\pm$ and $Z^0$ massive propagators, i.e.
\begin{align}
&\mbox{$Z^0$ boson}:\frac{-i\left[g_{\mu\nu}-\frac{q_\mu q_\nu}{M^2_z}\right]}{q^2-M^2_z}\approx\frac{ig_{\mu\nu}}{M^2_z},\quad
&\mbox{$W^\pm$ boson}:\frac{-i\left[g_{\mu\nu}-\frac{q_\mu q_\nu}{M^2_w}\right]}{q^2-M^2_w}\approx\frac{ig_{\mu\nu}}{M^2_w},
\end{align}
and consider the tree-level Feynman diagram contributions only. Then, following the Feynman rules of weak interaction~[\cite{Griffiths:2008zz}], we obtain the matrix elements $|M|^2$ for different  interaction processes.
%%%%%%%%%%%%%%%%%%%%%%%%%%%%%%%%%%%%%%%%%%%%%%%%%%%%%%%%%%%%%
\begin{table}[h]
\centering
\begin{tabular}{lp{8cm}lp{8cm}l}
\hline\hline
Annihilation/Production \\
\hline\hline
Scattering Process & Transition Amplitude $|M|^2$ \\
\hline
$l^-+l^+\longrightarrow\nu_l+\bar{\nu}_l$ &$ 32G^2_F\bigg[\left(1+2\sin^2\theta_w\right)^2\left(p_1\cdot p_4\right)\left(p_2\cdot p_3\right)$

$+\left(2\sin^2\theta_w\right)^2\left(p_1\cdot p_3\right)\left(p_2\cdot p_4\right)$

$+2\sin^2\theta_w\left(1+2\sin^2\theta_w\right)m^2_l\left(p_3\cdot p_4\right)\bigg]$ \\
\hline
$l^{\prime-}+l^{\prime+}\longrightarrow\nu_l+\bar{\nu}_l$ & $32G^2_F\bigg[\left(1-2\sin^2\theta_w\right)^2\left(p_1\cdot p_4\right)\left(p_2\cdot p_3\right)$

$+\left(2\sin^2\theta_w\right)^2\left(p_1\cdot p_3\right)\left(p_2\cdot p_4\right)$

$-2\sin^2\theta_w\left(1-2\sin^2\theta_w\right)m^2_{l^\prime}\left(p_3\cdot p_4\right)\bigg]$ \\
\hline
$\nu_l+\bar{\nu}_l\longrightarrow\nu_l+\bar{\nu}_l$ &
$32G^2_F\bigg[\left(p_1\cdot p_4\right)\left(p_2\cdot p_3\right)\bigg]$ \\
\hline
$\nu_{l^\prime}+\bar{\nu}_{l^\prime}\longrightarrow\nu_l+\bar{\nu}_l$ &
$32G^2_F\bigg[\left(p_1\cdot p_4\right)\left(p_2\cdot p_3\right)\bigg]$ \\
\hline\hline
\end{tabular}
\caption{The transition amplitude for different annihilation and production processes. The definition of particle number is given by $1+2\leftrightarrow3+4$, where $l,\,l^\prime=e,\,\mu,\,\tau\,(l\neq\,l^\prime)$.}
\label{T005}
\end{table}
%%%%%%%%%%%%%%%%%%%%%%%%%%%%%%%%%%%%%%%%%%%%%%%%%%%
%%%%%%%%%%%%%%%%%%%%%%%%%%%%%%%%%%%%%%%%%%%%%%%%%%%%%%%%%%%%%
\begin{table}[h]
\centering
\begin{tabular}{lp{8cm}lp{8cm}l}
\hline\hline
Elastic Scattering Process ($\nu_e$) \\
\hline\hline
Scattering Process & Transition Amplitude $|M|^2$\\
\hline
$\nu_l+l^-\longrightarrow\nu_l+l^-$ & 
$ 32G^2_F\bigg[
   \left(1+2\sin^2\theta_w\right)^2\left(p_1\cdot p_2\right)\left(p_3\cdot p_4\right)$
   
   $+\left(2\sin^2\theta_w\right)^2\left(p_1\cdot p_4\right)\left(p_2\cdot p_3\right)$
   
   $-2\sin^2\theta_w\left(1+2\sin^2\theta_w\right)m^2_l\left(p_1\cdot p_3\right)\bigg]$ \\
\hline
$\nu_l+l^+\longrightarrow\nu_l+l^+$ &
$ 32G^2_F\bigg[
   \left(1+2\sin^2\theta_w\right)^2\left(p_1\cdot p_4\right)\left(p_2\cdot p_3\right)$
   
   $+\left(2\sin^2\theta_w\right)^2\left(p_1\cdot p_2\right)\left(p_3\cdot p_4\right)$
   
   $-2\sin^2\theta_w\left(1+2\sin^2\theta_w\right)m^2_l\left(p_1\cdot p_3\right)\bigg]$ \\
\hline
$\nu_l+l^{\prime-}\longrightarrow\nu_l+l^{\prime-}$ &
$ 32G^2_F\bigg[
   \left(1-2\sin^2\theta_w\right)^2\left(p_1\cdot p_2\right)\left(p_3\cdot p_4\right)$
   
  $+\left(2\sin^2\theta_w\right)^2\left(p_1\cdot p_4\right)\left(p_2\cdot p_3\right)$
  
  $+2\sin^2\theta_w\left(1-2\sin^2\theta_w\right)m^2_{l^\prime}\left(p_1\cdot p_3\right)\bigg]$ \\
\hline
$\nu_l+l^{\prime+}\longrightarrow\nu_l+l^{\prime+}$ &
$ 32G^2_F\bigg[
   \left(1-2\sin^2\theta_w\right)^2\left(p_1\cdot p_4\right)\left(p_2\cdot p_3\right)$
   
   $+\left(2\sin^2\theta_w\right)^2\left(p_1\cdot p_2\right)\left(p_3\cdot p_4\right)$
   
   $+2\sin^2\theta_w\left(1-2\sin^2\theta_w\right)m^2_{l^\prime}\left(p_1\cdot p_3\right)\bigg]$ \\
\hline
$\nu_l+\nu_l\longrightarrow\nu_l+\nu_l$ &
$\frac{1}{2!}\frac{1}{2!}\times32G^2_F\bigg[4\left(p_1\cdot p_2\right)\left(p_3\cdot p_4\right)\bigg]$ \\
\hline
$\nu_l+\bar{\nu}_l\longrightarrow\nu_l+\bar{\nu}_l$ &
$32G^2_F\bigg[4\left(p_1\cdot p_4\right)\left(p_2\cdot p_3\right)\bigg]$ \\
\hline
$\nu_l+\nu_{l^\prime}\longrightarrow\nu_l+\nu_{l^\prime}$ &
$32G^2_F\bigg[\left(p_1\cdot p_2\right)\left(p_3\cdot p_4\right)\bigg]$ \\
\hline
$\nu_l+\bar{\nu}_{l^\prime}\longrightarrow\nu_l+\bar{\nu}_{l^\prime}$ &
$32G^2_F\bigg[\left(p_1\cdot p_4\right)\left(p_2\cdot p_3\right)\bigg]$ \\
\hline\hline
\end{tabular}
\caption{The transition amplitude for different elastic scattering processes. The definition of particle number is given by $1+2\leftrightarrow3+4$, where $l,\,l^\prime=e,\,\mu,\,\tau\,(l\neq\,l^\prime)$.}
\label{T006}
\end{table}

\clearpage

%~~~~~~~~~~~~~~~~~~~~~~~~~~~~~~~~~~~~~~~~~~~~~~~~~

\section{Neutrinos in the early Universe }
%In this section we will focus on the following:
%\begin{itemize}
%    \item Neutrino decoupling in early Universe: standard scenarios
%    \item Overview of neutrino freeze-out and free streaming
%    \item After freeze-out extra neutrinos from microscope process
%    \item Lepton number and effective number of neutrinos
%\end{itemize}
In the early Universe the neutrinos are kept in equilibrium with cosmic plasma via the weak interaction processes. However, as the Universe expanded, these weak interactions gradually became too slow to maintain equilibrium, then the neutrinos ceased interacting and decoupled from the cosmic background. These relic neutrino background carries great information about our early Universe which can provide the information when our Universe is about $1$ sec old. 

Today the Universe is believed to be filled with relic cosmic neutrinos. Although the cosmic neutrino background is hard to detect, we can use entropy conservation to infer their temperature and find that they should have the temperature $T_\nu^0=1.9\,\mathrm{K}$ in the present epoch for  massless neutrinos. However, from the neutrino oscillation experiment, we know that the the neutrinos are not massless particles. 

The square mass difference $\Delta m^2_{ij}$ has been experimentally measured, from the neutrino oscillation experiment~[\cite{ParticleDataGroup:2022pth}]:
\begin{align}
&\Delta{m}_{21}^2=7.39^{+0.21}_{-0.20}\times10^{-5}\,\mathrm{eV}^2,\\
&\Delta{m}_{32}^2=2.45^{+0.03}_{-0.03}\times10^{-3}\,\mathrm{eV}^2.
\end{align}
and neutrino mass eigenvalue can be ordered in the normal mass hierarchy ($m_1\ll m_2<m_3$) or inverted mass hierarchy ($m_3\ll m_1<m_2$). All three mass states remained relativistic until the temperature dropped below their rest mass. These results allow for the possibility that one mass eigenstate or two mass eigenstates of neutrinos may become non-relativistic today. 

\subsection{Overview of neutrino freeze-out in the early Universe}


The properties of the neutrino background are influenced by the details of the freeze-out or decoupling process at a temperature $T=\mathcal{O}(2\mathrm{MeV})$. In the literature one finds estimates of freeze-out temperatures based on a comparison of Hubble expansion with neutrino scattering length and considering only number changing (i.e. chemical) processes. In the paper~[\cite{Birrell:2014uka}], we employ a similar definition of freeze-out temperature in the context of the Boltzmann equation and refine the results by noting that there are three different freeze-out processes for neutrinos:

%In general, the particle freezeout process include both chemical and kinetic which lead to particle become free-streaming in the early Universe (for detail discussion see Chapter~\ref{Introduction}), we have:

1. Neutrino chemical freeze-out: the temperature at which neutrino number changing processes such as $e^-e^+\to\nu\overline\nu$ effectively cease. After chemical freeze-out, there are no reactions that, in a noteworthy fashion, can
change the neutrino abundance and so particle number is conserved. %Prior to the chemical freezeout temperature, number changing processes are significant and keep the particle in chemical (and thermal) equilibrium, implying the distribution function of neutrino has the Fermi-Dirac form:
%\begin{equation}\label{equilibrium}
%f_{c}(t,E)=\frac{1}{\exp(E/T)+1}, \qquad\text{ for } T> T_{ch}.
%\end{equation}

2. Neutrino kinetic freeze-out: the temperature at which the neutrino momentum exchanging interactions such as $e^\pm\nu\to e^\pm\nu$ are no longer occurring rapidly enough to maintain an equilibrium momentum distribution. %When $T_k<T(t)<T_{ch}$, the number-changing process no longer occurs rapidly enough to keep the distribution in chemical equilibrium but there is still sufficient momentum exchange to keep the distribution in thermal equilibrium. The distribution function is therefore obtained by maximizing entropy, with fixed energy, particle number, and antiparticle number separately. This implies that the distribution function has the form
%\begin{equation}\label{kinetic_equilib}
%f_k(t,E)=\frac{1}{\Upsilon^{-1}\exp(E/T)+1},\qquad \text{ for }T_k< T< T_{ch}.
%\end{equation}
%The time dependent generalized fugacity $\Upsilon(t)$ controls the occupancy of phase space and is necessary once $T(t)<T_{ch}$ in order to conserve particle number.

3. Collisions between neutrinos:
Those collisions $\nu\nu\to\nu\nu$ are capable of reequilibrating energy within and between neutrino flavor families. These processes end at a yet lower temperature and the neutrinos will be truly free-streaming from that point on.
 
%3. Free streaming: for $T<T_k$ there are no longer any significant interactions that couple the particle species of interest and so they begin to free-stream through the Universe. The Einstein-Vlasov equation can be solved~[\cite{choquet2008general,Birrell:2012gg}], to yield the free-streaming momentum distribution
%\begin{equation}\label{free_stream_dist}
%f(t,E)=\frac{1}{\Upsilon^{-1}e^{\sqrt{p^2/T_{fs}^2+m_\nu^2 /T_k^2}}+ 1},\qquad T_{fs}=\frac{T_ka(t_k)}{a(t)}
%\end{equation}
%where the free-streaming effective temperature $T_{fs}$ is obtained by redshifting the temperature at kinetic freeze-out, and $m_\nu$ is the mass of neutrino.





To estimate the freeze-out temperature, we need to solve the Boltzmann equation with different types of collision terms with the transition matrices from Table.~\ref{T005} and Table.~\ref{T006}. The paper~[\cite{Birrell:2014uka} ] developed a new method for analytically simplifying the collision integrals and showing that the neutrino freeze-out temperature is controlled by standard model (SM) parameters. The freeze-out temperature depends only on the magnitude of the Weinberg angle in the form $\sin^2\theta_W$ , and a dimensionless relative interaction strength parameter $\eta$,
\begin{align}
\eta\equiv M_p m_e^3 G_F^2, \qquad M_p^2\equiv \frac{1}{8\pi G_N}, \end{align}
a combination of the electron mass $m_e$, Newton constant $G_N$ (expressed above in terms of Planck mass $M_p$), and the Fermi constant $G_F$. The dimensionless interaction strength parameter $\eta$ in the present-day vacuum has the value
\begin{align}
\eta_0\equiv \left.M_p m_e^3 G_F^2\right|_0 = 0.04421 .
\end{align}

The magnitude of $\sin^2\theta_W$ is not fixed within the SM and could be subject to variation as a function of time or temperature. In Fig.~\ref{fig:freezeoutT} we show the dependence of neutrino freeze-out temperatures for $\nu_e$ and $\nu_{\mu,\tau}$ on SM model parameters $\sin^2\theta_W$ and $\eta$ in detail. The impact of SM parameter values on neutrino freeze-out and the discussion of the implications and connections of this work to other areas of physics, namely Big Bang Nucleosynthesis and dark radiation can be found in detail in~[\cite{Dreiner:2011fp,Boehm:2012gr,Blennow:2012de,Birrell:2014uka}]. A comprehensive investigation of neutrino freezeout, and a novel approach to analytically simplify the collision integrals for the Boltzmann equation can be found in Dr. Jeremiah Birrell‘s PhD thesis~[\cite{Birrell:2014ona}].


%We expect that incorporating oscillations into the freeze-out calculation would yield a smaller freeze-out temperature difference between neutrino flavors as oscillation provides a mechanism in which the heavier flavors remain thermally active despite their direct production becoming suppressed. In work by Mangano et. al.~[\cite{Mangano:2005cc}], neutrino freeze-out including flavor oscillations is shown to be a negligible effect.
%\clearpage
%~~~~~~~~~~~~~~~~~~~~~~~~~~~~~~~~~~~~~~~~~~~~~~~~~
\begin{figure}[ht]
\centerline{\includegraphics[width=0.47\columnwidth]{./plots/nu_e_freezeout.pdf}
\hspace{1mm}\includegraphics[width=0.47\columnwidth]{./plots/nu_mu_freezeout.pdf}}
\centerline{\includegraphics[width=0.47\columnwidth]{./plots/nu_e_freezeout_GF.pdf}
\hspace{1mm}\includegraphics[width=0.47\columnwidth]{./plots/nu_mu_freezeout_GF.pdf}}
\caption{Freeze-out temperatures for electron neutrinos (left) and $\mu$, $\tau$ neutrinos (right) for the three types of freeze-out processes adapted from paper [\cite{Birrell:2014uka}]. Top panels print temperature curves as a function of $\sin^2\theta_W$ for $\eta=\eta_0$, the vertical dashed line is $\sin^2\theta_W=0.23$; bottom panels are printed as a function of relative change in interaction strength $\eta/\eta_0$ obtained for $\sin^2\theta_W=0.23$.}
\label{fig:freezeoutT}
 \end{figure}
%~~~~~~~~~~~~~~~~~~~~~~~~~~~~~~~~~~~~~~~~~~~~~
\clearpage



%%%%%%%%%%%%%%%%%%%
\subsection{Lepton number and effective number of neutrinos}

Neutrinos decoupled from the cosmic plasma in the early Universe at a temperature of $T=\mathcal{O}(2\mathrm{MeV})$ and became free-streaming. However, after freezeout neutrinos still continue to play a significant role in the evolution of the Universe and have a huge impact on cosmological observations such as Big Bang Nucleosynthesis (BBN), the Cosmic Microwave Background (CMB), and the matter spectrum for large scale structure. This is due to the sensitivity of the Hubble parameter to the total energy density in the Universe. Besides photons, neutrinos are the most abundant species and contribute significantly to the relativistic energy density throughout the early Universe, affecting the Hubble expansion rate significantly. 

The contribution of energy density from the neutrino sector can be described by the effective number of neutrinos $N_{\nu}^{\mathrm{eff}}$, which captures the number of relativistic degrees of freedom for neutrinos as well as any reheating that occurred in the sector after freeze-out. The effective number of neutrino is defined as 
\begin{align}\label{Neff}
N_\nu^{\mathrm{eff}}\equiv\frac{\rho^{\mathrm{tot}}_\nu}{\frac{7\pi^2}{120}\left(\frac{4}{11}\right)^{4/3}T_\gamma^4}\;,
\end{align}
where $\rho_\nu^{\mathrm{tot}}$ is the total energy density in neutrinos and $T_\gamma$ is the photon temperature. $N_\nu^{\mathrm{eff}}$ is defined such that three neutrino flavors with zero participation of neutrinos in reheating during $e^\pm$ annihilation results in $N_\nu^{\mathrm{eff}}=3$. The factor of $\left(4/11\right)^{1/3}$ relates the photon temperature to the free-streaming neutrinos temperature, under the assumption of zero neutrino reheating after $e^\pm$ annihilation. The currently accepted theoretical value is $N_\nu^{\mathrm{eff}}=3.046$, after including the slight effect of neutrino reheating [\cite{Mangano:2005cc,Birrell:2014uka}]. The favored value of $N_\nu^{\mathrm{eff}}$ can be found by fitting to CMB data. In 2013 the Planck collaboration found $N_\nu^{\mathrm{eff}}=3.36\pm0.34$ (CMB only) and $N_\nu^{\mathrm{eff}}= 3.62\pm0.25$ (CMB and $H_0$)~[\cite{Planck:2013pxb}].

To explain the experimental value of $N_\nu^{\mathrm{eff}}$, many studies aim to improve the calculation of neutrino decoupling in the early Universe, including exploring the dependence of freeze-out on natural constants~[\cite{Birrell:2014uka}], the entropy transfer from $e^\pm$ annihilation and finite temperature correction~[\cite{Dicus:1982bz,Heckler:1994tv,Fornengo:1997wa}], neutrino decoupling with flavor oscillations~[\cite{Mangano:2001iu,Mangano:2005cc}], and investigating nonstandard neutrino interactions [\cite{Morgan:1981zy,Fukugita:1987uy,Elmfors:1997tt,Vogel:1989iv,Mangano:2006ar,Giunti:2008ve,Mangano:2006ar}].% However, the effective number of neutrino $N_\nu^{\mathrm{eff}}$ can be a consequence of fundamental physics principle.


The standard cosmological model assumes that the lepton asymmetry $L\equiv  [N_\mathrm{L}-N_{\overline{\mathrm{L}}}] /N_\gamma $  (normalized with the photon number) 
between leptons and anti-leptons is small, similar to the baryon asymmetry $B=[N_\mathrm{B}-N_{\overline{\mathrm{B}}}]/N_\gamma $; most often it is assumed $L=B$. Barenboim, Kinney, and Park~[\cite{Barenboim:2016shh,Barenboim:2017dfq}] noted that the lepton asymmetry of the Universe is one of the most weakly constrained parameters is cosmology and they propose that models with leptogenesis are able to accommodate a large lepton number asymmetry surviving up to today.  Moreover, the discrepancy between $H_\mathrm{CMB}$ and $H_0$ has increased~[\cite{riess2018new,Riess:2018byc,Planck:2018vyg}]. The Hubble tension and the possibility that leptogenesis in the early Universe resulted in neutrino asymmetry motivate our study of the dependence of $N_\nu^{\mathrm{eff}}$ on lepton asymmetry, $L$. In our work~[\cite{Yang:2018oqg}] we consider $L\simeq 1$ and explore how this large cosmological lepton yield relates to the effective number of (Dirac) neutrinos $N^{\mathrm{eff}}_\nu$. 

\subsubsection{Relation between $N_\nu^{\mathrm{eff}}$ and neutrino chemical potential}
We consider now neutrinos decouple~[\cite{Birrell:2014gea}] at a temperature of $T_f\simeq 2\,\mathrm{MeV}$ and are subsequently free-streaming. Assuming exact thermal equilibrium at the time of decoupling, the neutrino distribution can be subsequently written as (see~[\cite{Birrell:2012gg}] and references therein)
\begin{align}
\label{fnudef}
&f_\nu=\frac{1}{\exp{\left(\sqrt{\frac{E^2-m_\nu^2}{T_\nu^2}+\frac{m^2_\nu}{T^2_f}}-\sigma\frac{\mu_\nu}{T_f}\right)+1}}\;,\qquad T_\nu\equiv\frac{a(t_f)}{a(t)}T_f,
\end{align}
where $\sigma=+1(-1)$ denotes particles (antiparticles) and we define the effective neutrino temperature $T_\nu$  by the red-shifting of momentum in the comoving volume element of the Universe.

Since the freeze-out temperature $T_f\gg m_\nu$ and also neutrino temperature $T_\nu\gg m_\nu$ in the domain of our analysis, we consider the massless limit in Eq.\;(\ref{fnudef}). Under this approximation, the total neutrino energy density can be written as
\begin{align}
\label{Energy_Density}
\rho_\nu^{\mathrm{tot}}
&=\frac{g_\nu\,T_\nu^4}{2\pi^2}\left[\frac{7\pi^4}{60}+\frac{\pi^2}{2}\left(\frac{\mu_\nu}{T_f}\right)^{\!\!2}+\frac{1}{4}\left(\frac{\mu_\nu}{T_f}\right)^{\!\!4}\right].
\end{align}
Substituting Eq.\;(\ref{Energy_Density}) into the definition of the effective number of neutrinos Eq.~(\ref{Neff}), we obtain 
\begin{align}
\label{Neff_002}
N_\nu^{\mathrm{eff}}\!\!
=\!3\!\left(\frac{11}{4}\right)^{\!\!\frac{4}{3}}\!\!\left(\frac{T_\nu}{T_\gamma}\right)^{\!\!4}\!
\left[1\!+\!\frac{30}{7\pi^2}\!\!\left(\frac{\mu_\nu}{T_f}\right)^{\!\!2} 
\!\!+\frac{15}{7\pi^4}\!\!\left(\frac{\mu_\nu}{T_f}\right)^{\!\!4}\right].
\end{align}
From Eq.\;(\ref{Neff_002}) we have for the standard photon reheating ratio $T_\nu/T_\gamma=(4/11)^{1/3}$ [\cite{Kolb:1990vq}] and degeneracy $g_\nu=3$ (flavor), the relation between the effective number of neutrinos and the chemical potential at freezeout
\begin{align}
\label{Neff_Potential}
N_\nu^{\mathrm{eff}}=3\left[1+\frac{30}{7\pi^2}\left(\frac{\mu_\nu}{T_f}\right)^{\!\!2}+ \frac{15}{7\pi^4} \left(\frac{\mu_\nu}{T_f}\right)^{\!\!4}\right].
\end{align}
To solve the neutrino chemical potential $\mu_\nu/T_f$ as a function of the effective number of neutrinos, we can neglect the $(\mu_\nu/T_f)^4$ term in Eq.\;(\ref{Neff_Potential}) because $m_\nu\ll T_f$ and obtain
\begin{align}\label{Solution}
\frac{\mu_\nu}{T_f}=\pm\sqrt{\frac{7\pi^2}{30}\left(\frac{N_\nu^{\mathrm{eff}}}{3}-1\right)}.
%=&\pm \pi\sqrt{\sqrt{1+\frac{7}{15}\left(\frac{N_\nu^{\mathrm{eff}}}{3}-1\right)}-1}%\approx
\end{align}
In Fig.\;\ref{Chemical_Potential_Neff} we plot the free-streaming neutrino chemical potential $|\mu_\nu|/T_f$ as a function of the effective number of neutrinos $N_\nu^{\mathrm{eff}}$. For comparison, the solid (blue) line is the exact solution of $|\mu_\nu|/T_f$ by solving Eq.~(\ref{{Neff_Potential}}) numerically, and the (red) dashed line is the approximate solution Eq.~(\ref{Solution}) by neglecting the $(\mu_\nu/T_f)^4$ in calculation. In the parameter range of interest, we show that the term $(\mu_\nu/T_f)^4$ only contributes $\approx 2\%$ to the calculation and henceforth we neglect it, and use the approximation Eq.\;(\ref{Solution}). 

The SM value of the effective number of neutrinos, $N_\nu^{\mathrm{eff}}=3$, is obtained under the assumption that the neutrino chemical potentials are not essential, {\it i.e.\/}, $\mu_\nu\ll T_f$. From Fig.\;\ref{Chemical_Potential_Neff}, to interpret the literature values $N_\nu^{\mathrm{eff}}=3.36\pm0.34$ (CMB only) and $N_\nu^{\mathrm{eff}}= 3.62\pm0.25$ (CMB and $H_0$), we require $0.52\leqslant\mu_\nu/T_f\leqslant0.69$. These values suggest  a possible neutrino-antineutrino asymmetry at freezeout, {\it i.e.\/} a difference between the number densities of neutrinos and antineutrinos.
%%%%%%%%%%%%%%%%%%%%%%%%%%%%%%%%%%%%%%%%%%%%%%%%%%%%%%%%%%%%%%%%%%
\begin{figure}[t]
\begin{center}
\includegraphics[width=\textwidth]{./plots/Chemical_Potential_Neff}
\caption{The free-streaming neutrino chemical potential $|\mu_\nu|/T_f$ as a function of the effective number of neutrinos $N_\nu^{\mathrm{eff}}$. The solid (blue) line is the exact solution and the (red) dashed line is the approximate solution neglecting the $(\mu_\nu/T_f)^4$ term; the maximum difference in the domain shown is about $2\%$.}
\label{Chemical_Potential_Neff}
\end{center}
\end{figure}
%%%%%%%%%%%%%%%%%%%%%%%%%%%%%%%%%%%%%%%%%%%%%%%%%%%%%%%%%%%%%%%%%%%



\subsubsection{Dependence of $N_\nu^{\mathrm{eff}}$ on lepton asymmetry}
We now obtain the relation between neutrino chemical potential and the baryon to lepton ratio. Let us consider the neutrino freezeout temperature $T_f\simeq 2.0$ MeV; here we treat neutrino freezeout as occurring instantaneously and prior to $e^\pm$ annihilation (implying zero neutrino reheating). Comoving lepton (and baryon) number is conserved after the epoch of leptogenesis (baryogenesis, respectively) which precedes the epoch  under consideration in this work ($T\lesssim 2$\;MeV). %However, the photon number $N_\gamma$ changes due to reheating, and accommodates the large number of photons originating from $e^+e^-$-annihilation.% We present results in terms of the current epoch photon number. 

The lepton-density asymmetry $\ell $ at neutrino freeze-out can be written as
\begin{align}
\ell_f \equiv\big(n_e-n_{\overline{e}}\big)_f+\sum_{i=e,\mu, \tau}\big(n_{\nu_i}-n_{\overline{\nu}_i}\big)_f,
\end{align}
where we use the subscript $f$ to indicate that the quantities should be evaluated at the neutrino freeze-out temperature. As a first approximation, here we assume that all neutrinos freeze-out at the same temperature and their chemical potentials are the same; {\it i.e.\/},
\begin{align}
\mu_\nu=\mu_{\nu_e}=\mu_{\nu_\mu}=\mu_{\nu_\tau}.
\end{align}
Furthermore, neutrino oscillation implies that neutrino number is freely exchanged between flavors; {\it i.e.\/}, $\nu_e\rightleftharpoons\nu_\mu\rightleftharpoons\nu_\tau$, and we can assume that all neutrino flavors share the same population. Under these assumptions, the lepton-density asymmetry can be written as
\begin{align}
\label{L_asymmetry} 
\ell_f=\big(n_e-n_{\overline{e}}\big)_f+\big(n_{\nu}-n_{\overline{\nu}}\big)_f,
\end{align}
where the three flavors are accounted for by taking the degeneracy $g_\nu=3$ in the last term. The difference in yield of neutrinos and antineutrinos can be written as
\begin{align}
\label{Excess_Neutrino}
\left(n_\nu-n_{\overline{\nu}}\right)_f=\frac{g_\nu}{6\pi^2}T^3_f\bigg[\pi^2\left(\frac{\mu_\nu}{T_f}\right)+\left(\frac{\mu_\nu}{T_f}\right)^{\!\!3}\bigg].
\end{align}


On the other hand, the baryon-density asymmetry $b$ at neutrino freezeout is given by
\begin{align}
\label{B_asymmetry}
b_f \equiv\big(n_p-n_{\overline{p}}\big)_f+\big(n_n-n_{\overline{n}}\big)_f \approx \big(n_p+n_n\big)_f,
\end{align}
where $n_{\overline{n}}$ and $n_{\overline{p}}$ are negligible in the temperature range we consider here. Taking the ratio $\ell_f/b_f$, using charge neutrality, and introducing the entropy density we obtain
\begin{align}\label{Lf_Bf}
\left(\frac{\ell_f}{b_f}\right)  
\approx\left(\frac{n_p}{n_B} \right)_f+\left(n_{\nu}-n_{\overline{\nu}}\right)_f \left(\frac{s}{n_B}\right)_f \frac{1}{s_f},\qquad n_B=(n_p+n_n),
\end{align}
where we introduce the notation $n_B$ for the baryon number density. The proton concentration at neutrino freeze-out is given by
\begin{align}
\label{X_proton}
\left(\frac{n_p}{n_B}\right)_f&=\frac{1}{1+(n_n/n_p)_f}=\frac{1}{1+\exp{\big[-\left(Q+\mu_\nu\right)/T_f\big]}},
\end{align}
with $Q=m_n-m_p=1.293\,\mathrm{MeV}$. We neglect the electron chemical potential in the last step because the $e^\pm$ asymmetry is determined by the proton density, and at energies of order a few MeV, the proton density is small, {\it i.e.\/}, $\mu_e\ll T_f$. 

However, as we will see, for our study of $N_\nu^{\mathrm{eff}}$ we will be interested in the case of a large lepton-to-baryon ratio. From Eq.\;(\ref{X_proton}) it is apparent that this can only be achieved through the second term in Eq.\;(\ref{Lf_Bf}), with the first term then being negligible, as it is smaller than $1$. So we further approximate
\begin{align}\label{L_B_ratio}
\left(\frac{\ell_f}{b_f}\right)  
\approx\left(n_{\nu}-n_{\overline{\nu}}\right)_f \left(\frac{s}{n_B}\right)_f \frac{1}{s_f}.
\end{align}
We retained the full expression Eq.\;(\ref{X_proton}) in our above discussion to show that the presence of a chemical potential $\mu_\nu\simeq 0.2\,Q$ could lead to small, perhaps noticeable, effects on pre-BBN proton and neutron abundance. We defer this unrelated discussion to a separate future work. Note that for large $|\mu_\nu|$, Eq.\;(\ref{L_B_ratio}) implies that the signs of $\mu_\nu$ and $\ell_f$ are the same. However, for very small $\mu_\nu$ the sign of $\ell_f$ is determined by the interplay between (anti)electrons and (anti)neutrinos; {\it i.e.\/}, there is competition between the two terms in Eq.\;(\ref{L_asymmetry}).

In general, the total entropy density at freeze-out can be written
\begin{align}
\label{Entropy_density}
s_f=\frac{2\pi^2}{45}g^s_\ast(T_f)\,T_f^3,
\end{align}
where the $g^s_\ast$ counts the degree of freedom for relativistic particles~[\cite{Kolb:1990vq}]. At $T_f\simeq 2\mathrm{MeV}$, the relativistic species in the early Universe are photons, electron/positrons, and $3$ neutrino species. We have
\begin{align}
g^s_{\ast}&= g_\gamma+\frac{7}{8}\,g_{e^\pm}+\frac{7}{8}\,g_{\nu\bar{\nu}}\left(\frac{T_\nu}{T_\gamma}\right)^{\!\!3}\bigg[1+\frac{15}{7\pi^2}\left(\frac{\mu_\nu}{T_f}\right)^{\!\!2}\bigg]=10.75+\frac{45}{4\pi^2}\left(\frac{\mu_\nu}{T_f}\right)^{\!\!2}\;,
\end{align}
where the degrees of freedom are given by $g_\gamma=2$, $g_{e^\pm}=4$, and $g_{\nu\bar{\nu}}=6$, and we have $T_\nu=T_\gamma=T_f$ at neutrino freeze-out.

%%%%%%%%%%%%%%%%%%%%%%%%%%%%%%%%%
\begin{figure}[h]
\begin{center}
\includegraphics[width=\textwidth]{./plots/Ratio_BL}
\caption{The ratio $B/|L|$ between the net baryon number and the net lepton number as a function of $N^{\mathrm{eff}}_\nu$: The solid blue line shows $B/|L|$. The vertical (red) dotted lines represent the values $3.36\leqslant N_\nu^{\mathrm{eff}}\leqslant3.62$, which correspond to $1.16 \times 10^{-9}\leqslant B/|L|\leqslant 1.51 \times 10^{-9}$ (horizontal dashed lines).}
\label{BL_Ratio}
\end{center}
\end{figure}
%%%%%%%%%%%%%%%%%%%%%%%%%%%%%%%%%%%%%%%%%%%%%%%%%%%%%%%%%%%%%%%%%%

Finally, since the entropy-per-baryon from neutrino freeze-out up to the present epoch is constant, we can obtain this value by considering the Universe's entropy content today~[\cite{Fromerth:2012fe}]. For $T\ll1\,\mathrm{MeV}$, the entropy content today is carried by photons and neutrinos, yielding
\begin{align}
\label{Nb_S}
\left(\frac{s}{n_B}\right)_{t_0}&=\frac{\sum_i\,s_i}{n_B}=\frac{n_\gamma}{n_B}\,\bigg(\frac{s_\gamma}{n_\gamma}+\frac{s_\nu}{n_\gamma}+\frac{s_{\bar{\nu}}}{n_\gamma}\bigg)\;\\
&=\left(\frac{1}{B}\right)_{\!\!t_0}\!\!\left[\frac{s_\gamma}{n_\gamma}+\frac{4}{3T_\nu}\frac{\rho_\nu^{\mathrm{tot}}}{n_\gamma}-\frac{\mu_\nu}{T_f}\left(\frac{n_\nu-n_{\bar{\nu}}}{n_\gamma}\right)\right]_{t_0}\;,
\end{align}
where $t_0$ denotes the present day values, we have $B=n_B/n_\gamma= 0.605\times10^{-9}$ (CMB)~[\cite{ParticleDataGroup:2016lqr}] from today's observation. The entropy per particle for a massless boson at zero chemical potential is $(s/n)_{\mathrm{boson}}\approx 3.602$.

Substituting Eq.\;(\ref{Excess_Neutrino}) and Eq.\;(\ref{Entropy_density}) into Eq.\;(\ref{L_B_ratio}) yields the lepton-to-baryon ratio
\begin{align}\label{L_B_ratio_final}
&\frac{L}{B}=\frac{45}{4\pi^4}\frac{\pi^2(\mu_\nu/T_f)+(\mu_\nu/T_f)^3}{10.75+{45}(\mu_\nu/T_f)^2/{4\pi^2}}\left(\frac{s}{n_B}\right)_{\!\!t_0}\;,
\end{align}
in terms of $\mu_\nu/T_f$ which is given by Eq.(\ref{Solution}) and the present day entropy-per-baryon ratio. In Fig.\;\ref{BL_Ratio} we show the ratio between the net baryon number and the net lepton number as a function of the effective number of neutrino species $N^{\mathrm{eff}}_\nu$ with the parameter $ B|_{t_0} =0.605\times 10^{-9}$(CMB). We find that the values $N_\nu^{\mathrm{eff}}=3.36\pm0.34$ and $N_\nu^{\mathrm{eff}}= 3.62\pm0.25$ require the ratio between baryon number and lepton number to be $1.16 \times 10^{-9} \leqslant\, B/|L| \leqslant 1.51\times 10^{-9}$. These values are close to the baryon-to-photon ratio $0.57 \times 10^{-9} \leqslant B  \leqslant 0.67\times 10^{-9}$. 


In summary, motivated by the necessity to explain a slightly faster Universe expansion, we believe that there is need for additional unobserved particles,  leading to an increase  in the Universe expansion rate. Considerable effort has been made in this direction, e.g., by introducing exotic and new \lq dark\rq\ particles, see~[\cite{Birrell:2014cja}] and references therein. In this work a similar effect is achieved by introducing lepton asymmetry in the Universe. We connected the lepton asymmetry in the Universe with the chemical neutrino potential $\mu_\nu$,
and further evaluated the consequences for the Universe expansion. We have explored the other natural scenario regarding the baryon number-to-lepton number ratio. Instead of $B\simeq |L|$, we found that $0.4\leqslant|L| \leqslant0.52$ and $B\simeq 1.33\times 10^{-9}|L|$ reconciles the CMB and current epoch results for the Hubble expansion parameter.
%The standard cosmological model assumes (arbitrarily) that the asymmetry between leptons and anti-leptons is small, similar to the baryon asymmetry; most often it is assumed $L=B$. We consider $L\simeq 1$ and explore how this large cosmological lepton yield relates to the effective number of (Dirac) neutrinos $N^{\mathrm{eff}}_\nu$.  we have explored the other natural scenario regarding the baryon number-to-lepton number ratio. Instead of $B\simeq |L|$, we found that $0.4\leqslant|L| \leqslant0.52$ and $B\simeq 1.33\times 10^{-9}|L|$ reconciles the CMB and current epoch results for the Hubble expansion parameter.

The large lepton asymmetry from cosmic neutrino can also affect the neutron lifespan in cosmic plasma which is one of the important parameter controlling BBN element abundances. 
In general the neutron lifespan dependence on temperature of the cosmic medium. When temperature $T=\mathcal{O}(\mathrm{MeV})$, neutron decay occurs in the plasma of electron/positron and 
 neutrino/antineutrino. Electrons and neutrinos in the background plasma can reduce the neutron decay rate by Fermi suppression to the neutron decay rate. Furthermore, the neutrino background can still provide the suppression after electron/positron pair annihilation becomes nearly complete. In this case,the large neutrino chemical potential from lepton asymmetry would play an important role and needs to be accounted for in the precision study of the neutron lifespan in the cosmic plasma.
\chapter{Charged leptons in cosmic plasma}\label{Electron}

Charged leptons played significant roles in the dynamics and evolution of the early Universe. They were kept in equilibrium via electromagnetic and weak interactions.  In this chapter, I examine a dynamical model of the abundance of charged leptons $\mu^\pm$ and $e^\pm$ in the early Universe obtaining their disappearance temperature, the condition when they disappear from the particle inventory. Of particular interest is the dense electron-positron plasma present during the early Universe evolution. I study the damping rate and the magnetization process in this dense $e^\pm$ plasma in the early Universe.


%{Introduction\daggerfootnote{This chapter has been published previously as \citet{Gottbrath1999}.}}
%~~~~~~~~~~~~~~~~~~~~~~~~~~~~~~~~~~~~~~~~~~~~~~~~~

\section{Overview of charge leptons in early Universe}
%In this section we will focus on the following:
%\begin{itemize}
%    \item Charged leptons in early universe
%    \item Remarks on tau leptons
%\end{itemize}


%In the early universe, charged leptons are kept equilibrium with the cosmic plasma via electromagnetic and weak interactions which played significant roles in the dynamics and evolution of the Universe.  For example, the present $e^\pm$ in early universe can affect  the neutrino decoupling, photon heating, and big bang nucleosynthesis. Although, the massive lepton $\tau^\pm, \mu^\pm$ decay into light leptons ($\nu$, $l^\pm$) and hadrons in their lifespan. These high-energy leptons ($\nu$, $l^\pm$) originating from the decay of $\tau^\pm, \mu^\pm$ continue played a significant role in shaping the particle energy distribution which can affect the property of cosmic plasma.



The $\tau^\pm$ leptons can undergo various decay processes via the weak interaction in the early Universe, and is the only charged lepton that can decay into hadrons because of its heavy mass ($m_\tau=1776.86$ MeV). The principle decay channels of $\tau^\pm$ are given by
\begin{align}
&\tau^-\rightarrow\nu_\tau+e^-+\bar{\nu}_e,\qquad \tau^-\rightarrow\nu_\tau+\mu^-+\bar{\nu}_\mu,\\
&\tau^-\rightarrow\nu_\tau+\pi^-,\qquad\qquad\,\tau^+\rightarrow\bar{\nu}_\tau+\pi^+,
\end{align}
 where the vacuum lifespan for $\tau^\pm$ is given by ~[\cite{ParticleDataGroup:2022pth}]
\begin{align}
&\tau_{\tau}=(290.3\pm0.5)\times10^{-15}\,\mathrm{sec}.
\end{align}

Moreover, following the decay of $\tau^\pm$ into pions, these pions subsequently decay into a muon and a neutrino through the reaction
\begin{align}
\pi^-\rightarrow\nu_\mu+\mu^-,\qquad\qquad\,\pi^+\rightarrow\bar{\nu}_\mu+\mu^+,
\end{align}
with pion vacuum lifespan $\tau_\pi=2.6033\times10^{-8}$ sec~[\cite{ParticleDataGroup:2022pth}].
In this scenario, $\tau^\pm$ disappears from the Universe via multiparticle decay processes.
These decay processes can contribute as one of the sources for the production of neutrinos and muons in the early Universe.

The $\mu^\pm$ lepton abundance is an important quantity required for the understanding of several fundamental questions regarding properties of the primordial Universe,  particularly in relation to the freeze-out of strangeness flavor in the early Universe. We recall that the strangeness decay often proceeds into muons, energy thresholds permitting, as the charged kaons K$^\pm$ have a 63\% branching into $\mu+\bar \nu_\mu$. Should muons fall out of thermal abundance equilibrium this would directly impact the detailed balance back-reaction processes. Another, indirect influence on strangeness in early Universe arises through the nearly exclusive decay of charged pions into $\mu+\bar \nu_\mu$. Without chemical abundance equilibrium this back reaction stops too impacting pions and thus all other hadronic particles in the Universe. 

On the other hand, we will show that the lightest charged leptons $e^\pm$ can persist via the reaction $\gamma\gamma\to e^-e^+$ until the temperature $T=20$ keV in the early Universe.  After $T=20$ keV, the positron rapidly disappears through annihilation, leaving only residual electrons to maintain the Universe's charge neutrality. The existence of an electron-positron plasma plays a pivotal role in several aspects of the early Universe as follows: 

1. The role of electron-positron plasma has not received the appropriate attention in the days of precision Big-Bang nucleosynthesis studies. The standard BBN model indicates that the synthesis of light elements typically takes place at temperatures around  $86\,\mathrm{keV}>T_{BBN}>50\,\mathrm{keV}$~[\cite{Pitrou:2018cgg}]. Within this temperature range there are millions of electron-positron pairs per charged nucleon, providing an electron-positron-rich plasma environment for nucleosynthesis which leads to modifications in the Coulomb potential due to the screening effect. Furthermore, the electron-positron densities can reach millions of times normal atomic densities. The presence of  these $e\bar e$-pairs before and during BBN has been acknowledged by Wang, Bertulani and Balantekin~[\cite{Wang:2010px}] nearly a decade ago.

2. The Universe today is filled with magnetic fields at various scales and strengths both within galaxies and in deep extra-galactic space. The origin of these magnetic fields is currently unknown. In the early Universe, when temperature $T>20$ keV, we have dense $e^\pm$ plasma. The significant magnetic moments of electrons and positrons also provide opportunities to investigate spin magnetization process.

Understanding the abundances of muons and electrons/positrons provides essential insights into the evolution of the primordial Universe.  In the following we discuss the muon density at persistence temperature in section \ref{section_muon}, and explore the electron/positron plasma properties, including the damped rate and magnetization in section \ref{section_electron}.

%~~~~~~~~~~~~~~~~~~~~~~~~~~~~~~~~~~~~~~~~~~~~~~~~~

\section{Muon–antimuon in the early Universe}\label{section_muon}
%In this section we will focus on the following:
%\begin{itemize}
%    \item Vanishing of muon in early Universe
%    \item Muon density at persistence temperature
%\end{itemize}

%\subsection{Vanishing of muon in early Universe}
Our interest in strangeness flavor freeze-out in the early Universe requires the understanding of the abundance of muons in the early Universe. The specific question needing an answer is at which temperature muons remain in abundance (chemical) equilibrium established predominantly by electromagnetic and weak interaction processes, allowing detailed-balance back-reactions to influence strangeness abundance.


In the early Universe in the the cosmic plasma muons of mass $m_\mu=105.66$\,MeV can be produced by the following interaction processes
\begin{align} 
&\gamma+\gamma\longrightarrow\mu^++\mu^-,\qquad & e^++e^-\longrightarrow \mu^++\mu^-\;,\\
&\pi^-\longrightarrow\mu^-+\bar{\nu}_\mu,\qquad & \pi^+\longrightarrow\mu^++\nu_\mu\;.
\end{align}
The back reactions for all above processes are in detailed balance, provided all particles shown on the right hand side (RHS) exist in chemical abundance equilibrium in the Universe. We recall the empty space (no plasma) at rest lifetime of pions $\tau_\pi=2.6033\times10^{-8}$ sec. 

However, all produced muons can also decay via the reactions
\begin{equation}
\mu^-\rightarrow\nu_\mu+e^-+\bar{\nu}_e,\qquad \mu^+\rightarrow\bar{\nu}_\mu+e^++\nu_e\,,
\end{equation} 
with the empty space (no plasma) at rest lifetime $\tau_{\mu}=2.197 \times 10^{-6}\,\mathrm{sec}$. We thus must establish the range of temperature in which production processes exceed in speed the decay process.
 
 The temperature range of our interests is the Universe when $m_\mu\gg T$. In this case the the Boltzmann approximation is appropriate for studying massive particles muons and pions. The thermal decay rate per volume and time  for muons $\mu^\pm$ (and pions $\pi^\pm$) in the Boltzmann limit  are given by~[\cite{PhysRevC.82.035203}]:
\begin{align}
&R_\mu=\frac{g_\mu}{2\pi^2}\left(\frac{T^3}{\tau_\mu}\right)\left(\frac{m_\mu}{T}\right)^2K_1(m_\mu/T)\;,\\
&R_\pi=\frac{g_\pi}{2\pi^2}\left(\frac{T^3}{\tau_\pi}\right)\left(\frac{m_\pi}{T}\right)^2K_1(m_\pi/T)\;, 
\end{align}
where the lifespan of $\mu^\pm$ and $\pi^\pm$ in the vacuum were given above. This rate accounts for both the density of particles in chemical abundance equilibrium and the effect of time dilation present when particles are in thermal motion with respect to observer at rest in the local reference frame. The effects of Fermi blocking or boson stimulated emission have been neglected.

The thermal averaged reaction rate per volume for the reaction $a\overline{a}\rightarrow b\overline{b}$ in Boltzmann approximation is given by [\cite{Letessier:2002ony}]
\begin{align}\label{pairR}
R_{a\overline{a}\rightarrow b\overline{b}}=\frac{g_ag_{\overline{a}}}{1+I}\frac{T}{32\pi^4}\int_{s_{th}}^\infty ds\frac{s(s-4m^2_a)}{\sqrt{s}}\sigma_{a\overline{a}\rightarrow b\overline{b}}~K_1(\sqrt{s}/T),
\end{align}
where $s_{th}$ is the threshold energy for the reaction, $\sigma_{a\overline{a}\rightarrow b\overline{b}}$ is the cross section for the given reaction, and $K_1$ is the modified
Bessel function of integer order $”1”$. We introduce the factor $1/1+I$ to avoid the double counting of indistinguishable pairs of particles; we have $I=1$ for an identical pair and $I=0$ for a distinguishable pair.

The leading order invariant matrix elements for the reactions $e^++e^-\to\mu^++\mu^-$ and $\gamma+\gamma\to\mu^++\mu^-$, are introduced in this work by [\cite{Kuznetsova:2008jt}]
\begin{align}\label{Mee}
|M_{e\bar e\to\mu\bar\mu}|^2=&32\pi^2\alpha^2\frac{(m_\mu^2-t)^2+(m_\mu^2-u)^2+2m_\mu^2s}{s^2},\quad m_\mu\gg m_e\;,\\[0.2cm]
\label{Mgg}
|M_{\gamma\gamma\to\mu\bar\mu}|^2=&32\pi^2\alpha^2\bigg[\left(\frac{m_\mu^2-u}{m_\mu^2-t}+\frac{m_\mu^2-t}{m_\mu^2-u}\right)+4\left(\frac{m_\mu^2}{m_\mu^2-t}+\frac{m_\mu^2}{m^2_\mu-u}\right)\\[0.1cm]  \nonumber
&\hspace{1cm}-4\left(\frac{m_\mu^2}{m^2_\mu-t}+\frac{m^2_\mu}{m^2_\mu-u}\right)^2\bigg]\;,
\end{align}
 where $s, t, u$ are the Mandelstam variables. The cross section required in Eq.\,(\ref{pairR}) can be obtained by integrating the matrix elements Eq.\,(\ref{Mee}) and Eq.\,(\ref{Mgg}) over the Mandelstam variable $t$ ~[\cite{PhysRevC.82.035203}]. We have
\begin{align}
&\sigma_{e\bar e\to\mu\bar\mu} 
=\frac{64\pi\alpha^2}{48\pi}\left(\frac{1+2m^2_\mu/s}{s-4m_e^2}\right)\sqrt{1-\frac{4m^2_\mu}{s}},\\
&\sigma_{\gamma\gamma\to\mu\bar\mu}=\frac{\pi}{2}\left(\frac{\alpha}{m_\mu}\right)^2(1-\beta^2)\left[(3-\beta^4)\ln\frac{1+\beta}{1-\beta}-2\beta(2-\beta^2)\right],\\
&\beta=\sqrt{1-4m^2_\mu/s}
\end{align}
Substituting the cross sections into Eq.\,(\ref{pairR}) we obtain the production rates for $e\bar e\to\mu\bar\mu$ and $\gamma\gamma\to\mu\bar\mu$ respectively.

 
In Fig.~\ref{MuonRatenew_fig} we show the invariant thermal reaction rates per volume and time for rates of relevance, as a function of temperature $T$.
As the temperature decreases in the expanding Universe, the initially dominant production rates ($e\bar e,\gamma\gamma\to\mu\bar\mu$) decrease with decreasing temperature, and eventually cross the $\mu^\pm$ decay rates. 
Muon abundance disappears as soon as any decay rate is faster than the fastest production rate. Specifically after the Universe cools below the temperature $T_\mathrm{disappear}=4.195$ MeV, the dominant reaction is the muon decay. Due to the relatively slow expansion of the Universe, the disappearance of muons is sudden, and the abundance of muons vanishes as soon as a decay rate surpasses the dominant production rate.
 

%~~~~~~~Figure~~~~~~ ~~~~~~~~~~~~~~~~~~~~~~~~~~~~~~~~~~~~~~~~~~~~~~~~~~~~~~~~~~~~~~~~~~~
\begin{figure}[ht]
\begin{center}
\includegraphics[width=5.0in]{./plots/MuonRate_new2.pdf}
\caption{We plot the thermal reaction rate per volume for different reactions as a function of temperature. We found that dominant reactions for $\mu^\pm$ production are ${\gamma+\gamma\to\mu^++\mu^-}$ and $e^++e^-\to\mu^++\mu^-$, and the total production rate crosses the decay rate of $\mu^\pm$ at temperature $T_{dissapear}\approx 4.195$ MeV.}
\label{MuonRatenew_fig}
\end{center}
\end{figure}
%~~~~~~~~~~~~~~~~~~~~~~~~~~~~~~~~~~~~~~~~~~~~~~~~~~~~~~~~~~~~~~~~~~~~~~~~~~~~~~~~~~~~~~~~~~~~~~~~

On the other hand, considering the number density for nonrelativistic $\mu^\pm$ in the Boltzmann approximation, we have
\begin{align}\label{nmupm}
n_{\mu^\pm}=\frac{g_{\mu^\pm}}{2\pi^2}T^3\left(\frac{m_\mu}{T}\right)^2 K_2(m_\mu/T)=g_{\mu^\pm}\left(\frac{m_\mu T}{2\pi}\right)^{3/2}e^{-{m_\mu}/{T}}\;. 
\end{align}
then the number density between $n_{\mu^\pm}$ and baryon $n_B$ can be written as
\begin{align}
\frac{n_{\mu^\pm}}{n_\mathrm{B}}=\frac{n_{\mu^\pm}}{s}\frac{s}{n_\mathrm{B}}=
\frac{n_{\mu^\pm}}{s}\left(\frac{s}{n_\mathrm{B}}\right)_{\!t_0},
\end{align}
where we used that $s/n_\mathrm{B}$ remains constant and $t_0$ represent present day value. The present value is given by $(n_B/s)_{t_0}\approx8.69\times10^{-11}$ (detail please see Chapter~\ref{Introduction}). The entropy density $s$ can be characterized introducing $g^s_\ast$, the total number of \lq entropic\rq\ degrees of freedom
\begin{align}\label{entrop}
s=\frac{2\pi^2}{45}g^s_\ast T^3\;.
\end{align}
For temperature $10\,\mathrm{MeV} >T>3 $\,MeV, the massless photons, nearly relativistic electron/positrons, and practically massless neutrinos contribute to the degree of freedom $g^s_\ast$.  In this case, the number density between $n_{\mu^\pm}$ and baryon $n_B$ in the temperature interval we consider $10\,\mathrm{MeV} >T>3 $\,MeV is given by
\begin{align}\label{nmuperbF} 
\frac{n_{\mu^\pm}}{n_\mathrm{B}}=\frac{45}{2\pi^2}\frac{g_{\mu^\pm}}{g^s_\ast}\left(\frac{m_\mu}{2\pi T}\right)^{3/2}e^{-{m_\mu}/{T}}\;\left(\frac{s}{n_\mathrm{B}}\right)_{\!t_0}.
\end{align}


%Figure~~~~~~~~~~~~~~~~~~~~~~~~~~~~~~~~~~~~~~~~~~~~~~~~~~~~~~~~~~~~~~~~~~~~~~~~~
\begin{figure}[t]
\begin{center}
\includegraphics[width=\linewidth]{./plots/DensityRatio_new2.pdf}
\caption{
The density ratio between $\mu^\pm$ and baryons as a function of temperature. The density ratio at muon disappearance temperature is about $n_{\mu^\pm}/n_\mathrm{B}(T_\mathrm{disappear})\approx0.911$, and around the temperature $T\approx4.212$ MeV the density ratio $n_{\mu^\pm}/n_\mathrm{B}\approx1$.}
\label{DensityRatio_fig}
\end{center}
\end{figure}
%~~~~~~~~~~~~~~~~~~~~~~~~~~~~~~~~~~~~~~~~~~~~~~~~~~~~~~~~~~~~~~~~~~~~~~~~~~~~~


In Fig.\,\ref{DensityRatio_fig} we show the muon to baryon density ratio Eq.\,(\ref{nmuperbF}) as a function of $T$. We see that the muon abundance $T=10$\,MeV exceeds that of baryons by a factor 500,000 while at muon disappearance temperature $n_{\mu^\pm}/n_\mathrm{B}(T_\mathrm{disappear})\approx0.911$. The number density $n_{\mu^\pm}$ and $n_\mathrm{B}$  abundances are equal at around the temperature $T_\mathrm{equal}\approx4.212\,\mathrm{MeV} >  T_\mathrm{disappear}$.  This means that the muon abundance may still be able to influence baryon evolution because their number density is comparable to the baryon density.% However, we also find that at the temperature $T_\mathrm{equal}\approx4.212$\,MeV the density ratio is unity $n_{\mu^\pm}/n_\mathrm{B}\approx1$.

The primary insight of this work is that aside of protons, neutrons and other nonrelativistic particles, both positively and negatively charged muons $\mu^\pm$ are present in thermal equilibrium and in non-negligible abundance for $T>T_\mathrm{dissapear}\approx 4.195$\,MeV. This offers a new and tantalizing model building opportunity for anyone interested in baryon-antibaryon separation in the primordial Universe, strangelet formation, and perhaps other exotic primordial structure formation mechanisms.





%~~~~~~~~~~~~~~~~~~~~~~~~~~~~~~~~~~~~~~~~~~~~~~~~~

\section{ Electron-positron plasma in the early Universe}\label{section_electron}
%In this section we will focus on the following:
%\begin{itemize}
%    \item Chemical potential of electron in early universe
%    \item Electron-positron plasma in BBN (Damped sccreening)
%    \item Electron-positron magnetization
%%    \item Neutron Lifespan in magnetized electron/positron plasma.
%\end{itemize}

In the early Universe, after the neutrino freeze-out at $T\approx 2$\,MeV, the Universe is controlled by the electron-positron-photon plasma. In this section, we demonstrate the rich electron-positron plasma in the early Universe by examining the chemical potential $\mu_e$ in the charge-neutral and entropy-conserving Universe. We study the  microscope collision property of electron-positron plasma and explore the spin response of the electron-positron plasma to external and self-magnetization fields, thus developing methods for future detailed study.

%In this section, we will quantify the dynamical picture of $e^\pm$ plasma and show that the $e^+$ abundance can persist in early universe at relatively low temperature $T = 20$ keV which provide the dense $e^\pm$ plasma environment for the big-bang nucleosynthesis (BBN) in the early universe. 

%The role of electron-positron plasma has not received the appropriate attention in the days of precision big bang nucleosynthesis studies. The standard BBN model indicates that the synthesis of light elements typically takes place at temperatures around  $86\,\mathrm{keV}>T_{BBN}>50\,\mathrm{keV}$~[\cite{Pitrou:2018cgg}]. Within this temperature range there are millions of electron-positron pairs per charged nucleon, providing an electron-positron-rich plasma environment for nucleosynthesis. Furthermore, the electron-positron densities can reach millions of times normal atomic densities. The presence of  these $e\bar e$-pairs before and during BBN has been acknowledged by Wang, Bertulani and Balantekin~[\cite{Wang:2010px}] nearly a decade ago.





%On the other hand, the Universe today filled with magnetic fields at various scales and strengths both within galaxies and in deep extra-galactic space.It is currently unknown the origin for these magnetic fields today. In early Universe when temperature $T>20$ keV , we have dense $e^\pm$ plasma. The significant magnetic moments of electrons and positrons also provide opportunities to investigate spin magnetization process.


%~~~~~~~~~~~~~~~~~~~~~~~~~~~~~~~~~~~~~~~~~~~~~~~~~~~~~~~~~~~~~~~~~~~~~~~~~

\subsection{Electron chemical potential in the early Universe}
In this section, we derive the dependence of electron chemical potential, and hence $e^\pm$ density, on the photon background temperature by employing the following physical principles:
\begin{enumerate}
\item Charge neutrality of the Universe:
\begin{align}\label{neutrality}
n_e-n_{\overline{e}}=n_p-n_{\overline{p}}\approx\,n_p,
\end{align}
where $n_e$ and $n_{\overline{e}}$ denotes the number density of electron and positron.
\item Neutrinos decouple (freeze-out) at a temperature $T_f\simeq 2$ MeV, after which they free stream through the Universe with an effective temperature~[\cite{Birrell:2012gg}]
\begin{align}
T_\nu(t)=T_f a(t_f)/a(t),
\end{align}
 where $a(t)$ is the FLRW Universe scale factor.
\item Total comoving entropy is conserved. At $T\leq T_f$ the dominant contributors to entropy are photons, $e^\pm$, and neutrinos.
In addition, after neutrino freeze-out, neutrino comoving entropy is independently conserved ~[\cite{Birrell:2012gg}]. This  implies that the combined comoving entropy in $\gamma$, $e^\pm$ is also conserved for $T_\gamma\leq T_f$.
\end{enumerate}

Motivated by the fact that comoving entropy in $\gamma$, $e^\pm$ is conserved after neutrino freeze-out, we rewrite the charge neutrality condition, Eq.(\ref{neutrality}) in the form
\begin{align}\label{charge_neutral_cond2}
n_e-n_{\overline{e}}=X_p\frac{n_B}{s_{\gamma,e,\overline{e}}} s_{\gamma,e,\overline{e}},\qquad X_p\equiv\frac{n_p}{n_B},
\end{align}
where $n_B$ is the number density of baryons, and $s_{\gamma,e,\overline{e}}$ is the combined entropy density in photons, electrons, and positrons. During the Universe expansion, the comoving entropy and baryon number are conserved quantities, hence the ratio $n_B/s_{\gamma,e,\overline{e}}$ is conserved. We have
\begin{align}
\frac{n_B}{s_{\gamma,e,\overline{e}}}=\left(\frac{n_B}{s_{\gamma,e,\overline{e}}}\right)_{t_0}\!\!\!\!=\left(\frac{n_B}{s_{\gamma}}\right)_{t_0}\!\!\!\!=\left(\frac{n_B}{n_\gamma}\right)_{t_0}\left(\frac{n_\gamma}{s_{\gamma}}\right)_{t_0},
\end{align}
where the subscript $t_0$ denotes the present day value, and the second equality is obtained by observing that the present day $e^\pm$-entropy density is negligible compared to the photon entropy density. We can evaluate the ratio by giving the present day baryon-to-photon ratio: $n_B/n_\gamma= 6.05\times10^{-10}$(CMB) ~[\cite{ParticleDataGroup:2022pth}] and the entropy per particle for a massless boson:$(s/n)_{\mathrm{boson}}\approx 3.602$~[\cite{Letessier:2002ony}].

The total entropy density of photons and electron/positron can be written as
\begin{align}\label{entropy_per_baryon}
s_{\gamma,e,\overline{e}}=\frac{2\pi^2}{45}g_\gamma\,T_\gamma^3+\frac{\rho_{e,\overline{e}}+P_{e,\overline{e}}}{T_\gamma}-\frac{\mu_e}{T_\gamma}(n_e-n_{\overline{e}}),
\end{align}
where $ \rho_{e,\overline{e}}=\rho_{e}+\rho_{\overline{e}}$ and $P_{e,\overline{e}}=P_{e}+P_{\overline{e}}$ are the total energy density and pressure of electrons/positron respectively.
The energy density and pressure in electrons and positrons are given by
\begin{align}\label{rho_e}
\frac{\rho_{e,\overline{e}}}{T_\gamma^4}=\frac{g_e}{2\pi^2}M_e^4 \bigg[&\int_{1}^\infty \frac{ u^2\sqrt{ u^2-1} du}{\exp(M_e u-b_e)+1}+\int_{1}^\infty \frac{ u^2\sqrt{ u^2-1} du}{\exp(M_e u+b_e)+1}\bigg]\,,
\end{align}
and
\begin{align}\label{P_e}
\frac{P_{e,\overline{e}}}{T_\gamma^4}=\frac{g_e}{6\pi^2}M_e^4\bigg[&\int_{1}^\infty   \frac{(u^2-1)^{3/2} du}{\exp(M_e u-b_e)+1}+\int_{1}^\infty   \frac{(u^2-1)^{3/2} du}{\exp(M_e u+b_e)+1}\bigg],
\end{align}
where we introduce the dimensionless variables as follows: 
\begin{align}\label{Variables}
u=\frac{E}{m_e},\qquad M_e=\frac{m_e}{T_\gamma},\qquad b_e=\frac{\mu_e}{T_\gamma}.
\end{align}

By incorporating Eq.(\ref{charge_neutral_cond2}) and Eq.(\ref{entropy_per_baryon}), the charge neutrality condition can be expressed as
\begin{align}\label{charge_neutral_cond3}
&\left[1+X_p\left(\frac{n_B}{n_\gamma}\right)_{t_0}\left(\frac{n_\gamma}{s_{\gamma}}\right)_{t_0}\frac{\mu_e}{T_\gamma}\right]\frac{n_e-n_{\overline{e}}}{T_\gamma^3}\notag\\
&\qquad\qquad\qquad=X_p\left(\frac{n_B}{n_\gamma}\right)_{t_0}\left(\frac{n_\gamma}{s_{\gamma}}\right)_{t_0} \left(\frac{2\pi^2}{45}g_\gamma+\frac{\rho_{e,\overline{e}}+P_{e,\overline{e}}}{T_\gamma^4}\right).
\end{align}
Using the Fermi distribution, the number density of electrons over positrons in the early Universe is given by
\begin{align}\label{ee_density}
n_e-n_{\overline{e}}&=\frac{g_e}{2\pi^2}\left[\int_0^\infty\frac{p^2dp}{\exp{\left((E-\mu_e)\right)/T_\gamma}+1}\right.\left.-\int_0^\infty\frac{p^2dp}{\exp{\left((E+\mu_e)/T_\gamma\right)}+1}\right]\notag\\
&=\frac{g_e}{2\pi^2}{T_\gamma^3}\tanh(b_e)M_e^3\int_{1}^\infty \!\!\!\!\frac{  u \sqrt{u^2-1} du}{1+\cosh(M_eu)/\cosh(b_e)}.
\end{align}
Substituting Eq.(\ref{ee_density}) into Eq.(\ref{charge_neutral_cond3}) and giving the value of $X_p$, the charge neutrality condition can be solved to determine $\mu_e/T_\gamma$ as a function of $M_e$ and $T_\gamma$. 
%Fig~~~~~~~~~~~~~~~~~~~~~~~~~~~~~~~~~~~~~~~~~~~~~~~~~~~~~
\begin{figure}[ht]
\begin{center}
\includegraphics[width=\linewidth]{./plots/May152023_EPDensity_Chemical}
\caption{Left axis: The chemical potential of an electron as a function of photon temperature $T=T_\gamma$ with $X_p=0.878$ and $n_B/n_\gamma=6.05\times10^{-10}$. Right axis: the ratio of electron(positron) number density to baryon density as a function of temperature. The blue solid line is the electron density, the red dashed line is the positron density, and the green dotted line is the number density with $\mu_e=0$. We found that when electron chemical potential $\mu_e\approx T=0.02\,\mathrm{MeV}$ the positron density decreases because of the annihilation.}
\label{BBN_Electron}
\end{center}
\end{figure}
%~~~~~~~~~~~~~~~~~~~~~~~~~~~~~~~~~~~~~~~~~~~~~~~~~~~~~

In Fig.~\ref{BBN_Electron} (left axis) we solve Eq.(\ref{charge_neutral_cond3}) numerically and plot the electron chemical potential as a function of temperature with the following parameters: proton concentration $X_p=0.878$ from 
 observation~[\cite{ParticleDataGroup:2022pth}] and  $n_B/n_\gamma=6.05\times10^{-10}$ from CMB. We can see the value of chemical potential is comparatively small $\mu_e/T\approx10^{-6}\sim10^{-7}$ during the BBN temperature range, implying an equal number of electrons and positrons in plasma. From the ratio of electron (positron) number density to baryon density in Fig.~\ref{BBN_Electron} (right axis) we can see that during the accepted BBN temperature range the Universe was filled with an electron-positron rich plasma.
For example when the temperature is around $T=70\,\mathrm{keV}$ the density of electrons and positrons is comparatively large in the early Universe $n_{e^\pm}\approx10^7\,n_B$. Later when the temperature is around $T=20.3\,\mathrm{keV}$, the positron density decreases, leading to the transformation of the pair plasma to an electron-proton plasma.
%~~~~~~~~~~~~~~~~~~~~~~~~~~~~~~~~~~~~~~~~~~~~~~~~~
\subsection{Microscope damping rate of electron-positron plasma}\label{relax}
In electron-positron plasma, the major reactions between photons and $e^+e^-$ pairs are inverse Compton scattering, M{\o}ller scattering, and Bhabha scattering:
\begin{align}
&e^\pm+\gamma\longrightarrow e^\pm+\gamma,\qquad e^\pm+e^\pm\longrightarrow e^\pm+e^\pm,\qquad e^\pm+e^\mp\longrightarrow e^\pm+e^\mp.
\end{align}
The general formula for thermal reaction rate per volume is discussed in~[\cite{Letessier:2002ony}] (Eq.(17.16), Chapter 17). For inverse Compton scattering we have
\begin{align}
R_{e^{\pm}\gamma}=\frac{g_eg_\gamma}{16\left(2\pi\right)^5}T\int_{m_e^2}^\infty\!\!\!\!ds\frac{K_1(\sqrt{s}/T)}{\sqrt{s}}\int^0_{-(s-m_e^2)^2/s}\!\!\!\!\!\!\!\!\!\!\!\!\!\!\!\!dt\, |M_{e^{\pm}\gamma}|^2,
\end{align} 
and for M{\o}ller and Bhabha reactions we have
\begin{align}
&R_{e^\pm e^\pm}=\frac{g_eg_e}{16\left(2\pi\right)^5}T\!\!\int_{4m_e^2}^\infty\!\!\!\!ds\frac{K_1(\sqrt{s}/T)}{\sqrt{s}}\int^0_{-(s-4m_e^2)}\!\!\!\!\!\!\!\!\!\!\!\!\!\!\!\!dt\,|M_{e^\pm e^\pm}|^2,\\
&R_{e^\pm e^\mp}=\frac{g_eg_e}{16\left(2\pi\right)^5}T\!\!\int_{4m_e^2}^\infty\!\!\!\!ds\frac{K_1(\sqrt{s}/T)}{\sqrt{s}}\int^0_{-(s-4m_e^2)}\!\!\!\!\!\!\!\!\!\!\!\!\!\!\!\!dt\,|M_{e^\pm e^\mp}|^2,
\end{align}
where $g_i$ is the degeneracy of particle $i$, $|M|^2$ is the matrix element for a given reaction, $K_1$ is the Bessel function of order $1$, and $s,t,u$ are Mandelstam variables. The leading order matrix element associated with inverse Compton scattering can be expressed in the Mandelstam variables~[\cite{Kuznetsova:2011wt, Kuznetsova:2009bq}] we have
\begin{align}
|M_{e^\pm\gamma}|^2\!=32 \pi^2\alpha^2\bigg[&4\left(\frac{m_e^2}{m_e^2-s}+\frac{m_e^2}{m_e^2-u}\right)^2\notag\\
&\qquad\qquad-\frac{4m_e^2}{m_e^2-s}-\frac{4m_e^2}{m_e^2-u} -
 \frac{m_e^2-u}{m_e^2-s} -\frac{m_e^2-s}{m_e^2-u}\bigg],
\end{align}
and for M{\o}ller and Bhabha scattering we have 
\begin{align}
|M_{e^{\pm}e^{\pm}}|^{2}\!=64\pi^{2}\alpha^{2}\bigg[&
\frac{s^{2}+u^{2}+8m_e^{2}(t-m_e^{2})}{2(t-m^2_{\gamma})^{2}}\notag\\
&\quad+\frac{{s^{2}+t^{2}}+8m_e^{2}
(u-m_e^{2})}{2(u-m_{\gamma}^2)^{2}} + \frac{\left( {s}-2m_e^{2}\right)\left({s}-6m_e^{2}\right)}
{(t-m_{\gamma}^2)(u-m_{\gamma}^2)} \bigg],
\end{align}
and
\begin{align}
|M_{e^\pm e^\mp}|^{2}=64\pi^{2}\alpha^{2}
\bigg[&\frac{s^{2}+u^{2}+8m_e^{2}(t-m_e^{2})}{2(t-m^2_{\gamma})^{2}}\notag\\
&\quad+\frac{u^{2}+t^{2}+8m_e^{2}
(s-m_e^{2})}{2(s-m^2_{\gamma})^{2}}  +   \frac{\left({u}-2m_e^{2}\right)\left({u}-6m_e^{2}\right)}
   {(t-m^2_{\gamma})(s-m^2_{\gamma})} \bigg],
\label{M_fi_b}
\end{align}
where we introduce the photon mass $m_\gamma$ to account the plasma effect and avoid singularity in reaction matrix elements. 

The photon mass $m_\gamma$ in plasma is equal to the plasma frequency $\omega_p$, where we have~[\cite{Kislinger:1975uy}]
\begin{align}
m^2_\gamma=\omega^2_{p}=8\pi\alpha\int\frac{d^3p_e}{(2\pi)^3}\left(1-\frac{p_e^2}{3E_e^2}\right)\frac{f_e+f_{\bar e}}{E_e},
\end{align}
where $E_e=\sqrt{p_e^2+m^2_e}$. In the BBN temperature range $86\,\mathrm{keV}>T_{BBN}>50\,\mathrm{keV}$ we have $m_e\gg T$ and considering the nonrelativistic limit for electron-positron plasma, we obtain
\begin{align}
m^2_\gamma=\frac{4\pi\alpha}{2m_e}\left(\frac{2m_eT}{\pi}\right)^{3/2}e^{-m_e/T}\cosh\left(\frac{\mu_e}{T}\right).
\end{align}
In the BBN temperature range, we have $\mu_e/T\ll1$, which implies the equal number of electrons and positrons in plasma.

To discuss the collisional plasma by the linear response theory, it is convenient to define the average relaxation rate for the electron-positron plasma as follows:
\begin{align}\label{Kappa}
\kappa=\frac{R_{e^\pm e^\pm}+R_{e^\pm e^\mp}+R_{e^\pm\gamma}}{\sqrt{n_{e^-}n_{e^+}}}\approx\frac{R_{e^\pm e^\pm}+R_{e^\pm e^\mp}}{\sqrt{n_{e^-}n_{e^+}}},
\end{align}
where the density function ${\sqrt{n_{e^-}n_{e^+}}}$ in the Boltzmann limit is given by
\begin{align}
{\sqrt{n_{e^-}n_{e^+}}}=\frac{g_e}{2\pi^3}T^3\left(\frac{m_e}{T}\right)^2K_2(m_e/T).
\end{align}
In Fig.~\ref{RelaxationRate_fig}, we show the reaction rates for M{\o}ller reaction, Bhabha reaction, and inverse Compton scattering as a function of temperature. For temperatures $T>12.0$ keV, the dominant reactions in plasma are M{\o}ller and Bhabha scatterings between electrons and positrons. Thus in the BBN temperature range, we can neglect the inverse Compton scattering. The total relaxation rate is approximately constant $\kappa=10\sim12$ keV during the BBN. For $T<20.3$ keV the relaxation rate $\kappa$ decreases rapidly because of positron annihilation. At this temperature, the composition of plasma begins to change from an electron-positron plasma to an electron-baryon plasma.
%~~~~Figure~~~~~~~~~~~~~~~~~~~~~~~~~~~
\begin{figure}[h]
\begin{center}
%\includegraphics[width=0.95\linewidth]{KappaRateToT_May082023}
\includegraphics[width=\linewidth]{./plots/May152023Kappa_EPPlasma}
\caption{The relaxation rate $\kappa$ as a function of temperature in nonrelativistic electron-positron plasma. For comparison, we show  reaction rates  for M{\o}ller reaction $e^-+e^-\to e^-+e^-$ (blue line), Bhabha reaction $e^-+e^+\to e^-+e^+$ (red line), and inverse Compton scattering $e^-+\gamma\to e^-+\gamma$ (green line) respectively. It shows that the dominant reactions during BBN are the M{\o}ller and Bhabha scatterings between electrons and positrons. The total relaxation rate Eq.(\ref{Kappa}) is shown in the black line. It shows that we have $\kappa=10\sim12$ keV during the BBN temperature range. For comparison, the Debye mass $m_D=\omega_{p}\sqrt{m_e/T}$(purple line) is shown as a function of temperature.
}
\label{RelaxationRate_fig}
\end{center}
\end{figure}
%~~~~Figure~~~~~~~~~~~~~~~~~~~~~~~~~~~


\subsubsection{From static to damped dynamic screening}

At present, the observation of light element (e.g. D, $^3$He, $^4$He, and $^7$Li) abundances produced in Big-Bang nucleosynthesis (BBN) offers a reliable probe of the early Universe before the recombination. Much effort of the BBN study is currently being made to reconcile the discrepancies and tensions between theoretical predictions and observations of light element abundances, e.g. $^7$Li problem ~[\cite{Pitrou:2018cgg, Fields:2011zzb}].
Current models assume that the Universe was essentially void of anything but reacting light nucleons and electrons needed to keep the local baryon density charge-neutral, a situation similar to the experimental environment where empirical nuclear reaction rates are obtained.

The electron-positron plasma influences light element abundances through electromagnetic screening of the nuclear potential. The electron cloud surrounding the charge of an ion screens other nuclear charges far from its own radius and reduces the Coulomb barrier. In nuclear reactions, the reduction of Coulomb barrier makes the penetration probability easier and enhance the thermonuclear reaction rates. In this case, the modification of the nuclei interaction due to the plasma screening effect may plays a key role in the formation of light element in the BBN. 

The enhancement factor of thermonuclear reaction rates and screening potential are calculated by Salpeter in 1954~[\cite{Salpeter:1954nc}], which describes the static screening effects for the thermonuclear reactions. In an isotropic and homogeneous plasma the Coulomb potential of a point-like particle with charge $Ze$ at rest is modified into~[\cite{Salpeter:1954nc}]
\begin{align}
\phi_\text{stat}(r)=\frac{Ze}{4\pi\epsilon_0 r}e^{-m_Dr},
\end{align}
where $m_D$ is the Debye mass. After that it has been exploited widely in BBN for static screening ~[\cite{1969ApJ...155..183S,Famiano:2016hhs}]. 

Subsequently, the study of dynamical screening for moving ions has been taken into account~[\cite{1988ApJ...331..565C,Gruzinov:1997as,Hwang:2021kno}]. When a test charge moves with a velocity that is enough to react with the background charge in plasma, the Coulomb potential is modified by the dynamical effect. However, the applications focus on the weakly interacting electron-positron plasma only. 

In our separate work~[\cite{Grayson:2023flr}] we use the linear response theory adapted by C.Grayson to to describe the inter nuclear potential in electron-positron plasma during BBN. We improve the prior efforts by evaluation and inclusion of the collision damping rate due to scattering in the dense plasma medium and provide an approximate analytic formula that can be readily used to estimate the effect of screening on internuclear potential. For comprehensive discussion and the application of the damped dynamic screening see~[\cite{Grayson:2023flr}].










%~~~~~~~~~~~~~~~~~~~~~~~~~~~~~~~~~~~~~~~~~~~~~~~~~~~~
\subsection{Magnetization of the electron-positron plasma}

In the present-day Universe, we have magnetic fields~[\cite{giovannini2003magnetized, Kronberg:1993vk,kronberg1994extragalactic}] at various scales and strengths both within galaxies and in deep extra-galactic space far away from matter sources. Current observations suggest the upper and lower bounds for the Extra-Galactic Magnetic Field (EGMF) are given by~[\cite{neronov2010evidence,taylor2011extragalactic,pshirkov2015new,jedamzik2019stringent,vernstrom2021discovery}]
\begin{align}
    \label{egmf}
    10^{-8}{\mathrm G}>B_{\mathrm{EGFM}}>10^{-16}{\mathrm G}\,.
\end{align}
The origin for EGMF today is currently unknown; different models are considered in lectures~[\cite{Widrow:2011hs,Vazza:2021vwy}]. In our work~[\cite{Rafelski:2023emw}], we investigate the hypothesis that the observed EGMF are primordial in nature, predating even the recombination
epoch. Under this hypothesis, the first best candidate is the electron-positron plasma. This is because for the temperature range $ 200\,\mathrm{keV} > T > 20$ keV, we still have relatively large quantity of both $e^\pm$ in the the early Universe plasma. In addition, electrons and positrons have the largest magnetic moments in nature, are likely to have been magnetized in the early Universe due to spin orientation. These  provide the possibility origins for a primordial magnetic field.

As the Universe undergoes the isentropic expansion,  the temperature gradually decreases as $T\propto1/a(t)$, where $a(t)$ represents the scale factor. The assumption is made that the magnetic flux is conserved over comoving surfaces, implying that the primordial relic field is expected to dilute as $B\propto1/a(t)^{2}$~[\cite{Rafelski:2023emw}]. Combining these cosmological redshift relations, we can introduce a dimensionless cosmic magnetic scale that remains unchanged during the evolution of the Universe 
\begin{align}
    \label{tbscale}
    b \equiv\frac{e{B}}{T^{2}}=\left(\frac{e{B}}{T^{2}}\right)_{t_0}=b_0={\rm\ const.}\qquad10^{-3}>b_{0}>10^{-11}\,.
\end{align}
The upper and lower bounds for $b_0$ are estimated by using the present day EGMF observations Eq.~(\ref{egmf}) and the present CMB temperature $T_{0}=2.7\,\mathrm{K}\approx2.3\times10^{-4}$ eV~[\cite{aghanim2018planck}].
As $b_0$ is a constant of expansion, this means the contemporary small bounded values of may have once represented large magnetic fields in the early Universe and require detailed study in a different epoch of the Universe. Therefore, correctly describing the dynamics of this $e^{\pm}$ plasma is of interest when considering the origin of extra-galactic magnetic fields (EGMF). 

In the following,  we will demonstrate that fundamental quantum statistical analysis can lead to further insights on the behavior of magnetized plasma, and show that the $e^\pm$ plasma is overall paramagnetic and yields a positive overall magnetization, which
is contrary to the traditional assumption that matter-antimatter plasma lack significant magnetic responses. For more detailed discussion  of electron-positron plasma magnetization, please see~[\cite{Andrew:2023abc}].



\subsubsection{Electron-positron partition function}
To study the statistical behavior of the $e^\pm$ system in a magnetic field, we utilize the general Fermion partition function~[\cite{Elze:1980er}]
\begin{align}
 \label{PartFunc} \ln\mathcal{Z}=\sum_{\alpha}\ln\left(1+e^{-\beta(E-\eta)}\right)\,,
\end{align}
where $\beta=1/T$, $\alpha$ is the set of all quantum numbers in the system, and $\eta$ is the generalized chemical potential. In the case of a magnetized $e^{\pm}$ system, we consider it as a system of four quantum species: Particles and antiparticles, and spin aligned and anti-aligned. Taken together, we consider a system where electrons and positrons can be spin aligned or anti-aligned with the magnetic field $B$ and the partition function of the system can be written as
\begin{align}\label{PartFuncB}
%&\ln\mathcal{Z}_{tot}=&\frac{2eBV}{(2\pi)^2}\sum_{\sigma}^{\pm1}\sum_{s}^{\pm1/2}\sum_{n=0}^\infty\int^\infty_{0}dp_z\left[\ln\left(1+\Upsilon_{\sigma}^{s}(x)e^{-\beta E_{n}^{s}}\right)\right]\,\\
\ln\mathcal{Z}_{tot}=\frac{2eBV}{(2\pi)^2}\sum_{\sigma}^{\pm1}\sum_{s}^{\pm1/2}\sum_{n=0}^\infty\int^\infty_{0}dp_z\left[\ln\left(1+\Upsilon(x)e^{(\sigma\eta_{e}+s\eta_s)/T}e^{-\beta E_{n}^{s}}\right)\right]\,,
\end{align}
where $n$ is the principle quantum number for the Landau levels. The parameter $\eta_{e}$ is the electron chemical potential and $\eta_s$ is the spin chemical potential~[\cite{Andrew:2023abc}]. The parameter $\Upsilon(x)$ is the fugacity of the Fermi gas. In this thesis we will focus on the case $\Upsilon(x)=1$ and $\eta_s=0$ , 
we leave the general case $\Upsilon(x)\neq1$ and $\eta_s\neq0$ for future work.



%In general, $\Upsilon=1$ represents the maximum entropy and corresponds to the normal Fermi distribution. The deviation of $\Upsilon\neq1$ represents the configurations of reduced entropy without pulling the system off a thermal temperature. This scenario is well studied for quarks in QGP. The situation for $e^\pm$ plasma is similar to the case of the quarks during QGP, but instead here the deviation is spatial rather than temporal. Inhomogeneity can arise from the influence of other forces on the gas such as gravitational forces. This is precisely the kind of behavior that may arise in the $e^{\pm}$ epoch as the dominant photon thermal bath keeps the Fermi gas in thermal equilibrium while spatial inequilibrium could spontaneously develop. 


In the following, we will retain $\Upsilon(x)=1$ and consider the case $\eta_s/T\ll1$ for the first approximation. Then the partition function becomes
\begin{align}
\ln\mathcal{Z}_{tot}=\frac{2eBV}{(2\pi)^2}\sum_{s}^{\pm1/2}\sum_{n=0}^\infty\int^\infty_{0} \!\!dp_z\bigg[\ln\left(1+e^{-\beta(E_{n}^s-\eta_e)}\right)+\ln\left(1+e^{-\beta(E_{n}^s+\eta_e)}\right)\bigg].
\end{align}
Considering the $e^\pm$ plasma in a uniform magnetic field $B$ pointing along the $z$-axis, the energy $E_{n}^\pm$ of electron/positron system can be written as~[\cite{Rafelski:2023emw}]
\begin{align}
&E_{n}^\pm=\sqrt{p^2_z+\tilde m^2_\pm+2eBn},\qquad\tilde{m}^2_\pm=m^2_e+eB\left(1\mp\frac{g}{2}\right)\,,
\end{align}
where the $\pm$ script refers to spin aligned and anti-aligned eigenvalues. The parameter $g$ is the gyro-magnetic ($g$-factor) of the particle. 

To simplify the partition function, we consider the expansion of the logarithmic function as follows:
\begin{align}
\ln\left(1+x\right)=\sum^{\infty}_{k=1}\frac{(-1)^{k+1}}{k}x^k, \,\,\,\,\,\,\,\mathrm{for}\,|x|<1.
\end{align}
Then the partition function of electron/positron system can be written as
\begin{align}
\ln\mathcal{Z}_{tot}=&\frac{2eBV}{(2\pi)^2}\sum_{n=0}^\infty\int^\infty_{0} \!\!dp_z\sum^{\infty}_{k=1}\frac{(-1)^{k+1}}{k}\bigg[e^{k\beta\mu_e}+e^{-k\beta\mu_e}\bigg]e^{-k\beta E_n^\pm}\notag\\
&=\frac{2eBV}{(2\pi)^2}\sum_{n=0}^\infty\sum^{\infty}_{k=1}\frac{(-1)^{k+1}}{k}\bigg[2\cosh{(k\beta\mu_e)}\bigg]\int_0^\infty dp_z\,e^{-k\beta E_n^\pm}.
\end{align}
Using the general definition of Bessel function:
\begin{align}
K_\nu(\beta m)=\frac{\sqrt{\pi}}{\Gamma({\nu-1/2})}\frac{1}{m}\left(\frac{\beta}{2m}\right)^{\nu-1}\int_0^\infty\,dp\,p^{2\nu-2}e^{-\beta E} \,\,\,\,\,\,\,\mathrm{for}\,\nu>1/2,
\end{align}
the integral over $dp_z$ can be written as
\begin{align}
\int_0^\infty dp_z\,e^{-k\beta E_n^\pm}&=\frac{\Gamma{(1/2)}}{\sqrt{\pi}}\sqrt{\tilde{m}^2_\pm+2eBn}\,\,K_1\!\!\left({k\sqrt{\tilde{m}^2_\pm+2eBn}}/{T}\right)\notag\\&=\sqrt{\tilde{m}^2_\pm+2eBn}\,\,K_1\!\!\left({k\sqrt{\tilde{m}^2_\pm+2eBn}}/{T}\right).
\end{align}
In this case, the partition function becomes
\begin{align}
\ln\mathcal{Z}_{tot}&=\frac{2eBV}{(2\pi)^2}\sum_{n=0}^\infty\sum^{\infty}_{k=1}\frac{(-1)^{k+1}}{k}\bigg[2\cosh{(k\beta\mu_e)}\bigg]\sqrt{\tilde{m}^2_\pm+2eBn}\,\,K_1({k\sqrt{\tilde{m}^2_\pm+2eBn}}/{T})\notag\\
&=\frac{2eBTV}{(2\pi)^2}\sum^{\infty}_{k=1}\frac{(-1)^{k+1}}{k^2}\bigg[2\cosh{(k\beta\mu_e)}\bigg]\sum_{n=0}^\infty W^\pm_1(n),
\end{align}
where we introduce the function $W^\pm_1(n)$ as follows
\begin{align}
W^\pm_1(n)\equiv\frac{k\sqrt{\tilde{m}^2_\pm+2eBn}}{T}\,\,K_1\!\!\left({k\sqrt{\tilde{m}^2_\pm+2eBn}}/{T}\right).
\end{align}

Considering the Euler-Maclaurin formula to replace the sum over Landau levels, we have
\begin{align}
\sum^{\infty}_{n=0}W^\pm_1(n)=\int^\infty_0\!\!dn\,W^\pm_1(n)&+\frac{1}{2}\bigg[W^\pm_1(\infty)+W^\pm_1(0)\bigg]\notag\\
&\qquad+\frac{1}{12}\bigg[\left.\frac{\partial W^\pm_1}{\partial n}\right|_{\infty}-\left.\frac{\partial W^\pm_1}{\partial n}\right|_{0}\bigg]+R,
\end{align}
where $R$ is the error remainder which is defined by integrals over Bernoulli polynomials which is small and can be neglected~[\cite{Elze:1980er}]. Using the properties of Bessel function we have
\begin{align}
&\frac{\partial W^\pm_1}{\partial n}=-\frac{k^2eB}{T^2}K_0\left({k\sqrt{\tilde{m}^2_\pm+2eBn}}/{T}\right),\qquad W^\pm_1(\infty)=0,\\
&\int^\infty_a\!\!dx\,x^2K_1(x)=a^2K_2(a),
\end{align}
then we obtain
\begin{align}
\sum^{\infty}_{n=0}W^\pm_1(n)
&=\left(\frac{T^2}{k^2eB}\right)\left[\left(\frac{k\tilde{m}_\pm}{T}\right)^2K_2(k\tilde m_\pm/T)\right]+\frac{1}{2}\left[\left(\frac{k\tilde{m}_\pm}{T}\right)K_1(k\tilde m_\pm/T)\right]\notag\\
&\qquad+\frac{1}{12}\left[\left(\frac{k^2eB}{T^2}\right)K_0(k\tilde m_\pm/T)\right].
\end{align}
Replacing the sum over Landau levels by the integral, the partition function becomes
\begin{align}
\ln\mathcal{Z}_{tot}=\ln\mathcal{Z}_{free}+\ln\mathcal{Z}_B\,,
\end{align}
where we define the partition functions as  
\begin{align}
 \label{FreePart}&\ln\mathcal{Z}_{free}=\frac{T^3V}{2\pi^2}\left[2\cosh{\left(\frac{\eta_{e}}{T}\right)}\right]\sum_{i=\pm}x_i^2K_2\left(x_i\right)\,,\qquad x_i=\frac{\tilde{m}_i}{T}\\
 \label{MagPart}&\ln\mathcal{Z}_B=\frac{eBTV}{2\pi^2}\left[2\cosh{\left(\frac{\eta_{e}}{T}\right)}\right]\sum_{i=\pm}\bigg[\frac{x_i}{2}K_1\left(x_i\right)+\frac{b_0}{12}K_0\left(x_i\right)\bigg]\,.
\end{align}
The partition function $\ln(\mathcal{Z}_{free})$ in Eq.~(\ref{FreePart}) represents the general form of the Fermi partition function for $e^\pm$ with "effective mass" $\tilde{m}_\pm$ in our system. When the magnetic field $B=0$ the function $\ln(\mathcal{Z}_{free})$ will go back to the general form of the Fermi partition function without the external field. The partition function $\ln\mathcal{Z}_B$ gives us the partition with magnetic field effect to the order  $\mathcal{O}(eB)$ and  $\mathcal{O}(eB)^2$.


In the temperature domain $ 200\,\mathrm{keV} > T > 20$ keV, we have $m_e\gg T$, and it suffices to consider the Boltzmann limit of the quantum distributions. Considering the Boltzmann approximation for non-relativistic electrons and positrons we can rewrite Eq.~(\ref{FreePart}) - Eq.~(\ref{MagPart}) and obtain
\begin{align}
 \label{lnZ}
&\ln\mathcal{Z}_{tot}\!=\!\frac{T^3V}{2\pi^2}\left[2\cosh\left(\frac{\eta_{e}}{T}\right)\right]\sum_{i=\pm}\left\{x_i^{2} K_2\left(x_i\right)+\frac{b_0}{2}x_iK_1\left(x_i\right)+\frac{b^2_0}{12}K_0\left(x_i\right)\right\}.
\end{align}
Given the partition function Eq.~(\ref{lnZ}), we can explore the chemical potential and magnetization of $e^\pm$ plasma in the early Universe
under the hypothesis of charge neutrality and entropy conservation.

\subsubsection{Electron chemical potential under magnetic field}
We explore the chemical potential of electron-positron plasma in a uniform magnetic field $B$ in the early Universe under the hypothesis of charge neutrality and entropy conservation. Considering the temperature after neutrino freeze-out, the charge neutrality condition can be written as
\begin{align}
 \label{density_proton}
 \left(n_{e}-n_{\bar{e}}\right)=n_{p}=X_p\,\left(\frac{n_{B}}{s_{\gamma,e}}\right)\,s_{\gamma,e},\qquad X_p\equiv\frac{n_p}{n_B}\,,
\end{align}
where $n_{p}$ and $n_B$ is the number density of protons and baryons respectively. Using the partition function Eq.~(\ref{lnZ}), the net number density of electrons in Boltzmann approximation can be written as
\begin{align}\label{NetElectron}
\left(n_e-n_{\bar e}\right)&=\frac{T}{V}\frac{\partial}{\partial \eta_{e}}\ln\mathcal{Z}_{tot}\notag\\
&=\frac{T^3}{2\pi^2}\left[2\sinh{(\eta_{e}/T)}\right]\sum_{i=\pm}\left[x_i^2K_2(x_i)+\frac{b_0}{2}x_i K_1(x_i)+\frac{b^2_0}{12}K_0(x_i)\right]\,.
\end{align}
Substituting Eq.~(\ref{NetElectron}) into the charge neutrality condition Eq.~(\ref{density_proton}), we can solve the chemical potential of electron $\eta_e/T$ numerically. We have
\begin{align}\label{ChemicalPotential}
\sinh{(\eta_{e}/T)}&=\frac{2\pi^2}{2T^3}\,\frac{X_p(n_B/s_{\gamma,e})s_{\gamma,e}}{\sum_{i=\pm}\left[x_i^2K_2(x_i)+\frac{b_0}{2}x_i K_1(x_i)+\frac{b^2_0}{12}K_0(x_i)\right]}\,,\\
&\longrightarrow\frac{2\pi^2n_p}{2T^3}\,\frac{X_p(n_B/s_{\gamma,e})s_{\gamma,e}}{2x^2K_2(x)},\qquad x=m_e/T,\qquad \mathrm{for}\,\,b_0=0\label{ChemiticalPotential_000}.
\end{align}
We see in Eq.~(\ref{ChemiticalPotential_000}) that for the case $b_0=0$, the chemical potential agrees with the free particle result in~[\cite{Grayson:2023flr}]. In Fig.~\ref{ChemicalPotential_B} we plot the chemical potential of electron as a function of temperature with different value of $b_0$. It shows that the chemical potential is not sensitive to the magnetic field because the small value of $10^{-3}>b_0>10^{-11}$ can be neglected in Eq.~(\ref{ChemicalPotential}).
%~~figure~~~~~~~~~~~~~~~~~~~~~~~~~~~~~~
\begin{figure}[ht]
\begin{center}
\includegraphics[width=\linewidth]{./plots/ChemicalPotential_new_survey}
\caption{The chemical potential of electron as a function of temperature in the magnetic field $b_0$ with $X_p=0.878$ and $n_B/n_\gamma=6.05\times10^{-10}$. The red dashed line represents the magnetic field $b_0=1.1\times10^{-11}$ and blue line labels the magnetic field $b_0=5.5\times10^{-3}$}
\label{ChemicalPotential_B}
\end{center}
\end{figure}
%~~~~~~~~~~~~~~~~~~~~~~~~~~~~~~~~~~~~~~~~

\subsubsection{Electron-positron magnetization}
%We consider the electron-positron plasma in the mean field approximation where the external field is representative of the \lq\lq bulk\rq\rq\ internal magnetization of the gas. Each particle is therefore responding to the averaged magnetic flux generated by its neighbors as well as any global external field contribution. 

Considering the magnetized electron-positron partition function Eq.~(\ref{lnZ}), it is convenient to introduce the dimensionless magnetization $\overline{\mathcal{M}}$ and the critical field $B_c$ as follows
\begin{align}
\label{Mdef}
\overline{\mathcal{M}}\equiv\frac{M}{\mathcal{B}_{c}}=\frac{1}{\mathcal{B}_{c}}\left(\frac{T}{V}\frac{\partial \ln\mathcal{Z}_{tot}}{\partial B}\right)\,\qquad \mathcal{B}_{c}=\frac{m_{e}^{2}}{e}\,.
\end{align}
Substituting the partition function Eq.~(\ref{lnZ}) into Eq.~(\ref{Mdef}), the total magnetization ${\overline{\mathcal M}}$ can be broken into the sum of spin parallel $\overline{\mathcal M}_{+}$ and spin anti-parallel $\overline{\mathcal M}_{-}$ magnetization. We have
\begin{align}\label{Magnetization}
&{\overline{\mathcal M}}={\overline{\mathcal M}_+}+{\overline{\mathcal M}_-},\\
&\overline{\mathcal M}_{\pm}=\frac{e^2T^{2}}{2\pi^2m_e^2}\left[2\cosh\left(\frac{\eta_{e}}{T}\right)\right]\left\{c_{1}(x_{\pm})K_1(x_i)+c_{0}K_0(x_\pm)\right\}\,,
\end{align}
where the coefficients are given by
\begin{align}
    c_{1}(x_{\pm}) &= \left[\frac{1}{2}-\left(\frac{1}{2}\pm\frac{g}{4}\right)\left(1+\frac{b^2_0}{12x^2_\pm}\right)\right]x_\pm\,,\qquad c_{0} = \left[\frac{1}{6}-\left(\frac{1}{4}\pm\frac{g}{8}\right)\right]b_0\,.
\end{align}
Substituting the chemical potential Eq.~(\ref{ChemicalPotential}) into Eq.~(\ref{Magnetization}), we can solve the magnetization numerically.
%~Figure~~~~~~~~~~~~~~~~~~~~~~~~~~~~~
\begin{figure}[ht]
    \centering
    \includegraphics[width=\textwidth]{./plots/Magnetization_Hc_new002.jpg}
    \caption{The magnetization $\overline{\cal M}=\mathcal{M}/\mathcal{B}_C$, with $g=2$, of the primordial $e^{+}e^{-}$ plasma is plotted as a function of temperature. The lower (solid red) and upper (solid blue) bounds for cosmic magnetic scale $b_{0}$ are included. The external magnetic field strength ${\cal B}/{\cal B}_{C}$ is also plotted in for lower (dashed red) and upper (dashed blue) bounds. The spin fugacity is set to unity.}
    \label{fig:magnet} 
\end{figure}
%~Figure~~~~~~~~~~~~~~~~~~~~~~~~~~~~~

In this thesis we focus on considering the case for $g=2$. In this case, the electron-positron magnetization can be written as 
\begin{align}\label{Magnetization_g2}
&{\overline{\mathcal M}}={\overline{\mathcal M}_+}+{\overline{\mathcal M}_-}\\
&{\overline{\cal M}}_{+}=\frac{e^{2}}{\pi^{2}}\frac{T^{2}}{m_{e}^{2}}\cosh{\frac{\eta_e}{T}}\left[\frac{1}{2}x_{+}K_{1}(x_{+})+\frac{b_{0}}{6}K_{0}(x_{+})\right]\,,\\
&{\overline{\cal M}}_{-}=-\frac{e^{2}}{\pi^{2}}\frac{T^{2}}{m_{e}^{2}}\cosh{\frac{\eta_e}{T}}\left[\left(\frac{1}{2}+\frac{b_{0}^{2}}{12x_{-}^{2}}\right)x_{-}K_{1}(x_{-})+\frac{b_{0}}{3}K_{0}(x_{-})\right]\,,
\end{align}
where $x_\pm$ are given by
\begin{align}
x_{+}=\frac{m_{e}}{T},\qquad   x_{-}=\sqrt{\frac{m_{e}^{2}}{T^{2}}+2b_{0}}
\end{align}
The discussion for the case $g\neq2$ can be found in~~[\cite{Andrew:2023abc}].

In Fig.~\ref{fig:magnet}, we present the magnetization Eq.~(\ref{Magnetization_g2}) for the case $g=2$ as a function of temperature. It shows that the magnetization depends on the magnetic scale $b_0$ and the $e^{+}e^{-}$ plasma possesses an overall paramagnetic property, resulting in a positive magnetization $\overline{\mathcal{M}}$. This paramagnetic property is contrary to the conventional assumption that matter-antimatter plasmas lack significant inherent magnetic responses. However, the magnetization never exceeds the external field under the parameters considered, which shows a lack of ferromagnetic behavior. As the Universe cooled, the dropping magnetization slowed at $T_{\mathrm{split}}=20.3$ keV, where positrons vanished. Thereafter the remaining electron density diluted with cosmic expansion.

In this section, we have explored the electron-positron plasma considering  external and self-magnetization fields  without spin potential $\eta_s/T\ll1$. However the nonzero spin potential $\eta_s\neq0$  would have an impact on the primordial $e^{+}e^{-}$ plasma. In general, the  magnetization is also a function of the spin potential $\eta_s$, and would be one important parameter that control the spin direction of primordial gas which allows for magnetization even in the absence of external magnetic fields. For further discussion see ~[\cite{Andrew:2023abc}].





\section{Temperature Dependence of the Neutron Lifespan}

Element production during BBN is influenced by several parameters, e.g. baryon to photon ratio $\eta_b$, number of neutrino species $N_\nu$, and neutron to proton ratio, as controlled by both the dynamics of neutron freeze-out at temperature $T_f\approx 0.8\,\mathrm{MeV}$ and neutron lifetime. Since about 200 seconds pass between neutron freeze-out and BBN at $T\approx0.07\,\mathrm{MeV}$, the neutron lifetime is one of the important parameter controlling BBN element yields~\cite{Pitrou:2018cgg}. However, the neutron decay in cosmic plasma medium has seemingly never been considered before. Therefore, here we study the Fermi suppression effects which lengthen the neutron lifetime in the early universe. The medium dependence of particle decay was recognized by Kuznetsova et al~\cite{Kuznetsova:2010pi}, we will use their method to study the plasma effects on neutron decay in the cosmic medium.

After freeze-out, the neutron abundance is subject to natural decay
\begin{align}\label{Ndec}
n\longrightarrow p+e+\overline{\nu}_e\;,
\end{align}
where the current experimental neutron lifetime quoted by the Particle Data Group~\cite{Patrignani:2016xqp} is $\tau_n^0=880.2\pm1.0\,\mathrm{sec}$. The latest measurement for neutron lifetime by using magnetogravitational trap is $877.7\pm0.7\,\mathrm{sec}$ \cite{Pattie:2018vsj}. In the standard Big Bang Nucleosynthesis, the neutron abundance when nucleosynthesis begins is given by~\cite{Pitrou:2018cgg}
\begin{align}
\label{Xn_abundance}
X_n(T_{BBN})=X_n^f\exp\left(-\frac{t_{BBN}-t_f}{\tau_n^0}\right)\approx0.13\;,
\end{align}
The normalizing neutron freeze-out yield $X_n^f$ 
\begin{align}
\label{Xn_abundance2}
X_n^f \equiv  \frac{n_n^f}{n_n^f+n_p^f}= \frac{n_n^f/n_p^f}{1+n_n^f/n_p^f}\;.
\end{align}
where $n_n^f$ and $n_p^f$ are freeze-out neutron and proton densities, respectively. The thermal equilibrium yield ratio is
\begin{align}
\label{Xn_abundance3}
 \frac{n_n^f}{n_p^f}= \exp\left(-Q/T_f\right)\;,\qquad Q=m_n-m_p\;,
\end{align}
assuming a instantanous freeze-out, depends on temperature $T_f$ at which neutrons decouple from the heat bath, and the neutron-proton mass difference (in medium). The values considered  are in the range $X_n^f=0.15\sim0.17$ \cite{Pitrou:2018cgg}. A dynamical approach to neutron freezeout is necessary to fully understand $X_n^f$, we hope to return to this challenge in the near  future.

Following freeze-out the neutron is subject to natural decay and normally the neutron lifetime in vacuum $\tau_n^0$ is used c.f. Eq.\,(\ref{Xn_abundance}) to calculate the neutron abundance resulting in the \lq desired\rq\ value $X_n(T_{BBN})\approx0.13$ when BBN starts. To improve precision a dynamically evolving neutron yield needs to be studied and for this purpose we explore here the neutron decay which occurs in  medium, not vacuum. This leads to  neutron lifespan dependence on temperature of the cosmic medium as the decay occurs for a particle emerged in plasma of electron/positron, neutrino/antineutrino, (and protons).

Two physical effects of the medium  influence the neutron lifetime in the early universe noticeably:
\begin{itemize}
\item Fermi suppression factors from the medium: 
During the temperature range $T_f\geqslant T\geqslant T_{BBN}$, electrons and neutrinos in the background plasma can reduce the neutron decay rate by Fermi suppression to the neutron decay rate. Furthermore, the neutrino background can still provide the suppression after electron/positron pair annihilation becomes nearly complete.
\item Photon reheating:
When $T\ll m_e$ the electron/positron annihilation occurs, the entropy from $e^\pm$ is fed into photons, leading to photon reheating. The already decoupled (frozen-out) neutrinos remain undisturbed. Therefore, after annihilation we have two different temperatures in cosmic plasma: neutrino temperature $T_\nu$ and the photon and proton temperature $T$ respectively.
\end{itemize}
These two effect will be included in the following exploration of the neutron lifetime in the early universe as a function of $T$. We show how these effects alter the neutron lifespan and obtain the modification of the neutron yield at the time of BBN. Yet another effect was considered by Kuznetsova et al~\cite{Kuznetsova:2010pi} which is due to time dilation originating in particle thermal motion. In our case for neutrons with $T/m<10^{-3}$ this effect is negligible. Below we will explicitly assume that the neutron decay is studied in the neutron rest frame.

\subsection{Decay Rate in Medium}\label{Rate_Medium}

The invariant matrix element for the neutron decay Eq.\,(\ref{Ndec}) for non-relativistic neutron and proton is given by
\begin{align}
\langle|\mathcal{M}|^2\rangle\approx16\,G^2_FV^2_{ud}\,m_nm_p(1+3g^2_A)(1+RC)E_{\bar{\nu}}E_e,
\end{align}
where the Fermi constant is $G_F=1.1663787\times10^{-5}\,\mathrm{GeV}^{-2}$, $V_{ud}=0.97420$ is an element of the Cabibbo-Kobayashi-Maskawa (CKM) matrix~\cite{Czarnecki:2018okw,Marciano:2005ec,Czarnecki:2004cw}, and $g_A=1.2755$ is the axial current constant for the nucleons~\cite{Czarnecki:2018okw,Marciano:2014ria}. We also consider the total effect of all radiative corrections relative to muon decay that have not been absorbed into Fermi constant $G_F$. The most precise calculation of this correction~\cite{Marciano:2014ria,Marciano:2005ec} gives $(1+RC)=1.03886$. 

In the early universe the neutron decay rate in medium, at finite temperature can be written as~\cite{Kuznetsova:2010pi}
\begin{align}
\frac{1}{\tau^\prime_n}=\frac{1}{2m_n}\int&\frac{d^3p_{\bar{\nu}}}{(2\pi)^32E_{\bar{\nu}}}\frac{d^3p_p}{(2\pi)^32E_p}\frac{d^3p_e}{(2\pi)^32E_e}\notag\\
&(2\pi)^4\delta^4\left(p_n-p_p-p_e-p_{\bar{\nu}}\right)\langle|\mathcal{M}|^2\rangle\notag\\
&\big[1-f_p(p_p)\big]\big[1-f_e(p_e)\big]\big[1-f_{\bar{\nu}}(p_{\bar{\nu}})\big]\;,
\end{align}
where we consider this expression in the rest rest frame of neutron, {\it i.e.\/} $p_n=(m_n,0)$. The phase-space factors $(1-f_i)$ are Fermi suppression factors in the medium. The Fermi-Dirac distributions for electron and non-relativistic proton are given by
\begin{align}
&f_e=\frac{1}{e^{E_e/T}+1},\\
&f_p=e^{-E_p/T}=e^{-m_p/T}\,e^{-p_p^2/2m_pT}.
\end{align}
For neutrinos, after neutrino/antineutrino kinetic freeze-out they become free streaming particles. If we assume that kinetic freeze out occurs at some time $t_k$ and temperature $T_k$, then for $t>t_k$ the free streaming distribution function can be written as~\cite{Birrell:2012gg}
\begin{align}
f_{\bar{\nu}}=\frac{1}{\exp{\left(\sqrt{\frac{E^2-m_\nu^2}{T_\nu^2}+\frac{m^2_\nu}{T^2_k}}+\frac{\mu_{\bar{\nu}}}{T_k}\right)+1}},
\end{align}
for antineutrinos and we define the effective neutrino temperature $T_\nu$ as
\begin{align}
T_\nu\equiv\frac{a(t_k)}{a(t)}T_k.
\end{align}
In the following calculation, we assume the condition $T_k\gg\mu_{\bar{\nu}},\,m_\nu$, {\it i.e.\/} we consider the massless neutrino in plasma. Substituting the distributions into the decay rate formula and using the neutron rest frame, the decay rate can be written as 
\begin{align}
\label{Decay_rate_01}
\frac{1}{\tau_n^\prime}&=\frac{G^2_FQ^5V^2_{ud}}{2\pi^3}\,(1+3g^2_A)\,(1+RC)\\
&\times\int^1_{m_e/Q}d\xi\,\frac{\xi(1-\xi)^2}{\exp\left(-Q\xi/{T}\right)+1}\frac{\sqrt{\xi^2-(m_e/Q)^2}}{\exp\left(-Q(1-\xi)/T_\nu\right)+1},\notag
\end{align} 
where $Q$ was defined in Eq.\;(\ref{Xn_abundance3}) and we integrate using dimensionless variable $\xi=E_e/Q$. From Eq.(\ref{Decay_rate_01}), the decay rate in vacuum can be written as
\begin{align}
&\frac{1}{\tau_n^0}=\frac{G^2_Fm_e^5V^2_{ud}}{2\pi^3}(1+3g^2_A)\,(1+RC)\,f^\prime,
\end{align}
where the phase space factor $f^\prime$ is given by
\begin{align}
f^\prime&\equiv\left(\frac{Q}{m_e}\right)^5\int^1_{m_e/Q}d\xi\,{\xi(1-\xi)^2}\sqrt{\xi^2-(m_e/Q)^2}
\notag\\&=1.6360.
\end{align}
The phase space factor is also modified by the Coulomb correction between electron and proton, proton recoil, nucleon size correction etc. This has been studied by Wilkinson~\cite{Wilkinson:1982hu}, and the phase space factor is given by~\cite{Czarnecki:2018okw,Czarnecki:2004cw,Wilkinson:1982hu}
\begin{align}
f=1.6887.
\end{align}
These effect amount to adding the factor $\mathcal{F}$ to our calculation
\begin{align}
\mathcal{F}=\frac{f}{f^\prime}=1.0322,
\end{align}
then the neutron lifespan can be written as
\begin{align}
\tau^{\mathrm{Vacuum}}_n=\frac{\tau^0_n}{\mathcal{F}}=879.481\,\mathrm{sec},
\end{align}
which compare well to the experiment result $877.7\pm0.7\,\mathrm{sec}$ \cite{Pattie:2017vsj}. In the case of plasma medium, we do not expect that these effect (Coulomb correction between electron and proton, proton recoil, nucleon size correction etc) are modified in the cosmic plasma. Thus we adapt the factor into our calculation and the neutron decay rate in the cosmic plasma is given by
\begin{align}
\label{Decay_rate_02}
&\frac{1}{\tau_n^{\mathrm{Medium}}}=\frac{G^2_FQ^5V^2_{ud}}{2\pi^3}\,(1+3g^2_A)\,(1+RC)\,\mathcal{F}\\
&\times\int^1_{m_e/Q}d\xi\,\frac{\xi(1-\xi)^2}{\exp\left(-Q\xi/{T}\right)+1}\frac{\sqrt{\xi^2-(m_e/Q)^2}}{\exp\left(-Q(1-\xi)/T_\nu\right)+1}.\notag
\end{align}
From Eq.(\ref{Decay_rate_02}) we see that the neuron decay rate in the early universe depends on both the photon temperature $T$ and the neutrino effective temperature $T_\nu$.

%%%%%%%%%%%%%%%%%%%%%%%%%%%%%%%%%%%%%%%%%%%%%%%%%%%%%%%%%%%%%%%%%%

\subsection{Photon Reheating}\label{Reheating}

After neutrino free-out and when $m_e\gg T$, the $e^{\pm}$ becomes non-relativistic and annihilate. In this case, their entropy is transferred to the other relativistic particles still present in the cosmic plasma, {\it i.e.\/} photons, resulting in an increase in photon temperature as compared to the freestreaming neutrinos. From entropy conservation we have
\begin{align}
\label{Entropy}
\frac{2\pi}{45}g^s_\ast(T_k)T^3_kV_k+S_{\nu}(T_k)=\frac{2\pi}{45}g^s_\ast(T)T^3V+S_{\nu}(T),
\end{align}
where we use the subscripts $k$ to denote quantities for neutrino freezeout and $g^s_\ast$ counts the degree of freedom for relativistic species in early universe. After neutrino freezeout, their entropy is conserved independently and the temperature can be written as
\begin{align}
T_\nu\equiv\frac{a(t_k)}{a(t)}T_k=\left(\frac{V_k}{V}\right)^{1/3}T_k.
\end{align}
In this case, from entropy conservation, Eq.(\ref{Entropy}), we obtain
\begin{align}
\label{Neutrino_Photon}
T_\nu=\frac{T}{\kappa},\,\,\,\,\,\,\kappa\equiv\left[\frac{g^s_\ast(T_k)}{g^s_\ast(T)}\right]^{1/3}.
\end{align}
From Eq.(\ref{Neutrino_Photon}) the neutron decay rate in a heat bath can be written as
\begin{align}
\label{Decay_rate_03}
&\frac{1}{\tau_n^\mathrm{Medium}}= \frac{G^2_FQ^5V^2_{ud}}{2\pi^3}(1+3g^2_A)\,(1+RC)\,\mathcal{F}\\
&\times\int^1_{m_e/Q}d\xi\,\frac{\xi(1-\xi)^2}{\exp\left(-Q\xi/{T}\right)+1}\frac{\sqrt{\xi^2-(m_e/Q)^2}}{\exp\left(-Q(1-\xi)\kappa/T\right)+1}.\notag
\end{align}
In the high temperature regime, $T\gg Q$, the exponential terms in the Fermi distribution becomes $1$ and the decay rate is given by
\begin{align}
&\frac{1}{\tau_n^\mathrm{Medium}}=\frac{1}{4}\left(\frac{1}{\tau_n^\mathrm{Vacuum}}\right)\;,
\qquad
T\gg Q\;.
\end{align}
In Fig.~\ref{Decay_Rate}, we plot the the neutron lifetime $\tau^\mathrm{Medium}_n$ in plasma as a function of temperature. Fermi-suppression from electron and neutrino increases the neutron lifetime as compared to value in vacuum. At low temperature, $T<m_e$, most of the electrons and positrons have annihilated and the main Fermi-blocking comes from the cosmic neutrino background. In this regime, the neutron lifetime depends also on the neutrino temperature, $T_\nu$. For cold neutrinos $T_\nu<T$, the Fermi suppression is smaller than the hot one $T_\nu=T$. 
%%%%%%%%%%%%%%%%%%%%%%%%%%%%%%%%%%%%%%%%%%%%%%%%%%%%%%%%%%%%%%%%%
\begin{figure}[t]
\begin{center}
\includegraphics[width=\linewidth]{./plots/Neutron_Lifetime_001}
\caption{The neutron lifetime $\tau_n^\mathrm{Medium}$ in the cosmic plasma as a function of temperature. At high temperature $T=100\,\mathrm{MeV}$ the neutron lifetime is $3495\,\mathrm{sec}$ which is $3.974$ times larger than the lifetime in vacuum. At low temperature, $T<m_e$, the neutron lifetime depends also on the neutrino temperature, $T_\nu$, the effect is amplified in the insert.}
\label{Decay_Rate}
\end{center}
\end{figure}
%%%%%%%%%%%%%%%%%%%%%%%%%%%%%%%%%%%%%%%%%%%%%%%%%%%%%%%%%%%%%%%%%%
%%%%%%%%%%%%%%%%%%%%%%%%%%%%%%%%%%%%%%%%%%%%%%%%%%%%%%%%%%%%%%%%%%
\begin{figure}[h]
\begin{center}
\includegraphics[width=\linewidth]{./plots/Neutron_Abundance}
\caption{The neutron abundance ratio as a function of temperature. Consider the neutron freezeout temperature $T_f=0.08\mathrm{MeV}$ and the BBN temperature $T_{BBN}\approx0.07\mathrm{MeV}$, we find the abundance ratio ${X_n^{\mathrm{Meduim}}}/{X_n^{\mathrm{vacuum}}}=1.064$ at temperature $T_{BBN}$.}
\label{Neutron_Abundance}
\end{center}
\end{figure}
%%%%%%%%%%%%%%%%%%%%%%%%%%%%%%%%%%%%%%%%%%%%%%%%%%%%%%%
\subsection{Neutron Abundance}\label{Neutron}

After the neutron freezeout, the neutron abundance is only affected by the neutron decay. The neutron concentration can be written as
\begin{align}
\label{Abundance}
X_n=X_n^f\,\exp\bigg[-\int^t_{t_f}\frac{dt^\prime}{\tau_n}\bigg],
\end{align}
where we use the subscripts $f$ to denote quantities at neutron freezeout. Using Eq.(\ref{Decay_rate_03}) and Eq.(\ref{Abundance}), the neutron abundance ratio between plasma medium and vacuum is given by
\begin{align}
\label{Abundance_Ratio}
\frac{X_n^{\mathrm{Meduim}}}{X_n^{\mathrm{Vacuum}}}=\exp\bigg[-\int^t_{t_f}dt^\prime\left(\frac{1}{\tau^\prime_n}-\frac{1}{\tau^0_n}\right)\bigg].
\end{align}
In Fig.(\ref{Neutron_Abundance}), we plot the neutron abundance ratio as a function of temperature. Consider the neutron freezeout temperature $T_f=0.08\mathrm{MeV}$ and the BBN temperature $T_{BBN}\approx0.07\mathrm{MeV}$, we found that the ratio ${X_n^{\mathrm{Meduim}}}/{X_n^{\mathrm{Vacuum}}}=1.064$ at temperature $T_{BBN}$. Then from Eq.(\ref{Xn_abundance}) the neutron abundance in plasma medium is given by
\begin{align}
X_n^{\mathrm{Meduim}}=1.064\,X_n^{\mathrm{Vacuum}}\approx0.138.
\end{align}
In this case, the neutron abundance will increase about $6.4\%$ in the cosmic plasma which should affect the final abundances of the Helium-4 and other light elements in BBN.

%%%%%%%%%%%%%%%%%%%%%%%%%%%%%%%%%%%%%%%%%%%%%%%%%%%%%%%%%%%%%%%%%%

%\section{Conclusion and Discussion}\label{Disscusion}

One of the important parameters of standard BBN is the neutron lifetime, as it affects the neutron abundance after neutron freezeout at temperature $T_f\approx 0.8 \mathrm{MeV}$ and before the BBN $T\approx0.07 \mathrm{MeV}$. In the standard BBN model, it is necessary to have a neutron-to-proton ratio $n/p\approx1/7$ when BBN begins in order to explain the observed values of hydrogen and helium abundance~\cite{Pitrou:2018cgg}.

We have evaluated the effect of Fermi suppression on the neutron lifetime due to the background electron and neutrino plasma. We found that in medium the neutron lifetime is lengthened by upt to a factor 4 at a high temperature ($T>10$\,MeV). Our method should in principle also be considered in the study of medium modification of just about any of the BBN weak interaction rates, this remains a task for another day.

In the temperature range between neutron freeze-out just below $T=1$\;MeV and BBN conditions the effect of neutron lifespan is smaller but still noticeable. Near neutron freeze-out both decay electron and neutrino are blocked. However, after $e^\pm$ annihilation is nearly complete closer to BBN Fermi-blocking comes predominantly from the cosmic neutrino background and the neutron lifetime depends on the temperature $T_\nu<T$.

We found that the increased neutron lifetime results in an increased neutron abundance of ${X_n^{\mathrm{Meduim}}}/{X_n^{\mathrm{vacuum}}}=1.064$ at $T_{BBN}\approx0.07 \mathrm{MeV}$ {\it i.e.\/} we find a $6.4\%$ \emph{increase} in neutron abundance due to the medium effect at the time of BBN. We believe that this effect needs to be accounted for in the precision study of the final abundances of hydrogen, helium and other light elements produced in BBN.
%%%%%%%%%%%%%%%%%%%%%%%%
% Chapter from Andrew Steinmetz's dissertation
%~~~~~~~~~~~~~~~~~~~~~~~~~~~~~~~~~~~~~~~~~~~~~~~~~

% \subsection{ Electron-positron plasma in the early Universe}\label{section_electron}
% %In this section we will focus on the following:
% %\begin{itemize}
% %    \item Chemical potential of electron in early universe
% %    \item Electron-positron plasma in BBN (Damped sccreening)
% %    \item Electron-positron magnetization
% %%    \item Neutron Lifespan in magnetized electron/positron plasma.
% %\end{itemize}

% In the early Universe, after the neutrino freeze-out at $T\approx 2$\,MeV, the Universe is controlled by the electron-positron-photon plasma. In this section, we demonstrate the rich electron-positron plasma in the early Universe by examining the chemical potential $\mu_e$ in the charge-neutral and entropy-conserving Universe. We study the  microscope collision property of electron-positron plasma and explore the spin response of the electron-positron plasma to external and self-magnetization fields, thus developing methods for future detailed study.

%In this section, we will quantify the dynamical picture of $e^\pm$ plasma and show that the $e^+$ abundance can persist in early universe at relatively low temperature $T = 20$ keV which provide the dense $e^\pm$ plasma environment for the big-bang nucleosynthesis (BBN) in the early universe. 

%The role of electron-positron plasma has not received the appropriate attention in the days of precision big bang nucleosynthesis studies. The standard BBN model indicates that the synthesis of light elements typically takes place at temperatures around  $86\,\mathrm{keV}>T_{BBN}>50\,\mathrm{keV}$~\cite{Pitrou:2018cgg}. Within this temperature range there are millions of electron-positron pairs per charged nucleon, providing an electron-positron-rich plasma environment for nucleosynthesis. Furthermore, the electron-positron densities can reach millions of times normal atomic densities. The presence of  these $e\bar e$-pairs before and during BBN has been acknowledged by Wang, Bertulani and Balantekin~\cite{Wang:2010px} nearly a decade ago.





%On the other hand, the Universe today filled with magnetic fields at various scales and strengths both within galaxies and in deep extra-galactic space.It is currently unknown the origin for these magnetic fields today. In early Universe when temperature $T>20$ keV , we have dense $e^\pm$ plasma. The significant magnetic moments of electrons and positrons also provide opportunities to investigate spin magnetization process.


%~~~~~~~~~~~~~~~~~~~~~~~~~~~~~~~~~~~~~~~~~~~~~~~~~~~~~~~~~~~~~~~~~~~~~~~~~

% \subsubsection{Electron chemical potential in the early Universe}\index{electron chemical potential}
% In this section, we derive the dependence of electron chemical potential, and hence $e^\pm$ density, on the photon background temperature by employing the following physical principles:
% \begin{enumerate}
% \item Charge neutrality of the Universe:\index{charge neutrality}
% \begin{align}%\label{neutrality}
% n_e-n_{\overline{e}}=n_p-n_{\overline{p}}\approx\,n_p,
% \end{align}
% where $n_e$ and $n_{\overline{e}}$ denotes the number density of electron and positron.
% \item Neutrinos decouple (freeze-out) at a temperature $T_f\simeq 2$ MeV, after which they free stream through the Universe with an effective temperature~\cite{Birrell:2012gg}
% \begin{align}
% T_\nu(t)=T_f a(t_f)/a(t),
% \end{align}
%  where $a(t)$ is the FLRW Universe scale factor.
% \item Total comoving entropy is conserved. At $T\leq T_f$ the dominant contributors to entropy are photons, $e^\pm$, and neutrinos.
% In addition, after neutrino freeze-out, neutrino comoving entropy is independently conserved ~\cite{Birrell:2012gg}. This  implies that the combined comoving entropy in $\gamma$, $e^\pm$ is also conserved for $T_\gamma\leq T_f$.
% \end{enumerate}

% Motivated by the fact that comoving entropy in $\gamma$, $e^\pm$ is conserved after neutrino freeze-out, we rewrite the charge neutrality condition, Eq.(\ref{neutrality}) in the form
% \begin{align}%\label{charge_neutral_cond2}
% n_e-n_{\overline{e}}=X_p\frac{n_B}{s_{\gamma,e,\overline{e}}} s_{\gamma,e,\overline{e}},\qquad X_p\equiv\frac{n_p}{n_B},
% \end{align}
% where $n_B$ is the number density of baryons, and $s_{\gamma,e,\overline{e}}$ is the combined entropy density in photons, electrons, and positrons. During the Universe expansion, the comoving entropy and baryon number are conserved quantities, hence the ratio $n_B/s_{\gamma,e,\overline{e}}$ is conserved. We have
% \begin{align}
% \frac{n_B}{s_{\gamma,e,\overline{e}}}=\left(\frac{n_B}{s_{\gamma,e,\overline{e}}}\right)_{t_0}\!\!\!\!=\left(\frac{n_B}{s_{\gamma}}\right)_{t_0}\!\!\!\!=\left(\frac{n_B}{n_\gamma}\right)_{t_0}\left(\frac{n_\gamma}{s_{\gamma}}\right)_{t_0},
% \end{align}
% where the subscript $t_0$ denotes the present day value, and the second equality is obtained by observing that the present day $e^\pm$-entropy density is negligible compared to the photon entropy density. We can evaluate the ratio by giving the present day baryon-to-photon ratio: $n_B/n_\gamma= 6.05\times10^{-10}$(CMB) ~\cite{ParticleDataGroup:2022pth} and the entropy per particle for a massless boson:$(s/n)_{\mathrm{boson}}\approx 3.602$~\cite{Letessier:2002ony}.

% The total entropy density of photons and electron/positron can be written as
% \begin{align}%\label{entropy_per_baryon}
% s_{\gamma,e,\overline{e}}=\frac{2\pi^2}{45}g_\gamma\,T_\gamma^3+\frac{\rho_{e,\overline{e}}+P_{e,\overline{e}}}{T_\gamma}-\frac{\mu_e}{T_\gamma}(n_e-n_{\overline{e}}),
% \end{align}
% where $ \rho_{e,\overline{e}}=\rho_{e}+\rho_{\overline{e}}$ and $P_{e,\overline{e}}=P_{e}+P_{\overline{e}}$ are the total energy density and pressure of electrons/positron respectively.
% The energy density and pressure in electrons and positrons are given by
% \begin{align}\label{rho_e}
% \frac{\rho_{e,\overline{e}}}{T_\gamma^4}=\frac{g_e}{2\pi^2}M_e^4 \bigg[&\int_{1}^\infty \frac{ u^2\sqrt{ u^2-1} du}{\exp(M_e u-b_e)+1}+\int_{1}^\infty \frac{ u^2\sqrt{ u^2-1} du}{\exp(M_e u+b_e)+1}\bigg]\,,
% \end{align}
% and
% \begin{align}\label{P_e}
% \frac{P_{e,\overline{e}}}{T_\gamma^4}=\frac{g_e}{6\pi^2}M_e^4\bigg[&\int_{1}^\infty   \frac{(u^2-1)^{3/2} du}{\exp(M_e u-b_e)+1}+\int_{1}^\infty   \frac{(u^2-1)^{3/2} du}{\exp(M_e u+b_e)+1}\bigg],
% \end{align}
% where we introduce the dimensionless variables as follows: 
% \begin{align}%\label{Variables}
% u=\frac{E}{m_e},\qquad M_e=\frac{m_e}{T_\gamma},\qquad b_e=\frac{\mu_e}{T_\gamma}.
% \end{align}

% By incorporating Eq.(\ref{charge_neutral_cond2}) and Eq.(\ref{entropy_per_baryon}), the charge neutrality condition can be expressed as
% \begin{align}%\label{charge_neutral_cond3}
% &\left[1+X_p\left(\frac{n_B}{n_\gamma}\right)_{t_0}\left(\frac{n_\gamma}{s_{\gamma}}\right)_{t_0}\frac{\mu_e}{T_\gamma}\right]\frac{n_e-n_{\overline{e}}}{T_\gamma^3}\notag\\
% &\qquad\qquad\qquad=X_p\left(\frac{n_B}{n_\gamma}\right)_{t_0}\left(\frac{n_\gamma}{s_{\gamma}}\right)_{t_0} \left(\frac{2\pi^2}{45}g_\gamma+\frac{\rho_{e,\overline{e}}+P_{e,\overline{e}}}{T_\gamma^4}\right).
% \end{align}
% Using the Fermi distribution, the number density of electrons over positrons in the early Universe is given by
% \begin{align}%\label{ee_density}
% n_e-n_{\overline{e}}&=\frac{g_e}{2\pi^2}\left[\int_0^\infty\frac{p^2dp}{\exp{\left((E-\mu_e)\right)/T_\gamma}+1}\right.\left.-\int_0^\infty\frac{p^2dp}{\exp{\left((E+\mu_e)/T_\gamma\right)}+1}\right]\notag\\
% &=\frac{g_e}{2\pi^2}{T_\gamma^3}\tanh(b_e)M_e^3\int_{1}^\infty \!\!\!\!\frac{  u \sqrt{u^2-1} du}{1+\cosh(M_eu)/\cosh(b_e)}.
% \end{align}
% Substituting Eq.(\ref{ee_density}) into Eq.(\ref{charge_neutral_cond3}) and giving the value of $X_p$, the charge neutrality condition can be solved to determine $\mu_e/T_\gamma$ as a function of $M_e$ and $T_\gamma$. 
% %Fig~~~~~~~~~~~~~~~~~~~~~~~~~~~~~~~~~~~~~~~~~~~~~~~~~~~~~
% \begin{figure}[ht]
% \begin{center}
% \includegraphics[width=\linewidth]{./plots/May152023_EPDensity_Chemical}
% \caption{\cccite{Grayson:2023flr}, adapted from Ref.~\cite{Grayson:2023flr} and thesis of C.T.Yang \cite{Yang:2024ret}. Left axis: The chemical potential of an electron as a function of photon temperature $T=T_\gamma$ with $X_p=0.878$ and $n_B/n_\gamma=6.05\times10^{-10}$. Right axis: the ratio of electron(positron) number density to baryon density as a function of temperature. The blue solid line is the electron density, the red dashed line is the positron density, and the green dotted line is the number density with $\mu_e=0$. We found that when electron chemical potential $\mu_e\approx T=0.02\,\mathrm{MeV}$ the positron density decreases because of the annihilation.}
% %\label{BBN_Electron}
% \end{center}
% \end{figure}
% %~~~~~~~~~~~~~~~~~~~~~~~~~~~~~~~~~~~~~~~~~~~~~~~~~~~~~

% In Fig.~\ref{BBN_Electron} (left axis) we solve Eq.(\ref{charge_neutral_cond3}) numerically and plot the electron chemical potential as a function of temperature with the following parameters: proton concentration $X_p=0.878$ from 
%  observation~\cite{ParticleDataGroup:2022pth} and  $n_B/n_\gamma=6.05\times10^{-10}$ from CMB. We can see the value of chemical potential is comparatively small $\mu_e/T\approx10^{-6}\sim10^{-7}$ during the BBN temperature range, implying an equal number of electrons and positrons in plasma. From the ratio of electron (positron) number density to baryon density in Fig.~\ref{BBN_Electron} (right axis) we can see that during the accepted BBN temperature range the Universe was filled with an electron-positron rich plasma.
% For example when the temperature is around $T=70\,\mathrm{keV}$ the density of electrons and positrons is comparatively large in the early Universe $n_{e^\pm}\approx10^7\,n_B$. Later when the temperature is around $T=20.3\,\mathrm{keV}$, the positron density decreases, leading to the transformation of the pair plasma to an electron-proton plasma.
% %~~~~~~~~~~~~~~~~~~~~~~~~~~~~~~~~~~~~~~~~~~~~~~~~~
% \subsubsection{Microscope damping rate of electron-positron plasma}\label{relax}
% In electron-positron plasma, the major reactions between photons and $e^+e^-$ pairs are inverse Compton scattering, M{\o}ller scattering, and Bhabha scattering:
% \begin{align}
% &e^\pm+\gamma\longrightarrow e^\pm+\gamma,\qquad e^\pm+e^\pm\longrightarrow e^\pm+e^\pm,\qquad e^\pm+e^\mp\longrightarrow e^\pm+e^\mp.
% \end{align}
% The general formula for thermal reaction rate per volume is discussed in~\cite{Letessier:2002ony} (Eq.(17.16), Chapter 17). For inverse Compton scattering we have
% \begin{align}
% R_{e^{\pm}\gamma}=\frac{g_eg_\gamma}{16\left(2\pi\right)^5}T\int_{m_e^2}^\infty\!\!\!\!ds\frac{K_1(\sqrt{s}/T)}{\sqrt{s}}\int^0_{-(s-m_e^2)^2/s}\!\!\!\!\!\!\!\!\!\!\!\!\!\!\!\!dt\, |M_{e^{\pm}\gamma}|^2,
% \end{align} 
% and for M{\o}ller and Bhabha reactions we have
% \begin{align}
% &R_{e^\pm e^\pm}=\frac{g_eg_e}{16\left(2\pi\right)^5}T\!\!\int_{4m_e^2}^\infty\!\!\!\!ds\frac{K_1(\sqrt{s}/T)}{\sqrt{s}}\int^0_{-(s-4m_e^2)}\!\!\!\!\!\!\!\!\!\!\!\!\!\!\!\!dt\,|M_{e^\pm e^\pm}|^2,\\
% &R_{e^\pm e^\mp}=\frac{g_eg_e}{16\left(2\pi\right)^5}T\!\!\int_{4m_e^2}^\infty\!\!\!\!ds\frac{K_1(\sqrt{s}/T)}{\sqrt{s}}\int^0_{-(s-4m_e^2)}\!\!\!\!\!\!\!\!\!\!\!\!\!\!\!\!dt\,|M_{e^\pm e^\mp}|^2,
% \end{align}
% where $g_i$ is the degeneracy of particle $i$, $|M|^2$ is the matrix element for a given reaction, $K_1$ is the Bessel function of order $1$, and $s,t,u$ are Mandelstam variables. The leading order matrix element associated with inverse Compton scattering can be expressed in the Mandelstam variables~\cite{Kuznetsova:2011wt,Kuznetsova:2009bq} we have\index{inverse Compton scattering }
% \begin{align}
% |M_{e^\pm\gamma}|^2\!=32 \pi^2\alpha^2\bigg[&4\left(\frac{m_e^2}{m_e^2-s}+\frac{m_e^2}{m_e^2-u}\right)^2\notag\\
% &\qquad\qquad-\frac{4m_e^2}{m_e^2-s}-\frac{4m_e^2}{m_e^2-u} -
%  \frac{m_e^2-u}{m_e^2-s} -\frac{m_e^2-s}{m_e^2-u}\bigg],
% \end{align}
% and for M{\o}ller and Bhabha scattering we have \index{M{\o}ller scattering}\index{Bhabha scattering}
% \begin{align}
% |M_{e^{\pm}e^{\pm}}|^{2}\!=64\pi^{2}\alpha^{2}\bigg[&
% \frac{s^{2}+u^{2}+8m_e^{2}(t-m_e^{2})}{2(t-m^2_{\gamma})^{2}}\notag\\
% &\quad+\frac{{s^{2}+t^{2}}+8m_e^{2}
% (u-m_e^{2})}{2(u-m_{\gamma}^2)^{2}} + \frac{\left( {s}-2m_e^{2}\right)\left({s}-6m_e^{2}\right)}
% {(t-m_{\gamma}^2)(u-m_{\gamma}^2)} \bigg],
% \end{align}
% and
% \begin{align}
% |M_{e^\pm e^\mp}|^{2}=64\pi^{2}\alpha^{2}
% \bigg[&\frac{s^{2}+u^{2}+8m_e^{2}(t-m_e^{2})}{2(t-m^2_{\gamma})^{2}}\notag\\
% &\quad+\frac{u^{2}+t^{2}+8m_e^{2}
% (s-m_e^{2})}{2(s-m^2_{\gamma})^{2}}  +   \frac{\left({u}-2m_e^{2}\right)\left({u}-6m_e^{2}\right)}
%    {(t-m^2_{\gamma})(s-m^2_{\gamma})} \bigg],
% \label{M_fi_b}
% \end{align}
% where we introduce the photon mass $m_\gamma$ to account the plasma effect and avoid singularity in reaction matrix elements. 

% The photon mass $m_\gamma$ in plasma is equal to the plasma frequency $\omega_p$, where we have~\cite{Kislinger:1975uy}\index{photon mass}
% \begin{align}
% m^2_\gamma=\omega^2_{p}=8\pi\alpha\int\frac{d^3p_e}{(2\pi)^3}\left(1-\frac{p_e^2}{3E_e^2}\right)\frac{f_e+f_{\bar e}}{E_e},
% \end{align}
% where $E_e=\sqrt{p_e^2+m^2_e}$. In the BBN temperature range $86\,\mathrm{keV}>T_{BBN}>50\,\mathrm{keV}$ we have $m_e\gg T$ and considering the nonrelativistic limit for electron-positron plasma, we obtain
% \begin{align}
% m^2_\gamma=\frac{4\pi\alpha}{2m_e}\left(\frac{2m_eT}{\pi}\right)^{3/2}e^{-m_e/T}\cosh\left(\frac{\mu_e}{T}\right).
% \end{align}
% In the BBN temperature range, we have $\mu_e/T\ll1$, which implies the equal number of electrons and positrons in plasma.

% To discuss the collisional plasma by the linear response theory, it is convenient to define the average relaxation rate for the electron-positron plasma as follows:\index{electron-positron plasma!damping rate}
% \begin{align}\label{Kappa}
% \kappa=\frac{R_{e^\pm e^\pm}+R_{e^\pm e^\mp}+R_{e^\pm\gamma}}{\sqrt{n_{e^-}n_{e^+}}}\approx\frac{R_{e^\pm e^\pm}+R_{e^\pm e^\mp}}{\sqrt{n_{e^-}n_{e^+}}},
% \end{align}
% where the density function ${\sqrt{n_{e^-}n_{e^+}}}$ in the Boltzmann limit is given by
% \begin{align}
% {\sqrt{n_{e^-}n_{e^+}}}=\frac{g_e}{2\pi^3}T^3\left(\frac{m_e}{T}\right)^2K_2(m_e/T).
% \end{align}
% In Fig.~\ref{RelaxationRate_fig}, we show the reaction rates for M{\o}ller reaction, Bhabha reaction, and inverse Compton scattering as a function of temperature. For temperatures $T>12.0$ keV, the dominant reactions in plasma are M{\o}ller and Bhabha scatterings between electrons and positrons. Thus in the BBN temperature range, we can neglect the inverse Compton scattering. The total relaxation rate is approximately constant $\kappa=10\sim12$ keV during the BBN. For $T<20.3$ keV the relaxation rate $\kappa$ decreases rapidly because of positron annihilation. At this temperature, the composition of plasma begins to change from an electron-positron plasma to an electron-baryon plasma.
% %~~~~Figure~~~~~~~~~~~~~~~~~~~~~~~~~~~
% \begin{figure}[ht]
% \begin{center}
% %\includegraphics[width=0.95\linewidth]{KappaRateToT_May082023}
% \includegraphics[width=\linewidth]{./plots/May152023Kappa_EPPlasma}
% \caption{\cccite{Grayson:2023flr}, adapted from Ref.~\cite{Grayson:2023flr} and thesis of C.T.Yang \cite{Yang:2024ret}. The relaxation rate $\kappa$ as a function of temperature in nonrelativistic electron-positron plasma from \cite{Grayson:2023flr}. For comparison, we show  reaction rates  for M{\o}ller reaction $e^-+e^-\to e^-+e^-$ (blue line), Bhabha reaction $e^-+e^+\to e^-+e^+$ (red line), and inverse Compton scattering $e^-+\gamma\to e^-+\gamma$ (green line) respectively. It shows that the dominant reactions during BBN are the M{\o}ller and Bhabha scatterings between electrons and positrons. The total relaxation rate Eq.(\ref{Kappa}) is shown in the black line. It shows that we have $\kappa=10\sim12$ keV during the BBN temperature range. For comparison, the Debye mass $m_D=\omega_{p}\sqrt{m_e/T}$(purple line) is shown as a function of temperature.
% }
% \label{RelaxationRate_fig}
% \end{center}
% \end{figure}
% %~~~~Figure~~~~~~~~~~~~~~~~~~~~~~~~~~~


% \paragraph{From static to damped dynamic screening}

% At present, the observation of light element (e.g. D, $^3$He, $^4$He, and $^7$Li) abundances produced in Big-Bang nucleosynthesis (BBN) offers a reliable probe of the early Universe before the recombination. Much effort of the BBN study is currently being made to reconcile the discrepancies and tensions between theoretical predictions and observations of light element abundances, e.g. $^7$Li problem ~\cite{Pitrou:2018cgg,Fields:2011zzb}.
% Current models assume that the Universe was essentially void of anything but reacting light nucleons and electrons needed to keep the local baryon density charge-neutral, a situation similar to the experimental environment where empirical nuclear reaction rates are obtained.

% The electron-positron plasma influences light element abundances through electromagnetic screening of the nuclear potential. The electron cloud surrounding the charge of an ion screens other nuclear charges far from its own radius and reduces the Coulomb barrier. In nuclear reactions, the reduction of Coulomb barrier makes the penetration probability easier and enhance the thermonuclear reaction rates. In this case, the modification of the nuclei interaction due to the plasma screening effect may plays a key role in the formation of light element in the BBN. 

% The enhancement factor of thermonuclear reaction rates and screening potential are calculated by Salpeter in 1954~\cite{Salpeter:1954nc}, which describes the static screening effects for the thermonuclear reactions. In an isotropic and homogeneous plasma the Coulomb potential of a point-like particle with charge $Ze$ at rest is modified into~\cite{Salpeter:1954nc}
% \begin{align}
% \phi_\text{stat}(r)=\frac{Ze}{4\pi\epsilon_0 r}e^{-m_Dr},
% \end{align}
% where $m_D$ is the Debye mass. After that it has been exploited widely in BBN for static screening ~\cite{1969ApJ...155..183S,Famiano:2016hhs}. 

% Subsequently, the study of dynamical screening for moving ions has been taken into account~\cite{1988ApJ...331..565C,Gruzinov:1997as,Hwang:2021kno}. When a test charge moves with a velocity that is enough to react with the background charge in plasma, the Coulomb potential is modified by the dynamical effect. However, the applications focus on the weakly interacting electron-positron plasma only. 

% In our separate work~\cite{Grayson:2023flr} we use the linear response theory adapted by C.Grayson to to describe the inter nuclear potential in electron-positron plasma during BBN. We improve the prior efforts by evaluation and inclusion of the collision damping rate due to scattering in the dense plasma medium and provide an approximate analytic formula that can be readily used to estimate the effect of screening on internuclear potential. For comprehensive discussion and the application of the damped dynamic screening see~\cite{Grayson:2023flr}.










% %~~~~~~~~~~~~~~~~~~~~~~~~~~~~~~~~~~~~~~~~~~~~~~~~~~~~
\subsubsection{Magnetization of the electron-positron plasma (ANDREW WORKING HERE)}

In the present-day Universe, we have magnetic fields~\cite{Giovannini:2003yn,Kronberg:1993vk} at various scales and strengths both within galaxies and in deep extra-galactic space far away from matter sources. Current observations suggest the upper and lower bounds for the Extra-Galactic Magnetic Field (EGMF) are given by~\cite{neronov2010evidence,taylor2011extragalactic,pshirkov2015new,jedamzik2019stringent,vernstrom2021discovery}
\begin{align}
    \label{egmf}
    10^{-8}{\mathrm G}>B_{\mathrm{EGFM}}>10^{-16}{\mathrm G}\,.
\end{align}
The origin for EGMF today is currently unknown; different models are considered in lectures~\cite{Widrow:2011hs,Vazza:2021vwy}. In our work~\cite{Rafelski:2023emw}, we investigate the hypothesis that the observed EGMF are primordial in nature, predating even the recombination
epoch. Under this hypothesis, the first best candidate is the electron-positron plasma. This is because for the temperature range $ 200\,\mathrm{keV} > T > 20$ keV, we still have relatively large quantity of both $e^\pm$ in the the early Universe plasma. In addition, electrons and positrons have the largest magnetic moments in nature, are likely to have been magnetized in the early Universe due to spin orientation. These  provide the possibility origins for a primordial magnetic field.

As the Universe undergoes the isentropic expansion,  the temperature gradually decreases as $T\propto1/a(t)$, where $a(t)$ represents the scale factor. The assumption is made that the magnetic flux is conserved over comoving surfaces, implying that the primordial relic field is expected to dilute as $B\propto1/a(t)^{2}$~\cite{Rafelski:2023emw}. Combining these cosmological redshift relations, we can introduce a dimensionless cosmic magnetic scale that remains unchanged during the evolution of the Universe 
\begin{align}
    %\label{tbscale}
    b \equiv\frac{e{B}}{T^{2}}=\left(\frac{e{B}}{T^{2}}\right)_{t_0}=b_0={\rm\ const.}\qquad10^{-3}>b_{0}>10^{-11}\,.
\end{align}
The upper and lower bounds for $b_0$ are estimated by using the present day EGMF observations Eq.~(\ref{egmf}) and the present CMB temperature $T_{0}=2.7\,\mathrm{K}\approx2.3\times10^{-4}$ eV~\cite{aghanim2018planck}.
As $b_0$ is a constant of expansion, this means the contemporary small bounded values of may have once represented large magnetic fields in the early Universe and require detailed study in a different epoch of the Universe. Therefore, correctly describing the dynamics of this $e^{\pm}$ plasma is of interest when considering the origin of extra-galactic magnetic fields (EGMF). 

In the following,  we will demonstrate that fundamental quantum statistical analysis can lead to further insights on the behavior of magnetized plasma, and show that the $e^\pm$ plasma is overall paramagnetic and yields a positive overall magnetization, which
is contrary to the traditional assumption that matter-antimatter plasma lack significant magnetic responses. For more detailed discussion  of electron-positron plasma magnetization, please see~\cite{Steinmetz:2023nsc}.



\paragraph{Electron-positron partition function}
To study the statistical behavior of the $e^\pm$ system in a magnetic field, we utilize the general Fermion partition function~\cite{Elze:1980er}
\begin{align}
 \label{PartFunc} \ln\mathcal{Z}=\sum_{\alpha}\ln\left(1+e^{-\beta(E-\eta)}\right)\,,
\end{align}
where $\beta=1/T$, $\alpha$ is the set of all quantum numbers in the system, and $\eta$ is the generalized chemical potential. In the case of a magnetized $e^{\pm}$ system, we consider it as a system of four quantum species: Particles and antiparticles, and spin aligned and anti-aligned. Taken together, we consider a system where electrons and positrons can be spin aligned or anti-aligned with the magnetic field $B$ and the partition function of the system can be written as\index{electron-positron plasma! partition function}
\begin{align}\label{PartFuncB}
%&\ln\mathcal{Z}_{tot}=&\frac{2eBV}{(2\pi)^2}\sum_{\sigma}^{\pm1}\sum_{s}^{\pm1/2}\sum_{n=0}^\infty\int^\infty_{0}dp_z\left[\ln\left(1+\Upsilon_{\sigma}^{s}(x)e^{-\beta E_{n}^{s}}\right)\right]\,\\
\ln\mathcal{Z}_{tot}=\frac{2eBV}{(2\pi)^2}\sum_{\sigma}^{\pm1}\sum_{s}^{\pm1/2}\sum_{n=0}^\infty\int^\infty_{0}dp_z\left[\ln\left(1+\Upsilon(x)e^{(\sigma\eta_{e}+s\eta_s)/T}e^{-\beta E_{n}^{s}}\right)\right]\,,
\end{align}
where $n$ is the principle quantum number for the Landau levels. The parameter $\eta_{e}$ is the electron chemical potential and $\eta_s$ is the spin chemical potential~\cite{Steinmetz:2023nsc}. The parameter $\Upsilon(x)$ is the fugacity of the Fermi gas. In this thesis we will focus on the case $\Upsilon(x)=1$ and $\eta_s=0$ , 
we leave the general case $\Upsilon(x)\neq1$ and $\eta_s\neq0$ for future work.



In general, $\Upsilon=1$ represents the maximum entropy and corresponds to the normal Fermi distribution. The deviation of $\Upsilon\neq1$ represents the configurations of reduced entropy without pulling the system off a thermal temperature. This scenario is well studied for quarks in QGP. The situation for $e^\pm$ plasma is similar to the case of the quarks during QGP, but instead here the deviation is spatial rather than temporal. Inhomogeneity can arise from the influence of other forces on the gas such as gravitational forces. This is precisely the kind of behavior that may arise in the $e^{\pm}$ epoch as the dominant photon thermal bath keeps the Fermi gas in thermal equilibrium while spatial inequilibrium could spontaneously develop. 


In the following, we will retain $\Upsilon(x)=1$ and consider the case $\eta_s/T\ll1$ for the first approximation. Then the partition function becomes
\begin{align}
\ln\mathcal{Z}_{tot}=\frac{2eBV}{(2\pi)^2}\sum_{s}^{\pm1/2}\sum_{n=0}^\infty\int^\infty_{0} \!\!dp_z\bigg[\ln\left(1+e^{-\beta(E_{n}^s-\eta_e)}\right)+\ln\left(1+e^{-\beta(E_{n}^s+\eta_e)}\right)\bigg].
\end{align}
Considering the $e^\pm$ plasma in a uniform magnetic field $B$ pointing along the $z$-axis, the energy $E_{n}^\pm$ of electron/positron system can be written as~\cite{Rafelski:2023emw}
\begin{align}
&E_{n}^\pm=\sqrt{p^2_z+\tilde m^2_\pm+2eBn},\qquad\tilde{m}^2_\pm=m^2_e+eB\left(1\mp\frac{g}{2}\right)\,,
\end{align}
where the $\pm$ script refers to spin aligned and anti-aligned eigenvalues. The parameter $g$ is the gyro-magnetic ($g$-factor) of the particle. 

To simplify the partition function, we consider the expansion of the logarithmic function as follows:
\begin{align}
\ln\left(1+x\right)=\sum^{\infty}_{k=1}\frac{(-1)^{k+1}}{k}x^k, \,\,\,\,\,\,\,\mathrm{for}\,|x|<1.
\end{align}
Then the partition function of electron/positron system can be written as
\begin{align}
\ln\mathcal{Z}_{tot}=&\frac{2eBV}{(2\pi)^2}\sum_{n=0}^\infty\int^\infty_{0} \!\!dp_z\sum^{\infty}_{k=1}\frac{(-1)^{k+1}}{k}\bigg[e^{k\beta\mu_e}+e^{-k\beta\mu_e}\bigg]e^{-k\beta E_n^\pm}\notag\\
&=\frac{2eBV}{(2\pi)^2}\sum_{n=0}^\infty\sum^{\infty}_{k=1}\frac{(-1)^{k+1}}{k}\bigg[2\cosh{(k\beta\mu_e)}\bigg]\int_0^\infty dp_z\,e^{-k\beta E_n^\pm}.
\end{align}
Using the general definition of Bessel function:
\begin{align}
K_\nu(\beta m)=\frac{\sqrt{\pi}}{\Gamma({\nu-1/2})}\frac{1}{m}\left(\frac{\beta}{2m}\right)^{\nu-1}\int_0^\infty\,dp\,p^{2\nu-2}e^{-\beta E} \,\,\,\,\,\,\,\mathrm{for}\,\nu>1/2,
\end{align}
the integral over $dp_z$ can be written as
\begin{align}
\int_0^\infty dp_z\,e^{-k\beta E_n^\pm}&=\frac{\Gamma{(1/2)}}{\sqrt{\pi}}\sqrt{\tilde{m}^2_\pm+2eBn}\,\,K_1\!\!\left({k\sqrt{\tilde{m}^2_\pm+2eBn}}/{T}\right)\notag\\&=\sqrt{\tilde{m}^2_\pm+2eBn}\,\,K_1\!\!\left({k\sqrt{\tilde{m}^2_\pm+2eBn}}/{T}\right).
\end{align}
In this case, the partition function becomes
\begin{align}
\ln\mathcal{Z}_{tot}&=\frac{2eBV}{(2\pi)^2}\sum_{n=0}^\infty\sum^{\infty}_{k=1}\frac{(-1)^{k+1}}{k}\bigg[2\cosh{(k\beta\mu_e)}\bigg]\sqrt{\tilde{m}^2_\pm+2eBn}\,\,K_1({k\sqrt{\tilde{m}^2_\pm+2eBn}}/{T})\notag\\
&=\frac{2eBTV}{(2\pi)^2}\sum^{\infty}_{k=1}\frac{(-1)^{k+1}}{k^2}\bigg[2\cosh{(k\beta\mu_e)}\bigg]\sum_{n=0}^\infty W^\pm_1(n),
\end{align}
where we introduce the function $W^\pm_1(n)$ as follows
\begin{align}
W^\pm_1(n)\equiv\frac{k\sqrt{\tilde{m}^2_\pm+2eBn}}{T}\,\,K_1\!\!\left({k\sqrt{\tilde{m}^2_\pm+2eBn}}/{T}\right).
\end{align}

Considering the Euler-Maclaurin formula to replace the sum over Landau levels, we have \index{Euler-Maclaurin formula}
\begin{align}
\sum^{\infty}_{n=0}W^\pm_1(n)=\int^\infty_0\!\!dn\,W^\pm_1(n)&+\frac{1}{2}\bigg[W^\pm_1(\infty)+W^\pm_1(0)\bigg]\notag\\
&\qquad+\frac{1}{12}\bigg[\left.\frac{\partial W^\pm_1}{\partial n}\right|_{\infty}-\left.\frac{\partial W^\pm_1}{\partial n}\right|_{0}\bigg]+R,
\end{align}
where $R$ is the error remainder which is defined by integrals over Bernoulli polynomials which is small and can be neglected~\cite{Elze:1980er}. Using the properties of Bessel function we have
\begin{align}
&\frac{\partial W^\pm_1}{\partial n}=-\frac{k^2eB}{T^2}K_0\left({k\sqrt{\tilde{m}^2_\pm+2eBn}}/{T}\right),\qquad W^\pm_1(\infty)=0,\\
&\int^\infty_a\!\!dx\,x^2K_1(x)=a^2K_2(a),
\end{align}
then we obtain
\begin{align}
\sum^{\infty}_{n=0}W^\pm_1(n)
&=\left(\frac{T^2}{k^2eB}\right)\left[\left(\frac{k\tilde{m}_\pm}{T}\right)^2K_2(k\tilde m_\pm/T)\right]+\frac{1}{2}\left[\left(\frac{k\tilde{m}_\pm}{T}\right)K_1(k\tilde m_\pm/T)\right]\notag\\
&\qquad+\frac{1}{12}\left[\left(\frac{k^2eB}{T^2}\right)K_0(k\tilde m_\pm/T)\right].
\end{align}
Replacing the sum over Landau levels by the integral, the partition function becomes
\begin{align}
\ln\mathcal{Z}_{tot}=\ln\mathcal{Z}_{free}+\ln\mathcal{Z}_B\,,
\end{align}
where we define the partition functions as  
\begin{align}
 \label{FreePart}&\ln\mathcal{Z}_{free}=\frac{T^3V}{2\pi^2}\left[2\cosh{\left(\frac{\eta_{e}}{T}\right)}\right]\sum_{i=\pm}x_i^2K_2\left(x_i\right)\,,\qquad x_i=\frac{\tilde{m}_i}{T}\\
 \label{MagPart}&\ln\mathcal{Z}_B=\frac{eBTV}{2\pi^2}\left[2\cosh{\left(\frac{\eta_{e}}{T}\right)}\right]\sum_{i=\pm}\bigg[\frac{x_i}{2}K_1\left(x_i\right)+\frac{b_0}{12}K_0\left(x_i\right)\bigg]\,.
\end{align}
The partition function $\ln(\mathcal{Z}_{free})$ in Eq.~(\ref{FreePart}) represents the general form of the Fermi partition function for $e^\pm$ with "effective mass" $\tilde{m}_\pm$ in our system. When the magnetic field $B=0$ the function $\ln(\mathcal{Z}_{free})$ will go back to the general form of the Fermi partition function without the external field. The partition function $\ln\mathcal{Z}_B$ gives us the partition with magnetic field effect to the order  $\mathcal{O}(eB)$ and  $\mathcal{O}(eB)^2$.


In the temperature domain $ 200\,\mathrm{keV} > T > 20$ keV, we have $m_e\gg T$, and it suffices to consider the Boltzmann limit of the quantum distributions. Considering the Boltzmann approximation for non-relativistic electrons and positrons we can rewrite Eq.~(\ref{FreePart}) - Eq.~(\ref{MagPart}) and obtain
\begin{align}
 \label{lnZ}
&\ln\mathcal{Z}_{tot}\!=\!\frac{T^3V}{2\pi^2}\left[2\cosh\left(\frac{\eta_{e}}{T}\right)\right]\sum_{i=\pm}\left\{x_i^{2} K_2\left(x_i\right)+\frac{b_0}{2}x_iK_1\left(x_i\right)+\frac{b^2_0}{12}K_0\left(x_i\right)\right\}.
\end{align}
Given the partition function Eq.~(\ref{lnZ}), we can explore the chemical potential and magnetization of $e^\pm$ plasma in the early Universe
under the hypothesis of charge neutrality and entropy conservation.

\paragraph{Electron chemical potential under magnetic field}
We explore the chemical potential of electron-positron plasma in a uniform magnetic field $B$ in the early Universe under the hypothesis of charge neutrality and entropy conservation. Considering the temperature after neutrino freeze-out, the charge neutrality condition can be written as
\begin{align}
 \label{density_proton}
 \left(n_{e}-n_{\bar{e}}\right)=n_{p}=X_p\,\left(\frac{n_{B}}{s_{\gamma,e}}\right)\,s_{\gamma,e},\qquad X_p\equiv\frac{n_p}{n_B}\,,
\end{align}
where $n_{p}$ and $n_B$ is the number density of protons and baryons respectively. Using the partition function Eq.~(\ref{lnZ}), the net number density of electrons in Boltzmann approximation can be written as
\begin{align}\label{NetElectron}
\left(n_e-n_{\bar e}\right)&=\frac{T}{V}\frac{\partial}{\partial \eta_{e}}\ln\mathcal{Z}_{tot}\notag\\
&=\frac{T^3}{2\pi^2}\left[2\sinh{(\eta_{e}/T)}\right]\sum_{i=\pm}\left[x_i^2K_2(x_i)+\frac{b_0}{2}x_i K_1(x_i)+\frac{b^2_0}{12}K_0(x_i)\right]\,.
\end{align}
Substituting Eq.~(\ref{NetElectron}) into the charge neutrality condition Eq.~(\ref{density_proton}), we can solve the chemical potential of electron $\eta_e/T$ numerically. We have
\begin{align}\label{ChemicalPotential}
\sinh{(\eta_{e}/T)}&=\frac{2\pi^2}{2T^3}\,\frac{X_p(n_B/s_{\gamma,e})s_{\gamma,e}}{\sum_{i=\pm}\left[x_i^2K_2(x_i)+\frac{b_0}{2}x_i K_1(x_i)+\frac{b^2_0}{12}K_0(x_i)\right]}\,,\\
&\longrightarrow\frac{2\pi^2n_p}{2T^3}\,\frac{X_p(n_B/s_{\gamma,e})s_{\gamma,e}}{2x^2K_2(x)},\qquad x=m_e/T,\qquad \mathrm{for}\,\,b_0=0\label{ChemiticalPotential_000}.
\end{align}
We see in Eq.~(\ref{ChemiticalPotential_000}) that for the case $b_0=0$, the chemical potential agrees with the free particle result in~\cite{Grayson:2023flr}. In Fig.~\ref{ChemicalPotential_B} we plot the chemical potential of electron as a function of temperature with different value of $b_0$. It shows that the chemical potential is not sensitive to the magnetic field because the small value of $10^{-3}>b_0>10^{-11}$ can be neglected in Eq.~(\ref{ChemicalPotential}).
%~~figure~~~~~~~~~~~~~~~~~~~~~~~~~~~~~~
\begin{figure}[ht]
\begin{center}
\includegraphics[width=\linewidth]{./plots/ChemicalPotential_new_survey}
\caption{The chemical potential of electron as a function of temperature in the magnetic field $b_0$ with $X_p=0.878$ and $n_B/n_\gamma=6.05\times10^{-10}$. The red dashed line represents the magnetic field $b_0=1.1\times10^{-11}$ and blue line labels the magnetic field $b_0=5.5\times10^{-3}$}
\label{ChemicalPotential_B}
\end{center}
\end{figure}
%~~~~~~~~~~~~~~~~~~~~~~~~~~~~~~~~~~~~~~~~

\paragraph{Electron-positron magnetization}
%We consider the electron-positron plasma in the mean field approximation where the external field is representative of the \lq\lq bulk\rq\rq\ internal magnetization of the gas. Each particle is therefore responding to the averaged magnetic flux generated by its neighbors as well as any global external field contribution. 

Considering the magnetized electron-positron partition function Eq.~(\ref{lnZ}), it is convenient to introduce the dimensionless magnetization $\overline{\mathcal{M}}$ and the critical field $B_c$ as follows \index{electron-positron plasma!magnetization}
\begin{align}
\label{Mdef}
\overline{\mathcal{M}}\equiv\frac{M}{\mathcal{B}_{c}}=\frac{1}{\mathcal{B}_{c}}\left(\frac{T}{V}\frac{\partial \ln\mathcal{Z}_{tot}}{\partial B}\right)\,\qquad \mathcal{B}_{c}=\frac{m_{e}^{2}}{e}\,.
\end{align}
Substituting the partition function Eq.~(\ref{lnZ}) into Eq.~(\ref{Mdef}), the total magnetization ${\overline{\mathcal M}}$ can be broken into the sum of spin parallel $\overline{\mathcal M}_{+}$ and spin anti-parallel $\overline{\mathcal M}_{-}$ magnetization. We have
\begin{align}\label{Magnetization}
&{\overline{\mathcal M}}={\overline{\mathcal M}_+}+{\overline{\mathcal M}_-},\\
&\overline{\mathcal M}_{\pm}=\frac{e^2T^{2}}{2\pi^2m_e^2}\left[2\cosh\left(\frac{\eta_{e}}{T}\right)\right]\left\{c_{1}(x_{\pm})K_1(x_i)+c_{0}K_0(x_\pm)\right\}\,,
\end{align}
where the coefficients are given by
\begin{align}
    c_{1}(x_{\pm}) &= \left[\frac{1}{2}-\left(\frac{1}{2}\pm\frac{g}{4}\right)\left(1+\frac{b^2_0}{12x^2_\pm}\right)\right]x_\pm\,,\qquad c_{0} = \left[\frac{1}{6}-\left(\frac{1}{4}\pm\frac{g}{8}\right)\right]b_0\,.
\end{align}
Substituting the chemical potential Eq.~(\ref{ChemicalPotential}) into Eq.~(\ref{Magnetization}), we can solve the magnetization numerically.
%~Figure~~~~~~~~~~~~~~~~~~~~~~~~~~~~~
\begin{figure}[ht]
    \centering
    \includegraphics[width=\textwidth]{./plots/Magnetization_Hc_new002.jpg}
    \caption{The magnetization $\overline{\cal M}=\mathcal{M}/\mathcal{B}_C$, with $g=2$, of the primordial $e^{+}e^{-}$ plasma is plotted as a function of temperature. The lower (solid red) and upper (solid blue) bounds for cosmic magnetic scale $b_{0}$ are included. The external magnetic field strength ${\cal B}/{\cal B}_{C}$ is also plotted in for lower (dashed red) and upper (dashed blue) bounds. The spin fugacity is set to unity.}
    \label{fig:magnet} 
\end{figure}
%~Figure~~~~~~~~~~~~~~~~~~~~~~~~~~~~~

In this thesis we focus on considering the case for $g=2$. In this case, the electron-positron magnetization can be written as 
\begin{align}\label{Magnetization_g2}
&{\overline{\mathcal M}}={\overline{\mathcal M}_+}+{\overline{\mathcal M}_-}\\
&{\overline{\cal M}}_{+}=\frac{e^{2}}{\pi^{2}}\frac{T^{2}}{m_{e}^{2}}\cosh{\frac{\eta_e}{T}}\left[\frac{1}{2}x_{+}K_{1}(x_{+})+\frac{b_{0}}{6}K_{0}(x_{+})\right]\,,\\
&{\overline{\cal M}}_{-}=-\frac{e^{2}}{\pi^{2}}\frac{T^{2}}{m_{e}^{2}}\cosh{\frac{\eta_e}{T}}\left[\left(\frac{1}{2}+\frac{b_{0}^{2}}{12x_{-}^{2}}\right)x_{-}K_{1}(x_{-})+\frac{b_{0}}{3}K_{0}(x_{-})\right]\,,
\end{align}
where $x_\pm$ are given by
\begin{align}
x_{+}=\frac{m_{e}}{T},\qquad   x_{-}=\sqrt{\frac{m_{e}^{2}}{T^{2}}+2b_{0}}
\end{align}
The discussion for the case $g\neq2$ can be found in~~\cite{Steinmetz:2023nsc}.

In Fig.~\ref{fig:magnet}, we present the magnetization Eq.~(\ref{Magnetization_g2}) for the case $g=2$ as a function of temperature. It shows that the magnetization depends on the magnetic scale $b_0$ and the $e^{+}e^{-}$ plasma possesses an overall paramagnetic property, resulting in a positive magnetization $\overline{\mathcal{M}}$. This paramagnetic property is contrary to the conventional assumption that matter-antimatter plasmas lack significant inherent magnetic responses. However, the magnetization never exceeds the external field under the parameters considered, which shows a lack of ferromagnetic behavior. As the Universe cooled, the dropping magnetization slowed at $T_{\mathrm{split}}=20.3$ keV, where positrons vanished. Thereafter the remaining electron density diluted with cosmic expansion.

In this section, we have explored the electron-positron plasma considering  external and self-magnetization fields  without spin potential $\eta_s/T\ll1$. However the nonzero spin potential $\eta_s\neq0$  would have an impact on the primordial $e^{+}e^{-}$ plasma. In general, the  magnetization is also a function of the spin potential $\eta_s$, and would be one important parameter that control the spin direction of primordial gas which allows for magnetization even in the absence of external magnetic fields. For further discussion see ~\cite{Steinmetz:2023nsc}.


%%%%%%%%%%%%%%%%%%%%%%%%%%%%%%%%%%%%%%%
%\chapter{Matter-antimatter origin of cosmic magnetism}
%\label{chap:cosmo}
%%%%%%%%%%%%%%%%%%%%%%%%%%%%%%%%%%%%%%%
\noindent We investigate the hypothesis that the observed intergalactic magnetic fields (IGMF) are primordial in nature, predating the recombination epoch. Specifically, we explore the role of the extremely large electron-positron $(e^{+}e^{-})$ pair abundance in the temperature range of $2000\keV>T>20\keV$ which only disappeared after Big Bang nucleosynthesis (BBN). We review the status of cosmic magnetism in \rsec{sec:mag_universe} which motivates our study. \rsec{sec:abundance} discusses the extreme electron-positron abundance during this epoch. The statistical and thermodynamic theory of the electron-positron gas is described in \rsec{sec:theory}. \rsec{sec:magnetization} describes the relativistic paramagnetism of the electron-positron gas. We propose in \rsec{sec:ferro} a model of self-magnetization caused by spin polarization within the individual species in the gas.

This chapter serves primarily as a review of our work in~\cite{Steinmetz:2023nsc,Steinmetz:2023ucp} and portions of~\cite{Rafelski:2023emw} where we propose that the early universe electron-positron plasma was a highly magnetized environment. We will use natural units $(c=\hbar=k_{B}=1)$ unless otherwise noted.

%%%%%%%%%%%%%%%%%%%%%%%%%%%%%%%%%%%%%%%
\section{Magnetism in the Plasma Universe}\label{partX}
\label{sec:mag_universe}
%%%%%%%%%%%%%%%%%%%%%%%%%%%%%%%%%%%%%%%
\noindent Macroscopic domains of magnetic fields have been found in all astrophysical environments from compact objects (stars, planets, etc.); interstellar and intergalactic space; and surprisingly in deep extra-galactic void spaces. Considering the ubiquity of magnetic fields in the universe~\cite{Giovannini:2017rbc,Giovannini:2003yn,Kronberg:1993vk}, we search for a common primordial mechanism initiate the diversity of magnetism observed today. In this chapter, IGMF will refer to experimentally observed intergalactic fields of any origin while primordial magnetic fields (PMF) refers to fields generated via early universe processes possibly as far back as inflation. The conventional elaboration of the origins for cosmic PMFs are detailed in~\cite{Gaensler:2004gk,Durrer:2013pga,AlvesBatista:2021sln}.

IGMF are notably difficult to measure and difficult to explain. The bounds for IGMF at a length scale of $1{\rm\ Mpc}$ are today~\cite{Neronov:2010gir,Taylor:2011bn,Pshirkov:2015tua,Jedamzik:2018itu,Vernstrom:2021hru}
\begin{gather}
 \label{igmf}
 10^{-8}{\rm\ G}>B_\mathrm{IGMF}>10^{-16}{\rm\ G}\,.
\end{gather}
We note that generating PMFs with such large coherent length scales is nontrivial~\cite{Giovannini:2022rrl} though currently the length scale for PMFs are not well constrained~\cite{AlvesBatista:2021sln}. Faraday rotation from distant radio active galaxy nuclei (AGN)~\cite{Pomakov:2022cem} suggest that neither dynamo nor astrophysical processes would sufficiently account for the presence of magnetic fields in the universe today if the IGMF strength was around the upper bound of $B_\mathrm{IGMF}\simeq30-60{\rm\ nG}$ as found in Ref.~\cite{Vernstrom:2021hru}. Such strong magnetic fields would then require that at least some portion of the IGMF arise from primordial sources that predate the formation of stars.

Magnetized baryon inhomogeneities which in turn would produce anisotropies in the cosmic microwave background (CMB)~\cite{Jedamzik:2013gua,Abdalla:2022yfr}. \cite{Jedamzik:2020krr} propose further that the presence of a magnetic field of $B_\mathrm{PMF}\simeq0.1{\rm\ nG}$ could be sufficient to explain the Hubble tension.

%%%%%%%%%%%%%%%%%%%%%%%%%%%%%%%%%%%%%%%
\begin{figure}[ht]
    \centering
    \includegraphics[width=0.95\textwidth]{plots/chap04cosmo/pmf.png}
    \caption{Qualitative plot of the primordial magnetic field strength over cosmic time. All figures are printed in temporal sequence in the expanding universe beginning with high temperatures (and early times) on the left and lower temperatures (and later times) on the right.}
    \label{fig:pmf}
\end{figure}
%%%%%%%%%%%%%%%%%%%%%%%%%%%%%%%%%%%%%%%

Our motivating hypothesis is outlined qualitatively in \rf{fig:pmf} where PMF evolution is plotted over the temperature history of the universe. The descending blue band indicates the range of possible PMF strengths. The different epochs of the universe according to $\Lambda\mathrm{CDM}$ are delineated by temperature. The horizontal lines mark two important scales: (a) the Schwinger critical field strength given by
\begin{align}
    \label{crit:1}
    B_\mathrm{C} = \frac{m_{e}^{2}}{e}\simeq4.41\times10^{13}\,\mathrm{G}\,.
\end{align}
where electrodynamics is expected to display nonlinear characteristics and (b) the upper field strength seen in magnetars of $\sim10^{15}\,\mathrm{G}$. A schematic of magnetogenesis is drawn with the dashed red lines indicating spontaneous formation of the PMF within the early universe plasma itself. The $e^{+}e^{-}$ era is notably the final epoch where antimatter exists in large quantities in the cosmos~\cite{Rafelski:2023emw}.

%%%%%%%%%%%%%%%%%%%%%%%%%%%%%%%%%%%%%%%
\subsection{Electron-positron abundance}
\label{sec:abundance}
%%%%%%%%%%%%%%%%%%%%%%%%%%%%%%%%%%%%%%%
\noindent As the universe cooled below temperature $T\!=\!m_{e}$ (the electron mass), the thermal electron and positron comoving density depleted by over eight orders of magnitude. At $T_\mathrm{split}=20.3\keV$, the charged lepton asymmetry (mirrored by baryon asymmetry and enforced by charge neutrality) became evident as the surviving excess electrons persisted while positrons vanished entirely from the particle inventory of the universe due to annihilation.

%%%%%%%%%%%%%%%%%%%%%%%%%%%%%%%%%%%%%%%
\begin{figure}[ht]
 \centering
\includegraphics[width=0.95\textwidth]{plots/chap04cosmo/EEPlasmaDensityRatio_new01.jpg}
 \caption{Number density of electron $e^{-}$ and positron $e^{+}$ to baryon ratio $n_{e^{\pm}}/n_{B}$ as a function of photon temperature in the universe. See text for further details. In this work we measure temperature in units of energy (keV) thus we set the Boltzmann constant to $k_{B}=1$. Figure courtesy of Cheng Tao Yang.}
 \label{fig:densityratio} 
\end{figure}
%%%%%%%%%%%%%%%%%%%%%%%%%%%%%%%%%%%%%%%

The electron-to-baryon density ratio $n_{e^{-}}/n_{B}$ is shown in \rf{fig:densityratio} as the solid blue line while the positron-to-baryon ratio $n_{e^{+}}/n_{B}$ is represented by the dashed red line. These two lines overlap until the temperature drops below $T_\mathrm{split}=20.3\keV$ as positrons vanish from the universe marking the end of the $e^{+}e^{-}$ plasma and the dominance of the electron-proton $(e^{-}p)$ plasma. The two vertical dashed green lines denote temperatures $T\!=\!m_{e}\simeq511\keV$ and $T_\mathrm{split}=20.3\keV$. These results were obtained using charge neutrality and the baryon-to-photon content (entropy) of the universe; see details in~\cite{Rafelski:2023emw}. The two horizontal black dashed lines denote the relativistic $T\gg m_e$ abundance of $n_{e^{\pm}}/n_{B}=4.47\times10^{8}$ and post-annihilation abundance of $n_{e^{-}}/n_{B}=0.87$. Above temperature $T\simeq85\keV$, the $e^{+}e^{-}$ primordial plasma density exceeded that of the Sun's core density $n_{e}\simeq6\times10^{26}{\rm\ cm}^{-3}$~\cite{Bahcall:2000nu}. 

Conversion of the dense $e^{+}e^{-}$ pair plasma into photons reheated the photon background~\cite{Birrell:2014uka} separating the photon and neutrino temperatures. The $e^{+}e^{-}$ annihilation and photon reheating period lasted no longer than an afternoon lunch break. Because of charge neutrality, the post-annihilation comoving ratio $n_{e^{-}}/n_{B}=0.87$~\cite{Rafelski:2023emw} is slightly offset from unity in~\rf{fig:densityratio} by the presence of bound neutrons in $\alpha$ particles and other neutron containing light elements produced during BBN epoch.

The abundance of baryons is itself fixed by the known abundance relative to photons~\cite{ParticleDataGroup:2022pth} and we employed the contemporary recommended value $n_B/n_\gamma=6.09\times 10^{-10}$. The resulting chemical potential needs to be evaluated carefully to obtain the behavior near to $T_\mathrm{split}=20.3\keV$ where the relatively small value of chemical potential $\mu$ rises rapidly so that positrons vanish from the particle inventory of the universe while nearly one electron per baryon remains. The detailed solution of this problem is found in \cite{Fromerth:2012fe,Rafelski:2023emw} leading to the results shown in \rf{fig:densityratio}.

%%%%%%%%%%%%%%%%%%%%%%%%%%%%%%%%%%%%%%%
\subsection{Theory of thermal matter-antimatter plasmas}
\label{sec:theory}
%%%%%%%%%%%%%%%%%%%%%%%%%%%%%%%%%%%%%%%
\noindent To evaluate magnetic properties of the thermal $e^{+}e^{-}$ pair plasma we take inspiration from Ch. 9 of Melrose's treatise on magnetized plasmas~\cite{melrose2008quantum}. We focus on the bulk properties of thermalized plasmas in (near) equilibrium.

We consider a homogeneous magnetic field domain defined along the $z$-axis as
\begin{gather}
    \label{homoB:1}
    \bb{B}=(0,\,0,\,B)\,,
\end{gather}
with magnetic field magnitude $|\bb{B}|=B$. Following \cite{Steinmetz:2018ryf}, we reprint the microscopic energy of the charged relativistic fermion within a homogeneous magnetic field given by
\begin{align}
 \label{cosmokgp}
 E^{n}_{\sigma,s}(p_{z},{B})=\sqrt{m_{e}^{2}+p_{z}^{2}+e{B}\left(2n+1+\frac{g}{2}\sigma s\right)}\,,
\end{align}
where $n\in0,1,2,\ldots$ is the Landau orbital quantum number, $p_{z}$ is the momentum parallel to the field axis and the electric charge is $e\equiv q_{e^{+}}=-q_{e^{-}}$. The index $\sigma$ in \req{cosmokgp} differentiates electron $(e^{-};\ \sigma=+1)$ and positron $(e^{+};\ \sigma=-1)$ states. The index $s$ refers to the spin along the field axis: parallel $(\uparrow;\ s=+1)$ or anti-parallel $(\downarrow;\ s=-1)$ for both particle and antiparticle species.

%%%%%%%%%%%%%%%%%%%%%%%%%%%%%%%%%%%%%%%
\begin{figure}[ht]
 \centering
 \includegraphics[width=0.95\linewidth]{plots/chap04cosmo/schematic.png}\Bstrut\\
 \begin{tabular}{ r|c|c| }
 \multicolumn{1}{r}{}
 & \multicolumn{1}{c}{aligned: $s=+1$}
 & \multicolumn{1}{c}{anti-aligned: $s=-1$} \\
 \cline{2-3}
 electron: $\sigma=+1$ & $U_{\rm Mag}>0$ & $U_{\rm Mag}<0$ \TBstrut\\
 \cline{2-3}
 positron: $\sigma=-1$ & $U_{\rm Mag}<0$ & $U_{\rm Mag}>0$ \TBstrut\\
 \cline{2-3}
 \end{tabular}\\
 \caption{Organizational schematic of matter-antimatter $(\sigma)$ and polarization $(s)$ states with respect to the sign of the non-relativistic magnetic dipole energy $U_{\rm Mag}$ obtainable from~\req{cosmokgp}.}
 \label{fig:schematic}
\end{figure}
%%%%%%%%%%%%%%%%%%%%%%%%%%%%%%%%%%%%%%%

The reason \req{cosmokgp} distinguishes between electrons and positrons is to ensure the correct non-relativistic limit for the magnetic dipole energy is reached. Following the conventions found in \cite{Tiesinga:2021myr}, we set the gyro-magnetic factor $g\equiv g_{e^{+}}=-g_{e^{-}}>0$ such that electrons and positrons have opposite $g$-factors and opposite magnetic moments relative to their spin; see \rf{fig:schematic}.

We recall the conventions established in \rsec{sec:flrw}. Conservation of magnetic flux requires that the magnetic field through a comoving surface $L_{0}^{2}$ remain unchanged. The magnetic field strength under expansion~\cite{Durrer:2013pga} starting at some initial time $t_{0}$ is then given by
\begin{gather}
 \label{bscale}
 B(t)=B_{0}\frac{a^{2}_{0}}{a^{2}(t)}\rightarrow B(z)=B_{0}\left(1+z\right)^{2}\,,
\end{gather}
where $B_{0}$ is the comoving value obtained from the contemporary value of the magnetic field today. Magnetic fields in the cosmos generated through mechanisms such as dynamo or astrophysical sources do not follow this scaling~\cite{Pomakov:2022cem}. It is only in deep intergalactic space where matter density is low are magnetic fields preserved (and thus uncontaminated) over cosmic time.

From \req{tscale} and \req{bscale} there emerges a natural ratio of interest which is conserved over cosmic expansion 
\begin{gather}
 \label{tbscale}
 \boxed{b\equiv\frac{e{B}(t)}{T^{2}(t)}=\frac{e{B}_{0}}{T_{0}^{2}}\equiv b_0={\rm\ const.}}\\
 10^{-3}>b_{0}>10^{-11}\,,
\end{gather}
given in natural units ($c=\hbar=k_{B}=1$). We computed the bounds for this cosmic magnetic scale ratio by using the present day IGMF observations given by \req{igmf} and the present CMB temperature $T_{0}=2.7{\rm\ K}\simeq2.3\times10^{-4}\eV$~\cite{Planck:2018vyg}.

%%%%%%%%%%%%%%%%%%%%%%%%%%%%%%%%%%%%%%%
\subsubsection{Eigenstatess of magnetic moment in cosmology}
\label{sec:protection}
%%%%%%%%%%%%%%%%%%%%%%%%%%%%%%%%%%%%%%%

As statistical properties depend on the characteristic Boltzmann factor $E/T$, another interpretation of \req{tbscale} in the context of energy eigenvalues (such as those given in \req{cosmokgp}) is the preservation of magnetic moment energy relative to momentum under adiabatic cosmic expansion. The Boltzmann statistical factor is given by
\begin{alignat}{1}
    \label{Boltz} x\equiv\frac{E}{T}\,.
\end{alignat}
We can explore this relationship for the magnetized system explicitly by writing out \req{Boltz} using the KGP energy eigenvalues written in \req{cosmokgp} as
\begin{alignat}{1}
    \label{XExplicit} x_{\sigma,s}^{n} = \frac{E_{\sigma,s}^{n}}{T} = \sqrt{\frac{m_{e}^{2}}{T^{2}}+\frac{p_{z}^{2}}{T^{2}}+\frac{eB}{T^{2}}\left(2n+1+\frac{g}{2}\sigma s\right)}\,.
\end{alignat}

Introducing the expansion scale factor $a(t)$ via \req{tscale}, \req{bscale} and \req{tbscale}. The Boltzmann factor can then be written as
\begin{alignat}{1}
    \label{xscale:1} x_{\sigma,s}^{n}(a(t)) = \sqrt{\frac{m_{e}^{2}}{T^{2}(t_{0})}\frac{a(t)^{2}}{a_{0}^{2}}+\frac{p_{z,0}^{2}}{T_{0}^{2}}+\frac{eB_{0}}{T_{0}^{2}}\left(2n+1+\frac{g}{2}\sigma s\right)}\,.
\end{alignat}
This reveals that only the mass contribution is dynamic over cosmological time. The constant of motion $b_{0}$ defined in \req{tbscale} is seen as the coefficient to the Landau and spin portion of the energy. For any given eigenstate, the mass term drives the state into the non-relativistic limit while the momenta and magnetic contributions are frozen by initial conditions. 

In comparison, the Boltzmann factor for the DP energy eigenvalues are given by
\begin{alignat}{1}
    \label{xscaledp:1} x_{\sigma,s}^{n}\vert_\mathrm{DP} = \sqrt{\left(\sqrt{\frac{m_{e}^{2}}{T^{2}}+\frac{eB}{T^{2}}\left(2n+1+\sigma s\right)}+\frac{eB}{2m_{e}T}\left(\frac{g}{2}-1\right)\sigma s\right)^{2}+\frac{p_{z}^{2}}{T^{2}}}\,,
\end{alignat}
which scales during FLRW expansion as
\begin{multline}
    \label{xscaledp:2} x_{\sigma,s}^{n}(a(t))\vert_\mathrm{DP} =\\ \sqrt{\left(\sqrt{\frac{m_{e}^{2}}{T_{0}^{2}}\frac{a(t)^{2}}{a_{0}^{2}}+\frac{eB_{0}}{T_{0}^{2}}\left(2n+1+\sigma s\right)}+\frac{eB_{0}}{2m_{e}T_{0}}\frac{a_{0}}{a(t)}\left(\frac{g}{2}-1\right)\sigma s\right)^{2}+\frac{p_{z,0}^{2}}{T_{0}^{2}}}\,.
\end{multline}
While the above expression is rather complicated, we note that the KGP~\req{xscale:1} and DP~\req{xscaledp:1} Boltzmann factors both reduce to the Sch{\"o}dinger-Pauli limit as $a(t)\rightarrow\infty$ thereby demonstrating that the total magnetic moment is protected under the adiabatic expansion of the universe.

Higher order non-minimal magnetic contributions can be introduced to the Boltzmann factor such as $\sim(e/m)^{2}B^{2}/T^{2}$. The reasoning above suggests that these terms are suppressed over cosmological time driving the system into minimal electromagnetic coupling with the exception of the anomalous magnetic moment. It is interesting to note that cosmological expansion then serves to `smooth out' the characteristics of more complex electrodynamics erasing them from a statistical perspective in favor of minimal-like dynamics.

%%%%%%%%%%%%%%%%%%%%%%%%%%%%%%%%%%%%%%%
\subsubsection{Magnetized fermion partition function}
\label{sec:partition}
%%%%%%%%%%%%%%%%%%%%%%%%%%%%%%%%%%%%%%%
\noindent To obtain a quantitative description of the above evolution, we study the bulk properties of the relativistic charged/magnetic gasses in a nearly homogeneous and isotropic primordial universe via the thermal Fermi-Dirac or Bose distributions.

The grand partition function for the relativistic Fermi-Dirac ensemble is given by the standard definition
\begin{align}
    \label{part:1} \ln\mathcal{Z}_\mathrm{total} &= \sum_{\alpha}\ln\left(1+\Upsilon_{\alpha_{1}\ldots\alpha_{m}}\exp\left(-\frac{E_{\alpha}}{T}\right)\right)\,,\\
    \Upsilon_{\alpha_{1}\ldots\alpha_{m}} &= \lambda_{\alpha_{1}}\lambda_{\alpha_{2}}\ldots\lambda_{\alpha_{m}}\,,
\end{align}
where we are summing over the set all relevant quantum numbers $\alpha=(\alpha_{1},\alpha_{2},\ldots,\alpha_{m})$. We note here the generalized the fugacity $\Upsilon_{\alpha_{1}\ldots\alpha_{m}}$ allowing for any possible deformation caused by pressures effecting the distribution of any quantum numbers.

In the case of the Landau problem, there is an additional summation over $\widetilde{G}$ which represents the occupancy of Landau states~\cite{greiner2012thermodynamics} which are matched to the available phase space within $\Delta p_{x}\Delta p_{y}$. If we consider the orbital Landau quantum number $n$ to represent the transverse momentum $p_{T}^{2}=p_{x}^{2}+p_{y}^{2}$ of the system, then the relationship that defines $\widetilde{G}$ is given by
\begin{alignat}{1}
    \label{phase:1} \frac{L^{2}}{(2\pi)^{2}}\Delta p_{x}\Delta p_{y}=\frac{eBL^{2}}{2\pi}\Delta n\,,\qquad\widetilde{G}=\frac{eBL^{2}}{2\pi}\,.
\end{alignat}
The summation over the continuous $p_{z}$ is replaced with an integration and the double summation over $p_{x}$ and $p_{y}$ is replaced by a single sum over Landau orbits
\begin{alignat}{1}
    \label{phase:2}
    \sum_{p_{z}}\rightarrow\frac{L}{2\pi}\int^{+\infty}_{-\infty}dp_{z}\,,\qquad\sum_{p_{x}}\sum_{p_{y}}\rightarrow\frac{eBL^{2}}{2\pi}\sum_{n}\,,
\end{alignat}
where $L$ defines the boundary length of our considered volume $V=L^{3}$.

The partition function of the $e^{+}e^{-}$ plasma can be understood as the sum of four gaseous species
\begin{align}
    \label{partition:0}    
    \ln\mathcal{Z}_{e^{+}e^{-}}=\ln\mathcal{Z}_{e^{+}}^{\uparrow}+\ln\mathcal{Z}_{e^{+}}^{\downarrow}+\ln\mathcal{Z}_{e^{-}}^{\uparrow}+\ln\mathcal{Z}_{e^{-}}^{\downarrow}\,,
\end{align}
of electrons and positrons of both polarizations $(\uparrow\downarrow)$. The change in phase space written in \req{phase:2} modify the magnetized $e^{+}e^{-}$ plasma partition function from \req{part:1} into
\begin{gather}
     \label{partition:1}
     \ln\mathcal{Z}_{e^{+}e^{-}}=\frac{e{B}V}{(2\pi)^{2}}\sum_{\sigma}^{\pm1}\sum_{s}^{\pm1}\sum_{n=0}^{\infty}\int_{-\infty}^{\infty}\mathrm{d}p_{z}\left[\ln\left(1+\lambda_{\sigma}\xi_{\sigma,s}\exp\left(-\frac{E_{\sigma,s}^{n}}{T}\right)\right)\right]\,\\
    \label{partition:2}    
    \Upsilon_{\sigma,s} =\lambda_{\sigma}\xi_{\sigma,s} = \exp{\frac{\mu_{\sigma}+\eta_{\sigma,s}}{T}}\,,
\end{gather}
where the energy eigenvalues $E_{\sigma,s}^{n}$ are given in \req{cosmokgp}. The index $\sigma$ in \req{partition:1} is a sum over electron and positron states while $s$ is a sum over polarizations. The index $s$ refers to the spin along the field axis: parallel $(\uparrow;\ s=+1)$ or anti-parallel $(\downarrow;\ s=-1)$ for both particle and antiparticle species.

We are explicitly interested in small asymmetries such as baryon excess over antibaryons, or one polarization over another. These are described by \req{partition:2} as the following two fugacities:
\begin{itemize}%[nosep]
 \item[(a)] Chemical fugacity $\lambda_{\sigma}$
 \item[(b)] Polarization fugacity $\xi_{\sigma,s}$
\end{itemize}
For matter $(e^{-};\ \sigma=+1)$ and antimatter $(e^{+};\ \sigma=-1)$ particles, a nonzero relativistic chemical potential $\mu_{\sigma}=\sigma\mu$ is caused by an imbalance of matter and antimatter. While the primordial electron-positron plasma era was overall charge neutral, there was a small asymmetry in the charged leptons (namely electrons) from baryon asymmetry~\cite{Fromerth:2012fe,Canetti:2012zc} in the universe. Reactions such as $e^{+}e^{-}\leftrightarrow\gamma\gamma$ constrains the chemical potential of electrons and positrons~\cite{Elze:1980er} as 
\begin{align}
 \label{cpotential}
 \mu\equiv\mu_{e^{-}}=-\mu_{e^{+}}\,,\qquad
 \lambda\equiv\lambda_{e^{-}}=\lambda_{e^{+}}^{-1}=\exp\frac{\mu}{T}\,,
\end{align}
where $\lambda$ is the chemical fugacity of the system.

We can then parameterize the chemical potential of the $e^{+}e^{-}$ plasma as a function of temperature $\mu\rightarrow\mu(T)$ via the charge neutrality of the universe which implies
\begin{align}
 \label{chargeneutrality}
 n_{p}=n_{e^{-}}-n_{e^{+}}=\frac{1}{V}\lambda\frac{\partial}{\partial\lambda}\ln\mathcal{Z}_{e^{+}e^{-}}\,.
\end{align}
In \req{chargeneutrality}, $n_{p}$ is the observed total number density of protons in all baryon species. The chemical potential defined in \req{cpotential} is obtained from the requirement that the positive charge of baryons (protons, $\alpha$ particles, light nuclei produced after BBN) is exactly and locally compensated by a tiny net excess of electrons over positrons.

We then introduce a novel polarization fugacity $\xi_{\sigma,s}$ and polarization potential $\eta_{\sigma,s}=\sigma s\eta$. We propose the polarization potential follows analogous expressions as seen in \req{cpotential} obeying
\begin{align}
 \label{spotential}
 \eta\equiv\eta_{+,+}=\eta_{-,-}\,,\quad\eta=-\eta_{\pm,\mp}\,,\quad\xi_{\sigma,s}\equiv\exp{\frac{\eta_{\sigma,s}}{T}}\,.
\end{align}
An imbalance in polarization within a region of volume $V$ results in a nonzero polarization potential $\eta\neq0$. Conveniently since antiparticles have opposite signs of charge and magnetic moment, the same magnetic moment is associated with opposite spin orientations. A completely particle-antiparticle symmetric magnetized plasma will have therefore zero total angular momentum.

%%%%%%%%%%%%%%%%%%%%%%%%%%%%%%%%%%%%%%%
\paragraph{Euler-Maclaurin integration.}
\label{sec:eulermac}
%%%%%%%%%%%%%%%%%%%%%%%%%%%%%%%%%%%%%%%
\noindent Before we proceed with the Boltzmann distribution approximation which makes up the bulk of our analysis, we will comment on the full Fermi-Dirac distribution analysis. The Euler-Maclaurin formula~\cite{abramowitz1988handbook} is used to convert the summation over Landau levels $n$ into an integration given by
\begin{multline}
    \label{eulermaclaurin}\sum^{b}_{n=a}f(n)-\int^{b}_{a}f(x)dx = \frac{1}{2}\left(f(b)+f(a)\right)\\
    +\sum_{i=1}^{j}\frac{b_{2i}}{(2i)!}\left(f^{(2i-1)}(b)-f^{(2i-1)}(a)\right)+R(j)\,,
\end{multline}
where $b_{2i}$ are the Bernoulli numbers and $R(j)$ is the error remainder defined by integrals over Bernoulli polynomials. The integer $j$ is chosen for the level of approximation that is desired. Euler-Maclaurin integration is rarely convergent, and in this case serves only as an approximation within the domain where the error remainder is small and bounded; see~\cite{greiner2012thermodynamics} for the non-relativistic case. In this analysis, we keep the zeroth and first order terms in the Euler-Maclaurin formula. We note that regularization of the excess terms in \req{eulermaclaurin} is done in the context of strong field QED~\cite{greiner2008quantum} though that is outside our scope.

Using \req{eulermaclaurin} allows us to convert the sum over $n$ quantum numbers in \req{partition:1} into an integral. Defining
\begin{alignat}{1}
    \label{Func} f_{\sigma,s}^{n}=\ln\left(1+\Upsilon_{\sigma,s}\exp\left(-\frac{E_{\sigma,s}^{n}}{T}\right)\right)\,,
\end{alignat}
\req{partition:1} for $j=1$ becomes
\begin{multline}
    \label{PartFuncTwo} \ln\mathcal{Z}_{e^{+}e^{-}} = \frac{e{B}V}{(2\pi)^{2}}\sum_{\sigma,s}^{\pm1}\int_{-\infty}^{+\infty}dp_{z}\\
    \left(\int_{0}^{+\infty}dn f_{\sigma,s}^{n} + \frac{1}{2}f_{\sigma,s}^{0} + \frac{1}{12}\frac{\partial f_{\sigma,s}^{n}}{\partial n}\bigg\rvert_{n=0} + R(1)\right)
\end{multline}
It will be useful to rearrange \req{cosmokgp} by pulling the spin dependency and the ground state Landau orbital into the mass writing
\begin{gather}
 \label{effmass:1}
 E^{n}_{\sigma,s}={\tilde m}_{\sigma,s}\sqrt{1+\frac{p_{z}^{2}}{{\tilde m}_{\sigma,s}^{2}}+\frac{2e{B}n}{{\tilde m}_{\sigma,s}^{2}}}\,,\\
 \label{effmass:2}
 \varepsilon_{\sigma,s}^{n}(p_{z},{B})=\frac{E_{\sigma,s}^{n}}{{\tilde m}_{\sigma,s}}\,,\qquad{\tilde m}_{\sigma,s}^{2}=m_{e}^{2}+e{B}\left(1+\frac{g}{2}\sigma s\right)\,,
\end{gather}
where we introduced the dimensionless energy $\varepsilon^{n}_{\sigma,s}$ and effective polarized mass ${\tilde m}_{\sigma,s}$ which is distinct for each spin alignment and is a function of magnetic field strength ${B}$. The effective polarized mass ${\tilde m}_{\sigma,s}$ allows us to describe the $e^{+}e^{-}$ plasma with the spin effects almost wholly separated from the Landau characteristics of the gas when considering the plasma's thermodynamic properties.

With the energies written in this fashion, we recognize the first term in \req{PartFuncTwo} as mathematically equivalent to the free particle fermion partition function with a re-scaled mass $m_{\sigma,s}$. The phase-space relationship between transverse momentum and Landau orbits in \req{phase:1} and \req{phase:2} can be succinctly described by
\begin{gather}
    p_{T}^{2} \sim 2eBn\,,\qquad2p_{T}dp_{T} \sim 2eBdn\,,\qquad d\bb{p}^{3}=2\pi p_{T}dp_{T}dp_{z}\\
    \frac{eBV}{(2\pi)^{2}}\int_{-\infty}^{+\infty}dp_{z}\int_{0}^{+\infty}dn \rightarrow \frac{V}{(2\pi)^{3}}\int d\bb{p}^{3}
\end{gather}
which recasts the first term in \req{PartFuncTwo} as
\begin{align}
    %\label{FreePart}
    \ln\mathcal{Z}_{e^{+}e^{-}} = \frac{V}{(2\pi)^{3}}\sum_{\sigma,s}^{\pm1}\int d\bb{p}^{3}\ln\left(1+\Upsilon_{\sigma,s}\exp{\left(-\frac{m_{\sigma,s}\sqrt{1+p^{2}/m_{\sigma,s}^{2}}}{T}\right)}\right)+\ldots
\end{align}
As we will see in the proceeding section, this separation of the `free-like' partition function can be reproduced in the Boltzmann distribution limit as well. This marks the end of the analytic analysis without approximations.

%%%%%%%%%%%%%%%%%%%%%%%%%%%%%%%%%%%%%%%
\subsubsection{Boltzmann approach to electron-positron plasma}
\label{sec:boltzmann}
%%%%%%%%%%%%%%%%%%%%%%%%%%%%%%%%%%%%%%%
\noindent Since we address the temperature interval $200\keV>T>20\keV$ where the effects of quantum Fermi statistics on the $e^{+}e^{-}$ pair plasma are relatively small, but the gas is still considered relativistic, we will employ the Boltzmann approximation to the partition function in \req{partition:1}. However, we extrapolate our results for presentation completeness up to $T\simeq 4m_{e}$.

%%%%%%%%%%%%%%%%%%%%%%%%%%%%%%%%%%%%%%%
\begin{table}[ht]
 \centering
 \begin{tabular}{ r|c|c| }
 \multicolumn{1}{r}{}
 & \multicolumn{1}{c}{aligned: $s=+1$}
 & \multicolumn{1}{c}{anti-aligned: $s=-1$} \\
 \cline{2-3}
 electron: $\sigma=+1$ & $+\mu+\eta$ & $+\mu-\eta$ \TBstrut\\
 \cline{2-3}
 positron: $\sigma=-1$ & $-\mu-\eta$ & $-\mu+\eta$ \TBstrut\\
 \cline{2-3}
 \end{tabular}\\\,\Bstrut\\
 \caption{Organizational schematic of matter-antimatter $(\sigma)$ and polarization $(s)$ states with respect to the chemical $\mu$ and polarization $\eta$ potentials as seen in~\req{partitionpower:2}. Companion to \rt{fig:schematic}.}
 \label{fig:org}
\end{table}
%%%%%%%%%%%%%%%%%%%%%%%%%%%%%%%%%%%%%%%

The partition function shown in equation \req{partition:1} can be rewritten removing the logarithm as
\begin{gather}
\label{partitionpower:1}
\ln{\mathcal{Z}_{e^{+}e^{-}}}=\frac{e{B}V}{(2\pi)^{2}}\sum_{\sigma,s}^{\pm1}\sum_{n=0}^{\infty}\sum_{k=1}^{\infty}\int_{-\infty}^{+\infty}\mathrm{d}p_{z}
\frac{(-1)^{k+1}}{k}\exp\left({k\frac{\sigma\mu+\sigma s\eta-{\tilde m}_{\sigma,s}\varepsilon^{n}_{\sigma,s}}{T}}\right)\,,\\
\label{bapprox} 
\sigma\mu+\sigma s\eta-{\tilde m}_{\sigma,s}\varepsilon_{\sigma,s}^{n}<0\,,
\end{gather}
which is well behaved as long as the factor in \req{bapprox} remains negative. We evaluate the sums over $\sigma$ and $s$ as
\begin{multline}
    \label{partitionpower:2}
    \ln{\mathcal{Z}_{e^{+}e^{-}}}=\frac{e{B}V}{(2\pi)^{2}}\sum_{n=0}^{\infty}\sum_{k=1}^{\infty}\int_{-\infty}^{+\infty}\mathrm{d}p_{z}\frac{(-1)^{k+1}}{k}\times\\
    \left(\ \exp\left(k\frac{+\mu+\eta}{T}\right)\exp\left(-k\frac{{\tilde m}_{+,+}\varepsilon_{+,+}^{n}}{T}\right)\right.
    +\exp\left(k\frac{+\mu-\eta}{T}\right)\exp\left(-k\frac{{\tilde m}_{+,-}\varepsilon_{+,-}^{n}}{T}\right)\qquad\\
    +\exp\left(k\frac{-\mu-\eta}{T}\right)\exp\left(-k\frac{{\tilde m}_{-,+}\varepsilon_{-,+}^{n}}{T}\right)
    +\left.\exp\left(k\frac{-\mu+\eta}{T}\right)\exp\left(-k\frac{{\tilde m}_{-,-}\varepsilon_{-,-}^{n}}{T}\right)\right)
\end{multline}
We note from \rf{fig:schematic} that the first and forth terms and the second and third terms share the same energies via
\begin{align}
    \label{partitionpower:3}
    \varepsilon_{+,+}^{n}=\varepsilon_{-,-}^{n}\,,\qquad
    \varepsilon_{+,-}^{n}=\varepsilon_{-,+}^{n}\,.\qquad
    \varepsilon_{+,-}^{n}<\varepsilon_{+,+}^{n}\,,
\end{align}

\req{partitionpower:3} allows us to reorganize the partition function with a new magnetization quantum number $s'$ which characterizes paramagnetic flux increasing states $(s'=+1)$ and diamagnetic flux decreasing states $(s'=-1)$. This recasts \req{partitionpower:2} as
\begin{multline}
    \label{partitionpower:4}
    \ln{\mathcal{Z}_{e^{+}e^{-}}}=\frac{e{B}V}{(2\pi)^{2}}\sum_{s'}^{\pm1}\sum_{n=0}^{\infty}\sum_{k=1}^{\infty}\int_{-\infty}^{+\infty}\mathrm{d}p_{z}\frac{(-1)^{k+1}}{k}\\
    \left[2\xi_{s'}\cosh\frac{k\mu}{T}\right]\exp\left(-k\frac{{\tilde m}_{s'}\varepsilon_{s'}^{n}}{T}\right)
\end{multline}
with dimensionless energy $\varepsilon_{s'}^{n}$, polarization mass $\tilde{m}_{s'}$, and polarization $\eta_{s'}$ redefined in terms of the moment orientation quantum number $s'$
\begin{gather}
    {\tilde m}_{s'}^{2}=m_{e}^{2}+e{B}\left(1-\frac{g}{2}s'\right)\,,\\
    \eta\equiv\eta_{+}=-\eta_{-}\qquad\xi\equiv\xi_{+}=\xi_{-}^{-1}\,,\qquad\xi_{s'}=\xi^{\pm1}=\exp\left(\pm\frac{\eta}{T}\right)\,.
\end{gather}

We introduce the modified Bessel function $K_{\nu}$ (see Ch. 10 of~\cite{Letessier:2002ony}) of the second kind
\begin{gather}
\label{besselk}
K_{\nu}\left(\frac{m}{T}\right)=\frac{\sqrt{\pi}}{\Gamma(\nu-1/2)}\frac{1}{m}\left(\frac{1}{2mT}\right)^{\nu-1}
\int_{0}^{\infty}\mathrm{d}p\,p^{2\nu-2}\exp\left({-\frac{m\varepsilon}{T}}\right)\,,\\
\nu>1/2\,,\qquad\varepsilon=\sqrt{1+p^{2}/m^{2}}\,,
\end{gather}
allowing us to rewrite the integral over momentum in \req{partitionpower:4} as
\begin{align}
 \label{besselkint}
 \frac{1}{T}\int_{0}^{\infty}\!\!\mathrm{d}p_{z}\exp\!\left(\!{-\frac{k{\tilde m}_{s'}\varepsilon_{s'}^{n}}{T}}\!\right)\!=\!W_{1}\!\!\left(\frac{k{\tilde m}_{s'}\varepsilon_{s'}^{n}(0,{B})}{T}\right)\,.
\end{align}
The function $W_{\nu}$ serves as an auxiliary function of the form $W_{\nu}(x)=xK_{\nu}(x)$. The notation $\varepsilon(0,{B})$ in \req{besselkint} refers to the definition of dimensionless energy found in \req{effmass:2} with $p_{z}=0$. The standard Boltzmann distribution is obtained by summing only $k=1$ and neglecting the higher order terms.

We take advantage again of Euler-Maclaurin integration \req{eulermaclaurin} and integrate the partition function. After truncation of the series and error remainder, the partition function \req{partitionpower:1} can then be written in terms of modified Bessel $K_{\nu}$ functions of the second kind and cosmic magnetic scale $b_{0}$, yielding
\begin{gather}
    \label{boltzmann}
    \boxed{\ln\mathcal{Z}_{e^{+}e^{-}}\simeq\frac{T^{3}V}{\pi^{2}}\sum_{s'}^{\pm1}\left[\xi_{s'}\cosh{\frac{\mu}{T}}\right]
    \left(x_{s'}^{2}K_{2}(x_{s'})+\frac{b_{0}}{2}x_{s'}K_{1}(x_{s'})+\frac{b_{0}^{2}}{12}K_{0}(x_{s'})\right)}\,,\\
    \label{xfunc}
    x_{s'}=\frac{{\tilde m}_{s'}}{T}=\sqrt{\frac{m_{e}^{2}}{T^{2}}+b_{0}\left(1-\frac{g}{2}s'\right)}\,.
\end{gather}
The latter two terms in \req{boltzmann} proportional to $b_{0}K_{1}$ and $b_{0}^{2}K_{0}$ are the uniquely magnetic terms present in powers of magnetic scale \req{tbscale} containing both spin and Landau orbital influences in the partition function. The $K_{2}$ term is analogous to the free Fermi gas~\cite{greiner2012thermodynamics} being modified only by spin effects.

This `separation of concerns' can be rewritten as
\begin{gather}
    \label{spin}
    \ln\mathcal{Z}_\mathrm{S}=\frac{T^{3}V}{\pi^{2}}\sum_{s'}^{\pm1}\left[\xi_{s'}\cosh{\frac{\mu}{T}}\right]\left(x_{s'}^{2}K_{2}(x_{s'})\right)\,,\\
    \label{spinorbit}
    \ln\mathcal{Z}_\mathrm{SO}=\frac{T^{3}V}{\pi^{2}}\sum_{s'}^{\pm}\left[\xi_{s'}\cosh{\frac{\mu}{T}}\right]
    \left(\frac{b_{0}}{2}x_{s'}K_{1}(x_{s'})+\frac{b_{0}^{2}}{12}K_{0}(x_{s'})\right)\,,        
\end{gather}

where the spin (S) and spin-orbit (SO) partition functions can be considered independently. When the magnetic scale $b_{0}$ is small, the spin-orbit term \req{spinorbit} becomes negligible leaving only paramagnetic effects in \req{spin} due to spin. In the non-relativistic limit, \req{spin} reproduces a quantum gas whose Hamiltonian is defined as the free particle (FP) Hamiltonian plus the magnetic dipole (MD) Hamiltonian which span two independent Hilbert spaces $\mathcal{H}_\mathrm{FP}\otimes\mathcal{H}_\mathrm{MD}$. The non-relativistic limit is further discussed in \rsec{sec:nrboltz}.

Writing the partition function as \req{boltzmann} instead of \req{partitionpower:1} has the additional benefit that the partition function remains finite in the free gas $({B}\rightarrow0)$ limit. This is because the free Fermi gas and \req{spin} are mathematically analogous to one another. As the Bessel $K_{\nu}$ functions are evaluated as functions of $x_{\pm}$ in \req{xfunc}, the `free' part of the partition $K_{2}$ is still subject to spin magnetization effects. In the limit where ${B}\rightarrow0$, the free Fermi gas is recovered in both the Boltzmann approximation $k=1$ and the general case $\sum_{k=1}^{\infty}$.

%%%%%%%%%%%%%%%%%%%%%%%%%%%%%%%%%%%%%%%
\subsubsection{Non-relativistic limit of the magnetized partition function}
\label{sec:nrboltz}
%%%%%%%%%%%%%%%%%%%%%%%%%%%%%%%%%%%%%%%
While we label the first term in \req{FreePart} as the `free' partition function, this is not strictly true as the partition function dependant on the magnetic-mass we defined in \req{effmass:2}. When determining the magnetization of the quantum Fermi gas, derivatives of the magnetic field $B$ will not fully vanish on this first term which will resulting in an intrinsic magnetization which is distinct from the Landau levels.

This represents magnetization that arises from the spin magnetic energy rather than orbital contributions. To demonstrate this, we will briefly consider the weak field limit for $g=2$. The effective polarized mass for electrons is then
\begin{align}
  \label{MagMassPlus}
  \tilde{m}_{+}^{2}&=m_{e}^{2}\,,\\
  \label{MagMassMinus}
  \tilde{m}_{-}^{2}&=m_{e}^{2}+2eB\,,
\end{align}
with energy eigenvalues
\begin{align}
  \label{EPlus}
  E_{n}^{+}&=\sqrt{p_{z}^{2}+m_{e}^{2}+2eBn}\,,\\
  \label{EMinus}
  E_{n}^{-}&=\sqrt{\left(E_{n}^{+}\right)^{2}+2eB}\,.
\end{align}
The spin anti-aligned states in the non-relativistic (NR) limit reduce to
\begin{align}
  \label{EMinusNR} E_{n}^{-}\vert_\mathrm{NR}\approx E_{n}^{+}\vert_\mathrm{NR}+\frac{eB}{m_{e}}\,.
\end{align}
This shift in energies is otherwise not influenced by summation over Landau quantum number $n$, therefore we can interpret this energy shift as a shift in the polarization potential from \req{spotential}. The polarization potential is then
\begin{align}
  \label{SpinChem} \eta_{e}^{\pm}=\eta_{e}\pm\frac{eB}{2m_{e}}\,,
\end{align}
allowing us to rewrite the partition function in \req{partitionpower:1} as
\begin{gather}
  \label{PartTotalNR} \ln\mathcal{Z}_{e^{-}}\vert_{NR}=\frac{eBV}{(2\pi)^{2}}\sum_{s'}^{\pm}\sum_{n=0}^{\infty}\sum_{k=1}^{\infty}\int_{-\infty}^{+\infty}dp_{z}\frac{(-1)^{k+1}}{k}2\cosh(k\beta\eta_{e}^{s'})\lambda^{k}\exp(-k\epsilon_{n}/T)\,,\\
  \label{NREnergy} \epsilon_{n}=m_{e}+\frac{p_{z}^{2}}{2m_{e}}+\frac{eB}{2m_{e}}\left(n+1\right)\,.
\end{gather}

\req{PartTotalNR} is then the traditional NR quantum harmonic oscillator partition function with a spin dependant potential shift differentiating the aligned and anti-aligned states. We note that in this formulation, the spin contribution is entirely excised from the orbital contribution. Under Euler-Maclaurin integration, the now spin-independant Boltzmann factor can be further separated into `free' and Landau quantum parts as was done in \req{FreePart} for the relativistic case. We note however that the inclusion of anomalous magnetic moment spoils this clean separation.

%%%%%%%%%%%%%%%%%%%%%%%%%%%%%%%%%%%%%%%
\subsubsection{Electron-positron chemical potential}
\label{sec:chem}
%%%%%%%%%%%%%%%%%%%%%%%%%%%%%%%%%%%%%%%
\noindent In presence of a magnetic field in the Boltzmann approximation, the charge neutrality condition \req{chargeneutrality} becomes
\begin{gather}
 \label{chem}
 \sinh\frac{\mu}{T}=n_{p}\frac{\pi^{2}}{T^{3}}
 \left[\sum_{s'}^{\pm1}\xi_{s'}\!\left(\!x_{s'}^{2}K_{2}(x_{s'})\!+\!\frac{b_{0}}{2}x_{s'}K_{1}(x_{s'})\!+\!\frac{b_{0}^{2}}{12}K_{0}(x_{s'}\!)\!\right)\!\right]^{-1}\!.
\end{gather}
\req{chem} is fully determined by the right-hand-side expression if the spin fugacity is set to unity $\eta=0$ implying no external bias to the number of polarizations except as a consequence of the difference in energy eigenvalues. In practice, the latter two terms in \req{chem} are negligible to chemical potential in the bounds of the primordial $e^{+}e^{-}$ plasma considered and only becomes relevant for extreme (see \rf{fig:chemicalpotential}) magnetic field strengths well outside our scope.

%%%%%%%%%%%%%%%%%%%%%%%%%%%%%%%%%%%%%%%
\begin{figure}[ht]
 \centering
 \includegraphics[clip, trim=0.0cm 0.0cm 0.0cm 0.0cm,width=0.95\linewidth]{plots/chap04cosmo/thesis_chempot_fixed.pdf}
 \caption{The chemical potential over temperature $\mu/T$ is plotted as a function of temperature with differing values of spin potential $\eta$ and magnetic scale $b_{0}$.}
 \label{fig:chemicalpotential}
\end{figure}
%%%%%%%%%%%%%%%%%%%%%%%%%%%%%%%%%%%%%%%

\req{chem} simplifies if there is no external magnetic field $b_{0}=0$ into
\begin{align}
    \label{simpchem:1}
    \sinh\frac{\mu}{T}=n_{p}\frac{\pi^{2}}{T^{3}}\left[2\cosh\frac{\eta}{T}\left(\frac{m_{e}}{T}\right)^{2}K_{2}\left(\frac{m_{e}}{T}\right)\right]^{-1}\,.
\end{align}

In \rf{fig:chemicalpotential} we plot the chemical potential $\mu/T$ in \req{chem} and \req{simpchem:1} which characterizes the importance of the charged lepton asymmetry as a function of temperature. Since the baryon (and thus charged lepton) asymmetry remains fixed, the suppression of $\mu/T$ at high temperatures indicates a large pair density which is seen explicitly in \rf{fig:densityratio}. The black line corresponds to the $b_{0}=0$ and $\eta=0$ case. 

The para-diamagnetic contribution from \req{spinorbit} does not appreciably influence $\mu/T$ until the magnetic scales involved become incredibly large well outside the observational bounds defined in \req{igmf} and \req{tbscale} as seen by the dotted blue curves of various large values $b_{0}=\{25,\ 50,\ 100,\ 300\}$. The chemical potential is also insensitive to forcing by the spin potential until $\eta$ reaches a significant fraction of the electron mass $m_{e}$ in size. The chemical potential for large values of spin potential $\eta=\{100,\ 200,\ 300,\ 400,\ 500\}\,\keV$ are also plotted as dashed black lines with $b_{0}=0$.

It is interesting to note that there are crossing points where a given chemical potential can be described as either an imbalance in spin-polarization or presence of external magnetic field. While spin potential suppresses the chemical potential at low temperatures, external magnetic fields only suppress the chemical potential at high temperatures.

The profound insensitivity of the chemical potential to these parameters justifies the use of the free particle chemical potential (black line) in the ranges of magnetic field strength considered for cosmology. Mathematically this can be understood as $\xi$ and $b_{0}$ act as small corrections in the denominator of \req{chem} if expanded in powers of these two parameters.

%%%%%%%%%%%%%%%%%%%%%%%%%%%%%%%%%%%%%%%
\subsection{Relativistic paramagnetism of electron-positron gas}
\label{sec:magnetization}
%%%%%%%%%%%%%%%%%%%%%%%%%%%%%%%%%%%%%%%
\noindent The total magnetic flux within a region of space can be written as the sum of external fields and the magnetization of the medium via
\begin{align}
 \label{totalmag}
 {B}_\mathrm{total} = {B} + \mathcal{M}\,.
\end{align}
For the simplest mediums without ferromagnetic or hysteresis considerations, the relationship can be parameterized by the susceptibility $\chi$ of the medium as
\begin{align}
 \label{susceptibility}
 {B}_\mathrm{total} = (1+\chi){B}\,,\qquad \mathcal{M} = \chi{B}\,,\qquad \chi\equiv\frac{\partial\mathcal{M}}{\partial{B}}\,,
\end{align}
with the possibility of both paramagnetic materials $(\chi>1)$ and diamagnetic materials $(\chi<1)$. The $e^{+}e^{-}$ plasma however does not so neatly fit in either category as given by \req{spin} and \req{spinorbit}. In general, the susceptibility of the gas will itself be a field dependant quantity.

In our analysis, the external magnetic field always appears within the context of the magnetic scale $b_{0}$, therefore we can introduce the change of variables
\begin{align}
 \frac{\partial b_{0}}{\partial{B}}=\frac{e}{T^{2}}\,.
\end{align}
The magnetization of the $e^{+}e^{-}$ plasma described by the partition function in \req{boltzmann} can then be written as
\begin{align}
 \label{defmagetization}
 \mathcal{M}\equiv\frac{T}{V}\frac{\partial}{\partial{B}}\ln{\mathcal{Z}_{e^{+}e^{-}}} = \frac{T}{V}\left(\frac{\partial b_{0}}{\partial{B}}\right)\frac{\partial}{\partial b_{0}}\ln{\mathcal{Z}_{e^{+}e^{-}}}\,,
\end{align}
Magnetization arising from other components in the cosmic gas (protons, neutrinos, etc.) could in principle also be included. Localized inhomogeneities of matter evolution are often non-trivial and generally be solved numerically using magneto-hydrodynamics (MHD)~\cite{melrose2008quantum,Vazza:2017qge,Vachaspati:2020blt} or with a suitable Boltzmann-Vlasov transport equation. An extension of our work would be to embed magnetization into transport theory~\cite{Formanek:2021blc}. In the context of MHD, primordial magnetogenesis from fluid flows in the electron-positron epoch was considered in~\cite{Gopal:2004ut,Perrone:2021srr}.

We introduce dimensionless units for magnetization ${\mathfrak M}$ by defining the critical field strength
\begin{align}
 {B}_{C}\equiv\frac{m_{e}^{2}}{e}\,,\qquad{\mathfrak M}\equiv\frac{\mathcal{M}}{{B}_{C}}\,.
\end{align}
The scale ${B}_{C}$ is where electromagnetism is expected to become subject to non-linear effects, though luckily in our regime of interest, electrodynamics should be linear. We note however that the upper bounds of IGMFs in \req{igmf} (with $b_{0}=10^{-3}$; see \req{tbscale}) brings us to within $1\%$ of that limit for the external field strength in the temperature range considered.

The total magnetization ${\mathfrak M}$ can be broken into the sum of magnetic moment parallel ${\mathfrak M}_{+}$ and magnetic moment anti-parallel ${\mathfrak M}_{-}$ contributions
\begin{align}
\label{g2mag}
{\mathfrak M}={\mathfrak M}_{+}+{\mathfrak M}_{-}\,.
\end{align}
We note that the expression for the magnetization simplifies significantly for $g\!=\!2$ which is the `natural' gyro-magnetic factor~\cite{Evans:2022fsu,Rafelski:2022bsv} for Dirac particles. For illustration, the $g\!=\!2$ magnetization from \req{defmagetization} is then
\begin{align}
 \label{g2magplus}
 {\mathfrak M}_{+}&=\frac{e^{2}}{\pi^{2}}\frac{T^{2}}{m_{e}^{2}}\xi\cosh{\frac{\mu}{T}}\left[\frac{1}{2}x_{+}K_{1}(x_{+})+\frac{b_{0}}{6}K_{0}(x_{+})\right]\,,\\
 \label{g2magminus}
 -{\mathfrak M}_{-}&=\frac{e^{2}}{\pi^{2}}\frac{T^{2}}{m_{e}^{2}}\xi^{-1}\cosh{\frac{\mu}{T}}
 \left[\left(\frac{1}{2}+\frac{b_{0}^{2}}{12x_{-}^{2}}\right)x_{-}K_{1}(x_{-})+\frac{b_{0}}{3}K_{0}(x_{-})\right]\,,\\
 x_{+}&=\frac{m_{e}}{T}\,,\qquad
 x_{-}=\sqrt{\frac{m_{e}^{2}}{T^{2}}+2b_{0}}\,.
\end{align}
As the $g$-factor of the electron is only slightly above two at $g\simeq2.00232$~\cite{Tiesinga:2021myr}, the above two expressions for ${\mathfrak M}_{+}$ and ${\mathfrak M}_{-}$ are only modified by a small amount because of anomalous magnetic moment (AMM) and would be otherwise invisible on our figures.

%%%%%%%%%%%%%%%%%%%%%%%%%%%%%%%%%%%%%%%
\subsubsection{Evolution of electron-positron magnetization}
\label{sec:paramagnetism}
%%%%%%%%%%%%%%%%%%%%%%%%%%%%%%%%%%%%%%%
\noindent In \rf{fig:magnet}, we plot the magnetization as given by \req{g2magplus} and \req{g2magminus} with the spin potential set to unity $\xi=1$. The lower (solid red) and upper (solid blue) bounds for cosmic magnetic scale $b_{0}$ are included. The external magnetic field strength ${B}/{B}_{C}$ is also plotted for lower (dotted red) and upper (dotted blue) bounds. Since the derivative of the partition function governing magnetization may manifest differences between Fermi-Dirac and the here used Boltzmann limit more acutely, out of abundance of caution, we indicate extrapolation outside the domain of validity of the Boltzmann limit with dashes.

%%%%%%%%%%%%%%%%%%%%%%%%%%%%%%%%%%%%%%%
\begin{figure}[ht]
 \centering
 \includegraphics[width=0.95\linewidth]{plots/chap04cosmo/thesis_mag.pdf}
 \caption{The magnetization ${\mathfrak M}$, with $g\!=\!2$, of the primordial $e^{+}e^{-}$ plasma is plotted as a function of temperature}
 %label{fig:magnet} 
\end{figure}
%%%%%%%%%%%%%%%%%%%%%%%%%%%%%%%%%%%%%%%

We see in \rf{fig:magnet} that the $e^{+}e^{-}$ plasma is overall paramagnetic and yields a positive overall magnetization which is contrary to the traditional assumption that matter-antimatter plasma lack significant magnetic responses of their own in the bulk. With that said, the magnetization never exceeds the external field under the parameters considered which shows a lack of ferromagnetic behavior. 

The large abundance of pairs causes the smallness of the chemical potential seen in~\rf{fig:chemicalpotential} at high temperatures. As the universe expands and temperature decreases, there is a rapid decrease of the density $n_{e^{\pm}}$ of $e^{+}e^{-}$ pairs. This is the primary the cause of the rapid paramagnetic decrease seen in \rf{fig:magnet} above $T\!=\!21\keV$. At lower temperatures $T<21\keV$ there remains a small electron excess (see~\rf{fig:densityratio}) needed to neutralize proton charge. These excess electrons then govern the residual magnetization and dilutes with cosmic expansion.

An interesting feature of \rf{fig:magnet} is that the magnetization in the full temperature range increases as a function of temperature. This is contrary to Curie's law~\cite{greiner2012thermodynamics} which stipulates that paramagnetic susceptibility of a laboratory material is inversely proportional to temperature. However, Curie's law applies to systems with fixed number of particles which is not true in our situation; see \rsec{sec:perlepton}.

A further consideration is possible hysteresis as the $e^{+}e^{-}$ density drops with temperature. It is not immediately obvious the gas's magnetization should simply `degauss' so rapidly without further consequence. If the very large paramagnetic susceptibility present for $T\simeq m_{e}$ is the origin of an overall magnetization of the plasma, the conservation of magnetic flux through the comoving surface ensures that the initial residual magnetization is preserved at a lower temperature by Faraday induced kinetic flow processes however our model presented here cannot account for such effects.

Early universe conditions may also apply to some extreme stellar objects with rapid change in $n_{e^{\pm}}$ with temperatures above $T\!=\!21\keV$. Production and annihilation of $e^{+}e^{-}$ plasmas is also predicted around compact stellar objects~\cite{Ruffini:2009hg,Ruffini:2012it} potentially as a source of gamma-ray bursts.

%%%%%%%%%%%%%%%%%%%%%%%%%%%%%%%%%%%%%%%
\subsubsection{Dependency on g-factor}
\label{sec:gfac}
%%%%%%%%%%%%%%%%%%%%%%%%%%%%%%%%%%%%%%%

\noindent As discussed at the end of \rsec{sec:magnetization}, the AMM of $e^{+}e^{-}$ is not relevant in the present model. However out of academic interest, it is valuable to consider how magnetization is effected by changing the $g$-factor significantly.

%%%%%%%%%%%%%%%%%%%%%%%%%%%%%%%%%%%%%%%
\begin{figure}[ht]
 \centering
 \includegraphics[width=0.95\textwidth]{plots/chap04cosmo/thesis_gfac.pdf}
 \caption{The magnetization $\mathfrak M$ as a function of $g$-factor plotted for several temperatures with magnetic scale $b_{0}=10^{-3}$ and polarization fugacity $\xi=1$.}
 \label{fig:gfac} 
\end{figure}
%%%%%%%%%%%%%%%%%%%%%%%%%%%%%%%%%%%%%%%

The influence of AMM would be more relevant for the magnetization of baryon gasses since the $g$-factor for protons $(g\approx5.6)$ and neutrons $(g\approx3.8)$ are substantially different from $g\!=\!2$. The influence of AMM on the magnetization of thermal systems with large baryon content (neutron stars, magnetars, hypothetical bose stars, etc.) is therefore also of interest~\cite{Ferrer:2019xlr,Ferrer:2023pgq}.

\req{g2magplus} and \req{g2magminus} with arbitrary $g$ reintroduced is given by
\begin{gather}
\label{arbg:1}
{\mathfrak M}=\frac{e^{2}}{\pi^{2}}\frac{T^{2}}{m_{e}^{2}}\sum_{s'}^{\pm1}\xi_{s'}\cosh{\frac{\mu}{T}}
\left[C^{1}_{s'}(x_{s'})K_{1}(x_{s'})+C^{0}_{s'}K_{0}(x_{s'})\right]\,,\\
\label{arbg:2}
C^{1}_{s'}(x_{\pm}) = \left[\frac{1}{2}-\left(\frac{1}{2}-\frac{g}{4}s'\right)\left(1+\frac{b^2_0}{12x^{2}_{s'}}\right)\right]x_{s'}\,,\qquad
C^{0}_{s'} = \left[\frac{1}{6}-\left(\frac{1}{4}-\frac{g}{8}s'\right)\right]b_0\,,
\end{gather}
where $x_{s'}$ was previously defined in \req{xfunc}.

In \rf{fig:gfac}, we plot the magnetization as a function of $g$-factor between $4>g>-4$ for temperatures $T\!=\!\{511,\ 300,\ 150,\ 70\}\keV$. We find that the magnetization is sensitive to the value of AMM revealing a transition point between paramagnetic $({\mathfrak M}>0)$ and diamagnetic gasses $({\mathfrak M}<0)$. Curiously, the transition point was numerically determined to be around $g\simeq1.1547$ in the limit $b_{0}\rightarrow0$. The exact position of this transition point however was found to be both temperature and $b_{0}$ sensitive, though it moved little in the ranges considered.

It is not surprising for there to be a transition between diamagnetism and paramagnetism given that the partition function (see \req{spin} and \req{spinorbit}) contained elements of both. With that said, the transition point presented at $g\approx1.15$ should not be taken as exact because of the approximations used to obtain the above results. 

It is likely that the exact transition point has been altered by our taking of the Boltzmann approximation and Euler-Maclaurin integration steps. It is known that the Klein-Gordon-Pauli solutions to the Landau problem in \req{cosmokgp} have periodic behavior~\cite{Steinmetz:2018ryf,Evans:2022fsu,Rafelski:2022bsv} for $|g|=k/2$ (where $k\in1,2,3\ldots$).

These integer and half-integer points represent when the two Landau towers of orbital levels match up exactly. Therefore, we propose a more natural transition between the spinless diamagnetic gas of $g=0$ and a paramagnetic gas is $g=1$. A more careful analysis is required to confirm this, but that our numerical value is close to unity is suggestive.

%%%%%%%%%%%%%%%%%%%%%%%%%%%%%%%%%%%%%%%
\subsubsection{Magnetization per lepton}
\label{sec:perlepton}
%%%%%%%%%%%%%%%%%%%%%%%%%%%%%%%%%%%%%%%
\noindent Despite the relatively large magnetization seen in \rf{fig:magnet}, the average contribution per lepton is only a small fraction of its overall magnetic moment indicating the magnetization is only loosely organized. Specifically, the magnetization regime we are in is described by
\begin{align}
 \label{fractionalmagnetization}
 \mathcal{M}\ll\mu_{B}\frac{N_{e^{+}}+N_{e^{-}}}{V}\,,\qquad\mu_{B}\equiv\frac{e}{2m_{e}}\,,
\end{align}
where $\mu_{B}$ is the Bohr magneton and $N=nV$ is the total particle number in the proper volume V. To better demonstrate that the plasma is only weakly magnetized, we define the average magnetic moment per lepton given by along the field ($z$-direction) axis as
\begin{align}
 \label{momentperlepton}
 \vert\vec{m}\vert_{z}\equiv\frac{\mathcal{M}}{n_{e^{-}}+n_{e^{+}}}\,,\qquad\vert\vec{m}\vert_{x}=\vert\vec{m}\vert_{y}=0\,.
\end{align}
Statistically, we expect the transverse expectation values to be zero. We emphasize here that despite $|\vec{m}|_{z}$ being nonzero, this doesn't indicate a nonzero spin angular momentum as our plasma is nearly matter-antimatter symmetric. The quantity defined in \req{momentperlepton} gives us an insight into the microscopic response of the plasma.

%%%%%%%%%%%%%%%%%%%%%%%%%%%%%%%%%%%%%%%
\begin{figure}[ht]
 \centering
 \includegraphics[clip, trim=0.0cm 0.0cm 0.0cm 0.0cm,width=0.95\textwidth]{plots/chap04cosmo/thesis_perlepton.png}
 \caption{The magnetic moment per lepton $\vert\vec{m}\vert_{z}$ along the field axis as a function of temperature}
 \label{fig:momentperlepton}
\end{figure}
%%%%%%%%%%%%%%%%%%%%%%%%%%%%%%%%%%%%%%%

The average magnetic moment $\vert\vec{m}\vert_{z}$ defined in \req{momentperlepton} is plotted in \rf{fig:momentperlepton} which displays how essential the external field is on the `per lepton' magnetization. The $b_{0}=10^{-3}$ case (blue curve) is plotted in the Boltzmann approximation. The dashed lines indicate where this approximation is only qualitatively correct. For illustration, a constant magnetic field case (solid green line) with a comoving reference value chosen at temperature $T_{0}=10\keV$ is also plotted.

If the field strength is held constant, then the average magnetic moment per lepton is suppressed at higher temperatures as expected for magnetization satisfying Curie's law. The difference in \rf{fig:momentperlepton} between the non-constant (blue solid curve) case and the constant field (solid green curve) case demonstrates the importance of the conservation of primordial magnetic flux in the plasma, required by \req{bscale}. While not shown, if \rf{fig:momentperlepton} was extended to lower temperatures, the magnetization per lepton of the constant field case would be greater than the non-constant case which agrees with our intuition that magnetization is easier to achieve at lower temperatures. This feature again highlights the importance of flux conservation in the system and the uniqueness of the primordial cosmic environment.

%%%%%%%%%%%%%%%%%%%%%%%%%%%%%%%%%%%%%%%
\subsection{Polarization potential and ferromagnetism}
\label{sec:ferro}
%%%%%%%%%%%%%%%%%%%%%%%%%%%%%%%%%%%%%%%
\noindent Up to this point, we have neglected the impact that a nonzero spin potential $\eta\neq0$ (and thus $\xi\neq1$) would have on the primordial $e^{+}e^{-}$ plasma magnetization. In the limit that $(m_{e}/T)^2\gg b_0$ the magnetization given in \req{arbg:1} and \req{arbg:2} is entirely controlled by the spin fugacity $\xi$ asymmetry generated by the spin potential $\eta$ yielding up to first order $\mathcal{O}(b_{0})$ in magnetic scale
\begin{multline}
 \label{ferro}
 \lim_{m_{e}^{2}/T^{2}\gg b_0}{\mathfrak M}=\frac{g}{2}\frac{e^{2}}{\pi^{2}}\frac{T^{2}}{m_{e}^{2}}\sinh{\frac{\eta}{T}}\cosh{\frac{\mu}{T}}\left[\frac{m_{e}}{T}K_{1}\left(\frac{m_{e}}{T}\right)\right]\\
 +b_{0}\left(g^{2}-\frac{4}{3}\right)\frac{e^{2}}{8\pi^{2}}\frac{T^{2}}{m_{e}^{2}}\cosh{\frac{\eta}{T}}\cosh{\frac{\mu}{T}}K_{0}\left(\frac{m_{e}}{T}\right)
 +\mathcal{O}\left(b_{0}^{2}\right)
\end{multline}

Given \req{ferro}, we can understand the spin potential as a kind of `ferromagnetic' influence on the primordial gas which allows for magnetization even in the absence of external magnetic fields. This interpretation is reinforced by the fact the leading coefficient is $g/2$.

We suggest that a variety of physics could produce a small nonzero $\eta$ within a domain of the gas. Such asymmetries could also originate statistically as while the expectation value of free gas polarization is zero, the variance is likely not.

As $\sinh{\eta/T}$ is an odd function, the sign of $\eta$ also controls the alignment of the magnetization. In the high temperature limit \req{ferro} with strictly $b_{0}=0$ assumes a form of to lowest order for brevity
\begin{align}
 \label{hiTferro}
 \lim_{m_{e}/T\rightarrow0}{\mathfrak M}\vert_{b_{0}=0}=\frac{g}{2}\frac{e^{2}}{\pi^{2}}\frac{T^{2}}{m_{e}^{2}}\frac{\eta}{T}\,,
\end{align}

While the limit in \req{hiTferro} was calculated in only the Boltzmann limit, it is noteworthy that the high temperature (and $m\rightarrow0$) limit of Fermi-Dirac distributions only differs from the Boltzmann result by a proportionality factor. 

The natural scale of the $e^{+}e^{-}$ magnetization with only a small spin fugacity ($\eta<1\eV$) fits easily within the bounds of the predicted magnetization during this era if the IGMF measured today was of primordial origin. The reason for this is that the magnetization seen in \req{g2magplus}, \req{g2magminus} and \req{ferro} are scaled by $\alpha{B}_{C}$ where $\alpha$ is the fine structure constant.

%%%%%%%%%%%%%%%%%%%%%%%%%%%%%%%%%%%%%%%
\subsubsection{Hypothesis of ferromagnetic self-magnetization}
\label{sec:self}
%%%%%%%%%%%%%%%%%%%%%%%%%%%%%%%%%%%%%%%
\noindent One exploratory model we propose is to fix the spin polarization asymmetry, described in \req{spotential}, to generate a homogeneous magnetic field which dissipates as the universe cools down. In this model, there is no external primordial magnetic field $({B}_\mathrm{PMF}=0)$ generated by some unrelated physics, but rather the $e^{+}e^{-}$ plasma itself is responsible for the field by virtue of spin polarization.

%%%%%%%%%%%%%%%%%%%%%%%%%%%%%%%%%%%%%%%
\begin{figure}[ht]
 \centering
 \includegraphics[width=0.9\textwidth]{plots/chap04cosmo/Spinchemical_03.png}
 \caption{The spin potential $\eta$ and chemical potential $\mu$ are plotted under the assumption of self-magnetization through a nonzero spin polarization in bulk of the plasma}
 \label{fig:self} 
\end{figure}
%%%%%%%%%%%%%%%%%%%%%%%%%%%%%%%%%%%%%%%

This would obey the following assumption of
\begin{align}
 \label{selfmag}
 {\mathfrak M}(b_{0})=\frac{\mathcal{M}(b_0)}{{B}_{C}}\longleftrightarrow\frac{B}{{B}_{C}}=b_{0}\frac{T^{2}}{m_{e}^{2}}\,,
\end{align}
which sets the total magnetization as a function of itself. The spin polarization described by $\eta\rightarrow\eta(b_{0},T)$ then becomes a fixed function of the temperature and magnetic scale. The underlying assumption would be the preservation of the homogeneous field would be maintained by scattering within the gas (as it is still in thermal equilibrium) modulating the polarization to conserve total magnetic flux.

The result of the self-magnetization assumption in \req{selfmag} for the potentials is plotted in \rf{fig:self}. The solid lines indicate the curves for $\eta/T$ for differing values of $b_{0}=\{10^{-11},\ 10^{-7},\ 10^{-5},\ 10^{-3}\}$ which become dashed above $T\!=\!300\keV$ to indicate that the Boltzmann approximation is no longer appropriate though the general trend should remain unchanged.

%%%%%%%%%%%%%%%%%%%%%%%%%%%%%%%%%%%%%%%
\begin{figure}[ht]
 \centering
 \includegraphics[width=0.95\textwidth]{plots/chap04cosmo/ElectronDensity_SpinChemicalPotential004.jpg}
 \caption{The number density $n_{e^{\pm}}$ of polarized electrons and positrons under the self-magnetization model for differing values of $b_{0}$. Figure courtesy of Cheng Tao Yang.}
 \label{fig:polarswap} 
\end{figure}
%%%%%%%%%%%%%%%%%%%%%%%%%%%%%%%%%%%%%%%

The dotted lines are the curves for the chemical potential $\mu/T$. At high temperatures we see that a relatively small $\eta/T$ is needed to produce magnetization owing to the large densities present. \rf{fig:self} also shows that the chemical potential does not deviate from the free particle case until the spin polarization becomes sufficiently high which indicates that this form of self-magnetization would require the annihilation of positrons to be incomplete even at lower temperatures.

This is seen explicitly in~\rf{fig:polarswap} where we plot the numerical density of particles as a function of temperature for spin aligned $(+\eta)$ and spin anti-aligned $(-\eta)$ species for both positrons $(-\mu)$ and electrons $(+\mu)$. Various self-magnetization strengths are also plotted to match those seen in~\rf{fig:self}. The nature of $T_{\rm split}$ changes under this model since antimatter and polarization states can be extinguished separately. Positrons persist where there is insufficient electron density to maintain the magnetic flux. Polarization asymmetry therefore appears physical only in the domain where there is a large number of matter-antimatter pairs.

%%%%%%%%%%%%%%%%%%%%%%%%%%%%%%%%%%%%%%%
\subsubsection{Matter inhomogeneities in the cosmic plasma}
\label{sec:inhomogeneous}
%%%%%%%%%%%%%%%%%%%%%%%%%%%%%%%%%%%%%%%
\noindent In general, an additional physical constraint is required to fully determine $\mu$ and $\eta$ simultaneously as both potentials have mutual dependency (see \rsec{sec:ferro}). We note that spin polarizations are not required to be in balanced within a single species to preserve angular momentum.

The CMB~\cite{Planck:2018vyg} indicates that the early universe was home to domains of slightly higher and lower baryon densities which resulted in the presence of galactic super-clusters, cosmic filaments, and great voids seen today. However, the CMB, as measured today, is blind to the localized inhomogeneities required for gravity to begin galaxy and supermassive black hole formation.

Such acute inhomogeneities distributed like a dust~\cite{Grayson:2023flr} in the plasma would make the proton density sharply and spatially dependant $n_{p}\rightarrow n_{p}(x)$ which would directly affect the potentials $\mu(x)$ and $\eta(x)$ and thus the density of electrons and positrons locally. This suggests that $e^{+}e^{-}$ may play a role in the initial seeding of gravitational collapse. If the plasma were home to such localized magnetic domains, the nonzero local angular momentum within these domains would provide a natural mechanism for the formation of rotating galaxies today.

Recent measurements by the James Webb Space Telescope (JWST)~\cite{Yan:2022sxd,adams2023discovery,arrabal2023spectroscopic} indicate that galaxy formation began surprisingly early at large redshift values of $z\gtrsim10$ within the first 500 million years of the universe requiring gravitational collapse to begin in a hotter environment than expected. The observation of supermassive black holes already present~\cite{CEERSTeam:2023qgy} in this same high redshift period (with millions of solar masses) indicates the need for local high density regions in the early universe whose generation is not yet explained and likely need to exist long before the recombination epoch.

%%%%%%%%%%%%%%%%%%%%%%%%
% Chapters from Jeremiah Birrell's dissertation
\section{Part 3: Jeremy's Thesis}\label{part3}
\subsection{Boltzmann-Einstein Equation}
We now begin a detailed study of the non-equilibrium properties of the neutrino freeze-out and it's impact on the effective number of neutrinos, an important cosmological observable. We model the dynamics of the neutrino freeze-out using the Boltzmann-Einstein equation\index{Boltzmann-Einstein equation}, also called the general relativistic Boltzmann equation, which describes the dynamics of a gas of particles that travel on geodesics in an general spacetime, with the only interactions being point collisions~\cite{Andreasson:2011ng,cercignani,Choquet-Bruhat:2009xil,ehlers},
\begin{equation}\label{boltzmann_einstein}
p^\alpha\partial_{x^\alpha}f-\sum_{j=1}^3\Gamma^j_{\mu\nu}p^\mu p^\nu\partial_{p^j}f=C[f]\,.
\end{equation}
Here $ \Gamma^\alpha_{\mu\nu}$ is the affine connection (Christoffel symbols) corresponding to a metric $g_{\alpha\beta}$,   the distribution function $f$ is a function of four-momentum on the mass shell, i.e., that satisfy
 \begin{equation}
g_{\alpha\beta}p^\alpha p^\beta=m^2\,.
\end{equation}
Here and in the following,  repeated Greek indices are summed from $0$ to $3$.  $C[f]$ is the collision operator and encodes all information about point interactions between particles.  If $C[f]$ vanishes then the equation is called the Vlasov equation and describes particles that move on geodesics (or free stream).  At this point, we are  not invoking the assumption that the distribution function has a kinetic equilibrium form, nor are we assuming a FRW universe; in this section we will discuss  general properties of \req{boltzmann_einstein} before turning to the study of neutrino freeze-out in subsequent sections.   We will need the following definitions of entropy current $s^\mu$, stress-energy tensor ${T}^{\mu\nu}$, and number current $n^\mu$,\index{entropy current}\index{stress-energy tensor}\index{number current}
\begin{align}
\label{smdef} s^\mu&=-\int \left(f\ln(f)\pm(1\mp f)\ln(1\mp f)\right)p^\mu d\pi\,,\\
\label{Tmndef}{T}^{\mu\nu}&=\int p^\mu p^\nu f d\pi\,,\\
\label{nmdef} n^\nu&=\int f p^\nu d\pi\,,\\
d\pi&=\frac{\sqrt{-g}}{p_0}\frac{g_pd^3{\bf p}}{8\pi^3}\,,
\end{align}
where $d\pi$ is the volume element on the future mass shell, $g$ denotes the determinant of the metric tensor, $p_0=g_{0\alpha} p^\alpha$, non-bold $p$ are four-momenta while bold ${\bf p}$ denotes the spacial components, the upper signs are for fermions and the lower signs for bosons. See Appendix \ref{ch:vol_forms} for the derivation of the form of the volume element.
\subsubsection{Collision Operator}
We now elaborate on the form of the collision operator.  Our presentation is an expanded version of the survey in \cite{ehlers}.  Suppose we have a collection of  distinct particle and antiparticle types $\mathcal{C}$ with distribution functions $f_{C}$, $C\in\mathcal{C}$, and they partake in some number of reactions or interactions $I=n_{B_1} B_1, n_{B_2}B_2...\longrightarrow n_{A_1} A_1,n_{A_2}A_2...$, $A_i\in\mathcal{C}$ distinct and $B_j\in\mathcal{C}$ distinct, where $n_{A_i}$ is the number of particles of type $A_i$ occurring in the interaction (all nonzero) and similarly for $n_{B_i}$.  Given an interaction, $I$, we let $r(I)$ be the collection of particle types that are reactants in the interaction, $p(I)$ be the collection of particle types that are products, and we let $\overleftarrow{I}$ denote the reverse reaction, i.e., with reactants and products reversed.   We let $int$ denote the set of all interactions and, for any given species $A$, $int(A)$ be the set of all interactions involving $A$ as a   product.   We will assume that $\overleftarrow{I}\in int$ whenever $I\in int$.  With these conventions, the collision operator for particle type $A$ takes the form\index{collision operator}
\begin{align}\label{collision_operator}
&C[f_A]\\
=&\sum_{I\in int(A)} \frac{n_A}{\prod_i n_{A_i}!\prod _j n_{B_j}!}\int\left[\left(\prod_j \prod_{l=1}^{n_{B_j}}f_{B_j}(p_{B_j}^l)\right)\left(\prod_i \prod_{k=1}^{n_{A_i}}f^{A_i}(p_{A_i}^k)\right)W^I(p_{B_j}^l,p_{A_i}^k) \right.\notag\\
& \left. -\left(\prod_i \prod_{k=1}^{n_{A_i}}f_{A_i}(p_{A_i}^k)\right)\left(\prod_j \prod_{l=1}^{n_{B_j}}f^{B_j}(p_{B_j}^l)\right)W^{\overleftarrow{I}}(p_{A_i}^k,p_{B_j}^l) \right] \delta(\Delta p)\prod_i \widehat{dV}_{A_i}\prod_j dV_{B_j},\notag\\
&f^C=1\mp f_C, \hspace{2mm} \Delta p=\sum_i \sum_{k=1}^{n_{A_i}}p^k_{A_i}-\sum_j \sum_{l=1}^{n_{B_j}}p^l_{B_j}\,,\notag\\
&\widehat{dV}_{A_i}=\tilde{\pi}_{A_i}\prod_{k=2}^{n_{A_i}}\frac{1}{2}d\pi^k_{A_i}, \hspace{2mm}  dV_{B_j}=(2\pi)^4\prod_{l=1}^{n_{B_j}}\frac{1}{2}d\pi^l_{B_j}\,,\notag\\
&\tilde{\pi}_{A_i}=\frac{1}{2} \text{ if } A_i=A \text{ and }  \tilde{\pi}_{A_i}=\frac{1}{2}d\pi^1_{A_i} \text{ otherwise,} \notag\\
&d\pi_{C}^r=\frac{\sqrt{-g}}{(p_{C}^r)_0}\frac{g_{C}d^3{\bf p}_{C}^r}{8\pi^3}, \hspace{2mm}p_0=g_{0\alpha}p^\alpha.\notag
\end{align}
  The integrations are over the future mass shells of all the particles, so the $p$ are related by $g_{\alpha \beta}p^\alpha p^\beta=m^2$. The factorials take into account the indistinguishably of the particles and prevent one from over counting the independent ways a reaction can happen when integrating over momentum.  The terms $f^A$ are due to quantum statistics and account for Fermi repulsion or Bose attraction (again, upper signs are for fermions and lower signs for bosons).  $W^I(p_{B_j}^l,p_{A_i}^k)$, an abbreviation for  $W^I(p_{B_1}^1,p_{B_1}^2,...,p_{B_1}^{n_{B_1}},p_{B_2}^1,...,p_{A_1}^1,...)$, is the scattering kernel that encodes the probability of $n_{B_j}$ particles of types $B_j$ with momenta $p_{B_j}^l$ interacting to form $n_{A_i}$ particles of types $A_i$ with momenta $p_{A_i}^k$ in the process $I=n_{B_1}B_1,n_{B_2}B_2,...\longrightarrow n_{A_1}A_1,n_{A_1}A_1,...$, and so it is non-negative.  The delta function enforces conservation of four-momentum. The factors of $(2\pi^4)$ and $\frac{1}{2}$ in the definitions of the volume elements come from normalization of the transition functions from quantum scattering calculations.  For computational purposes, the expression \eqref{collision_operator} must be further simplified, taking into account the structure of each interaction.  For example, see Appendix \ref{ch:coll_simp} for a detailed study of the collision operator in the case of neutrino freeze-out.

 As defined, $C[f_A]$ is a function of $p_{A_i}^1$ where $A=A_i$. The choice to not integrate over $p_{A_i}^1$ rather than any of the other $p_{A_i}^k$ is completely arbitrary, but makes no difference in the result since the interaction does not depend on how we number the participating particles. In terms of the scattering kernels, this means we assume $W^I$ has the property
\begin{equation}\label{reorder_property}
W^I(p_{A_1}^{\sigma_1},p^{\sigma 2}_{A_1},...)=W^I(p_{A_1}^1,p_{A_1}^2,...)\,,
\end{equation}
for any permutation $\sigma$, and similarly for any other permutation with one of the collections $p_{A_i}^k$ or $p_{B_j}^l$ for any choice of $i$ or $j$. For economy of notation in these derivations, we will employ the additional abbreviations for a given interaction  $I=n_{B_i}B_i\longrightarrow n_{A_i}A_i$:
\begin{align}
f_{p,I}(p^k_{A_i})&\equiv f_{p,I}(p^1_{A_i},p^2_{A_i},...,p^{n_{A_i}}_{A_i})\equiv \prod_i \prod_{k=1}^{n_{A_i}}f_{A_i}(p_{A_i}^k)\,,\\
f^{p,I}(p^k_{A_i})&=f^{p,I}(p^1_{A_i},p^2_{A_i},...,p^{n_{A_i}}_{A_i})=\prod_i \prod_{k=1}^{n_{A_i}}f^{A_i}(p_{A_i}^k)\,,\notag\\
f_{r,I}(p^l_{B_j})&\equiv f_{r,I}(p^1_{B_j},p^2_{B_j},...,p^{n_{B_j}}_{B_j})\equiv \prod_j \prod_{l=1}^{n_{B_j}}f_{B_j}(p_{B_j}^l)\,,\notag\\
f^{r,I}(p^l_{B_j})&=f^{r,I}(p^1_{B_j},p^2_{B_j},...,p^{n_{B_j}}_{B_j})=\prod_j \prod_{l=1}^{n_{B_j}}f^{B_j}(p_{B_j}^l)\,,\notag\\
n_I&=\prod_i n_{A_i}!\prod _j n_{B_j}!\,,\notag\\
\widehat{dV}_I&=\delta(\Delta p)\prod_i\widehat{dV}_{A_i}\prod_jdV_{B_j}\,,\notag\\
dV_I&=\delta(\Delta p)\prod_idV_{A_i}\prod_jdV_{B_j}\,,\notag
\end{align}
where the $r$ and $p$ sub and superscripts stand for reactants and products respectively.   See Appendix \ref{ch:vol_forms} for more information on the precise meaning and properties of the delta function factors.

In the following subsections we derive several important  properties of the equation \eqref{boltzmann_einstein}.  While in principle these properties are well known~\cite{Andreasson:2011ng,cercignani,Choquet-Bruhat:2009xil,ehlers}, here we prove them at a level of generality that, to the authors knowledge, is not available in other references, i.e., for a general collection of interactions as encapsulated in \req{collision_operator}.  We  note that Riemannian normal coordinates will a key tool   in these derivations.   These are coordinates centered at a chosen point, $x$, in spacetime wherein the geodesics through $x$ are straight lines in the coordinate system and the derivatives of the metric in the coordinate system vanish at $x$. In particular, the Christoffel symbols vanish at $x$; see, e.g., page 42 in \cite{Wald:1984rg} or pages 72-73 of \cite{o1983semi}.  
%%%%%%%%%%%%%%%%%%%%%%%%%%%%%%%%%%%%%%%%%%%%%%%%%%%%
\subsubsection{Conserved Currents}
Suppose all the interactions of interest conserve some charge $b_A$, i.e.,
 \begin{align}\label{eq:conserved_charge}
\sum_{A\in p(I)} n_Ab_A=\sum_{A\in r(I)} n_Ab_A
\end{align}
for all $I\in int$.   We can construct and $4$-vector current corresponding to this charge as follows:
\begin{equation}
B^\mu=\sum_A b_A N_A^\mu\,,
\end{equation}
where $N^\mu_A$ are the number currents of the particle species \req{nmdef}.  In this section we show that $B^\mu$ has vanishing divergence,  i.e., a $B^\mu$ satisfies a conservation law.

For any point $x$ in spacetime,  by transforming to Riemannian normal coordinates at   $x$ and using \eqref{boltzmann_einstein} along with the fact that the first derivatives of the metric vanish at $x$,  one can compute
\begin{equation}\label{use_normal_coords}
\nabla_\mu N_A^\mu=\int p^\mu \partial_{x^\mu} f d\pi_A=\int C[f_A] d\pi_A
\end{equation}
at $x$. The left and right hand sides are scalars   and therefore they are equal in any coordinate system. Noting this, we can then calculate
\begin{align}
\nabla_\mu B^\mu=&\sum_A b_A\int C[f_A]d\pi_A=\sum_A\sum_{I\in int(A)} \frac{n_Ab_A}{n_I}\int\int\left(f_{r,I}(p_{B_j}^l)f^{p,I}(p_{A_i}^k)W^I(p_{B_j}^l,p_{A_i}^k) \right.\\
&\left. -f_{p,I}(p_{A_i}^k)f^{r,I}(p_{B_j}^l)W^{\overleftarrow{I}}(p_{A_i}^k,p_{B_j}^l)\right)\widehat{dV}_I d\pi_A\notag\\
=&\sum_A\sum_{I\in int(A)} \frac{n_Ab_A}{n_I}\int\left(f_{r,I}(p_{B_j}^l)f^{p,I}(p_{A_i}^k)W^I(p_{B_j}^l,p_{A_i}^k) \right.\notag\\
&\left. -f_{p,I}(p_{A_i}^k)f^{r,I}(p_{B_j}^l)W^{\overleftarrow{I}}(p_{A_i}^k,p_{B_j}^l)\right)  dV_I.\notag
\end{align}
Now observe that, for any collection of finite sets $D_j$ indexed by a finite set $J$ with $\bigcup_{j\in J}D_j=D$ and any function $h:J\times D\rightarrow \mathbb{R}^m$ we have
\begin{equation}\label{sum_lemma}
\sum_{j\in J}\sum_{x\in D_j} h(j,x)=\sum_{x\in D}\sum_{\{j:x\in D_j\}}h(j,x)\,.
\end{equation}
Using this fact, we can switch the order of the sums to obtain
\begin{align}\label{del_B}
&\nabla_\mu B^\mu=\sum_{I\in int}\sum_{A\in p(I)} n_Ab_A R_I \,,\\
&R_I\equiv \frac{1}{n_I}\int\left(f_{r,I}(p_{B_j}^l)f^{p,I}(p_{A_i}^k)W^I(p_{B_j}^l,p_{A_i}^k)  -f_{p,I}(p_{A_i}^k)f^{r,I}(p_{B_j}^l)W^{\overleftarrow{I}}(p_{A_i}^k,p_{B_j}^l)\right)  dV_I\,.\notag
\end{align}
The sum over all interactions splits over a sum over symmetric interactions, $int_{s}$, and a sum over asymmetric interactions.  For each asymmetric interaction, pair it up with its reverse and arbitrarily choose one of them to call the forward direction.  Let the set of these forward interactions be denoted $\overrightarrow{int}$.  Then the sum in \req{del_B} splits as follows
\begin{equation}
\nabla_\mu B^\mu=\sum_{I\in int_s}R_I\sum_{A\in p(I)} n_A b_A+\sum_{I\in\overrightarrow{int}}R_I\sum_{A\in p(I)} n_Ab_A+\sum_{I\in\overrightarrow{int}}R_{\overleftarrow{I}}\sum_{A\in p(\overleftarrow{I})} n_Ab_A\,.
\end{equation}
For every $I\in int_s$ we have $W^I=W^{\overleftarrow{I}}$, $f_{A_i}=f_{B_i}$, and $f^{A_i}=f^{B_i}$, and therefore
\begin{align}
R_I=&\frac{1}{n_I}\left(\int f_{r,I}(p_{B_j}^l)f^{p,I}(p_{A_i}^k)W^I(p_{B_j}^l,p_{A_i}^k)  dV_I \right.\\
&\left. -\int f_{p,I}(p_{A_i}^k)f^{r,I}(p_{B_j}^l)W^{\overleftarrow{I}}(p_{A_i}^k,p_{B_j}^l)  dV_I\right)\notag\\
=&\frac{1}{n_I}\left(\int f_{r,I}(p_{B_j}^l)f^{p,I}(p_{A_i}^k)W^I(p_{B_j}^l,p_{A_i}^k)  dV_I \right.\notag\\
&\left. -\int f_{r,I}(p_{A_i}^k)f^{p,I}(p_{B_j}^l)W^{I}(p_{A_i}^k,p_{B_j}^l)  dV_I\right)\notag\\
=&0\,,\notag
\end{align}
as the two integrals differ only by a relabeling of integration variables.  Asymmetric interactions satisfy
\begin{align}
R_{\overleftarrow{I}}=&\frac{1}{n_I}\int\left(f_{p,I}(p_{A_i}^k)f^{r,I}(p_{B_j}^l)W^{\overleftarrow{I}}(p_{A_i}^k,p_{B_j}^l) 
-f_{r,I}(p_{B_j}^l)f^{p,I}(p_{A_i}^k)W^I(p_{B_j}^l,p_{A_i}^k)\right)  dV_I\notag\\
=&-R_I.
\end{align}
Combining this with \req{eq:conserved_charge} we find
\begin{align}
\nabla_\mu B^\mu&=\sum_{I\in\overrightarrow{int}}R_I\left(\sum_{A\in p(I)} n_Ab_A-\sum_{A\in p(\overleftarrow{I})} n_Ab_A\right)\\
&=\sum_{I\in\overrightarrow{int}}R_I\left(\sum_{A\in p(I)} n_Ab_A-\sum_{A\in r(I)} n_Ab_A\right)=0\,.\notag
\end{align}
Therefore  $B^\mu$ is a conserved current, as claimed.

\subsubsection{Divergence Freedom of Stress Energy Tensor}
The Einstein equation implies that the total stress energy tensor of all matter coupled to gravity is divergence free.  Here we show that the relativistic Boltzmann stress energy tensor \req{Tmndef} has this property, and is therefore a natural candidate matter model for  coupling to gravity.  

First use Riemannian normal coordinates to compute
\begin{align}
\nabla_\mu T^{\mu\nu}=&\sum_A \int p_A^\nu C[f_A]d\pi_A\\
=&\sum_A\sum_{I\in int(A)}\frac{n_A}{n_I}\int (p^1_{A_\ell})^\nu\left(f_{r,I}(p_{B_j}^l)f^{p,I}(p_{A_i}^k)W^I(p_{B_j}^l,p_{A_i}^k) \right.\\
&\left. -f_{p,I}(p_{A_i}^k)f^{r,I}(p_{B_j}^l)W^{\overleftarrow{I}}(p_{A_i}^k,p_{B_j}^l)\right)  dV_I\,,\notag
\end{align}
where $\ell$ is the unique index such that $A_\ell=A$ ($\ell$ depends on $A$ and $I$, but we suppress this dependence for simplicity of notation). Using \req{sum_lemma} we can switch the summation order to get
\begin{align}
\nabla_\mu T^{\mu\nu}=&\sum_{I\in int}\sum_{A\in p(I)} \frac{n_A}{n_I}\int (p^1_{A_\ell})^{\nu}\left(f_{r,I}(p_{B_j}^l)f^{p,I}(p_{A_i}^k)W^I(p_{B_j}^l,p_{A_i}^k) \right.\\
&\left. -f_{p,I}(p_{A_i}^k)f^{r,I}(p_{B_j}^l)W^{\overleftarrow{I}}(p_{A_i}^k,p_{B_j}^l)\right)  dV_I\notag\,.
\end{align}
By \req{reorder_property} and the surrounding remarks, we can rewrite this as
\begin{align}\label{del_T_sum}
\nabla_\mu T^{\mu\nu}=&\sum_{I\in int}\sum_{A\in p(I)} \frac{1}{n_I}\sum_{a=1}^{n_A}\int (p^a_{A_\ell})^{\nu}\left(f_{r,I}(p_{B_j}^l)f^{p,I}(p_{A_i}^k)W^I(p_{B_j}^l,p_{A_i}^k) \right.\\
&\left. -f_{p,I}(p_{A_i}^k)f^{r,I}(p_{B_j}^l)W^{\overleftarrow{I}}(p_{A_i}^k,p_{B_j}^l)\right)  dV_I\notag\\
=&\sum_{I\in int} \frac{1}{n_I}\sum_{\ell}\sum_{a=1}^{n_{A_\ell}}\int (p_{A_\ell}^a)^{\nu}\left(f_{r,I}(p_{B_j}^l)f^{p,I}(p_{A_i}^k)W^I(p_{B_j}^l,p_{A_i}^k) \right.\notag\\
&\left. -f_{p,I}(p_{A_i}^k)f^{r,I}(p_{B_j}^l)W^{\overleftarrow{I}}(p_{A_i}^k,p_{B_j}^l)\right)  dV_I\,.\notag
\end{align}
As before, we can break the sum over $I$ into a sum over symmetric processes and two other sums over forward and backward asymmetric processes respectively.  For a symmetric interaction $I=\overleftarrow{I}$ and $f_{A_i}=f_{B_i}$ for all $i$, hence
\begin{align}
&\sum_\ell\sum_{a=1}^{n_{A_\ell}}\int (p_{A_\ell}^a)^{\nu}\left(f_{r,I}(p_{B_j}^l)f^{p,I}(p_{A_i}^k)W^I(p_{B_j}^l,p_{A_i}^k) \right.\\
&\left. -f_{p,I}(p_{A_i}^k)f^{r,I}(p_{B_j}^l)W^{\overleftarrow{I}}(p_{A_i}^k,p_{B_j}^l)\right)  dV_I\notag\\
=&\int\sum_\ell\sum_{a=1}^{n_{A_\ell}}\left((p_{A_\ell}^a)^\nu- (p_{B_\ell}^a)^{\nu}\right) f_{r,I}(p_{B_j}^l)f^{p,I}(p_{A_i}^k)W^I(p_{B_j}^l,p_{A_i}^k) dV_I\notag\\
=& 0\,,\notag
\end{align}
due to the delta function $\delta(\Delta p)$ in the volume form $dV_I$.  Therefore the terms in the sum \req{del_T_sum} corresponding to symmetric interactions vanish.  For every pair of forward and backward asymmetric interactions we obtain
\begin{align}
&\sum_\ell\sum_{a=1}^{n_{A_\ell}}\int (p_{A_\ell}^a)^{\nu}\left(f_{r,I}(p_{B_j}^l)f^{p,I}(p_{A_i}^k)W^I(p_{B_j}^l,p_{A_i}^k) \right.\\
&\left. -f_{p,I}(p_{A_i}^k)f^{r,I}(p_{B_j}^l)W^{\overleftarrow{I}}(p_{A_i}^k,p_{B_j}^l)\right)  dV_I\notag\\
&+\sum_{\tilde\ell}\sum_{c=1}^{n_{B_{\tilde\ell}}} \int (p_{B_{\tilde\ell}}^c)^{\nu}\left(f_{p,I}(p_{A_i}^k)f^{r,I}(p_{B_j}^l)W^{\overleftarrow{I}}(p_{A_i}^k,p_{B_j}^l) \right.\notag\\
&\left. -f_{r,I}(p_{B_j}^l)f^{p,I}(p_{A_i}^k)W^I(p_{B_j}^l,p_{A_i}^k)\right)  dV_I\notag\\
=&\int\left( \sum_\ell\sum_{a=1}^{n_{A_\ell}}(p_{A_\ell}^a)^{\nu} -\sum_{\tilde \ell}\sum_{c=1}^{n_{B_{\tilde\ell}}} (p_{B_{\tilde\ell}}^c)^{\nu}\right)f_{r,I}(p_{B_j}^l)f^{p,I}(p_{A_i}^k)W^I(p_{B_j}^l,p_{A_i}^k)  dV_I\notag\\
&+\int \left(\sum_{\tilde\ell}\sum_{c=1}^{n_{B_{\tilde\ell}}}(p_{B_{\tilde\ell}}^c)^{\nu}-\sum_\ell\sum_{a=1}^{n_{A_\ell}}(p_{A_\ell}^a)^{\nu}\right)f_{p,I}(p_{A_i}^k)f^{r,I}(p_{B_j}^l)W^{\overleftarrow{I}}(p_{A_i}^k,p_{B_j}^l) dV_I\notag\\
=&0\,,\notag
\end{align}
again because of $\delta(\Delta p)$ in the volume forms.  This shows $\nabla_\mu T^{\mu\nu}=0$, as claimed.

\subsubsection{Entropy and Boltzmann's H-Theorem}\index{Boltzmann's H-theorem}
Finally, we prove that the entropy four-current satisfies $\nabla_\mu s^\mu\geq 0$, known as Boltzmann's H-theorem. This result requires the additional assumption that the interactions are time-reversal symmetric,  i.e.,
\begin{equation}\label{time_symmetry}
W^I(p_{B_j}^l,p_{A_i}^k)=W^{\overleftarrow{I}}(p_{A_i}^k,p_{B_j}^l)
\end{equation}
for all $I$.  

Working in Riemannian normal coordinates once again, we can compute
\begin{align}
\nabla_\mu s^\mu=&-\sum_A\int p^\mu \partial_{x^\mu}\left(f_A\ln\left(f_A\right)\pm\left(1\mp f_A\right)\ln\left(1\mp f_A\right)\right)d\pi_A\\
=&\sum_A\int\ln\left(1/f_A\mp 1\right)C[f_A]d\pi_A\,.\notag
\end{align}
Similar reasoning to the above two subsections then gives
\begin{align}
\nabla_\mu s^\mu=&\sum_{I\in int}\frac{1}{n_I}\sum_\ell\sum_{a=1}^{n_{A_\ell}}\int\ln\left(1/f_{A_\ell}(p_{A_\ell}^a)\mp1\right)\left(f_{r,I}(p_{B_j}^l)f^{p,I}(p_{A_i}^k)W^I(p_{B_j}^l,p_{A_i}^k) \right.\\
&\left. -f_{p,I}(p_{A_i}^k)f^{r,I}(p_{B_j}^l)W^{\overleftarrow{I}}(p_{A_i}^k,p_{B_j}^l)\right)  dV_I\,.\notag
\end{align}
Once again, we break the summation into a sum over symmetric processes and two other sums over forward and backward asymmetric processes respectively. Each symmetric process contributes a term of the form
\begin{align}
&\int\sum_\ell\sum_{a=1}^{n_{A_\ell}}f_{r,I}(p_{B_j}^l)f^{p,I}(p_{A_i}^k)\left(\ln\left(1/f_{A_\ell}(p_{A_\ell}^a)\mp 1\right)\right.\\
&\left.\qquad\qquad\qquad\qquad-\ln\left(1/f_{B_\ell}(p_{B_\ell}^a)\mp 1\right)\right) W^I(p_{B_j}^l,p_{A_i}^k) dV_I\notag\\
=& \int\ln\left(\frac{f^{p,I}(p_{A_i}^k)f_{r,I}(p_{B_j}^l)}{f_{p,I}(p_{A_i}^k)f^{r,I}(p_{B_j}^l)}\right) f_{r,I}(p_{B_j}^l)f^{p,I}(p_{A_i}^k)W^I(p_{B_j}^l,p_{A_i}^k) dV_I\notag\\
=& \frac{1}{2}\int\ln\left(\frac{f^{p,I}(p_{A_i}^k)f_{r,I}(p_{B_j}^l)}{f_{p,I}(p_{A_i}^k)f^{r,I}(p_{B_j}^l)}\right)\left( f_{r,I}(p_{B_j}^l)f^{p,I}(p_{A_i}^k)\right.\notag\\
&\left.\qquad\qquad\qquad\qquad- f_{p,I}(p_{A_j}^l)f^{r,I}(p_{B_i}^k)\right)W^I(p_{B_j}^l,p_{A_i}^k) dV_I\,,\notag
\end{align}
\begin{comment}
\begin{align}
&\int\ln\left(\frac{f^{p,I}(p_{A_i}^k)f_{r,I}(p_{B_j}^l)}{f_{p,I}(p_{A_i}^k)f^{r,I}(p_{B_j}^l)}\right) f_{r,I}(p_{B_j}^l)f^{p,I}(p_{A_i}^k)W^I(p_{B_j}^l,p_{A_i}^k) dV\\
=&\int\ln\left(\frac{f^{p,I}(p_{B_i}^k)f_{r,I}(p_{A_j}^l)}{f_{p,I}(p_{B_i}^k)f^{r,I}(p_{A_j}^l)}\right) f_{r,I}(p_{A_j}^l)f^{p,I}(p_{B_i}^k)W^I(p_{B_j}^l,p_{A_i}^k) dV\\
=&-\int\ln\left(\frac{f_{p,I}(p_{B_i}^k)f^{r,I}(p_{A_j}^l)}{f^{p,I}(p_{B_i}^k)f_{r,I}(p_{A_j}^l)}\right) f_{r,I}(p_{A_j}^l)f^{p,I}(p_{B_i}^k)W^I(p_{B_j}^l,p_{A_i}^k) dV
\end{align}
\end{comment}
where to obtain the last line we used the time-reversal property  \eqref{time_symmetry}.

A pair of forward and backward asymmetric interactions combine to give a term of the form
\begin{align}
&\sum_\ell\sum_{a=1}^{n_{A_b}}\int\ln\left(1/f_{A_\ell}(p_{A_\ell}^a)\mp 1\right)\left(f_{r,I}(p_{B_j}^l)f^{p,I}(p_{A_i}^k)W^I(p_{B_j}^l,p_{A_i}^k) \right.\\
&\left. -f_{p,I}(p_{A_i}^k)f^{r,I}(p_{B_j}^l)W^{\overleftarrow{I}}(p_{A_i}^k,p_{B_j}^l)\right)  dV_I\notag\\
&+\sum_{\tilde\ell}\sum_{c=1}^{n_{B_{\tilde\ell}}} \int\ln\left(1/f_{B_{\tilde\ell}}(p_{B_{\tilde\ell}}^c)\mp 1\right)\left(f_{p,I}(p_{A_i}^k)f^{r,I}(p_{B_j}^l)W^{\overleftarrow{I}}(p_{A_i}^k,p_{B_j}^l) \right.\notag\\
&\left. -f_{r,I}(p_{B_j}^l)f^{p,I}(p_{A_i}^k)W^I(p_{B_j}^l,p_{A_i}^k)\right)  dV_I\notag\\
=&\int\left( \sum_\ell\sum_{a=1}^{n_{A_\ell}}\ln\left(1/f_{A_\ell}(p_{A_\ell}^a)\mp 1\right) \right.\notag\\
&\left.\qquad\qquad\qquad-\sum_{\tilde\ell}\sum_{c=1}^{n_{B_{\tilde\ell}}} \ln\left(1/f_{B_{\tilde\ell}}(p_{B_{\tilde\ell}}^c)\mp 1\right)\right)f_{r,I}(p_{B_j}^l)f^{p,I}(p_{A_i}^k)W^I(p_{B_j}^l,p_{A_i}^k)  dV_I\notag\\
&-\int \left(\sum_\ell\sum_{a=1}^{n_{A_\ell}}\ln\left(1/f_{A_\ell}(p_{A_\ell}^a)\mp 1\right) \right.\notag\\
&\left.\qquad\qquad\qquad-\sum_{\tilde\ell}\sum_{c=1}^{n_{B_{\tilde\ell}}} \ln\left(1/f_{B_{\tilde\ell}}(p_{B_{\tilde\ell}}^c)\mp 1\right)\right)f_{p,I}(p_{A_i}^k)f^{r,I}(p_{B_j}^l)W^{I}(p_{B_j}^l,p_{A_i}^k)  dV_I\notag\\
\end{align}
where  to obtain the first equality we  used  the time-reversal property  \eqref{time_symmetry}.  Combining the symmetric and asymmetric cases we find
\begin{align}
\nabla_\mu s^\mu=&\sum_{I\in int_s} \frac{1}{2n_I}\int\ln\left(\frac{f^{p,I}(p_{A_i}^k)f_{r,I}(p_{B_j}^l)}{f_{p,I}(p_{A_i}^k)f^{r,I}(p_{B_j}^l)}\right)\left( f_{r,I}(p_{B_j}^l)f^{p,I}(p_{A_i}^k)\right.\\
&\left.- f_{p,I}(p_{A_j}^l)f^{r,I}(p_{B_i}^k)\right)W^I(p_{B_j}^l,p_{A_i}^k) dV_I\notag\\
&+\sum_{I\in \overrightarrow{int}}\frac{1}{n_I}\int\ln\left(\frac{f_{r,I}(p_{B_j}^l)f^{p,I}(p_{A_i}^k)}{f_{p,I}(p_{A_i}^k)f^{r,I}(p_{B_j}^l)}\right)\left(f_{r,I}(p_{B_j}^l)f^{p,I}(p_{A_i}^k)\right.\notag\\
&\left.-f_{p,I}(p_{A_i}^k)f^{r,I}(p_{B_j}^l)\right)W^I(p_{B_j}^l,p_{A_i}^k)  dV_I\notag\,.
\end{align}
Each term in either sum is the integral of a non-negative quantity $W^{I}$ times a quantity of the form $(a-b)\ln(a/b)$, $a,b>0$, which is easily seen to be non-negative.  Therefore we obtain the claimed result $\nabla_\mu s^\mu\geq 0$.  The entropy four current is future directed, due to the volume element being supported on the future mass shell. Therefore, given any splitting of spacetime into space and time $M=S\times T$,   Boltzmann's H-theorem implies that the total entropy  is non-decreasing on $T$.


\section{ Study of Neutrino Distribution using Conservation Laws}\label{ch:model_ind}

\subsection{Effective Number of Neutrinos}\label{sec:N_nu}
In the previous chapter we gave an overview of cosmology that included a simple model of neutrino freeze-out, wherein neutrinos decouple prior to the $e^\pm$ annihilation reheating period, leading to the reheating ratio in the decoupled limit \req{T_nu_T_gamma} which we now denote by $R_\nu$. However, as we mentioned several times this is only an approximate model. In reality, the freeze-out and reheating periods overlap to some degree which greatly complicates the picture, as some energy and entropy from the annihilating $e^\pm$ goes into neutrinos.  This overlap has observable consequences, as any extra energy in neutrinos impacts the speed of expansion of the Universe, through the Hubble equation \req{Hubble_eq}.

The additional energy and entropy fed into neutrinos is typically quantified by the effective number of neutrinos, $N_\nu$, defined by comparing the total neutrino energy density to the energy density of a massless fermion with two degrees of freedom and neutrino to photon temperature ratio $R_\nu$,
\begin{equation}
N_{\nu}=\frac{\rho_\nu}{\frac{7}{120}\pi^2  \left(R_\nu T_\gamma\right)^4}.
\end{equation}
 By definition, any transfer of energy from $e^\pm$ into neutrinos results in $N_\nu>N_\nu^f=3$, the number of physical neutrino flavors.  $N_\nu$ can be  measured by fitting to observational data, such as the Planck CMB measurements. A numerical computation based on the Boltzmann equation with two body scattering~\cite{Mangano2005} gives to $N_{\nu}^{\rm th}=3.046$. However the Planck CMB results contain several fits~\cite{Planck} based on different data sets which suggest that $N_\nu$ is in the range $3.30\pm 0.27$ to $3.62\pm0.25$ ($68\%$ confidence level). 

This tension between the Planck results and theoretical reheating studies motivates our work. This tension has inspired various theories, such as \cite{Weinberg:2013kea}, where it is postulated to be due to the presence of as yet undiscovered particle species. In this work, we avoid postulating the existence of additional particles, but rather explore the possibility that the increase in $N_\nu$ is the consequences of additional energy and entropy being transferred into neutrinos during $e^\pm$ annihilation.  In other words, we postulate additional neutrino reheating and explore its consequences.



\subsection{Matter Content}
In this work, matter will be modeled by a particle distribution function $f(t,x,p)$ that, roughly speaking, gives the probability of finding a particle per unit spacial volume per unit momentum space volume at a given time.  The distribution function gives the stress energy tensor, particle four-current, and entropy four-current via 
\begin{align}
T^{\mu,\nu}(t,x)=&\frac{g_p}{(2\pi)^3}\int p^\mu p^\nu f(t,x,p) \sqrt{|g|}\frac{d^3p}{p_0},\\
n^\nu(t,x)=&\frac{g_p}{(2\pi)^3}\int p^\nu f(t,x,p) \sqrt{|g|}\frac{d^3p}{p_0},\\
s^\nu(t,x)=&-\frac{g_p}{(2\pi)^3}\int(f\ln(f)\pm(1\mp f)\ln(1\mp f))p^\mu\sqrt{|g|}\frac{d^3p}{p_0}
\end{align}
where the upper signs are for fermions, the lower for bosons, $g_p$ is the degeneracy of the particle, and $g$ is the determinant of the metric.  In a flat FRW Universe, the expressions for the  energy density, pressure, number density, and entropy density of a particle of mass $m$ are
\begin{align}\label{moments}
\rho=&\frac{g_p}{(2\pi)^3}\int f(t,x,p)Ed^3p,\\
P=&\frac{g_p}{(2\pi)^3}\int f(t,x,p)\frac{p^2}{3E}d^3p,\\
n=&\frac{g_p}{(2\pi)^3}\int f(t,x,p) d^3p, \hspace{2mm} E=\sqrt{m^2+p^2},\\
s=&-\frac{g_p}{(2\pi)^3}\int (f\ln(f)\pm(1\mp f)\ln(1\mp f)) d^3p.
\end{align}

The dynamics of the distribution function,  and therefore the precise nature of neutrino freeze-out and the energy and entropy transferred into the neutrino sector, are governed by the Boltzmann equation
\begin{equation}
p^\alpha\partial_{x^\alpha} f-\Gamma^{j}_{\mu,\nu}p^\mu p^\nu\partial_{p^j}f=C[f]
\end{equation}
where repeated Greek indices indicate a sum over $0,...,3$ and Roman indices indicate a sum over the spacial components $1,...,3$.  The right hand side is the collision operator and incorporates the physics of any short range interactions that the particles participate in. The left hand side gives the dynamics under any long range forces. For us the only long range force will be gravity, encoded in the Christoffel symbols $\Gamma^j_{\mu\nu}$, and so the Boltzmann equation expresses the fact that particles undergo geodesic motion in between collisions. For much greater detail on the definition of the distribution function in a general spacetime, the geometric origin of the Boltzmann equation, and various properties and relations satisfied by moments of the distribution function, see for example \cite{andre,cercignani,bruhat,ehlers,kolb,bernstein2004kinetic}.

We will study neutrino freeze-out in detail using the Boltzmann equation in the second part of this dissertation, starting in chapter \ref{ch:boltz_orthopoly}. However, in this chapter we pursue a model independent approach wherein we assume instantaneous chemical/kinetic equilibrium and sharp freeze-out transitions between them.  Though limited in the kinds of questions we can address and answer, this approach makes up for these limitations by letting us derive several important properties that are independent of microscopic dynamics, i.e. independent of $C[f]$, so long as these assumptions are sufficiently accurate.  The dynamics will be derived from conservation laws involving the moments \ref{moments}, but first we must describe the distinction between chemical and kinetic equilibrium.


\subsubsection{Chemical and Kinetic Equilibrium}
%%%%%%%%%%%%%%%%%%%%%%%%%%%%%%%%%%%%
At sufficiently high temperatures, such as existed in the early Universe, both particle creation and annihilation (i.e. chemical) processes and momentum exchanging (i.e. kinetic) scattering processes can occur sufficiently rapidly to establish complete thermal equilibrium of a given particle species. The most probable canonical distribution function $f_{ch}^\pm$ of  fermions (+) and bosons (-) in both chemical and kinetic equilibrium is found by maximizing entropy subject to energy being conserved
\begin{equation}\label{ch_eq}
f_{ch}^\pm=\frac{1}{\exp(E/T)\pm 1}, \hspace{2mm} T>T_{ch}
\end{equation}
where $E$ is the particle energy, $T$ the temperature, and $T_{ch}$ the chemical freeze-out temperature. 


For a physical system comprising {\em interacting} particles whose temperature is decreasing with time, there will be a period where the temperature is greater than the kinetic freeze-out temperature, $T_k$, but below chemical freeze-out. During this period, momentum exchanging processes continue to maintain an equilibrium distribution of energy among the available particles, which we call kinetic equilibrium, but particle number changing processes no longer occur rapidly enough to keep the equilibrium particle number yield, i.e. for $T<T_{ch}$ the particle number changing processes have `frozen-out'. In this condition the momentum distribution, which is in kinetic equilibrium but chemical non-equilibrium, is obtained by maximizing  entropy subject to  particle number and energy constraints and thus two parameters appear
\begin{equation}\label{k_eq}
f_{k}^\pm=\frac{1}{\Upsilon^{-1} \exp(E/T)\pm 1},\hspace{2mm} T_k<T\leq T_{ch}.
\end{equation}
The need to preserve the total particle number within the distribution introduces an additional parameter $\Upsilon$ called fugacity. 



The fugacity, $\Upsilon(t)\equiv e^{\sigma(t)}$, controls the occupancy of phase space and is necessary once $T(t)<T_{ch}$ in order to conserve particle number.  A fugacity different from $1$ implies an over-abundance ($\Upsilon>1$) or under-abundance ($\Upsilon<1$) of particles compared to chemical equilibrium and in either of these  situations one speaks of chemical non-equilibrium. 

The effect of $\sigma$ is similar after that of chemical potential $\mu$, except that $\sigma$ is equal for particles and antiparticles, and not opposite. This means $\sigma>0$ ($\Upsilon>1$) increases the density of both particles and antiparticles, rather than increasing one and decreasing the other as is common when the chemical potential is associated with conserved quantum numbers.  Similarly, $\sigma<0$ $(\Upsilon<1)$ decreases both. The fact that $\sigma$ is not opposite for particles and antiparticles reflects the fact that both  the number of particles and the number of antiparticles are conserved after chemical freeze-out, and not just their difference.  Ignoring the small particle antiparticle asymmetry their equality reflects the fact that any process that modifies  the distribution would affect both particle and antiparticle distributions in the same fashion.   Such an asymmetry would be incorporated by replacing $\Upsilon\rightarrow \Upsilon e^{\pm\mu/T}$ where $\mu$ is the chemical potential, but we ignore it in this work as the matter antimatter asymmetry is on the order of $1$ part in $10^9$.

 We also emphasize that the fugacity is time dependent and not just an initial condition.  At high temperatures $\Upsilon=1$ and we will find that $\Upsilon<1$ emerges dynamically as a result of the freeze-out process. The importance of fugacity was first introduced in \cite{PhysRevLett.48.1066} in the context of quark-gluon plasma.  Its presence in cosmology was noted in  \cite{Bernstein:1985,Dolgov:1993} but its importance has been largely forgotten and the consequences unexplored in the literature.  



Once the temperature drops below the kinetic freeze-out temperature $T_k$ we reach  the free streaming period where  particle scattering processes have completely frozen out and the resultant distribution is obtained by solving the collisionless Boltzmann equation with initial condition as given by the chemical non-equilibrium   distribution \req{k_eq}.  As already indicated, the two transitions between these three regimes constitute  the freeze-out process -- first we have at $T_{ch}$ the chemical freeze-out and at lower $T_k$ the kinetic freeze-out.


%%%%%%%%%%%%%%%%%%%%%%%%%%%%%%%%%%%%%%%%%%%%%%%%%%%
\subsubsection{Entropy Conservation}
In this section we show that in an FRW Universe and under the assumption of chemical or kinetic equilibrium, the total comoving entropy of all particle species is conserved. More specifically, we will consider a collection of particles with distinct fugacities $\Upsilon_i$, all of which are in kinetic equilibrium at a common temperature $T$.   For the following derivation, it is useful to define $\mu_i=\sigma_i T$.  This gives the expressions a familiar thermodynamic form with $\mu$ playing the role of chemical potential and helps with the calculations, but should not be confused with a chemical potential as discussed above.  

Integration by parts establishes the following identities for the kinetic equilibrium distribution \req{k_eq}
\begin{equation}\label{identities}
s_i=\frac{\partial P_i}{\partial T}=(P_i+\rho_i-\mu_i n_i)/T, \hspace{3mm} n_i=\frac{\partial P_i}{\partial \mu_i}.
\end{equation}
 Using \req{divTmn} and \req{identities}, we calculate $d/dt(a^3s)$ where $s=\sum_i s_i$ is the total entropy density.
\begin{align}\frac{1}{a^3}\frac{d}{dt}(a^3sT)&=\frac{1}{a^{3}}\frac{d}{dt}(a^3(P+\rho-\sum_i \mu_i n_i))\\
&=\dot{P}+\dot{\rho}-\sum_i \left(\dot{\mu_i}n_i+\mu_i\dot{n_i}\right)+3\left(P+\rho-\sum_i \mu_i n_i\right)\dot{a}/a\notag\\
&=\frac{\partial P}{\partial T} \dot{T}+\sum_i\frac{\partial P_i}{\partial \mu_i} \dot{\mu_i}-\sum_i \left(\dot{\mu_i}n_i+\mu_i\dot{n_i}+3\mu_i n_i \dot{a}/a\right)+\nabla_\mu \mathcal{T}^{\mu 0}\notag\\
&=s\dot{T}-\sum_i \left(\mu_i\dot{n_i}+3\mu_i n_i \dot{a}/a\right)\notag\\
&=s\dot{T}- a^{-3}\sum_i\mu_i\frac{d}{dt}(a^3n_i).
\end{align}
Therefore
\begin{align}\label{S_n_eq}
\frac{d}{dt}(a^3s)=&\frac{1}{T}\frac{d}{dt}(a^3sT)-a^3s\frac{\dot T}{T}=-\sum_i\sigma_i\frac{d}{dt}(a^3n_i).
\end{align}
If every particle is either in chemical equilibrium (i.e. $\sigma_i= 0$) or has frozen out chemically, and thus has a conserved comoving particle number, then this implies comoving entropy conservation.  

This observation completely fixes the dynamics of the system in the chemical or kinetic equilibrium regimes.  The dynamical quantities are the scale factor $a(t)$, the common temperature $T(t)$, and the fugacities of each particle species $\Upsilon_i(t)$ that is not in chemical equilibrium.  The dynamics are given by the Einstein equation, conservation of the total comoving entropy of all particle species, and conservation of comoving particle number for each species not in chemical equilibrium (otherwise $\Upsilon_i=1$ is constant)
\begin{equation}\label{eq_dynamics}
H^2=\frac{\rho}{3M_p^2}, \hspace{2mm} \frac{d}{dt}(a^3s)=0,\hspace{2mm} \frac{d}{dt}(a^3n_i)=0 \text{ when } \Upsilon_i\neq 1.
\end{equation}

\subsection{Key Results From our Study of Neutrino Freeze-out}\label{nu_freezeout_summary}
Using the dynamical equations \req{eq_dynamics} we studied the neutrino distribution after freeze-out under the instantaneous equilibrium approximation in the papers \cite{Birrell2013} and \cite{Birrell:2013_2}, attached as appendices \ref{app:chem_freezeout} and \ref{app:model_ind} respectively.  In these works we assumed that the chemical freeze-out occurs before reheating begins and hence the system is in kinetic but not chemical equilibrium from the beginning of reheating until kinetic freeze-out.  In \cite{Birrell2013} we showed numerically that this is the case for the reaction $e^+e^-\rightarrow \nu_e\bar\nu_e$ to high accuracy.  In  \cite{Birrell:2013_2} we characterized the dependence of the neutrino distribution on the kinetic freeze-out temperature $T_k$.  If one is interested in a model where the chemical freeze-out temperature also varies significantly, then the analysis presented in these papers should be repeated with both the chemical and kinetic freeze-out temperatures treated as free parameters. Below we give some of the key results from our analysis.



As discussed in  \cite{Birrell:2013_2}, a deviation from $\Upsilon=1$ and the reheating ratio in the decoupled limit, \req{T_nu_T_gamma}, is a necessary result of the transfer of entropy from the annihilating $e^\pm$ into neutrinos.  In that paper we used conservation laws to analytically derive an approximate relation between the fugacity $\Upsilon=e^\sigma$ and the photon to neutrino temperature ratio
\begin{align}\label{Upsilon_ratio}
\frac{T_\gamma}{T_\nu}&=a\Upsilon^{b}\left(1+c\sigma^2+O(\sigma^3)\right),\\
\label{value_a}
a&=\left(1+\frac{7}{8}\frac{g_{e^\pm}}{g_\gamma}\right)^{1/3}=\left(\frac{11}{4}\right)^{1/3}=R_\nu^{-1}\approx 1.4010,\\
\label{value_b}
b&\approx 0.367,\\
c&\approx -0.0209.
\end{align}
An approximate power law fit was first obtained numerically in \cite{Birrell2013}. In \cite{Birrell:2013_2} we also derived a relation between the effective number of neutrinos and the fugacity $\Upsilon=e^\sigma$ that results from neutrino freeze-out
\begin{equation}\label{N_nu_approx}
N_\nu=\frac{360}{7\pi^4}\frac{e^{-4b\sigma}}{(1+c\sigma^2)^4}\int_0^\infty \frac{u^3}{e^{u-\sigma}+1}du\left(1+O(\sigma^3)\right).
\end{equation}


%%%%%%%%%%%%%%%%%%%%%%%%%%%%%%%%%%%%%%%
\begin{figure}\label{fig:Tk_dependence}
\begin{minipage}{\linewidth}
\makebox[0.5\linewidth]%
{\includegraphics[height=5.8cm]{04-birrell/ModelIndStudy/Upsilon_q.eps}}
\makebox[0.5\linewidth]%
{\includegraphics[height=5.8cm]{04-birrell/ModelIndStudy/N_eff.eps}}
\caption{Dependence of neutrino fugacity (left) and effective number of neutrinos and reheating ratio (right) on the neutrino kinetic freeze-out temperature. We also show the evolution of the deceleration parameter through the freeze-out period (left).}
\end{minipage}
 \end{figure}
%%%%%%%%%%%%%%%%%%%%%%%%%%%%%%%%%%%%%%%


These two papers also contain several figures that show other relations relation between the quantities $T_k$, $N_\nu$, $\Upsilon$, and $T_\gamma/T_\nu$ for which we do not have simple analytic relations. In figure \ref{fig:Tk_dependence} we give slightly modified versions of two of these plots, showing the dependence of $N_\nu$, $\Upsilon$, and $T_\gamma/T_\nu$ on $T_k$.  In particular, the fugacity evolves following the solid black curve in the left hand plot until it reaches the kinetic freeze-out temperature, at which point the neutrinos decouple and $\Upsilon$ remains constant thereafter, as shown in the dashed black curves for two sample values of $T_k$.

We showed in \cite{Birrell:2013_2} that after kinetic freeze-out, the free-streaming neutrino momentum distribution takes the form
\begin{equation}\label{neutrino_dist}
f(t,E)=\frac{1}{\Upsilon^{-1}e^{p/T_\nu}+ 1}
\end{equation}
where the neutrino effective temperature is redshifted as the universe expands
\begin{equation}\label{Tneutrino_dist}
T_\nu(t)\propto \frac{1}{a(t)}
\end{equation}
and the value of the fugacity  that developed during the freeze-out process is frozen into the distribution and remains constant while free-streaming. The resulting expressions for the energy density, pressure, and number density in the rest frame of the neutrino background are
\begin{align}
\rho&=\frac{g_\nu}{2\pi^2}\!\int_0^\infty\!\!\frac{\left(m_\nu^2+p^2\right)^{1/2}p^2dp }{\Upsilon^{-1}e^{p/T_\nu}+ 1},\label{neutrino_rho}\\[0.2cm]
P&=\frac{g_\nu}{6\pi^2}\!\int_0^\infty\!\!\frac{\left(m_\nu^2+p^2\right)^{-1/2}p^4dp }{\Upsilon^{-1} e^{p/T_\nu}+ 1},\label{neutrino_P}\\[0.2cm]
n&=\frac{g_\nu}{2\pi^2}\!\int_0^\infty\!\!\!\frac{p^2dp }{\Upsilon^{-1}e^{p/T_\nu}+ 1}.
\label{num_density}
\end{align}

  Finally, in  \cite{Birrell:2013_2} we presented for the first time a  physically consistent derivation of the equation of state of free-streaming neutrinos, including dependence on both $N_\nu$ and neutrino mass ($\beta=m_\nu/T_\gamma$). 
\begin{align}
&\rho^{EV}/\rho_0= N_\nu+0.1016\sum_i\beta_i^2+0.0015\delta N_\nu\sum_i\beta_i^2\notag\\
&-0.0001\delta N_\nu^2\sum_i\beta_i^2-0.0022\sum_i\beta_i^4,\\
&P^{EV}/P_0= N_\nu-0.0616\sum_i\beta_i^2-0.0049\delta N_\nu\sum_i\beta_i^2\notag\\
&+0.0005\delta N_\nu^2\sum_i\beta_i^2+0.0022\sum_i\beta_i^4.\label{tau_Ups}
\end{align}
The inclusion of fugacity was a crucial aspect in obtaining a physically consistent description, as it was in all of the above results.

\section{Dependence of Neutrino Freeze-out on Parameters}\label{ch:param_studies}
Having developed an improved method for solving the Boltzmann equation and computing scattering integrals that greatly reduces the computational cost, we are now able to characterize the dependence of neutrino freeze-out on parameters.  This will allow us to identify potential avenues by which the tension between observed and theoretical values of $N_\nu$ may be alleviated.  See also our paper \cite{Birrell:2014uka}.

Our study will also us to constrain the time and/or temperature variation of certain natural constants by comparing the results with measurements of $N_\nu$.  The topic of time variation of natural constants is a very active field with a long history. For a comprehensive review of this area, with which we make only slight contact,  see \cite{Uzan:2010pm}.



\subsection{Weinberg Angle}

As mentioned above, the Weinberg angle is one of the standard model parameters that impacts the neutrino freeze-out process.  More specifically, it is found in the matrix elements of weak force processes, including the reactions $e^+e^-\rightarrow \nu\bar\nu$ and $\nu e^\pm\rightarrow \nu e^\pm$ found in tables \ref{table:nu_e_reac} and \ref{table:nu_mu_reac}.  It is determined by the $SU(2)\times U(1)$ coupling constants $g$, $g^{'}$  by
\begin{equation}
\sin(\theta_W)=\frac{g^{'}}{\sqrt{g^2+(g^{'})^2}}.
\end{equation}
It is also related to the mass of the $W$ and $Z$ bosons and the Higgs vacuum expectation value $v$ by
\begin{equation}
M_Z=\frac{1}{2}\sqrt{g^2+(g^{'})^2}v,\hspace{2mm}  M_W=\frac{1}{2}gv,\hspace{2mm} \cos(\theta_W)=\frac{M_W}{M_Z}
\end{equation}
as well as the electromagnetic coupling strength
\begin{equation}
e=2M_W\sin(\theta_W)/v=\frac{gg^{'}}{\sqrt{g^2+(g^{'})^2}}.
\end{equation}
It has a measured value in vacuum $\theta_W\approx 30^\circ$, giving $\sin(\theta_W)\approx 1/2$, but its value is not fixed within the Standard Model. For this reason, a time or temperature variation can be envisioned and this would have an observable impact on the neutrino freeze-out process, as measured by $N_\nu$.

In letting $\sin(\theta_W)$, and hence $g$ and $g^{'}$, vary we must fix the electromagnetic coupling $e$ so as not to impact sensitive cosmological observables such as Big Bang Nucleosynthesis.  Fixing $v$, the smallest $M_W$ can become is when $\sin(\theta_W)=1$, yielding a reduction in $M_W$ by a factor of $2$.  This implies that $M_Z>M_W\gg |p|$ for neutrino momentum $p$ in the energy range of neutrino freeze-out, around $1\MeV$, even as we vary $\sin(\theta_W)$.  This approximation is inherent in the formulas for the matrix elements  in tables  \ref{table:nu_e_reac} and \ref{table:nu_mu_reac} and continues to be valid here. We will characterize the dependence of $N_\nu$ on $\sin(\theta_W)$ in section \ref{sec:param_char} below, but first we identify the remaining parameter dependence in the Einstein Boltzmann system


\subsection{Interaction Strength}
 In order to isolate the dependence of the Einstein Boltzmann system for neutrino freeze-out on dimensioned quantities, we now convert it to dimensionless form. Letting $m_e$ be the mass scale and $M_p/m_e^2$ be the time scale the Einstein equations take the form
\begin{equation}
H^2=\frac{\rho}{3},\hspace{2mm}\dot\rho=-3H(\rho+P).
\end{equation}
 Since $e^\pm$ are the only (effectively) massive particles in the system, by scaling all energies, momenta, energy densities, pressures, and temperatures by $m_e$ we have removed all scale dependent parameters from the Einstein equations.  The Boltzmann equation becomes
\begin{equation}\label{eta_def}
\partial_tf-pH\partial_pf=\eta\frac{C[f]}{E},\hspace{2mm}\eta\equiv M_p m_e^3G_F^2
\end{equation}
where we have also factored out of $C[f]$ the $G_F^2$ term that is common to all of the neutrino interaction matrix elements. 

Aside from the $\theta_W$ dependence of the matrix elements seen in tables \ref{table:nu_e_reac} and \ref{table:nu_mu_reac}, the complete dependence on natural constants  is now contained in a single dimensionless interaction strength parameter $\eta$ with the vacuum present day value,
\begin{equation}\label{eta0_def}
\eta_0\equiv \left.M_p m_e^3 G_F^2\right|_0  = 0.04421 .
\end{equation}
In the following section we characterize the dependence of $N_\nu$ on the interaction strength.  




\subsection{Dependence of $N_\nu$ on Parameters}\label{sec:param_char}

The main result  of this chapter is the  dependence of $N_\nu$ on  the SM parameters   $\sin^2\theta_W$ and $\eta$. These results are shown in  figure \ref{N_nu_params}, presented as a function of  Weinberg angle $\sin^2 \theta_W $ for $\eta/\eta_0=1,2,5,10$. The effects of an increase in both parameters above the vacuum values superpose  in the parameter range  considered, amplifying the effect and generating a significant increase in  $N_\nu\to 3.5$. The present day vacuum value of Weinberg angle puts the $\nu_\mu,\nu_\tau$ freeze-out temperature, seen in the right pane of figure \ref{fig:freezeoutT},  near its maximum value.  This is why a comparatively large change in $\sin^2(\theta_W)$ is needed to produce a change in $N_\nu$ for $\sin^2(\theta_W)\approx0.23$.
 
%%%%%%%%%%%%%%%%%%%%
\begin{figure}%[ht]
\centerline{\includegraphics[width=0.70\columnwidth]{04-birrell/ParametricStudies/N_eff2.eps}
}
\caption{Change in effective number of neutrinos  $N_\nu$ as a function of Weinberg angle for  several values of $\eta/\eta_0=1,2,5,10$. Vertical line is $\sin^2(\theta_W)=0.23$.}
\label{N_nu_params}  
 \end{figure}
%%%%%%%%%%%%%%%%%%%%
We performed a least squares fit of $N_\nu$ over the range $0\leq \sin^2(\theta_W)\leq 1$, $1\leq \eta/\eta_0\leq 10$ shown in figure \ref{N_nu_params}, obtaining a result with relative error less than $0.2\%$,
\begin{align}
N_\nu=&3.003-0.095\sin^2\theta_W +0.222\sin^4\theta_W  -0.164\sin^6\theta_W \notag\\
+&\sqrt{\frac{\eta}{\eta_0}}\left(0.043+0.011\sin^2\theta_W +0.103\sin^4\theta_W\right).
\end{align}
$N_\nu$ is monotonically increasing in $\eta/\eta_0$ with dominant behavior  scaling as $\sqrt{ \eta/\eta_0}$. Monotonicity is to be expected, as increasing $\eta$ decreases the freeze-out temperature and the longer neutrinos are able to remain coupled to $e^\pm$, the more energy and entropy from annihilation is transferred to neutrinos.

We complement this with fits to the photon to neutrino temperature ratios $ T_\gamma / T_{\nu_e}, T_\gamma / T_{\nu_\mu}= T_\gamma / T_{\nu_\tau} $, and the neutrino fugacities, $\Upsilon_{\nu_e}, \Upsilon_{\nu_\mu}=\Upsilon_{\nu_\tau}$, again with relative error less than $0.2\%$  
\begin{align}
\frac{T_\gamma}{T_{\nu_\mu}}=&1.401+0.015x-0.040x^2+0.029x^3-0.0065y+0.0040xy-0.017x^2y, \label{fit1}\\
\Upsilon_{\nu_e}=&1.001+0.011x-0.024x^2+0.013x^3-0.005y-0.016xy+0.0006x^2y,\label{fit2}\\ 
\frac{T_\gamma}{T_{\nu_e}}=&1.401+0.015x-0.034x^2+0.021x^3-0.0066y-0.015xy-0.0045x^2y,\label{fit3}\\
\Upsilon_{\nu_\mu}=&1.001+0.011x-0.032x^2+0.023x^3-0.0052y+0.0057xy-0.014x^2y.\label{fit4}
%N_\nu=&3.003-0.095x+0.222x^2-0.164x^3+0.043y+0.011xy+0.103x^2y
\end{align}
where
\begin{equation}%{align}
x\equiv \sin^2 \theta_W ,\qquad
y\equiv  \sqrt{\frac{\eta}{\eta_0}}.
\end{equation}%{align}

%%%%%%%%%%%%%%%%%%%%
\begin{figure}%[ht]
\centerline{\includegraphics[width=0.75\columnwidth]{04-birrell/ParametricStudies/region_plot_legend.eps}
}
\caption{$N_\nu$ bounds in the $\eta/\eta_0, \sin^2(\theta_W)$ plane. Dark (green) for $N_\nu\in (3.03,3.57)$ corresponding to Ref.\cite{Planck} CMB+BAO analysis and light(teal) extends the region to $N_\nu<3.87$ i.e. to CMB+$H_0$. Dot-dashed line delimits the 1s.d. lower boundary of the second analysis.}
\label{N_nu_domain}
 \end{figure}
%%%%%%%%%%%%%%%%%%%%
The bounds on $N_\nu$ from the Planck analysis \cite{Planck} can be  used to constrain time or temperature variation of $\sin^2\theta_W$ and $\eta$.  
In  Figure \ref{N_nu_domain} the dark (green) color shows the combined range of  variation of natural constants  compatible with CMB+BAO and the light (teal) color shows  the extension in the range of  variation of  natural constants for CMB+$H_0$, both at a $68\%$ confidence level. The dot-dashed line within the dark (green) color  delimits   this latter domain. The dotted line shows the limit of a 5\% change in $N_\nu$.    Any increase in  $\eta/\eta_0$ and/or $\sin^2(\theta_W)$ moves the value of $N_\nu$ into the domain favored by current experimental results. Further parameter study is found in \ref{app:weinberg} and \ref{app:int_strength}.

%%%%%%%%%%%%%%%%%%%%
%%%%%%%%%%%%%%%%%%%%

%%%%%%%%%%%%%%%%%%%%
\subsection{Summary, Discussion, and Conclusions}\label{sec:concl}
We have employed a novel spectral method Boltzmann solver and a new procedure for evaluating the Boltzmann scattering integrals in order to characterize the impact of a potential time and/or temperature variation of SM parameters on the effective number of neutrinos. Specifically, we identified a dimensionless combination of $m_e$, $M_p$, and $G_F$, called the interaction strength $\eta$, that, along with the Weinberg angle $\sin^2 \theta_W$, control neutrino freeze-out and the resulting value of the effective number of neutrinos, $N_\nu$.  
%%%%%%%%%%%%%%%%%%%%%%%%%%%%%%%%%%%%%%%%%%%%%%%%%

%%%%%%%%%%%%%%%%%%%%%%%%%%%%%%%%%%%%%%%%%%%%%%%%%
\subsubsection{Primordial Variation of Natural Constants}
The question which we addressed in this section is: What neutrino decoupling in the early Universe can tell us about the values of natural constants when the Universe was about one second old and at an ambient temperature near to 1 MeV (11.6 billion degrees K). Our results were presented assuming that the Universe contains no other effectively massless particles but the three left handed neutrinos and corresponding, three right handed anti-neutrinos. 

We found that near to the physical value of the Weinberg angle  $\sin^2 \theta_W\simeq 0.23$ the effect of changing $\sin^2\theta_W$ on the decoupling of neutrinos is small. Thus as seen in Figure \ref{N_nu_params}  the dominant variance is due to the change  in the coupling strength $\eta/\eta_0$, \req{eta_def}  and \req{eta0_def}. The dotted line in  Figure \ref{N_nu_domain} shows that in order to achieve a change in $N_\nu$ at the level of up to 5\% that is  $N_\nu\lesssim 3.2 $  both $\sin^2 \theta_W$ and $\eta/\eta_0$ must change significantly, with e.g. $\eta$ increasing by an order of magnitude.

Let us look closer at what an increase in the strength parameter $\eta$ by factor 10 means, looking case by case on all the natural constant contributions as if each were responsible for the entire change:
\begin{itemize}
\item 
Considering that  $\eta\propto M_p\propto G_N^{-1/2}$ this translates into a decrease  in the strength of Gravity at neutrino freeze-out by a factor 100.  This effect would need to become much smaller by the time the age of the Universe is 1000 times longer (1s compared to 10 min) for Big Bang nucleosynthesis to be unaffected. This presumably means that, conversely, as we go further back in time we would need the gravity to continue to rapidly become very much weaker yet. In models of emergent gravity we can  imagine a  `melting' of gravity in the hot primordial Universe. Whether such a model can be realized will be a topic for future consideration. The attractive aspect of Gravity weakening rapidly with increasing temperature is that for  exponentially disappearing $G_N\to 0$ as $t\to 0$ and/or $T\to \infty$ the dynamics can be arranged to be similar to an inflationary  Universe.
\item 
Since $\eta\propto m_e^3$ electron mass would need to go up `only' by factor 2.15 . Compared to all other particles the electron mass has an anomalously  low value. Appearance of a mechanism just when $T\simeq m_e$ that `restores' the electron mass to where intuition would like it to be, a few MeV, arising from  the systematics of other Yukawa Higgs coupling $g_{Ye}$ compared to the Yukawa coupling of other charged light particles, where $m_e= g_{Ye} v $ seems to us also  a possible scenario. Interestingly,   laboratory limits for these conditions could be attainable in the foreseeable future.
\item
Since $\eta\propto G_F^2\propto 1/v^4$  we would need to find a mechanism that would decrease the vacuum value $v_0\simeq 246$ GeV by factor 1.8 already at temperature $T\simeq m_e$.  Allowing three powers of $v$ to cancel by using the Higgs minimal coupling formula for electron mass  we need to change $v$ by an order of magnitude near to $T\simeq m_e$. This appears impossible.
\end{itemize}
While ideas justifying strong variation of $\eta$ can be developed as two of the above three cases argue, a model for temperature or time dependence of  $\sin^2 \theta_W$ seems at this time without a theoretical anchor point, mainly so since we do not have a valid grand unified theoretical framework in which the electro-weak mixing or equivalently the masses $M_W, M_Z$ would be anchored.


%%%%%%%%%%%%%%%%%%%%%%%%%%%%%%%%%%%%%%%%%%%%%%%%%
\subsubsection{Two Different Ways to Change $N_\nu$}

However, there are additional challenges we have not at this time addressed. This is so since the immediate observable is the energy content of the invisible Universe as defined by the effective number of neutrinos $N_\nu$. Considering a value of   $N_\nu>3$,  there could  be a contribution from presently not discovered, more weakly interacting massless particles that decoupled even before neutrinos, and which therefore could contribute fractionally to $N_\nu$, see our discussion in Ref.\cite{Birrell:2014connect}. 

Of particular relevance could be a so called light sterile neutrino~\cite{Abazajian:2012ys}, possibly the right handed complement to the left handed neutrinos. If such particles exist and freeze-out well before regular neutrinos, their contribution to $N_\nu$ would be subject to dilution by reheating~\cite{Birrell:2014connect} and thus their contribution to $N_\nu$ would depend on when precisely they begin free-streaming.

These unknown dark `radiation' particles as well as neutrinos could have a mass that is at the scale of the temperature of photon decoupling $T_{\gamma 0}=0.25$ eV, for which an analysis of the Universe density fluctuations akin to Planck~\cite{Planck} would need to be adapted. We have  discussed in Ref.\cite{Birrell:2013_2} a consistent treatment of neutrino mass and $N_\nu$,  in the case of a particular type of delayed massive neutrino  freeze-out. This approach is exactly the same as would be the case for dark radiation:  Near to $T_{\gamma 0}=0.25$ eV massive neutrinos are indistinguishable from massive dark radiation, which contributes as an additional particle with reduced contribution to $N_\nu$~\cite{Birrell:2014connect}.

The alternative explanation of $N_\nu>3$ in terms of variation of  of natural constants that we have presented comprises  speculative beyond the standard model ideas akin in this aspect to new dark `radiation' particles. We believe that our present contribution provides a viable alternative  mechanism  capable of influencing $N_\nu$. In order to achieve an increase in $N_\nu$ the change in natural constants must cause a delay in neutrino freeze-out and thus  a greater participation of neutrinos in reheating during $e^\pm$ annihilation. The changes in the natural constants  which are required to make a large and visible contribution in $N_\nu$ appear at first sight to reach beyond a variation that one could tolerate at the time of big bang nucleosynthesis only a factor 1000 in time later. We have argued that a change in the electron mass $m_e$ by factor larger than two, and/or Newtons constant $G_N$ even by several orders of magnitude  could be present.




\subsection{Weinberg Angle Plots}\label{app:weinberg}



%%%%%%%%%%%%%%%%%%%%%%%%%%%%%%%%%%%%%%%
\begin{figure}[ht]
\centerline{\includegraphics[height=5.8cm]{04-birrell/ParametricStudies/delta_rho.eps}\hspace{-5mm}\includegraphics[height=5.8cm]{04-birrell/ParametricStudies/upsilon.eps}}
\caption{Fractional increase in neutrino energy (left) and neutrino fugacities (right), as functions of Weinberg angle. Vertical line is $\sin^2(\theta_W)=0.23$.}
 \end{figure}
%%%%%%%%%%%%%%%%%%%%%%%%%%%%%%%%%%%%%%%




%%%%%%%%%%%%%%%%%%%%%%%%%%%%%%%%%%%%%%
\begin{figure}[ht]
\centerline{\includegraphics[height=5.8cm]{04-birrell/ParametricStudies/aT_gamma.eps}\hspace{-5mm}\includegraphics[height=5.8cm]{04-birrell/ParametricStudies/T_ratio.eps}}
\caption{Dependence on Weinberg angle of photon reheating  (left) and photon-neutrino temperature ratios (right) after freeze-out. Vertical line is $\sin^2(\theta_W)=0.23$.}
 \end{figure}
%%%%%%%%%%%%%%%%%%%%%%%%%%%%%%%%%%%%%%%



%%%%%%%%%%%%%%%%%%%%%%%%%%%%%%%%%%%%%%%
\begin{figure}[ht]
\centerline{\includegraphics[height=6.3cm]{04-birrell/ParametricStudies/nu_e_freezeout.eps}\hspace{5mm}\includegraphics[height=6.3cm]{04-birrell/ParametricStudies/nu_mu_freezeout.eps}}
\caption{Freeze-out temperatures for electron neutrinos (left) and $\mu$, $\tau$ neutrinos (right) for various types of processes, as functions of Weinberg angle. Vertical line is $\sin^2(\theta_W)=0.23$.}\label{fig:freezeoutT}
 \end{figure}
%%%%%%%%%%%%%%%%%%%%%%%%%%%%%%%%%%%%%%%

\subsection{Interaction Strength Plots}\label{app:int_strength}



%%%%%%%%%%%%%%%%%%%%%%%%%%%%%%%%%%%%%%%
\begin{figure}[ht]
\centerline{\includegraphics[height=5.8cm]{04-birrell/ParametricStudies/delta_rho_GF.eps}\hspace{-5mm}\includegraphics[height=5.8cm]{04-birrell/ParametricStudies/upsilon_GF.eps}}
\caption{Fractional increase in neutrino energy (left) and neutrino fugacities (right), as functions of interaction strength.}
 \end{figure}
%%%%%%%%%%%%%%%%%%%%%%%%%%%%%%%%%%%%%%%




%%%%%%%%%%%%%%%%%%%%%%%%%%%%%%%%%%%%%%
\begin{figure}[ht]
\centerline{\includegraphics[height=5.8cm]{04-birrell/ParametricStudies/aT_gamma_GF.eps}\hspace{-5mm}\includegraphics[height=5.8cm]{04-birrell/ParametricStudies/T_ratio_GF.eps}}
\caption{Dependence on  interaction strength of photon reheating (left) and photon-neutrino temperature ratios (right) after freeze-out.}
 \end{figure}
%%%%%%%%%%%%%%%%%%%%%%%%%%%%%%%%%%%%%%%



%%%%%%%%%%%%%%%%%%%%%%%%%%%%%%%%%%%%%%%
\begin{figure}[ht]
\centerline{\includegraphics[height=6.3cm]{04-birrell/ParametricStudies/nu_e_freezeout_GF.eps}\hspace{5mm}\includegraphics[height=6.3cm]{04-birrell/ParametricStudies/nu_mu_freezeout_GF.eps}}
\caption{Freeze-out temperatures for electron neutrinos (left) and $\mu$, $\tau$ neutrinos (right) for various types of processes, as functions of interaction strength.}
 \end{figure}
%%%%%%%%%%%%%%%%%%%%%%%%%%%%%%%%%%%%%%%




%%%%%%%%%%%%%%%%%%%%%%%%%%%%%%%%%%%%%%%%%%%%%%%%%%%%%%%%%%%%%%%%%%%%
\section{QGP Hadronization Transition and $N_{eff}$}
Based on \cite{Birrell:2014cja}.
????need to edit???
Phase transitions in the early Universe,  such as the Electroweak and QCD transitions, constitute a drastic change in the properties of the vacuum.  In the case of QCD, the strong symmetry  breaking is accompanied by the presence of relatively heavy Goldstone bosons. It is natural to wonder whether such a transition comprises further and much weaker symmetry breaking, accompanied by low mass (sub eV scale mass) Goldstone bosons, expected to decouple at or near  the phase boundary. We refer to any light weakly coupled  particle species as a `sterile particle' (SP),   generalizing the  sterile neutrino concept-- to avoid misunderstandings we stress that these SPs are not (cold) dark matter  but rather behave as `dark radiation'~\cite{Steigman:2013yua}.

To motivate the assumption that the transformation of the vacuum structure is the origin of SPs, it is best to compare our discussion with Ref.~\cite{Weinberg:2013kea}, where a concrete but yet to-be-discovered model is proposed.  In contrast, here we consider the latest  phase transformation in the early Universe and evaluate quantitatively  the production and freeze-out of possible SPs in such a transition. If SPs are interpreted as Goldstone bosons, it would imply that in the deconfined phase there is  an additional hidden symmetry, weakly broken at hadronization.  For example, if this symmetry were to be part of the baryon conservation riddle, then we can expect that these Goldstone bosons will  couple to particles with baryon number, and possibly only in the domain where the vacuum is modified from its present day condition. 

Another viable candidate for SPs are sterile neutrinos. It was shown that the freeze-out temperature required for three `new' right-handed neutrinos to fully account for the effective number of neutrinos, $N_{\text{eff}}$  (see the following section),  is in the vicinity of the quark gluon plasma (QGP) phase transition~\cite{Anchordoqui:2011nh,Anchordoqui:2012qu}.  The former paper proposed a concrete model of how this might be obtained from an expanded gauge group for QCD.  However,  the QGP equation of state (EoS) which were used are not consistent with recent numerical lattice-QCD results.   We use the  lattice-QCD derived QGP EoS from Ref.~\cite{Borsanyi:2013bia} to characterize the relation between $N_{\text{eff}}$ and the number of DoF that froze out at the time that the quark-gluon deconfined phase froze into hadrons near $T=150\MeV$, and  compute the coupling strength required to achieve freeze-out at the QGP transformation. 

The question of the validity of the symmetry breaking model~\cite{Anchordoqui:2011nh}  is far from resolved, and other extensions of strong interactions   are present in the literature, see for example \cite{Georgi:1989gp}. The cancellation seen in lattice simulations between QED and QCD CP-odd terms~\cite{Bonati:2013hsa,D'Elia:2012zw}  provides concrete evidence of a possible connection between the QED and QCD sectors, showing that the theory of strong interactions is not fully understood and can contain symmetry breaking outside the known realm.

Even a very weakly coupled SP, should it be associated with symmetry breaking below and at QGP hadronization, could be seen in several experiments, a point we develop in this work:\\
a)  An experimental motivation for prior interest is the analysis of CMB temperature fluctuations, such as by the Planck satellite collaboration (Planck)~\cite{Planck}, especially the observed tension in the effective number of neutrinos, $N_{\text{eff}}$. $N_{\text{eff}}$ is constrained by the expansion of the Universe and includes the effects of all light sub-eV mass particles present in the Universe, such as our SPs.\\
b)   By applying methods of kinetic theory, we obtain the minimal coupling strength of SPs required to maintain chemical equilibrium down to the time of  hadronization of cosmological or laboratory formed QGP.  We argue that in this situation a significant fraction of the total energy of the QGP phase could be unaccounted for, carried out by SPs  that are `invisible'  in the confined vacuum.  This result motivates a closer study of the energy balance and expansion dynamics in laboratory QGP experiments.

Our discussion of the role of SPs in understanding the early Universe expansion complements and competes with other explanations of the tension in  $N_{\text{eff}}$, which has already inspired various theories, including the consideration of:
i)  a model in which the temperature of neutrino decoupling was a variable parameter \cite{Birrell:2013_2};
ii) a very light neutralino that freezes-out prior to muon annihilation~\cite{Dreiner:2011fp};
iii) the production of sterile neutrinos after the QCD transition~\cite{Lello:2014yha};
iv) as noted, the introduction of Goldstone bosons associated with a new spontaneously broken symmetry that freeze out prior to the disappearance of muons~\cite{Weinberg:2013kea}. 
 
The last two cases are examples of the general mechanism whereby ultra-weakly interacting particles of any type that freeze-out in an earlier epoch of the Universe, such as our SPs, make a contribution to $N_{\text{eff}}$ that depends on the decoupling temperature~\cite{Anchordoqui:2011nh,Anchordoqui:2012qu,Blennow:2012de,Steigman:2013yua}. This  naturally results in a fractional contribution to $N_{\text{eff}}$. After decoupling, SPs do not participate in the reheating process, in which the entropy of a disappearing  particle component is transferred into the remaining components. The noticeably lower temperature of SPs, compared to the reference particle (photon), means they have a smaller contribution to thermal pressure and energy, an effect measured by $N_{\text{eff}}$, resulting in a fractional contribution to the `neutrino' DoF. \\[-0.2cm]


%%%%%%%%%%%%%%%%%%%%%%%%%%%%%%%%%%%%%%%%%%%%%%%%%%%%%
\subsection{Effective Number of Neutrinos:}
$N_{\text{eff}}$ quantifies the amount of radiation energy density, $\rho_r$, in the Universe prior to photon freeze-out and after $e^\pm$ annihilation and is defined by $\rho_r=(1+(7/8)R_\nu^{4}N_{\text{eff}})\rho_\gamma$, where $\rho_\gamma$ is the photon energy density and  $R_\nu\equiv T_\nu/T_\gamma=({4}/{11})^{1/3}$ is the photon to neutrino temperature ratio in the limit where no entropy from the annihilating $e^\pm$ pairs is transferred to neutrinos.  The factor 7/8 is the ratio of Fermi to Bose reference normalization in $\rho$ and the neutrino to photon temperature ratio $R_\nu$ is the result of the transfer of $e^\pm$ entropy into photons after Standard Model (SM) left handed neutrino freeze-out.

If photons and SM left-handed neutrinos are the only significant massless particle species in the Universe between the freeze-out of left-handed neutrinos at  $T_\gamma=\mathcal{O}(1)$ MeV and photon freeze-out at $T_\gamma=0.25$ eV, and assuming zero reheating of neutrinos, then $N_{\text{eff}}=3$, corresponding to the number of SM neutrino flavors  by definition.  A numerical computation of the neutrino freeze-out process employing SM two body scattering interactions and carried out using the Boltzmann equation framework presented in~\cite{Dicus:1982bz} gives $N_{\text{eff}}^{\rm th}=3.046$~\cite{Mangano2005}, a value close to the number of flavors.  

The value of $N_{\text{eff}}$ can be  measured by fitting to observational data, such as the distribution of CMB temperature fluctuations. The  Planck~\cite{Planck}  analysis gives $N_{\text{eff}}=3.36\pm 0.34$ (CMB only) and $N_{\text{eff}}=3.62\pm 0.25$ (CMB+$H_0$) ($68\%$ confidence levels). With more dedicated CMB experiments forthcoming and an analysis that can self consistently account for any additional particle inventory in the early universe, it is believed that a significantly more precise value of $N_{\text{eff}}$  will be available in the next decade. \\[-0.2cm]


%%%%%%%%%%%%%%%%%%%%%%%%%%%%%%%%%%%%%%%%%%%%%%%%%%%%%
\subsection{Contribution to $N_{\text{eff}}$ of a Sub-eV  Mass SPs:}  The Einstein equations  imply a practically entropy conserving expansion of the Universe. Entropy conservation during periods when dimensional (mass) scales are irrelevant means that all temperatures scale inversely with the metric  scale factor $a(t)$. As temperature  passes through $m\simeq T$ thresholds, successively less massive particles annihilate and their entropy is shifted into the remaining effectively massless particles, causing the  $T\propto 1/a(t)$ scaling to break down. 


After an effectively massless particle species decouples, its temperature scales as $1/a(t)$ at all later times as a result of the free-streaming solution of the Einstein-Vlasov equation.  This leads to a temperature difference between the free streaming particles, and the photon background, which is the last to freeze-out. This reheating effect builds up during each  period  in which particle species disappear from the Universe inventory.

We denote by $S$ the conserved `comoving' entropy in a volume element $dV$, which scales with the factor $a(t)^3$. We define the effective number of entropy DoF, $g_*^S$, by
\begin{equation}
S=\frac{2\pi^2}{45}g^S_*T_\gamma^3 a^3.
\end{equation} 
For ideal Fermi and Bose gases
\begin{equation}
g_*^S=\!\!\!\!\sum_{i=\text{bosons}}\!\!\!\!g_i \left(\frac{T_i}{T_\gamma}\right)^3\!\!\!f_i^-+\frac{7}{8}\!\!\!\sum_{i=\text{fermions}}\!\!\!\! g_i \left(\frac{T_i}{T_\gamma}\right)^3\!\!\!f_i^+.
\end{equation}
The $g_i$ are degeneracies, $f_i^\pm$ are known functions, valued in $(0,1)$, that turn off the various species as the temperature drops below their mass-- compare to the analogous Eqs. (2.3) and (2.4) in Ref.\cite{Blennow:2012de}. 


%%%%%%%%%%%%%%%%%%%%%%%%%%%%%%%%%%%%%%%
\begin{figure}\label{fig:gs}
\centering
\begin{minipage}[b]{.49\textwidth}
\centerline{\hspace*{-0.10cm}\includegraphics[height=6.4cm]{04-birrell/QGP_Neff/Figures/gS_T_ratio.eps}}
\end{minipage}
\caption{\cccite{Birrell:2014cja}. Left axis: Effective number of entropy-DoF, including lattice QCD effects applying Ref.~\cite{Borsanyi:2013bia} (solid line) and  Ref.~\cite{Bazavov:2014pvz} (circles), compared to the early Ref.\cite{Bazavov:2009zn} (triangles) results used by~\cite{Anchordoqui:2011nh}, and the ideal gas model of Ref.\cite{Coleman:2003hs} (dashed line) as function of temperature $T$. Right axis: Photon to SP temperature ratio, $T_\gamma/T_s$, as a function of SP decoupling temperature (dash-dotted (blue) line). The vertical dotted lines at $T=142$ and 163 MeV delimit the QGP transformation region.\label{fig:gS}}
 \end{figure}
%%%%%%%%%%%%%%%%%%%%%%%%%%%%%%%%%%%%%%%

Such a simple characterization does not hold in the vicinity of the QGP phase transformation where  quark-hadron degrees of freedom are strongly coupled  and the system must be studied using lattice QCD. This result is incorporated in the solid line in figure \ref{fig:gS} (left axis), where we have used a table of entropy density values through the QGP phase transition presented by Borsanyi et al.~\cite{Borsanyi:2013bia}, while circles show recent results from Bazavov et al.~\cite{Bazavov:2014pvz}. This should be compared to the ideal gas approximation from~\cite{Coleman:2003hs} together with the fit in~\cite{Wantz:2009it} to interpolate though the QGP phase transition and older (year 2009) lattice data from Ref.\cite{Bazavov:2009zn} (triangles). The free gas approximation carries with it a maximum error of $10\%$ in the temperature range of the QGP phase transition  $T\simeq 150$\,MeV where quarks appear.  The error in the 2009 lattice data used in Ref.\cite{Anchordoqui:2011nh} is on the order of $25\%$.  This leads to a non-negligible difference in the relation between freeze-out temperature and $N_{\text{eff}}$.

%%%%%%%%%%%%%%%%%%%%%%%%%%%%%%%%%%%%%%%
\begin{figure}
\centering
\begin{minipage}[b]{.49\textwidth}
\centerline{\hspace*{0.4cm}\includegraphics[height=6.8cm]{04-birrell/QGP_Neff/Figures/Neff_Td_combined.eps}}
\end{minipage}
\caption{\cccite{Birrell:2014cja}. Solid lines: Increase in $\delta N_{\text{eff}}$ due to the effect of $1,\dots,6$ light sterile boson DoF ($g_s=1,\dots,6$, bottom to top curves) as a function of freeze-out temperature $T_{d,s}$. Dashed lines: Increase in $\delta N_{\text{eff}}$ due to the effect of $1,\dots,6$ light sterile fermion DoF ($g_s=7/8\times 1,\dots,7/8\times 6$, bottom to top curves) as a function of freeze-out temperature $T_{d,s}$. The horizontal dotted lines correspond to $\delta N_{\text{eff}}+0.046=0.36,0.62,1$. The vertical dotted lines show the reported range of QGP transformation temperatures $T_c=142-163\MeV$.\label{fig:Neff_Td_zoom}}
\end{figure}
%%%%%%%%%%%%%%%%%%%%%%%%%%%%%%%%%%%%


Once the SPs decouple from the  particle inventory at a photon temperature of $T_{d,s}$, a difference in their temperature from that of photons will build up during subsequent photon reheating periods, as discussed above. Conservation of entropy leads to a temperature ratio at $T_\gamma<T_{d,s}$, shown in the dot-dashed line in figure \ref{fig:gS} (right axis), of
\begin{equation}\label{T_ratio}
R_s\equiv T_{s}/T_{\gamma}=\left(\frac{g_*^S(T_\gamma)}{g_*^S(T_{d,s})}\right)^{1/3}.
\end{equation}


If $T_s$ and $T_\gamma$ are the light SP and photon temperatures, both after $e^\pm$ annihilation, and $g_s$ is the number of DoF of the SPs normalized to bosons (i.e. for fermions it includes an additional factor of $7/8$) then this gives
\begin{equation}\label{Neff1}
\delta N_{\text{eff}}\equiv N_{\text{eff}}-3.046=\frac{4g_s}{7}\left(\frac{T_s}{R_s T_{\gamma}}\right)^4
\end{equation}
where $3.046$ is the SM neutrino contribution. Using \req{T_ratio} we can rewrite $\delta N_{\text{eff}}$ as
\begin{equation}\label{delta_N}
\delta N_{\text{eff}}=\frac{4g_s}{7R_\nu^4}\left(\frac{g_*^S(T_{\gamma})}{g_*^S(T_{d,s})}\right)^{4/3}.
\end{equation}
where   $T_{d,s}$ is the decoupling temperature of the SP and $T_{\gamma}$ is any photon temperature $T_{\gamma}\ll m_e$. The SM particles remaining at $T_{\gamma}$ are  photons and SM neutrinos, the latter with temperature $R_\nu T_{\gamma}$, and so $g_*^S(T_{\gamma})=2+7/8\times 6\times 4/11$ and (see also Eq.(2.7) in~\cite{Blennow:2012de})
\begin{align}\label{delta_N2}
\delta N_{\text{eff}}\approx&g_s\left(\frac{7.06}{g_*^S(T_{d,s})}\right)^{4/3}.
\end{align}

Figure \ref{fig:Neff_Td_zoom} shows $\delta N_{\text{eff}}$ as a function of $T_{d,s}$ for $1,\dots,6$ boson (solid lines) and fermion (dashed lines) DoF. For a low decoupling temperature $T_{d,s}<100$\,MeV  a single bose or fermi SP  can help alleviate the tension in $N_{\text{eff}}$. Within QGP hadronization interval $T_c=142-163\MeV$ (marked by vertical lines) we see that three bose degrees of freedom or four fermi degrees of freedom are the most likely cases to resolve the tension. 

It is also clear from  figure \ref{fig:Neff_Td_zoom} that the rapid growth of the number of degrees of freedom in the QGP phase implies that earlier decoupling temperatures lead to a rapid increase in the required number of SPs. While one cannot exclude the possible presence of 20--30 new dark light particles, it seems to us unlikely that there are that many undiscovered weakly broken symmetries producing light Goldstone, or/and sterile neutrino-like particles. We believe that figure \ref{fig:Neff_Td_zoom}  pinpoints the QGP  temperature range and below as the primary domain of interest for the freeze-out of such hypothetical degrees of freedom, should these be responsible for the modification $\delta N_{\text{eff}}$.


We emphasize that this model independent result is only valid in the instantaneous decoupling limit.  This limit holds if there is a sharp cutoff in the interaction of the SPs with matter, perhaps associated with a change in vacuum structure, but otherwise a computation of the freeze-out process using the Boltzmann equation with two body scattering is needed for quantitatively precise results in a given model.  We discuss one such example in the following section.\\[-0.2cm]


%%%%%%%%%%%%%%%%%%%%%%%%%%%%%%%%%%%%%%%%%%%%%
{\bf Chemical Equilibration of SPs:}
Using the Boltzmann equation solver presented in \cite{Birrell:2014uka}, we can compute the minimum coupling required within the QGP phase to maintain chemical equilibrium in the early Universe.  We define the chemical freeze-out temperature as the temperature at which the scattering length exceeds some $\mathcal{O}(1)$ multiple of the Hubble length, as discussed in~\cite{Birrell:2014uka}.   

At the present level of precision, we find it reasonable to take an illustrative model for the interactions, wherein we think of the SPs as if they were sterile neutrinos and model their  interaction and  freeze-out in a manner similar that of the left-handed neutrinos at $T_f=\mathcal{O}(1\MeV)$~\cite{Mangano2005}.  However, in the QGP phase with $T_f\simeq 150$\,MeV, in addition to leptons we also include  coupling to the more abundant quarks. To account for quarks, in this first estimate of the coupling strength we have effectively increased  the number of active degrees of freedom in our computation by the appropriate amount.  In such a model of SP freeze-out, the strength of the interaction is controlled by a modified Fermi constant $  G_{SP}=C G_F$. 


In a cosmological setting, the lower bound for $C$ that will assure that SPs remain in chemical equilibrium until the confining QGP transition into regular matter at $T=\mathcal{O}(150\MeV)$ is approximately
\begin{equation}
C\gtrsim 10^{-3}, \qquad   G_{SP}^{-1/2}\lesssim 9\,\mathrm{TeV}.
\end{equation}
This large $\mathrm{TeV}$  energy scale  for the coupling of e.g. sterile neutrinos seems reasonable and renders such particles within a range that can perturb experimental laboratory data. 

A much greater coupling $C$ is required to assure that a chemical equilibrium abundance of SPs is achieved in the short lifespan of QGP formed in laboratory  heavy ion collisions. In order to model this situation we need a   temperature profile. We assume a simple model $T\tau =T_0\tau_0$. We choose  $\tau_0=8\, 10^{-24}$s  so that the temperature  falls from $T_0=600\MeV$ to $T=150\MeV$ in   $2.4\cdot 10^{-23}$s. We estimate the freeze-out temperature by comparing the scattering length to $\dot{T}/T$. With this, the limit on the required coupling is
\begin{equation}
C\gtrsim 3\, 10^6. \quad \quad   G_{SP}^{-1/2}\lesssim  170\,\mathrm{MeV}.
\end{equation}
The appearance of a coupling on the order of the QCD scale  is consistent with the intuition about the interaction strength that is required for   particles to reach chemical equilibrium in laboratory  QGP experiments.\\[-0.2cm]

%%%%%%%%%%%%%%%%%%%%%%%%%%%%%%%%%%%%%%%%%%%%%
{\bf QGP Signature of SP Production}:  As we have seen, for QCD-scale coupling, SPs associated with the deconfined phase transition could be produced abundantly in laboratory relativistic heavy ion experiments, saturating the volume occupied by the QGP with their practically massless yield.  However, to be consistent with the present day invisibility of SPs, their interaction with other particles must only turn on in the domain where the vacuum is modified at finite temperature, analogous to the enhancement of anomalous baryon-number non-conservation at GUT scale temperatures~\cite{Kuzmin:1985mm}.  Therefore, in our proposed scenario, the QGP transition must be associated with a sharp cutoff of the coupling, and thus  scattering cross section,  of the SPs.

Reinspecting the results we have presented,  in particular the  figures, we note the best $N_{\text{eff}}$ constraint suggests a multiplicity of $3\pm 2$ SPs at QGP hadronization,  at which point there are about $g_*^S=25$ strong interaction (entropy) degrees of freedom. This means that approximately $12\pm 8\%$ of all entropy content of the QGP  is within the escaping SPs.  Moreover, since SP's stop interacting at the QGP surface they can escape freely during the entire lifespan of the QGP. As a consequence, the energy loss could be even greater.

We recognize  that the likely loss of energy and entropy could be substantial. A full model of this dynamical process is beyond the present discussion.  However, a quarter or more of the energy brought in by heavy ions into the space-time domain could literally `evaporate'. A systematic exploration of the thermal energy in the QGP fireball  is presented in tables 8 and 9 in Ref.\cite{Letessier:2005qe}. This study did not consider the kinetic energy due to collective matter flow.  However, near   the QGP formation threshold   the  kinetic flow energy component should be small. Inspecting the fireball thermal energy per baryon content near this threshold, we note missing energy of the  here estimated magnitude:   only 75\% of the energy per baryon is found in the visible QGP reaction products.


There is another, indirect, signature of the SPs, should they be strongly coupled only within the domain of the QGP phase. We recall that lattice-QCD results predict a relatively small continuous pressure for the QGP phase near the transformation, a situation consistent with absence of a phase transition~\cite{Borsanyi:2013bia}. However, SPs contribute to the pressure internal to QGP, scattering from QGP partons, yet  not in the external region. Therefore, by including SPs there is now a pressure discontinuity at the QGP surface.  This restores the appearance of a phase transition between the deconfined and confined domains and drives the QGP expansion.  In this way, SPs could also be observed through their indirect, dynamical effect on the flow of matter, including particle $v_2$, the  dynamical azimuthal asphericity~\cite{Ollitrault:1992bk,Voloshin:1994mz},  imparting this effect on all components of the QGP, including heavy quarks.\\[-0.2cm]


%%%%%%%%%%%%%%%%%%%%%%%%%%%%%%%%%%%%%%%%%%%%%%%%%%%%%
\subsection{Discussion:} 
The natural concordance of the reported CMB range of $N_{\text{eff}}$ with the range of QGP hadronization temperatures, as seen in figure \ref{fig:Neff_Td_zoom}, motivates the exploration of a connection between $N_{\text{eff}}$ and the decoupling of SPs at and below  the QGP phase transition.  Our analysis uses a QCD equation of state derived from by Borsanyi et al.~\cite{Borsanyi:2013bia}, which improves on the approximately $25\%$ error in the lattice data used in the prior characterization of this effect in Ref.\cite{Anchordoqui:2011nh}, see figure~\ref{fig:gs}.  From \req{delta_N}, an underestimation of the degrees of freedom by $25\%$ results in approximately $45\%$ error in $\delta N_{\text{eff}}$, making this is a significant update to the prior study.

 Additional motivation for our focus on QGP comes from the recognition that the transition into a confining vacuum involves symmetry breaking, and therefore is a natural place to look for new Goldstone bosons. The sharp drop in matter density after hadronization also makes this a natural period to look for the freeze-out of other particles, such as sterile neutrinos.

We estimated the minimal strength of the coupling of  novel particles, such as sterile neutrinos, that would lead to decoupling during the QGP transition in the early Universe. We further considered the possibility of SPs that are coupled much more strongly, but exclusively within the context of the deconfined vacuum structure.  Under this hypothesis the strength of the coupling could be governed by the QCD scale. We argued that such SPs, possibly a novel type of Goldstone bosons, would have considerable observable impact on both energy balance in the formation of the QGP phase, and on its dynamical evolution.


In summary, we can say that   $N_{\text{eff}}>3.05$ can be associated  with the appearance of several  (best fit a total of $3\pm2$) light  particles at QGP hadronization in the early Universe that either are weakly interacting in the entire space or are allowed to interact only inside the deconfined domain, in which case their coupling would be strong. Such particles could leave a clear experimental signature in, for example, relativistic heavy ion experiments that produce the deconfined QGP phase.




%%%%%%%%%%%%%%%%%%%%%%%%
% Chapters from Christopher Grayson's dissertation
\include{05-grayson/chap01intro}
\chapter{Dynamic response of quark-gluon plasma to electromagnetic fields}\label{chap:QCD}

 In this chapter, we discuss the application of the ultrarelativistic limit of the polarization tensor in Chapter \ref{chap:PlasmaSF} to the electromagnetic properties of quark-gluon plasma (QGP), as found in \cite{Grayson:2022asf}. QGP is an extreme state of matter composed of free quarks and gluons, which occurs in the aftermath of colliding nuclei in particle accelerators and existed a few microseconds after the big bang \cite{Letessier:2002ony}. 

The electromagnetic fields generated by colliding relativistic heavy ions in particle colliders are some of the largest in the known Universe, on the order of $ec|B| \approx m_\pi^2$, but exist for very short times $t_{\text{coll}}= 2 R/\gamma \sim 10^{-25}\,\textrm{s}$ due to the Lorentz contraction of the colliding nuclei. The magnetic field generated in these collisions is interesting due to its role in separating electric charge in the QGP through the chiral magnetic effect (CME) \cite{Kharzeev:2007jp}. The electric current generated by the CME could lead to a charge separation along magnetic field lines. If a magnetic field survives in QGP until the time of hadronization of the QGP, which we will refer to as the freeze-out time $t_f$, it could also lead to a difference in the global polarization of $\Lambda$ hyperons and antihyperons \cite{Muller:2018ibh}. Charge separation in the hadron was recently studied in \cite{PhysRevX.14.011028}. 

\begin{figure}[h!]
    \centering
    \includegraphics[width=0.85\linewidth]{plots/chap02QCD/Bfield.png}
    \caption{The vacuum magnetic field for two colliding lead Pb nuclei is shown for impact parameter $b=3R$ and $\gamma =37$. (At larger Lorentz factors, a graphical representation is difficult to visualize without scaling the fields with $\gamma$). The vector potential is plotted in the collision plane, and red arrows indicate the direction of the moving nuclei. This plot mainly shows the magnetic field distribution, which is Lorentz contracted along the direction of motion. The magnetic field lines circulate out of the collision plane perpendicular to the velocity, adding together at the collision center.  }
    \label{fig:vacmag}
\end{figure}

The distribution of the vacuum magnetic field given by the Li\'enard-Wiechert fields is plotted in \reff{fig:vacmag}. This is the same magnetic field found by Lorentz boosting the Coulomb field of a nucleus at rest. We neglect the portion of the field that depends on acceleration since it is small for vacuum scattering of heavy nuclei, compared to the field that depends on velocity.

This magnetic field is treated as an external perturbation on the quark-gluon plasma, filling the overlap region between the two nuclei after they collide. For simplicity, the QGP is modeled as an infinite medium so that complications do not arise at the boundary. The temperature of QGP depends strongly on the collision energy of the nuclei. In \cite{Grayson:2022asf} we study Au+Au collisions at $\sqrt{s_{\text{NN}}}=200\,$GeV with QGP temperature $T=300$\,MeV.  After Heavy Ions collide, the conducting QGP medium generates long-range decaying tails or wakefields in the magnetic field that extend far beyond the collision time \cite{Tuchin:2010vs}. The conductivity of QGP determines the strength of these wakefields. We aim to model these fields in QGP using the formulation discussed in Chapter \ref{chap:PlasmaSF}.

\section{EM conductivity of quark-gluon plasma}

Past analytic calculations \cite{Tuchin:2010vs,Deng:2012pc,McLerran:2013hla,Tuchin:2013apa,Gursoy:2014aka,Li:2016tel,Roy:2015kma} solve Maxwell's equations in the presence of static electric conductivity 
\begin{equation}
   \sigma_0 = \frac{m_D^2}{3\kappa}\,,
\end{equation} 
in a  hydrodynamically evolving QGP. For a collisionless plasma $\kappa\rightarrow0$, the conductivity is infinite, and the medium behaves as a perfect conductor. This work introduces the frequency and wavevector dependence of the QGP analytically using the polarization tensor previously obtained in \cite{Formanek:2021blc}.

Numerical calculations \cite{Inghirami:2016iru, Inghirami:2019mkc} have incorporated the dynamical response of QGP by numerically solving the coupled magneto-hydrodynamic equations for a conducting quark-gluon plasma in the presence of the colliding nuclear charges. More recent calculations \cite{Yan:2021zjc,Wang:2021oqq} also incorporate the frequency and wave-vector dependence of QGP response to electromagnetic fields by solving the coupled Vlasov-Boltzmann--Maxwell equations numerically.



\subsection{The Ultrarelativistic EM polarization tensor in QGP}\label{sec:linresp}

In this Section, we review the ultrarelativistic polarization tensor, including damping, for the idealized case where the QGP is infinite, homogeneous, and stationary. This calculation differs from \cite{Formanek:2021blc} only in that we consider three quark species: up, down, and strange. We start with the Vlasov-Boltzmann equation for each quark flavor \req{eq:boltzmanncov} where we assume all quarks collide on a momentum-averaged time scale $\tau_{\text{rel}} = \kappa^{-1}$. The induced current $ j_{\mathrm{ind}}^\mu$ can be written in terms of the phase-space distribution of quarks and anti-quarks as
\begin{equation}\label{eq:current}
   j_{\mathrm{ind}}^\mu(x) = 2 N_c \int (dp)p^\mu \\ \times \sum_{u,d,s} q_f (f_{f}(x,p) - f_{\bar{f}}(x,p))=  4 N_Q e^2 \int (dp)p^\mu \delta f(x,p)\,,
\end{equation}
where  $N_c$ is the number of colors. We sum over the quark flavors with charges $q_f$, and in the final result we replace $q_f \delta f = \delta f_f$. The result \req{eq:current} differs from that found in the case of an electron-positron plasma by the factor
\begin{equation}
N_Q \equiv N_c\sum_f (q_f/e)^2 = 2\,,
\end{equation}
for three light quark flavors ($u,d,s$).

In the ultrarelativistic limit, neglecting quark masses, one finds the polarization functions \cite{Formanek:2021blc}:
\begin{align}\label{eq:polfuncsUltra}
&\Pi_{\parallel}(\omega,|\boldsymbol{k}|) = m_D^2\frac{\omega^2}{\boldsymbol{k}^2}\left(1 - \frac{\omega \Lambda}{2|\boldsymbol{k}|-i\kappa \Lambda}\right)\,,\\
&\Pi_{\perp}(\omega,|\boldsymbol{k}|) = \frac{m_D^2\,\omega}{4 |\boldsymbol{k}|}\left( \Lambda \left(\frac{\omega'^2}{\boldsymbol{k}^2} - 1\right) - \frac{2\omega'}{ |\boldsymbol{k}|}\right)\,,
\end{align}
where $\Lambda(\omega,\boldsymbol{k})$ is defined as
\begin{align}\label{eq:definitions}
 \Lambda \equiv \ln \frac{\omega'+  |\boldsymbol{k}|}{\omega'- |\boldsymbol{k}|}\,, \quad \text{with} \quad \omega' = \omega+i\kappa\,.
\end{align}
The parallel and transverse polarization functions have the same form as in \cite{Formanek:2021blc} except for an overall factor $N_Q$  as found in \cite{Kapusta:1992fm, Grayson:2022asf}:
\begin{equation}\label{eq:DebyemQCD}
    {m_D}^2_{(\text{EM})} = \sum_{u,d,s} q^2_f T^2 \frac{N_c}{3} = N_Q\frac{e^2T^2}{3} \equiv C_{\text{em}}T^2\,,
\end{equation}
where $C_{\text{em}} =  2e^2/3$. In the following, we will use $m_D$ as short-hand notation for the electromagnetic screening mass since we do not discuss color screening here.
The transverse conductivity $\sigma_{\perp}$, which controls the response of the plasma to magnetic fields, is related to the imaginary part of the transverse polarization function as in \req{eq:sigmaperp}



\subsection{QCD Damping rate in QGP}

The strength of the plasma response to an external magnetic field depends on the quark damping rate $\kappa$ and the electromagnetic screening mass $m_D$. The scale of the collisional quark damping $\kappa$ is much larger than the electromagnetic Debye mass $m_D$ and electromagnetic damping $\kappa_{\text{EM}}$, because it depends on the strong coupling constant $\alpha_s$, not the electromagnetic coupling $\alpha$.

In \cite{Grayson:2022asf}, we use the first-order electromagnetic Debye mass \req{eq:DebyemQCD} to estimate the electromagnetic screening mass $m_D$. The collision rate $\kappa$ is related to the inverse of the mean-free time of quarks in QGP. We adopt a value for $\kappa$ from \cite{Mrowczynski:1988xu} where the mean-free time is given by the product of the parton density in the QGP and the quark-parton transport cross-section, leading to the expression 
\begin{equation}\label{eq:kappadef}
    \kappa(T) = \frac{10}{17\pi} (9 N_f +16) \zeta(3) \alpha_s^2 \ln\left(\frac{1}{\alpha_s}\right) T\,,
\end{equation}

\begin{figure}[h!]
    \centering
    \includegraphics[width=0.85\linewidth]{plots/chap02QCD/kappaDEBYE.png}
    \caption{\textit{From \cite{Grayson:2022asf}.} Plot of the electromagnetic Debye mass and the QCD dampening rate $\kappa$ as a function of temperature. At temperature $T=300\,$MeV used here, $\kappa = 4.86\, m_D$.\label{fig:kappaDebye}}
\end{figure}

where $N_f$ is the number of flavors, $\zeta(x)$ denotes the Riemann zeta function, and $\alpha_s(T)$ is the running QCD coupling.  We model the running of the QCD coupling constant as a function of temperature in the range $T<5T_c$ using a fit provided in \cite{Letessier:2002ony}:
\begin{equation}\label{eq:alphas}
    \alpha_s(T) \approx \frac{\alpha_s(T_c)}{1+C \ln(T/T_c)}\,,
\end{equation}
where $C=0.760 \pm 0.002$. For the QCD (pseudo-)critical temperature we use $T_c = 160\,$MeV. The QED Debye mass is compared to $\kappa(T)$ in Fig.~\ref{fig:kappaDebye}. 
This is plotted along with the electromagnetic Debye mass in \reff{fig:kappaDebye}. We can expect the electromagnetic response of QGP response to be over-damped since $\kappa> \frac{2}{\sqrt{3} m_D}$ giving a plasma frequency \req{eq:plasmafreq} which is imaginary over the range of temperatures relevant for QGP.


\begin{figure}[h!]
    \centering
    \includegraphics[width=0.85\linewidth]{plots/chap02QCD/condcomp.png}
    \caption{The black line shows the static conductivity $\sigma_0$ as a function of temperature predicted by \req{eq:condstat}, which is compared to lattice results adapted from \cite{Aarts:2020dda} for $T>T_c$. The factor of $C_{\text{em}}$, defined in \req{eq:DebyemQCD}, normalizes the conductivity by the charge of the plasma constituents, such that results using different numbers of dynamical quark flavors can be compared. We indicate each set of points by its arXiv reference: blue diamonds \cite{Amato:2013naa, Aarts:2014nba}, green circles \cite{Brandt:2015aqk}, and red triangles \cite{Astrakhantsev:2019zkr}.}
    \label{fig:lattice comp}
\end{figure}


We can then use the Debye mass \req{eq:DebyemQCD} and the damping rate \req{eq:kappadef} to calculate the static conductivity \req{eq:condstat}, shown as a black line in \reff{fig:lattice comp}, which we then compare to Lattice calculations of the conductivity in QGP.


These lattice-QCD results \cite{Amato:2013naa, Aarts:2014nba,Brandt:2015aqk,Astrakhantsev:2019zkr} are scaled with temperature $T$ to remove the linear temperature dependence. We also scale the conductivity with $C_{\text{em}}$, as defined in \req{eq:DebyemQCD}, such that computations with different numbers of flavors can be compared. One can see that the conductivity value predicted by \req{eq:kappadef}, plotted in Fig.~\ref{fig:lattice comp} as a black line, lies well within the lattice-QCD results. We will use the value predicted by \reff{fig:lattice comp}, $\sigma = 5.01\,$MeV at $T=300\,$GeV, in the next section to compute the screened heavy ion fields in QGP.

% Here, we focus on the damping effect of the collision rate $\kappa$ on the induced magnetic fields in nuclear collisions. The collision term generates a nonvanishing conductivity via the imaginary part of the polarization tensor. This conductivity manifests itself in poles of the resummed propagator in the lower complex $\omega$-plane that generate long-range tails or wake fields that extend far beyond the collision time. In Table~\ref{tab:timescales} we collect the relevant timescales of the problem in ascending order.  The collision time $t_{\text{coll}}$ is much shorter than all other relevant time scales. The collective oscillations of the plasma are highly damped by the large value of the relaxation time giving rise to over-damped behavior.

% \begin{table}[h!]
% \caption{\label{tab:timescales}
% Approximate time scales relevant to the electromagnetic response of QGP for an Au+Au collisions at $\sqrt{s_{\text{NN}}}=200\,$GeV with QGP temperature $T=300$\,MeV. Time scales are shown in ascending order.}
% \begin{ruledtabular}
% \begin{tabular}{ccc}
% \textrm{Time Scale}&
% \textrm{Formula}&
% \multicolumn{1}{c}{\textrm{Time (fm/c)}}\\
% \colrule
%  \textrm{Collision Time} & $t_\text{coll} = 2 R/\gamma$ & 0.086\footnote{Calculated using the Gaussian radius $R = 4.33$\,fm defined in \req{eq:radius}.} \\
% \textrm{Relaxation Time} & $\tau_\text{rel} = 1/\kappa$ & 0.36 \\

% \textrm{Freeze-out time} & $t_f$ & $5$\footnote{Estimated using 2+1 dimensional hydrodynamic evolution \cite{Song:2007ux}.}\\
% \textrm{ Decay Time} & $t_{\sigma} = 1/\sigma_0 = \kappa/\omega_p^2$ & 59\footnote{The decay time is the large damping $\kappa/\omega_p$ expansion of the plasma oscillation frequency \req{eq:plasmafreq}. } \\
% \end{tabular}
% \end{ruledtabular}
% \end{table}



%%%%%%%%%%%%%%%%%%%%%%%%%%%%%%%%%%%%%%%%%%%%%%%%%%%%%%%%%%%%%%%%%%%%%%%%%%%%%%%%%%%%%%%%%%%%%%%%

\section{Magnetic field in QGP during a nuclear collision}\label{sec:Maxwell2}
Assuming that the QGP is an infinite homogeneous and stationary medium near equilibrium, we can solve Maxwell's equations for the self-consistent fields as in Section~\ref{sec:Maxwell}. Then the magnetic field is given by Fourier transforming the momentum space expressions given in \reqs{eq:aperp}{eq:ftfields} to position space
\begin{equation}\label{eq:magorgin}
   \boldsymbol{B}(t, z) = \int \frac{d^4k}{(2\pi)^4}  e^{-i\omega t+ik_z z}
 \frac{\mu_0 i \boldsymbol{k} \times\ft{j}_{\perp \text{ext}}(\omega, \boldsymbol{k})}{\boldsymbol{k}^2 - \omega^2 - \mu_0 \Pi_{\perp}(\omega, \boldsymbol{k})}\,.
\end{equation}
We choose the collision center as the origin of our spatial coordinate system and align the spatial $z$-axis with the beam direction. Due to the symmetry of the colliding ions, the only nonzero component of the magnetic field along the $z$-axis points out of the collision plane ($x-y$ plane). In our coordinate system used in \cite{Grayson:2022asf}, this corresponds to the $y$-component of the magnetic field. 

For ease of calculation, we specify the external 4-current using two colliding Gaussians charge distributions normalized to the nuclear rms radius $R$ and charge $Z$:
\begin{equation}\label{eq:rhoext}
\rho_{\text{ext}\pm }(t,\boldsymbol{x}) = \frac{Zq\gamma}{\pi^{3/2}R^3}e^{-\frac{1}{R^2}(x\mp b/2)^2}e^{-\frac{1}{R^2}y^2}
\times e^{-\frac{\gamma^2}{R^2}(z\mp \beta t)^2}\,,
\end{equation}
where $\gamma$ is the Lorentz factor, $\beta$ is the ratio of the ion speed to the speed of light, respectively, and $b$ is the impact parameter of the collision. The plus and minus signs indicate motion in the $\pm \hat{z}$-direction (beam-axis). This charge distribution corresponds to the vector current
\begin{equation}\label{eq:jext}
\boldsymbol{j}_{\text{ext}\pm}(t, \boldsymbol{x}) = \pm \beta \hatv{z} \rho_{\text{ext}\pm}(t, \boldsymbol{x})\,.
\end{equation}
Further details of the external charge distribution for two colliding nuclei are presented in Appendix B. of \cite{Grayson:2022asf}.

The numerical result for the position-space magnetic field found by Fourier transforming \req{eq:magorgin} using the full transverse polarization function \req{eq:polfuncsUltra} is shown as a red dashed line in Fig.~\ref{fig:bfcomp} and compared with various models of conductivity. These other models and their connections to published works are discussed in detail in \cite{Grayson:2022asf}.

\phantom{Phantom text}
\begin{figure}[h]
\centering              
\includegraphics[width=0.46\linewidth]{plots/chap02QCD/bf100.png}
%}
\hspace{0.05\linewidth}
\includegraphics[width=0.44\linewidth]{plots/chap02QCD/bf100lin.png}
%}
\caption{\textit{From \cite{Grayson:2022asf}.} The magnetic field at the collision center as a function of time, with $T = 300$\,MeV for Au-Au collisions ($Z=79$) at $\sqrt{s_\text{NN}} = 200$\,GeV and impact parameter $b = 6.4\,$fm. The left panel shows the magnetic field on a semi-logarithmic scale up to $ct = 5$\,fm. The right panel shows the early-time magnetic field on a linear scale. At the chosen temperature, the electromagnetic Debye mass is $m_D = 74\,$MeV, and the quark damping rate is $\kappa = 4.86\,m_D$. This gives a static conductivity of $\sigma_0 = 5.01\,$MeV. Comparing the different approximations, we see they all have similar asymptotic behavior. Only the Drude conductivity, the light-cone limit of the conductivity, and the full conductivity $\sigma_\perp(\omega,\boldsymbol{k})$ describe the field at early times. Note that the plasma is considered homogeneous and stationary here. In a more realistic situation, the field would become screened only after the QGP is formed in the collision.\label{fig:bfcomp}}
\end{figure}

One of the important results of this paper was that the fields of the ions, traveling near the speed of light, probe the polarization tensor along the light cone. The transverse conductivity on the light cone is
\begin{equation}\label{eq:lightcone}
    \sigma_\perp (\omega = |\boldsymbol{k}|)  =  i \frac{m_D^2}{4 \omega}\left( \frac{\kappa^2}{\omega^2} \xi \ln\xi +\frac{i\kappa}{\omega}\left(\xi+1\right)\right)\,,
\end{equation}
where $\xi$ is defined as
\begin{equation}\label{eq:xidef}
    \xi \equiv 1- 2i \frac{\omega}{\kappa}\,.
\end{equation}
The light-cone conductivity simplifies the calculation of plasma response since it only depends on a single variable ($\omega = |\boldsymbol{k}|$). One can see that \req{eq:lightcone} shown as an opaque grey line traces out the full numerical solution \req{eq:magorgin} shown as a dashed red line. The light-cone conductivity accurately models the magnetic field in QGP since the ions traveling near the light's speed only sample the polarization tensor on the light-cone. One subject of future research is to use the light-cone conductivity to attain analytical formulas for electromagnetic fields in position space in light-cone coordinates.


The simplest method to calculate the late-time magnetic field of colliding nuclei is to assume a static conductivity \cite{Tuchin:2013apa}. In this case, the magnetic field in Fourier space has the form
\begin{equation}\label{eq:bstat}
    \ft{B}(\omega,\boldsymbol{k}) = \frac{ \mu_0 i\boldsymbol{k} \times \ft{j}_{\perp \text{ext}}}{\boldsymbol{k}^2 - \omega^2 - i\omega\sigma_0}\,,
\end{equation}
which is Fourier transformed using contour integration in the appendix of \cite{Grayson:2022asf} to
\begin{equation}\label{eq:banalyticapp}
   B_y(t) = -\mu_0 \frac{ Zq \beta }{(2\pi)} \frac{ b\sigma_0}{4t^2} e^{\frac{-b^2 \sigma_0}{16 t}}\,.
\end{equation}
Looking at the left panel of Fig.~\ref{fig:bfcomp}, the static conductivity initially overestimates the magnetic field after the external field begins to disappear since the effect of dynamic screening is not captured. Every model of the response function predicts similar values for the magnetic field approaching the freeze-out time $t_f \approx 5\,$fm/c \cite{Song:2007ux}. This is because the static conductivity determines the dependence of the magnetic field at times later than $t>1/\sigma \approx 59$\,fm/c after which damping of the initial magnetic field pulse is irrelevant. 

Alternatively, by assuming a point-like charge distribution $R\rightarrow0$ and approximating the magnetic field for $ 1/\sigma_0 > t\gg 1/\kappa$ one can derive the late-time magnetic field using the Drude conductivity \req{eq:drude}
\begin{equation}\label{eq:latetimeB}
   B_y(t) \approx  \mu_0 \frac{ Ze \beta b \kappa \omega_p }{8\pi}\bigg[ \frac{1- e^{-\kappa t}}{\kappa t} - e^{-\kappa t} \text{Ei}\left(t\kappa\right)\bigg]\,.
\end{equation}
This result, derived in Appendix \ref{sec:magf}, has the advantage of accurately describing the late-time magnetic field $t>t_f$  at large $\gamma$ as shown in \reff{fig:bcolcomp}.

Both these results illustrate that the late-time magnetic field has a finite limit when $\gamma\rightarrow\infty$ as it depends only on $\beta$, but not on $\gamma$.
\begin{figure}
\centering
\includegraphics[width=0.85\linewidth]{plots/chap02QCD/bfgaamacomp.png}
    \caption{\textit{Adapted from \cite{Grayson:2022asf}.} Plot of the freeze-out magnetic field for $T= 150$\,MeV. We expect that around this temperature QGP will hadronize into a mixed phase \cite{Letessier:1992xd}. The approximate late time solution \req{eq:banalyticapp} shown as an orange dashed line is compared to numerical calculations using the full polarization tensor \req{eq:magorgin} and to the late time analytic expansion \req{eq:latetimeB}. The approximate solution does not fully match the ultrarelativistic limit until times $t > t_{\sigma} \approx 59$\,fm/c. The magnetic field is independent of the beam energy over a wide range of $\gamma$ but begins to diverge slowly from the ultrarelativistic case at around $\gamma \leq 15$ for the time window shown in the figure. Lower beam energies result in a somewhat larger field at late times.\label{fig:bcolcomp}}
\end{figure}
The approximation used to derive this solution holds for $\gamma\beta \gg \sqrt{ \kappa/\sigma_0} \approx 12$. In Fig.~\ref{fig:bcolcomp} we compare \req{eq:banalyticapp} to the full numerical result to explore its dependence on $\gamma$.  One can see that the static case \req{eq:banalyticapp} (black solid line) begins to diverge from the numerical solution, shown as dashed colored lines at around $\gamma \approx 15$.  In Fig.~\ref{fig:bcolcomp} one can see that the late-time magnetic field has a very soft dependence on collision energy. The time at which hadronization occurs $t_f$, which varies with collision energy, has a much stronger effect on the magnitude of the freeze-out field. Since the remnant magnetic field at hadronization does not depend strongly on the collision energy, an experimental measurement of the magnetic field at different collision energies could permit a determination of the electrical conductivity of the QGP or a determination of the freeze-out time of QGP if the conductivity is assumed to be known. 

As the QGP begins to hadronize at time $t_f$, one may expect hadrons to be statistically polarized with respect to the magnetic field. In \cite{Muller:2018ibh} the measured difference in global polarization of hyperons and antihyperons is used to give an upper bound on the magnetic field at QGP freeze-out, $B \sim 2.7\times 10^{-3}\,m_{\pi}^2$ for Au+Au collisions at $\sqrt{s_\text{NN}} = 200$\,GeV. Looking at Fig.~\ref{fig:bcolcomp} the magnetic field for $\gamma = 100$ at QGP freeze-out $t_f \approx 5 $\,fm/c is predicted to be $B \approx 1.2\times 10^{-3}\,m_{\pi}^2$, somewhat below this upper bound. Note that this assumes the polarization rapidly equilibrates in the plasma. It also neglects any interactions during the hadron gas phase of the collision. 
%%%%%%%%%%%%%%%%%%%%%%%%%%%%%%%%%%%%%%%%%%%%%%%%%%%%%%%%%%%%%%%%%%%%%%%%%%%%%%%%%%%%%%%%%%%%%%%%%%%%%%%%%%%%%%%%%%%%

% \section{The QGP Polarization tensor}\label{sec:linresp}
% \subsection{Derivation of the Polarization tensor}

% In this Section we derive the polarization tensor, including damping, for the idealized case where the QGP is homogeneous and stationary. We follow the derivation presented in \cite{Formanek:2021blc} for the damped polarization tensor of an electron-positron plasma. The calculation differs slightly from \cite{Formanek:2021blc}, since in QGP we consider three quark species: up, down, and strange. We start from the Vlasov-Boltzmann equation for each quark flavor:
% \begin{equation}\label{eq:VBE}
% (p \cdot \partial) f_f(x,p) + q_f F^{\mu\nu} p_\nu \frac{\partial f_f(x,p)}{\partial p^\mu} = (p\cdot u)C_f(x,p)\,,
% \end{equation}
% The collision term $C_f(x,p)$ in the BGK form is given by
% \begin{equation}\label{eq:collision}
%     C_f(x,p) =\kappa_f\left(\eq{f}_f (p)\frac{n_f(x)}{{\eq{n}_f}} - f_f(x,p)\right)\,,
% \end{equation}
% where plasma constituents collide on a momentum-averaged time scale $\tau_{\text{rel}} = \kappa^{-1}$. The collision term is constructed such that \req{eq:VBE} retains current conservation \cite{Bhatnagar:1954zz}. We show in Sect.~\ref{sec:energymomcons} that energy is also conserved for the case of a neutral particle-antiparticle plasma at linear order in the external field.

% The induced current $ j_{\mathrm{ind}}^\mu$ can be written in terms of the phase-space distribution of quarks and anti-quarks as
% \begin{multline}\label{eq:current}
%    j_{\mathrm{ind}}^\mu(x) = 2 N_c \int (dp)p^\mu \\ \times \sum_{u,d,s} q_f (f_{f}(x,p) - f_{\bar{f}}(x,p))\,,
% \end{multline}
% where  $N_c$ is the number of colors, and we sum over the quark flavors with charges $q_f$. One can calculate the induced current for small perturbations away from equilibrium for each quark flavor
% \begin{equation}\label{eq:perturbation}
% f_f(x,p) = {\eq{f}_f}(p) + \delta f_f(x,p)\,,
% \end{equation}
% Note that the equilibrium contributions ${\eq{f}_f}(p)$ do not contribute to \req{eq:current} because of the opposite sign of the charges of particles and antiparticles, but the perturbations $\delta f$ add up due the change in sign of the external force $qF^{\mu\nu}p_\nu$:
% \begin{multline}\label{eq:current2}
%     j_{\mathrm{ind}}^\mu(x) = 2 N_c \int (dp)p^\mu \sum_{u,d,s} q_f (\delta f_{f}(x,p) - \delta f_{\bar{f}}(x,p))\\
%  = 4 N_c \int (dp)p^\mu \sum_{u,d,s} q_f^2 \delta f(x,p)\\
%   =  4 N_Q e^2 \int (dp)p^\mu \delta f(x,p)\,.
% \end{multline}
% In the second line we pulled out a factor of electric charge $\delta f_{f} = q_f \delta f$. The perturbations $\delta f$ are identical for all quark species in the ultrarelativistic limit. The result \req{eq:current2} differs from that found in the case of an electron-positron plasma by the factor
% \begin{equation}
% N_Q \equiv N_c\sum_f (q_f/e)^2 = 2\,,
% \end{equation}
% where the numerical value holds for three light quarks flavors ($u,d,s$). We refer to \cite{Formanek:2021blc} for the derivation of the polarization tensor in terms of integrals over the phase-space distribution $\delta f$, because the only difference is the overall factor $N_Q$.

% As noted in the previous Section, the polarization tensor in \req{eq:poltensgen} may be written in terms of two independent components: the longitudinal polarization function $\Pi_{\parallel}$, which describes response parallel to wave-vector $\boldsymbol{k}$, and the transverse polarization function $\Pi_{\perp}$, which describes response in the plane perpendicular to wave-vector $\boldsymbol{k}$. When the $\mu=3$ ($z$) axis is chosen along the wave-vector $\boldsymbol{k}$, the longitudinal and transverse polarization functions relate to the components of the polarization tensor \req{eq:poltenmat} along the coordinate axes as
% \begin{equation}\label{eq:piLT}
%     \Pi_{\parallel} =\Pi^3_3, \quad \Pi_{\perp} =\Pi^1_1=\Pi^2_2\,.
% \end{equation}
% In the ultrarelativistic limit, neglecting quark masses, one finds \cite{Formanek:2021blc}:
% \begin{align}\label{eq:polfuncs}
% &\Pi_{\parallel}(\omega,|\boldsymbol{k}|) = m_D^2\frac{\omega^2}{\boldsymbol{k}^2}\left(1 - \frac{\omega \Lambda}{2|\boldsymbol{k}|-i\kappa \Lambda}\right)\,,\\
% &\Pi_{\perp}(\omega,|\boldsymbol{k}|) = \frac{m_D^2\,\omega}{4 |\boldsymbol{k}|}\left( \Lambda \left(\frac{\omega'^2}{\boldsymbol{k}^2} - 1\right) - \frac{2\omega'}{ |\boldsymbol{k}|}\right)\,,
% \end{align}
% where $\Lambda(\omega,\boldsymbol{k})$ is defined as
% \begin{align}\label{eq:definitions}
%  \Lambda \equiv \ln \frac{\omega'+  |\boldsymbol{k}|}{\omega'- |\boldsymbol{k}|}\,, \quad \text{with} \quad \omega' = \omega+i\kappa.
% \end{align}
% The natural logarithm leads to branch cut in the complex $\omega$ plane running from $-|\boldsymbol{k}|-i\kappa$ to  $|\boldsymbol{k}|-i\kappa$ as noted in \cite{Romatschke:2015gic}. The parallel and transverse polarization functions have the same form as in \cite{Formanek:2021blc} except for an overall factor $N_Q$ that is contained in the leading order electromagnetic Debye mass for the QGP plasma \cite{Kapusta:1992fm}:
% \begin{equation}\label{eq:DebyemQCD}
%     {m_D}^2_{(\text{EM})} = \sum_{u,d,s} q^2_f T^2 \frac{N_c}{3} = N_Q\frac{e^2T^2}{3} \equiv C_{\text{em}}T^2\,,
% \end{equation}
% where $C_{\text{em}} =  2e^2/3$. In the following we will use $m_D$ as short-hand notation for the electromagnetic screening mass since we do not discuss color screening here.

% The polarization tensor may be written in any general frame by using \req{eq:poltensgen}, but for our purposes it will be simpler to carry out calculations in the coordinate system where $\boldsymbol{k}$ aligns with the $z$-axis so that the polarization tensor takes the form shown in \req{eq:poltenmat}.


%  \subsection{QGP parameters}
 
% The strength of the plasma response to an external magnetic field depends on the values of two physical parameters: the quark damping rate $\kappa$, and the electromagnetic screening mass $m_D$. In this Section we provide estimates for these parameters. 

% We adopt the perturbative result \req{eq:DebyemQCD} to estimate $m_D$. Higher-order corrections to this expression can been derived from higher-order calculation of the vector spectral function in thermal perturbation theory (see \cite{Jackson:2019mop} and references cited therein).

% The scale of the collisional quark damping $\kappa$ is much larger than the electromagnetic Debye mass $m_D$ because it depends on the strong coupling constant $\alpha_s$, not the electromagnetic coupling $\alpha$. Solving the dispersion relation
% \begin{equation}
%     \frac{1}{(k\cdot u)^2}(k^2+ \mu_0\Pi_\parallel(\omega, k))(k^2 + \mu_0 \Pi_\perp(\omega, k))^2=0 \,,
% \end{equation}
% see \cite{melrose2008quantum}, in the limit $\boldsymbol{k}\rightarrow 0$ one finds for the plasma oscillation frequency \cite{Formanek:2021blc}
% \begin{equation}\label{eq:plasmafreq}
%     \omega_{p}^\pm = -\frac{i\kappa}{2} \pm \sqrt{\frac{m_D^2}{3} - \frac{\kappa}{4}^2}\,.
% \end{equation}
% We see that if $\kappa > \tfrac{2}{\sqrt{3}}m_D $, the plasma oscillations are over-damped.

% \begin{figure}[h!]
%     \centering
%     \includegraphics[width=0.85\linewidth]{plots/chap02QCD/kappaDEBYE.png}
%     \caption{Plot of the QED Debye mass and the QCD dampening rate $\kappa$ as a function of temperature. At temperature $T=300\,$MeV used in the plots below, $\kappa = 4.86\, m_D$.\label{fig:kappaDebye}}
% \end{figure}

% The collision rate $\kappa$ is related to the inverse of the mean-free time of quarks in QGP. In kinetic theory the mean-free time is given by the product of the parton density in the QGP and the quark-parton transport cross section, leading to the expression \cite{Mrowczynski:1988xu}
% \begin{equation}\label{eq:kappadef}
%     \kappa(T) = \frac{10}{17\pi} (9 N_f +16) \zeta(3) \alpha_s^2 \ln\left(\frac{1}{\alpha_s}\right) T\,,
% \end{equation}
% where $N_f$ is the number of flavors, $\zeta(x)$ denotes the Riemann zeta function, and $\alpha_s(T)$ is the running QCD coupling.  We model the running of the QCD coupling constant as a function of temperature in the range $T<5T_c$ using a fit provided in \cite{Letessier:2002ony}:
% \begin{equation}\label{eq:alphas}
%     \alpha_s(T) \approx \frac{\alpha_s(T_c)}{1+C \ln(T/T_c)}\,,
% \end{equation}
% where $C=0.760 \pm 0.002$. For the QCD (pseudo-)critical temperature we use $T_c = 160\,$MeV. The QED Debye mass is compared to $\kappa(T)$ in Fig.~\ref{fig:kappaDebye}. 

% From $\kappa(T)$ in \req{eq:kappadef} and the running of the coupling in \req{eq:alphas}, we calculate the static conductivity using the leading order electromagnetic Debye mass $m_D$. The momentum dependent transverse conductivity $\sigma_{\perp}$, which controls the response of the plasma to magnetic fields, is related to the imaginary part of the transverse polarization function $\Pi_{\perp}$ as follows \cite{melrose2008quantum}:
% \begin{equation}\label{eq:conddef}
%     \sigma_{\perp}(\omega,\boldsymbol{k}) = -i \frac{\Pi_{\perp}(\omega,\boldsymbol{k})}{\omega}\,.
% \end{equation}
% In the long wavelength limit $\boldsymbol{k}\rightarrow0$, the branch cut in \req{eq:definitions} shrinks to a single pole at $\omega = -i \kappa$, and the conductivity has the simple form
% \begin{equation}\label{eq:conddrude}
%     \sigma_{\perp}(\omega, 0 ) = \sigma_{\parallel}(\omega, 0 ) = \frac{\sigma_0}{1-i\omega/\kappa}\,.
% \end{equation}
% We will refer to $\sigma_{\perp}(\omega, 0 )$ as the Drude model \cite{Drude:1900}. In the static limit $\omega\rightarrow0$ the parallel and perpendicular conductivities are the same, and the static conductivity $\sigma_0$ is given by
% \begin{equation}\label{eq:condstat}
%    \sigma_0 = \frac{m_D^2}{3\kappa}\,.
% \end{equation} 
%  The static conductivity determines the late time behavior of the magnetic field.

% %%%%%%%%%%%%%%%%%%%%%%%%%%%%%%%%%%%%%%%%%%%%%%%%%%%%%%%%%%%%%%%%%%%%%%%%%%%%%%%%%%%%%%%%%

% \subsection{Energy-momentum conservation}\label{sec:energymomcons}

% In general, the modified BGK collision term \req{eq:collision} violates energy and momentum conservation. Rocha {\it et al.} \cite{Rocha:2021zcw} recently showed how energy-momentum conservation can be restored by introducing a linearized collision operator that is projected on eigenfunctions of the conserved quantities with eigenvalue zero. Here we show explicitly that for a symmetric particle-antiparticle plasma the energy momentum violations cancel at linear order in the external field. 

% Recall that the energy momentum tensor $T^{\mu\nu}$ of the plasma is given by
% \begin{equation}
%     T^{\mu \nu} = 2 \int (dp) p^{\mu} p^{\nu} (f_-(x,p) + f_+(x,p) )\,,
% \end{equation}
% where the factor of two accounts for spin and $f_\pm(x,p)$ represent the distributions of particles ($+$) and antiparticles ($-$), respectively. We recall that for $T^{\mu\nu}$ to be conserved the covariant divergence 
% \begin{equation}
%     \partial_{\mu} T^{\mu \nu} = 2\,\partial_{\mu} \int (dp)p^{\mu} p^{\nu}\left(f_-(x,p) + f_+(x,p)\right)
% \end{equation}
% must vanish. In linear response the distribution functions $ f_{\pm} (x,p)$ are given by
% \begin{equation}
%     f_{\pm} (x,p) = \delta  f_{\pm} (x,p) + \eq{f}(p)\,.
% \end{equation}
% Equation \req{eq:current2} indicates that the perturbation $ \delta  f_{\pm}$ is linear in the quark charge
% \begin{equation}
%     \delta f_\pm = \pm q\, \delta f\,.
% \end{equation}
% This leads to a cancellation of the particle and antiparticle perturbations in the energy-momentum tensor at linear order:
% \begin{equation}
%     \partial_{\mu} T^{\mu \nu} = 4\,\partial_{\mu}\left( \int (dp)p^{\mu} p^{\nu}\eq{f}(p)\right) = 0\,.
% \end{equation}
% Thus for a symmetric particle-antiparticle plasma corrections to the energy-momentum tensor appear only at second order in external field. This is a general consequence of CPT symmetry of the medium. 


% %%%%%%%%%%%%%%%%%%%%%%%%%%%%%%%%%%%%%%%%%%%%%%%%%%%%%%%%%%%%%%%%%%%%%%%%%%%%%%%%%%%%%%%%%%%%%%%%%%%


% \section{Magnetic field in a nuclear collision}\label{sec:results}

% In this Section we calculate the magnetic field at the center of the heavy ion collision by Fourier transforming the momentum space magnetic field \req{eq:ftfields} to position space. We calculate the self-consistent magnetic field using the potentials given in \reqs{eq:phi}{eq:aperp} and model the response of QGP using the idealized case of a homogeneous, stationary plasma detailed in Sects.~\ref{sec:Maxwell2} and \ref{sec:linresp}. The external fields are specified by the moving Gaussian charge distributions defined in \reqs{eq:rhoext}{eq:jext}.

% The magnetic field is of particular interest due to its role in the separation of electric charge in the QGP through the chiral magnetic effect (CME) \cite{Kharzeev:2007jp}. In the large magnetic fields that occur in heavy ion collisions the electric current generated by the CME could lead to a charge separation along the direction of the magnetic field. Whether this effect is observable depends strongly on the size of the magnetic field. If a magnetic field of meaningful strength survives until the time of hadronization of the QGP, it could also lead to a difference in the global polarization of $\Lambda$ hyperons and antihyperons \cite{Muller:2018ibh}.

% We chose the collision center as the origin of our spatial coordinate system and align the spatial $z$-axis with the beam direction. We calculate the magnetic field along the $z$-axis by Fourier transforming the momentum space expressions given in \reqs{eq:aperp}{eq:ftfields}:
% \begin{multline}\label{eq:magorgin}
%    \boldsymbol{B}(t, z) = \int \frac{d^4k}{(2\pi)^4}  e^{-i\omega t+ik_z z}
%  \frac{\mu_0 i \boldsymbol{k} \times\ft{j}_{\perp \text{ext}}(\omega, \boldsymbol{k})}{\boldsymbol{k}^2 - \omega^2 - \mu_0 \Pi_{\perp}(\omega, \boldsymbol{k})}
% \end{multline}
% to position space. It is convenient to perform the Fourier integrals in cylindrical coordinates $(\boldsymbol{x}_\perp,z)$. The angular integral $d\theta$ and the integral over momentum along the beam axis $d k_z $ can be performed exactly. The $d k_z $ integral is trivial due to the delta function in the external charge distribution \req{eq:extchgfreq}. The frequency integral $d\omega$ and the transverse momentum integral $dk_\rho$ must, in general, be done numerically. We present the details of this calculation in Appendix \ref{sec:magf}. Due to the symmetry of the colliding ions, the only nonzero component of the magnetic field along the $z$-axis points out of the collision plane ($x-y$ plane). In our coordinate system, described in Appendix \ref{sec:freechg}, this corresponds to the $y$-component of the magnetic field. The numerical results for the position-space magnetic field are shown in Fig.~\ref{fig:bfcomp} and compared with earlier results.

% To connect to these previous studies, we compute the magnetic field in position space at the origin in various levels of approximation defined in \reqs{eq:conddef}{eq:condstat} and \req{eq:lightcone}.
% \begingroup
% \renewcommand{\arraystretch}{1.5} % Default value: 1
% \begin{table}[b]
% \caption{\label{tab:cond}
% Conductivity models used to calculate the resulting magnetic field. Each conductivity represents the response of QGP with a different spacetime dependence. }
% \begin{ruledtabular}
% \begin{tabular}{ccc}
% \textrm{Conductivity}&
% \textrm{Dependence}&
% \multicolumn{1}{c}{\textrm{Formula}}\\
% \colrule
%  \textrm{Full} & $\sigma_{\perp}(\omega,\boldsymbol{k})$ & $-i \Pi_{\perp}(\omega,\boldsymbol{k})/\omega$  \\
%  \textrm{Light-cone} & $\sigma_{\perp}(\omega = |\boldsymbol{k}|)$ & \req{eq:lightcone}  \\
% \textrm{Drude} & $\sigma_{\perp}(\omega, 0 )$ & $\sigma_0/(1-i\omega/\kappa)$ \\
% \textrm{Static} & $\sigma_0$ & $m_D^2/(3\kappa)$ \\
% \end{tabular}
% \end{ruledtabular}
% \end{table}
% \endgroup
% These conductivities, collected in Table\,\ref{tab:cond}, refer to different treatments of the frequency $\omega$ and wave-vector $\boldsymbol{k}$ dependence of the conductivity $\sigma_\perp(\omega, \boldsymbol{k})$. For instance, solving for the magnetic field in the limit $\boldsymbol{k}\rightarrow0$ assumes that the spatial dependence of the external field can be neglected, not superficially a good approximation because at any given time $t$ the field varies rapidly with $z$.  The levels of approximation we consider include: the full space- and time-dependence of the conductivity $\sigma_\perp(\omega, \boldsymbol{k})$, the Drude model \req{eq:conddrude}, and the static response $\sigma_\perp(0,0)$. We list these limits in \reqs{eq:conddef}{eq:condstat}, respectively. 

% The fourth limit we are considering is the conductivity along the light-cone $\sigma_\perp(\omega,k_z=\pm\omega,k_\rho=0)$. We now show that the light-cone limit closely resembles the Drude model. We first recall that the frequency dependence of the free charge distribution in cylindrical coordinates \req{eq:extchgfreq} has the form
% \begin{multline}
% \wt{\rho}_{\text{ext}\pm}(\omega,\boldsymbol{k}) = 2\pi Zq\, e^{-(k_{\rho}^2 + k_z^2/\gamma^2)\frac{R^2}{4}} \\
% \times e^{\mp \frac{ik_{\rho} b \cos\theta }{2}} \delta(\omega \mp k_z \beta)\,.
% \end{multline} 
% After performing the Fourier transform over the parallel component of the wave-vector $k_z$ using the delta function, the magnitude of the wave-vector $|\boldsymbol{k}|$ is effectively set to the light-cone $\omega \approx |\boldsymbol{k}|$, with a small deviation due to the transverse dependence of the field,
% \begin{equation}
%     |\boldsymbol{k}|^2 = k_z^2+ k_{\rho}^2 \rightarrow  (\omega/\beta)^2+ k_{\rho}^2\,.
% \end{equation}
% Inspecting the external charge distribution after this replacement
% \begin{multline}
% \wt{\rho}_{\text{ext}\pm}(\omega,k_{\rho}) = 2\pi Zq\, e^{-(k_{\rho}^2 + \omega^2/(\beta \gamma)^2)\frac{R^2}{4}} \\
% \times e^{\mp \frac{ik_{\rho} b \cos\theta }{2}}\,,
% \end{multline}
% we can see that the size of the deviation from the light-cone due to $k_{\rho}$ is or order $O(1/R)$, while the width of the current distribution in frequency space is of order $O(\beta\gamma/R)$.
% \begin{figure}[h!]
%     \centering
%     \includegraphics[width=0.85\linewidth]{plots/chap02QCD/lightcone2.png}
%     \caption{The magnitude of the polarization tensor is plotted in momentum space showing deviations in $k_\rho$ from the light-cone $\omega = |\boldsymbol{k}|$ on the horizontal axis. The contours show lines of constant magnitude of $\Pi_\perp(\omega, |\boldsymbol{k}|)$; lighter shading indicates increasing magnitude. The dashed line encapsulates the $2\sigma$ support of the external charge distribution. The width of the external charge distribution in momentum space is $\sqrt{2}/R$ in the transverse direction and  $\beta \gamma \sqrt{2}/R$ along the light-cone. One can see that in the region sampled by the external charge distribution the polarization tensor is effectively constant as a function of $k_\rho$. \label{fig:lightcone}}
% \end{figure} 
% The region of two-sigma support of the Gaussian charge distribution is shown as the region enclosed by the dashed line in Fig.~\ref{fig:lightcone}. The polarization tensor is approximately constant as a function of $k_\rho$ in this region. This implies that one can approximate the integral in \req{eq:magorgin} using the polarization function at $k_\rho = 0$, i.~e.\, on the light-cone. This means that the fields of the ions, traveling near the speed of light, probe the polarization tensor along the light-cone. In this limit, the transverse conductivity near the light-cone is
% \begin{equation}\label{eq:lightcone}
%     \sigma_\perp (\omega = |\boldsymbol{k}|)  =  i \frac{m_D^2}{4 \omega}\left( \frac{\kappa^2}{\omega^2} \xi \ln\xi +\frac{i\kappa}{\omega}\left(\xi+1\right)\right)\,,
% \end{equation}
% where $\xi$ is defined as
% \begin{equation}\label{eq:xidef}
%     \xi \equiv 1- 2i \frac{\omega}{\kappa}\,.
% \end{equation}
% Since the light-cone conductivity only depends on a single variable ($\omega = |\boldsymbol{k}|$) it simplifies integrals involved in the Fourier transform of fields back into position space.

% Our results for the magnetic field at the collision center $B_y(t,0)$ are shown in Fig.~\ref{fig:bfcomp}. The right panel of the figure shows the field at early times ($|t| < 0.25~\text{fm}/c$) on a linear scale, the left panel shows the field over a wider time range on a logarithmic scale. The most general case $\sigma_\perp(\omega, \boldsymbol{k})$, shown as the dashed red curve in Fig.~\ref{fig:bfcomp}, includes the full time- and space-dependent response of the medium to the fields of the colliding ions. The blue dashed curve shows the magnetic field in the  Drude model approximation \req{eq:conddrude}, where the response depends only on time. The magnetic field using the light-cone conductivity is seen as the gray line overlapping the red dashed line in Fig.~\ref{fig:bfcomp},  where $\sigma_0$ is defined in \req{eq:condstat}. The result of Fourier transforming this expression is shown as the brown dotted curve in Fig.~\ref{fig:bfcomp}. Our results differ slightly from those of \cite{Tuchin:2013apa} because here we account for the finite size of the ions and use a slightly different conductivity value. 

% \phantom{Phantom text}
% \begin{figure}[htb]
% \centering
% \includegraphics[width=0.46\linewidth]{plots/chap02QCD/bf100.png}
% %}
% \hspace{0.05\linewidth}
% \includegraphics[width=0.44\linewidth]{plots/chap02QCD/bf100lin.png}
% %}
% \caption{The magnetic field at the collision center as a function of time, with $T = 300$\,MeV for a Au-Au collisions ($Z=79$) at $\sqrt{s_\text{NN}} = 200$\,GeV and impact parameter $b = 6.4\,$fm. The left panel shows the magnetic field on a semi-logarithmic scale up to $ct = 5$\,fm. The right panel shows the early-time magnetic field on a linear scale. At the chosen temperature the electromagnetic Debye mass is $m_D = 74\,$MeV and the quark damping rate is $\kappa = 4.86\,m_D$. This gives a static conductivity of $\sigma_0 = 5.01\,$MeV. Comparing the different approximations we see that all of them have similar asymptotic behavior. Only the Drude conductivity, the light-cone limit of the conductivity, and the full conductivity $\sigma_\perp(\omega,\boldsymbol{k})$ describe the field at early times. Note that here that the plasma is considered homogeneous and stationary. In a more realistic situation the field would become screened only after the QGP is formed in the collision.\label{fig:bfcomp}}
% \end{figure}

% The magnetic field in the presence of a QGP was previously calculated using a static conductivity in \cite{Tuchin:2013apa}. In this case, the magnetic field in Fourier space has the form
% \begin{equation}\label{eq:bstat}
%     \ft{B}(\omega,\boldsymbol{k}) = \frac{ \mu_0 i\boldsymbol{k} \times \ft{j}_{\perp \text{ext}}}{\boldsymbol{k}^2 - \omega^2 - i\omega\sigma_0}\,,
% \end{equation}

% Looking at the left panel of Fig.~\ref{fig:bfcomp}, one can see that every model of the response function predicts similar values for the magnetic field approaching the freeze-out time $t_f$. This is because the static conductivity determines the late-time dependence of the magnetic field. As we discuss in Appendix \ref{sec:magf}, we can expect the static solution to match the full solution when $t > 1/\kappa$. The static conductivity initially overestimates the magnetic field after the external field begins to fall, since the effect of dynamic screening is not captured. This matches the qualitative picture given by the detailed numerical transport calculation done in \cite{Wang:2021oqq}. The full space-time dependent model and the Drude model model behave similarly for most times, and are almost identical for $t>1/\kappa \approx 0.36$\,fm/c. The magnetic field calculated using the polarization tensor evaluated on the light cone tracks the general solution at all times.

% We can use the light-cone conductivity in \req{eq:lightcone} to understand why the Drude model $\sigma_\perp(\omega, 0)$ matches the full solution for times $t>1/\kappa$. Late times probe the small frequency limit of the conductivity. An expansion of \req{eq:lightcone} in $\omega/\kappa$ yields
% \begin{multline}
% \sigma_\perp (\omega = |\boldsymbol{k}|)   = \sigma_0\left(1+i\omega/\kappa\right)\\
% - \frac{6 \sigma_0}{5 }\frac{\omega^2}{\kappa^2}+O\left(\frac{\omega^3}{\kappa^3}\right)\,.
% \end{multline}
% We then compare to the same expansion for the Drude conductivity
% \begin{multline}
% \sigma_\perp (\omega,0) = \frac{\sigma_0}{1- i\omega/\kappa}  \approx  \sigma_0\left(1+i\omega/\kappa\right) \\
% - \sigma_0 \frac{\omega^2}{\kappa^2}+O\left(\frac{\omega^3}{\kappa^3}\right)\,.
% \end{multline}
% The lowest-order term, which coincides with the expression for the Drude model, closely approximates the full solution when $\kappa \gg \omega$ as shown in Fig~\ref{fig:condlightcomp}. Since $\kappa t_f \gg 1$ for the QGP, the series converges rapidly for times of the order of the freeze-out time $t_f$. 
% \begin{figure}[h!]
%     \centering
%     \includegraphics[width=0.85\linewidth]{plots/chap02QCD/condlightcomp.png}
%     \caption{Comparison of the conductivity on the light-cone to $ \sigma_{\perp}(\omega,\boldsymbol{k}\rightarrow 0) $, scaled with the static conductivity. We see that at small $\omega/\kappa$, i.e. times much larger than $1/\kappa$, both approximations converge to the static case, while they diverge $\omega/\kappa > 1$. This predicts that the Drude model will underestimate screening at small times, which is exactly what we observe in Fig.~\protect\ref{fig:bfcomp}. \label{fig:condlightcomp}}
% \end{figure} 

% The simple form of the Drude approximation \req{eq:conddrude} allows one to find the poles of the denominator in \req{eq:magorgin}, analytically. The frequency integral can then be done using the residue theorem, allowing for an approximate analytical expression for the late-time magnetic field. This is done in Appendix \ref{sec:magf}. 
% In the ultrarelativistic limit $\gamma\gg 1$ and large times $t\gg 1/\kappa$ gives
% \begin{equation}\label{eq:banalyticapp}
%    B_y(t) = -\mu_0 \frac{ Zq \beta }{(2\pi)} \frac{ b\sigma_0}{4t^2} e^{\frac{-b^2 \sigma_0}{16 t}}\,,
% \end{equation}
% This result differs from the ``diffusive'' solution of Tuchin \cite{Tuchin:2013apa} by a factor 1/4 in the exponent, due to their convention for impact parameter $b\rightarrow2b$. The reason why $\kappa$ does not appear in the expression \req{eq:banalyticapp} for the late-time magnetic field lies in the hierarchy of time scales $t_\text{coll} \ll 1/\kappa \ll t_f$, which makes plasma damping irrelevant during the spike of the external field as well as at freeze-out.

% Interestingly, this solution has a finite limit when $\gamma\rightarrow\infty$ as it depends only on $\beta$, but not on $\gamma$. This property, which was first observed by Tuchin \cite{Tuchin:2013apa}, can be understood as follows: For late times the Fourier integral of \req{eq:extchgfreq} is dominated by contributions from small frequencies $\omega$, and it is sufficient to consider the $\omega \rightarrow 0$ limit of the Fourier spectrum of the external charge distributions $\wt{\rho}_{f\pm}$ given in \req{eq:By}. In this limit \req{eq:extchgfreq} takes the form
% \begin{equation}
% \wt{\rho}_{\text{ext}\pm}(0,\boldsymbol{k}) \rightarrow 2\pi Ze\, e^{-k_\rho^2R^2/4} e^{\mp \frac{i k_\rho b \cos \theta }{2}} \delta(k_z \beta)\,,
% \end{equation} 
% which is independent of $\gamma$. This occurs because
% \begin{equation}
% \wt{\rho}_{\text{ext}\pm}(0,\boldsymbol{k}) = \int dt \int d^3x e^{-\boldsymbol{k}\cdot\boldsymbol{x}} \rho_{\text{ext}\pm}(0,\boldsymbol{x})
% \end{equation}
% integrates over the passage of the entire nucleus at a given location $\boldsymbol{x}$ and thus is independent of $\gamma$ as the total charge is Lorentz invariant. We conclude that, quite generally, for high collision energies the remnant magnetic field at late times is determined by the time-integrated action of the external electromagnetic pulse on the QGP. In a more realistic calculation, where the QGP is not present for the entire duration of the pulse, because it is created during the collision, the remnant magnetic field will be diminished as only a fraction of the pulse acts on the QGP. We therefore expect our result to represent an upper bound to the late-time magnetic field in a realistic collision scenario.

% \begin{figure}
%     \centering
% \includegraphics[width=0.95\linewidth]{plots/chap02QCD/bfgaamacomp.png}
%     \caption{Plot of the freeze-out magnetic field for $T= 150$\,MeV. We expect that around this temperature QGP will hadronize into a mixed phase \cite{Letessier:1992xd}. The approximate late time solution \req{eq:banalyticapp} shown as an orange dashed line is compared to numerical calculations using the full polarization tensor \req{eq:magorgin}. the approximate solution does not fully match the ultrarelativistic limit until times $t > t_{\sigma} \approx 59$\,fm/c.  The The magnetic field is independent of the beam energy over a wide range of $\gamma$ but begins to diverge slowly from the ultrarelativistic case at around $\gamma \leq 15$ for the time window shown in the figure. Lower beam energies result in a somewhat larger field at late time.\label{fig:bcolcomp}}
% \end{figure}

% The approximation used to derive this solution holds for $\gamma\beta \gg \sqrt{ \kappa/\sigma_0} \approx 12$. In Fig.~\ref{fig:bcolcomp} we compare \req{eq:banastat} to the full numerical result to explore its dependence on $\gamma$.  One can see that the ultrarelativistic case (black solid line) begins to diverge from the numerical solution at around $\gamma \approx 15$ for the times shown. The early time magnetic field is not shown because the initial temperature of QGP will depend strongly on the collision energy. The times are chosen such that they cover the range of freeze-out times predicted for QGP for the range of experimental collision energies used \cite{Bass:2000ib}. We do not show curves for $\gamma<10$ because we expect the effects of chemical potential will become important, yet here chemical potential $\mu$ is set to zero. In Fig.~\ref{fig:bcolcomp} one can see that the late-time magnetic field has a very soft dependence on collision energy. The time at which the magnetic field freezes out, which varies with collision energy, has a much stronger effect on the magnitude of the freeze-out field.

% As the QGP begins to hadronize at time $t_f$, one may expect hadrons to be statistically polarized with respect to the magnetic field. In \cite{Muller:2018ibh} the measured difference in global polarization of hyperons and antihyperons is used to give an upper bound on the magnetic field at QGP freeze-out, $B \sim 2.7\times 10^{-3}\,m_{\pi}^2$ for Au+Au collisions at $\sqrt{s_\text{NN}} = 200$\,GeV. Looking at Fig.~\ref{fig:bcolcomp} the magnetic field for $\gamma = 100$ at QGP freeze-out $t_f \approx 5 $\,fm/c is predicted to be $B \approx 1.2\times 10^{-3}\,m_{\pi}^2$, somewhat below this upper bound. Note that this assumes the polarization  rapidly equilibriates in the plasma. It also neglects any interactions during the hadron gas phase of the collision. 

% \begin{figure}
% \vskip 16pt
%     \centering
% \includegraphics[width=0.85\linewidth]{plots/chap02QCD/kappacomp.png}
%     \caption{Comparison of the magnetic field for different values of quark damping rate or, equivalently, electric conductivity. Larger values of  the damping rate $\kappa$ represent smaller conductivities and vice versa as indicated by \req{eq:condstat}. The black dashed line and the solid black line represent the limits of zero and infinite conductivity, respectively. One can see that as $\kappa$ increases the asymptotic value of the magnetic field decreases.\label{fig:kappacomp}}
    
% \end{figure}

% In Fig.~\ref{fig:kappacomp} we look at the magnetic field at the origin for different values of $\kappa$. Increasing $\kappa$ reduces the static conductivity $\sigma_0$ which decreases the asymptotic value of the magnetic field as indicated by \req{eq:banalyticapp}. As $\kappa$ goes to zero the results converge to the case of ideal conductivity  $\sigma_0 \rightarrow \infty$ where the magnetic field quickly approaches a constant value. This case was studied in \cite{Deng:2012pc} where the authors considered a magnetic field that falls to a constant value and then decreases with $1/t$ due to Bjorken flow. More recent calculations \cite{Yan:2021zjc,Wang:2021oqq} solve the Vlasov-Boltzmann equation numerically with parton-parton scattering. The magnetic field predicted by \cite{Yan:2021zjc} is around $\sim 10^{-4} m_\pi^2$ after $t\approx 2$\,fm/c, which is two orders of magnitude lower than the value found here (see Fig.~\ref{fig:bcolcomp}). However, the magnetic field predicted by \cite{Wang:2021oqq} is around $\sim 10^{-2} m_\pi^2$ after $t\approx 2$\,fm/c, which is in agreement with our model.

% In Fig.~\ref{fig:lighfield} we show a space-time contour plot of the magnetic field. The field at the higher collision energy (on the left) has a higher peak magnetic field. For lower collision energy (on the right) the field is less Lorentz contracted, and leads to a magnetic field at late times that is a factor of $\sim1.1$ larger. The freeze-out magnetic field will increase at lower collision energy mainly due to the decreasing freeze-out time.


% \phantom{Phantom text}

% \begin{figure}[H]
% \centering
% \includegraphics[width=0.45\linewidth]{plots/chap02QCD/lightBfplot.png}
% %}
% \hspace{0.05\linewidth}
% \includegraphics[width=0.45\linewidth]{plots/chap02QCD/lightBfplot10.png}
% %}
% \caption{Space-time plot of the magnetic field on the beam axis ($x=y=0$) in eternal (pre-existent) QGP with $T = 300$\,MeV for a Au-Au collision at impact parameter $b = 6.4\,$fm. Left panel: collision energy $\sqrt{s_\text{NN}} = 200$\,GeV; right panel: collision energy $\sqrt{s_\text{NN}} = 17$\,GeV. The same value of $\kappa$ is used as in Fig.~\ref{fig:bfcomp}. In a more realistic scenario, where the QGP is formed during the collision, the field would only create induced currents in the upper light-cone.\label{fig:lighfield}}
% \end{figure}


%%%%%%%%%%%%%%%%%%%%%%%%%%%%%%%%%%%%%%%%%%%%%%%%%%%%%%%%%%%%%%%%%%%%%%%%%%%%%%%%%%%%%%%%%%%%%%%%%%%%%%%%%%%%%%%%%%%%%%%%

\section{Towards a more realistic QGP: discussion and outlook }\label{sec:ConclusionsQGP}
The work reviewed here and presented in Appendix \ref{appendixB} calculates the magnetic field of two colliding nuclei in a stationary, homogeneous QGP using relativistic kinetic theory with collisional damping. Our first main finding in \cite{Grayson:2022asf} was that the response to the external magnetic field is controlled by the polarization function along the light-cone, $\Pi^\mu_\nu(\omega ,|\boldsymbol{k}|\approx\omega)$. This allowed us to derive an approximate analytic solution for the magnetic field that considers the dynamics of the medium's response. We also discussed how the late-time magnetic field at hadronization does not depend strongly on the collision energy. This gives the possibility that an experimental measurement of the magnetic field at different collision energies could permit a determination of the electrical conductivity of the QGP \cite{PhysRevX.14.011028}. We must also know how the freeze-out time depends on collision energy to make this measurement.

\subsection{The QGP medium}
This calculation can be improved in numerous ways. One of our main interests is to incorporate a finite size and a time-dependent onset in the QGP medium, which we describe here as infinite and homogenous. Boundary effects at the QGP surface are likely crucial for many collisions since the Debye sphere is not much smaller than the size of QGP, or similarly, the skin depth is probably large in comparison to the radius of QGP. Plasma skin effects could lead to novel electromagnetic phenomena at the QGP surface. We have begun some work on implementing an initial onset and formation time for QGP in the Vlasov-Boltzmann equation, effectively creating a boundary in time. This work should be extendable to studying plasma with a finite boundary in space which could be interesting with respect to the study of surface plasmons.

QGP is also not stationary; peripheral heavy ion collisions are one of the most highly rotational systems ever observed \cite{PhysRevC.87.034906, PhysRevC.93.064907,PhysRevC.94.044910, doi:10.1146/annurev-nucl-021920-095245}. This is due to the huge angular momentum of the colliding system. This rotation can be incorporated into the equilibrium distribution \cite{Hakim2011}, which creates a temperature that depends on radius \cite{chernikov1964equilibrium} changing our description of the magnetic field.

In \cite{Grayson:2022asf} it would have been simple to use the adiabatic expansion of a relativistic ideal gas \cite{Bjorken:1982qr} to parameterize the temperature dependence as a function of time. To reduce the number of free parameters, we found the magnetic field at large times by simply assuming the plasma temperature was the freeze-out temperature \reff{fig:bcolcomp}.

Many enhancements can be made that require numerical solutions of the linear response equations, Such improvements would include a realistic space-time dependence of the medium (formation and hydrodynamical evolution), nonzero net baryon density, quark thermal mass corrections \cite{PhysRevD.26.2789}, and viscous corrections to the unperturbed phase-space distribution used to calculate the polarization tensor.

\subsection{Electric field in QGP}

\phantom{Phantom text}
\begin{figure}[h!]
\centering
\includegraphics[width=0.45\linewidth]{plots/chap02QCD/Eyy.png}
%}
\hspace{0.05\linewidth}
\includegraphics[width=0.45\linewidth]{plots/chap02QCD/Ezz.png}
%}
\caption{Plots comparing the electric field in vacuum, shown as a black dashed line, to the electric field in QGP shown as the red points. The left plot shows the transverse electric field screened by the plasma. The plot on the right shows the electric field in the direction of motion enhanced by the plasma. We choose $T = 300$\,MeV and $Z=79$, for Au-AU collisions at $\sqrt{s} = 200$\,GeV at an impact parameter of half nuclear overlap $b = 1 R = 6.4\,$fm. The vertical line in the left plot indicates $ y = R$, approximately the transverse size of QGP. \label{fig:efcomp}}
\end{figure}



Of course, we could have also studied electric fields in QGP which are in the same order as the magnetic fields $e|E| \approx m_\pi^2$. These fields are of interest in strong field QED since they are far beyond the Schwinger limit $e|E| \approx m_e^2$. Preliminary QGP electric field calculations are shown in 
\reff{fig:efcomp}. In QGP, the transverse electric field $E_y$ is screened while the eclectic field is enhanced in the direction of motion. The electric field is also interesting since it could do a significant amount of work on the QGP possibly reheating it after its formation through ohmic heating. 

Additionally, we were interested in studying the distribution of electric charge around relativistic heavy nuclei in QGP. This can be found by Fourier transforming \req{eq:indch} for the external charge distribution \req{eq:rhoext}. The induced charge density for a single traveling nucleus at low $\gamma$ is shown in \reff{fig:efcomp}. The external charge distribution increases with the Lorentz factor $\gamma$, but the total induced charge, which is the integral of the red dashed line, remains constant but trails behind further at larger velocities.

\phantom{Phantom text}
\begin{figure}[h!]
\centering
\includegraphics[width=0.45\linewidth]{plots/chap02QCD/indchg12.png}
%}
\hspace{0.05\linewidth}
\includegraphics[width=0.45\linewidth]{plots/chap02QCD/indchg5.png}
%}
\caption{The external (black), induced (red dashed), and total charge density (blue dashed) for a single nucleus traveling in the $+\boldsymbol{\hat{z}}$ direction at $\gamma = 1.2$ on the left and $\gamma = 5$ on the right. The induced charge distribution trails behind the nuclei. The external charge density increases with $\gamma$. The induced charge distribution trails behind the nuclei more for larger $\gamma$. \label{fig:potcomp}}
\end{figure}

As seen in \reff{fig:potcomp}, a wakefield of induced charge forms behind the traveling nucleus in QGP. In \reff{fig:chgwake}, we show a two-dimensional contour plot of the charged wake. The wakefield depicted in \reff{fig:chgwake} is damped at traverse distances instead of conical as in the collisionless case.
\begin{figure}[h!]
\centering
\includegraphics[width=0.85\linewidth]{plots/chap02QCD/chwake.png}
%}
\caption{2D plot of the wake field of a single traveling gold nucleus $\gamma = 5$ in QGP. The blue arrow indicates the direction of motion and the grey disk represents the Lorentz contracted nucleus. Lines of constant charge density are shown as contours. \label{fig:chgwake}}
\end{figure}


The Electromagnetic polarization tensor in QGP also has applicability in cosmology, where a QGP existed during the first $10~\mu$s of the early Universe. In the next chapter, we will study somewhat later times a few seconds after the Big Bang, when the universe was filled with electron-positron plasma. In these situations, the assumption of homogeneity and stationary of the medium on the scale of the relevant parameters, $m_D$, and $\kappa$, is well justified.





% %===================================================================
% %===================================APPENDICES======================

% %===================================================================
% \section{Electric current of two colliding nuclei}\label{sec:freechg}

% Here we define the free charge and current density used to describe heavy ion collisions. We wish to model two nuclei moving at constant velocity $\pm \beta$ along the collision axis ($\hat{z}$ direction) that are offset by $\pm b/2$ within the collision plane ($\hat{x}$ direction).  For simplicity we model the charge distribution as a gaussian in all directions
% \begin{multline}
% \rho_{\text{ext}\pm }(t,\boldsymbol{x}) = \frac{Ze\gamma}{\pi^{3/2}R^3}e^{-\frac{1}{R^2}(x\mp b/2)^2}e^{-\frac{1}{R^2}y^2}\\
% \times e^{-\frac{\gamma^2}{R^2}(z\mp \beta t)^2}\,,
% \end{multline}
% where the normalization is chosen in such a way that 
% \begin{equation}
% \int \rho_{\text{ext}\pm}(t,\boldsymbol{x}) d^3\boldsymbol{x} = Ze\,,
% \end{equation}
% is the total charge of the heavy ion nucleus and $\gamma$ is the usual relativistic factor. The Gaussian radius parameter $R$ is related to the mean squared radius of the nucleus, $\langle r^2\rangle$ at rest, ($\gamma = 1$) by
% \begin{equation}\label{eq:radius}
% \langle r^2 \rangle = \frac{1}{Ze}\int r^2 \rho_{\text{ext}\pm}(\boldsymbol{x}) d^3\boldsymbol{x} = \frac{3}{2}R^2\,,
% \end{equation}
% which is measured experimentally for a gold nucleus to be $\sqrt{\langle r^2 \rangle} \approx 5.30 \,$fm \cite{DeVries:1987atn} .

% At time $t = 0$ both nuclei are localized at the $z = 0$ plane and we assume that before and after the collision they continue moving on a straight line along the $z$-axis. The gaussian form of the charge distributions allows us to evaluate the Fourier transformations easily. The transforms in the transverse directions are
% \begin{align}
% \int_{-\infty}^\infty dy \ e^{-ik_y y}e^{-y^2/R^2} &= R \sqrt{\pi} e^{-k_y^2 R^2/4}\,, \\ 
% \int_{-\infty}^\infty \ dx e^{-ik_x x}e^{-(x\mp b/2)^2/R^2} &=  R \sqrt{\pi}e^{-k_x^2 R^2/4}e^{\pm i k_x b /2}\,.
% \end{align}
% The last two integrals are a bit more complicated because they are coupled
% \begin{multline}
% \int_{-\infty}^\infty e^{i\omega t} \left(\int_{-\infty}^\infty e^{-ik_z z}e^{- \frac{\gamma^2}{R^2}(z^2 \pm 2z\beta t)} dz \right)\\
% \times e^{-\frac{\gamma^2}{R^2}\beta^2t^2}dct = \frac{R \sqrt{\pi}}{\gamma}e^{-\frac{k_z^2R^2}{4\gamma^2}} \int_{-\infty}^\infty e^{i(\omega \pm k_z \beta)t}dt\\
% = \frac{2 R\pi^{3/2}}{\gamma}e^{-\frac{k_z^2R^2}{4\gamma^2}} \delta(\omega \pm k_z \beta)\,,
% \end{multline}
% where delta function appears because both nuclei move at a constant velocity. Altogether the Fourier transformed charge distributions are 
% \begin{multline}
% \wt{\rho}_{\text{ext}\pm}(\omega,\boldsymbol{k}) = 2\pi Ze\, e^{-(k_x^2 + k_y^2 + k_z^2/\gamma^2)\frac{R^2}{4}} \\
% \times e^{\mp \frac{i k_x b}{2}} \delta(\omega \mp k_z \beta)\,,
% \end{multline} 
% which may be written in cylindrical coordinates
% \begin{multline}\label{eq:extchgfreq}
% \wt{\rho}_{\text{ext}\pm}(\omega,\boldsymbol{k}) = 2\pi Ze\, e^{-(k_\rho^2 + k_z^2/\gamma^2)\frac{R^2}{4}} \\
% \times e^{\mp \frac{i k_\rho b \cos \theta }{2}} \delta(\omega \mp k_z \beta)\,.
% \end{multline} 
% The current densities are obtained from
% \begin{equation}\label{eq:extcurrent}
% \ft{j}_{\text{ext}\pm}(\omega, \boldsymbol{k}) = \pm \beta \hatv{z} \wt{\rho}_{\text{ext}\pm}(\omega, \boldsymbol{k})\,.
% \end{equation}
% The transverse component of the current given by
% \begin{multline}\label{eq:jperpext}
% \ft{j}_{\perp,\text{ext}} = \ft{j}_\text{ext} - (\hatv{k} \cdot \ft{j}_\text{ext}) \hatv{k} \\= (\hatv{z} - \hat{k}_z \hatv{k})\beta(\wt{\rho}_{\text{ext}+} - \wt{\rho}_{\text{ext}-})\,.
% \end{multline}

% %%%%%%%%%%%%%%%%%%%%%%%%%%%%%%%%%%%%%%%%%%%%%%%%%%%%%%%%%%%%%%%%%%%

% \section{Magnetic field at the collision center}\label{sec:magf}

% The magnetic field inside the plasma is given in Fourier space by
% \begin{equation}\label{eq:magapp}
%    \ft{B} = i \boldsymbol{k} \times \wt{\boldsymbol{A}} = i \boldsymbol{k} \times \wt{\boldsymbol{A}}_{\perp}\,,
% \end{equation}
% where the potential $\wt{\boldsymbol{A}}$ has been projected into components transverse and longitudinal to $\boldsymbol{k}$. In the following we represent the wave-vector in cylindrical coordinates $\boldsymbol{k} = (k_{\rho} \cos \theta,k_{\rho}\sin \theta, k_{z}) $. Using the expression for the self-consistent vector potential \req{eq:aperp} we find for the magnetic field,
% \begin{equation}
%    \ft{B} =
%  \frac{\mu_0 i \boldsymbol{k} \times\ft{j}_{\perp \text{ext}}}{\boldsymbol{k}^2 - \omega^2 - \mu_0 \Pi_{\perp}}\,.
% \end{equation}
% Given the definition of $\ft{j}_{\perp,\text{ext}}$ in \req{eq:jperpext} we can replace the perpendicular component of the current by its full form, adding the $\pm$ components of \req{eq:extcurrent}:
% \begin{equation}
%    \ft{B} =
%  \frac{\mu_0 i \boldsymbol{k} \times\ft{j}_{\text{ext}}}{\boldsymbol{k}^2 - \omega^2 - \mu_0 \Pi_{\perp}}\,.
% \end{equation}
% We now Fourier transform this quantity back to position space in order to calculate the magnetic field at the collision center as a function of time. Due to symmetry the only nonvanishing component of the magnetic field at this location will be the $y$-component:
% \begin{equation}\label{eq:By}
%    \wt{B}_y = 
%  \mu_0 \frac{ i k_x \beta (\wt{\rho}_{\text{ext}-}-\wt{\rho}_{\text{ext}+})}{\boldsymbol{k}^2 - \omega^2 - \mu_0 \Pi_{\perp}}\,.
% \end{equation}
% The Fourier transform at any point along the collision axis ($x=y=0$) is given by
% \begin{equation}
%   B_y(z,t) =  \int \frac{d^4k}{(2\pi)^4}  e^{-i\omega t+ ik_z z} \wt{B}_y(\omega, \boldsymbol{k})\,.
% \end{equation}
% In cylindrical coordinates the integral can be written as
% \begin{multline}
%   B_y(z,t) =  \frac{1}{(2\pi)^4}\int k_{\rho}dk_{\rho} d\omega dk_z d\theta  e^{-i\omega t+ ik_z z}\\
%  \mu_0 \frac{ i \beta k_{\rho} \cos \theta (\wt{\rho}_{f-}-\wt{\rho}_{f+})}{\boldsymbol{k}^2 - \omega^2 - \mu_0 \Pi_{\perp}(\omega, |\boldsymbol{k}|)}\,.
% \end{multline}
% We can use the the delta function in the Fourier transformed current \req{eq:extchgfreq} to trivially perform the $k_z$ integral:
% \begin{multline}
%    B_y(z,t) = -\mu_0 \frac{ Ze \beta }{(2\pi)^3}  \int dk_{\rho} d\omega d\theta \  e^{-i\omega t} \\ \frac{ 2 k_{\rho}^2  \cos \theta \sin \left(  k_{\rho} \cos \theta \frac{b}{2} - \frac{\omega z}{\beta} \right)
%   e^{-(k_{\rho}^2+ \omega^2/(\beta\gamma)^2)\frac{R^2}{4}}}{\omega^2/(\gamma\beta)^2 + k_{\rho}^2  - \mu_0 \Pi_{\perp}\left(\omega, \sqrt{k_{\rho}^2 + \omega^2/\beta^2} \right)}\,.
% \end{multline}
% We next perform the angular integration
%  \begin{multline}\label{eq:intnum}
%    B_y(z,t) = -\mu_0 \frac{ Ze \beta }{(2\pi)^2}  \int dk_{\rho} d\omega \  e^{-i\omega t} \\ \frac{ 2 k_{\rho}^2 J_1 \left(\frac{k_{\rho} b}{2} \right) \cos \left( \frac{\omega z}{\beta}\right)
%   e^{-(k_{\rho}^2+ \omega^2/(\beta\gamma)^2)\frac{R^2}{4}}}{\omega^2/(\gamma\beta)^2 + k_{\rho}^2  - \mu_0 \Pi_{\perp}\left(\omega, \sqrt{k_{\rho}^2 + \omega^2/\beta^2} \right)}\,,
% \end{multline}
% where $J_1$ is a Bessel function of the first kind. The remaining integrals have to be performed numerically. From here forward we pull the factor of $\mu_0$ into the Debye mass $m_D^2$, such that the factor of $e^2$ goes to $4 \pi \alpha$ in \req{eq:DebyemQCD}.

% We can obtain an analytical expression for the Drude approximation \req{eq:conddrude} in the limit $\gamma\beta \gg \sqrt{ \kappa/\sigma_0}$, which is valid when $\gamma \gg 12$ for the values of $\sigma_0$ and $\kappa$ adopted here.  In this limit we can neglect the first term in the denominator of \req{eq:intnum}, which now takes the simple form
% \begin{equation}
%       k_\rho^2 - i \omega \frac{\omega_p^2}{\kappa - i\omega}\,.
% \end{equation}
% Note that using \req{eq:condstat} we can see $\omega_p^2 = m_D^2/3 = \kappa \sigma_0 $. The integrand of \req{eq:intnum} then has a single pole at 
% \begin{equation}
% \omega = -i\frac{k_\rho^2\kappa}{k_\rho^2+\omega_p^2}\,,
% \end{equation}
% and the frequency integral can be performed by contour integration in the lower complex plane. Consistently neglecting the term proportional to $\omega^2/(\beta\gamma)^2$ in the exponent, the integration yields:
% \begin{multline}\label{eq:bintfull}
%    B_y(z,t) \approx - \mu_0 \frac{ Ze \beta }{2\pi}  \int dk_{\rho}\, 2\kappa k_{\rho}^2\\
%     \frac{\omega_p^2}{(k_\rho^2+\omega_p^2)^2} 
%     J_1 \left(\frac{k_{\rho} b}{2} \right) e^{-k_{\rho}^2R^2/4} \\
%    \cosh \left( \frac{k_\rho^2\kappa}{k_\rho^2+\omega_p^2} \frac{z}{\beta}\right)
%    \exp \left( - \frac{k_\rho^2\kappa t}{k_\rho^2+\omega_p^2} \right)\,.
% \end{multline}
% For late times $t$ the exponential factor only samples the small $k_\rho$ region of the integrand ($k_\rho^2 < \sigma_0/t$). We can then neglect $k_\rho$ with respect to $ \omega_p$ in the integrand provided that $\sigma_0/t \ll \omega_p^2$, which is satisfied when $t \gg 1/\kappa = t_{\text{rel}} \approx 1\,\text{fm}/c$. The expression then takes the simplified form:
% yielding
% \begin{multline}\label{eq:bappint}
%    B_y(z,t) \approx - \mu_0 \frac{ Ze \beta }{2\pi}  \int dk_{\rho}\, \frac{2k_{\rho}^2}{\sigma_0}
%    J_1 \left(\frac{k_{\rho} b}{2} \right) \\
%    e^{-k_{\rho}^2R^2/4} \cosh \left( \frac{k_\rho^2}{\sigma_0} \frac{z}{\beta} \right) e^{-k_\rho^2 t/\sigma_0}\,.
% \end{multline}
% The integral over $k_\rho$ can now be performed analytically resulting in:
% \begin{equation}
%    B_y(z,t) \approx - \mu_0 \frac{ Ze \beta }{2\pi}  \frac{b}{8\sigma_0}
%    \left[ \frac{e^{-\frac{b^2}{16L_+}}}{L_+^2} + \frac{e^{-\frac{b^2}{16L_-}}}{L_-^2} \right]
% \end{equation}
% with 
% \begin{equation}
% L_\pm = \frac{R^2}{4} + \frac{t\pm z/\beta}{\sigma_0} \,.
% \end{equation}
% At the collision center $(z=0)$ and for $t \gg \sigma_0R^2/4$ our result simplifies to
% \begin{equation}\label{eq:banastat}
%    B_y(0,t) \approx - \mu_0 \frac{ Ze \beta }{2\pi}  \frac{b\sigma_0}{4t^2} e^{-\frac{\sigma_0b^2}{16t}}\,.
% \end{equation}

% This result differs from Tuchin's \cite{Tuchin:2013apa} by a factor 1/4 in the exponent due to their convention for impact parameter $b\rightarrow2b$.




 

%%%%%%%%%%%%%%%%%%%%%%%%%%%%%%%%%%%%%%%%%%%%%%%%
%%%%%%%%%%%%%%%%%%%%%%%%%%%%%%%%%%%%%%%%%%%%%%%%

% \begin{thebibliography}{99}

% %%%%%%%%%%%%%%%%%  Analytic - Constant Conductivity 

%     \bibitem{Tuchin:2010vs}
%     K.~Tuchin,
%     %``Synchrotron radiation by fast fermions in heavy-ion collisions,''
%     Phys. Rev. C \textbf{82}, 034904 (2010)
%     [erratum: Phys. Rev. C \textbf{83} (2011), 039903]
%     % doi:10.1103/PhysRevC.83.039903
%     [arXiv:1006.3051 [nucl-th]].
    
%     \bibitem{Deng:2012pc}
%     W.~T.~Deng and X.~G.~Huang,
%     %``Event-by-event generation of electromagnetic fields in heavy-ion collisions,''
%     Phys. Rev. C \textbf{85}, 044907 (2012)
%     % doi:10.1103/PhysRevC.85.044907
%     [arXiv:1201.5108 [nucl-th]].
    
%     \bibitem{Tuchin:2013apa}
%     K.~Tuchin,
%     %``Time and space dependence of the electromagnetic field in relativistic heavy-ion collisions,''
%     Phys. Rev. C \textbf{88}, 024911 (2013) 
%     % doi:10.1103/PhysRevC.88.024911
%      [arXiv:1305.5806 [hep-ph]].
    
%     \bibitem{McLerran:2013hla}
%     L.~McLerran and V.~Skokov,
%     %``Comments About the Electromagnetic Field in Heavy-Ion Collisions,''
%     Nucl. Phys. A \textbf{929}, 184 (2014)
%     % doi:10.1016/j.nuclphysa.2014.05.008
%     [arXiv:1305.0774 [hep-ph]].
    
%     \bibitem{Gursoy:2014aka}
%     U.~Gursoy, D.~Kharzeev and K.~Rajagopal,
%     %``Magnetohydrodynamics, charged currents and directed flow in heavy ion collisions,''
%     Phys. Rev. C \textbf{89}, 054905 (2014) 
%     % doi:10.1103/PhysRevC.89.054905
%     [arXiv:1401.3805 [hep-ph]].

%     \bibitem{Roy:2015kma}
%     V.~Roy, S.~Pu, L.~Rezzolla and D.~Rischke,
%     %``Analytic Bjorken flow in one-dimensional relativistic magnetohydrodynamics,''
%     Phys. Lett. B \textbf{750}, 45 (2015)
%     % doi:10.1016/j.physletb.2015.08.046
%     [arXiv:1506.06620 [nucl-th]].
    
%     \bibitem{Li:2016tel}
%     H.~Li, X.~l.~Sheng and Q.~Wang,
%     %``Electromagnetic fields with electric and chiral magnetic conductivities in heavy ion collisions,''
%     Phys. Rev. C \textbf{94}, 044903 (2016)
%     % doi:10.1103/PhysRevC.94.044903
%     [arXiv:1602.02223 [nucl-th]].

% %%%%%%%%%%%%%%%%%  Numerical (MDH) - Constant Conductivity 

%     \bibitem{Inghirami:2016iru}
%     G.~Inghirami, L.~Del Zanna, A.~Beraudo, M.~H.~Moghaddam, F.~Becattini and M.~Bleicher,
%     %``Numerical magneto-hydrodynamics for relativistic nuclear collisions,''
%     Eur. Phys. J. C \textbf{76}, 659 (2016)
%     % doi:10.1140/epjc/s10052-016-4516-8
%     [arXiv:1609.03042 [hep-ph]].
    
%     \bibitem{Inghirami:2019mkc}
%     G.~Inghirami, M.~Mace, Y.~Hirono, L.~Del Zanna, D.~E.~Kharzeev and M.~Bleicher,
%     %``Magnetic fields in heavy ion collisions: flow and charge transport,''
%     Eur. Phys. J. C \textbf{80}, 293 (2020)
%     % doi:10.1140/epjc/s10052-020-7847-4
%     [arXiv:1908.07605 [hep-ph]].
    
% %%%%%%%%%%%%%%%%%  Numerical (MDH) - Dynamic Conductivity 
 
%     \bibitem{Yan:2021zjc}
%     L.~Yan and X.~G.~Huang,
%     %``Dynamical evolution of magnetic field in the pre-equilibrium quark-gluon plasma,''
%     [arXiv:2104.00831 [nucl-th]].
    
%     \bibitem{Wang:2021oqq}
%     Z.~Wang, J.~Zhao, C.~Greiner, Z.~Xu and P.~Zhuang,
%     %``Incomplete electromagnetic response of hot QCD matter,''
%     [arXiv:2110.14302 [hep-ph]].
    
% %%%%%%%%%%%%%%%%%%%%%%%%%%%%%%%%%%%%%%%%%%%%%%%%%%%%%%%%%%%%%%%%%%%%%%%%%%%

%     \bibitem{Formanek:2021blc}
%     M.~Formanek, C.~Grayson, J.~Rafelski and B.~M\"uller,
%     %``Current-conserving relativistic linear response for collisional plasmas,''
%     Annals Phys. \textbf{434}, 168605 (2021)
%     % doi:10.1016/j.aop.2021.168605
%      [arXiv:2105.07897 [physics.plasm-ph]].
    
%     \bibitem{Florkowski:2017olj}
%     W.~Florkowski, M.~P.~Heller and M.~Spalinski,
%     %``New theories of relativistic hydrodynamics in the LHC era,''
%     Rept. Prog. Phys. \textbf{81}, 046001 (2018)
%     % doi:10.1088/1361-6633/aaa091
%     [arXiv:1707.02282 [hep-ph]].
    
%     \bibitem{Rocha:2021zcw}
%     G.~S.~Rocha, G.~S.~Denicol and J.~Noronha,
%     %``Novel Relaxation Time Approximation to the Relativistic Boltzmann Equation,''
%     Phys. Rev. Lett. \textbf{127}, 042301 (2021)
%     % doi:10.1103/PhysRevLett.127.042301
%     [arXiv:2103.07489 [nucl-th]].

%     \bibitem{Bhatnagar:1954zz}
% 	P.~L.~Bhatnagar, E.~P.~Gross and M.~Krook,
% 	%``A Model for Collision Processes in Gases. 1. Small Amplitude Processes in Charged and Neutral One-Component Systems,''
% 	Phys. Rev. \textbf{94}, 511 (1954).
% 	%doi:10.1103/PhysRev.94.511
	
%     \bibitem{Song:2007ux}
%     H.~Song and U.~W.~Heinz,
%     %``Causal viscous hydrodynamics in 2+1 dimensions for relativistic heavy-ion collisions,''
%     Phys. Rev. C \textbf{77}, 064901 (2008)
%     % doi:10.1103/PhysRevC.77.064901
%     [arXiv:0712.3715 [nucl-th]].
    
%     \bibitem{Starke:2014tfa}
% 	R.~Starke and G.~A.~H.~Schober,
% 	%\lq\lq Relativistic covariance of Ohm's law,\rq\rq
% 	Int. J. Mod. Phys. D \textbf{25}, 1640010 (2016)
% 	%doi:10.1142/S0218271816400101
% 	[arXiv:1409.3723 [math-ph]].
	
%     \bibitem{Weldon:1982aq}
% 	H.~A.~Weldon,
% 	%\lq\lq Covariant Calculations at Finite Temperature: The Relativistic Plasma,\rq\rq
% 	Phys. Rev. D \textbf{26}, 1394 (1982).
% 	%doi:10.1103/PhysRevD.26.1394
	
% 	\bibitem{melrose2008quantum}
% 	D.~Melrose,
% 	{\it Quantum Plasmadynamics: Unmagnetized Plasmas},
% 	Lect. Notes Phys. \textbf{735} 
% 	(Springer, New York, 2008).
% 	%doi:10.1007/978-0-387-73903-8 

%     \bibitem{Romatschke:2015gic}
%     P.~Romatschke,
%     %``Retarded correlators in kinetic theory: branch cuts, poles and hydrodynamic onset transitions,''
%     Eur. Phys. J. C \textbf{76}, 352 (2016)
%     % doi:10.1140/epjc/s10052-016-4169-7
%     [arXiv:1512.02641 [hep-th]].
    
%    \bibitem{Kapusta:1992fm}
%     J.~I.~Kapusta,
%     %``Screening of static QED electric fields in hot QCD,''
%     Phys. Rev. D \textbf{46}, 4749 (1992)
%     % doi:10.1103/PhysRevD.46.4749
    
%     \bibitem{Jackson:2019mop}
%     G.~Jackson,
%     %``Two-loop thermal spectral functions with general kinematics,''
%     Phys. Rev. D \textbf{100}, 116019 (2019)
%     %doi:10.1103/PhysRevD.100.116019
%     [arXiv:1910.07552 [hep-ph]].
    
%     \bibitem{Mrowczynski:1988xu}
%     S.~Mr\'owczy\'nski,
%     %``On the Transport Coefficients of a Quark Plasma,''
%     Acta Phys. Polon. B \textbf{19}, 91 (1988).
	
%     \bibitem{Rafelski:2002}
%     J.~Letessier, and  J.~Rafelski,
%     {\it Hadrons and Quark-Gluon Plasma} 
%     % (Cambridge Monographs on Particle Physics, Nuclear Physics and Cosmology). 
%    ( Cambridge University Press, 2002)
%     %doi:10.1017/CBO9780511534997
    
%     \bibitem{Drude:1900}
% 	P.~Drude,
% 	Ann.\ Phys.\ (Leipzig) \textbf{1}, 566 (1900).
	
% 	\bibitem{Aarts:2020dda}
%     G.~Aarts and A.~Nikolaev,
%     %``Electrical conductivity of the quark-gluon plasma: perspective from lattice QCD,''
%     Eur. Phys. J. A \textbf{57}, 118 (2021)
%     % doi:10.1140/epja/s10050-021-00436-5
%      [arXiv:2008.12326 [hep-lat]].
    
%     %\cite{Brandt:2012jc}
%     \bibitem{Brandt:2012jc}
%     B.~B.~Brandt, A.~Francis, H.~B.~Meyer and H.~Wittig,
%     %``Thermal Correlators in the \textbackslash{}rho\textbackslash{} channel of two-flavor QCD,''
%     JHEP \textbf{03} (2013), 100
%     % doi:10.1007/JHEP03(2013)100
%     % [arXiv:1212.4200 [hep-lat]].
%     %103 citations counted in INSPIRE as of 10 Jan 2022
    
%     %\cite{Amato:2013naa}
%     \bibitem{Amato:2013naa}
%     A.~Amato, G.~Aarts, C.~Allton, P.~Giudice, S.~Hands and J.~I.~Skullerud,
%     %``Electrical conductivity of the quark-gluon plasma across the deconfinement transition,''
%     Phys. Rev. Lett. \textbf{111}, 172001 (2013)
%     % doi:10.1103/PhysRevLett.111.172001
%      [arXiv:1307.6763 [hep-lat]].
%     %182 citations counted in INSPIRE as of 10 Jan 2022
    
%     %\cite{Aarts:2014nba}
%     \bibitem{Aarts:2014nba}
%     G.~Aarts, C.~Allton, A.~Amato, P.~Giudice, S.~Hands and J.~I.~Skullerud,
%     %``Electrical conductivity and charge diffusion in thermal QCD from the lattice,''
%     JHEP \textbf{02}, 186 (2015)
%     % doi:10.1007/JHEP02(2015)186
%      [arXiv:1412.6411 [hep-lat]].
%     %168 citations counted in INSPIRE as of 10 Jan 2022
    
%     %\cite{Brandt:2015aqk}
%     \bibitem{Brandt:2015aqk}
%     B.~B.~Brandt, A.~Francis, B.~J\"ager and H.~B.~Meyer,
%     %``Charge transport and vector meson dissociation across the thermal phase transition in lattice QCD with two light quark flavors,''
%     Phys. Rev. D \textbf{93}, 054510 (2016) 
%     % doi:10.1103/PhysRevD.93.054510
%     [arXiv:1512.07249 [hep-lat]].
%     %53 citations counted in INSPIRE as of 10 Jan 2022
    
%     %\cite{Astrakhantsev:2019zkr}
%     \bibitem{Astrakhantsev:2019zkr}
%     N.~Astrakhantsev, V.~V.~Braguta, M.~D'Elia, A.~Y.~Kotov, A.~A.~Nikolaev and F.~Sanfilippo,
%     %``Lattice study of the electromagnetic conductivity of the quark-gluon plasma in an external magnetic field,''
%     Phys. Rev. D \textbf{102}, 054516 (2020) 
%     % doi:10.1103/PhysRevD.102.054516
%      [arXiv:1910.08516 [hep-lat]].
%     %32 citations counted in INSPIRE as of 10 Jan 2022
    
%     %\cite{Greif:2014oia}
%     \bibitem{Greif:2014oia}
%     M.~Greif, I.~Bouras, C.~Greiner and Z.~Xu,
%     %``Electric conductivity of the quark-gluon plasma investigated using a perturbative QCD based parton cascade,''
%     Phys. Rev. D \textbf{90}, 094014 (2014) 
%     % doi:10.1103/PhysRevD.90.094014
%     [arXiv:1408.7049 [nucl-th]].
%     %89 citations counted in INSPIRE as of 10 Jan 2022
    
    
%     \bibitem{Satow:2014lia}
% 	D.~Satow,
% 	%\lq\lq Nonlinear electromagnetic response in quark-gluon plasma,\rq\rq
% 	Phys. Rev. D \textbf{90}, 034018 (2014)
% 	%doi:10.1103/PhysRevD.90.034018
% 	[arXiv:1406.7032 [hep-ph]].
    
%     \bibitem{Kharzeev:2007jp}
%     D.~E.~Kharzeev, L.~D.~McLerran and H.~J.~Warringa,
%     %``The Effects of topological charge change in heavy ion collisions: 'Event by event P and CP violation',''
%     Nucl. Phys. A \textbf{803}, 227 (2008)
%     % doi:10.1016/j.nuclphysa.2008.02.298
%     [arXiv:0711.0950 [hep-ph]].
    
%     \bibitem{Bass:2000ib}
%     S.~A.~Bass and A.~Dumitru,
%     %``Dynamics of hot bulk QCD matter: From the quark gluon plasma to hadronic freezeout,''
%     Phys. Rev. C \textbf{61}, 064909 (2000)
%     % doi:10.1103/PhysRevC.61.064909
%     [arXiv:nucl-th/0001033 [nucl-th]].
    
%     \bibitem{Muller:2018ibh}
%     B.~M\"uller and A.~Sch\"afer,
%     %``Chiral magnetic effect and an experimental bound on the late time magnetic field strength,''
%     Phys. Rev. D \textbf{98}, 071902 (2018) 
%     % doi:10.1103/PhysRevD.98.071902
%     [arXiv:1806.10907 [hep-ph]].
    
%     \bibitem{Letessier:1992xd}
%     J.~Letessier, A.~Tounsi, U.~W.~Heinz, J.~Sollfrank and J.~Rafelski,
%     %``Evidence for a high entropy phase in nuclear collisions,''
%     Phys. Rev. Lett. \textbf{70}, 3530 (1993)
%     % doi:10.1103/PhysRevLett.70.3530
%     [arXiv:hep-ph/9711349 [hep-ph]].



%   %\cite{Bjorken:1982qr}
%     \bibitem{Bjorken:1982qr}
%     J.~D.~Bjorken,
%     %``Highly Relativistic Nucleus-Nucleus Collisions: The Central Rapidity Region,''
%     Phys. Rev. D \textbf{27} (1983), 140-151
%     doi:10.1103/PhysRevD.27.140
%     %3331 citations counted in INSPIRE as of 03 Feb 2022
    
%     %\cite{Song:2010mg}
%     \bibitem{Song:2010mg}
%     H.~Song, S.~A.~Bass, U.~Heinz, T.~Hirano and C.~Shen,
%     %``200 A GeV Au+Au collisions serve a nearly perfect quark-gluon liquid,''
%     Phys. Rev. Lett. \textbf{106} (2011), 192301
%     [erratum: Phys. Rev. Lett. \textbf{109} (2012), 139904]
%     doi:10.1103/PhysRevLett.106.192301
%     [arXiv:1011.2783 [nucl-th]].
%     %453 citations counted in INSPIRE as of 17 Feb 2022
    
%     %\cite{Stewart:2021mjz}
%     \bibitem{Stewart:2021mjz}
%     E.~Stewart and K.~Tuchin,
%     %``Continuous evolution of electromagnetic field in heavy-ion collisions,''
%     Nucl. Phys. A \textbf{1016}, 122308 (2021)
%     % doi:10.1016/j.nuclphysa.2021.122308
%     [arXiv:2106.09124 [nucl-th]].
%     %3 citations counted in INSPIRE as of 12 Jan 2022
    
%     \bibitem{DeVries:1987atn}
%     H.~De Vries, C.~W.~De Jager and C.~De Vries,
%     %``Nuclear charge and magnetization density distribution parameters from elastic electron scattering,''
%     Atom. Data Nucl. Data Tabl. \textbf{36}, 495 (1987)
%     % doi:10.1016/0092-640X(87)90013-1
    
%     \end{thebibliography}
    
    
% % %%%%%%%%%%%%%%%

%%%%%%%%%%%%%%%%%



%%%%%%%%%%%%%%%%%%%%%%% 

\subsection{Electron-positron plasma in BBN}\label{chap:bbn}


%%%%%%%%%%%%%%%%%%%%%%%%%%%%%%%%%%%%%%%%%%
\paragraph{Electron positron plasma screening in BBN:}\label{sec:Discussion}

In this chapter, we review \cite{Grayson:2023flr}, which applies the non-relativistic longitudinal polarization function to study the dynamics of the electron-positron plasma in the early Universe. In particular, we discussed the damping rate, the electron-positron to baryon density ratio, and their potential implications for Big Bang Nucleosynthesis (BBN) through screening within linear response theory. We derived an approximate analytic formula for the potential of a moving heavy charge in a collisional plasma in \req{eq:pos_point_DDS} describing screening effects previously found only numerically \cite{Hwang:2021kno}. Our analytic formula can be readily used to estimate the effect of screening on thermonuclear reactions using \req{eq:DDSenhance}. The correction to thermonuclear reactions due to damped-dynamic screening is small due to the low velocity of nuclei and a large amount of collisional scattering. This is in line with the findings of \cite{Hwang:2021kno}, who conclude that even though the densities are large, they are not enough to modify the potential at short distances related to screening. The analytic expression we find for the nuclear reaction rate enhancement \req{eq:DDSenhance} in a collisional plasma could be useful in other fusion environments such as stellar fusion and laboratory fusion experiments, such as those discussed in ~\cite{Labaune:2013dla,Margarone:2022mdpi}.

Overall we were very surprised to find that the screening effects in BBN were so small even in the static case, considering that the number densities present during BBN are $\sim 10^4$ times normal matter. If we compare this to screening effects on Earth, we can see that although plasmas occur at lower densities, they also occur in much colder environments. The strength of the screening effect is related to the Debye mass
\begin{equation}
m_D^2 \sim \frac{n_\text{eq} }{T}\,,
\end{equation}
which is on the order of a few keV during BBN. On earth, $n_\text{eq}$ is decreased by $\sim 10^4$, but T is decreased by $\sim 10^6$. Thus, we would expect to see similar, if not larger, screening effects on Earth. For instance, the Debye screening length in extracellular fluid in the body is 8 \AA ngstrom \cite{Wennerstrom:2020}, only a factor of $\sim 20$ times larger than the Debye length during BBN. We can have these large densities at low temperatures on earth due to gravity's agglomeration of matter in the universe.

%%%%%%%%%%%%%%%%%%%%%%%%%%%%%%%%%%%%%%%%%%%%%%%%
\paragraph{The short-range screening potential}
In \cite{Grayson:2023flr}, a proposal is made to study the short-range potential relevant to quantum tunneling in thermonuclear reactions. Since the Gamow energy at which nuclei are most likely to tunnel is above the thermal energy, the portion of the screening potential relevant for tunneling does not satisfy the "weak-field" limit where the electromagnetic energy is small compared to the thermal energy
\begin{equation}
 \frac{q \phi(x)}{T} \ll 1\,.
\end{equation}
When this condition is not satisfied one must consider the full equilibrium distribution when calculating the short-range potential \cite{Hakim:1967prd,DeGroot:1980dk}
\begin{equation}\label{eq:Boltz}
 f_B^\pm(x,p) = e^{-(p_0\pm e\phi(x))/T}\,.
\end{equation}
The $e\phi$ term in the exponential accounts for the change in energy of a charge in the plasma due to its presence in an external field. For this equilibrium distribution, a linear response is no longer possible since the equilibrium distribution depends on the external electromagnetic field. In equilibrium one can find the static screening potential for strong electromagnetic fields using the nonlinear Poisson-Boltzmann equation,
\begin{equation}\label{eq:Poisson-Boltz}
 -\nabla^2 e\phi_{(\text{eq})}(x)/T +m_D^2\sinh\left[e\phi_{(\text{eq})}(x)/T\right] =e\rho_\mathrm{ext}(x)/T\,.
 \end{equation}
This equation has a well-known solution for an infinite sheet which we used to argue the importance of strong screening in BBN. 
In a future publication, we will solve the Poisson-Boltzmann equation with strong screening to calculate the short-range screening potential in BBN. We note that the toy model in \cite{Grayson:2023flr} overestimates strong screening effects for two reasons: an infinite sheet has a constant electric field requiring more polarizing charge density to screen the field, and the Boltzmann distribution in \req{eq:Boltz} does not account for the stacking of electron-positron states when the density of electrons and positrons becomes very large near the nucleus. Both of these effects significantly reduce the effect of strong screening on reaction rates, but at the time of writing, it seems that strong screening will create a larger effect on nuclear reaction rates than damped-dynamic screening. Predicting enhanced screening may be relevant for the anomalous screening observed in the measurements of astrophysical S(E) factors \cite{Zhang:2020nuc}.
 
%%%%%%%%%%%%%%%%%%%%%%%%%%%%%%%%%%%%%%%%%%%%%%%%%%%%%
\paragraph{Early Universe plasma: non-relativistic polarization tensor:}\label{sec:kinetic_theory}
The properties of the BBN plasma are described by the relativistic Vlasov-Boltzmann transport equations \req{eq:VBEf}. Since photons do not couple directly to the electromagnetic field, they do not contribute to the polarization tensor at first order in $\delta f$ as indicated in Eq.\,(\ref{eq:VBEg}). We neglect photon influence on the electron-positron distribution through the scattering term since the rate of inverse Compton scattering $R_{e^{\pm}\gamma }$ shown in green in \reff{RelaxationRate_fig} is much smaller, in the BBN temperature range, than the total rate $\kappa$ shown as a black line. Each fermion Boltzmann equation \req{eq:VBEf} can be solved independently. Since the equations for electrons and positrons are equivalent, except for the charge sign, only one needs to be solved to understand the dynamics.

%We do not consider the influence of light nuclei on the polarization tensor since their density during the BBN epoch is much smaller than that of electrons and positrons \reff{BBN_Electron}. One can see in \reff{MeanFreePath_fig} that the separation of baryons $n_B^{-1/3}$ in black is much larger than the size of the polarizing Debye sphere, so baryons do not participate significantly in screening.

We take the equilibrium one particle distribution function $\eq{f_\pm}$ of electrons and positrons to be the relativistic Fermi-distribution
\begin{equation}\label{eq:equildist}
\eq{f}_\pm(p) = \frac{1}{\exp{\left(\frac{\sqrt{\boldsymbol{p}^2 + m^2}}{T}\right)}
+1}\,,
\end{equation}
with chemical potential $\mu = 0 $. The electron and positron mass will be indicated by $m$ unless otherwise stated. At temperatures interesting for nucleosynthesis $T = 50-86$\,keV, we expect the plasma temperature to be much less than the mass of the plasma constituents. Only the non-relativistic form of Eq.\,(\ref{eq:equildist}) will be relevant at these temperature scales
\begin{equation}
\eq{f}_\pm(p) \approx \exp\left(- \frac{m}{T}\left(1+\frac{|\pmb{p}|^2}{2m^2}\right)\right)\,.
\end{equation}
Keeping terms up to quadratic order in $|\boldsymbol{p}|/m$ we solve the Vlasov-Boltzmann equation Eq.\,(\ref{eq:VBEf}) for the induced current and identify the polarization tensor. This is done in detail in our previous work in~\cite{Formanek:2021blc}.

% First, we expand Eq.\,(\ref{eq:VBEf})
% around small perturbations from equilibrium
% \begin{equation}\label{eq:perturbation0}
% f_\pm(x,p) = {\eq{f}_\pm}(p) + \delta f_\pm(x,p)\,,
% \end{equation}
% and solve \req{eq:VBEf} for $\delta f_\pm(x,p)$ in Fourier space. The induced current in Fourier space is given by
% \begin{equation}\label{eq:perturbation1}
% \tilde{j}_{\mathrm{ind}}^\mu(k) = 2\int \frac{d^4 p}{(2 \pi)^4}p^\mu 4\pi \delta_+(p^2-m^2) 
% \sum_{i = \, +, \, -} q_i \tilde{f}_{i}(k,p)\,,
% \end{equation}
% with the factor of two accounting for spin.
% After inserting \req{eq:perturbation0}, and specifying $q_\pm = \pm e$ the induced current is a function of the perturbation
% \begin{multline}\label{eq:perturbation2}
% \tilde{j}_{\mathrm{ind}}^\mu(k) = 2\int \frac{d^3 p}{(2 \pi)^3 p^0}p^\mu \Big( e \left[\eq{\tilde{f}}_+(k,p)-\eq{\tilde{f}}_-(k,p)\right]\\
% + e\left[\delta\tilde{f}_+(k,p)-\delta\tilde{f}_-(k,p)\right]
% \Big)
% \\
% =4 e\int \frac{d^3 p}{(2 \pi)^3 p^0}p^\mu \delta\tilde{f}(k,p)
% \,,
% \end{multline}
% because the equilibrium currents cancel in the weak field limit and the perturbations add since they differ by the charge $\delta f_\pm=\pm e \delta f' $. This term is studied for finite chemical potential in~\cite{Wang:2010px}. We focus on the second term related to the polarization response of the plasma.

% The polarization tensor can be obtained from \req{eq:perturbation2} by computing the 4-momentum integrals over $p$ in the rest frame of the plasma. Once integrated, the 4-potential $\widetilde{A}^{\nu}(k)$, coming from the electromagnetic field strength tensor $F^{\mu \nu}$ in \req{eq:VBEf}, factors out since it only depends on the 4-wavevector $k$ and \req{eq:perturbation2} attains the form of \req{eq:linresp}. For details, see Ref.~\cite{Formanek:2021blc}.

In the infinite homogeneous plasma filling the early Universe, the polarization tensor only has two independent components: the longitudinal polarization function $\Pi_{\parallel}$ parallel to field wave-vector $\boldsymbol{k}$ in the rest frame of the plasma and the transverse polarization function $\Pi_{\perp}$ perpendicular to $\boldsymbol{k}$~\cite{melrose2008quantum}. In the non-relativistic limit, these functions are~\cite{Formanek:2021blc}
\begin{align}\label{eq:polfuncs}
	\Pi_\parallel(\omega,\boldsymbol{k}) &= -\omega_p^2\frac{\omega^2}{(\omega+ i \kappa)^2} \frac{1}{1-\frac{i\kappa}{\omega+ i \kappa}\left(1+\frac{ T |\boldsymbol{k}|^2}{m (\omega+ i \kappa)^2} \right)}\,,\\
	\Pi_{\perp}(\omega) &= -\omega_p^2 \frac{\omega}{\omega+ i \kappa}\,.
\end{align}
In these expressions, the plasma frequency $\omega_p$ (defined as $m_L$ in~\cite{Formanek:2021blc}) is related to the Debye screening mass in the non-relativistic limit as
\begin{equation}\label{eq:plasmafreq}
 \omega_p^2 = m_D^2\frac{T}{m}\,.
\end{equation}

%%%%%%%%%%%%%%%%%%%%%%%%%%%%%%%%%
\begin{figure} 
\centerline{\includegraphics[width=0.90\linewidth]{plots/chap03BBN/Distance_Plasma002.jpg}}
\caption{ The average distance between baryons $n_B^{-1/3}$ and the Debye length $\lambda_D$ ($\mu_e \neq 0$) as a function of temperature (red solid line). During the BBN epoch (vertical dotted lines) $n_B^{-1/3}>\lambda_D$. For temperature below $T<32.76$ keV we have $n_B^{-1/3}<\lambda_D$. For comparison, the Debye length for zero chemical potential $\mu_e=0$ is also plotted as a blue dashed line. Figure by C.T.\,Yang, appeared in~\cite{Grayson:2023flr}}
\label{MeanFreePath_fig} 
\end{figure}
%%%%%%%%%%%%%%%%%%%%%%%%%%%%%%%%%%%%%%

The transverse response $\Pi_{\perp}$ relates to the dispersion of photons in the plasma. Here we need only consider $\Pi_\parallel$ since the vector potential $\boldsymbol{A}(t,\boldsymbol{x})$ of the traveling ion will be small in the non-relativistic limit. This work does not consider the effect of a primordial magnetic field discussed in~\cite{Steinmetz:2023abc}. We note that Debye mass $m_D$ is related to the usual Debye screening length of the field in the plasma as
\begin{equation}\label{eq:mL}
	1/\lambda_D^{2} = m_D^2= 4 \pi \alpha \left(\frac{2mT}{\pi}\right)^{3/2}\frac{e^{-m/T}}{2T}\,.
\end{equation}
This formula describes the characteristic length scale of screening in the plasma.

%%%%%%%%%%%%%%%%%%%%%%%%%%%%%%%%%%%%%%%%%%%%%
\paragraph{Longitudinal dispersion relation:}
As discussed in Chapter \ref{chap:PlasmaSF} the poles in the propagator or roots of the dispersion equation represent the plasma's propagating modes, often called `quasi-particles' or `plasmons.' In the non-relativistic limit, one can solve the longitudinal part of the dispersion equation \req{eq:disp}, which is relevant for finding charge oscillation modes in the plasma
\begin{equation}
    1+ \frac{\Pi_\parallel( k)}{(p\cdot u)^2}= 1+ \frac{\Pi_\parallel(\omega, \boldsymbol{k})}{\omega^2}=\varepsilon_\parallel(\omega,\boldsymbol{k}) =0 \,,
\end{equation}
evaluated in the rest frame. Then we insert \req{eq:polfuncs} to find
\begin{equation}
   1- \frac{\omega_p^2}{(\omega+ i \kappa)^2} \frac{1}{1-\frac{i\kappa}{\omega+ i \kappa}\left(1+\frac{T |\boldsymbol{k}|^2}{m(\omega+ i \kappa)^2} \right)}=0 \,.
\end{equation}
We can simplify the above expression since this is only a function of $\omega' =\omega+i\kappa$
\begin{equation}
   1- \frac{\omega_p^2}{\omega'^2-i\kappa\omega'+\frac{i\kappa T |\boldsymbol{k}|^2}{m \omega'} }=0 \,.
\end{equation}
Then we get a cubic equation for $\omega'(|\boldsymbol{k}|)$
\begin{equation}\label{eq:dispfact}
   \frac{1}{\omega'^3-i\kappa\omega'^2+\frac{i\kappa T |\boldsymbol{k}|^2}{m} }
    \left(\omega'^3-i\kappa\omega'^2 - \omega_p^2\omega'+\frac{i\kappa T |\boldsymbol{k}|^2}{m} \right)=0 \,.
\end{equation}
Cardano's formula gives the solutions to this cubic equation
\begin{equation}\label{eq:cardano}
\omega_n(\boldsymbol{k}) = \frac{1}{3}\left(i\kappa-\xi^n C-\frac{\Delta_0}{\xi^n C}\right), \qquad n \in \{0,1,2\} \,,
\end{equation}
with the quantities:
\begin{align}\label{eq:delta}
  \xi &=\frac{i\sqrt{3}-1}{2}\,,\\
    C &= \sqrt[3]{\frac{\Delta_1 \pm \sqrt{\Delta_1^2 - 4 \Delta_0^3}}2}\,,\\
    \Delta_0 &= -\kappa^2 + 3 \omega_p^2\,,\\
\Delta_1 &= 2i\kappa^3 - 9 i\kappa \omega_p^2 + 27\frac{i\kappa T |\boldsymbol{k}|^2}{m}.
\end{align}
Since the longitudinal dispersion relation is analytically solvable the full non-relativistic potential can be found in position space using contour integration. The residue of each pole will lead to the strength of that mode, and the location of the pole will lead to space and time dependence, which in simple cases is exponential. In practice, factoring out these roots in the Fourier transform of the potential leads to five poles, which do not seem to lead to simple expressions in position space. We found using the approximate expression derived in \req{sec:potential} was more practical. Deriving the full expression is the subject of future work.

%\begin{figure}[H]
%    \centering
%    \includegraphics[width=0.95\linewidth]{plasmafreq.png}
%    \caption{Plot of the plasma frequencies $\omega_\pm$ in the complex plane as a function of $\kappa$. At $\kappa=2\omega_p$ both solutions become imaginary, one becoming more quickly damped and the other more slowly damped.}
%    \label{fig:plasma-freq}
%\end{figure}

%%%%%%%%%%%%%%%%%%%%%%%%%%%%%%%%%%%%%%%%%%%%%%%%%%%%%%%%%
\paragraph{Damped-dynamic screening} 
We discuss the application of the non-relativistic limit of the polarization tensor \rsec{chap:PlasmaSF} to the electron-positron plasma which existed during Big Bang nucleosynthesis (BBN)~\cite{Grayson:2023flr}. The BBN Epoch occurred within the first 20 min after the Big Bang when the Universe was hot and dense enough for nuclear reactions to produce light elements up to lithium \cite{Pitrou:2018cgg}. 

The BBN nuclear reactions typically take place within the temperature interval $86\, \mathrm{keV}>\mathrm{T_{BBN}}>50\, \mathrm{keV}$~\cite{Pitrou:2018cgg}. We refer to these elements produced in BBN as primordial light elements to distinguish them from those made later in the Universe's history. Primordial light element abundances are the most accessible probes of the early Universe before recombination. Though the current BBN model successfully predicts D, $^3$He, $^4$He abundances, well-documented discrepancies, such as $^7$Li, remain. Efforts to resolve the theoretical BBN model with present-day observations are discussed in detail in \cite{Pitrou:2021vqr,Bertulani:2022qly}.

A rather large electron-positron $e^-e^+$- number densities existed in the early Universe during Big Bang nucleosynthesis (BBN)~\cite{Wang:2010px,Hwang:2021kno,Rafelski:2023emw} are $10^2$ times larger than those present in the Sun \cite{Bahcall:2001smc} and $10^4$ times normal atomic densities \cite{Grayson:2023flr}. Charge screening is an essential collective plasma effect that modifies the inter-nuclear potential $\phi(r)$ changing thermonuclear reaction rates during BBN. An electron cloud around an ion's charge effectively diminishes the influence of nuclear charges beyond their immediate vicinity, lowering the Coulomb barrier. 

In the context of nuclear reactions, a reduced Coulomb barrier leads to a higher likelihood of penetration, boosting thermonuclear reaction rates. Consequently, this process influences the abundance of light elements in the early universe by modifying their formation rates. Since the BBN temperature range is much less than the electron mass, we will use the non-relativistic limit of the polarization tensor derived in Chapter \ref{chap:PlasmaSF}. The screened potential relevant for thermonuclear reactions will be given by the longitudinal polarization function \req{eq:phi}.

The influence of screening on nuclear reactions is a well-established field of study. The concept of plasma screening effects on nuclear reactions was initially introduced in~\cite{Salpeter:1954nc}, who suggested determining the increase in nuclear reaction rates through the use of the static Debye-Hückel potential~\cite{Debye:1923,Salpeter:1969apj,Famiano:2016hhs}. Subsequent research expanded this framework to account for the thermal velocity of nuclei traversing the plasma~\cite{Hwang:2021kno,Carraro:1988apj,Gruzinov:1997as,Opher:1999jh,Yao:2016cjs}, introducing the concept of `dynamic' screening. 

In our current study, we address the high density of the $e^-e^+\gamma$ plasma by including collisional damping using the current conserving collision term developed in \cite{Formanek:2021blc} shown in \req{eq:collision}. The dense aspect of the BBN plasma has only recently been acknowledged by incorporating collision effects into numerical models \cite{Sasankan:2019oee,Kedia:2020xdc}. We will refer to this model of screening as 'damped-dynamic' screening. In \cite{Grayson:2023flr}, we find an analytic formula for the induced screening potential, which allows for estimating the enhancement of thermonuclear reaction rates.

%%%%%%%%%%%%%%%%%%%%%%%%%%%%%%%%%%%%%%%%%%%%%%%%%%%%%%%%
\paragraph{Nuclear potential:}\label{sec:DDS}
We consider the effective nuclear potential for a light nucleus moving in the plasma at a constant velocity. This is done by Fourier transforming \req{eq:potentk}. The velocity of the nucleus is assumed to be the most probable velocity given by a Boltzmann distribution
\begin{equation}\label{eq:vel}
 \beta_{\text{N}} = \sqrt{\frac{2T}{m_N}}\,. 
\end{equation}
Since the poles of the \req{eq:potent} can be solved analytically, ideally, one would perform contour integration to get the position space field. Due to the intricacy of these poles $\omega_n(\boldsymbol{k})$, we find it insightful to look at the field in a series expansion around velocities of the light nuclei smaller than the thermal velocity of electrons and positrons and large damping.

% \begin{equation}
% \beta_{\text{N}}\frac{m_D}{\kappa} = \frac{v_{\text{N}}}{c}\frac{m_D}{\kappa} \ll 1\,.
% \end{equation}
\begin{equation}\label{eq:expansion}
% (\boldsymbol{k}\cdot\boldsymbol{\beta}_{\text{N}})^2 \ll \boldsymbol{k}^2 \frac{T}{m} \ll \kappa^2\, 
\frac{(\boldsymbol{k}\cdot\boldsymbol{\beta}_{\text{N}})^2}{\omega_p^2} \ll \frac{\boldsymbol{k}^2}{m_D^2} \ll \frac{\kappa^2}{\omega_p^2}\, .
\end{equation}


This expansion is useful during BBN since the temperature is much lower than the mass of light nuclei and the damping rate $\kappa$ is approximately twice the Debye mass $m_D$, as seen in Fig.~\ref{RelaxationRate_fig}. Applying this expansion to \req{eq:potentk} and evaluating this expression for a point charge $r \rightarrow 0$ we find
% % \begin{multline} \label{eq:potexp}
% % \phi(t,\boldsymbol{x}) = \phi_{\text{stat}}(t,\boldsymbol{x}) +\\ -Ze\int \frac{d^3\boldsymbol{k}}{(2\pi)^3} e^{ i\boldsymbol{k}\cdot(\boldsymbol{x}-\boldsymbol{\beta_{\text{N}}} t)}\frac{i \boldsymbol{k}\cdot \boldsymbol{\beta_{\text{N}}} m_D^2 (\frac{\boldsymbol{k}^2}{\kappa} - \frac{m}{T} \kappa)}{\boldsymbol{k}^2(\boldsymbol{k}^2+m_D^2)^2} e^{-\boldsymbol{k}^2\frac{R^2}{4}} \\ + O
% % \left(\beta_{\text{N}}\frac{m_D}{\kappa}^2\right)\,.
% % \end{multline}
% \begin{multline} \label{eq:potexp}
%  \phi(t,\boldsymbol{x}) = \phi_{\text{stat}}(t,\boldsymbol{x})\\ -Ze\int \frac{d^3\boldsymbol{k}}{(2\pi)^3} e^{ i\boldsymbol{k}\cdot(\boldsymbol{x}-\boldsymbol{\beta_{\text{N}}} t)}\frac{i \boldsymbol{k}\cdot \boldsymbol{\beta_{\text{N}}} m_D^4 (\frac{\boldsymbol{k}^2}{m_D^2} - \frac{\kappa^2}{\omega_p^2})}{\boldsymbol{k}^2(\boldsymbol{k}^2+m_D^2)^2\kappa} e^{-\boldsymbol{k}^2\frac{R^2}{4}} \,.
% \end{multline}
% First, we focus on the second term evaluating for a point charge $R\rightarrow 0$
\begin{equation}\label{eq:ddsint}
\phi(t,\boldsymbol{x}) =\phi_{\text{stat}}(t,\boldsymbol{x})-Ze\int \frac{d^3\boldsymbol{k}}{(2\pi)^3} e^{ i\boldsymbol{k}\cdot(\boldsymbol{x}-\boldsymbol{\beta_{\text{N}}} t)}\frac{i \boldsymbol{k}\cdot \boldsymbol{\beta_{\text{N}}} m_D^4 (\frac{\boldsymbol{k}^2}{m_D^2} - \frac{\kappa^2}{\omega_p^2})}{\boldsymbol{k}^2(\boldsymbol{k}^2+m_D^2)^2\kappa}\,.
\end{equation}
The second term is the damped-dynamic screening correction, which we refer to as $\Delta \phi$, where
\begin{equation}\label{eq:pos_point}
\phi(t,\boldsymbol{x}) = \phi_{\text{stat}}(t,\boldsymbol{x}) +\Delta \phi(t,\boldsymbol{x}) \,,
\end{equation}
and $\phi_{\text{stat}}$ is the standard static screening potential. The details of the integration of \req{eq:ddsint} can be found in \cite{Grayson:2023flr}, the result is
% In order to perform this integration we first note that this expression can be re-written as the laboratory time derivative
% \begin{equation}
%  \Delta\phi(t,\boldsymbol{x}) = Ze\frac{d}{dt}\left[\int \frac{d^3\boldsymbol{k}}{(2\pi)^3} e^{ i|\boldsymbol{k}||\boldsymbol{x}-\boldsymbol{\beta_{\text{N}}} t| \cos(\theta)}\frac{ m_D^4 (\frac{\boldsymbol{k}^2}{m_D^2} - \frac{\kappa^2}{\omega_p^2})}{\boldsymbol{k}^2(\boldsymbol{k}^2+m_D^2)^2\kappa}\right] \,,
% \end{equation}
% where the dot products were replaced by their angular dependence.
% The angular integration can be evaluated in spherical coordinates, see \ref{sec:static} for details,
% \begin{equation}
%  \Delta\phi(t,\boldsymbol{x}) = 2 Ze\frac{d}{dt}\left[\int \frac{d\boldsymbol{k}}{(2\pi)^2} \frac{\sin(|\boldsymbol{k}||\boldsymbol{x}-\boldsymbol{\beta}_N t|)}{|\boldsymbol{k}||\boldsymbol{x}-\boldsymbol{\beta}_N t|}\,\frac{ m_D^4 (\frac{\boldsymbol{k}^2}{m_D^2} - \frac{\kappa^2}{\omega_p^2})}{\boldsymbol{k}^2(\boldsymbol{k}^2+m_D^2)^2\kappa}\right] \,,
% \end{equation}
% \begin{multline}
%  \Delta\phi(t,\boldsymbol{x}) = \frac{Ze }{4\pi }\frac{m_D^2}{\kappa}\frac{d}{dt}\Bigg[\left(\frac{1+\nu_\tau^2}{ 2m_D}+\frac{\nu_\tau^2}{m_D^2 |\boldsymbol{x}-\boldsymbol{\beta}_N t|}\right)e^{-m_D |\boldsymbol{x}-\boldsymbol{\beta}_N t|}\\ - \frac{\nu_\tau^2}{m_D^2 |\boldsymbol{x}-\boldsymbol{\beta}_N t|} \Bigg]\,,
% \end{multline}
% here we introduce the ratio of the damping rate to the rate of oscillations in the plasma $\nu_\tau = \kappa/\omega_p$. We can apply the time derivative; note that
% \begin{equation}\label{eq:der_cos}
%  \frac{d}{dt} |\boldsymbol{x}-\boldsymbol{\beta}_N t| = - \frac{\boldsymbol{\beta}_N \cdot (\boldsymbol{x}-\boldsymbol{\beta}_N t)}{|\boldsymbol{x}-\boldsymbol{\beta}_N t|}= -\beta_N \cos(\psi)\,,
% \end{equation}
% where $\psi$ is the angle between $\boldsymbol{x}-\boldsymbol{\beta}_N t$ and $\boldsymbol{\beta}_N$. After taking the derivative and using the above expression \req{eq:der_cos} one finds

%%%%%%%%%%%%%%%%%%%%%%%%%%%%%%%%%%%%%
\begin{figure} 
 \centerline{
\includegraphics[width=.90\linewidth]{plots/chap03BBN/phidat_100_1_1_0_full_lin.pdf}}
 \caption{ \cccite{Grayson:2023flr}. Plot of the total screening potential scaled with charge Z and distance along the direction of motion. We show a comparison of the following screening models plotted along the direction of motion of a nucleus $\boldsymbol{r}\cdot\hat{\boldsymbol{\beta}_{\text{N}}}$: static screening (black), dynamic screening (red dotted) from \cite{Hwang:2021kno}, damped-dynamic screening (blue dashed), and the approximate analytic solution of \req{eq:pos_point} (orange dashed). A black arrow indicates the direction of motion of the nucleus $\hat{\boldsymbol{\beta}_{\text{N}}}$.}
 \label{fig:dynamiclinear}
\end{figure} 
%%%%%%%%%%%%%%%%%%%%%%%%%%%%%%%%%%%%%%%%%%%%

\begin{multline}\label{eq:pos_point_DDS}
\Delta \phi(t,\boldsymbol{x}) = \frac{Ze \beta_N \cos (\psi) m_D^2}{4 \pi \varepsilon_0 \kappa} \Bigg[\left(\frac{\nu_\tau^2}{m_D^2 r(t)^2} + \frac{\nu_\tau^2}{m_D r(t)}+\frac{1 + \nu_\tau^2}{2}\right)e^{-m_D r(t)} \\ -\frac{\nu_\tau^2}{m_D^2 r(t)^2}\Bigg]\,,
\end{multline}
where $\psi$ is the angle between $\boldsymbol{x}-\boldsymbol{\beta}_N t$ and $\boldsymbol{\beta}_N$ and $r(t) = |\boldsymbol{x}-\boldsymbol{\beta}_N t|$.
We introduce the ratio of the damping rate to the rate of oscillations in the plasma $\nu_\tau = \kappa/\omega_p$.  This expression is valid for large damping and slow motion of the nucleus or if the velocity of the nuclei is small. A similar result valid at large distances, which only includes the last term, was previously derived in~\cite{Stenflo:1973} for dusty (complex) plasmas. For large distances and large $\nu_\tau$, the last term in the second line is dominant, indicating that the overall potential would be over-damped. In this regime, the potential is heavily screened in the forward direction and unscreened in the backward direction relative to the motion of the nucleus. As $\nu_\tau$ becomes small, the $1/2$ in the first portion of the third term, proportional to $m_D^2/\kappa$, dominates. This flips the sign of the damped-dynamic screening contribution causing a wake potential to form behind the nuclei. This shift indicates the change from damped to undamped screening where \req{eq:pos_point_DDS} is no longer valid. 

%%%%%%%%%%%%%%%%%%%%%%%%%%%%%%%%%%
\begin{figure} 
 \centerline{\includegraphics[width=.90\linewidth]{plots/chap03BBN/Pot_2DPlotFix.png}}
 \caption{Two dimensional plot of the total potential \req{eq:pos_point} scaled with Z, at $T=74\,$keV. The potential is cylindrically symmetric about the direction of motion $\boldsymbol{\hat{z}}$, which is indicated by a black arrow. The direction transverse to the motion is $\rho$. The sign of the damped-dynamic correction \req{eq:pos_point_DDS} changes sign due to the cosine term}
 \label{fig:numericalComp}
\end{figure} 
%%%%%%%%%%%%%%%%%%%%%%%%%%%%%%%%%%%%%

Figure~\ref{fig:dynamiclinear} demonstrates that the damped-dynamic response in the analytic approximation \req{eq:pos_point_DDS} (shown as orange dashed line) is sufficient to approximate the full numerical solution (blue dashed line) found by numerical integration of \req{eq:potent}. The temperature $T = 100$\, keV, above our upper limit of BBN temperatures, is chosen to relate to the dynamic screening result found in~\cite{Hwang:2021kno}. Our analytic solution differs from the numerical result in Fig.~4 of \cite{Hwang:2021kno} by a factor of $\sqrt{2}$ and is horizontally flipped. This reflection is due to a difference in convention in the permittivity, as seen in \req{eq:potentk}. We can see that dynamic screening is slightly stronger at large distances than damped screening, as expected. Damped and undamped screening are very similar at short distances, which is relevant to thermonuclear reaction rates. 

Dynamic screening in both the damped and undamped cases predicts less screening behind and more in front of the moving nucleus than static screening. This is shown in the two-dimensional plot \reff{fig:numericalComp}, of the total potential in plasma at $T=76\,$keV This effect was previously observed for subsonic screening in electron-ion-dust plasmas ~\cite{Stenflo:1973,Shukla:2002ppcf,Lampe:2000pop}. As a result, a negative polarization charge builds up in front of the nucleus. The small negative potential in front alters the potential energy between light nuclei, possibly changing the equilibrium distribution of light elements in the early universe plasma. This effect is much larger in the undamped case and is known in some cases to lead to the formation of dust crystals~\cite{Shukla:1996ccc}. 

%%%%%%%%%%%%%%%%%%%%%%%%%%%%%%%%%%%%%%%%%%%%%%%%%%
\subsection{Effective internuclear potential}\label{sec:potential}
We calculate the potential of light nuclei in the early Universe electron-positron plasma by Fourier transforming the screened scalar potential $\phi$ of a single traveling nuclei \req{eq:phi}
\begin{equation}\label{eq:potent}
 \phi(t,\boldsymbol{x}) = \int \frac{d^4k}{(2\pi)^4} e^{-i\omega t+ i\boldsymbol{k}\cdot\boldsymbol{x}} \frac{\widetilde{\rho}_\text{ext}(\omega,\boldsymbol{k})}{\varepsilon_\parallel(\omega,\boldsymbol{k})(\boldsymbol{k}^2-\omega^2) }\,,
\end{equation}
where $\widetilde{\rho}_{\text{ext}}(\omega,\boldsymbol{k})$ is the Fourier-transformed charge distribution of nuclei traveling at a constant velocity, and $\varepsilon_\parallel(\omega,\boldsymbol{k})$ is the longitudinal relative permittivity. The relative permittivity can be written in terms of the polarization tensor as
\begin{equation}\label{eq:epsilon}
 \varepsilon_\parallel(\omega,\boldsymbol{k})= \left(\frac{\Pi_{\parallel}(\omega,\boldsymbol{k})}{ \omega^2}+1\right)\,.
\end{equation}

In the linear response framework \req{eq:ohm}, the electromagnetic field still obeys the principle of superposition so the potential between two nuclei can be inferred simply from the potential of a single nucleus. 

We can perform the $\omega$ integration in \req{eq:potent} using the delta function in the definition of the external charge distribution, whose effect is to set $\omega = \boldsymbol{\beta_{\text{N}}}\cdot \boldsymbol{k}$ where $\boldsymbol{\beta}_N = \boldsymbol{v}_N/c$ is the nuclei velocity. Then we have
\begin{equation}\label{eq:potentk}
 \phi(t,\boldsymbol{x}) = Ze\int \frac{d^3\boldsymbol{k}}{(2\pi)^3} e^{ i\boldsymbol{k}\cdot(\boldsymbol{x}-\boldsymbol{\beta_{\text{N}}} t)} \frac{ e^{-\boldsymbol{k}^2\frac{R^2}{4}}}{\boldsymbol{k}^2\varepsilon_\parallel(-\boldsymbol{\beta_{\text{N}}} \cdot \boldsymbol{k},\boldsymbol{k}) }\,,
\end{equation}
where $R$ is the Gaussian radius parameter.
In non-relativistic approximation the Lorentz factor $\gamma \approx 1$ and we use the convention $\varepsilon_\parallel(-\boldsymbol{\beta_{\text{N}}} \cdot \boldsymbol{k},\boldsymbol{k})$ used in~\cite{Montgomery:1970jpp,Stenflo:1973,Shukla:2002ppcf,Shukla:1996ccc} which gives the correct causality for the potential. This ensures that, without damping, the wakefield occurs behind the moving nucleus.

%%%%%%%%%%%%%%%%%%%%%%%%%%%%%%%%%%%%%%%%%%%%%%%%
\paragraph{Reaction rate enhancement}
We use the same argument as \cite{Salpeter:1954nc} to find the enhancement factor due to damped-dynamic screening. The enhancement of a nuclear reaction process by screening is related to the WKB probability of tunneling through the Coulomb barrier
\begin{equation} \label{eq:penprob}
    P(E) = \exp{\left( - \frac{2\sqrt{2 \mu_r}}{\hbar c}\int_{R}^{r_c}dr \sqrt{U(r)-E}\right)}\,,
\end{equation}
often referred to as the penetration factor. $U(r)$ is the potential energy of the two colliding nuclei, $\mu_r$ is their reduced mass, $E$ is the relative energy of the collision, $R$ is the radius of the nucleus, and $r_c$ is the classical turning point. In the weak screening limit, the screening charge density varies on the scale of $\lambda_D$, which is here on the order of \AA ngstrom. The distance scales relevant for tunneling are between $R$ and $r_c$, which is on the order of $10\,$fm. This allows us to approximate the contribution to the potential energy from screening, $H(r)$ defined as
\begin{equation}
    H(r) \equiv U(r) - U_\text{vac}(r)\,,
\end{equation}
as constant over the integral in \req{eq:penprob} taking the value of \req{eq:pos_point_DDS} at the origin,
\begin{equation}
     H(0) = Z_1\phi_2(0) = Z_1 Z_2 \alpha \left(m_D - \frac{\beta_N m_D^2}{2 \kappa}\right)\,.
\end{equation}
Then, the screening effect reduces to a constant shift in the relative energy $E \rightarrow E+H(0)$. In this approximation, the enhancement to reaction rates can be represented by a single factor \cite{Salpeter:1954nc,Kravchuk:2014sps}
\begin{equation}\label{eq:DDSenhance}
   \mathcal{F} = \exp\left[\frac{H(0)}{T} \right]=\exp\left[\frac{Z_1 Z_2 \alpha}{T} \left(m_D - \frac{\beta_N m_D^2}{2 \kappa}\right)\right]\,.
\end{equation}
This result is only valid in the weak damping limit $\omega_p<\kappa$. The first term is the normal weak field screening result, and the second is the contribution of damped-dynamic screening. Due to the large damping rate in comparison to the Debye mass and the small velocities of nuclei \req{eq:vel} during BBN, the correction due to damped dynamic screening is small, changing $H(0)$ by $10^{-5}$. 
% \subsection{Limitations of linear response: toy model}
% \label{ssec:limitLR}
% We now consider the equilibrium phase space distribution in the presence of a `strong' potential. In covariant form and not neglecting the magnitude of the 4-potential $A^\mu$ as compared to the energy-momentum $p^\mu$ of the particle in statistical ensemble, one obtains~\cite{Hakim:1967prd,DeGroot:1980dk}
% \begin{equation}
%  \eq{f}(x,p) = e^{-u_{\mu}(p^{\mu}+q A_{(\text{eq})}^{\mu}(x))/T}\,,
% \end{equation}
% where $u^{\mu} = (1,0,0,0)$ is the global velocity of the plasma in its rest frame. The density according to this distribution is given by the usual expression but altered by the exponential factor in potential
% \begin{equation}
%  n_{eq} = \int \frac{dp^3}{(2 \pi)^3 p^0} p^0 \eq{f}(x,p) = n_{eq} e^{-u^{\mu}q A_{(\text{eq})}^{\mu}(x)/T}\,.
% \end{equation}
% Inserting the equilibrium density into the Poisson equation
% \begin{equation}
%  -\nabla^2 \phi_{(\text{eq})}(x) =\rho_\mathrm{tot}(x)=\rho_\mathrm{ext}(x)+\rho_\mathrm{ind}(x)\,,
% \end{equation}
% gives the nonlinear Poisson-Boltzmann equation in equilibrium
% \begin{equation}
%  -\nabla^2 \phi_{(\text{eq})}(x) +4e n_{(\text{eq})} \sinh\left[e\phi_{(\text{eq})}(x)/T\right] =\rho_\mathrm{ext}(x)\,.
% \end{equation}
% We can re-scale the potential with temperature to rewrite as
% \begin{equation}\label{eq:Poisson-Boltz}
%  -\nabla^2 \Phi_{(\text{eq})}(x) +m_D^2\sinh\left[\Phi_{(\text{eq})}(x)\right] =e\rho_\mathrm{ext}(x)/T\,,
% \end{equation}
% where $\Phi_{(\text{eq})}(x) = e\phi_{(\text{eq})}(x)/T$. This equation has a well-known solution for an infinite sheet, where the external charge $\rho_\mathrm{ext}$ becomes a fixed boundary condition. We will set the potential at the sheet $\phi(0) = \phi_0$ and solve the Poisson-Boltzmann equation in the region to the right of the sheet ($x > 0$)
% \begin{equation}
%  -\frac{d^2}{dx^2}\Phi_{(\text{eq})}(x) +m_D^2\sinh\left[\Phi_{(\text{eq})}(x)\right] =0\,.
% \end{equation}
% The solution to this problem can be found for example in Ref.~\cite{Gouy:1910}
% \begin{equation}\label{eq:nonlinscreen}
%  \Phi_{(\text{eq})}(x) = \frac{e\phi_{(\text{eq})}(x)}{T} = 4 \tanh^{-1} \left[\tanh\left(\frac{e\phi_0}{4T}\right)e^{- m_D x}\right]\,,
% \end{equation}
% which we can compare to the usual Debye screening for a charged plane
% \begin{equation}\label{eq:linscreen}
%  \phi_{(\text{eq})}(x) =\phi_0e^{-m_D x}\,.
% \end{equation}

% \begin{figure}[ht]
%  \centering
%  \includegraphics[width=.95\linewidth]{plots/chap03BBN/nonLinScreen.jpg}
%  \caption{Comparison of nonlinear screening to Debye linearized screening (solid top line) in the range $50\le T\le 400$\,keV; the BBN temperature range is shaded. In this toy model example $Z=2$ was used in \req{eq:phi0}.
%  } \label{fig:nonlinscreen}
% \end{figure}

% We contrast linear screening \req{eq:linscreen} and nonlinear screening \req{eq:nonlinscreen} in \reff{fig:nonlinscreen}. The boundary value of the potential $\phi_0$ is chosen to be the maximum value of the potential of a Gaussian charge distribution
% \begin{equation}\label{eq:phi0}
%  \phi_0 = \frac{Z e}{8 \pi^{3/2} \varepsilon_0 R}\,.
% \end{equation}
% Figure \ref{fig:nonlinscreen} shows that nonlinear screening and linear screening predict different decay rates for the potential $\phi$ at short distances but have the same long-range decay rate. Since the screening near the origin is larger in the nonlinear case, the long-distance potential $\phi$ does not approach the weak field limit but is offset. As \reff{fig:nonlinscreen} demonstrates this is especially true in the BBN temperature range. All solutions have an exponential slope $-m_D$ indicating linear or weak field behavior at large distances.

% The important prediction of the nonlinear screening in \reff{fig:nonlinscreen} is a rapid drop of the potential at short distances, thus a strong screening effect near the classical turning point of thermal reacting ions, where quantum tunneling for thermonuclear reactions will be important. This effect increases nonlinearly with $Z$, see \req{eq:nonlinscreen}. Note that the use of relativistic Fermi statistics may be required since the tunneling regime involves potentials that are comparable to and larger than the electron mass. We expect that nonlinear screening effects could lead to drastic changes in the Coulomb penetration factor for $Z>1$ and, in turn, light element abundances. 

% To describe nonlinear screening beyond the presented toy model, in the context of light element screening during BBN, we need to solve \req{eq:Poisson-Boltz} with a realistic spherical charge distribution. That way the difference in asymptotic behavior will be captured for the solution at large distances. It turns out that even the simple first look at the 3D-static asymptotic Debye-type solution requires formidable numerical effort, let alone merging the short-range strong field limit with damped-dynamic linear response theory as developed here. Therefore the strong field screening will be addressed separately, opening the path to the computation of screened-BBN reaction rates.

 
% To improve the understanding of the internuclear screened Coulomb potential $\phi$ governing BBN reaction rates, this work expands the prior studies~\cite{Hwang:2021kno, Carraro:1988apj,Opher:1999jh, Yao:2016cjs} carried out in the linear response approximation applicable in the regime where the modification of the Coulomb potential is a small effect. We extend the 'dynamic screening' effort (and in particular that of Hwang et al~\cite{Hwang:2021kno} which introduced dynamic screening during BBN) to study `damped-dynamic' screening as follows:
% \begin{enumerate}
% \item 
% We detail the scattering damping effect of plasma components in the temperature range beginning near $T\simeq m_ec^2=511$\,keV extending down to a temperature near $T_\mathrm{split}\simeq 20.3$\,keV where practically all $e^-e^+$-pairs disappeared.
% \item 
% We obtain analytical results for damped-dynamic screening applicable to the BBN epoch temperature range which demonstrates a connection between the BBN plasma and dusty plasma theory. 
% \item 
% We recognize and describe the limits of applicability of the linear response method to BBN by considering the full equilibrium distribution necessary in the presence of strong fields. We explore a one-dimensional toy model at distances a fraction of an \AA ngstrom in the quantum tunneling regime where the potential $\phi$ is large compared to the thermal collision energy $|\phi|/3T> 1$,  which predicts enhanced screening at small distances.
% \end{enumerate}

% Such effort to improve the BBN model through screening is well justified for the following reasons:
% \begin{enumerate}
% \item Currently, the most accessible path to observe the early Universe before recombination is through light element abundances. 
% \item
% Despite the great initial success of the BBN model regarding the prediction of the abundances of D, $^3$ He, $^4$ He, there remain well-documented discrepancies. Currently, substantial efforts are being directed toward reconciling the theoretical BBN model with present-day observations. For a more detailed discussion, see Refs.\,\cite{Pitrou:2021vqr,Bertulani:2022qly}.
% \item
% Enhanced production of light elements during the BBN era could help resolve two problems: a) Elements Beryllium and Boron cannot be produced in stellar burning processes while their high abundance suggests additional mechanisms of production beyond secondary processes after stellar explosions. b) Understanding the age of stars obtained using the abundance of light elements (metallicity) may be improved by adaptation of the BBN network to account for nuclear reactions in ultra-dense QED plasma.
% \item
% We note that screening effects are more pronounced for nuclei with greater charge $Z$ leading to stronger modification in predicted abundances, just where the most significant current disagreements exist ({\it e.g.\/} the abundance of $^7$Li). Inclusion of screening effects in our current models of the BBN reaction network could explain this discrepancy without the need to invoke more exotic modifications of the Coulomb internuclear potential. An example of such a modification would be the variation of $\alpha$, the fine-structure constant~\cite{Meissner:2023voo}.
% \item 
% Recent James Webb space telescope-Hubble space telescope (JWST-HST) observations implying early onset of galactic structure formation and/or presence of luminous objects as early as 300 million years after BBN~\cite{Haro:2023JWST,Sabti:2023xwo,Ilie:2023DM} could be seen as evidence supporting nuclear dust matter clustering in dense QED plasma.
% \end{enumerate}


% In the standard BBN model, thermonuclear reaction rates are evaluated using nuclear reactions specific to the vacuum state. In a high-precision BBN model, the effect of $e^-e^+$-pair-plasma screening of nuclear charges must be allowed. We describe in Section~\ref{sec:density} the presence of up to several millions of electron-positron ($e^-e^+$) pairs per every charged nucleon. This situation prompts an effort to reconsider BBN in an ultra dense plasma environment allowing for a dense $e^-e^+\gamma$ plasma medium.

% We account for collisions between plasma components using the current-conserving Bhatnagar, Gross, and Krook (BGK) collision term~\cite{Bhatnagar:1954zz}, which allows us to study damping in the dynamic plasma. For an in-depth discussion presented in contemporary covariant notation and preserving current conservation, see Formanek et al.~\cite{Formanek:2021blc}. Our approach is different from Monte-Carlo simulation of two particle collisions~\cite{Sasankan:2019oee,Kedia:2020xdc} as in our case particles participating in scattering also experience simplified static screening which in terms of collision description would require more than two particle collisions. In addition, the simplified BGK collision term we employ allows us to obtain an analytic `damped-dynamic' internuclear potential relevant to the BBN reaction network in the linear response approximation. 

% Several well known processes, including M\o ller, Bhabha, and inverse Compton scattering, characterize the damping in plasma. These textbook results will be adapted to the plasma environment: When studied in vacuum, electrons and positrons interact with each other by exchanging a massless photon. However, when a photon propagates through a plasma of electrons and positrons, its properties are modified by interactions with the medium. When evaluating Moller and Bhabha scattering, we include the temperature-dependent mass of the photon obtained in plasma theory without damping. We find that the total relaxation rate during BBN is much larger than the screening mass, which is the inverse of the Debye screening length $m_D = 1/\lambda_D$. This suggests that electromagnetic perturbations of the plasma will be overdamped.

% Since, as noted above, the computed damping strength is the dominant scale, it is also the main parameter determining the photon mass. However, introducing the damping strength in the photon mass would make the calculation of the damping strength self-referential. The complexity of such a self-consistent evaluation of damping is beyond the scope of this work. This underscores the need to develop a self-consistent approach where both damping and photon properties in plasma are determined in a mutually consistent manner.

% The QED $e^+e^-\gamma$-plasma present during BBN is perturbed by a variety of comparatively very heavy charged protons $p$ and even heavier isotopes and light elements. In calculating the electromagnetic potential $\phi$, we naturally view BBN active participants as heavier, higher charged impurities {\it i.e.\/} \lq\lq charged dust\rq\rq in the QED-plasma. Since theories of such a system have been developed by others working on charged dust grains in planetary and space plasma~\cite{Montgomery:1970jpp, Stenflo:1973, Shukla:2002ppcf, Lampe:2000pop} we can validate our results and adapt previously developed theoretical tools necessary for describing heavy charged impurities within a plasma of lighter particles.

% We provide the first step in merging these two fields by including damping to analyze the electromagnetic potential during BBN, similar to work done by Stenflo in 1973 for dusty plasma~\cite{Stenflo:1973}. We advance this work by applying it to the BBN epoch, allowing for the finite size of the "dust grains," and by finding the short-distance behavior of the potential in the weak field limit. Using a general model of simplified collisional damping in linear response, we calculate the screened electromagnetic potential of light nuclei undergoing thermal motion subject to damping relying on previous work ~\cite{Formanek:2021blc,Grayson:2022asf}.

%  
% In Section~\ref{sec:relax}, we calculate the relaxation rate $\kappa= 1/\tau$, where $\tau$ is the mean time between collisions in the plasma. This is done by finding the average of the most relevant reaction rates for $2\leftrightarrow 2$ scattering: M{\o}ller, Bhabha, and inverse Compton scattering. 

% We proceed in Section~\ref{sec:kinetic_theory} to solve the Vlasov-Boltzmann Equation (VBE) in the weak field limit with damping. Our approach considers small plasma perturbations in linear response to find the polarization tensor present during the BBN era. In Section~\ref{sec:potential}, we find an analytic expression for the damped-dynamic potential applicable to BBN and study the limitation of the linear response method. We show that additional strong field phenomena beyond linear response alter the behavior of the screened potential $\phi$. We discuss our results and describe their interdisciplinary importance in Section~\ref{sec:Discussion}. 
 
%%%%%%%%%%%%%%%%%%%%%%%%
\include{04-birrell/NeutrinoDistributionToday/NuDistToday.tex}
% Appendix
\appendix
\section{Geometry Background: Volume Forms on Submanifolds}\label{ch:vol_forms}
In this appendix we develop the geometric machinery which will be used to derive computationally efficient formulas for the scattering integrals. This facilitates the study  of the neutrino freeze-out using the Boltzmann-Einstein equation in Section \ref{ch:param_studies}.  This appendix is much more mathematical than the main text and, when standard, we use geometrical language and notation here without further explanation;  see, e.g., \cite{lee2003introduction,lee1997riemannian,o1983semi}.  We found this formalism to be useful for our development of an improved method for computing scattering integrals, as presented in Appendix \ref{ch:coll_simp}.  However, if one is content with simply using the results then this appendix is non-essential. See also   \cite{Birrell:2014uka}.



\subsection{Inducing Volume Forms on Submanifolds}

Given a Riemannian manifold $(M,g)$ with volume form $dV_g$ and a  hypersurface $S$, the standard Riemannian hypersurface area form, $dA_g$, is defined on $S$ as the volume form of the pullback metric tensor on $S$.   Given vectors $v_1,...,v_k$ we define the interior product (i.e. contraction)  operator acting on a form $\omega$ of degree $n\geq k$ as the $n-k$ form 
\begin{equation}
i_{(v_1,...,v_k)}\omega=\omega(v_1,...,v_k,\cdot)\,.
\end{equation}
 With this notation, the hypersurface area form can equivalently be computed as
\begin{equation}
dA_g=i_v dV_g\,,
\end{equation}
where $v$ is a unit normal vector to $S$.  This method extends to submanifolds of codimension greater than one as well as to semi-Riemannian manifolds, as long as the metric restricted to the submanifold is non-degenerate. 

However, there are many situations where one would like to define a natural volume form on a submanifold that is induced by a volume form in the ambient space, but where the above method is inapplicable, such as defining a natural volume form on the light cone or other more complicated degenerate submanifolds in general relativity. In this section, we will describe a method for inducing volume forms on regular level sets of a function that is applicable in cases where there is no metric structure and show its relation to more widely used semi-Riemannian case.  We prove analogues of the coarea formula and Fubini's theorem in this setting. 

Let $M$, $N$ be smooth manifolds, $c$ be a regular value of a smooth function $F:M\rightarrow N$, and $\Omega^M$ and $\Omega^N$ be volume forms on $M$ and $N$ respectively.  Using this data, we will be able to induce a natural volume form on the level set $F^{-1}(c)$.  The absence of a metric on $M$ is made up for by the additional information that the function $F$ and volume form $\Omega^N$ on $N$ provide. The following theorem makes our definition precises and proves the existence and uniqueness of the induced volume form.

\begin{theorem}\label{induced_vol_form}
Let $M$, $N$ be $m$ (resp. $n$)-dimensional smooth manifolds with volume forms $\Omega^M$ (resp. $\Omega^N$). Let $F:M\rightarrow N$ be smooth and $c$ be a regular value.  Then there is a unique volume form $\omega$  (also denoted $\omega^M$) on $F^{-1}(c)$ such that $\omega_x=i_{(v_1,...,v_n)}\Omega^M_x$ whenever $v_i\in T_xM$ are such that 
\begin{equation}\label{unit_volume}
\Omega^N(F_*v_1,...,F_* v_n)=1\,.
\end{equation}
We call $\omega$ the {\bf volume form induced by $F:(M,\Omega^M)\rightarrow (N,\Omega^N)$}.
\end{theorem}
\begin{proof}
$F_*$ is onto $T_{F(x)}N$ for any $x\in F^{-1}(c)$.  Hence there exists $\{v_i\}_1^n\subset T_xM$ such that $\Omega^N(F_*v_1,...,F_* v_n)=1$.  In particular, $F_* v_i$ is a basis for $T_{F(x)} N$.  Define $\omega_x=i_{(v_1,...,v_n)}\Omega_x$. This is obviously a nonzero $m-n$ form on $T_xF^{-1}(c)$ for each $x\in F^{-1}(c)$.  We must show that this definition is independent of the choice of $v_i$ and the result is smooth.

 Suppose $F_*v_i$ and $F_*w_i$ both satisfy \req{unit_volume}.  Then $F_*v_i=A_i^jF_*w_j$ for $A\in SL(n)$. Therefore $v_i-A_i^jw_j\in \ker F_{*x}$.  This implies
\begin{equation}
i_{(v_1,...,v_n)}\Omega^M_x=\Omega^M_x(A_1^{j_1}w_{j_1},...,A_n^{j_n}w_{j_n},\cdot)
\end{equation}
since the terms involving $\ker F_*$ will vanish on $T_x F^{-1}(c)=\ker F_{*x}$.  Therefore
\begin{align}\label{ind_of_v_proof}
i_{(v_1,...,v_n)}\Omega^M_x&=A_1^{j_1}...A_n^{j_n}\Omega^M_x(w_{j_1},...,w_{j_n},\cdot)\\
&=\sum_{\sigma\in S_n} \pi(\sigma)A_1^{\sigma(1)}...A_n^{\sigma(n)}\Omega^M_x(w_1,...,w_n,\cdot)\notag\\
&=\det(A)i_{(w_1,...,w_n)}\Omega^M_x\notag\\
&=i_{(w_1,...,w_n)}\Omega^M_x\,.\notag
\end{align}
This proves that $\omega$ is independent of the choice of $v_i$.  If we can show $\omega$ is smooth then we are done.  We will do better than this by proving that for any  $v_i\in T_xM$ the following holds
\begin{equation}
i_{(v_1,...,v_n)}\Omega^M_x=\Omega^N(F_*v_1,...,F_*v_n)\omega_x\,.
\end{equation}
To see this, take $w_i$ satisfying \req{unit_volume}.  Then $F_*v_i=A_i^j F_*w_j$. This determinant can be computed from
\begin{align}
\Omega^N(F_*v_1,...,F_*v_n)=\det(A)\Omega^N(F_*w_1,...,F_*w_n)=\det(A)\,.
\end{align}
 Therefore, the same computation as \req{ind_of_v_proof} gives
\begin{align}
i_{(v_1,...,v_n)}\Omega^M_x=\det(A)\omega_x=\Omega^N(F_*v_1,...,F_*v_n)\omega_x
\end{align}
as desired.  To prove that $\omega$ is smooth, take a smooth basis of vector fields $\{V_i\}_1^m$ in a neighborhood of $x$.  After relabeling, we can assume $\{F_*V_i\}_1^n$ are linearly independent at $F(x)$ and hence, by continuity, they are linearly independent at $F(y)$ for all $y$ in some neighborhood of $x$.  In that neighborhood, $\Omega^N(F_*V_1,...,F_*V_n)$ is non-vanishing and therefore
\begin{equation}
\omega=(\Omega^N(F_*V_1,...,F_*V_n))^{-1}i_{(V_1,...,V_n)}\Omega
\end{equation} 
which is smooth.
\end{proof}

\begin{corollary}\label{induced_vol_eq}
For any  $v_i\in T_xM$ the following holds
\begin{equation}\label{vol_formula1}
i_{(v_1,...,v_n)}\Omega^M_x=\Omega^N(F_*v_1,...,F_*v_n)\omega_x\,.
\end{equation}
\end{corollary}



\begin{corollary}
If $\phi:M\rightarrow\mathbb{R}$ is smooth and $c$ is a regular value then by equipping $\mathbb{R}$ with its canonical volume form we have 
\begin{equation}
\omega_x=i_v\Omega^M_x\,,
\end{equation}
where $v\in T_xM$ is any vector satisfying $d\phi(v)=1$.
\end{corollary}

It is useful to translate  \req{vol_formula1} into a form that is more readily applicable to computations in coordinates.  Choose arbitrary coordinates $y^i$ on $N$ and write $\Omega^N=h^N(y) dy^n$. Choose coordinates $x^i$ on $M$ such that $F^{-1}(c)$ is the coordinate slice
\begin{equation}
F^{-1}(c)=\{x:x^1=...=x^n=0\}
\end{equation}
and write $\Omega^M=h^M(x)dx^m$. The coordinate vector fields $\partial_{x^i}$ are transverse to $F^{-1}(c)$ and so
\begin{equation}
\Omega^N(F_*\partial_{x^1},...,F_*\partial_{x^n})=h^N(F(x))\det \left(\frac{\partial F^i}{\partial x^j}\right)_{i,j=1..n}
\end{equation}
and
\begin{equation}
i_{(\partial_{x^1},...,\partial_{x^n})}\Omega^M=h^M(x) dx^{n+1}...dx^m\,.
\end{equation}
Therefore we obtain
\begin{equation}\label{vol_form_coords}
\omega_x=\frac{h^M(x)}{h^N(F(x))}\det \left(\frac{\partial F^i}{\partial x^j}\right)^{-1}_{i,j=1..n}dx^{n+1}...dx^m\,.
\end{equation}

Just like in the (semi)-Riemannian case, the induced measure allows us to prove a coarea formula where we break integrals over $M$ into slices. In this theorem and the remainder of the section, we consider integration with respect to the density defined by any given volume form, i.e., we ignore the question of defining consistent orientations.
\begin{theorem}[Coarea formula]\label{vol_form_coarea}
Let $M$ be a smooth manifold with volume form $\Omega^M$, $N$ a smooth manifold with volume form $\Omega^N$ and $F:M\rightarrow N$ be a smooth map.  If $F_*$ is surjective at a.e. $x\in M$ then for $f\in L^1(\Omega^M)\bigcup L^+(M)$ we have
\begin{equation}\label{coarea_formula}
\int_Mf(x) \Omega^M(dx)=\int_{N}\int_{F^{-1}(z)} f(y)\omega^M_z(dy) \Omega^N(dz)\,,
\end{equation}
where $\omega^M_z$ is the volume form induced on $F^{-1}(z)$ as in Lemma \ref{induced_vol_form}.
\end{theorem}
\begin{proof}
First suppose $F$ is a submersion. By the rank theorem there exists a countable collection of charts $(U_i,\Phi_i)$ that cover $M$ and corresponding charts $(V_i,\Psi_i)$ on $N$ such that 
\begin{align}
\Psi_i\circ F\circ \Phi_i^{-1}(y^1,...,y^{m-n},z^1,...,z^n)=(z^1,...,z^n)\,.
\end{align}
Let $\sigma_i$ be a partition of unity subordinate to $U_i$.  For each $i$ and $z$ we have $\Phi_i(U_i\cap F^{-1}(z))=\left(\mathbb{R}^{m-n}\times\{\Psi_i(z)\}\right)\cap \Phi_i(U_i)$.  We can assume that the $\Phi_i(U_i)=U_i^1\times U_i^2\subset \mathbb{R}^{m-n}\times \mathbb{R}^n$ and therefore each $\Phi_i$ is a slice chart for $F^{-1}(z)$ for all $y$ such that $F^{-1}(z)\cap U_i\neq \emptyset$.  In other words, $\Phi_i(U_i\cap F^{-1}(z))= U_i^1\times \{\Psi(z)\}$.  This lets us compute the left and right hand sides of \req{coarea_formula} for $f\in L^+(M)$:
\begin{align}
\int_Mf(x) \Omega^M(dx)&=\sum_i\int_{U_i}(\sigma_if)(x) \Omega^M(dx)\\
&=\sum_i\int_{\Phi_i(U_i)}(\sigma_if)\circ \Phi^{-1}(y,z) \Phi^{-1*}\Omega^M(dy,dz)\notag\\
&=\sum_i\int_{\Phi_i(U_i)}(\sigma_if)\circ \Phi^{-1}(y,z)|g^M(y,z)| dy^{m-n}dz^n\notag\\
&=\sum_i\int_{U_i^2}\left[\int_{U_i^1}(\sigma_if)\circ \Phi^{-1}(y,z)|g^M(y,z)| dy^{m-n}\right]dz^n\notag\\
&\text{where }\Omega^M=g^M dy^1\wedge...\wedge dy^{m-n}\wedge dz^1\wedge...\wedge dz^n\,,\notag
\end{align}
and
\begin{align}
&\int_{N}\int_{F^{-1}(z)} f(y)\omega^M_z(dy) \Omega^N(dz)\\
=&\sum_i \int_{N}\left[\int_{\Phi_i(U_i\cap F^{-1}(z))} (\sigma_if)\circ\Phi_i^{-1}(y,\Psi(z))\Phi_i^{-1*}\omega^M_z(dy)\right] \Omega^N(dz)\notag\\
=&\sum_i \int_{V_i}\left[\int_{\Phi_i(U_i\cap F^{-1}(z))} (\sigma_if)\circ\Phi_i^{-1}(y,\Psi(z))\Phi_i^{-1*}\omega^M_z(dy)\right] \Omega^N(dz)\notag\\
=&\sum_i \int_{\Psi_i(V_i)}\left[\int_{\Phi_i(U_i\cap F^{-1}(\Psi^{-1}(z))} (\sigma_if)\circ\Phi_i^{-1}(y,z)\Phi_i^{-1*}\omega^M_z(dy)\right] \Psi^{-1*}\Omega^N(dz)\notag\\
=&\sum_i \int_{U_i^2}\left[\int_{U_i^1\times \{z\}} (\sigma_if)\circ\Phi_i^{-1}(y,z)|g^M_z(y)| dy^{m-n}\right] |g^N(z)| dz^n\,,\notag\\
&\text{where }\omega^M_z=g^M_z dy^1\wedge...\wedge dy^{m-n} \text{ and }\Omega^N=g^N dz^1\wedge...\wedge dz^n \text{ for } g_1^M,g_N>0\,.\notag
\end{align}
\begin{comment}
Here we used that if \pi_1\circ \Phi is the chart on F^{-1}(z) then the inverse is y->\Phi^{-1}(y,Psi(z)) since \pi_2\circ\Phi(x)=\Psi(z)
for any x\in F^{-1}(z)
\end{comment}
Therefore, if we can show $|g^M(y,z)|=|g_z^M(y)g^N(z)|$ on $U_i^1\times U_i^2$ we are done. From Corollary \ref{induced_vol_eq} we have
\begin{align}
&(-1)^{n(m-n)} g^M(y,z)\\
=&\Omega^M(\partial_{z^1},...,\partial_{z^n},\partial_{y^1},...,\partial_{y^{m-n}})=\Omega^N(F_*\partial_{z^n},...,F_*\partial_{z^n})g_z^M(y)\,.\notag
\end{align}
Since $\Psi\circ F\circ\Phi^{-1}=\pi_2$ we have $F_*\partial_{z^j}=\partial_{z_j}$ and so $\Omega^N(F_*\partial_{z^n},...,F_*\partial_{z^n})=g^N$ which completes the proof in the case where $F$ is a submersion.  The generalization to the case where $F_*$ is surjective a.e. follows from Sard's theorem and the fact that the set of $x\in M$ at which $F_*$ is surjective is open.

\begin{comment}
The set K of $x\in M$ where $F_*$ is surjective is open in M by the rank theorem and by assumption, its complement has M-measure zero.  Therefore the lhs equal int_K f dOmega^M. F is a submersion on $K$ so   int_K f dOmega^M=int_N int_F|K^{-1}(z) f omega^M_z Omega^N.  Split this into an integral over regular values of $F:M->N$ and crit values.  The int over crit values is zero by Sard.  For z a regular value, F|_K^{-1}(z)=F^{-1}(z) Hence the lhs is int_N_ref int_F^{-1}(z) f omega^M_z Omega^N. 

  Split the rhs into integral over critical  values of F:M->N and regular values.  The integral over crit values is
 zero by sard, hence the rhs is
int_N_reg int F^{-1}(z).  Therefore they are equal.  QED
\end{comment}
\end{proof}
\subsection{Comparison to Riemannian Coarea Formula}
We now recall the classical coarea formula for semi-Riemannian metrics, see, e.g., \cite{chavel1995riemannian},  and give its relation to theorem \ref{vol_form_coarea}.
\begin{definition}
Let $F:(M,g)\rightarrow (N,h)$ be a smooth map between semi-Riemannian manifolds.  The {\bf normal Jacobian} of $F$ is
\begin{equation}
NJF(x)=|\det(F_*|_x(F_*|_x)^T)|^{1/2}\,,
\end{equation}
where $(F_*|_x)^T$ denotes the adjoint map $T_xN\rightarrow T_xM$ obtained pointwise from the pullback $T^*N\rightarrow T^*M$ combined with the tangent-cotangent bundle isomorphisms defined by the metrics.
\end{definition}

\begin{lemma}
The normal Jacobian has the following properties.
\begin{itemize}
\item $(F_*|_x)^T:T_{F(x)}N\rightarrow (\ker F_*|_x)^\perp$.
\item If $F_*|_x$ is surjective then $(F_*|_x)^T$ is 1-1.
\item In coordinates
\begin{equation}
NJF(x)=\left|\det\left(h_{ik}(F(x))\frac{\partial F^k}{\partial x^l}(x)g^{lm}(x)\frac{\partial F^j}{\partial x^m}(x)\right)\right|^{1/2}\,.
\end{equation}
\item  If $F_*|_x$ is surjective and $g$ is nondegenerate on $ker F_*|_x$ then $F_*|_x(F_*|_x)^T$ is invertible.
\item If $c\in N$ is a regular value of $F$ and $g$ is nondegenerate on $F^{-1}(c)$ then $NJF(x)$ is non-vanishing and smooth on $F^{-1}(c)$.
\end{itemize}
\end{lemma}

Combining these lemmas with the rank theorem, one can prove the standard semi-Riemannian coarea formula
\begin{theorem}[Coarea formula]
Let $F:(M,g)\rightarrow (N,h)$ be a smooth map between semi-Riemannian manifolds such that $F_*$ is surjective at a.e. $x\in M$ and $g$ is nondegenerate on $F^{-1}(c)$ for a.e $c\in N$.  Then for $\phi\in L^1(dV_g)$ we have
\begin{equation}
\int_M\phi(x)dV_g=\int_{y\in N}\int_{x\in F^{-1}(y)}\frac{\phi(x)}{NJF(x)}dA_g dV_h\,,
\end{equation}
where $dA_g$ is the volume measure induced on $F^{-1}(y)$ by pulling back the metric $g$.  In particular, if $N=\mathbb{R}$ with its canonical metric then $NJF=|\nabla F|$ and 
\begin{equation}
\int_M \phi dV_g=\int_\mathbb{R}\int_{F^{-1}(r)}\frac{\phi(x)}{|\nabla F(x)|} dA_g dr\,.
\end{equation}
\end{theorem}

The relation between the Riemannian coarea theorem and Theorem \ref{vol_form_coarea} follows from the following theorem.
\begin{theorem}
Let $F:(M,g)\rightarrow (N,h)$  be a smooth map between semi-Riemannian manifolds and $c$ be a regular value.  Suppose $g$ is nondegenerate on $F^{-1}(c)$.  Let $\omega$ be the volume form on $F^{-1}(c)$ induced by $F:(M,dV_g)\rightarrow (N,dV_h)$.  Then
\begin{equation}
\omega=NJF^{-1}dA_g
\end{equation}
as densities.
\end{theorem}
\begin{proof}
By Corollary \ref{induced_vol_eq}, for any  $v_i\in T_xM$ we have
\begin{equation}
i_{(v_1,...,v_n)}\Omega^M_x=dV_h(F_*v_1,...,F_*v_n)\omega_x\,.
\end{equation}
If we let $v_i$ be an orthonormal basis of vectors orthogonal to $F^{-1}(c)$ at $x$ then $F_*v_i$ are linearly independent and so
\begin{align}
\omega=&(dV_h(F_*v_1,...,F_*v_n))^{-1}i_{(v_1,...,v_n)}dV_g\\
=&(dV_h(F_*v_1,...,F_*v_n))^{-1}dA_g\,.\notag
\end{align}
Choose coordinates about $x$ and $F(x)$ so that $\partial_{x^i}=v_i$ for $i=1...n$, $\{\partial_{x^i}\}_{n+1}^m$ span $\ker F_*$, and $\partial_{y_i}$ are orthonormal.  Then 
\begin{align}
dV_h(F_*v_1,...,F_*v_n)&=\sqrt{|\det(h)|}\frac{\partial F^{j_1}}{\partial x^1}...\frac{\partial F^{j_n}}{\partial x^n}dy^1\wedge...\wedge dy^n(\partial_{y^{j_1}},...,\partial_{y^{j_n}})\\
&=\det\left(\frac{\partial F^{j}}{\partial x^i}\right)_{i,j=1}^n\,.\notag
\end{align}
$F_*\partial_{x^i}=0$ for $i=n+1...m$ and so $\frac{\partial F^j}{\partial x_i}=0$ for $i=n+1...m$.  Letting $\eta=\diag(\pm 1)$ be the signature of $g$, we find
\begin{align}
NJF(x)=&\left|\det\left(h_{ik}(F(x))\frac{\partial F^k}{\partial x^l}(x)g^{lm}(x)\frac{\partial F^j}{\partial x^m}(x)\right)\right|^{1/2}\\
=&\left|\det\left(\sum_{l,m=1}^n\frac{\partial F^k}{\partial x^l}(x)\eta^{lm}(x)\frac{\partial F^j}{\partial x^m}(x)\right)\right|^{1/2}\notag\\
=&\left|\det\left(\frac{\partial F^k}{\partial x^l}\right)_{k,l=1}^n\det(\eta^{lm})_{l,m=1}^n\det\left(\frac{\partial F^j}{\partial x^m}\right)_{j,m=1}^n\right|^{1/2}\notag\\
=&\left|\det\left(\frac{\partial F^k}{\partial x^l}\right)_{k,l=1}^n\right|\notag\\
=&|dV_h(F_*v_1,...,F_*v_n)|\,.\notag
\end{align}
Therefore 
\begin{equation}
\omega=NJF^{-1}dA_g
\end{equation}
as densities.
\end{proof}
In particular, this shows that even though $NJF$ and $dA_g$ are undefined individually when $g$ is degenerate on $F^{-1}(c)$, one can make sense of their ratio in this situation as the induced volume form $\omega$.


\subsection{Delta Function Supported on a Level Set}
 The induced measure defined above allows for a coordinate independent definition of a delta function supported on a regular level set.  Such an object is of great use in performing calculations in relativistic phase space.  We give the definition and prove several properties that justify several common formal manipulations that one would like to make with such an object.
\begin{definition}
Motivated by the coarea formula, we define the composition of the {\bf Dirac delta function} supported on $c\in N$ with a smooth map $F:M\rightarrow N$ such that $c$ is a regular value of $F$ by
\begin{equation}\label{delta_def}
 \delta_c(F(x))\Omega^M \equiv \omega^M
\end{equation}
on $F^{-1}(c)$.  This is just convenient shorthand, but it commonly used in the physics literature (typically without the justification presented above or in the following results).   For $f\in L^1(\omega^M)$ we will write 
\begin{equation}
\int_M f(x)\delta_c(F(x))\Omega^M(dx)
\end{equation} 
in place of 
\begin{equation}
\int_{F^{-1}(c)} f(x) \omega^M(dx)\,.
\end{equation}



More generally, if the subset of $F^{-1}(c)$ consisting of critical points, a closed set whose complement we call $U$, has $\dim M-\dim N$ dimensional Hausdorff measure zero in $M$ then we define
\begin{equation}
\int_M f(x)\delta_c(F(x))\Omega^M(dx)=\int_{F|_U^{-1}(c)} f(x)\omega^M\,.
\end{equation}
This holds, for example, if $U^c$ is contained in a submanifold of dimension less than  $\dim M-\dim N$.  

Equivalently, we can replace $U$ in this definition with any {\bf open} subset of $U$ whose complement still has $\dim M-\dim N$ dimensional Hausdorff measure zero. In this situation, we will say $c$ is a regular value except for a lower dimensional exceptional set.  Note that while Hausdorff measure depends on a choice of Riemannian metric on $M$, the measure zero subsets are the same for each choice.
\begin{comment}
Hausdorff measures commutes with restriction to subspaces (w/subspace metric)
Hausdorff measure on a submanifold induced by pullback metric tensor equals that induced by restriction of distance metric from ambient space.
Hausdorff measure of subsets of a manifold being zero is well defined (i.e. ind of choice of metric) so we can define k-dimensional measure zero for any k (integer or not). This follows since vol forms are abs cont wrt hausdorff measure of that dim
\end{comment}
\end{definition}

Using \req{vol_form_coords}, along with the coordinates described there, we can (at least locally) write the integral with respect to the delta function in the more readily usable form
\begin{equation}\label{delta_integral_coords}
\int_M f(x)\delta_c(F(x))\Omega^M=\int_{F^{-1}(c)} f(x)\frac{h^M(x)}{h^N(F(x))}\bigg|\det \left(\frac{\partial F^i}{\partial x^j}\right)^{-1}\bigg|dx^{n+1}...dx^m\,.
\end{equation}
The absolute value comes from the fact that we use $\delta_c(F(x))\Omega^M$ to define the orientation on $F^{-1}(c)$.


As expected, such an operation behaves well under diffeomorphisms.
\begin{lemma}\label{diffeo_property}
Let $c$ be a regular value of $F:M\rightarrow N$ and $\Phi:M^{'}\rightarrow M$ be a diffeomorphism.  Then the delta functions induced by $F:(M,\Omega^M)\rightarrow (N,\Omega^N)$ and $F\circ\Phi:(M^{'},\Phi^*\Omega^M)\rightarrow  (N,\Omega^N)$ satisfy
\begin{equation}
\delta_c(F\circ\Phi)(\Phi^*\Omega^M)=\Phi^*(\delta_c(F)\Omega^M)\,.
\end{equation}
\end{lemma}

\begin{lemma}
Let $c$ be a regular value of $F:(M,\Omega^M)\rightarrow (N,\Omega^N)$ and $\Phi:N\rightarrow (N^{'},\Omega^{N^{'}})$ be a diffeomorphism where $\Phi^*\Omega^{N^{'}}=\Omega^N$.  Then the delta functions induced by $F:(M,\Omega^M)\rightarrow (N,\Omega^N)$ and $\Phi\circ F:(M,\Omega^M)\rightarrow (N^{'},\Omega^{N^{'}})$ satisfy
\begin{equation}
\delta_c(F)\Omega^M=\delta_{\Phi(c)}(\Phi\circ F)\Omega^M\,.
\end{equation}
\end{lemma}

We also have a version of Fubini's theorem.

\begin{theorem}[Fubini's Theorem for Delta functions]
Let $M_1,M_2,N$ be smooth manifolds with volume forms $\Omega_1,\Omega_2, \Omega^N$. Let $M\equiv M_1\times M_2$ and $\Omega\equiv \Omega_1\wedge\Omega_2$. Suppose that the set of $(x,y)\in F^{-1}(c)$ such that $F|_{M_1\times\{y\}}$ is not regular at $x$ has $\dim M_1+\dim M_2-\dim N$ dimensional Hausdorff measure zero in $M_1\times M_2$ (we denote the complement of this closed set by $U$).  Then for $f\in L^1(\omega)\bigcup L^+(F^{-1}(c))$ we have
\begin{equation}\label{Fubini_eq}
\int_Mf(x,y)\delta_c(F(x,y)) \Omega(dx,dy)=\int_{M_2}\left[\int_{U^y} f(x,y) \delta_c(F(x,y))\Omega_1(dx) \right]\Omega_2(dy)\,,
\end{equation}
where $U^y=\{x\in M_1:(x,y)\in U\}$.
\end{theorem}
\begin{proof}
Our assumption about $F|_{M_1\times\{y\}}$ implies that $c$ is a regular value of $F:M_1\times M_2\rightarrow N$ except for the lower dimensional exceptional set $U^c$
\begin{comment}
U is a subset of the set given in the condition.  Both are closed
\end{comment}
and for $y\in M_2$, $c$ is also a regular value of $F|_{U^y\times\{y\}}$, hence both sides of \req{Fubini_eq} are well defined.  Rewriting \req{Fubini_eq} without the delta function, we then need to show that 
\begin{equation}
\int_{F|_U^{-1}(c)} f(x,y) d\omega=\int_{M_2}\left[\int_{F|_{U^y\times\{y\}}^{-1}(c)} f(x,y) \omega^1_{c,y}(dx)\right]\Omega_2(dy)\,,
\end{equation}
where $\omega^1_{c,y}$ is the induced volume form on $F|_{U^y\times\{y\}}^{-1}(c)$.  

Consider the projection map restricted to the $c$-level set, $\pi_2:F|_U^{-1}(c)\rightarrow M_2$.  By assumption, $F|_{M_1\times\{y\}}$ is regular at $x$ for all $(x,y)\in F|_U^{-1}(c)$. For such an $(x,y)$, take a basis $w_i\in T_yM_2$. Since $F|_{M_1\times\{y\}}$ has full rank at $x$, for each $i$ there exists $v_i\in T_xM_1$ such that $F(\cdot,y)_*v_i=F_*(0,w_i)$.  Therefore $(-v_i,w_i)\in \ker F_*|_{(x,y)}=T_{(x,y)}F|_U^{-1}(c)$.  Hence $w_i\in\pi_{2*} T_{(x,y)}F^{-1}(c)$ and so $\pi_2:F|_U^{-1}(c)\rightarrow M_2$ is regular at $(x,y)$.  

Since $\pi_2$ is regular for all $(x,y)\in F|_U^{-1}(c)$ the coarea formula applies, giving
\begin{align}
\int_{F|_U^{-1}(c)}f d\omega=&\int_{M_2}\left[\int_{\pi_2^{-1}(y)}f\tilde{\omega}_{c,y}^1\right]\Omega_2(dy)
\end{align}
for all $f\in L^1(\omega)\bigcup L^+(F^{-1}(c))$, where $\tilde{\omega}_{c,y}^1$ is the volume form on $\pi_2^{-1}(y)$ induced by $\pi_2:(F|_U^{-1}(c),\omega)\rightarrow (M_2,\Omega_2)$.

As a point set, $\pi_2^{-1}(y)=F|_{ U^y\times\{y\}}^{-1}(c)$ and both are embedded submanifolds of $M_1\times M_2$ for a.e. $y\in M_2$, hence are equal as manifolds.  So if we can show $\tilde{\omega}_{c,y}^1=\omega^1_{c,y}$ as densities whenever $F|_{M_1\times\{y\}}$ is regular at $x$ for some $(x,y)$ then we are done.  

Given any such $(x,y)$, take $v_i\in T_xM_1$ such that $\Omega^N(F(\cdot,y)_*v_i)=1$.  By definition, $\omega_{c,y}^1=i_{(v_1,...,v_n)}\Omega_1$.  We also have $(v_i,0)\in T_{(x,y)}M_1\times M_2$ and $\Omega^N(F_*(v_i,0))=1$.  Hence 
\begin{align}
\omega=&i_{((v_1,0),...,(v_n,0))}(\Omega_1\wedge\Omega_2)
=(i_{((v_1,0),...,(v_n,0))}\Omega_1)\wedge\Omega_2\,.
\end{align}
Let $w_i\in T_y M_2$ such that $\Omega_2(w_1,...,w_{m_2})=1$.  By the same argument as above, there exists $\tilde{v}_i\in T_xM_1$ such that $(\tilde{v}_i,w_i)\in \ker F_*=T_{(x,y)}F^{-1}(c)$.  $\pi_{2*}(\tilde{v}_i,w_i)=w_i$ and $\Omega_2(w_1,...,w_{m_2})=1$ so by definition,
\begin{equation}
\tilde{\omega}_{c,y}^1=i_{((\tilde{v}_1,w_1),...,(\tilde{v}_{m_2},w_{m_2}))}\omega\,.
\end{equation}
Since any term containing $\Omega_2$ will vanishes on $T_F(\cdot,y)^{-1}(c)\subset T M_1$, we have  
\begin{align}
\tilde{\omega}_{c,y}^1=&(-1)^{m_1-n}i_{((v_1,0),...,(v_n,0))}\Omega_1\\
=&(-1)^{m_1-n}\omega_{c,y}^1\wedge\left(i_{((\tilde{v}_1,w_1),...,(\tilde{v}_{m_2},w_{m_2}))}\Omega_2\right)\notag\\
=&(-1)^{m_1-n}\omega_{c,y}^1\,.\notag
\end{align}
As we are integrating with respect to the densities defined by $\omega_{c,y}^1$ and $\tilde{\omega}_{c,y}^1$ we are done.  
\end{proof}

Before moving on, we give a few more useful identities.

\begin{theorem}\label{delta_associative}
Let $(c_1,c_2)$ be a regular value of $F\equiv F_1\times F_2:(M,\Omega^M)\rightarrow (N_1\times N_2,\Omega^{N_1}\wedge\Omega^{N_2})$.  Then $c_2$ is a regular value of $F_2$, $c_1$ is a regular value of $F_1|_{F_2^{-1}(c_2)}$ and we have
\begin{equation}
\delta(F)\Omega^M=\delta(F_1)(\delta(F_2)\Omega^M)\,.
\end{equation}
\end{theorem}
\begin{proof}
$(c_1,c_2)$ is a regular value of $F$, hence there exists $v_i$, $w_i$ such that $F_* v_i=(\tilde{v}_i,0)$, $F_* w_i=(0,\tilde{w}_i)$ satisfy 
\begin{equation}
\Omega^{N_1}\wedge\Omega^{N_2}( (\tilde{v}_1,0),..., (0,\tilde{w}_1),...)=1\,.
\end{equation}
After rescaling, we can assume
\begin{equation}
\Omega^{N_1}( \tilde{v}_1,...,\tilde{v}_{n_1})=1,\hspace{2mm} \Omega^{N_2}(\tilde{w}_1,...,\tilde{w}_{n_2})=1\,.
\end{equation}
Therefore $c_2$ is a regular value of $F_2$ and 
\begin{equation}
\delta(F_2)\Omega^M=i_{w_1,...,w_n}\Omega^M\,.
\end{equation}
The tangent space to $F_2^{-1}(c_2)$ is $\ker (F_2)_*$ which contains $v_i$.  Hence $c_1$ is a regular value of $F_1|_{F_2^{-1}(c_2)}$ and 
\begin{align}
\delta(F_1)(\delta(F_2)\Omega^M)=&i_{v_1,...,v_n}\delta(F_2)\Omega^M=\pm i_{v_1,...,v_n,w_1,...,w_n}\Omega^M\,,
\end{align}
therefore they agree as densities.
\end{proof}

\begin{theorem}
Let $c_i\in N_i$ be regular values of $F_i:M_i\rightarrow N_i$ and define $F=F_1\times F_2:M_1\times M_2\rightarrow N_1\times N_2$, $c=(c_1,c_2)$.  If $\Omega^{M_i}$ and $\Omega^{N_i}$ are volume forms on $M_i$ and $N_i$ respectively then 
\begin{equation}
\delta_c( F) \left(\Omega^{M_1}\wedge\Omega^{M_2}\right)=\left(\delta_{c_1}( F_1)\Omega^{M_1}\right)\wedge\left(\delta_{c_2}( F_2)\Omega^{M_2}\right)
\end{equation}
as densities.
\end{theorem}
\begin{proof}
Our assumptions ensure that both sides are $m_1+m_2-n_1-n_2$-forms on $F_1^{-1}(c_1)\times F_2^{-1}(c_2)$.  Choose $v_i^j\in TM_i$ that satisfy $\Omega^{N_i}(F_{i*}v^1_i,...,F_{i*}v^{n_i}_i)=1$ then
\begin{align}
&\Omega^{N_1}\wedge \Omega^{N_2}(F_*(v_1^1,0),...,F_*(v_1^{n_1},0),F_*(0,v_2^1),...,F_*(0,v_2^{n_2}))\\
=&\Omega^{N_1}\wedge \Omega^{N_2}(F_{1*}v_1^1,...,F_{2*}v_2^{n_2})\notag\\
=&\Omega^{N_1}(v_1^1,...,v_1^{n_1})\Omega^{N_2}(v_2^1,...,v_2^{n_2})=1\,.\notag
\end{align}
Therefore, by definition
\begin{align}
\delta_c\circ F \left(\Omega^{M_1}\wedge\Omega^{M_2}\right)=&i_{(v_1^1,0),...,(v_1^{n_1},0),(0,v_2^1),...,(0,v_2^{n_2})}\left(\Omega^{M_1}\wedge\Omega^{M_2}\right)\\
=&(-1)^{n_2}\left(i_{v_1^1,...,v_1^{n_1}}\Omega^{M_1}\right)\wedge\left(i_{v_2^1,...,v_2^{n_2}}\Omega^{M_2}\right)\notag\\
=&(-1)^{n_2}\left(\delta_{c_1}\circ F_1\right)\wedge\left(\delta_{c_2}\circ F_2\right)\,.\notag
\end{align}
Therefore they agree as densities.
\end{proof}

\begin{theorem}\label{delta_product}
Let $F_i:M_i\rightarrow N_i$ and $g:N_1\times N_2\rightarrow K$ be smooth.  Let $\Omega^{M_i}$, $\Omega^{N_1}$, $\Omega^K$ be volume forms on $M_i$, $N_1$, $K$ respectively.  Suppose $c$ is a regular value of $F_1$ and $d$ is a regular value of $g(c,F_2)$ and of $g\circ F_1\times F_2$. Then 
\begin{equation}
\delta_c(F_1)\left[\delta_d( g\circ F_1\times F_2)\left(\Omega^{M_1}\wedge\Omega^{M_2}\right)\right]=\left(\delta_c(F_1)\Omega^{M_1}\right)\wedge\left(\delta_d(g(c, F_2))\Omega^{M_2}\right)\,.
\end{equation}
\end{theorem}
\begin{proof}
 Let $(x,y)\in (f\circ F_1\times F_2)^{-1}(d)$ with $x\in F^{-1}(c)$. For any $w\in T_c N_1$ there exists $v\in T_x M_1$ such that $F_{1*}v=w$.  $d$ is a regular value of $g(c,F_2)$ hence there exists $\tilde{v}$ such that $g(c,F_2)_*\tilde{v}=(g\circ F_1\times F_2)_*(v,0)$.  Therefore $(g\circ F_1\times F_2)_*(v,-\tilde{v})=0$ and $F_1*(v,-\tilde{v})=w$.  This proves $c$ is a regular value of $F_1$ on $(g\circ F_1\times F_2)^{-1}(d)$.  This proves both sides are defined and are forms on $F^{-1}(c)\times g(c,F_2)^{-1}(d)$.

  Let $x\in F^{-1}(c)$ and $y\in  g(c,F_2)^{-1}(d)$ and choose $v_i$, $w_j$ such that
\begin{equation}
\Omega^{N_1}(F_{1*}v_1,...,F_{1*}v_{n_1})=1\,, \hspace{2mm} \Omega^K(g(c,F_2)_*w_1,...,g(c,F_2)_*w_k)=1\,.
\end{equation}
Then 
\begin{equation}
\Omega^K((g\circ F_1\times F_2)_*(0,w_1),...,(g\circ F_1\times F_2)_*(0,w_k))=1
\end{equation}
and so 
\begin{align}
\delta_d( g\circ F_1\times F_2)\left(\Omega^{M_1}\wedge\Omega^{M_2}\right)&=i_{(0,w_1),...,(0,w_k)}\left(\Omega^{M_1}\wedge\Omega^{M_2}\right)\\
&=\Omega^{M_1}\wedge\left(i_{w_1,...,w_k}\Omega^{M_2}\right)\notag\\
&=\Omega^{M_1}\wedge\left(\delta_d(g(c,F_2))\Omega^{M_2}\right)\,.\notag
\end{align}
By the same argument as above, we get $\tilde{v}_i$ such that $(v_i,\tilde{v}_i)\in T_{(x,y)} (g\circ F_1\times F_2)^{-1}(d)$.  Hence
\begin{equation}
\delta_c(F_1)\left[\delta_d( g\circ F_1\times F_2)\left(\Omega^{M_1}\wedge\Omega^{M_2}\right)\right]=i_{(v_1,\tilde{v}_1),...,(v_{n_1},\tilde{v}_{n_1})}\left[\Omega^{M_1}\wedge\left(i_{w_1,...,w_k}\Omega^{M_2}\right)\right]\,.
\end{equation}
The only non-vanishing term is
\begin{equation}
\left(i_{(v_1,\tilde{v}_1),...,(v_{n_1},\tilde{v}_{n_1})}\Omega^{M_1}\right)\wedge\left(i_{w_1,...,w_k}\Omega^{M_2}\right)=\left(i_{v_1,...,v_{n_1}}\Omega^{M_1}\right)\wedge\left(i_{w_1,...,w_k}\Omega^{M_2}\right)
\end{equation}
since the other terms all contain a $m_1-n_1+l$ form on the $m_1-n_1$-dimensional manifold $F^{-1}(c)$ for some $l>0$.  This proves the result.
\end{proof}

Sometimes it is convenient to use the delta function to introduce ``dummy integration variables", by which we mean utilizing the following simple corollary of the coarea formula.
\begin{corollary}\label{dummy_int}
Let $\Omega^M$ be a volume form on $M$, $F:M\rightarrow (N,\Omega^N)$ be smooth, and $f:N\times M\rightarrow \mathbb{R}$  such that $f(F(\cdot),\cdot)\in L^1(\Omega^M)\bigcup L^+(M)$.  If $F_*$ is surjective at a.e. $x\in M$ then
\begin{equation}
\int_M f(F(x),x)\Omega^M(dx)=\int_N\int_{F^{-1}(z)} f(z,x)\delta_z(F)\Omega^M(dx) \Omega^N(dz)\,.
\end{equation}
\end{corollary}




\subsection{Applications}
\subsubsection{Relativistic Volume Element}\label{rel_vol_form}
We now discuss an application of the above results to the single particle phase space volume element. We first define it in the massive case, where the semi-Riemannian method of defining volume forms is applicable.  The massless case is often handled via a limiting argument \cite{tsamparlis}.  We will show that   our method is able to handle both the massive and massless case in a unified manner.

 Given a time oriented $n+1$ dimensional semi-Riemannian manifold $(M,g)$, there is a natural induced metric $\tilde{g}$ on the tangent bundle, called the diagonal lift.  At a given point $(x,p)\in TM$ its coordinate independent definition is
\begin{align}
\tilde{g}_{(x,p)}(v,w)=g_x(\pi_{*} v,\pi_{*} w)+g_x(D_t \gamma_v, D_t \gamma_w)\,,
\end{align}
where $\gamma_v$ is any curve in $TM$ with tangent $v$ at $x$, $\pi:TM\longrightarrow M$ is the projection, and $D_t\gamma_v$ is the covariant derivative of $\gamma_v$, treated as a vector field along the curve $\pi\circ\gamma_v$, and similarly for $\gamma_w$, see, e.g., \cite{pettini}. The result can be shown to be independent of the choice of curves.  In a coordinate system on $M$ where the the first coordinate is future timelike and the Christoffel symbols are $\Gamma^\beta_{\sigma\eta}$, consider the  induced coordinates $(x^\alpha,p^\alpha), \hspace{2mm}\alpha=0,...,n$ on $TM$.  In these coordinates we have 
\begin{equation}
\tilde{g}_{(x^\alpha,p^\alpha)}=g_{\beta,\delta}(x^\alpha)dx^\beta\otimes dx^\delta +g_{\beta,\delta}(x^{\alpha})\epsilon^\beta\otimes \epsilon^\delta, \hspace{2mm} \epsilon^\beta=dp^\beta+p^\sigma\Gamma^\beta_{\sigma\eta}(x^{\alpha})dx^\eta\,.
\end{equation}
The vertical and horizontal subspaces are spanned by
\begin{equation}\label{horizontal_subspace}
V_\alpha=\partial_{p^\alpha}, \hspace{2mm} H_\alpha=\partial_{x^\alpha}-p^\sigma\Gamma_{\sigma\alpha}^\beta\partial_{p^\beta}
\end{equation}
respectively.  The horizontal vector fields satisfy
\begin{equation}
\tilde{g}(H_\alpha,H_\beta)=g_{\alpha\beta}\,.
\end{equation}
For any manifold (oriented or not), the tangent bundle has a canonical orientation.  With this orientation, the volume form on $TM$ induced by $\tilde{g}$ is
\begin{equation}
\widetilde{dV}_{(x^\alpha,p^{\alpha})}=|g(x^\alpha)|dx^0\wedge...\wedge dx^n\wedge dp^0\wedge...\wedge dp^n\,,
\end{equation}
where $|g(x^\alpha)|$ denotes the absolute value of the determinant of the component matrix of $g$ in these coordinates.

Of primary interest in kinetic theory for a particle of mass $m\geq 0$ is the mass shell bundle
\begin{equation}
P_m=\{p\in TM :g(p,p)=m^2,\hspace{1mm}  p\text{ future directed}\}
\end{equation}
and it will be necessary to have a volume form on $P_m$.  $P_m$ is a connected component of the zero set of the of the smooth map 
\begin{equation}\label{defining_function}
h:TM\setminus \{0_x:x\in M\}\longrightarrow \mathbb{R},\hspace{2mm} h(x,p)= \frac{1}{2}(g_x(p,p)-m^2)\,.  
\end{equation} 
We remove the image of the zero section to avoid problems when $m=0$.  Its differential is
\begin{equation}\label{dh}
dh=\frac{1}{2}\frac{\partial g_{\sigma\delta}}{\partial x^\alpha}p^\sigma p^\delta dx^\alpha+g_{\sigma\delta}p^\sigma dp^\delta=g_{\sigma\delta}p^\sigma\epsilon^\delta\,.
\end{equation}
$g$ is nondegenerate, so for $p=p^{\alpha}\partial_{x^\alpha}\in TM_x\setminus{\{0_x\}}$ there is some $v=v^\alpha\partial{x^\alpha}\in TM_x$ with $g(v,p)\neq 0$.  Therefore
\begin{equation}
dh_{(x,p)}(v^\alpha\partial_{p^\alpha})=g(v,p)\neq 0\,.
\end{equation}
This proves $P_m$ is a regular level set of $h$, and hence is a closed embedded hypersurface of $TM\setminus \{0_x:x\in M\}$.  For $m\neq 0$ it is also closed in $TM$, but for $m=0$ every zero vector is a limit point of $P_m$.

\noindent{\bf Massive Case:}\\
For $m\neq 0$, we will show that $P_m$ is a semi-Riemannian hypersurface in $TM$ and hence inherits a volume form from $TM$. This is the standard method of inducing a volume form, as presented in \cite{tsamparlis}.  

The normal to $P_m$ is 
\begin{equation}
\grad h=\tilde{g}^{-1}(dh)=p^\alpha\partial_{p^\alpha}
\end{equation}
which has norm squared 
\begin{equation}
\tilde{g}(\grad h,\grad h)=g(p,p)=m^2\,.
\end{equation}
Therefore, for $m\neq 0$, $P_m$ has a unit normal $N=\grad h/m$ and so it is a semi-Riemannian hypersurface with volume form
\begin{equation}
\widetilde{dV}_m=i_N \widetilde{dV}=\frac{|g|}{m}dx^0\wedge...\wedge dx^n\wedge\left(\sum_\alpha (-1)^\alpha p^\alpha dp^0\wedge...\wedge\widehat{dp^\alpha}\wedge...\wedge dp^n\right)\,,
\end{equation}
where $i_N$ denotes the interior product (or contraction) and a hat denotes an omitted term.  We are also interested in the volume form on $P_{m,x}$ the fiber of $P_m$ over a point $x\in M$.  We obtain this by contracting $\widetilde{dV}$ with an orthonormal basis of vector fields normal to $P_{m,x}$. Such a basis is composed of $N$ together with an orthonormalization of the basis of horizontal fields, $W_\alpha=\Lambda^\beta_\alpha H_\beta$, where $H_\beta$ are defined in \req{horizontal_subspace}. Therefore we have
\begin{equation}\label{contract_horiz}
\widetilde{dV}_{m,x}=i_{W_0}...i_{W_n}\widetilde{dV}_m\,.
\end{equation}
 We can simplify these expressions by defining a coordinate system on the momentum bundle, writing $p^0$ as a function of the $p^i$.  The details, which are standard, are carried  out in Appendix  \ref{coord_comp}.  The results are
\begin{equation}
\widetilde{dV}_m=\frac{m|g|}{p_0}dx^0\wedge...\wedge dx^n\wedge dp^1\wedge...\wedge dp^n\,,
\end{equation}
\begin{equation}
\widetilde{dV}_{m,x}=\frac{m|g|^{1/2}}{p_0}dp^1\wedge...\wedge dp^n\,.
\end{equation}
We define $\pi$ and $\pi_x$ by
\begin{equation}
\pi=\frac{1}{m}\widetilde{dV}_m=\frac{|g|}{p_0}dx^0\wedge...\wedge dx^n\wedge dp^1\wedge...\wedge dp^n\,,
\end{equation}
\begin{equation}\label{pi_x}
\pi_x=\frac{1}{m}\widetilde{dV}_{m,x}=\frac{|g|^{1/2}}{p_0}dp^1\wedge...\wedge dp^n\,.
\end{equation}
We will typically omit the subscript $x$ and let the context distinguish whether we are integrating over the full momentum bundle (i.e. both over spacetime and momentum variables) or just momentum space at a single point in spacetime.  \\

\noindent{\bf Massless Case:}\\
When $m=0$ the above construction fails.  However, we can use Theorem \ref{induced_vol_form} to induce a volume form using the map \req{defining_function} defined above. Here we carry out the construction for the induced volume form on $P_{m,x}$ for any $m\geq 0$. The volume form on each tangent space $T_xM$ is
\begin{equation}
\tilde{dV}_x=|g(x)|^{1/2}dp^0\wedge...\wedge dp^n\,.
\end{equation}
We assume that the coordinates are chosen so that the vector field $\partial_{p^0}$ is timelike. By \req{dh} we find
\begin{equation}
dh(\partial_{p^0})=g_{\alpha 0}p^\alpha\neq 0
\end{equation}
on $P_{m,x}$.  Therefore, by Corollary \ref{induced_vol_eq} the induced volume form is
\begin{align}\label{mass_shell_vol}
\omega=&\frac{1}{dh(\partial_{p^0})} i_{\partial_{p^0}} \tilde{dV}_x
=\frac{|g|^{1/2}}{p_0}dp^1\wedge...\wedge dp^n\,.
\end{align}
We can also pull this back under the coordinate chart on $P_{m,x}$ defined in Appendix \ref{coord_comp} and obtain the same expression in coordinates. This result agrees with our prior definition of \req{pi_x} in the case where $m>0$ but is also able to handle the massless case in a uniform manner, without resorting to a limiting argument as $m\rightarrow 0$.

We also point out another convention in common use where $h$ is replaced by $2h$.  This leads to an additional factor of $1/2$ in the volume element, distinguishing this definition from the one based on semi-Riemannian geometry.  However, the convention
\begin{equation}
\omega=\frac{|g|^{1/2}}{2p_0}dp^1\wedge...\wedge dp^n
\end{equation}
 is in common use and will be employed in the scattering integral computations in Appendix \ref{ch:coll_simp}.

\subsubsection{Relativistic Phase Space}
Here we justify several manipulations that are useful for working with relativistic phase space integrals.

\begin{lemma}\label{parallel_lemma}
Let $V$ be an $n$-dimensional vector space.  The subset of $\prod_1^N V\setminus\{0\}$ consisting of $N$-tuples of parallel vectors is an $n+N-1$ dimensional closed submanifold of $\prod_1^N V\setminus\{0\}$.
\end{lemma}
\begin{proof}
The map $V\times \mathbb{R}^{N-1}\rightarrow  \prod_1^N V\setminus\{0\}$ given by
\begin{equation}
F(p,a^2,...,a^N)=(p,a^2p,...,a^{N}p)
\end{equation}
is an injective immersion and maps onto the desired set.
\end{proof}
For reactions converting $k$ particles to $l$ particles, the relevant phase space is $3(k+l)-4$ dimensional and so for $k+l\geq 4$ (in particular for $2$-$2$ reactions), the set of parallel $4$-momenta is lower dimensional and can be ignored. This will be useful as we proceed.

\begin{lemma}
Let $N\geq 4$. Then
\begin{equation}
\prod_i \delta(p_i^2-m_i^2)d^4p_i=\left(\prod_i \delta(p_i^2-m_i^2)\right)\prod_i d^4p_i
\end{equation}
 and 
\begin{equation}
\delta(\Delta p)\left[\left(\prod_i \delta(p_i^2-m_i^2)\right)\prod_i d^4p_i\right]=\left(\delta(\Delta p)\prod_i \delta(p_i^2-m_i^2)\right)\prod_i d^4p_i\,,
\end{equation}
where each $d^4p_i$ is the standard volume form on future directed vectors, $\{p:p^2\geq 0, p^0>0\}$, we give $\mathbb{R}$ its standard volume form, and $\Delta p=a^ip_i$, $a^i=1$, $i=1,...,l$, $a^i=-1$, $i=l,...,N$. 
\end{lemma}
\begin{proof}
Let $F_1(p_i)=(p_1^2,...,p_N^2)$ and $F_2(p_i)=(\Delta p,F_1(p_i))$.  We need to show that $(m_1^2,...,m_N^2)$ is a regular value of $F_1$ and $(0,m_1^2,...,m_k^2)$ is a regular value of $F_2$.  The result then follows from Theorem \ref{delta_associative}.

It holds for $F_1$ since each $p_i\neq 0$. For $F_2$, the differential is
\begin{equation}
(F_2)*=\left( \begin{array}{cccc}
a^{1}I&a^{2}I&...& a^{N}I \\
2 \eta_{ij}p^j_1&0&...&0\\
\vdots&&&\vdots\\
0&...&0&2 \eta_{ij}p^j_N\\
\end{array} \right)
\end{equation}
where $I$ is the $4$-by-$4$ identity.  The fact that $(F_1)_*$ is onto means that we need only show $(F_2)_*$ maps onto $\mathbb{R}^4\times(0,...,0)$.  

By Lemma \ref{parallel_lemma} we assume there exists $i,j$ such that $p_i,p_j$ are not parallel. We are done if for each standard basis vector $e_k\in\mathbb{R}^4$ there exists $q\in\mathbb{R}^4$ such that
\begin{equation}
p_i\cdot q=\frac{1}{a^j}p_i\cdot e_k,\hspace{2mm} p_j\cdot q=0\,.
\end{equation}
If $p_j$ is null then there is a $c$ such that $q=c p_j$ satisfies these conditions. If $p_j$ is non-null then complete it to an orthonormal basis.  $p_i$ must have a component along the orthogonal complement of $p_j$ and we can take $q$ to be proportional to that component.

\end{proof}



\subsection{Volume Form in Coordinates}\label{coord_comp}
Here we derive a useful formula for the volume form on the momentum bundle in a simple coordinate system.  We begin in a coordinate system $x^\alpha$ on $U\subset M$ and the induced coordinates $p^\alpha$ on $TM$ where our only assumption is that the $0$'th coordinate direction is future timelike, and so $g_{00}>0$.  For any $v^i\in \mathbb{R}^n$, let $v^0=-g_{0i}v^i/g_{00}$.  $v^\alpha$ is orthogonal to the $0$'th coordinate direction, and therefore spacelike. Hence 
\begin{equation}
0\geq g_{\alpha \beta}v^\alpha v^\beta=-(g_{0i}v^i)^2/g_{00}+g_{ij}v^iv^j\,.
\end{equation}
and is zero iff $v^\alpha=0$. Therefore, the following map is well defined
\begin{align}
(x^\alpha,p^j)&\longrightarrow (x^\alpha,p^0(x^\alpha,p^j),p^1,...,p^n),  \hspace{2mm} \alpha=0...n, \hspace{1mm} j=1...n \,,\notag\\
 p^0&=-g_{0j}p^j/g_{00}+\left((g_{0j}p^j/g_{00})^2+(m^2-g_{ij}p^ip^j)/g_{00}\right)^{1/2}\,,
\end{align}
and is smooth on $\mathbb{R}^{n+1}\times\mathbb{R}^n$ if $m\neq 0$, and on $\mathbb{R}^{n+1}\times\left(\mathbb{R}^n\setminus{0}\right)$ if $m=0$.  We also have $g_{00}p^0+g_{0j}p^j>0$ under either of these cases, and so the resulting element of $TM$ is future directed and has squared norm $m^2$, so it maps into $P_m$.  It is a bijection and has full rank, hence it is a coordinate system on $P_m$.  In these coordinates, the volume form is
\begin{align}
\widetilde{dV}_m=&\frac{|g|}{m}dx^0\wedge...\wedge dx^n\wedge\left(p^0dp^1\wedge...\wedge dp^n+\sum_j (-1)^j p^j dp^0\wedge...\wedge\widehat{dp^j}\wedge...\wedge dp^n\right)\notag\\
dp^0=&\partial_{x^\alpha} p^0dx^\alpha+\partial_{p^j}(p^0) dp^j\,.
\end{align}
The terms in $dp^0$ involving $dx^\alpha$ drop out once they are wedged with $dx^0\wedge...\wedge dx^n$, hence
\begin{align}
&\widetilde{dV}_m\\
=&\frac{|g|}{m}dx^0\wedge...\wedge dx^n\wedge\left(p^0dp^1\wedge...\wedge dp^n+\sum_{i,j} (-1)^j p^j \partial_{p^i}p^0 dp^i\wedge...\wedge\widehat{dp^j}\wedge...\wedge dp^n\right)\notag\\
=&\frac{|g|}{m}\left(p^0-\sum_{j}p^j \partial_{p^j}(p^0) \right)dx^0\wedge...\wedge dx^n\wedge dp^1\wedge...\wedge dp^n\,,\notag\\
&p^0-\sum_jp^j\partial_{p^j}(p^0)= p^0+g_{0j}p^j/g_{00}-\frac{(g_{0j}p^j/g_{00})^2-g_{ij}p^ip^j/g_{00}}{\left((g_{0j}p^j/g_{00})^2+(m^2-g_{ij}p^ip^j)/g_{00}\right)^{1/2}}\notag\\
=&\frac{1}{p_0}\left(\frac{1}{g_{00}}(g_{00}p^0+g_{0,j}p^j)^2-(g_{0j}p^j)^2/g_{00}+g_{ij}p^ip^j\right)=\frac{m^2}{p_0}\,.\notag
\end{align}
Therefore
\begin{equation}
\widetilde{dV}_m=\frac{m|g|}{p_0}dx^0\wedge...\wedge dx^n\wedge dp^1\wedge...\wedge dp^n\,.
\end{equation}

To compute the volume form on $P_{m,x}$, recall  that 
\begin{equation}\label{contract_horiz}
\widetilde{dV}_{m,x}=i_{W_0}...i_{W_n}\widetilde{dV}_m\,.
\end{equation}
Where $W_i$ is an orthonormalization of the basis of horizontal fields, $W_\alpha=\Lambda^\beta_\alpha H_\beta$, where $H_\beta$ are defined in \req{horizontal_subspace}. All of the contractions in \req{contract_horiz} that involve the $dp^\alpha$'s will be zero when restricted to $P_{m,x}$ since the $dx^\alpha$ are zero there. Hence we obtain
\begin{align}\label{dV_x}
\widetilde{dV}_{m,x}=&\frac{|g|}{m}\left(p^0-\sum_{j}p^j \partial_{p^j}(p^0) \right)dx^0\wedge...\wedge dx^n\left(W_0,...,W_n)\right) dp^1\wedge...\wedge dp^n\\
=&\frac{|g|\det(\Lambda)}{m}\left(p^0-\sum_{j}p^j \partial_{p^j}(p^0) \right)dx^0\wedge...\wedge dx^n\left(H_0,...,H_n)\right) dp^1\wedge...\wedge dp^n\notag\\
=&\frac{|g|^{1/2}}{m}\left(p^0-\sum_{j}p^j \partial_{p^j}(p^0) \right) dp^1\wedge...\wedge dp^n\,,\notag
\end{align}
where we used $\det(\Lambda^\sigma_\alpha g_{\sigma\delta}\Lambda^\delta_\beta)=1$.
 In the coordinate system on $P_{m,x}$
\begin{align}
(p^j)&\longrightarrow (p^0(x^\alpha,p^j),p^1,...,p^n)\,,\\
 p^0&=-g_{0j}(x)p^j/g_{00}(x)+\left((g_{0j}(x)p^j/g_{00}(x))^2+(m^2-g_{ij}(x)p^ip^j)/g_{00}(x)\right)^{1/2}\,.\notag
\end{align}
 the same calculation as above gives the formula
\begin{equation}
\widetilde{dV}_{m,x}=\frac{m|g|^{1/2}}{p_0}dp^1\wedge...\wedge dp^n\,.
\end{equation}




\include{06-appendix/SpectralMethodBoltzmann/BoltzmannOrthopoly.tex}
\include{06-appendix/NeutrinoScatteringIntegrals.tex}


\bibliographystyle{spphys}
\bibliography{bibs/birrell-refs,bibs/steinmetz-refs,bibs/yang-refs,bibs/grayson-refs}
%%%%%%%%%%%%%%%%%%%%%%%%
\end{document}
