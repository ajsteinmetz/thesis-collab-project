\chapter{Fugacity and Reheating of Primordial Neutrinos.}\label{app:chem_freezeout}
J. Birrell, C.T. Yang, P. Chen, J. Rafelski, Mod. Phys. Lett. A 28, 1350188 (2013) 
DOI: 10.1142/S0217732313501885 

\section*{ Summary}

In this paper we studied numerically the dependence of the neutrino distribution, including fugacity $\Upsilon$, effective number of neutrinos $N_\nu$, and reheating ratio $T_\gamma/T_\nu$ on the kinetic freezeout temperature $T_k$ under the assumption of kinetic equilibrium. In particular, we showed that  a measurement of $N_\nu$ constitutes a measurement of $T_k$ in this model. We demonstrated that the instantaneous chemical freeze-out approximation, and hence entropy conservation,  holds to a good approximation for a subset of the neutrino reactions.  We also numerically identified an approximate power law relation between $\Upsilon$ and $T_\gamma/T_\nu$.

Others worked on deriving the neutrino reaction rates, but I was responsible for deriving all other equations, numerically simulating the system, and producing both figures. I was also responsible for the creation of the initial draft of the manuscript.
