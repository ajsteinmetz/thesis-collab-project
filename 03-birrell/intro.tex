\part{Neutrino Freeze-out via Conservation Laws}
\chapter{Introduction to Cosmology and the Relic Neutrino Background}\label{ch:intro}

At a temperature of $5$ MeV the Universe consisted of a plasma of $e^\pm$, photons, and neutrinos.  At around $1$ MeV neutrinos stop interacting, or freeze-out, and begin to free-stream through the Universe. Today they comprise the relic neutrino background. Photons freeze-out around $0.25$ eV and today they make up the Cosmic Microwave Background (CMB), currently at $T_{\gamma,0}=0.235$ meV.  Relic neutrinos have not been directly measured, but their impact on the speed of expansion of the Universe is imprinted on the CMB.  Indirect measurements of the relic neutrino background, such as by the Planck satellite \cite{Planck},  constrain neutrino properties such as mass and number of massless degrees of freedom.

 In later chapters, we will study the details of the neutrino freeze-out process and their impact on observables in detail but first we present an overview of cosmology, from just prior to neutrino freeze-out until today, putting the relic neutrinos in their proper context. Much of this material, including most figures, was adapted from our paper \cite{ErasOfUniverse}.


\section{Standard Cosmology}\label{cosmo}
 To follow the history of the relic neutrino distribution, one must first understand the relation between the expansion dynamics of the Universe, its energy content, and the connection to the photon and neutrino temperature. For this purpose we need some preparation in the  Friedmann$-$Lemaitre$-$Robertson$-$Walker (FRW) cosmological  model, see for example \cite{hartle2003gravity,hobson,misner1973gravitation}. Assuming a homogeneous, isotropic Universe, one arrives at the spacetime metric
\beqn\label{metric}
ds^2=dt^2-a^2(t)\left[ \frac{dr^2}{1-kr^2}+r^2(d\theta^2+\sin^2(\theta)d\phi^2)\right]
%g_{00}=1, \quad g_{rr}=-\frac{a^2}{1-kr^2}, \quad g_{\theta\theta}=-a^2r^2, \quad g_{\phi\phi}=-a^2 r^2\sin^2\theta
\eeqn
characterized  by the scale parameter $a(t)$.  $a(t)$ determines the distance between objects at rest in the Universe frame, otherwise known as comoving observers. The geometric parameter $k=-1,0,1$ identifies the geometry of the spacial hypersurfaces defined by comoving observers. Space is a flat-sheet for the observationally preferred value $k=0$ \cite{Planck}, hyperbolic for $k=-1$, and spherical for $k=1$.

The dynamics are governed by the Einstein equations
\beqn\label{Einstine}
G^{\mu\nu}=R^{\mu\nu}-\left(\frac R 2 -\Lambda\right) g^{\mu\nu}=-\frac{1}{M_p^2} T^{\mu\nu},  
\quad R= g_{\mu\nu}R^{\mu\nu}
\eeqn
where $M_p\equiv 1/\sqrt{8\pi G_N}$ is the Planck mass, $G_N$ is the gravitational constant, and we work in units where $\hbar=c=1$. Recall that the Einstein tensor $G^{\mu\nu}$ is divergence free and hence so is the total stress energy tensor, $T^{\mu\nu}$.  Note that our definition of $M_p$, while more convenient in cosmology, differs by a factor of $1/\sqrt{8\pi}$ from the particle physics convention.  Finally, we point out that there are several sign conventions in use regarding the definition of geometrical quantities and Einstein's equation that are clarified in appendix \ref{app:conventions}.

 In a homogeneous isotropic spacetime, the matter content is necessarily characterized by two quantities, the energy density $\rho$ and isotropic pressure $P$
\begin{equation}
  T^\mu_\nu =\mathrm{diag}(\rho, -P, -P, -P).
\end{equation}
 It is common to absorb the Einstein cosmological constant $\Lambda$ into $\rho$ and $P$ by defining
\beqn\label{EpsLam}
\rho_\Lambda=M_p^2\Lambda, \qquad P_\Lambda=-M_p^2 \Lambda.
\eeqn
We implicitly consider this done from now on. 


The global Universe dynamics can be characterized by two  quantities, the Hubble parameter  $H$, a strongly time dependent quantity on cosmological time scales,  and the deceleration parameter $q$
\beqn\label{dynamic}
\frac{\dot a }{a}\equiv H(t) ,\quad 
q\equiv -\frac{a\ddot a}{\dot a^2}.
\eeqn
We note the relations
\beqn
\quad \frac{\ddot a}{a}=-qH^2,\quad \dot H=-H^2(1+q). 
\eeqn

Two dynamically independent equations arise using the metric \req{metric} in \req{Einstine}
\beqn\label{hubble}
\frac{8\pi G_N}{3} \rho =  \frac{\dot a^2+k}{a^2}
=H^2\left( 1+\frac { k }{\dot a^2}\right),
\qquad
\frac{4\pi G_N}{3} (\rho+3P)  =-\frac{\ddot a}{a}=qH^2.
\eeqn
We can eliminate the strength of the interaction, $G_N$,  solving both these equations for ${8\pi G_N}/{3}$, and equating the result to find a relatively simple constraint for the deceleration parameter
\beqn\label{qparam}
q=\frac 1 2 \left(1+3\frac{P}{\rho}\right)\left(1+\frac{k}{\dot a^2}\right).
\eeqn
 From this point on, we work within the  flat cosmological model with $k=0$ and so $q$ is determined entirely by the matter content of the Universe
\begin{equation}\label{qparam}
q=\frac 1 2 \left(1+3\frac{P}{\rho}\right).
\end{equation}


As must be the case for any solution of Einstein's equations,   \req{hubble} implies that the energy momentum tensor of matter is divergence free
\beqn\label{divTmn}
\nabla_\nu T^{\mu\nu} =0 \Rightarrow -\frac{\dot\rho}{\rho+P}=3\frac{\dot a}{a}=3H.
\eeqn
 The same relation also follows from  conservation of entropy, $dE+PdV=TdS=0,\  dE=d(\rho V),\  dV=d(a^3)$. Given an equation of state $P(\rho)$, solution of \req{divTmn} describes the dynamical evolution of matter in the Universe. Combined with the Hubble equation
\begin{equation}\label{Hubble_eq}
H^2=\frac{\rho}{3M_p}
\end{equation}
this allows us to solve for the large scale dynamics of the Universe. 

Using the flat FRW model of cosmology outlined above, we now present several perspectives on the history of the Universe.  First we focus on the reheating history. 

%%%%%%%%%%%%%%%%%%%%%%%%%%%%%%%%%%%%%%%%%%%%%%%%%%%%%%%%%%%%%%%
\section{Reheating History of the Universe}\label{Eralink}

At times where dimensional scales are irrelevant, entropy conservation means that  temperature scales inversely with the scale factor $a(t)$. This follows from \req{divTmn} when $ \rho\simeq 3P   \propto T^4$. However, as the temperature drops and at their respective $m\simeq T$ scales, successively less massive particles annihilate and disappear from the thermal Universe. Their entropy reheats the other degrees of freedom and thus in the process, the entropy originating in a massive degree of freedom is shifted into the effectively massless degrees of freedom that still remain.  This causes the  $T\propto 1/a(t)$ scaling to break down; during each of these `reorganization' periods the drop in temperature is slowed by the concentration of entropy in fewer degrees of freedom, leading to a change in the reheating ratio, $R$, defined as
\begin{equation}\label{redshiftratio}
R\equiv \frac{1+z}{ T_\gamma/T_{\gamma,0}}, \qquad 1+z\equiv \frac{a_{0}}{a(t)}.
\end{equation}
The reheating ratio connects the photon temperature redshift to the geometric redshift, where $a_0$ is the scale factor today (often normalized to $1$) and quantifies the deviation from the scaling relation between $a(t)$ and $T$.

As we will see, the change in $R$ can be computed by the drop in the number of degrees of freedom.  At a temperature on the order of the top quark mass, when all standard model particles were in thermal equilibrium, the Universe was pushed apart by 28 bosonic and 90 fermionic degrees of freedom. The total number of degrees of freedom can be computed as follows.  

For bosons we have the following: the doublet of charged Higgs particles has $4=2\times2=1+3$  degrees of freedom -- three will migrate to the longitudinal components of $W^\pm, Z$ when the electro-weak vacuum freezes and the EW symmetry breaking arises, while one is retained in the one single dynamical charge neutral Higgs component. In the massless stage, the SU(2)$\times$U(1) theory has 4$\times$2=8 gauge degrees of freedom where the first coefficient  is  the number of particles $(\gamma, Z, W^\pm)$ and each massless gauge boson has  two transverse polarizations. Adding in $8_c\times2_s=16$ gluonic degrees of freedom we obtain 4+8+16=28  bosonic degrees of freedom. 

The count of fermionic degrees of freedom includes three $f$ families, two spins $s$, another factor two for particle-antiparticle duality. We have in each family of flavors a doublet of $2\times 3_c$ quarks, 1-lepton and 1/2 neutrinos (due left-handedness which was not implemented counting spin). Thus we find that a total $3_f\times 2_p\times 2_s\times(2\times 3_c+1_l+1/2_\nu)=90$ fermionic degrees of freedom. We further recall that massless fermions contribute 7/8 of that of bosons in both pressure and energy density. Thus the total number of massless Standard Model particles at a temperature above the top quark mass scale, referring by convention to bosonic degrees of freedom, is $g_{\rm SM}=28+90\times 7/8=106.75$ 



In figure~\ref{fig:dof}  we show the cube of the reheating ratio \req{redshiftratio} as a function of photon temperature $T_\gamma$ from the primordial high temperature  early Universe on the right to the present on the left, where $R$  must be by definition unity.  The periods of change seen in figure \ref{fig:dof} come when the temperature crosses the mass of a particle species that is in equilibrium. One can see drops corresponding to the disappearance of particles as indicated.   After $e^+e^-$ annihilation on the left, there are no significant degrees of freedom remaining to annihilate and feed entropy into photons, and so $R$  remains constant until today. We show the result using a Fermi gas model with a very rough model for the QGP phase transition and hadronization period near $O(100\MeV)$. The fermi gas model is a poor approximation above the QGP phase transition; a more precise model using lattice QCD, see e.g. \cite{Borsanyi:2013bia}, together with a high temperature perturbative QCD expansion, see e.g. \cite{letessier2002hadrons}, would be needed to improve on this situation but the details do not impact the neutrino freeze-out period near $1\MeV$ which is our primary concern, and so we do not consider these issues further here.

%%%%%%%%%%%%%%%%%%%%%%%%%%%%%%%%%%%%%%%
\begin{figure} 
\centerline{\hspace*{0.4cm}\includegraphics[height=6.6cm]{./ErasOfUniverse/z_T_plot-eps-converted-to.pdf}}
\caption{Disappearance of degrees of freedom. The Universe volume inflated approximately by a factor of 27 above the thermal red shift scale as massive particles disappeared successively from the inventory.\label{fig:dof}}
 \end{figure}
%%%%%%%%%%%%%%%%%%%%%%%%%%%%%%%%%%%%%%%



As long as the dynamics are at least approximately entropy conserving, the total drop in $R$ is entirely determined by entropy conservation. Namely, the magnitude of the drop in $R$ figure~\ref{fig:dof} is a measure of the number of degrees of freedom that have disappeared from the Universe. Consider   two times $t_1$ and $t_2$ at which all particle species that have not yet annihilated are effectively massless.  By conservation of comoving entropy and  scaling $T\propto 1/a$ we have
\begin{equation}\label{r_ratio}
1=\frac{a_1^3S_{1}}{a_2^3 S_2}=\frac{a_1^3\sum_ig_i T_{i,1}^3}{a_2^3\sum_j g_j T_{j,2}^3},\qquad \left(\frac{R_1}{R_2}\right)^3=\frac{\sum_ig_i (T_{i,1}/T_{\gamma,1})^3}{\sum_j g_j (T_{j,2}/T_{\gamma,2})^3}
\end{equation}
where the sums are over the total number of degrees of freedom present at the indicated time and the degeneracy factors $g_i$ contain the $7/8$ factor for fermions. In the second form    we divided the numerator and denominator by $a_{0}T_{\gamma,0}$. We distinguish between the temperature of each particle species and our reference temperature, the photon temperature.  This is important since today neutrinos are colder than photons, due to photon reheating from  $e^\pm$ annihilation occurring after neutrinos decoupled (this is only an approximation, a point we will study in detail in subsequent chapters).  By conservation of entropy one obtains the neutrino to photon temperature ratio of
\begin{equation}\label{T_nu_T_gamma}
T_\nu/T_\gamma=({4}/{11})^{1/3}.
\end{equation}
We will call this the reheating ratio in the decoupled limit.  For details on the derivation of this standard result, see for example our paper in appendix \ref{app:model_ind}, where it is obtained as a special case of a more general analysis. 

Using \req{r_ratio}  we  compute the total drop in $R^3$ shown in figure \ref{fig:dof}.  At $T=T_\gamma=\mathcal{O}(100\GeV)$ the number of active degrees of freedom is slightly below $g_{\rm SM}=106.75$ due to the partial disappearance of top quarks, but this approximation will be good enough for our purposes.  At this time, all the species are in thermal equilibrium with photons and so $T_{i,1}/T_{\gamma,1}=1$ for all $i$.  Today we have $2$ photon and $7/8\times 6$ neutrino degrees of freedom and a  neutrino to photon temperature ratio \req{T_nu_T_gamma}.  Therefore we have
\begin{equation}
\left(\frac{R_{100GeV}}{R_{now}}\right)^3= \frac{g_{SM}}{g_{\rm now}}=\frac{106.75}{2+\frac{7}{8}\times 6\times \frac{4}{11}}\approx 27.3
\end{equation}
which is the  fractional change we see in the fermi gas model curve in figure \ref{fig:dof} (as mentioned above, the QCD model is reduced due to interactions). The meaning of this factor is that the Universe approximately inflated by a factor 27 above the thermal red shift scale as massive particles disappeared successively from the inventory. 


\section{Composition of the Universe}
From the perspective of reheating, the history of the Universe from the end of $e^\pm$ annihilation until today has been uneventful.  We can shed additional light on this period and others by looking at the composition of the Universe as a function of temperature

%%%%%%%%%%%%%%%%%%%%%%%%%%%%%%%%%%%%
\begin{figure}
\centerline{\hspace*{0.4cm}\includegraphics[height=7.6cm]{./ErasOfUniverse/energy_densities_total-eps-converted-to.pdf}}\label{fig:energy_frac}
\caption{Current era: $69\%$ dark energy, $26\%$ dark matter, $5\%$ baryons, $<1\%$ photons and neutrinos, $1$ massless and $2\times .1$ eV neutrinos (Neutrino mass choice is just for illustration.  Other values are possible).}
 \end{figure}
%%%%%%%%%%%%%%%%%%%%%%%%%%%%%%%%%%
In figure \ref{fig:energy_frac} we begin on the right at the end of the hadron era with the disappearance of muons and pions.  This constitutes a reheating period, with energy and entropy from these particles being transfered to the remaining $e^\pm$, photon, neutrino plasma.  Continuing to $T=O(1)$ MeV, we come to the annihilation  of $e^\pm$ and the photon reheating period.  Notice that only the photon energy density fraction increases here.  As discussed above, a common simplifying assumption is that neutrinos are already decoupled at this time and hence do not share in the reheating process, leading to a difference in photon and neutrino temperatures \req{T_nu_T_gamma}.

After passing through a long period, from $T=O(1)$ MeV until $T=O(1)$ eV, where the energy density is dominated by photons and free-streaming neutrinos, we then come to the beginning of the matter dominated regime, where the energy density is dominated by dark matter and baryonic matter.  This transition is the result of the redshifting of the photon and neutrino energy, $\rho\propto T^4$, whereas for non-relativistic matter $\rho\propto a^{-3}\propto T^3$.  Note that our inclusion of neutrino mass causes the leveling out of the neutrino energy density fraction during this period, as compared to the continued redshifting of the photon energy.

Finally, as we move towards the present day CMB temperature of $T_{\gamma,0}=0.235$ meV on the left hand side, we have entered the dark energy dominated regime.  For the present day values, we have used the fits from the Planck data \cite{Planck} of  $69\%$ dark energy, $26\%$ dark matter and $5\%$ baryons (and zero spatial curvature).  The photon energy density is fixed by the CMB temperature $T_{\gamma,0}$ and the neutrino energy density is fixed by $T_{\gamma,0}$ along with the photon to neutrino temperature ratio.  Both constitute $<1\%$ of the current energy budget.


\section{Deceleration Parameter}
We conclude our overview of cosmology with one final perspective, the Universe as seen through the deceleration parameter.  The deceleration parameter is another indicator of the transition between different eras of the Universe's history.  Recall the relation \req{qparam} (for $k=0$)  between deceleration parameter and matter content of the Universe. In particular we have the regimes

\begin{itemize}
\item Radiation dominated Universe: $P=\rho/3 \implies q=1$.\\


\item  (Non-relativistic) Matter dominated Universe: $P\ll\rho \implies q=1/2$.\\



\item Dark energy ($\Lambda$) dominated Universe: $P=-\rho \implies q=-1$.\\

\end{itemize}
We use $q$ first to characterize the era from today back to the end of neutrino freeze-out and then from freeze-out until the end of the hadron era.


%%%%%%%%%%%%%%%%%%%%%%%%%%%%%%%%%%%%%%%%%%%%%%%%%%%%%%%%%%%%%%%
\subsection{Back in time to Neutrino Freeze-out}\label{recomb}
In the following we use the mix of matter  (31\%) and dark energy (69\%) with photon and neutrino backgrounds favored by the latest Planck results \cite{Planck}, where we gave two neutrino species mass of $m_\nu=30\meV$ and a third neutrino remains  massless.  This is a different mass value than used above and again, it is only for illustration-- other mass choices are possible within present day constraints and will impact to some degree where exactly matter dominance emerges from the radiative Universe.  We presume  that neutrino kinetic freeze-out completed before the onset of $e^\pm$-annihilation into  photons, leading to the neutrino to photon temperature ratio \req{T_nu_T_gamma}. Again, this is a common simplifying assumption.  Much of the remainder of this work will involve improving on this approximation, but for the purposes of this overview it is sufficient.

Figure \ref{fig:today} shows in the left frame the temperature  (left axis) and deceleration parameter (right axis)  from shortly after the completion neutrino freeze-out until today.  The horizontal dot-dashed lines show  the pure radiation-dominated value of $q=1$ and the matter-dominated value of $q=1/2$. The expansion in this era starts off as radiation-dominated, but transitions to matter-dominated starting around $T=\mathcal{O}(10\eV)$ and begins to transition to a dark energy dominated era at $T=\mathcal{O}(1\meV)$. We are still in the midst of this transition today. The vertical dot-dashed lines show  the time of recombination at $T\simeq0.25\eV$, when the Universe became transparent to photons, and reionization at $T\simeq {\cal O}(1\meV)$, when hydrogen in the Universe was again ionized due to light from the first galaxies \cite{Zaroubi:2012in}. 

On the right in figure  \ref{fig:today}  we show the Hubble parameter $H$ and redshift $z+1\equiv a_0/a(t)$. We can see in figure \ref{fig:today} a visible deviation from power law behavior due to the transitions from radiation to matter dominated and from matter to dark energy dominated expansion.  These transitions are accentuated and more easily visualized in the form of the deceleration parameter $q$. The time span covered by the figure  \ref{fig:today} is in essence the entire lifespan of the Universe, but of course on a logarithmic time scale there is a lot of room for interesting physics in the tiny blip that happened beforehand.


%%%%%%%%%%%%%%%%%%%%%%%%%%%%%%%%%%%%%%%
\begin{figure}
\begin{minipage}{\linewidth}
\makebox[0.5\linewidth]%
{\includegraphics[keepaspectratio=true,scale=0.52]{./ErasOfUniverse/T_q_today-eps-converted-to.pdf}}
\makebox[0.5\linewidth]%
{\includegraphics[keepaspectratio=true,scale=0.52]{./ErasOfUniverse/H_z_today-eps-converted-to.pdf}}
\caption{Transition periods in the composition of the Universe: on left -- evolution of temperature $T$  and deceleration parameter $q$; on right --  evolution of the Hubble parameter $H$ and redshift $z$.
\label{fig:today} }
\end{minipage}
\end{figure}
%%%%%%%%%%%%%%%%%%%%%%%%%%%%%%%%%%%%%%%

%%%%%%%%%%%%%%%%%%%%%%%%%%%%%%%%%%%%%%%%%%%%%%%%%%%%%%%%%%%%%%%
\subsection{Neutrino Freeze-out Era }\label{nudecoup}
%%%%%%%%%%%%%%%%%%%%%%%%%%%%%%%
The era separating the photon-neutrino-matter-dark energy Universe we just described from the end of the hadron Universe is quite complex in its evolution.   We begin when the number of $e^\pm$-pairs has decayed to the same abundance as the number of baryons in the Universe at the temperature  $T=\mathcal{O}(10\keV)$ and reach back to $T={\cal O}(30\MeV)$ where muons and pions are disappearing from the Universe.

%%%%%%%%%%%%%%%%%%%%%%%%%%%%%%%%%%%%%%%
\begin{figure}
\begin{minipage}{\linewidth}
\makebox[0.5\linewidth]%
{\includegraphics[keepaspectratio=true,scale=0.54]{./ErasOfUniverse/T_q_BBN-eps-converted-to.pdf}} 
\makebox[0.5\linewidth]%
{\includegraphics[keepaspectratio=true,scale=0.54]{./ErasOfUniverse/H_z_BBN-eps-converted-to.pdf}} 
\caption{From the end of baryon antimatter annihilation through BBN, see figure \ref{fig:today}.%  on left -- evolution of temperature $T$  and deceleration parameter $q$; on right --  evolution of the Hubble parameter $H$ and redshift $z$.
\label{fig:BBN}  }
\end{minipage}
\end{figure}
%%%%%%%%%%%%%%%%%%%%%%%%%%%%%%%%%%%%%%%

 In figure~\ref{fig:BBN} the horizontal dot-dashed line for $q=1$  shows the pure radiation dominated value with two exceptions. First, the presence of massive pions  and muons reduce  the value of $q$ near to the maximal temperature shown.  Second, when the temperature is near the value of the electron mass, the $e^\pm$-pairs are not yet fully depleted but already sufficiently non-relativistic to cause another dip in $q$.  These are not large drops; the expansion is still predominately radiation dominated.  But $q$ provides a sensitive measure of when various mass scales become relevant and is a good indicator of the presence of a reheating period.

 The dashed line shows the neutrino temperature, which decouples from the $e^\pm$ and photon temperature at $T={\cal O}(1\MeV)$ when neutrinos freeze-out and begin free streaming. In figure~\ref{fig:BBN} the unit of time is seconds and the range spans the domain from fractions of a millisecond to a few hours. After neutrino freeze-out we come to Big Bang Nucleosynthesis, the period when the lighter elements were synthesized in a hot but relatively dilute plasma \cite{Iocco:2008va}. We left some time gap between this and the domain shown in figure \ref{fig:today}  describing the current era -- there is an uneventful evolution between the two domains. 

\section{Focusing on Neutrino Freeze-out}
Neutrino freeze-out is, as far as we know, the unique era in the history of the Universe when a significant matter fraction froze out at the same time that a reheating period was beginning, namely the start of $e^\pm$ annihilation.  It is this coincidence that makes neutrino freeze-out a rich and complicated period to study as compared to the many other reheating periods in the history of the Universe. This period has been studied before \cite{Madsen,Dolgov_Hansen,Gnedin,Esposito2000,Mangano2002,Mangano2005}, but the Planck satellite results \cite{Planck} motivate a reinvestigation of this period of cosmology.  We therefore make the interplay of neutrino freeze-out and reheating from $e^\pm$ annihilation the primary focus of the remainder of this work.

\begin{subappendices}
\section{Conventions}\label{app:conventions}

There are several sign conventions in use in general relativity.  As discussed in \cite{hobson}, these conventions differ by the sign factors $S1$, $S2$, $S3$, which appear in the following objects:
\vspace{3mm}

Metric Signature: $\eta^{\mu\nu}=(S1)\text{Diag}(-1,1,1,1)$
\vspace{3mm}

Riemann Tensor: $R^\mu_{\alpha\beta\gamma}=(S2)(\partial_{\beta}\Gamma^\mu_{\alpha\gamma}-\partial_{\gamma}\Gamma^\mu_{\alpha\beta}+\Gamma^\mu_{\sigma\beta}\Gamma^\sigma_{\gamma\alpha}-\Gamma^\mu_{\sigma\gamma}\Gamma^\sigma_{\beta\alpha})$
\vspace{3mm}

Einstein Equation: $G_{\mu\nu}-(S3)\Lambda g_{\mu\nu}=(S3)8\pi G_NT_{\mu\nu}$
\vspace{3mm}

Ricci Tensor: $R_{\mu\nu}=(S2)(S3)R^\alpha_{\mu\alpha\nu}$
\vspace{3mm}

The sign $S3$ comes from the choice of what index is contracted in forming the Ricci tensor.  Since that sign factor appears in both $R_{\mu\nu}$ and $R$ it affects the overall sign of $G_{\mu\nu}$ and therefore Einstein's equation as shown above. In this dissertation we will use the $(-,+,-)$ convention.


\end{subappendices}

