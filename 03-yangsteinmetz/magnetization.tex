%~~~~~~~~~~~~~~~~~~~~~~~~~~~~~~~~~~~~~~~~~~~~~~~~~

% \subsection{ Electron-positron plasma in the early Universe}\label{section_electron}
% %In this section we will focus on the following:
% %\begin{itemize}
% %    \item Chemical potential of electron in early universe
% %    \item Electron-positron plasma in BBN (Damped sccreening)
% %    \item Electron-positron magnetization
% %%    \item Neutron Lifespan in magnetized electron/positron plasma.
% %\end{itemize}

% In the early Universe, after the neutrino freeze-out at $T\approx 2$\,MeV, the Universe is controlled by the electron-positron-photon plasma. In this section, we demonstrate the rich electron-positron plasma in the early Universe by examining the chemical potential $\mu_e$ in the charge-neutral and entropy-conserving Universe. We study the  microscope collision property of electron-positron plasma and explore the spin response of the electron-positron plasma to external and self-magnetization fields, thus developing methods for future detailed study.

%In this section, we will quantify the dynamical picture of $e^\pm$ plasma and show that the $e^+$ abundance can persist in early universe at relatively low temperature $T = 20$ keV which provide the dense $e^\pm$ plasma environment for the big-bang nucleosynthesis (BBN) in the early universe. 

%The role of electron-positron plasma has not received the appropriate attention in the days of precision big bang nucleosynthesis studies. The standard BBN model indicates that the synthesis of light elements typically takes place at temperatures around  $86\,\mathrm{keV}>T_{BBN}>50\,\mathrm{keV}$~\cite{Pitrou:2018cgg}. Within this temperature range there are millions of electron-positron pairs per charged nucleon, providing an electron-positron-rich plasma environment for nucleosynthesis. Furthermore, the electron-positron densities can reach millions of times normal atomic densities. The presence of  these $e\bar e$-pairs before and during BBN has been acknowledged by Wang, Bertulani and Balantekin~\cite{Wang:2010px} nearly a decade ago.





%On the other hand, the Universe today filled with magnetic fields at various scales and strengths both within galaxies and in deep extra-galactic space.It is currently unknown the origin for these magnetic fields today. In early Universe when temperature $T>20$ keV , we have dense $e^\pm$ plasma. The significant magnetic moments of electrons and positrons also provide opportunities to investigate spin magnetization process.


%~~~~~~~~~~~~~~~~~~~~~~~~~~~~~~~~~~~~~~~~~~~~~~~~~~~~~~~~~~~~~~~~~~~~~~~~~

% \subsubsection{Electron chemical potential in the early Universe}\index{electron chemical potential}
% In this section, we derive the dependence of electron chemical potential, and hence $e^\pm$ density, on the photon background temperature by employing the following physical principles:
% \begin{enumerate}
% \item Charge neutrality of the Universe:\index{charge neutrality}
% \begin{align}%\label{neutrality}
% n_e-n_{\overline{e}}=n_p-n_{\overline{p}}\approx\,n_p,
% \end{align}
% where $n_e$ and $n_{\overline{e}}$ denotes the number density of electron and positron.
% \item Neutrinos decouple (freeze-out) at a temperature $T_f\simeq 2$ MeV, after which they free stream through the Universe with an effective temperature~\cite{Birrell:2012gg}
% \begin{align}
% T_\nu(t)=T_f a(t_f)/a(t),
% \end{align}
%  where $a(t)$ is the FLRW Universe scale factor.
% \item Total comoving entropy is conserved. At $T\leq T_f$ the dominant contributors to entropy are photons, $e^\pm$, and neutrinos.
% In addition, after neutrino freeze-out, neutrino comoving entropy is independently conserved ~\cite{Birrell:2012gg}. This  implies that the combined comoving entropy in $\gamma$, $e^\pm$ is also conserved for $T_\gamma\leq T_f$.
% \end{enumerate}

% Motivated by the fact that comoving entropy in $\gamma$, $e^\pm$ is conserved after neutrino freeze-out, we rewrite the charge neutrality condition, Eq.(\ref{neutrality}) in the form
% \begin{align}%\label{charge_neutral_cond2}
% n_e-n_{\overline{e}}=X_p\frac{n_B}{s_{\gamma,e,\overline{e}}} s_{\gamma,e,\overline{e}},\qquad X_p\equiv\frac{n_p}{n_B},
% \end{align}
% where $n_B$ is the number density of baryons, and $s_{\gamma,e,\overline{e}}$ is the combined entropy density in photons, electrons, and positrons. During the Universe expansion, the comoving entropy and baryon number are conserved quantities, hence the ratio $n_B/s_{\gamma,e,\overline{e}}$ is conserved. We have
% \begin{align}
% \frac{n_B}{s_{\gamma,e,\overline{e}}}=\left(\frac{n_B}{s_{\gamma,e,\overline{e}}}\right)_{t_0}\!\!\!\!=\left(\frac{n_B}{s_{\gamma}}\right)_{t_0}\!\!\!\!=\left(\frac{n_B}{n_\gamma}\right)_{t_0}\left(\frac{n_\gamma}{s_{\gamma}}\right)_{t_0},
% \end{align}
% where the subscript $t_0$ denotes the present day value, and the second equality is obtained by observing that the present day $e^\pm$-entropy density is negligible compared to the photon entropy density. We can evaluate the ratio by giving the present day baryon-to-photon ratio: $n_B/n_\gamma= 6.05\times10^{-10}$(CMB) ~\cite{ParticleDataGroup:2022pth} and the entropy per particle for a massless boson:$(s/n)_{\mathrm{boson}}\approx 3.602$~\cite{Letessier:2002ony}.

% The total entropy density of photons and electron/positron can be written as
% \begin{align}%\label{entropy_per_baryon}
% s_{\gamma,e,\overline{e}}=\frac{2\pi^2}{45}g_\gamma\,T_\gamma^3+\frac{\rho_{e,\overline{e}}+P_{e,\overline{e}}}{T_\gamma}-\frac{\mu_e}{T_\gamma}(n_e-n_{\overline{e}}),
% \end{align}
% where $ \rho_{e,\overline{e}}=\rho_{e}+\rho_{\overline{e}}$ and $P_{e,\overline{e}}=P_{e}+P_{\overline{e}}$ are the total energy density and pressure of electrons/positron respectively.
% The energy density and pressure in electrons and positrons are given by
% \begin{align}\label{rho_e}
% \frac{\rho_{e,\overline{e}}}{T_\gamma^4}=\frac{g_e}{2\pi^2}M_e^4 \bigg[&\int_{1}^\infty \frac{ u^2\sqrt{ u^2-1} du}{\exp(M_e u-b_e)+1}+\int_{1}^\infty \frac{ u^2\sqrt{ u^2-1} du}{\exp(M_e u+b_e)+1}\bigg]\,,
% \end{align}
% and
% \begin{align}\label{P_e}
% \frac{P_{e,\overline{e}}}{T_\gamma^4}=\frac{g_e}{6\pi^2}M_e^4\bigg[&\int_{1}^\infty   \frac{(u^2-1)^{3/2} du}{\exp(M_e u-b_e)+1}+\int_{1}^\infty   \frac{(u^2-1)^{3/2} du}{\exp(M_e u+b_e)+1}\bigg],
% \end{align}
% where we introduce the dimensionless variables as follows: 
% \begin{align}%\label{Variables}
% u=\frac{E}{m_e},\qquad M_e=\frac{m_e}{T_\gamma},\qquad b_e=\frac{\mu_e}{T_\gamma}.
% \end{align}

% By incorporating Eq.(\ref{charge_neutral_cond2}) and Eq.(\ref{entropy_per_baryon}), the charge neutrality condition can be expressed as
% \begin{align}%\label{charge_neutral_cond3}
% &\left[1+X_p\left(\frac{n_B}{n_\gamma}\right)_{t_0}\left(\frac{n_\gamma}{s_{\gamma}}\right)_{t_0}\frac{\mu_e}{T_\gamma}\right]\frac{n_e-n_{\overline{e}}}{T_\gamma^3}\notag\\
% &\qquad\qquad\qquad=X_p\left(\frac{n_B}{n_\gamma}\right)_{t_0}\left(\frac{n_\gamma}{s_{\gamma}}\right)_{t_0} \left(\frac{2\pi^2}{45}g_\gamma+\frac{\rho_{e,\overline{e}}+P_{e,\overline{e}}}{T_\gamma^4}\right).
% \end{align}
% Using the Fermi distribution, the number density of electrons over positrons in the early Universe is given by
% \begin{align}%\label{ee_density}
% n_e-n_{\overline{e}}&=\frac{g_e}{2\pi^2}\left[\int_0^\infty\frac{p^2dp}{\exp{\left((E-\mu_e)\right)/T_\gamma}+1}\right.\left.-\int_0^\infty\frac{p^2dp}{\exp{\left((E+\mu_e)/T_\gamma\right)}+1}\right]\notag\\
% &=\frac{g_e}{2\pi^2}{T_\gamma^3}\tanh(b_e)M_e^3\int_{1}^\infty \!\!\!\!\frac{  u \sqrt{u^2-1} du}{1+\cosh(M_eu)/\cosh(b_e)}.
% \end{align}
% Substituting Eq.(\ref{ee_density}) into Eq.(\ref{charge_neutral_cond3}) and giving the value of $X_p$, the charge neutrality condition can be solved to determine $\mu_e/T_\gamma$ as a function of $M_e$ and $T_\gamma$. 
% %Fig~~~~~~~~~~~~~~~~~~~~~~~~~~~~~~~~~~~~~~~~~~~~~~~~~~~~~
% \begin{figure}[ht]
% \begin{center}
% \includegraphics[width=\linewidth]{./plots/May152023_EPDensity_Chemical}
% \caption{\cccite{Grayson:2023flr}, adapted from Ref.~\cite{Grayson:2023flr} and thesis of C.T.Yang \cite{Yang:2024ret}. Left axis: The chemical potential of an electron as a function of photon temperature $T=T_\gamma$ with $X_p=0.878$ and $n_B/n_\gamma=6.05\times10^{-10}$. Right axis: the ratio of electron(positron) number density to baryon density as a function of temperature. The blue solid line is the electron density, the red dashed line is the positron density, and the green dotted line is the number density with $\mu_e=0$. We found that when electron chemical potential $\mu_e\approx T=0.02\,\mathrm{MeV}$ the positron density decreases because of the annihilation.}
% %\label{BBN_Electron}
% \end{center}
% \end{figure}
% %~~~~~~~~~~~~~~~~~~~~~~~~~~~~~~~~~~~~~~~~~~~~~~~~~~~~~

% In Fig.~\ref{BBN_Electron} (left axis) we solve Eq.(\ref{charge_neutral_cond3}) numerically and plot the electron chemical potential as a function of temperature with the following parameters: proton concentration $X_p=0.878$ from 
%  observation~\cite{ParticleDataGroup:2022pth} and  $n_B/n_\gamma=6.05\times10^{-10}$ from CMB. We can see the value of chemical potential is comparatively small $\mu_e/T\approx10^{-6}\sim10^{-7}$ during the BBN temperature range, implying an equal number of electrons and positrons in plasma. From the ratio of electron (positron) number density to baryon density in Fig.~\ref{BBN_Electron} (right axis) we can see that during the accepted BBN temperature range the Universe was filled with an electron-positron rich plasma.
% For example when the temperature is around $T=70\,\mathrm{keV}$ the density of electrons and positrons is comparatively large in the early Universe $n_{e^\pm}\approx10^7\,n_B$. Later when the temperature is around $T=20.3\,\mathrm{keV}$, the positron density decreases, leading to the transformation of the pair plasma to an electron-proton plasma.
% %~~~~~~~~~~~~~~~~~~~~~~~~~~~~~~~~~~~~~~~~~~~~~~~~~
% \subsubsection{Microscope damping rate of electron-positron plasma}\label{relax}
% In electron-positron plasma, the major reactions between photons and $e^+e^-$ pairs are inverse Compton scattering, M{\o}ller scattering, and Bhabha scattering:
% \begin{align}
% &e^\pm+\gamma\longrightarrow e^\pm+\gamma,\qquad e^\pm+e^\pm\longrightarrow e^\pm+e^\pm,\qquad e^\pm+e^\mp\longrightarrow e^\pm+e^\mp.
% \end{align}
% The general formula for thermal reaction rate per volume is discussed in~\cite{Letessier:2002ony} (Eq.(17.16), Chapter 17). For inverse Compton scattering we have
% \begin{align}
% R_{e^{\pm}\gamma}=\frac{g_eg_\gamma}{16\left(2\pi\right)^5}T\int_{m_e^2}^\infty\!\!\!\!ds\frac{K_1(\sqrt{s}/T)}{\sqrt{s}}\int^0_{-(s-m_e^2)^2/s}\!\!\!\!\!\!\!\!\!\!\!\!\!\!\!\!dt\, |M_{e^{\pm}\gamma}|^2,
% \end{align} 
% and for M{\o}ller and Bhabha reactions we have
% \begin{align}
% &R_{e^\pm e^\pm}=\frac{g_eg_e}{16\left(2\pi\right)^5}T\!\!\int_{4m_e^2}^\infty\!\!\!\!ds\frac{K_1(\sqrt{s}/T)}{\sqrt{s}}\int^0_{-(s-4m_e^2)}\!\!\!\!\!\!\!\!\!\!\!\!\!\!\!\!dt\,|M_{e^\pm e^\pm}|^2,\\
% &R_{e^\pm e^\mp}=\frac{g_eg_e}{16\left(2\pi\right)^5}T\!\!\int_{4m_e^2}^\infty\!\!\!\!ds\frac{K_1(\sqrt{s}/T)}{\sqrt{s}}\int^0_{-(s-4m_e^2)}\!\!\!\!\!\!\!\!\!\!\!\!\!\!\!\!dt\,|M_{e^\pm e^\mp}|^2,
% \end{align}
% where $g_i$ is the degeneracy of particle $i$, $|M|^2$ is the matrix element for a given reaction, $K_1$ is the Bessel function of order $1$, and $s,t,u$ are Mandelstam variables. The leading order matrix element associated with inverse Compton scattering can be expressed in the Mandelstam variables~\cite{Kuznetsova:2011wt,Kuznetsova:2009bq} we have\index{inverse Compton scattering }
% \begin{align}
% |M_{e^\pm\gamma}|^2\!=32 \pi^2\alpha^2\bigg[&4\left(\frac{m_e^2}{m_e^2-s}+\frac{m_e^2}{m_e^2-u}\right)^2\notag\\
% &\qquad\qquad-\frac{4m_e^2}{m_e^2-s}-\frac{4m_e^2}{m_e^2-u} -
%  \frac{m_e^2-u}{m_e^2-s} -\frac{m_e^2-s}{m_e^2-u}\bigg],
% \end{align}
% and for M{\o}ller and Bhabha scattering we have \index{M{\o}ller scattering}\index{Bhabha scattering}
% \begin{align}
% |M_{e^{\pm}e^{\pm}}|^{2}\!=64\pi^{2}\alpha^{2}\bigg[&
% \frac{s^{2}+u^{2}+8m_e^{2}(t-m_e^{2})}{2(t-m^2_{\gamma})^{2}}\notag\\
% &\quad+\frac{{s^{2}+t^{2}}+8m_e^{2}
% (u-m_e^{2})}{2(u-m_{\gamma}^2)^{2}} + \frac{\left( {s}-2m_e^{2}\right)\left({s}-6m_e^{2}\right)}
% {(t-m_{\gamma}^2)(u-m_{\gamma}^2)} \bigg],
% \end{align}
% and
% \begin{align}
% |M_{e^\pm e^\mp}|^{2}=64\pi^{2}\alpha^{2}
% \bigg[&\frac{s^{2}+u^{2}+8m_e^{2}(t-m_e^{2})}{2(t-m^2_{\gamma})^{2}}\notag\\
% &\quad+\frac{u^{2}+t^{2}+8m_e^{2}
% (s-m_e^{2})}{2(s-m^2_{\gamma})^{2}}  +   \frac{\left({u}-2m_e^{2}\right)\left({u}-6m_e^{2}\right)}
%    {(t-m^2_{\gamma})(s-m^2_{\gamma})} \bigg],
% \label{M_fi_b}
% \end{align}
% where we introduce the photon mass $m_\gamma$ to account the plasma effect and avoid singularity in reaction matrix elements. 

% The photon mass $m_\gamma$ in plasma is equal to the plasma frequency $\omega_p$, where we have~\cite{Kislinger:1975uy}\index{photon mass}
% \begin{align}
% m^2_\gamma=\omega^2_{p}=8\pi\alpha\int\frac{d^3p_e}{(2\pi)^3}\left(1-\frac{p_e^2}{3E_e^2}\right)\frac{f_e+f_{\bar e}}{E_e},
% \end{align}
% where $E_e=\sqrt{p_e^2+m^2_e}$. In the BBN temperature range $86\,\mathrm{keV}>T_{BBN}>50\,\mathrm{keV}$ we have $m_e\gg T$ and considering the nonrelativistic limit for electron-positron plasma, we obtain
% \begin{align}
% m^2_\gamma=\frac{4\pi\alpha}{2m_e}\left(\frac{2m_eT}{\pi}\right)^{3/2}e^{-m_e/T}\cosh\left(\frac{\mu_e}{T}\right).
% \end{align}
% In the BBN temperature range, we have $\mu_e/T\ll1$, which implies the equal number of electrons and positrons in plasma.

% To discuss the collisional plasma by the linear response theory, it is convenient to define the average relaxation rate for the electron-positron plasma as follows:\index{electron-positron plasma!damping rate}
% \begin{align}\label{Kappa}
% \kappa=\frac{R_{e^\pm e^\pm}+R_{e^\pm e^\mp}+R_{e^\pm\gamma}}{\sqrt{n_{e^-}n_{e^+}}}\approx\frac{R_{e^\pm e^\pm}+R_{e^\pm e^\mp}}{\sqrt{n_{e^-}n_{e^+}}},
% \end{align}
% where the density function ${\sqrt{n_{e^-}n_{e^+}}}$ in the Boltzmann limit is given by
% \begin{align}
% {\sqrt{n_{e^-}n_{e^+}}}=\frac{g_e}{2\pi^3}T^3\left(\frac{m_e}{T}\right)^2K_2(m_e/T).
% \end{align}
% In Fig.~\ref{RelaxationRate_fig}, we show the reaction rates for M{\o}ller reaction, Bhabha reaction, and inverse Compton scattering as a function of temperature. For temperatures $T>12.0$ keV, the dominant reactions in plasma are M{\o}ller and Bhabha scatterings between electrons and positrons. Thus in the BBN temperature range, we can neglect the inverse Compton scattering. The total relaxation rate is approximately constant $\kappa=10\sim12$ keV during the BBN. For $T<20.3$ keV the relaxation rate $\kappa$ decreases rapidly because of positron annihilation. At this temperature, the composition of plasma begins to change from an electron-positron plasma to an electron-baryon plasma.
% %~~~~Figure~~~~~~~~~~~~~~~~~~~~~~~~~~~
% \begin{figure}[ht]
% \begin{center}
% %\includegraphics[width=0.95\linewidth]{KappaRateToT_May082023}
% \includegraphics[width=\linewidth]{./plots/May152023Kappa_EPPlasma}
% \caption{\cccite{Grayson:2023flr}, adapted from Ref.~\cite{Grayson:2023flr} and thesis of C.T.Yang \cite{Yang:2024ret}. The relaxation rate $\kappa$ as a function of temperature in nonrelativistic electron-positron plasma from \cite{Grayson:2023flr}. For comparison, we show  reaction rates  for M{\o}ller reaction $e^-+e^-\to e^-+e^-$ (blue line), Bhabha reaction $e^-+e^+\to e^-+e^+$ (red line), and inverse Compton scattering $e^-+\gamma\to e^-+\gamma$ (green line) respectively. It shows that the dominant reactions during BBN are the M{\o}ller and Bhabha scatterings between electrons and positrons. The total relaxation rate Eq.(\ref{Kappa}) is shown in the black line. It shows that we have $\kappa=10\sim12$ keV during the BBN temperature range. For comparison, the Debye mass $m_D=\omega_{p}\sqrt{m_e/T}$(purple line) is shown as a function of temperature.
% }
% \label{RelaxationRate_fig}
% \end{center}
% \end{figure}
% %~~~~Figure~~~~~~~~~~~~~~~~~~~~~~~~~~~


% \paragraph{From static to damped dynamic screening}

% At present, the observation of light element (e.g. D, $^3$He, $^4$He, and $^7$Li) abundances produced in Big-Bang nucleosynthesis (BBN) offers a reliable probe of the early Universe before the recombination. Much effort of the BBN study is currently being made to reconcile the discrepancies and tensions between theoretical predictions and observations of light element abundances, e.g. $^7$Li problem ~\cite{Pitrou:2018cgg,Fields:2011zzb}.
% Current models assume that the Universe was essentially void of anything but reacting light nucleons and electrons needed to keep the local baryon density charge-neutral, a situation similar to the experimental environment where empirical nuclear reaction rates are obtained.

% The electron-positron plasma influences light element abundances through electromagnetic screening of the nuclear potential. The electron cloud surrounding the charge of an ion screens other nuclear charges far from its own radius and reduces the Coulomb barrier. In nuclear reactions, the reduction of Coulomb barrier makes the penetration probability easier and enhance the thermonuclear reaction rates. In this case, the modification of the nuclei interaction due to the plasma screening effect may plays a key role in the formation of light element in the BBN. 

% The enhancement factor of thermonuclear reaction rates and screening potential are calculated by Salpeter in 1954~\cite{Salpeter:1954nc}, which describes the static screening effects for the thermonuclear reactions. In an isotropic and homogeneous plasma the Coulomb potential of a point-like particle with charge $Ze$ at rest is modified into~\cite{Salpeter:1954nc}
% \begin{align}
% \phi_\text{stat}(r)=\frac{Ze}{4\pi\epsilon_0 r}e^{-m_Dr},
% \end{align}
% where $m_D$ is the Debye mass. After that it has been exploited widely in BBN for static screening ~\cite{1969ApJ...155..183S,Famiano:2016hhs}. 

% Subsequently, the study of dynamical screening for moving ions has been taken into account~\cite{1988ApJ...331..565C,Gruzinov:1997as,Hwang:2021kno}. When a test charge moves with a velocity that is enough to react with the background charge in plasma, the Coulomb potential is modified by the dynamical effect. However, the applications focus on the weakly interacting electron-positron plasma only. 

% In our separate work~\cite{Grayson:2023flr} we use the linear response theory adapted by C.Grayson to to describe the inter nuclear potential in electron-positron plasma during BBN. We improve the prior efforts by evaluation and inclusion of the collision damping rate due to scattering in the dense plasma medium and provide an approximate analytic formula that can be readily used to estimate the effect of screening on internuclear potential. For comprehensive discussion and the application of the damped dynamic screening see~\cite{Grayson:2023flr}.










% %~~~~~~~~~~~~~~~~~~~~~~~~~~~~~~~~~~~~~~~~~~~~~~~~~~~~
\subsubsection{Magnetization of the electron-positron plasma (ANDREW WORKING HERE)}

In the present-day Universe, we have magnetic fields~\cite{Giovannini:2003yn,Kronberg:1993vk} at various scales and strengths both within galaxies and in deep extra-galactic space far away from matter sources. Current observations suggest the upper and lower bounds for the Extra-Galactic Magnetic Field (EGMF) are given by~\cite{neronov2010evidence,taylor2011extragalactic,pshirkov2015new,jedamzik2019stringent,vernstrom2021discovery}
\begin{align}
    \label{egmf}
    10^{-8}{\mathrm G}>B_{\mathrm{EGFM}}>10^{-16}{\mathrm G}\,.
\end{align}
The origin for EGMF today is currently unknown; different models are considered in lectures~\cite{Widrow:2011hs,Vazza:2021vwy}. In our work~\cite{Rafelski:2023emw}, we investigate the hypothesis that the observed EGMF are primordial in nature, predating even the recombination
epoch. Under this hypothesis, the first best candidate is the electron-positron plasma. This is because for the temperature range $ 200\,\mathrm{keV} > T > 20$ keV, we still have relatively large quantity of both $e^\pm$ in the the early Universe plasma. In addition, electrons and positrons have the largest magnetic moments in nature, are likely to have been magnetized in the early Universe due to spin orientation. These  provide the possibility origins for a primordial magnetic field.

As the Universe undergoes the isentropic expansion,  the temperature gradually decreases as $T\propto1/a(t)$, where $a(t)$ represents the scale factor. The assumption is made that the magnetic flux is conserved over comoving surfaces, implying that the primordial relic field is expected to dilute as $B\propto1/a(t)^{2}$~\cite{Rafelski:2023emw}. Combining these cosmological redshift relations, we can introduce a dimensionless cosmic magnetic scale that remains unchanged during the evolution of the Universe 
\begin{align}
    %\label{tbscale}
    b \equiv\frac{e{B}}{T^{2}}=\left(\frac{e{B}}{T^{2}}\right)_{t_0}=b_0={\rm\ const.}\qquad10^{-3}>b_{0}>10^{-11}\,.
\end{align}
The upper and lower bounds for $b_0$ are estimated by using the present day EGMF observations Eq.~(\ref{egmf}) and the present CMB temperature $T_{0}=2.7\,\mathrm{K}\approx2.3\times10^{-4}$ eV~\cite{aghanim2018planck}.
As $b_0$ is a constant of expansion, this means the contemporary small bounded values of may have once represented large magnetic fields in the early Universe and require detailed study in a different epoch of the Universe. Therefore, correctly describing the dynamics of this $e^{\pm}$ plasma is of interest when considering the origin of extra-galactic magnetic fields (EGMF). 

In the following,  we will demonstrate that fundamental quantum statistical analysis can lead to further insights on the behavior of magnetized plasma, and show that the $e^\pm$ plasma is overall paramagnetic and yields a positive overall magnetization, which
is contrary to the traditional assumption that matter-antimatter plasma lack significant magnetic responses. For more detailed discussion  of electron-positron plasma magnetization, please see~\cite{Steinmetz:2023nsc}.



\paragraph{Electron-positron partition function}
To study the statistical behavior of the $e^\pm$ system in a magnetic field, we utilize the general Fermion partition function~\cite{Elze:1980er}
\begin{align}
 \label{PartFunc} \ln\mathcal{Z}=\sum_{\alpha}\ln\left(1+e^{-\beta(E-\eta)}\right)\,,
\end{align}
where $\beta=1/T$, $\alpha$ is the set of all quantum numbers in the system, and $\eta$ is the generalized chemical potential. In the case of a magnetized $e^{\pm}$ system, we consider it as a system of four quantum species: Particles and antiparticles, and spin aligned and anti-aligned. Taken together, we consider a system where electrons and positrons can be spin aligned or anti-aligned with the magnetic field $B$ and the partition function of the system can be written as\index{electron-positron plasma! partition function}
\begin{align}\label{PartFuncB}
%&\ln\mathcal{Z}_{tot}=&\frac{2eBV}{(2\pi)^2}\sum_{\sigma}^{\pm1}\sum_{s}^{\pm1/2}\sum_{n=0}^\infty\int^\infty_{0}dp_z\left[\ln\left(1+\Upsilon_{\sigma}^{s}(x)e^{-\beta E_{n}^{s}}\right)\right]\,\\
\ln\mathcal{Z}_{tot}=\frac{2eBV}{(2\pi)^2}\sum_{\sigma}^{\pm1}\sum_{s}^{\pm1/2}\sum_{n=0}^\infty\int^\infty_{0}dp_z\left[\ln\left(1+\Upsilon(x)e^{(\sigma\eta_{e}+s\eta_s)/T}e^{-\beta E_{n}^{s}}\right)\right]\,,
\end{align}
where $n$ is the principle quantum number for the Landau levels. The parameter $\eta_{e}$ is the electron chemical potential and $\eta_s$ is the spin chemical potential~\cite{Steinmetz:2023nsc}. The parameter $\Upsilon(x)$ is the fugacity of the Fermi gas. In this thesis we will focus on the case $\Upsilon(x)=1$ and $\eta_s=0$ , 
we leave the general case $\Upsilon(x)\neq1$ and $\eta_s\neq0$ for future work.



In general, $\Upsilon=1$ represents the maximum entropy and corresponds to the normal Fermi distribution. The deviation of $\Upsilon\neq1$ represents the configurations of reduced entropy without pulling the system off a thermal temperature. This scenario is well studied for quarks in QGP. The situation for $e^\pm$ plasma is similar to the case of the quarks during QGP, but instead here the deviation is spatial rather than temporal. Inhomogeneity can arise from the influence of other forces on the gas such as gravitational forces. This is precisely the kind of behavior that may arise in the $e^{\pm}$ epoch as the dominant photon thermal bath keeps the Fermi gas in thermal equilibrium while spatial inequilibrium could spontaneously develop. 


In the following, we will retain $\Upsilon(x)=1$ and consider the case $\eta_s/T\ll1$ for the first approximation. Then the partition function becomes
\begin{align}
\ln\mathcal{Z}_{tot}=\frac{2eBV}{(2\pi)^2}\sum_{s}^{\pm1/2}\sum_{n=0}^\infty\int^\infty_{0} \!\!dp_z\bigg[\ln\left(1+e^{-\beta(E_{n}^s-\eta_e)}\right)+\ln\left(1+e^{-\beta(E_{n}^s+\eta_e)}\right)\bigg].
\end{align}
Considering the $e^\pm$ plasma in a uniform magnetic field $B$ pointing along the $z$-axis, the energy $E_{n}^\pm$ of electron/positron system can be written as~\cite{Rafelski:2023emw}
\begin{align}
&E_{n}^\pm=\sqrt{p^2_z+\tilde m^2_\pm+2eBn},\qquad\tilde{m}^2_\pm=m^2_e+eB\left(1\mp\frac{g}{2}\right)\,,
\end{align}
where the $\pm$ script refers to spin aligned and anti-aligned eigenvalues. The parameter $g$ is the gyro-magnetic ($g$-factor) of the particle. 

To simplify the partition function, we consider the expansion of the logarithmic function as follows:
\begin{align}
\ln\left(1+x\right)=\sum^{\infty}_{k=1}\frac{(-1)^{k+1}}{k}x^k, \,\,\,\,\,\,\,\mathrm{for}\,|x|<1.
\end{align}
Then the partition function of electron/positron system can be written as
\begin{align}
\ln\mathcal{Z}_{tot}=&\frac{2eBV}{(2\pi)^2}\sum_{n=0}^\infty\int^\infty_{0} \!\!dp_z\sum^{\infty}_{k=1}\frac{(-1)^{k+1}}{k}\bigg[e^{k\beta\mu_e}+e^{-k\beta\mu_e}\bigg]e^{-k\beta E_n^\pm}\notag\\
&=\frac{2eBV}{(2\pi)^2}\sum_{n=0}^\infty\sum^{\infty}_{k=1}\frac{(-1)^{k+1}}{k}\bigg[2\cosh{(k\beta\mu_e)}\bigg]\int_0^\infty dp_z\,e^{-k\beta E_n^\pm}.
\end{align}
Using the general definition of Bessel function:
\begin{align}
K_\nu(\beta m)=\frac{\sqrt{\pi}}{\Gamma({\nu-1/2})}\frac{1}{m}\left(\frac{\beta}{2m}\right)^{\nu-1}\int_0^\infty\,dp\,p^{2\nu-2}e^{-\beta E} \,\,\,\,\,\,\,\mathrm{for}\,\nu>1/2,
\end{align}
the integral over $dp_z$ can be written as
\begin{align}
\int_0^\infty dp_z\,e^{-k\beta E_n^\pm}&=\frac{\Gamma{(1/2)}}{\sqrt{\pi}}\sqrt{\tilde{m}^2_\pm+2eBn}\,\,K_1\!\!\left({k\sqrt{\tilde{m}^2_\pm+2eBn}}/{T}\right)\notag\\&=\sqrt{\tilde{m}^2_\pm+2eBn}\,\,K_1\!\!\left({k\sqrt{\tilde{m}^2_\pm+2eBn}}/{T}\right).
\end{align}
In this case, the partition function becomes
\begin{align}
\ln\mathcal{Z}_{tot}&=\frac{2eBV}{(2\pi)^2}\sum_{n=0}^\infty\sum^{\infty}_{k=1}\frac{(-1)^{k+1}}{k}\bigg[2\cosh{(k\beta\mu_e)}\bigg]\sqrt{\tilde{m}^2_\pm+2eBn}\,\,K_1({k\sqrt{\tilde{m}^2_\pm+2eBn}}/{T})\notag\\
&=\frac{2eBTV}{(2\pi)^2}\sum^{\infty}_{k=1}\frac{(-1)^{k+1}}{k^2}\bigg[2\cosh{(k\beta\mu_e)}\bigg]\sum_{n=0}^\infty W^\pm_1(n),
\end{align}
where we introduce the function $W^\pm_1(n)$ as follows
\begin{align}
W^\pm_1(n)\equiv\frac{k\sqrt{\tilde{m}^2_\pm+2eBn}}{T}\,\,K_1\!\!\left({k\sqrt{\tilde{m}^2_\pm+2eBn}}/{T}\right).
\end{align}

Considering the Euler-Maclaurin formula to replace the sum over Landau levels, we have \index{Euler-Maclaurin formula}
\begin{align}
\sum^{\infty}_{n=0}W^\pm_1(n)=\int^\infty_0\!\!dn\,W^\pm_1(n)&+\frac{1}{2}\bigg[W^\pm_1(\infty)+W^\pm_1(0)\bigg]\notag\\
&\qquad+\frac{1}{12}\bigg[\left.\frac{\partial W^\pm_1}{\partial n}\right|_{\infty}-\left.\frac{\partial W^\pm_1}{\partial n}\right|_{0}\bigg]+R,
\end{align}
where $R$ is the error remainder which is defined by integrals over Bernoulli polynomials which is small and can be neglected~\cite{Elze:1980er}. Using the properties of Bessel function we have
\begin{align}
&\frac{\partial W^\pm_1}{\partial n}=-\frac{k^2eB}{T^2}K_0\left({k\sqrt{\tilde{m}^2_\pm+2eBn}}/{T}\right),\qquad W^\pm_1(\infty)=0,\\
&\int^\infty_a\!\!dx\,x^2K_1(x)=a^2K_2(a),
\end{align}
then we obtain
\begin{align}
\sum^{\infty}_{n=0}W^\pm_1(n)
&=\left(\frac{T^2}{k^2eB}\right)\left[\left(\frac{k\tilde{m}_\pm}{T}\right)^2K_2(k\tilde m_\pm/T)\right]+\frac{1}{2}\left[\left(\frac{k\tilde{m}_\pm}{T}\right)K_1(k\tilde m_\pm/T)\right]\notag\\
&\qquad+\frac{1}{12}\left[\left(\frac{k^2eB}{T^2}\right)K_0(k\tilde m_\pm/T)\right].
\end{align}
Replacing the sum over Landau levels by the integral, the partition function becomes
\begin{align}
\ln\mathcal{Z}_{tot}=\ln\mathcal{Z}_{free}+\ln\mathcal{Z}_B\,,
\end{align}
where we define the partition functions as  
\begin{align}
 \label{FreePart}&\ln\mathcal{Z}_{free}=\frac{T^3V}{2\pi^2}\left[2\cosh{\left(\frac{\eta_{e}}{T}\right)}\right]\sum_{i=\pm}x_i^2K_2\left(x_i\right)\,,\qquad x_i=\frac{\tilde{m}_i}{T}\\
 \label{MagPart}&\ln\mathcal{Z}_B=\frac{eBTV}{2\pi^2}\left[2\cosh{\left(\frac{\eta_{e}}{T}\right)}\right]\sum_{i=\pm}\bigg[\frac{x_i}{2}K_1\left(x_i\right)+\frac{b_0}{12}K_0\left(x_i\right)\bigg]\,.
\end{align}
The partition function $\ln(\mathcal{Z}_{free})$ in Eq.~(\ref{FreePart}) represents the general form of the Fermi partition function for $e^\pm$ with "effective mass" $\tilde{m}_\pm$ in our system. When the magnetic field $B=0$ the function $\ln(\mathcal{Z}_{free})$ will go back to the general form of the Fermi partition function without the external field. The partition function $\ln\mathcal{Z}_B$ gives us the partition with magnetic field effect to the order  $\mathcal{O}(eB)$ and  $\mathcal{O}(eB)^2$.


In the temperature domain $ 200\,\mathrm{keV} > T > 20$ keV, we have $m_e\gg T$, and it suffices to consider the Boltzmann limit of the quantum distributions. Considering the Boltzmann approximation for non-relativistic electrons and positrons we can rewrite Eq.~(\ref{FreePart}) - Eq.~(\ref{MagPart}) and obtain
\begin{align}
 \label{lnZ}
&\ln\mathcal{Z}_{tot}\!=\!\frac{T^3V}{2\pi^2}\left[2\cosh\left(\frac{\eta_{e}}{T}\right)\right]\sum_{i=\pm}\left\{x_i^{2} K_2\left(x_i\right)+\frac{b_0}{2}x_iK_1\left(x_i\right)+\frac{b^2_0}{12}K_0\left(x_i\right)\right\}.
\end{align}
Given the partition function Eq.~(\ref{lnZ}), we can explore the chemical potential and magnetization of $e^\pm$ plasma in the early Universe
under the hypothesis of charge neutrality and entropy conservation.

\paragraph{Electron chemical potential under magnetic field}
We explore the chemical potential of electron-positron plasma in a uniform magnetic field $B$ in the early Universe under the hypothesis of charge neutrality and entropy conservation. Considering the temperature after neutrino freeze-out, the charge neutrality condition can be written as
\begin{align}
 \label{density_proton}
 \left(n_{e}-n_{\bar{e}}\right)=n_{p}=X_p\,\left(\frac{n_{B}}{s_{\gamma,e}}\right)\,s_{\gamma,e},\qquad X_p\equiv\frac{n_p}{n_B}\,,
\end{align}
where $n_{p}$ and $n_B$ is the number density of protons and baryons respectively. Using the partition function Eq.~(\ref{lnZ}), the net number density of electrons in Boltzmann approximation can be written as
\begin{align}\label{NetElectron}
\left(n_e-n_{\bar e}\right)&=\frac{T}{V}\frac{\partial}{\partial \eta_{e}}\ln\mathcal{Z}_{tot}\notag\\
&=\frac{T^3}{2\pi^2}\left[2\sinh{(\eta_{e}/T)}\right]\sum_{i=\pm}\left[x_i^2K_2(x_i)+\frac{b_0}{2}x_i K_1(x_i)+\frac{b^2_0}{12}K_0(x_i)\right]\,.
\end{align}
Substituting Eq.~(\ref{NetElectron}) into the charge neutrality condition Eq.~(\ref{density_proton}), we can solve the chemical potential of electron $\eta_e/T$ numerically. We have
\begin{align}\label{ChemicalPotential}
\sinh{(\eta_{e}/T)}&=\frac{2\pi^2}{2T^3}\,\frac{X_p(n_B/s_{\gamma,e})s_{\gamma,e}}{\sum_{i=\pm}\left[x_i^2K_2(x_i)+\frac{b_0}{2}x_i K_1(x_i)+\frac{b^2_0}{12}K_0(x_i)\right]}\,,\\
&\longrightarrow\frac{2\pi^2n_p}{2T^3}\,\frac{X_p(n_B/s_{\gamma,e})s_{\gamma,e}}{2x^2K_2(x)},\qquad x=m_e/T,\qquad \mathrm{for}\,\,b_0=0\label{ChemiticalPotential_000}.
\end{align}
We see in Eq.~(\ref{ChemiticalPotential_000}) that for the case $b_0=0$, the chemical potential agrees with the free particle result in~\cite{Grayson:2023flr}. In Fig.~\ref{ChemicalPotential_B} we plot the chemical potential of electron as a function of temperature with different value of $b_0$. It shows that the chemical potential is not sensitive to the magnetic field because the small value of $10^{-3}>b_0>10^{-11}$ can be neglected in Eq.~(\ref{ChemicalPotential}).
%~~figure~~~~~~~~~~~~~~~~~~~~~~~~~~~~~~
\begin{figure}[ht]
\begin{center}
\includegraphics[width=\linewidth]{./plots/ChemicalPotential_new_survey}
\caption{The chemical potential of electron as a function of temperature in the magnetic field $b_0$ with $X_p=0.878$ and $n_B/n_\gamma=6.05\times10^{-10}$. The red dashed line represents the magnetic field $b_0=1.1\times10^{-11}$ and blue line labels the magnetic field $b_0=5.5\times10^{-3}$}
\label{ChemicalPotential_B}
\end{center}
\end{figure}
%~~~~~~~~~~~~~~~~~~~~~~~~~~~~~~~~~~~~~~~~

\paragraph{Electron-positron magnetization}
%We consider the electron-positron plasma in the mean field approximation where the external field is representative of the \lq\lq bulk\rq\rq\ internal magnetization of the gas. Each particle is therefore responding to the averaged magnetic flux generated by its neighbors as well as any global external field contribution. 

Considering the magnetized electron-positron partition function Eq.~(\ref{lnZ}), it is convenient to introduce the dimensionless magnetization $\overline{\mathcal{M}}$ and the critical field $B_c$ as follows \index{electron-positron plasma!magnetization}
\begin{align}
\label{Mdef}
\overline{\mathcal{M}}\equiv\frac{M}{\mathcal{B}_{c}}=\frac{1}{\mathcal{B}_{c}}\left(\frac{T}{V}\frac{\partial \ln\mathcal{Z}_{tot}}{\partial B}\right)\,\qquad \mathcal{B}_{c}=\frac{m_{e}^{2}}{e}\,.
\end{align}
Substituting the partition function Eq.~(\ref{lnZ}) into Eq.~(\ref{Mdef}), the total magnetization ${\overline{\mathcal M}}$ can be broken into the sum of spin parallel $\overline{\mathcal M}_{+}$ and spin anti-parallel $\overline{\mathcal M}_{-}$ magnetization. We have
\begin{align}\label{Magnetization}
&{\overline{\mathcal M}}={\overline{\mathcal M}_+}+{\overline{\mathcal M}_-},\\
&\overline{\mathcal M}_{\pm}=\frac{e^2T^{2}}{2\pi^2m_e^2}\left[2\cosh\left(\frac{\eta_{e}}{T}\right)\right]\left\{c_{1}(x_{\pm})K_1(x_i)+c_{0}K_0(x_\pm)\right\}\,,
\end{align}
where the coefficients are given by
\begin{align}
    c_{1}(x_{\pm}) &= \left[\frac{1}{2}-\left(\frac{1}{2}\pm\frac{g}{4}\right)\left(1+\frac{b^2_0}{12x^2_\pm}\right)\right]x_\pm\,,\qquad c_{0} = \left[\frac{1}{6}-\left(\frac{1}{4}\pm\frac{g}{8}\right)\right]b_0\,.
\end{align}
Substituting the chemical potential Eq.~(\ref{ChemicalPotential}) into Eq.~(\ref{Magnetization}), we can solve the magnetization numerically.
%~Figure~~~~~~~~~~~~~~~~~~~~~~~~~~~~~
\begin{figure}[ht]
    \centering
    \includegraphics[width=\textwidth]{./plots/Magnetization_Hc_new002.jpg}
    \caption{The magnetization $\overline{\cal M}=\mathcal{M}/\mathcal{B}_C$, with $g=2$, of the primordial $e^{+}e^{-}$ plasma is plotted as a function of temperature. The lower (solid red) and upper (solid blue) bounds for cosmic magnetic scale $b_{0}$ are included. The external magnetic field strength ${\cal B}/{\cal B}_{C}$ is also plotted in for lower (dashed red) and upper (dashed blue) bounds. The spin fugacity is set to unity.}
    \label{fig:magnet} 
\end{figure}
%~Figure~~~~~~~~~~~~~~~~~~~~~~~~~~~~~

In this thesis we focus on considering the case for $g=2$. In this case, the electron-positron magnetization can be written as 
\begin{align}\label{Magnetization_g2}
&{\overline{\mathcal M}}={\overline{\mathcal M}_+}+{\overline{\mathcal M}_-}\\
&{\overline{\cal M}}_{+}=\frac{e^{2}}{\pi^{2}}\frac{T^{2}}{m_{e}^{2}}\cosh{\frac{\eta_e}{T}}\left[\frac{1}{2}x_{+}K_{1}(x_{+})+\frac{b_{0}}{6}K_{0}(x_{+})\right]\,,\\
&{\overline{\cal M}}_{-}=-\frac{e^{2}}{\pi^{2}}\frac{T^{2}}{m_{e}^{2}}\cosh{\frac{\eta_e}{T}}\left[\left(\frac{1}{2}+\frac{b_{0}^{2}}{12x_{-}^{2}}\right)x_{-}K_{1}(x_{-})+\frac{b_{0}}{3}K_{0}(x_{-})\right]\,,
\end{align}
where $x_\pm$ are given by
\begin{align}
x_{+}=\frac{m_{e}}{T},\qquad   x_{-}=\sqrt{\frac{m_{e}^{2}}{T^{2}}+2b_{0}}
\end{align}
The discussion for the case $g\neq2$ can be found in~~\cite{Steinmetz:2023nsc}.

In Fig.~\ref{fig:magnet}, we present the magnetization Eq.~(\ref{Magnetization_g2}) for the case $g=2$ as a function of temperature. It shows that the magnetization depends on the magnetic scale $b_0$ and the $e^{+}e^{-}$ plasma possesses an overall paramagnetic property, resulting in a positive magnetization $\overline{\mathcal{M}}$. This paramagnetic property is contrary to the conventional assumption that matter-antimatter plasmas lack significant inherent magnetic responses. However, the magnetization never exceeds the external field under the parameters considered, which shows a lack of ferromagnetic behavior. As the Universe cooled, the dropping magnetization slowed at $T_{\mathrm{split}}=20.3$ keV, where positrons vanished. Thereafter the remaining electron density diluted with cosmic expansion.

In this section, we have explored the electron-positron plasma considering  external and self-magnetization fields  without spin potential $\eta_s/T\ll1$. However the nonzero spin potential $\eta_s\neq0$  would have an impact on the primordial $e^{+}e^{-}$ plasma. In general, the  magnetization is also a function of the spin potential $\eta_s$, and would be one important parameter that control the spin direction of primordial gas which allows for magnetization even in the absence of external magnetic fields. For further discussion see ~\cite{Steinmetz:2023nsc}.


%%%%%%%%%%%%%%%%%%%%%%%%%%%%%%%%%%%%%%%
%\chapter{Matter-antimatter origin of cosmic magnetism}
%\label{chap:cosmo}
%%%%%%%%%%%%%%%%%%%%%%%%%%%%%%%%%%%%%%%
\noindent We investigate the hypothesis that the observed intergalactic magnetic fields (IGMF) are primordial in nature, predating the recombination epoch. Specifically, we explore the role of the extremely large electron-positron $(e^{+}e^{-})$ pair abundance in the temperature range of $2000\keV>T>20\keV$ which only disappeared after Big Bang nucleosynthesis (BBN). We review the status of cosmic magnetism in \rsec{sec:mag_universe} which motivates our study. \rsec{sec:abundance} discusses the extreme electron-positron abundance during this epoch. The statistical and thermodynamic theory of the electron-positron gas is described in \rsec{sec:theory}. \rsec{sec:magnetization} describes the relativistic paramagnetism of the electron-positron gas. We propose in \rsec{sec:ferro} a model of self-magnetization caused by spin polarization within the individual species in the gas.

This chapter serves primarily as a review of our work in~\cite{Steinmetz:2023nsc,Steinmetz:2023ucp} and portions of~\cite{Rafelski:2023emw} where we propose that the early universe electron-positron plasma was a highly magnetized environment. We will use natural units $(c=\hbar=k_{B}=1)$ unless otherwise noted.

%%%%%%%%%%%%%%%%%%%%%%%%%%%%%%%%%%%%%%%
\section{Magnetism in the Plasma Universe}\label{partX}
\label{sec:mag_universe}
%%%%%%%%%%%%%%%%%%%%%%%%%%%%%%%%%%%%%%%
\noindent Macroscopic domains of magnetic fields have been found in all astrophysical environments from compact objects (stars, planets, etc.); interstellar and intergalactic space; and surprisingly in deep extra-galactic void spaces. Considering the ubiquity of magnetic fields in the universe~\cite{Giovannini:2017rbc,Giovannini:2003yn,Kronberg:1993vk}, we search for a common primordial mechanism initiate the diversity of magnetism observed today. In this chapter, IGMF will refer to experimentally observed intergalactic fields of any origin while primordial magnetic fields (PMF) refers to fields generated via early universe processes possibly as far back as inflation. The conventional elaboration of the origins for cosmic PMFs are detailed in~\cite{Gaensler:2004gk,Durrer:2013pga,AlvesBatista:2021sln}.

IGMF are notably difficult to measure and difficult to explain. The bounds for IGMF at a length scale of $1{\rm\ Mpc}$ are today~\cite{Neronov:2010gir,Taylor:2011bn,Pshirkov:2015tua,Jedamzik:2018itu,Vernstrom:2021hru}
\begin{gather}
 \label{igmf}
 10^{-8}{\rm\ G}>B_\mathrm{IGMF}>10^{-16}{\rm\ G}\,.
\end{gather}
We note that generating PMFs with such large coherent length scales is nontrivial~\cite{Giovannini:2022rrl} though currently the length scale for PMFs are not well constrained~\cite{AlvesBatista:2021sln}. Faraday rotation from distant radio active galaxy nuclei (AGN)~\cite{Pomakov:2022cem} suggest that neither dynamo nor astrophysical processes would sufficiently account for the presence of magnetic fields in the universe today if the IGMF strength was around the upper bound of $B_\mathrm{IGMF}\simeq30-60{\rm\ nG}$ as found in Ref.~\cite{Vernstrom:2021hru}. Such strong magnetic fields would then require that at least some portion of the IGMF arise from primordial sources that predate the formation of stars.

Magnetized baryon inhomogeneities which in turn would produce anisotropies in the cosmic microwave background (CMB)~\cite{Jedamzik:2013gua,Abdalla:2022yfr}. \cite{Jedamzik:2020krr} propose further that the presence of a magnetic field of $B_\mathrm{PMF}\simeq0.1{\rm\ nG}$ could be sufficient to explain the Hubble tension.

%%%%%%%%%%%%%%%%%%%%%%%%%%%%%%%%%%%%%%%
\begin{figure}[ht]
    \centering
    \includegraphics[width=0.95\textwidth]{plots/chap04cosmo/pmf.png}
    \caption{Qualitative plot of the primordial magnetic field strength over cosmic time. All figures are printed in temporal sequence in the expanding universe beginning with high temperatures (and early times) on the left and lower temperatures (and later times) on the right.}
    \label{fig:pmf}
\end{figure}
%%%%%%%%%%%%%%%%%%%%%%%%%%%%%%%%%%%%%%%

Our motivating hypothesis is outlined qualitatively in \rf{fig:pmf} where PMF evolution is plotted over the temperature history of the universe. The descending blue band indicates the range of possible PMF strengths. The different epochs of the universe according to $\Lambda\mathrm{CDM}$ are delineated by temperature. The horizontal lines mark two important scales: (a) the Schwinger critical field strength given by
\begin{align}
    \label{crit:1}
    B_\mathrm{C} = \frac{m_{e}^{2}}{e}\simeq4.41\times10^{13}\,\mathrm{G}\,.
\end{align}
where electrodynamics is expected to display nonlinear characteristics and (b) the upper field strength seen in magnetars of $\sim10^{15}\,\mathrm{G}$. A schematic of magnetogenesis is drawn with the dashed red lines indicating spontaneous formation of the PMF within the early universe plasma itself. The $e^{+}e^{-}$ era is notably the final epoch where antimatter exists in large quantities in the cosmos~\cite{Rafelski:2023emw}.

%%%%%%%%%%%%%%%%%%%%%%%%%%%%%%%%%%%%%%%
\subsection{Electron-positron abundance}
\label{sec:abundance}
%%%%%%%%%%%%%%%%%%%%%%%%%%%%%%%%%%%%%%%
\noindent As the universe cooled below temperature $T\!=\!m_{e}$ (the electron mass), the thermal electron and positron comoving density depleted by over eight orders of magnitude. At $T_\mathrm{split}=20.3\keV$, the charged lepton asymmetry (mirrored by baryon asymmetry and enforced by charge neutrality) became evident as the surviving excess electrons persisted while positrons vanished entirely from the particle inventory of the universe due to annihilation.

%%%%%%%%%%%%%%%%%%%%%%%%%%%%%%%%%%%%%%%
\begin{figure}[ht]
 \centering
\includegraphics[width=0.95\textwidth]{plots/chap04cosmo/EEPlasmaDensityRatio_new01.jpg}
 \caption{Number density of electron $e^{-}$ and positron $e^{+}$ to baryon ratio $n_{e^{\pm}}/n_{B}$ as a function of photon temperature in the universe. See text for further details. In this work we measure temperature in units of energy (keV) thus we set the Boltzmann constant to $k_{B}=1$. Figure courtesy of Cheng Tao Yang.}
 \label{fig:densityratio} 
\end{figure}
%%%%%%%%%%%%%%%%%%%%%%%%%%%%%%%%%%%%%%%

The electron-to-baryon density ratio $n_{e^{-}}/n_{B}$ is shown in \rf{fig:densityratio} as the solid blue line while the positron-to-baryon ratio $n_{e^{+}}/n_{B}$ is represented by the dashed red line. These two lines overlap until the temperature drops below $T_\mathrm{split}=20.3\keV$ as positrons vanish from the universe marking the end of the $e^{+}e^{-}$ plasma and the dominance of the electron-proton $(e^{-}p)$ plasma. The two vertical dashed green lines denote temperatures $T\!=\!m_{e}\simeq511\keV$ and $T_\mathrm{split}=20.3\keV$. These results were obtained using charge neutrality and the baryon-to-photon content (entropy) of the universe; see details in~\cite{Rafelski:2023emw}. The two horizontal black dashed lines denote the relativistic $T\gg m_e$ abundance of $n_{e^{\pm}}/n_{B}=4.47\times10^{8}$ and post-annihilation abundance of $n_{e^{-}}/n_{B}=0.87$. Above temperature $T\simeq85\keV$, the $e^{+}e^{-}$ primordial plasma density exceeded that of the Sun's core density $n_{e}\simeq6\times10^{26}{\rm\ cm}^{-3}$~\cite{Bahcall:2000nu}. 

Conversion of the dense $e^{+}e^{-}$ pair plasma into photons reheated the photon background~\cite{Birrell:2014uka} separating the photon and neutrino temperatures. The $e^{+}e^{-}$ annihilation and photon reheating period lasted no longer than an afternoon lunch break. Because of charge neutrality, the post-annihilation comoving ratio $n_{e^{-}}/n_{B}=0.87$~\cite{Rafelski:2023emw} is slightly offset from unity in~\rf{fig:densityratio} by the presence of bound neutrons in $\alpha$ particles and other neutron containing light elements produced during BBN epoch.

The abundance of baryons is itself fixed by the known abundance relative to photons~\cite{ParticleDataGroup:2022pth} and we employed the contemporary recommended value $n_B/n_\gamma=6.09\times 10^{-10}$. The resulting chemical potential needs to be evaluated carefully to obtain the behavior near to $T_\mathrm{split}=20.3\keV$ where the relatively small value of chemical potential $\mu$ rises rapidly so that positrons vanish from the particle inventory of the universe while nearly one electron per baryon remains. The detailed solution of this problem is found in \cite{Fromerth:2012fe,Rafelski:2023emw} leading to the results shown in \rf{fig:densityratio}.

%%%%%%%%%%%%%%%%%%%%%%%%%%%%%%%%%%%%%%%
\subsection{Theory of thermal matter-antimatter plasmas}
\label{sec:theory}
%%%%%%%%%%%%%%%%%%%%%%%%%%%%%%%%%%%%%%%
\noindent To evaluate magnetic properties of the thermal $e^{+}e^{-}$ pair plasma we take inspiration from Ch. 9 of Melrose's treatise on magnetized plasmas~\cite{melrose2008quantum}. We focus on the bulk properties of thermalized plasmas in (near) equilibrium.

We consider a homogeneous magnetic field domain defined along the $z$-axis as
\begin{gather}
    \label{homoB:1}
    \bb{B}=(0,\,0,\,B)\,,
\end{gather}
with magnetic field magnitude $|\bb{B}|=B$. Following \cite{Steinmetz:2018ryf}, we reprint the microscopic energy of the charged relativistic fermion within a homogeneous magnetic field given by
\begin{align}
 \label{cosmokgp}
 E^{n}_{\sigma,s}(p_{z},{B})=\sqrt{m_{e}^{2}+p_{z}^{2}+e{B}\left(2n+1+\frac{g}{2}\sigma s\right)}\,,
\end{align}
where $n\in0,1,2,\ldots$ is the Landau orbital quantum number, $p_{z}$ is the momentum parallel to the field axis and the electric charge is $e\equiv q_{e^{+}}=-q_{e^{-}}$. The index $\sigma$ in \req{cosmokgp} differentiates electron $(e^{-};\ \sigma=+1)$ and positron $(e^{+};\ \sigma=-1)$ states. The index $s$ refers to the spin along the field axis: parallel $(\uparrow;\ s=+1)$ or anti-parallel $(\downarrow;\ s=-1)$ for both particle and antiparticle species.

%%%%%%%%%%%%%%%%%%%%%%%%%%%%%%%%%%%%%%%
\begin{figure}[ht]
 \centering
 \includegraphics[width=0.95\linewidth]{plots/chap04cosmo/schematic.png}\Bstrut\\
 \begin{tabular}{ r|c|c| }
 \multicolumn{1}{r}{}
 & \multicolumn{1}{c}{aligned: $s=+1$}
 & \multicolumn{1}{c}{anti-aligned: $s=-1$} \\
 \cline{2-3}
 electron: $\sigma=+1$ & $U_{\rm Mag}>0$ & $U_{\rm Mag}<0$ \TBstrut\\
 \cline{2-3}
 positron: $\sigma=-1$ & $U_{\rm Mag}<0$ & $U_{\rm Mag}>0$ \TBstrut\\
 \cline{2-3}
 \end{tabular}\\
 \caption{Organizational schematic of matter-antimatter $(\sigma)$ and polarization $(s)$ states with respect to the sign of the non-relativistic magnetic dipole energy $U_{\rm Mag}$ obtainable from~\req{cosmokgp}.}
 \label{fig:schematic}
\end{figure}
%%%%%%%%%%%%%%%%%%%%%%%%%%%%%%%%%%%%%%%

The reason \req{cosmokgp} distinguishes between electrons and positrons is to ensure the correct non-relativistic limit for the magnetic dipole energy is reached. Following the conventions found in \cite{Tiesinga:2021myr}, we set the gyro-magnetic factor $g\equiv g_{e^{+}}=-g_{e^{-}}>0$ such that electrons and positrons have opposite $g$-factors and opposite magnetic moments relative to their spin; see \rf{fig:schematic}.

We recall the conventions established in \rsec{sec:flrw}. Conservation of magnetic flux requires that the magnetic field through a comoving surface $L_{0}^{2}$ remain unchanged. The magnetic field strength under expansion~\cite{Durrer:2013pga} starting at some initial time $t_{0}$ is then given by
\begin{gather}
 \label{bscale}
 B(t)=B_{0}\frac{a^{2}_{0}}{a^{2}(t)}\rightarrow B(z)=B_{0}\left(1+z\right)^{2}\,,
\end{gather}
where $B_{0}$ is the comoving value obtained from the contemporary value of the magnetic field today. Magnetic fields in the cosmos generated through mechanisms such as dynamo or astrophysical sources do not follow this scaling~\cite{Pomakov:2022cem}. It is only in deep intergalactic space where matter density is low are magnetic fields preserved (and thus uncontaminated) over cosmic time.

From \req{tscale} and \req{bscale} there emerges a natural ratio of interest which is conserved over cosmic expansion 
\begin{gather}
 \label{tbscale}
 \boxed{b\equiv\frac{e{B}(t)}{T^{2}(t)}=\frac{e{B}_{0}}{T_{0}^{2}}\equiv b_0={\rm\ const.}}\\
 10^{-3}>b_{0}>10^{-11}\,,
\end{gather}
given in natural units ($c=\hbar=k_{B}=1$). We computed the bounds for this cosmic magnetic scale ratio by using the present day IGMF observations given by \req{igmf} and the present CMB temperature $T_{0}=2.7{\rm\ K}\simeq2.3\times10^{-4}\eV$~\cite{Planck:2018vyg}.

%%%%%%%%%%%%%%%%%%%%%%%%%%%%%%%%%%%%%%%
\subsubsection{Eigenstatess of magnetic moment in cosmology}
\label{sec:protection}
%%%%%%%%%%%%%%%%%%%%%%%%%%%%%%%%%%%%%%%

As statistical properties depend on the characteristic Boltzmann factor $E/T$, another interpretation of \req{tbscale} in the context of energy eigenvalues (such as those given in \req{cosmokgp}) is the preservation of magnetic moment energy relative to momentum under adiabatic cosmic expansion. The Boltzmann statistical factor is given by
\begin{alignat}{1}
    \label{Boltz} x\equiv\frac{E}{T}\,.
\end{alignat}
We can explore this relationship for the magnetized system explicitly by writing out \req{Boltz} using the KGP energy eigenvalues written in \req{cosmokgp} as
\begin{alignat}{1}
    \label{XExplicit} x_{\sigma,s}^{n} = \frac{E_{\sigma,s}^{n}}{T} = \sqrt{\frac{m_{e}^{2}}{T^{2}}+\frac{p_{z}^{2}}{T^{2}}+\frac{eB}{T^{2}}\left(2n+1+\frac{g}{2}\sigma s\right)}\,.
\end{alignat}

Introducing the expansion scale factor $a(t)$ via \req{tscale}, \req{bscale} and \req{tbscale}. The Boltzmann factor can then be written as
\begin{alignat}{1}
    \label{xscale:1} x_{\sigma,s}^{n}(a(t)) = \sqrt{\frac{m_{e}^{2}}{T^{2}(t_{0})}\frac{a(t)^{2}}{a_{0}^{2}}+\frac{p_{z,0}^{2}}{T_{0}^{2}}+\frac{eB_{0}}{T_{0}^{2}}\left(2n+1+\frac{g}{2}\sigma s\right)}\,.
\end{alignat}
This reveals that only the mass contribution is dynamic over cosmological time. The constant of motion $b_{0}$ defined in \req{tbscale} is seen as the coefficient to the Landau and spin portion of the energy. For any given eigenstate, the mass term drives the state into the non-relativistic limit while the momenta and magnetic contributions are frozen by initial conditions. 

In comparison, the Boltzmann factor for the DP energy eigenvalues are given by
\begin{alignat}{1}
    \label{xscaledp:1} x_{\sigma,s}^{n}\vert_\mathrm{DP} = \sqrt{\left(\sqrt{\frac{m_{e}^{2}}{T^{2}}+\frac{eB}{T^{2}}\left(2n+1+\sigma s\right)}+\frac{eB}{2m_{e}T}\left(\frac{g}{2}-1\right)\sigma s\right)^{2}+\frac{p_{z}^{2}}{T^{2}}}\,,
\end{alignat}
which scales during FLRW expansion as
\begin{multline}
    \label{xscaledp:2} x_{\sigma,s}^{n}(a(t))\vert_\mathrm{DP} =\\ \sqrt{\left(\sqrt{\frac{m_{e}^{2}}{T_{0}^{2}}\frac{a(t)^{2}}{a_{0}^{2}}+\frac{eB_{0}}{T_{0}^{2}}\left(2n+1+\sigma s\right)}+\frac{eB_{0}}{2m_{e}T_{0}}\frac{a_{0}}{a(t)}\left(\frac{g}{2}-1\right)\sigma s\right)^{2}+\frac{p_{z,0}^{2}}{T_{0}^{2}}}\,.
\end{multline}
While the above expression is rather complicated, we note that the KGP~\req{xscale:1} and DP~\req{xscaledp:1} Boltzmann factors both reduce to the Sch{\"o}dinger-Pauli limit as $a(t)\rightarrow\infty$ thereby demonstrating that the total magnetic moment is protected under the adiabatic expansion of the universe.

Higher order non-minimal magnetic contributions can be introduced to the Boltzmann factor such as $\sim(e/m)^{2}B^{2}/T^{2}$. The reasoning above suggests that these terms are suppressed over cosmological time driving the system into minimal electromagnetic coupling with the exception of the anomalous magnetic moment. It is interesting to note that cosmological expansion then serves to `smooth out' the characteristics of more complex electrodynamics erasing them from a statistical perspective in favor of minimal-like dynamics.

%%%%%%%%%%%%%%%%%%%%%%%%%%%%%%%%%%%%%%%
\subsubsection{Magnetized fermion partition function}
\label{sec:partition}
%%%%%%%%%%%%%%%%%%%%%%%%%%%%%%%%%%%%%%%
\noindent To obtain a quantitative description of the above evolution, we study the bulk properties of the relativistic charged/magnetic gasses in a nearly homogeneous and isotropic primordial universe via the thermal Fermi-Dirac or Bose distributions.

The grand partition function for the relativistic Fermi-Dirac ensemble is given by the standard definition
\begin{align}
    \label{part:1} \ln\mathcal{Z}_\mathrm{total} &= \sum_{\alpha}\ln\left(1+\Upsilon_{\alpha_{1}\ldots\alpha_{m}}\exp\left(-\frac{E_{\alpha}}{T}\right)\right)\,,\\
    \Upsilon_{\alpha_{1}\ldots\alpha_{m}} &= \lambda_{\alpha_{1}}\lambda_{\alpha_{2}}\ldots\lambda_{\alpha_{m}}\,,
\end{align}
where we are summing over the set all relevant quantum numbers $\alpha=(\alpha_{1},\alpha_{2},\ldots,\alpha_{m})$. We note here the generalized the fugacity $\Upsilon_{\alpha_{1}\ldots\alpha_{m}}$ allowing for any possible deformation caused by pressures effecting the distribution of any quantum numbers.

In the case of the Landau problem, there is an additional summation over $\widetilde{G}$ which represents the occupancy of Landau states~\cite{greiner2012thermodynamics} which are matched to the available phase space within $\Delta p_{x}\Delta p_{y}$. If we consider the orbital Landau quantum number $n$ to represent the transverse momentum $p_{T}^{2}=p_{x}^{2}+p_{y}^{2}$ of the system, then the relationship that defines $\widetilde{G}$ is given by
\begin{alignat}{1}
    \label{phase:1} \frac{L^{2}}{(2\pi)^{2}}\Delta p_{x}\Delta p_{y}=\frac{eBL^{2}}{2\pi}\Delta n\,,\qquad\widetilde{G}=\frac{eBL^{2}}{2\pi}\,.
\end{alignat}
The summation over the continuous $p_{z}$ is replaced with an integration and the double summation over $p_{x}$ and $p_{y}$ is replaced by a single sum over Landau orbits
\begin{alignat}{1}
    \label{phase:2}
    \sum_{p_{z}}\rightarrow\frac{L}{2\pi}\int^{+\infty}_{-\infty}dp_{z}\,,\qquad\sum_{p_{x}}\sum_{p_{y}}\rightarrow\frac{eBL^{2}}{2\pi}\sum_{n}\,,
\end{alignat}
where $L$ defines the boundary length of our considered volume $V=L^{3}$.

The partition function of the $e^{+}e^{-}$ plasma can be understood as the sum of four gaseous species
\begin{align}
    \label{partition:0}    
    \ln\mathcal{Z}_{e^{+}e^{-}}=\ln\mathcal{Z}_{e^{+}}^{\uparrow}+\ln\mathcal{Z}_{e^{+}}^{\downarrow}+\ln\mathcal{Z}_{e^{-}}^{\uparrow}+\ln\mathcal{Z}_{e^{-}}^{\downarrow}\,,
\end{align}
of electrons and positrons of both polarizations $(\uparrow\downarrow)$. The change in phase space written in \req{phase:2} modify the magnetized $e^{+}e^{-}$ plasma partition function from \req{part:1} into
\begin{gather}
     \label{partition:1}
     \ln\mathcal{Z}_{e^{+}e^{-}}=\frac{e{B}V}{(2\pi)^{2}}\sum_{\sigma}^{\pm1}\sum_{s}^{\pm1}\sum_{n=0}^{\infty}\int_{-\infty}^{\infty}\mathrm{d}p_{z}\left[\ln\left(1+\lambda_{\sigma}\xi_{\sigma,s}\exp\left(-\frac{E_{\sigma,s}^{n}}{T}\right)\right)\right]\,\\
    \label{partition:2}    
    \Upsilon_{\sigma,s} =\lambda_{\sigma}\xi_{\sigma,s} = \exp{\frac{\mu_{\sigma}+\eta_{\sigma,s}}{T}}\,,
\end{gather}
where the energy eigenvalues $E_{\sigma,s}^{n}$ are given in \req{cosmokgp}. The index $\sigma$ in \req{partition:1} is a sum over electron and positron states while $s$ is a sum over polarizations. The index $s$ refers to the spin along the field axis: parallel $(\uparrow;\ s=+1)$ or anti-parallel $(\downarrow;\ s=-1)$ for both particle and antiparticle species.

We are explicitly interested in small asymmetries such as baryon excess over antibaryons, or one polarization over another. These are described by \req{partition:2} as the following two fugacities:
\begin{itemize}%[nosep]
 \item[(a)] Chemical fugacity $\lambda_{\sigma}$
 \item[(b)] Polarization fugacity $\xi_{\sigma,s}$
\end{itemize}
For matter $(e^{-};\ \sigma=+1)$ and antimatter $(e^{+};\ \sigma=-1)$ particles, a nonzero relativistic chemical potential $\mu_{\sigma}=\sigma\mu$ is caused by an imbalance of matter and antimatter. While the primordial electron-positron plasma era was overall charge neutral, there was a small asymmetry in the charged leptons (namely electrons) from baryon asymmetry~\cite{Fromerth:2012fe,Canetti:2012zc} in the universe. Reactions such as $e^{+}e^{-}\leftrightarrow\gamma\gamma$ constrains the chemical potential of electrons and positrons~\cite{Elze:1980er} as 
\begin{align}
 \label{cpotential}
 \mu\equiv\mu_{e^{-}}=-\mu_{e^{+}}\,,\qquad
 \lambda\equiv\lambda_{e^{-}}=\lambda_{e^{+}}^{-1}=\exp\frac{\mu}{T}\,,
\end{align}
where $\lambda$ is the chemical fugacity of the system.

We can then parameterize the chemical potential of the $e^{+}e^{-}$ plasma as a function of temperature $\mu\rightarrow\mu(T)$ via the charge neutrality of the universe which implies
\begin{align}
 \label{chargeneutrality}
 n_{p}=n_{e^{-}}-n_{e^{+}}=\frac{1}{V}\lambda\frac{\partial}{\partial\lambda}\ln\mathcal{Z}_{e^{+}e^{-}}\,.
\end{align}
In \req{chargeneutrality}, $n_{p}$ is the observed total number density of protons in all baryon species. The chemical potential defined in \req{cpotential} is obtained from the requirement that the positive charge of baryons (protons, $\alpha$ particles, light nuclei produced after BBN) is exactly and locally compensated by a tiny net excess of electrons over positrons.

We then introduce a novel polarization fugacity $\xi_{\sigma,s}$ and polarization potential $\eta_{\sigma,s}=\sigma s\eta$. We propose the polarization potential follows analogous expressions as seen in \req{cpotential} obeying
\begin{align}
 \label{spotential}
 \eta\equiv\eta_{+,+}=\eta_{-,-}\,,\quad\eta=-\eta_{\pm,\mp}\,,\quad\xi_{\sigma,s}\equiv\exp{\frac{\eta_{\sigma,s}}{T}}\,.
\end{align}
An imbalance in polarization within a region of volume $V$ results in a nonzero polarization potential $\eta\neq0$. Conveniently since antiparticles have opposite signs of charge and magnetic moment, the same magnetic moment is associated with opposite spin orientations. A completely particle-antiparticle symmetric magnetized plasma will have therefore zero total angular momentum.

%%%%%%%%%%%%%%%%%%%%%%%%%%%%%%%%%%%%%%%
\paragraph{Euler-Maclaurin integration.}
\label{sec:eulermac}
%%%%%%%%%%%%%%%%%%%%%%%%%%%%%%%%%%%%%%%
\noindent Before we proceed with the Boltzmann distribution approximation which makes up the bulk of our analysis, we will comment on the full Fermi-Dirac distribution analysis. The Euler-Maclaurin formula~\cite{abramowitz1988handbook} is used to convert the summation over Landau levels $n$ into an integration given by
\begin{multline}
    \label{eulermaclaurin}\sum^{b}_{n=a}f(n)-\int^{b}_{a}f(x)dx = \frac{1}{2}\left(f(b)+f(a)\right)\\
    +\sum_{i=1}^{j}\frac{b_{2i}}{(2i)!}\left(f^{(2i-1)}(b)-f^{(2i-1)}(a)\right)+R(j)\,,
\end{multline}
where $b_{2i}$ are the Bernoulli numbers and $R(j)$ is the error remainder defined by integrals over Bernoulli polynomials. The integer $j$ is chosen for the level of approximation that is desired. Euler-Maclaurin integration is rarely convergent, and in this case serves only as an approximation within the domain where the error remainder is small and bounded; see~\cite{greiner2012thermodynamics} for the non-relativistic case. In this analysis, we keep the zeroth and first order terms in the Euler-Maclaurin formula. We note that regularization of the excess terms in \req{eulermaclaurin} is done in the context of strong field QED~\cite{greiner2008quantum} though that is outside our scope.

Using \req{eulermaclaurin} allows us to convert the sum over $n$ quantum numbers in \req{partition:1} into an integral. Defining
\begin{alignat}{1}
    \label{Func} f_{\sigma,s}^{n}=\ln\left(1+\Upsilon_{\sigma,s}\exp\left(-\frac{E_{\sigma,s}^{n}}{T}\right)\right)\,,
\end{alignat}
\req{partition:1} for $j=1$ becomes
\begin{multline}
    \label{PartFuncTwo} \ln\mathcal{Z}_{e^{+}e^{-}} = \frac{e{B}V}{(2\pi)^{2}}\sum_{\sigma,s}^{\pm1}\int_{-\infty}^{+\infty}dp_{z}\\
    \left(\int_{0}^{+\infty}dn f_{\sigma,s}^{n} + \frac{1}{2}f_{\sigma,s}^{0} + \frac{1}{12}\frac{\partial f_{\sigma,s}^{n}}{\partial n}\bigg\rvert_{n=0} + R(1)\right)
\end{multline}
It will be useful to rearrange \req{cosmokgp} by pulling the spin dependency and the ground state Landau orbital into the mass writing
\begin{gather}
 \label{effmass:1}
 E^{n}_{\sigma,s}={\tilde m}_{\sigma,s}\sqrt{1+\frac{p_{z}^{2}}{{\tilde m}_{\sigma,s}^{2}}+\frac{2e{B}n}{{\tilde m}_{\sigma,s}^{2}}}\,,\\
 \label{effmass:2}
 \varepsilon_{\sigma,s}^{n}(p_{z},{B})=\frac{E_{\sigma,s}^{n}}{{\tilde m}_{\sigma,s}}\,,\qquad{\tilde m}_{\sigma,s}^{2}=m_{e}^{2}+e{B}\left(1+\frac{g}{2}\sigma s\right)\,,
\end{gather}
where we introduced the dimensionless energy $\varepsilon^{n}_{\sigma,s}$ and effective polarized mass ${\tilde m}_{\sigma,s}$ which is distinct for each spin alignment and is a function of magnetic field strength ${B}$. The effective polarized mass ${\tilde m}_{\sigma,s}$ allows us to describe the $e^{+}e^{-}$ plasma with the spin effects almost wholly separated from the Landau characteristics of the gas when considering the plasma's thermodynamic properties.

With the energies written in this fashion, we recognize the first term in \req{PartFuncTwo} as mathematically equivalent to the free particle fermion partition function with a re-scaled mass $m_{\sigma,s}$. The phase-space relationship between transverse momentum and Landau orbits in \req{phase:1} and \req{phase:2} can be succinctly described by
\begin{gather}
    p_{T}^{2} \sim 2eBn\,,\qquad2p_{T}dp_{T} \sim 2eBdn\,,\qquad d\bb{p}^{3}=2\pi p_{T}dp_{T}dp_{z}\\
    \frac{eBV}{(2\pi)^{2}}\int_{-\infty}^{+\infty}dp_{z}\int_{0}^{+\infty}dn \rightarrow \frac{V}{(2\pi)^{3}}\int d\bb{p}^{3}
\end{gather}
which recasts the first term in \req{PartFuncTwo} as
\begin{align}
    %\label{FreePart}
    \ln\mathcal{Z}_{e^{+}e^{-}} = \frac{V}{(2\pi)^{3}}\sum_{\sigma,s}^{\pm1}\int d\bb{p}^{3}\ln\left(1+\Upsilon_{\sigma,s}\exp{\left(-\frac{m_{\sigma,s}\sqrt{1+p^{2}/m_{\sigma,s}^{2}}}{T}\right)}\right)+\ldots
\end{align}
As we will see in the proceeding section, this separation of the `free-like' partition function can be reproduced in the Boltzmann distribution limit as well. This marks the end of the analytic analysis without approximations.

%%%%%%%%%%%%%%%%%%%%%%%%%%%%%%%%%%%%%%%
\subsubsection{Boltzmann approach to electron-positron plasma}
\label{sec:boltzmann}
%%%%%%%%%%%%%%%%%%%%%%%%%%%%%%%%%%%%%%%
\noindent Since we address the temperature interval $200\keV>T>20\keV$ where the effects of quantum Fermi statistics on the $e^{+}e^{-}$ pair plasma are relatively small, but the gas is still considered relativistic, we will employ the Boltzmann approximation to the partition function in \req{partition:1}. However, we extrapolate our results for presentation completeness up to $T\simeq 4m_{e}$.

%%%%%%%%%%%%%%%%%%%%%%%%%%%%%%%%%%%%%%%
\begin{table}[ht]
 \centering
 \begin{tabular}{ r|c|c| }
 \multicolumn{1}{r}{}
 & \multicolumn{1}{c}{aligned: $s=+1$}
 & \multicolumn{1}{c}{anti-aligned: $s=-1$} \\
 \cline{2-3}
 electron: $\sigma=+1$ & $+\mu+\eta$ & $+\mu-\eta$ \TBstrut\\
 \cline{2-3}
 positron: $\sigma=-1$ & $-\mu-\eta$ & $-\mu+\eta$ \TBstrut\\
 \cline{2-3}
 \end{tabular}\\\,\Bstrut\\
 \caption{Organizational schematic of matter-antimatter $(\sigma)$ and polarization $(s)$ states with respect to the chemical $\mu$ and polarization $\eta$ potentials as seen in~\req{partitionpower:2}. Companion to \rt{fig:schematic}.}
 \label{fig:org}
\end{table}
%%%%%%%%%%%%%%%%%%%%%%%%%%%%%%%%%%%%%%%

The partition function shown in equation \req{partition:1} can be rewritten removing the logarithm as
\begin{gather}
\label{partitionpower:1}
\ln{\mathcal{Z}_{e^{+}e^{-}}}=\frac{e{B}V}{(2\pi)^{2}}\sum_{\sigma,s}^{\pm1}\sum_{n=0}^{\infty}\sum_{k=1}^{\infty}\int_{-\infty}^{+\infty}\mathrm{d}p_{z}
\frac{(-1)^{k+1}}{k}\exp\left({k\frac{\sigma\mu+\sigma s\eta-{\tilde m}_{\sigma,s}\varepsilon^{n}_{\sigma,s}}{T}}\right)\,,\\
\label{bapprox} 
\sigma\mu+\sigma s\eta-{\tilde m}_{\sigma,s}\varepsilon_{\sigma,s}^{n}<0\,,
\end{gather}
which is well behaved as long as the factor in \req{bapprox} remains negative. We evaluate the sums over $\sigma$ and $s$ as
\begin{multline}
    \label{partitionpower:2}
    \ln{\mathcal{Z}_{e^{+}e^{-}}}=\frac{e{B}V}{(2\pi)^{2}}\sum_{n=0}^{\infty}\sum_{k=1}^{\infty}\int_{-\infty}^{+\infty}\mathrm{d}p_{z}\frac{(-1)^{k+1}}{k}\times\\
    \left(\ \exp\left(k\frac{+\mu+\eta}{T}\right)\exp\left(-k\frac{{\tilde m}_{+,+}\varepsilon_{+,+}^{n}}{T}\right)\right.
    +\exp\left(k\frac{+\mu-\eta}{T}\right)\exp\left(-k\frac{{\tilde m}_{+,-}\varepsilon_{+,-}^{n}}{T}\right)\qquad\\
    +\exp\left(k\frac{-\mu-\eta}{T}\right)\exp\left(-k\frac{{\tilde m}_{-,+}\varepsilon_{-,+}^{n}}{T}\right)
    +\left.\exp\left(k\frac{-\mu+\eta}{T}\right)\exp\left(-k\frac{{\tilde m}_{-,-}\varepsilon_{-,-}^{n}}{T}\right)\right)
\end{multline}
We note from \rf{fig:schematic} that the first and forth terms and the second and third terms share the same energies via
\begin{align}
    \label{partitionpower:3}
    \varepsilon_{+,+}^{n}=\varepsilon_{-,-}^{n}\,,\qquad
    \varepsilon_{+,-}^{n}=\varepsilon_{-,+}^{n}\,.\qquad
    \varepsilon_{+,-}^{n}<\varepsilon_{+,+}^{n}\,,
\end{align}

\req{partitionpower:3} allows us to reorganize the partition function with a new magnetization quantum number $s'$ which characterizes paramagnetic flux increasing states $(s'=+1)$ and diamagnetic flux decreasing states $(s'=-1)$. This recasts \req{partitionpower:2} as
\begin{multline}
    \label{partitionpower:4}
    \ln{\mathcal{Z}_{e^{+}e^{-}}}=\frac{e{B}V}{(2\pi)^{2}}\sum_{s'}^{\pm1}\sum_{n=0}^{\infty}\sum_{k=1}^{\infty}\int_{-\infty}^{+\infty}\mathrm{d}p_{z}\frac{(-1)^{k+1}}{k}\\
    \left[2\xi_{s'}\cosh\frac{k\mu}{T}\right]\exp\left(-k\frac{{\tilde m}_{s'}\varepsilon_{s'}^{n}}{T}\right)
\end{multline}
with dimensionless energy $\varepsilon_{s'}^{n}$, polarization mass $\tilde{m}_{s'}$, and polarization $\eta_{s'}$ redefined in terms of the moment orientation quantum number $s'$
\begin{gather}
    {\tilde m}_{s'}^{2}=m_{e}^{2}+e{B}\left(1-\frac{g}{2}s'\right)\,,\\
    \eta\equiv\eta_{+}=-\eta_{-}\qquad\xi\equiv\xi_{+}=\xi_{-}^{-1}\,,\qquad\xi_{s'}=\xi^{\pm1}=\exp\left(\pm\frac{\eta}{T}\right)\,.
\end{gather}

We introduce the modified Bessel function $K_{\nu}$ (see Ch. 10 of~\cite{Letessier:2002ony}) of the second kind
\begin{gather}
\label{besselk}
K_{\nu}\left(\frac{m}{T}\right)=\frac{\sqrt{\pi}}{\Gamma(\nu-1/2)}\frac{1}{m}\left(\frac{1}{2mT}\right)^{\nu-1}
\int_{0}^{\infty}\mathrm{d}p\,p^{2\nu-2}\exp\left({-\frac{m\varepsilon}{T}}\right)\,,\\
\nu>1/2\,,\qquad\varepsilon=\sqrt{1+p^{2}/m^{2}}\,,
\end{gather}
allowing us to rewrite the integral over momentum in \req{partitionpower:4} as
\begin{align}
 \label{besselkint}
 \frac{1}{T}\int_{0}^{\infty}\!\!\mathrm{d}p_{z}\exp\!\left(\!{-\frac{k{\tilde m}_{s'}\varepsilon_{s'}^{n}}{T}}\!\right)\!=\!W_{1}\!\!\left(\frac{k{\tilde m}_{s'}\varepsilon_{s'}^{n}(0,{B})}{T}\right)\,.
\end{align}
The function $W_{\nu}$ serves as an auxiliary function of the form $W_{\nu}(x)=xK_{\nu}(x)$. The notation $\varepsilon(0,{B})$ in \req{besselkint} refers to the definition of dimensionless energy found in \req{effmass:2} with $p_{z}=0$. The standard Boltzmann distribution is obtained by summing only $k=1$ and neglecting the higher order terms.

We take advantage again of Euler-Maclaurin integration \req{eulermaclaurin} and integrate the partition function. After truncation of the series and error remainder, the partition function \req{partitionpower:1} can then be written in terms of modified Bessel $K_{\nu}$ functions of the second kind and cosmic magnetic scale $b_{0}$, yielding
\begin{gather}
    \label{boltzmann}
    \boxed{\ln\mathcal{Z}_{e^{+}e^{-}}\simeq\frac{T^{3}V}{\pi^{2}}\sum_{s'}^{\pm1}\left[\xi_{s'}\cosh{\frac{\mu}{T}}\right]
    \left(x_{s'}^{2}K_{2}(x_{s'})+\frac{b_{0}}{2}x_{s'}K_{1}(x_{s'})+\frac{b_{0}^{2}}{12}K_{0}(x_{s'})\right)}\,,\\
    \label{xfunc}
    x_{s'}=\frac{{\tilde m}_{s'}}{T}=\sqrt{\frac{m_{e}^{2}}{T^{2}}+b_{0}\left(1-\frac{g}{2}s'\right)}\,.
\end{gather}
The latter two terms in \req{boltzmann} proportional to $b_{0}K_{1}$ and $b_{0}^{2}K_{0}$ are the uniquely magnetic terms present in powers of magnetic scale \req{tbscale} containing both spin and Landau orbital influences in the partition function. The $K_{2}$ term is analogous to the free Fermi gas~\cite{greiner2012thermodynamics} being modified only by spin effects.

This `separation of concerns' can be rewritten as
\begin{gather}
    \label{spin}
    \ln\mathcal{Z}_\mathrm{S}=\frac{T^{3}V}{\pi^{2}}\sum_{s'}^{\pm1}\left[\xi_{s'}\cosh{\frac{\mu}{T}}\right]\left(x_{s'}^{2}K_{2}(x_{s'})\right)\,,\\
    \label{spinorbit}
    \ln\mathcal{Z}_\mathrm{SO}=\frac{T^{3}V}{\pi^{2}}\sum_{s'}^{\pm}\left[\xi_{s'}\cosh{\frac{\mu}{T}}\right]
    \left(\frac{b_{0}}{2}x_{s'}K_{1}(x_{s'})+\frac{b_{0}^{2}}{12}K_{0}(x_{s'})\right)\,,        
\end{gather}

where the spin (S) and spin-orbit (SO) partition functions can be considered independently. When the magnetic scale $b_{0}$ is small, the spin-orbit term \req{spinorbit} becomes negligible leaving only paramagnetic effects in \req{spin} due to spin. In the non-relativistic limit, \req{spin} reproduces a quantum gas whose Hamiltonian is defined as the free particle (FP) Hamiltonian plus the magnetic dipole (MD) Hamiltonian which span two independent Hilbert spaces $\mathcal{H}_\mathrm{FP}\otimes\mathcal{H}_\mathrm{MD}$. The non-relativistic limit is further discussed in \rsec{sec:nrboltz}.

Writing the partition function as \req{boltzmann} instead of \req{partitionpower:1} has the additional benefit that the partition function remains finite in the free gas $({B}\rightarrow0)$ limit. This is because the free Fermi gas and \req{spin} are mathematically analogous to one another. As the Bessel $K_{\nu}$ functions are evaluated as functions of $x_{\pm}$ in \req{xfunc}, the `free' part of the partition $K_{2}$ is still subject to spin magnetization effects. In the limit where ${B}\rightarrow0$, the free Fermi gas is recovered in both the Boltzmann approximation $k=1$ and the general case $\sum_{k=1}^{\infty}$.

%%%%%%%%%%%%%%%%%%%%%%%%%%%%%%%%%%%%%%%
\subsubsection{Non-relativistic limit of the magnetized partition function}
\label{sec:nrboltz}
%%%%%%%%%%%%%%%%%%%%%%%%%%%%%%%%%%%%%%%
While we label the first term in \req{FreePart} as the `free' partition function, this is not strictly true as the partition function dependant on the magnetic-mass we defined in \req{effmass:2}. When determining the magnetization of the quantum Fermi gas, derivatives of the magnetic field $B$ will not fully vanish on this first term which will resulting in an intrinsic magnetization which is distinct from the Landau levels.

This represents magnetization that arises from the spin magnetic energy rather than orbital contributions. To demonstrate this, we will briefly consider the weak field limit for $g=2$. The effective polarized mass for electrons is then
\begin{align}
  \label{MagMassPlus}
  \tilde{m}_{+}^{2}&=m_{e}^{2}\,,\\
  \label{MagMassMinus}
  \tilde{m}_{-}^{2}&=m_{e}^{2}+2eB\,,
\end{align}
with energy eigenvalues
\begin{align}
  \label{EPlus}
  E_{n}^{+}&=\sqrt{p_{z}^{2}+m_{e}^{2}+2eBn}\,,\\
  \label{EMinus}
  E_{n}^{-}&=\sqrt{\left(E_{n}^{+}\right)^{2}+2eB}\,.
\end{align}
The spin anti-aligned states in the non-relativistic (NR) limit reduce to
\begin{align}
  \label{EMinusNR} E_{n}^{-}\vert_\mathrm{NR}\approx E_{n}^{+}\vert_\mathrm{NR}+\frac{eB}{m_{e}}\,.
\end{align}
This shift in energies is otherwise not influenced by summation over Landau quantum number $n$, therefore we can interpret this energy shift as a shift in the polarization potential from \req{spotential}. The polarization potential is then
\begin{align}
  \label{SpinChem} \eta_{e}^{\pm}=\eta_{e}\pm\frac{eB}{2m_{e}}\,,
\end{align}
allowing us to rewrite the partition function in \req{partitionpower:1} as
\begin{gather}
  \label{PartTotalNR} \ln\mathcal{Z}_{e^{-}}\vert_{NR}=\frac{eBV}{(2\pi)^{2}}\sum_{s'}^{\pm}\sum_{n=0}^{\infty}\sum_{k=1}^{\infty}\int_{-\infty}^{+\infty}dp_{z}\frac{(-1)^{k+1}}{k}2\cosh(k\beta\eta_{e}^{s'})\lambda^{k}\exp(-k\epsilon_{n}/T)\,,\\
  \label{NREnergy} \epsilon_{n}=m_{e}+\frac{p_{z}^{2}}{2m_{e}}+\frac{eB}{2m_{e}}\left(n+1\right)\,.
\end{gather}

\req{PartTotalNR} is then the traditional NR quantum harmonic oscillator partition function with a spin dependant potential shift differentiating the aligned and anti-aligned states. We note that in this formulation, the spin contribution is entirely excised from the orbital contribution. Under Euler-Maclaurin integration, the now spin-independant Boltzmann factor can be further separated into `free' and Landau quantum parts as was done in \req{FreePart} for the relativistic case. We note however that the inclusion of anomalous magnetic moment spoils this clean separation.

%%%%%%%%%%%%%%%%%%%%%%%%%%%%%%%%%%%%%%%
\subsubsection{Electron-positron chemical potential}
\label{sec:chem}
%%%%%%%%%%%%%%%%%%%%%%%%%%%%%%%%%%%%%%%
\noindent In presence of a magnetic field in the Boltzmann approximation, the charge neutrality condition \req{chargeneutrality} becomes
\begin{gather}
 \label{chem}
 \sinh\frac{\mu}{T}=n_{p}\frac{\pi^{2}}{T^{3}}
 \left[\sum_{s'}^{\pm1}\xi_{s'}\!\left(\!x_{s'}^{2}K_{2}(x_{s'})\!+\!\frac{b_{0}}{2}x_{s'}K_{1}(x_{s'})\!+\!\frac{b_{0}^{2}}{12}K_{0}(x_{s'}\!)\!\right)\!\right]^{-1}\!.
\end{gather}
\req{chem} is fully determined by the right-hand-side expression if the spin fugacity is set to unity $\eta=0$ implying no external bias to the number of polarizations except as a consequence of the difference in energy eigenvalues. In practice, the latter two terms in \req{chem} are negligible to chemical potential in the bounds of the primordial $e^{+}e^{-}$ plasma considered and only becomes relevant for extreme (see \rf{fig:chemicalpotential}) magnetic field strengths well outside our scope.

%%%%%%%%%%%%%%%%%%%%%%%%%%%%%%%%%%%%%%%
\begin{figure}[ht]
 \centering
 \includegraphics[clip, trim=0.0cm 0.0cm 0.0cm 0.0cm,width=0.95\linewidth]{plots/chap04cosmo/thesis_chempot_fixed.pdf}
 \caption{The chemical potential over temperature $\mu/T$ is plotted as a function of temperature with differing values of spin potential $\eta$ and magnetic scale $b_{0}$.}
 \label{fig:chemicalpotential}
\end{figure}
%%%%%%%%%%%%%%%%%%%%%%%%%%%%%%%%%%%%%%%

\req{chem} simplifies if there is no external magnetic field $b_{0}=0$ into
\begin{align}
    \label{simpchem:1}
    \sinh\frac{\mu}{T}=n_{p}\frac{\pi^{2}}{T^{3}}\left[2\cosh\frac{\eta}{T}\left(\frac{m_{e}}{T}\right)^{2}K_{2}\left(\frac{m_{e}}{T}\right)\right]^{-1}\,.
\end{align}

In \rf{fig:chemicalpotential} we plot the chemical potential $\mu/T$ in \req{chem} and \req{simpchem:1} which characterizes the importance of the charged lepton asymmetry as a function of temperature. Since the baryon (and thus charged lepton) asymmetry remains fixed, the suppression of $\mu/T$ at high temperatures indicates a large pair density which is seen explicitly in \rf{fig:densityratio}. The black line corresponds to the $b_{0}=0$ and $\eta=0$ case. 

The para-diamagnetic contribution from \req{spinorbit} does not appreciably influence $\mu/T$ until the magnetic scales involved become incredibly large well outside the observational bounds defined in \req{igmf} and \req{tbscale} as seen by the dotted blue curves of various large values $b_{0}=\{25,\ 50,\ 100,\ 300\}$. The chemical potential is also insensitive to forcing by the spin potential until $\eta$ reaches a significant fraction of the electron mass $m_{e}$ in size. The chemical potential for large values of spin potential $\eta=\{100,\ 200,\ 300,\ 400,\ 500\}\,\keV$ are also plotted as dashed black lines with $b_{0}=0$.

It is interesting to note that there are crossing points where a given chemical potential can be described as either an imbalance in spin-polarization or presence of external magnetic field. While spin potential suppresses the chemical potential at low temperatures, external magnetic fields only suppress the chemical potential at high temperatures.

The profound insensitivity of the chemical potential to these parameters justifies the use of the free particle chemical potential (black line) in the ranges of magnetic field strength considered for cosmology. Mathematically this can be understood as $\xi$ and $b_{0}$ act as small corrections in the denominator of \req{chem} if expanded in powers of these two parameters.

%%%%%%%%%%%%%%%%%%%%%%%%%%%%%%%%%%%%%%%
\subsection{Relativistic paramagnetism of electron-positron gas}
\label{sec:magnetization}
%%%%%%%%%%%%%%%%%%%%%%%%%%%%%%%%%%%%%%%
\noindent The total magnetic flux within a region of space can be written as the sum of external fields and the magnetization of the medium via
\begin{align}
 \label{totalmag}
 {B}_\mathrm{total} = {B} + \mathcal{M}\,.
\end{align}
For the simplest mediums without ferromagnetic or hysteresis considerations, the relationship can be parameterized by the susceptibility $\chi$ of the medium as
\begin{align}
 \label{susceptibility}
 {B}_\mathrm{total} = (1+\chi){B}\,,\qquad \mathcal{M} = \chi{B}\,,\qquad \chi\equiv\frac{\partial\mathcal{M}}{\partial{B}}\,,
\end{align}
with the possibility of both paramagnetic materials $(\chi>1)$ and diamagnetic materials $(\chi<1)$. The $e^{+}e^{-}$ plasma however does not so neatly fit in either category as given by \req{spin} and \req{spinorbit}. In general, the susceptibility of the gas will itself be a field dependant quantity.

In our analysis, the external magnetic field always appears within the context of the magnetic scale $b_{0}$, therefore we can introduce the change of variables
\begin{align}
 \frac{\partial b_{0}}{\partial{B}}=\frac{e}{T^{2}}\,.
\end{align}
The magnetization of the $e^{+}e^{-}$ plasma described by the partition function in \req{boltzmann} can then be written as
\begin{align}
 \label{defmagetization}
 \mathcal{M}\equiv\frac{T}{V}\frac{\partial}{\partial{B}}\ln{\mathcal{Z}_{e^{+}e^{-}}} = \frac{T}{V}\left(\frac{\partial b_{0}}{\partial{B}}\right)\frac{\partial}{\partial b_{0}}\ln{\mathcal{Z}_{e^{+}e^{-}}}\,,
\end{align}
Magnetization arising from other components in the cosmic gas (protons, neutrinos, etc.) could in principle also be included. Localized inhomogeneities of matter evolution are often non-trivial and generally be solved numerically using magneto-hydrodynamics (MHD)~\cite{melrose2008quantum,Vazza:2017qge,Vachaspati:2020blt} or with a suitable Boltzmann-Vlasov transport equation. An extension of our work would be to embed magnetization into transport theory~\cite{Formanek:2021blc}. In the context of MHD, primordial magnetogenesis from fluid flows in the electron-positron epoch was considered in~\cite{Gopal:2004ut,Perrone:2021srr}.

We introduce dimensionless units for magnetization ${\mathfrak M}$ by defining the critical field strength
\begin{align}
 {B}_{C}\equiv\frac{m_{e}^{2}}{e}\,,\qquad{\mathfrak M}\equiv\frac{\mathcal{M}}{{B}_{C}}\,.
\end{align}
The scale ${B}_{C}$ is where electromagnetism is expected to become subject to non-linear effects, though luckily in our regime of interest, electrodynamics should be linear. We note however that the upper bounds of IGMFs in \req{igmf} (with $b_{0}=10^{-3}$; see \req{tbscale}) brings us to within $1\%$ of that limit for the external field strength in the temperature range considered.

The total magnetization ${\mathfrak M}$ can be broken into the sum of magnetic moment parallel ${\mathfrak M}_{+}$ and magnetic moment anti-parallel ${\mathfrak M}_{-}$ contributions
\begin{align}
\label{g2mag}
{\mathfrak M}={\mathfrak M}_{+}+{\mathfrak M}_{-}\,.
\end{align}
We note that the expression for the magnetization simplifies significantly for $g\!=\!2$ which is the `natural' gyro-magnetic factor~\cite{Evans:2022fsu,Rafelski:2022bsv} for Dirac particles. For illustration, the $g\!=\!2$ magnetization from \req{defmagetization} is then
\begin{align}
 \label{g2magplus}
 {\mathfrak M}_{+}&=\frac{e^{2}}{\pi^{2}}\frac{T^{2}}{m_{e}^{2}}\xi\cosh{\frac{\mu}{T}}\left[\frac{1}{2}x_{+}K_{1}(x_{+})+\frac{b_{0}}{6}K_{0}(x_{+})\right]\,,\\
 \label{g2magminus}
 -{\mathfrak M}_{-}&=\frac{e^{2}}{\pi^{2}}\frac{T^{2}}{m_{e}^{2}}\xi^{-1}\cosh{\frac{\mu}{T}}
 \left[\left(\frac{1}{2}+\frac{b_{0}^{2}}{12x_{-}^{2}}\right)x_{-}K_{1}(x_{-})+\frac{b_{0}}{3}K_{0}(x_{-})\right]\,,\\
 x_{+}&=\frac{m_{e}}{T}\,,\qquad
 x_{-}=\sqrt{\frac{m_{e}^{2}}{T^{2}}+2b_{0}}\,.
\end{align}
As the $g$-factor of the electron is only slightly above two at $g\simeq2.00232$~\cite{Tiesinga:2021myr}, the above two expressions for ${\mathfrak M}_{+}$ and ${\mathfrak M}_{-}$ are only modified by a small amount because of anomalous magnetic moment (AMM) and would be otherwise invisible on our figures.

%%%%%%%%%%%%%%%%%%%%%%%%%%%%%%%%%%%%%%%
\subsubsection{Evolution of electron-positron magnetization}
\label{sec:paramagnetism}
%%%%%%%%%%%%%%%%%%%%%%%%%%%%%%%%%%%%%%%
\noindent In \rf{fig:magnet}, we plot the magnetization as given by \req{g2magplus} and \req{g2magminus} with the spin potential set to unity $\xi=1$. The lower (solid red) and upper (solid blue) bounds for cosmic magnetic scale $b_{0}$ are included. The external magnetic field strength ${B}/{B}_{C}$ is also plotted for lower (dotted red) and upper (dotted blue) bounds. Since the derivative of the partition function governing magnetization may manifest differences between Fermi-Dirac and the here used Boltzmann limit more acutely, out of abundance of caution, we indicate extrapolation outside the domain of validity of the Boltzmann limit with dashes.

%%%%%%%%%%%%%%%%%%%%%%%%%%%%%%%%%%%%%%%
\begin{figure}[ht]
 \centering
 \includegraphics[width=0.95\linewidth]{plots/chap04cosmo/thesis_mag.pdf}
 \caption{The magnetization ${\mathfrak M}$, with $g\!=\!2$, of the primordial $e^{+}e^{-}$ plasma is plotted as a function of temperature}
 %label{fig:magnet} 
\end{figure}
%%%%%%%%%%%%%%%%%%%%%%%%%%%%%%%%%%%%%%%

We see in \rf{fig:magnet} that the $e^{+}e^{-}$ plasma is overall paramagnetic and yields a positive overall magnetization which is contrary to the traditional assumption that matter-antimatter plasma lack significant magnetic responses of their own in the bulk. With that said, the magnetization never exceeds the external field under the parameters considered which shows a lack of ferromagnetic behavior. 

The large abundance of pairs causes the smallness of the chemical potential seen in~\rf{fig:chemicalpotential} at high temperatures. As the universe expands and temperature decreases, there is a rapid decrease of the density $n_{e^{\pm}}$ of $e^{+}e^{-}$ pairs. This is the primary the cause of the rapid paramagnetic decrease seen in \rf{fig:magnet} above $T\!=\!21\keV$. At lower temperatures $T<21\keV$ there remains a small electron excess (see~\rf{fig:densityratio}) needed to neutralize proton charge. These excess electrons then govern the residual magnetization and dilutes with cosmic expansion.

An interesting feature of \rf{fig:magnet} is that the magnetization in the full temperature range increases as a function of temperature. This is contrary to Curie's law~\cite{greiner2012thermodynamics} which stipulates that paramagnetic susceptibility of a laboratory material is inversely proportional to temperature. However, Curie's law applies to systems with fixed number of particles which is not true in our situation; see \rsec{sec:perlepton}.

A further consideration is possible hysteresis as the $e^{+}e^{-}$ density drops with temperature. It is not immediately obvious the gas's magnetization should simply `degauss' so rapidly without further consequence. If the very large paramagnetic susceptibility present for $T\simeq m_{e}$ is the origin of an overall magnetization of the plasma, the conservation of magnetic flux through the comoving surface ensures that the initial residual magnetization is preserved at a lower temperature by Faraday induced kinetic flow processes however our model presented here cannot account for such effects.

Early universe conditions may also apply to some extreme stellar objects with rapid change in $n_{e^{\pm}}$ with temperatures above $T\!=\!21\keV$. Production and annihilation of $e^{+}e^{-}$ plasmas is also predicted around compact stellar objects~\cite{Ruffini:2009hg,Ruffini:2012it} potentially as a source of gamma-ray bursts.

%%%%%%%%%%%%%%%%%%%%%%%%%%%%%%%%%%%%%%%
\subsubsection{Dependency on g-factor}
\label{sec:gfac}
%%%%%%%%%%%%%%%%%%%%%%%%%%%%%%%%%%%%%%%

\noindent As discussed at the end of \rsec{sec:magnetization}, the AMM of $e^{+}e^{-}$ is not relevant in the present model. However out of academic interest, it is valuable to consider how magnetization is effected by changing the $g$-factor significantly.

%%%%%%%%%%%%%%%%%%%%%%%%%%%%%%%%%%%%%%%
\begin{figure}[ht]
 \centering
 \includegraphics[width=0.95\textwidth]{plots/chap04cosmo/thesis_gfac.pdf}
 \caption{The magnetization $\mathfrak M$ as a function of $g$-factor plotted for several temperatures with magnetic scale $b_{0}=10^{-3}$ and polarization fugacity $\xi=1$.}
 \label{fig:gfac} 
\end{figure}
%%%%%%%%%%%%%%%%%%%%%%%%%%%%%%%%%%%%%%%

The influence of AMM would be more relevant for the magnetization of baryon gasses since the $g$-factor for protons $(g\approx5.6)$ and neutrons $(g\approx3.8)$ are substantially different from $g\!=\!2$. The influence of AMM on the magnetization of thermal systems with large baryon content (neutron stars, magnetars, hypothetical bose stars, etc.) is therefore also of interest~\cite{Ferrer:2019xlr,Ferrer:2023pgq}.

\req{g2magplus} and \req{g2magminus} with arbitrary $g$ reintroduced is given by
\begin{gather}
\label{arbg:1}
{\mathfrak M}=\frac{e^{2}}{\pi^{2}}\frac{T^{2}}{m_{e}^{2}}\sum_{s'}^{\pm1}\xi_{s'}\cosh{\frac{\mu}{T}}
\left[C^{1}_{s'}(x_{s'})K_{1}(x_{s'})+C^{0}_{s'}K_{0}(x_{s'})\right]\,,\\
\label{arbg:2}
C^{1}_{s'}(x_{\pm}) = \left[\frac{1}{2}-\left(\frac{1}{2}-\frac{g}{4}s'\right)\left(1+\frac{b^2_0}{12x^{2}_{s'}}\right)\right]x_{s'}\,,\qquad
C^{0}_{s'} = \left[\frac{1}{6}-\left(\frac{1}{4}-\frac{g}{8}s'\right)\right]b_0\,,
\end{gather}
where $x_{s'}$ was previously defined in \req{xfunc}.

In \rf{fig:gfac}, we plot the magnetization as a function of $g$-factor between $4>g>-4$ for temperatures $T\!=\!\{511,\ 300,\ 150,\ 70\}\keV$. We find that the magnetization is sensitive to the value of AMM revealing a transition point between paramagnetic $({\mathfrak M}>0)$ and diamagnetic gasses $({\mathfrak M}<0)$. Curiously, the transition point was numerically determined to be around $g\simeq1.1547$ in the limit $b_{0}\rightarrow0$. The exact position of this transition point however was found to be both temperature and $b_{0}$ sensitive, though it moved little in the ranges considered.

It is not surprising for there to be a transition between diamagnetism and paramagnetism given that the partition function (see \req{spin} and \req{spinorbit}) contained elements of both. With that said, the transition point presented at $g\approx1.15$ should not be taken as exact because of the approximations used to obtain the above results. 

It is likely that the exact transition point has been altered by our taking of the Boltzmann approximation and Euler-Maclaurin integration steps. It is known that the Klein-Gordon-Pauli solutions to the Landau problem in \req{cosmokgp} have periodic behavior~\cite{Steinmetz:2018ryf,Evans:2022fsu,Rafelski:2022bsv} for $|g|=k/2$ (where $k\in1,2,3\ldots$).

These integer and half-integer points represent when the two Landau towers of orbital levels match up exactly. Therefore, we propose a more natural transition between the spinless diamagnetic gas of $g=0$ and a paramagnetic gas is $g=1$. A more careful analysis is required to confirm this, but that our numerical value is close to unity is suggestive.

%%%%%%%%%%%%%%%%%%%%%%%%%%%%%%%%%%%%%%%
\subsubsection{Magnetization per lepton}
\label{sec:perlepton}
%%%%%%%%%%%%%%%%%%%%%%%%%%%%%%%%%%%%%%%
\noindent Despite the relatively large magnetization seen in \rf{fig:magnet}, the average contribution per lepton is only a small fraction of its overall magnetic moment indicating the magnetization is only loosely organized. Specifically, the magnetization regime we are in is described by
\begin{align}
 \label{fractionalmagnetization}
 \mathcal{M}\ll\mu_{B}\frac{N_{e^{+}}+N_{e^{-}}}{V}\,,\qquad\mu_{B}\equiv\frac{e}{2m_{e}}\,,
\end{align}
where $\mu_{B}$ is the Bohr magneton and $N=nV$ is the total particle number in the proper volume V. To better demonstrate that the plasma is only weakly magnetized, we define the average magnetic moment per lepton given by along the field ($z$-direction) axis as
\begin{align}
 \label{momentperlepton}
 \vert\vec{m}\vert_{z}\equiv\frac{\mathcal{M}}{n_{e^{-}}+n_{e^{+}}}\,,\qquad\vert\vec{m}\vert_{x}=\vert\vec{m}\vert_{y}=0\,.
\end{align}
Statistically, we expect the transverse expectation values to be zero. We emphasize here that despite $|\vec{m}|_{z}$ being nonzero, this doesn't indicate a nonzero spin angular momentum as our plasma is nearly matter-antimatter symmetric. The quantity defined in \req{momentperlepton} gives us an insight into the microscopic response of the plasma.

%%%%%%%%%%%%%%%%%%%%%%%%%%%%%%%%%%%%%%%
\begin{figure}[ht]
 \centering
 \includegraphics[clip, trim=0.0cm 0.0cm 0.0cm 0.0cm,width=0.95\textwidth]{plots/chap04cosmo/thesis_perlepton.png}
 \caption{The magnetic moment per lepton $\vert\vec{m}\vert_{z}$ along the field axis as a function of temperature}
 \label{fig:momentperlepton}
\end{figure}
%%%%%%%%%%%%%%%%%%%%%%%%%%%%%%%%%%%%%%%

The average magnetic moment $\vert\vec{m}\vert_{z}$ defined in \req{momentperlepton} is plotted in \rf{fig:momentperlepton} which displays how essential the external field is on the `per lepton' magnetization. The $b_{0}=10^{-3}$ case (blue curve) is plotted in the Boltzmann approximation. The dashed lines indicate where this approximation is only qualitatively correct. For illustration, a constant magnetic field case (solid green line) with a comoving reference value chosen at temperature $T_{0}=10\keV$ is also plotted.

If the field strength is held constant, then the average magnetic moment per lepton is suppressed at higher temperatures as expected for magnetization satisfying Curie's law. The difference in \rf{fig:momentperlepton} between the non-constant (blue solid curve) case and the constant field (solid green curve) case demonstrates the importance of the conservation of primordial magnetic flux in the plasma, required by \req{bscale}. While not shown, if \rf{fig:momentperlepton} was extended to lower temperatures, the magnetization per lepton of the constant field case would be greater than the non-constant case which agrees with our intuition that magnetization is easier to achieve at lower temperatures. This feature again highlights the importance of flux conservation in the system and the uniqueness of the primordial cosmic environment.

%%%%%%%%%%%%%%%%%%%%%%%%%%%%%%%%%%%%%%%
\subsection{Polarization potential and ferromagnetism}
\label{sec:ferro}
%%%%%%%%%%%%%%%%%%%%%%%%%%%%%%%%%%%%%%%
\noindent Up to this point, we have neglected the impact that a nonzero spin potential $\eta\neq0$ (and thus $\xi\neq1$) would have on the primordial $e^{+}e^{-}$ plasma magnetization. In the limit that $(m_{e}/T)^2\gg b_0$ the magnetization given in \req{arbg:1} and \req{arbg:2} is entirely controlled by the spin fugacity $\xi$ asymmetry generated by the spin potential $\eta$ yielding up to first order $\mathcal{O}(b_{0})$ in magnetic scale
\begin{multline}
 \label{ferro}
 \lim_{m_{e}^{2}/T^{2}\gg b_0}{\mathfrak M}=\frac{g}{2}\frac{e^{2}}{\pi^{2}}\frac{T^{2}}{m_{e}^{2}}\sinh{\frac{\eta}{T}}\cosh{\frac{\mu}{T}}\left[\frac{m_{e}}{T}K_{1}\left(\frac{m_{e}}{T}\right)\right]\\
 +b_{0}\left(g^{2}-\frac{4}{3}\right)\frac{e^{2}}{8\pi^{2}}\frac{T^{2}}{m_{e}^{2}}\cosh{\frac{\eta}{T}}\cosh{\frac{\mu}{T}}K_{0}\left(\frac{m_{e}}{T}\right)
 +\mathcal{O}\left(b_{0}^{2}\right)
\end{multline}

Given \req{ferro}, we can understand the spin potential as a kind of `ferromagnetic' influence on the primordial gas which allows for magnetization even in the absence of external magnetic fields. This interpretation is reinforced by the fact the leading coefficient is $g/2$.

We suggest that a variety of physics could produce a small nonzero $\eta$ within a domain of the gas. Such asymmetries could also originate statistically as while the expectation value of free gas polarization is zero, the variance is likely not.

As $\sinh{\eta/T}$ is an odd function, the sign of $\eta$ also controls the alignment of the magnetization. In the high temperature limit \req{ferro} with strictly $b_{0}=0$ assumes a form of to lowest order for brevity
\begin{align}
 \label{hiTferro}
 \lim_{m_{e}/T\rightarrow0}{\mathfrak M}\vert_{b_{0}=0}=\frac{g}{2}\frac{e^{2}}{\pi^{2}}\frac{T^{2}}{m_{e}^{2}}\frac{\eta}{T}\,,
\end{align}

While the limit in \req{hiTferro} was calculated in only the Boltzmann limit, it is noteworthy that the high temperature (and $m\rightarrow0$) limit of Fermi-Dirac distributions only differs from the Boltzmann result by a proportionality factor. 

The natural scale of the $e^{+}e^{-}$ magnetization with only a small spin fugacity ($\eta<1\eV$) fits easily within the bounds of the predicted magnetization during this era if the IGMF measured today was of primordial origin. The reason for this is that the magnetization seen in \req{g2magplus}, \req{g2magminus} and \req{ferro} are scaled by $\alpha{B}_{C}$ where $\alpha$ is the fine structure constant.

%%%%%%%%%%%%%%%%%%%%%%%%%%%%%%%%%%%%%%%
\subsubsection{Hypothesis of ferromagnetic self-magnetization}
\label{sec:self}
%%%%%%%%%%%%%%%%%%%%%%%%%%%%%%%%%%%%%%%
\noindent One exploratory model we propose is to fix the spin polarization asymmetry, described in \req{spotential}, to generate a homogeneous magnetic field which dissipates as the universe cools down. In this model, there is no external primordial magnetic field $({B}_\mathrm{PMF}=0)$ generated by some unrelated physics, but rather the $e^{+}e^{-}$ plasma itself is responsible for the field by virtue of spin polarization.

%%%%%%%%%%%%%%%%%%%%%%%%%%%%%%%%%%%%%%%
\begin{figure}[ht]
 \centering
 \includegraphics[width=0.9\textwidth]{plots/chap04cosmo/Spinchemical_03.png}
 \caption{The spin potential $\eta$ and chemical potential $\mu$ are plotted under the assumption of self-magnetization through a nonzero spin polarization in bulk of the plasma}
 \label{fig:self} 
\end{figure}
%%%%%%%%%%%%%%%%%%%%%%%%%%%%%%%%%%%%%%%

This would obey the following assumption of
\begin{align}
 \label{selfmag}
 {\mathfrak M}(b_{0})=\frac{\mathcal{M}(b_0)}{{B}_{C}}\longleftrightarrow\frac{B}{{B}_{C}}=b_{0}\frac{T^{2}}{m_{e}^{2}}\,,
\end{align}
which sets the total magnetization as a function of itself. The spin polarization described by $\eta\rightarrow\eta(b_{0},T)$ then becomes a fixed function of the temperature and magnetic scale. The underlying assumption would be the preservation of the homogeneous field would be maintained by scattering within the gas (as it is still in thermal equilibrium) modulating the polarization to conserve total magnetic flux.

The result of the self-magnetization assumption in \req{selfmag} for the potentials is plotted in \rf{fig:self}. The solid lines indicate the curves for $\eta/T$ for differing values of $b_{0}=\{10^{-11},\ 10^{-7},\ 10^{-5},\ 10^{-3}\}$ which become dashed above $T\!=\!300\keV$ to indicate that the Boltzmann approximation is no longer appropriate though the general trend should remain unchanged.

%%%%%%%%%%%%%%%%%%%%%%%%%%%%%%%%%%%%%%%
\begin{figure}[ht]
 \centering
 \includegraphics[width=0.95\textwidth]{plots/chap04cosmo/ElectronDensity_SpinChemicalPotential004.jpg}
 \caption{The number density $n_{e^{\pm}}$ of polarized electrons and positrons under the self-magnetization model for differing values of $b_{0}$. Figure courtesy of Cheng Tao Yang.}
 \label{fig:polarswap} 
\end{figure}
%%%%%%%%%%%%%%%%%%%%%%%%%%%%%%%%%%%%%%%

The dotted lines are the curves for the chemical potential $\mu/T$. At high temperatures we see that a relatively small $\eta/T$ is needed to produce magnetization owing to the large densities present. \rf{fig:self} also shows that the chemical potential does not deviate from the free particle case until the spin polarization becomes sufficiently high which indicates that this form of self-magnetization would require the annihilation of positrons to be incomplete even at lower temperatures.

This is seen explicitly in~\rf{fig:polarswap} where we plot the numerical density of particles as a function of temperature for spin aligned $(+\eta)$ and spin anti-aligned $(-\eta)$ species for both positrons $(-\mu)$ and electrons $(+\mu)$. Various self-magnetization strengths are also plotted to match those seen in~\rf{fig:self}. The nature of $T_{\rm split}$ changes under this model since antimatter and polarization states can be extinguished separately. Positrons persist where there is insufficient electron density to maintain the magnetic flux. Polarization asymmetry therefore appears physical only in the domain where there is a large number of matter-antimatter pairs.

%%%%%%%%%%%%%%%%%%%%%%%%%%%%%%%%%%%%%%%
\subsubsection{Matter inhomogeneities in the cosmic plasma}
\label{sec:inhomogeneous}
%%%%%%%%%%%%%%%%%%%%%%%%%%%%%%%%%%%%%%%
\noindent In general, an additional physical constraint is required to fully determine $\mu$ and $\eta$ simultaneously as both potentials have mutual dependency (see \rsec{sec:ferro}). We note that spin polarizations are not required to be in balanced within a single species to preserve angular momentum.

The CMB~\cite{Planck:2018vyg} indicates that the early universe was home to domains of slightly higher and lower baryon densities which resulted in the presence of galactic super-clusters, cosmic filaments, and great voids seen today. However, the CMB, as measured today, is blind to the localized inhomogeneities required for gravity to begin galaxy and supermassive black hole formation.

Such acute inhomogeneities distributed like a dust~\cite{Grayson:2023flr} in the plasma would make the proton density sharply and spatially dependant $n_{p}\rightarrow n_{p}(x)$ which would directly affect the potentials $\mu(x)$ and $\eta(x)$ and thus the density of electrons and positrons locally. This suggests that $e^{+}e^{-}$ may play a role in the initial seeding of gravitational collapse. If the plasma were home to such localized magnetic domains, the nonzero local angular momentum within these domains would provide a natural mechanism for the formation of rotating galaxies today.

Recent measurements by the James Webb Space Telescope (JWST)~\cite{Yan:2022sxd,adams2023discovery,arrabal2023spectroscopic} indicate that galaxy formation began surprisingly early at large redshift values of $z\gtrsim10$ within the first 500 million years of the universe requiring gravitational collapse to begin in a hotter environment than expected. The observation of supermassive black holes already present~\cite{CEERSTeam:2023qgy} in this same high redshift period (with millions of solar masses) indicates the need for local high density regions in the early universe whose generation is not yet explained and likely need to exist long before the recombination epoch.
