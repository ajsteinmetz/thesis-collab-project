%%%%%%%%%%%%%%%%
\section{Connecting Prior Works: Note to the Reader}\label{list:Works}
%%%%%%%%%%%%%%%%%%%%%%%%%%%%%%%%%%%%%%%%%%%%
In this review we have expanded considerably both in scope and content our recent review:
\begin{enumerate}
\item ``A Short Survey of Matter-Antimatter evolution in the Primordial Universe'' by \allcite{Rafelski:2023emw}, which focused on the role of antimatter in the primordial Universe. 
\end{enumerate}
The scope is widened by including in an edited and re-sequenced manner select material from the contents of four Ph.D. theses completed at the Department of Physics, The University of Arizona. {\color{black} These individual projects were written by authors of this review and address distinctly different events in the evolution of the primordial Universe:}
\begin{enumerate}
\setcounter{enumi}{1}
\item ``Non-Equilibrium Aspects of Relic Neutrinos: From Freeze-out to the Present Day'' by \aucite{Birrell:2014ona} studies the evolution of the relic (or cosmic) neutrino distribution from neutrino freeze-out at $T=\mathcal{O}(1)\MeV$ through the free-streaming era up to today. {\color{black} This work impacts our understanding of the values of natural constants in this primordial epoch. It is of great relevance to the ongoing discussion regarding speed of Universe expansion in the current epoch.}
%
\item ``Dense Relativistic Matter-Antimatter Plasmas'' by \aucite{Grayson:2024okq} explores dense electron-positron and quark-gluon plasma (QGP)\index{QGP!quark-gluon plasma} with strong electromagnetic fields generated during heavy-ion collisions and prevalent in extreme astrophysical environments. {\color{black} These methods also allow us to understand the long range cosmological magnetic fields if originating in the QGP phase.}
%
\item ``Modern topics in relativistic spin dynamics and magnetism'' by \aucite{Steinmetz:2023ucp} explores spin and magnetic moments in \emph{relativistic} mechanics from both
quantum and classical perspectives. A model of primordial magnetization in the Universe is presented, originating during the hot dense electron-positron plasma epoch. {\color{black} The methods developed in this work are fully portable to the study of magnetization originating in the earlier QGP epoch.}
%
\item ``Elementary Particles and Plasma in the First Hour of the Early Universe'' by \aucite{Yang:2024ret} deepens the understanding of the primordial composition of the Universe in the temperature range $300 \MeV>T>0.02\MeV$, which transits from quark-gluon plasma to hadron matter. {\color{black} Equilibrium and non-equilibrium particle abundances in the primordial Universe are explored and different primordial Universe epochs connected.}
\end{enumerate}

Due to graduation time constraints some of this presented material is only found in follow-up publications and in reports yet to be readied for publication. Many of these results are also included in this work. We further report in a large part on our research papers and related reports. Thus beyond the four theses, the key input too this review includes:
\begin{enumerate}
\setcounter{enumi}{5}
% \item {\small ``Self-consistent Strong Screening Applied to Thermonuclear Reactions''} by \allcitep{Grayson:2024uwg} explores strong screening effects in the BBN epoch due to dynamic and nonlinear polarization of the matter-antimatter (electron-positron) ambient medium. {\color{black} This work clarifies differences between models of screening and evaluates in a self-consistent manner short distance strong screening effects in the electron-positron plasma. Impact on the BBN reactions is presented.}
%
\item ``Matter-antimatter origin of cosmic magnetism'' by \allcite{Steinmetz:2023nsc} proposes a model of spin para-magnetization driven by the large matter-antimatter (electron-positron) content of the primordial Universe. {\color{black} The theoretical framework presented is general and allows us to seek the origins of Universe magnetization in all epochs containing high antimatter components, including the other most interesting case, the QGP.}
%
\item ``Electron-positron plasma in BBN: Damped-dynamic screening'' by \allcite{Grayson:2023flr} employs the linear response theory to describe the inter-nuclear potential screened by in electron-positron pair plasma in the BBN epoch. This work includes also the computation of the chemical potential and plasma damping rate required in semi-analytical study of the relativistic Boltzmann equation in the context of the linear response theory. {\color{black} The methods of linear response theory presented in this work lay the foundation for many yet to come studies of interacting particles and plasma in the primordial Universe, example is above Ref.\,\cite{Grayson:2024uwg}.}
%
\item ``Dynamic magnetic response of the quark-gluon plasma to electromagnetic fields'' by \allcite{Grayson:2022asf} applies linear response method to characterize the quark-gluon plasma environment in the presence of strong magnetic fields. {\color{black} Future study of magnetization phenomena in the primordial QGP can rely on results obtained in this work.}
%
\item ``Cosmological Strangeness Abundance'' by \allcite{Yang:2021bko}[Rafelski] presents strange particle abundances in the expanding primordial Universe including determination of strange particle freeze-out temperatures. {\color{black} Considerable difference with experimental relativistic heavy ion environment arises due to importance of reactions absent at the time scale of laboratory experiments, \eg\ lepton fusion $l^+ +l^-\to\phi(1020)\to \mathrm{K}+\overline{\mathrm K}$ strangeness producing reactions.}
\item ``Current-conserving Relativistic Linear Response for Collisional Plasmas'' by \allcite{Formanek:2021blc} develops relativistic linear response plasma theory implementing conservation laws, obtaining general solutions and laying the foundation for applications to primordial Universe plasma conditions. {\color{black} Of great relevance to both laboratory and cosmological environments is the allowance for the damping term.}
%
\item ``The Muon Abundance in the Primordial Universe'' by \allcite{Rafelski:2021aey}[Yang] is a conference proceedings paper dedicated to exploration of muon abundance and its persistence temperature in the primordial Universe. {\color{black} One can wonder if it is a mere coincidence that muons disappear from the Universe particle inventory when their ratio to net baryon number is nearly unity.}
%
\item ``Reactions Governing Strangeness Abundance in Primordial Universe'' by \allcite{Rafelski:2020ajx}[Yang] is a conference proceeding paper that lays the groundwork for the study of strangeness reactions in the primordial Universe. {\color{black} This work is a prequel to the Ref.\,\cite{Yang:2021bko} described above. It is more accessible in some of technical details.}
%
\item ``Possibility of bottom-catalyzed matter genesis near to primordial QGP hadronization'' by \allcitep{Yang:2020nne}[Rafelski] was our fist study of the bottom flavor abundance in the primordial Universe. {\color{black} We show unexpected appearance of chemical nonequilibrium behavior near to QGP hadronization. Follow-up investigations of allowed us to recognize the presence of non-stationary heavy particle abundances in the primordial QGP required for baryogenesis, a topic which awaits our near term attention.}
%
\item ``Lepton Number and Expansion of the Universe'' by \allcitep{Yang:2018oqg} proposes a model of large lepton asymmetry and explores how this large cosmological lepton yield relates to the effective number of (Dirac) neutrinos. {\color{black} This work explores several cosmological Universe properties freed from the $B-L=0$ constraint: The conservation of baryon minus lepton number. This is highly relevant to the present day Hubble parameter $H_0$ value as additional neutrino pressure wanes with time due to emergence of neutrino mass.}
%
\item ``Temperature Dependence of the Neutron Lifespan'' by \allcitep{Yang:2018qrr} is a study of neutron lifespan in primordial cosmic plasma with Fermi-blocking of the decay electrons and neutrinos. {\color{black} The widely discussed dual value of neutron lifespan depending on measurement method stimulated our interest in understanding primordial neutron response to ambient conditions. A study of the influence of strong magnetic fields in both cosmological and laboratory environment remains on our to-do list.}
%
\item ``Strong fields and neutral particle magnetic moment dynamics'' by \allcite{Formanek:2017mbv} is an overview of our research group's early efforts to understand neutral particle dynamics in the presence of (strong) electromagnetic fields. {\color{black} We explore the dynamics as function of the $g$-factor which can assume arbitrary values. The cosmic neutrino response to strong magnetic fields remains on to-do list.}
%
\item ``The hot Hagedorn Universe'' by \allcite{Rafelski:2016cho}[Birrell] {\color{black} recounts and updates the impact of Hagedorn's work on phase transformation at Hagedorn temperature in the primordial Universe. Hagedorn had a pivotal impact on the cosmic paradigm as his statistical bootstrap model of point-sized hadrons allowed a singular energy density at a temperature near to pion mass equivalent. This paradigm was dramatically expanded allowing for hadrons of finite size.}
%
\item ``Relic Neutrino Freeze-out: Dependence on Natural Constants'' by \allcite{Birrell:2014uka} is a study of neutrino freeze-out temperature as a function of standard model parameters and its application on the effective number of neutrinos. {\color{black} The presence of the a background neutrino radiation is impacting the speed of Universe expansion today. The magnitude of unseen neutrino component depends on our belief in the constancy of fundamental constants near to onset of neutrino free-streaming. In this work we investigate, in a quantitative manner, this relation showing  that the Weinberg angle and a particular combination of several other natural constants determine the present day Hubble parameter $H_0$. This work provides all neutrino-matter weak interaction matrix elements required for the Boltzmann transport code developed and needed to obtain these result.}
%
\item ``Quark–gluon Plasma as the Possible Source of Cosmological Dark Radiation'' by \allcite{Birrell:2014cja}[Rafelski] {\color{black} explores the role of hypothetical dark radiation particles decoupling at the time of QGP hadronization. The presence of dark radiation is often invoked to fine-tune models of Universe dynamics. Our contribution argues that the freezing of color degrees of freedom could liberate dark particles. We evaluate the effect of a single `dark' particle and show how it is reduced in impact due to reheating of visible particles. This work was dedicated to the study of nearly massless particles; a similar origin and study could be developed to model unobserved massive dark matter component in the Universe. }
%
\item ``Boltzmann Equation Solver Adapted to Emergent Chemical Non-equilibrium'' by \allcite{Birrell:2014gea} addresses the transport theory tools we developed to characterize the slow in time freeze-out of neutrinos in the primordial Universe. {\color{black} There are several species of neutrinos that undergo in the temperature interval $1\MeV<T<4\MeV$ a gradual decoupling from the electron-positron background: This manuscript presents the blueprint of a novel dynamical solution of the decoupling (freeze-out) using moving basis method.}
%
\item ``Proposal for Resonant Detection of Relic Massive Neutrinos'' by \allcite{Birrell:2014qna}[Rafelski] characterizes the primordial neutrino flux spectrum today and explores experimental approaches for experimental observations. {\color{black} The ambient cosmic microwave background (CMB) temperature today is $kT_0=0.235\meV$, well below the mass of two of the three neutrinos which thus move relatively slowly compared to the cosmic Earth motion. This situation is offering unusual detection strategy opportunities. Future work is required to reach definitive conclusions about primordial neutrino background detection.}
%
\item ``Traveling Through the Universe: Back in Time to the Quark-Gluon Plasma Era'' by \allcite{Rafelski:2013yka}[Birrell] presents a first study of the connection between quark-gluon plasma and neutrino freeze-out epochs. {\color{black} This is a decade old prequel to this work, comparing to it we see enormous progress made since.}
%
\item ``Connecting QGP-Heavy Ion Physics to the Early Universe'' by \aucite{Rafelski:2013qeu} explores the properties of the primordial Universe at QGP hadronization and connects to the ongoing experimental heavy-ion effort to study the hadronization process. {\color{black} This detailed look at the primordial QGP hadronization helped define several projects which are described in this report.}
%
\item ``Fugacity and Reheating of Primordial Neutrinos'' by \allcite{Birrell:2013gpa} is a study of neutrino fugacity as a function of neutrino kinetic freeze-out temperature. This short work includes {\color{black} neutrino interaction matrix elements and paves the way for the evaluation of neutrino relaxation time. This is a conference report driven prequel to later detailed work on neutrino freeze-out.}
%
\item ``Relic Neutrinos: Physically Consistent Treatment of Effective Number of Neutrinos and Neutrino Mass'' by \allcite{Birrell:2012gg} is a model-independent study of the neutrino momentum distribution at freeze-out, treating the freeze-out temperature as a free parameter. {\color{black} The speed of Universe expansion depends on correct accounting of the free-streaming massive neutrinos which this work offers.}
%
\item ``From Quark-Gluon Universe to Neutrino Decoupling: $200 < T < 2\MeV$'' by \allcite{Fromerth:2012fe} {\color{black} connects the Quark-Hadron phase transformation with neutrino decoupling as a function of the current era cosmological properties. A conference report addressing ongoing effort to connect primordial QGP to neutrino decoupling eras.}
%
\item ``Unstable Hadrons in Hot Hadron Gas in Laboratory and in the Early Universe'' by \allcite{Kuznetsova:2010pi}[Rafelski] shows that some unstable hadrons may persist in the evolution of the primordial Universe as the detailed balance condition is never broken due to strong coupling to the photon background. {\color{black} This is preparatory work providing methods and initial results for the study of hadrons and heavy leptons in the primordial Universe.}
%
\item ``Hadronization of the Quark Universe'' by \allcite{Fromerth:2002wb}[Rafelski] is {\color{black} a path-breaking work which established the tools needed to describe the primordial quark-hadron phase of matter.} It includes a first detailed study of chemical potentials and conditions of hadronization of QGP in the primordial Universe. 
%
\item {\color{black}``Hadrons and Quark–Gluon Plasma''~\allcite{Letessier:2002ony}[Rafelski] is a textbook where the introductory chapters connect the properties of the primordial Universe with the relativistic heavy-ion collision experimental program. This was a first effort to explain the importance of the ongoing experimental effort in the context of creating understanding of the primordial Universe as described in this report.}
\end{enumerate}
Additionally, material adapted from Refs.~\cite{Rafelski:2019twp,Rafelski:2016hnq,Rafelski:2015cxa} has been included. This lets us address strong interactions and quark-gluon plasma (QGP)\index{QGP!quark-gluon plasma} hadronization\index{QGP!hadronization} in the Universe: (i) Deconfined states of hot quarks and gluons, the QG; and (ii) Hot hadronic phase of matter, also called hadronic gas\index{hadrons!gas phase}, as applicable to the context of the primordial Universe. 
