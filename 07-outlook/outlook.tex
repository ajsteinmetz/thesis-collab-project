%%%%%%%%%%%%%%%%%%%%%%%%%%% 
\section{Discussion and Summary}\label{part6}
%%%%%%%%%%%%%%%%%%%%%%%%%%%
We have presented a compendium of theoretical models addressing the particle and plasma content of the  primordial Universe. The Universe at a temperature above 10\keV\ is dominated by `visible' matter, dependence on unknown dark matter and dark energy is minimal. However any underlying dark component will later surface, thus the understanding of this primordial epoch also as a source of darkness in present day Universe is of profound relevance.

Select introductory material addressing kinetic theory, statistical physics, and general relativity is presented. Kinetic and plasma theory is described in greater detail. Einstein's gravity theory found in many other sources is limited to the minimum required in the study of the primordial Universe within the confines of the FLRW cosmology model. 

In this work we are connecting several of our prior and ongoing studies of the cosmic particle plasma in the primordial Universe. The three primary  eras: radiation, matter, dark energy dominance, can be recognized in terms of the acceleration parameter $q$. We present this tool connecting these distinct epochs smoothly. Detailed results concerning time and temperature relation allowing for the reheating of the Universe were shown. Entropy transfer (reheating) inflates the Universe expansion whenever ambient temperature is too low to support the massive particle abundance.

In detailed studies we explore particle abundances and plasma properties which improve our comprehensive understanding of the Universe in its evolution. Many interesting phenomena in primordial Universe depend on nonequilibrium conditions and this topic is at the core of our theoretical interest. Nuance differences between kinetic and chemical equilibrium, dynamic but stationary detailed balance and non-stationary phenomena recur as topics of interest in our discussion. 

We recognize in detail the deviations from thermal equilibrium, particle freeze-out, and dynamic detailed balance away from the thermal equilibrium condition. These nonequilibrium concepts are pivotal in our opinion in recognizing any remnant observable of the primordial Universe. This is also what drives our interest in heavy quark dynamics and neutrino decoupling. 

The different epochs in Universe evolution are often considered as being distinctly separate. However, we have shown that this is not quite the case with significant overlaps identified. The primordial nucelosynthesis  epoch is embedded entirely into a very significant electron-positron pair plasma background which fades out well after BBN ends. 

Another example is the `squeeze' of neutrino decoupling between: The electron-positron annihilation reheating of photons at the low temperature edge; and heavy lepton (muon) disappearance on the  high-$T$ edge: this finetuning prompted our study of the neutrino decoupling as a function of the magnitude of governing natural constants. Our detailed study of the neutrino background shows future potential  to reconcile observational tensions that arise in the reported present day speed of Universe expansion. This depends on better understanding of the dynamics of free-streaming particles across mass thresholds. 

Outside of our primary domain of interest, the coincidental multiple crossing of different visible energy components in the Universe seen near to $T=0.25\meV$ in \rf{fig:energy:frac}, that is at condition of recombination is equally an unexpected coincidence trggered by the assumed contemporary ratio of dark and visible energy components employed to create our time dependent pie-chart Universe composition image. The analysis of cosmic microwave data which originates this, is not retold here.

Sceptics could interpret the appearance of several such coincidences as indicative of a situation akin to pre-Copernican epicycles. We note that current standard model of cosmology is being challenged by Fulvio Melia~\cite{Melia:2022itm} ``One cannot avoid the conclusion that the standard model needs a complete overhaul in order to survive.'' or by the same author~\cite{Melia:2024rzy}  ``\ldots the timeline in $\Lambda$CDM is overly compressed at $z\ge 6$, while strongly supporting the expansion history in the early Universe predicted by\ldots" the Melia model of cosmology. 

In our work there is no evidence found for a more rapid expansion of primordial Universe. We believe that in order to argue for or against certain models of primordial cosmology  we need first to establish the model properties very well and this has not been accomplished for particles and plasma Universe at $130\GeV\le T\le 10\keV$. Thus declarations about the particles and plasma Universe  based on a few  atomic, molecular, stelar phenomena observed in redshif $z=6\simeq7$ seem premature. 

One important aspect of the hot primordial Universe is experimental access to the study of the melting of matter into constituent quarks at high enough temperature. We recalled the 50 years of effort which begun with the recognition  of novel structure in the primordial Universe beyond the Hagedorn temperature, and the exploration of this high temperature deconfined quark-gluon phase using as the laboratory tool the ultra relativistic heavy-ion collision experiments. 

The study of the phase transformation between confined hadrons and deconfined quark-gluon plasma in laboratory facilitates the understanding of the primordial Universe to the earliest instants after its birth, about 20-30\, $\mu$s after the Big-Bang. The idea that one could recreate this Big-Bang condition in laboratory was the beginning of the modern interest in better understanding the structure of the primordial Universe. 

The experimental study in the laboratory micro-bang  stimulates development of detailed models of the strongly interacting hadron era of the Universe. We do not review here this very large volume of ongoing work beyond the scope of this article. The question, is the quark-gluon plasma observable in laboratory to be different from the hadron Universe content is mentioned. We use some of the developed tools to study properties of hadronic matter in the Universe and strangeness flavor freeze-out.



In Section \ref{sec:Boltzmann_Einstein} we provided background on the Boltzmann-Einstein equation, including proofs of conservation laws and Boltzmann's H-theorem for interactions between any number of particles; to our knowledge, proofs for general $m\to n$-particle interactions are not available in other references on the subject.

To facilitate the characterization of neutrino freeze-out and constrain the time variation of natural constants, in Appendix \ref{ch:boltz:orthopoly} we developed a novel computationally efficient moving-frame numerical method.

%%%%%%%%%%%%%%%%%%%%%%%%%%%%%%%%%%%%%%%
\para{REWORK INTO ABOVE: Matter-antimatter origin of cosmic magnetism}
%%%%%%%%%%%%%%%%%%%%%%%%%%%%%%%%%%%%%%%
We characterized the primordial magnetic properties of the early universe before recombination. We studied the temperature range of $2000\keV$ to $20\keV$ where all of space was filled with a hot dense electron-positron plasma (to the tune of 450 million pairs per baryon) which occurred within the first few minutes after the Big Bang. We note that our chosen period also includes the era of Big Bang Nucleosynthesis.

We found that subject to a primordial magnetic field, the early universe electron-positron plasma has a significant paramagnetic response (see \rf{fig:magnet}) due to magnetic moment polarization. We considered the interplay of charge chemical potential, baryon asymmetry, anomalous magnetic moment, and magnetic dipole polarization on the nearly homogeneous medium.

This novel approach to high temperature magnetization shows that the $e^{+}e^{-}$-plasma paramagnetic response (see~\req{g2magplus} and~\req{g2magminus}) is dominated by the varying abundance of electron-positron pairs, decreasing with decreasing $T$ for $T\!<\!m_{e}c^2$. This is unlike conventional laboratory cases where the number of magnetic particles is constant. 

We find that electron-positron magnetization rapidly vanishes as the number of pairs depletes as the universe cools. This therefore presents an opportunity for induced currents to facilitate inhomogeneities in the early universe. We also presented a simple model of self-magnetization of the primordial electron-positron plasma which indicates that only a small polarization asymmetry is required to generate significant magnetic flux when the universe was very hot and dense.

 