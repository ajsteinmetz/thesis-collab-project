%%%%%%%%%%%%%%%%%%%%%%%%%%% 
\section{Discussion and Summary}\label{part6}
%%%%%%%%%%%%%%%%%%%%%%%%%%%
We have presented a compendium of theoretical models addressing the particle and plasma content of the primordial Universe. The Universe at a temperature above 10\keV\ is dominated by `visible' matter, dependence on unknown dark matter and dark energy is minimal. However any underlying dark component will later surface, thus the understanding of this primordial epoch also as a source of darkness in the present day Universe is of profound relevance.

Select introductory material addressing kinetic theory, statistical physics, and general relativity is presented. Kinetic and plasma theory is described in greater detail. Einstein's gravity theory found in many other sources is limited to the minimum required in the study of the primordial Universe within the confines of the FLRW cosmology model. 

In this work we are connecting several of our prior and ongoing studies of the cosmic particle plasma in the primordial Universe. The three primary eras: radiation, matter, dark energy dominance, can be recognized in terms of the acceleration parameter $q$. We introduce this tool in cosmology primer \rsec{sec:flrw} connecting these distinct epochs smoothly in \rsec{sec:dynamic}. Detailed results concerning time and temperature relation allowing for the reheating of the Universe were shown. Entropy transfer (reheating) inflates the Universe expansion whenever ambient temperature is too low to support the massive particle abundance.

In detailed studies we explored particle abundances and plasma properties which improve our comprehensive understanding of the Universe in its evolution. Many interesting phenomena in primordial Universe depend on nonequilibrium conditions and this topic is at the core of our theoretical interest. Nuance differences between kinetic and chemical equilibrium, dynamic but stationary detailed balance and non-stationary phenomena recur as topics of interest in our discussion. 

One important aspect of the hot primordial Universe is experimental access to the study of the melting of matter into constituent quarks at high enough temperature. We recalled the 50 years of effort which begun with the recognition of novel structure in the primordial Universe beyond the Hagedorn temperature, and the exploration of this high temperature deconfined quark-gluon phase using as the laboratory tool the ultra relativistic heavy-ion collision experiments. 

The study of the phase transformation between confined hadrons and deconfined quark-gluon plasma in laboratory facilitates the understanding of the primordial Universe to the earliest instants after its birth, about 20-30\, $\mu$s after the Big-Bang. The idea that one could recreate this Big-Bang condition in laboratory was the beginning of the modern interest in better understanding the structure of the primordial Universe. 

The experimental study in the laboratory micro-bang stimulates development of detailed models of the strongly interacting hadron era of the Universe. We do not review here this very large volume of ongoing work beyond the scope of this article. The question, is the quark-gluon plasma observable in laboratory to be different from the hadron Universe content is mentioned. 

We use some of the developed tools to study properties of hadronic matter in the Universe and strangeness flavor freeze-out in particular in \rsec{Strangeness}. We recognize in detail the deviations from thermal equilibrium, particle freeze-out, and dynamic detailed balance away from the thermal equilibrium condition for bottom quarks in \rsec{Bottom}. These nonequilibrium concepts are pivotal in our opinion in recognizing any remnant observable of the primordial Universe. 

These insights drive our interest leading beyond our interest in strangness and bottom quarks to all heavy PP-SM particles. We question the potential that primordial QGP era harbors for baryogenesis both for the bottom quarks and the Higgs particle induced reactions, \rsec{HiggsQGP}.

The different epochs in the Universe evolution are often considered as being distinctly separate. However, we have shown that this is not quite the case. We note the `squeeze' of neutrino decoupling between: The electron-positron annihilation reheating of photons at the low temperature edge at about $T=1\MeV$; and heavy lepton (muon) disappearance on the high-$T$ edge at about $T=4.5\MeV$. 

This fine-tuning into a narrow available domain prompted our study of the neutrino decoupling as a function of the magnitude of governing natural constants. To facilitate the characterization of neutrino freeze-out and constrain the time variation of natural constants, in Appendix \ref{ch:boltz:orthopoly} we developed a novel computationally efficient moving-frame numerical method.

Our detailed study of the neutrino background shows future potential to reconcile observational tensions that arise in the reported present day speed of Universe expansion. This potentially could depend on better understanding of the dynamics of free-streaming particles across mass thresholds. We noted this opportunity to deepen the understanding of the background quantum neutrino liquid.

In~\rsec{sec:BoltzmannEinstein} we provided background on the Boltzmann-Einstein equation, including proofs of conservation laws and the Boltzmann's H-theorem for interactions between any number of particles; this is of interest as the evolution of the Universe often requires detailed balance involving more than two particle scattering. To our knowledge, proofs for general numbers $m$, $n$ with $m\to n$-particle interactions are not available in other references on the subject.

Another example of era overlap, in this case highly significant, is the recognition that primordial nucelosynthesis BBN epoch is embedded entirely into a electron-positron pair plasma background which fades out well after BBN ends. This effect is clearly visible but maybe is not fully appreciated when inspecting in~\rf{fig:energy:frac}: We see in the overlap of $e^+e^-$-component with the era of BBN as marked a ``small'' $e^+e^-$-energy fraction. It seems that the $e^+e^-$-pair plasma is in process of disappearance and does not matter. This is a wrong first impression,  starting with a giant $10^9$ pair ratio over nucleon dust there is still a huge $e^+e^-$-pair abundance left in BBN epoch with millions of pairs per nucleon at high $T$ range.

This ratio of $e^+e^-$-pair abundance to baryon number was studied  in detail in \rf{fig:densityratio} (see also \rf{BBN:Electron}  right ordinate):  As a curious tidbit let us note that as long as there are more than a few thousand $e^+e^-$-pairs per nucleon the antimatter content in the universe is practically symmetric with matter as the nuclear dust is not tilting the balance as matter are electrons and antimatter positrons. Thus it is not entirely correct to consider the disappearance of of antibaryons , see \rf{Baryon:fig}, at $T\simeq 38.2\MeV$ as the end of antimatter epoch. It is instead correct to view the temperature $T=30\keV$ as the onset of antimatter disappearance which completes at $T=20.3\keV$ as is seen in~\rf{fig:densityratio}.

Discussion of the dense charged particle background during BBN constitutes a major part of this work. In~\rsec{part4} we developed a covariant kinetic theory to analyze the influence of $e^+e^-$-pair plasma polarization. We solve the kinetic theory using linear response considering both spatial and temporal dispersion, focusing on how the covariant polarization tensor, which includes collisional damping, shapes the self-consistent electromagnetic fields within the medium. This approach allows us to elucidate the intricate dynamics and damping effects that characterize the behavior of these plasmas.

We explore the impact of damped-dynamic screening in electron-positron plasmas during Big-Bang Nucleosynthesis (BBN). Our findings indicate that screening plays a critical role in modifying internuclear potentials and nuclear fusion reaction rates, although the effect during BBN is a minor correction to the usual screening enhancement. Despite the significant damping and high temperatures characteristic of BBN, the enhancement in nuclear reaction rates due to this screening remains relatively small, around $10^{-5}$, yet it provides a valuable refinement to our understanding of the early universe's conditions.

Extending our analysis to QGP in \rsec{chap:QCD}, we particularly examine the magnetic field response under ultrarelativistic conditions during heavy ion collisions. By employing various conductivity models, we demonstrate that the conductivity evaluated on the light-cone effectively describes the evolution of magnetic fields within the QGP. This insight leads us to derive an analytic formula that predicts the freeze-out magnetic field, potentially enabling experimental determination of the QGP's electromagnetic conductivity—a key parameter in understanding the plasma's properties during these extreme events.


The long lasting (in relative terms) antimatter $e^+e^-$-pair  plasma offers an opportunity to consider a novel mechanism of magneto-genesis in primordial Universe: Extrapolating the intergalactic fields observed in the current era back in time to the $e^+e^-$-pair plasma era magnetic fields are encountered which approach the strength of the surface magnetar fields~\rsec{sec:mag_overview}.  

This has prompted our interest to study $e^+e^-$-pair plasma as source of Universe magnetization. We studied the temperature range of $2000\keV$ to $20\keV$ where all of space was filled with a hot dense electron-positron plasma (to the tune of 450 million pairs per baryon) which occurred within the first few minutes after the Big-Bang. We note that our chosen period also includes the era of Big-Bang Nucleosynthesis.

We found that subject to a primordial magnetic field, the early universe electron-positron plasma has a significant paramagnetic response, see~\rf{fig:magnet} due to magnetic moment polarization. We considered the interplay of charge chemical potential, baryon asymmetry, anomalous magnetic moment, and magnetic dipole polarization on the nearly homogeneous medium.

We presented a simple model of self-magnetization of the primordial electron-positron plasma which indicates that only a small polarization asymmetry is required to generate significant magnetic flux when the universe was very hot and dense.

Our novel approach to high temperature magnetization, see Chapter~\ref{sec:mag_universe} shows that the $e^{+}e^{-}$-plasma paramagnetic response (see~\req{g2magplus} and~\req{g2magminus}) is dominated by the varying abundance of electron-positron pairs, decreasing with decreasing $T$ for $T\!<\!m_{e}c^2$. This is unlike conventional laboratory cases where the number of magnetic particles is constant. We find that electron-positron spin magnetization cannot be maintained as the number of pairs depletes as the universe cools. This therefore presents an opportunity for induced currents to facilitate magnetic and potentially matter inhomogeneity in the early universe. 
 
Outside of our primary domain of interest, the coincidental multiple crossing of different visible energy components in the Universe seen near to $T=0.25\meV$ in \rf{fig:energy:frac}, that is at condition of recombination is equally an unexpected coincidence triggered by the assumed contemporary ratio of dark and visible energy components employed to create our time dependent pie-chart Universe composition image. The analysis of cosmic microwave data which originates this, is not retold here. This special situation depends directly on the interpretation of our current era in terms of specific of matter and darkness. However, this influence propagates on to the primordial times in the particles and plasma Universe and provide for the overlaps we reported in regard of earlier eras.

Sceptics could interpret the appearance of several such coincidences as indicative of a situation akin to pre-Copernican epicycles. We note that current standard model of cosmology is being challenged by Fulvio Melia~\cite{Melia:2022itm} ``One cannot avoid the conclusion that the standard model needs a complete overhaul in order to survive.'' or by the same author~\cite{Melia:2024rzy} ``\ldots the timeline in $\Lambda$CDM is overly compressed at $z\ge 6$, while strongly supporting the expansion history in the early Universe predicted by\ldots" the Melia model of cosmology. 

In our work there is no evidence found for a more rapid expansion of primordial Universe. We believe that in order to argue for or against certain models of primordial cosmology we need first to establish the model properties very well and this has not been accomplished for particles and plasma Universe at $130\GeV\le T\le 10\keV$. Thus declarations about the particles and plasma Universe based on a few atomic, molecular, stelar phenomena observed in redshift $z=6\simeq7$ are not convincing, as are many not addressed here speculations about the properties of the Universe prior to formation of particle zoo with properties we have explored in laboratory. 

This reminds of historical backdrop of this work we already noted: Before 1972 there was no inkling about particle physics standard model, we were searching to understand the primordial Universe based on a thermal hadron model. Hagdorn's bootstrap approach was particularly welcome as the exponential mass spectrum of hadronic resonances generated divergent energy density for point-sized hadrons. This well known result allowed the hypothesis that there is a maximum (Hagedorn) temperature in the Universe. 

This argument had excellent footing: We needed to accomodate the energy content we observe in the infinite Universe. A divergence of energy at a singular start point converts to a divergence in space. As soon as experiments in laboratory clarified our understanding of fundamental particle physics ,this narrative collapsed within weeks as one of us  (JR) saw in late 70's at CERN working with Hagedorn in his office long hours. 

The outcome of more than 50 years of ensuing effort is seen in these pages. We presented here  the Universe within the realm of the known laws of physics. There are many `loose' ends as the reader will note turning pages, we show and tell clearly. It is  clear that we cannot tell as yet what happened `before' our PP-SM begins at $T\simeq 130\GeV$. Many key dynamic details charcterizing evolution before recombination at $T=0.25\eV$ need to be resolved, and the evolution of atomic and molecular Universe presents another challenge we did not mention at all. Still, we note the particles and plasma Universe based on PP-SM spans a 12 orders of magnitude temperature window $  130\GeV > T > 0.25\eV $. And, we need to be sure how the following atomic and molecular Universe evolved with all the darkness in it. To conclude, it is not the time to argue that the primordial particles and plasma Universe is needing a new better framework in cosmology.   
