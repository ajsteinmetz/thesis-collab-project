% Cheng Tao Su
\sect{Sionummary and conclusion}\label{Summary}
%{Introduction\daggerfootnote{This chapter has been published previously as \citet{Gottbrath1999}.}}
We have studied the evolution of the early Universe from the QGP epoch to the $e^\pm$ plasma $300\,\mathrm{MeV}>T>0.02\,\mathrm{MeV}$. Our effort focuses on understanding how the plasma state impacts and modifies elementary interactions between particles and thus the properties of the early Universe.
In the QGP epoch, 
we investigate the heavy-quark bottom and charm near the QGP hadronization temperature $300\,\mathrm{MeV}>T>150$ MeV. We show that the  faster quark-gluon pair fusion keeps the charm quarks in chemical equilibrium up until hadronization. After hadronization, charm quarks form heavy mesons that decay quickly into multi-particles and causes charm quarks to vanish from inventory of particles in the Universe. For the bottom quarks, the
quark production rate competes with bottom decay rate as a function of temperature. When the Universe's temperature is near to the QGP phase transition $300\,\mathrm{MeV}>T>150$ MeV, the bottom quark breaks the detail balance and disappearance from particle inventory provides the arrow in time and which can be the ‘sweet-spot’ for Sakharov conditions.

After hadronization, the free quarks and gluons become confined within baryon/mesons and the Universe becomes hadronic-matter dominated. In the temperature range $ 150\,\mathrm{MeV}>T>20\,\mathrm{MeV}$, the Universe is rich in physics phenomena involving strange mesons and (anti)baryons including (anti)hyperon abundances. 
Considering the inventory of the Universe  with strange mesons and baryons, the antibaryons disappear from the Universe at temperature $T=38.2$ MeV. Strangeness can be produced by the inverse decay reactions via weak, electromagnetic, and strong interactions in the early Universe until $T\approx13$ MeV. For $T>20$ MeV, the strangeness is dominantly present in the meson. For $20 >T > 13$ MeV, the strangeness can present in the hyperons and anti-strangeness in kaon keeping symmetry $s=\overline{s}$. Below the temperature $T<13$ MeV a new regime opens up in which the tiny residual strangeness abundance in hyperons and is governed by weak decays with no reequilibration with mesons.


When the temperature in the Universe is about $T\approx10$ MeV, the main ingredients that control the Universe evolution are: photons, neutrinos, electrons and positrons. The massive meson, baryon, and $\mu^\pm$ and $\tau^\pm$ are also present, but their small number density can be neglected when considering the energy density of the Universe expansion. For temperature $10\,\mathrm{MeV}>T>2$ MeV the Universe is controlled by the weak interaction between neutrinos and matter. We explore the neutrino coherent and incoherent scatterings with matter and apply them to study the neutrino freeze-out (incoherent scattering) in the early Universe and  long wavelength cosmic neutrino-atom scattering (coherent scattering). After neutrino freeze-out $T_f\approx2\,\mathrm{MeV}$, it becomes free-streaming and the number of neutrinos is independently conserved. We investigate the relation between the lepton number $L$ and the effective number of neutrinos $N^{\mathrm{eff}}_\nu$, and explore the impact of a large cosmological lepton asymmetry on the Universe evolution. Instead of $B\simeq |L|$, we found that $0.4\leqslant|L| \leqslant0.52$ and $B\simeq 1.33\times 10^{-9}|L|$ reconciles the CMB and current epoch results for the Hubble expansion parameter. 
 


Considering the temperature after neutrino freeze-out: $2\,\mathrm{MeV}>T>0.02$ MeV, the cosmic plasma is dominated by the photon, electrons and positrons. The massive $\mu^\pm$ abundance disappears at $T_\mathrm{disappear}=4.195$ MeV as soon as the muon decay rate becomes faster than muon production rate. The muon-baryon density ratio at  muon persistence temperature is equal to $n_\mu^\pm/n_B(T_\mathrm{disappear})\approx0.911$, which implies that the muon abundance could influence baryon evolution because muon number density is comparable to the baryon number density. 

We demonstrate that the presence of rich electron-positron plasma can last until temperature $T=20.3$ keV by calculating the chemical potential of electrons in the universe that maintains charge neutrality and entropy conservation.
We evaluate the microscope damping rate with temperature dependent photon mass in electron-positron plasma which is one important variable to study the inter-nuclear electromagnetic potentials  in damped plasma. Finally  we examine the magnetization process within dense electron-positron plasma and show that it has paramagnetic properties when subjected to an external field. Our study of magnetization can provide insights to understand the magnetized plasma and the possible origin of primordial magnetic field and  developing methods for
future detailed study.

Looking forward, we will refine our understanding of the evolution of the early Universe by studying the topics that follow. For electron-positron plasma: we will improve our calculation on the damping rate $\kappa$ and develop a self-consistent approach where both damping and photon properties in plasma are determined in a mutually consistent manner. For the neutrino sector: we will examine the sources of extra neutrino and entropy transfer from microscope process after neutrino freezeout in details. 
%Applying our understanding of neutrino-matter coherent scattering and effective potential to further explore the relic neutrino in early Universe. 
By studying these phenomena, we aim to enhance our understanding of the role of neutrinos in the evolution of universe. For the heavy particle in QGP: we will focus on understanding the behavior of heavy quarks in extreme environments, and refine our study of nonequilibrium bottom and its potential applications on baryongenesis at low temperature. We will also study the dynamic Higgs abundance under the competition between production and decay to understand the departure from equilibrium of Higgs in the early Universe.


In summary, our research offers an initial glimpse into the first hour of the Universe's history, exploring the intricate relationship between fundamental particles and plasma in the early Universe. We hope that the output of our study will be beneficial to all parties concerned while at the same time contribute to the knowledge enhancement in the understanding of the early Universe.

