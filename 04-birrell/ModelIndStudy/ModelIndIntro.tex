\chapter{Relic neutrinos: Physically consistent treatment of effective number of neutrinos and neutrino mass.}\label{app:model_ind}

J. Birrell, C.T. Yang, P. Chen, J. Rafelski, Phys. Rev. D 89, 023008 (2014)
DOI: 10.1103/PhysRevD.89.023008

\section*{ Summary}

In this paper, we performed a model independent characterization of the neutrino distribution after freeze-out as a function of the kinetic freeze-out temperature $T_k$.  We showed how chemical non-equilibrium, in the form of a fugacity $\Upsilon<1$, emerges during freeze-out as a result of $T_k$ being on the order of the electron mass, the period when $e^\pm$ annihilation begins in earnest.

Using conservation of energy and entropy we were able to compute the neutrino fugacity, the photon to neutrino reheating temperature ratio $T_\gamma/T_\nu$, and the effective number of neutrinos $N_\nu$, all as functions of $T_k$.  In particular, we presented an analytic derivation of an approximate power law relation between the  reheating ratio and the fugacity.  We also showed that a delayed neutrino freeze-out is capable of matching the value of $N_\nu>3$, as seen in the recent Planck CMB results.

We also derived an analytic expression for the free-streaming neutrino distribution after freeze-out and used that to find fits of the neutrino energy density and pressure as functions of both the observed value of $N_\nu$ and the neutrino mass.  Such a parameterization is required in order to include the effects of both neutrino mass and neutrino reheating, $N_\nu>3,$ into CMB studies in a physically consistent way.  These constitute a new insight that was made possible by the model independent approach.  Prior studies had difficulty when attempting to include both effects simultaneously.

I was responsible for all mathematical derivations, numerical computations, and creation of figures.  I was also responsible for the creation of the initial draft of the manuscript.
