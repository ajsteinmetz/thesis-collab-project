\section{Neutrino Plasma}\label{Neutrino}
\subsection{Neutrino properties and reactions}\label{ssec:nuproperties}
Neutrinos are fundamental particles which play an important role in the evolution of the Universe. In the early Universe the neutrinos are kept in equilibrium with cosmic plasma via the weak interaction. The neutrino-matter interactions play a crucial role in our understanding of neutrinos evolution in the early Universe (such as neutrino freeze-out) and the later Universe (the property of today's neutrino background). In this chapter, we will examine the neutrino coherent and incoherent scattering with matter and their application in cosmology. The investigation of the relation between the effective number of neutrinos\index{neutrino!effective number} $N^{\mathrm{eff}}_\nu$ and lepton asymmetry\index{lepton!asymmetry} $L$ after neutrino freeze-out\index{neutrino!freeze-out} and its impact on Universe expansion is also discussed in this chapter. 

%%%%%%%%%%%%%%%%%%%%%%%%%%%%%%
\para{Matrix elements for neutrino coherent \& incoherent scattering}
 According to the standard model, neutrinos interact with other particles via the Charged-Current(CC) and Neutral-Current(NC) interactions. Their Lagrangian can be written as~\cite{Giunti:2007ry}
\begin{align}
&\mathcal{L}^{CC}=\frac{g}{2\sqrt{2}}\left(j^\mu_W\,W_\mu+{j^\mu_W}^\dagger\,W^\dagger_\mu\right),\qquad\mathcal{L}^{NC}=-\frac{g}{2\cos{\theta_W}}\,j^\mu_Z\,Z_\mu,
\end{align}
where $g=e\sin\theta_W$, $W^\mu$ and $Z^\mu$ are W and Z boson gauge fields, and $j^\mu_W$ and $j^\mu_Z$ are the charged-current and neutral-current, respectively. In the limit of energies lower than the $W(m_W=80\,\mathrm{GeV})$ and $Z(m_z=91\,\mathrm{GeV})$ gauge bosons, the effective Lagrangians are given by
\begin{align}\label{L_low}
\mathcal{L}^{CC}_{eff}=-\frac{G_F}{\sqrt{2}}\,j^\dagger_{W\,\mu}\,j^\mu_W,\qquad
\mathcal{L}^{NC}_{eff}=-\frac{G_F}{\sqrt{2}}\,j^\dagger_{Z\,\mu}\,j^\mu_Z,\qquad \frac{G_F}{\sqrt{2}}=\frac{g^2}{8m^2_W},
\end{align}
where $G_F=1.1664\times10^{-5}\,\mathrm{GeV}^{-2}$ is the Fermi constant, which is one of the important parameters that determine the strength of the weak interaction rate. When neutrinos interact with matter, based on the neutrino's wavelength, they can undergo two types of scattering processes: coherent scattering and incoherent scattering with the particles in the medium. 

With coherent scattering, neutrinos interact with the entire composite system rather than individual particles within the system. The coherent scattering is particularly relevant for low-energy neutrinos when the wavelength of neutrino is much larger than the size of system. In $1978$, Lincoln Wolfenstein pointed out that the coherent forward scattering of neutrinos off matter could be very important in studying the behavior of neutrino flavor oscillation\index{neutrino!flavor oscillation} in a dense medium~\cite{Wolfenstein:1977ue}. The fact that neutrinos propagating in matter may interact with the background particles can be described by the picture of free neutrinos traveling in an effective potential.

For incoherent scattering, neutrinos interact with particles in the medium individually. Incoherent scattering is typically more prominent for high-energy neutrinos, where the wavelength of neutrino is smaller compared to the spacing between particles. Study of incoherent scattering of high-energy neutrinos is important for understanding the physics in various astrophysical systems (e.g. supernova, stellar formation) and the evolution of the early Universe.

In this section, we discuss the coherent scattering between long wavelength neutrinos and atoms, and study the effective potential for neutrino coherent interaction. Then we present the matrix elements that describe the incoherent interaction between high energy neutrinos and other fundamental particles in the early Universe. Understanding these matrix elements is crucial for comprehending the process of neutrino freeze-out in the early Universe.

%%%%%%%%%%%%%%%%%%%%%%%%%%%%%%%%
\para{Long wavelength limit of neutrino-atom coherent scattering}
%\label{LongWavelength}
According to the standard cosmological model, the Universe today is filled with the cosmic neutrinos with temperature $T_{\nu}^0=1.9 \,\mathrm{K}=1.7\times10^{-4}\,\mathrm{eV}$. The average momentum of present-day relic neutrinos is given by $\langle p_\nu^0\rangle\approx3.15\,T_\nu^0$ and the typical wavelength $\lambda_{\nu}^{0}={2\pi}/{\langle p_{\nu}^{0}\rangle}\approx2.3\times10^5\,$\AA, which is much larger than the radius at the atomic scale, such as the Bohr radius $R_{\mathrm{atom}}=0.529\,$\AA. In this case we have the long wavelength condition $\lambda_\nu\gg\,R_{\mathrm{atom}}$ for cosmic neutrino background today. 

Under the condition $\lambda_\nu\gg\,R_{\mathrm{atom}}$, when the neutrino is scattering off an atom, the interaction can be coherent scattering~\cite{PhysRevD.38.32,Lewis:1979mu,Papavassiliou:2005cs}. According to the principles of quantum mechanics, with neutrino scattering it is impossible to identify which scatters the neutrino interacts with and thus it is necessary to sum over all possible contributions. In such circumstances, it is appropriate to view the scattering reaction as taking place on the atom as a whole, i.e.,\index{neutrino!coherent scattering}
\begin{align}
\nu+\mathrm{Atom}\longrightarrow\nu+\mathrm{Atom}.
\end{align}

Considering a neutrino elastic scattering off an atom which is composed of $Z$ protons, $N$ neutrons and $Z$ electrons. For the elastic neutrino atom scattering, the low-energy neutrinos scatter off both atomic electrons and nucleus. For nucleus parts, we consider that the neutrinos interact via the $Z^0$ boson with a nucleus as
\begin{align}
\nu+A^{Z}_N\longrightarrow\nu+A^{Z}_N.
\end{align}
In this process a neutrino of any flavor scatters off a nucleus with the same strength. Therefore, the scattering will be insensitive to neutrino flavor. On the other hand, the neutrons can also interact via the $W^\pm$ with nucleus as 
\begin{align}
\nu_l+A^{Z}_N\longrightarrow\,l^-+A^{{Z}+1}_N,
\end{align}
which is a quasi-elastic process for neutrino scattering with the nucleus; we have $A^{Z_e}_N\rightarrow\,A^{{Z_e}+1}_N$. Since this process will change the nucleus state into an excited one, we will not consider its effect here. For detail discussion pf quasi-elastic scattering see ~\cite{SajjadAthar:2022pjt}.

For atomic electrons, the neutrinos can interact via the $Z^0$ and $W^\pm$ bosons with electrons for different flavors, we have
\begin{align}
&\nu_e+e^-\longrightarrow\nu_e+e^-\,\,\,(\mathrm{Z^0,\,W^\pm\,exchange}),\\
&\nu_{\mu,\tau}+e^-\longrightarrow\nu_{\mu,\tau}+e^-\,\,\,(\mathrm{Z^0\,exchange}).
\end{align}
Because of the fact that the coupling of $\nu_e$ to electrons is quite different from that of $\nu_{\mu,\tau}$, one may expect large differences in the behavior of $\nu_e$ scattering compared to the other neutrino types.


%~~~~~~~~~~~~~~~~~~~~~~~~~~~~~~~~~~~~~~~~~~~~~~~~~~
\para{Neutrino-atom coherent scattering amplitude \& matrix element} 
This section considers how a neutrino scatters from a composite system, assumed to consist of $N$ individual constituents at positions $x_i,\,i=1,2,....N$. Due to the superposition principle, the scattering amplitude $\mathcal{M}_\mathrm{sys}(\mathbf{p}^\prime,\mathbf{p})$ for scattering from an incoming momentum $\mathbf{p}$ to an outgoing momentum $\mathbf{p}^\prime$ is given as the sum of the contributions from each constituent~\cite{Freedman:1977xn,Papavassiliou:2005cs}:
\begin{align}
\mathcal{M}_\mathrm{sys}(\mathbf{p}^\prime,\mathbf{p})=\sum_i^N\,\mathcal{M}_i(\mathbf{p}^\prime,\mathbf{p})\,e^{i\mathbf{q}\cdot\mathbf{x}_i},
\end{align}
where $\mathbf{q}=\mathbf{p}^\prime-\mathbf{p}$ is the momentum transfer and the individual amplitudes $\mathcal{M}_i(\mathbf{p}^\prime,\mathbf{p})$ are added with a relative phase factor determined by the corresponding wave function. %The transition probability is then given by
%\begin{align}
%\mathcal{P}_{\mathrm{sys}}(\mathbf{p}^\prime,\mathbf{p})&=|\mathcal{M}_\mathrm{sys}(\mathbf{p}^\prime,\mathbf{p})|^2\notag\\
%&=\sum_i|\mathcal{M}_i(\mathbf{p}^\prime,\mathbf{p})|^2+\sum_{i,j}^{i\neq\,j}\mathcal{M}_i(\mathbf{p}^\prime,\mathbf{p})\mathcal{M}_j^\dagger(\mathbf{p}^\prime,\mathbf{p})\,e^{i\mathbf{q}\cdot\left(\mathbf{x}_j-\mathbf{x}_i\right)}.
%\end{align} 
In principle, due to the presence of the phase factors, major cancellation may take place among the terms for the condition $|\mathbf{q}|R\gg1$, where $R$ is the size of the composite system, and the scattering would be incoherent. However, for the momentum small compared to the inverse target size, i.e., $|\mathbf{q}|R\ll1$, then all phase factors may be approximated by unity and contributions from individual scatters add coherently. 


In the case of neutrino coherent scattering with an atom: If we consider sufficiently small momentum transfer to an atom from a neutrino which satisfies the coherence condition, i.e., $|\mathbf{q}|R_{\mathrm{atom}}\ll1$, then the relevant phase factors have little effect, allowing us to write the transition amplitude as \cite{Nicolescu:2013rxa}
\begin{align}
\label{M_atom}
\mathcal{M}_\mathrm{atom}=\sum_t\frac{G_F}{\sqrt{2}}\left[\overline{u}(p^\prime_\nu)\gamma_\mu\left(1-\gamma_5\right)u(p_\nu)\right]\left[\overline{u}(p^\prime_t)\gamma^\mu\left(c^t_V-c^t_A\gamma^5\right)u(p_t)\right],
\end{align}
where $t$ is all the target constituents (Z protons, N neutrons and Z electrons). The transition amplitude includes contributions from both charged and neutral currents, with
\begin{align}\label{CC_int}
&\mathrm{Charged\,\,Current}: c^t_V=c^t_A=1\,,\\
\label{NC_int}
&\mathrm{Neutral\,\, Current}: c^t_V=I_3-2\mathcal{Q}\sin^2\theta_W,\qquad c^t_A=I_3\,,
\end{align}
where $I_3$ is the weak isospin, $\theta_W$ is the Weinberg angle, and $\mathcal{Q}$ is the particle electric charge. 

Considering the target can be regarded as an equal mixture of spin states $s_z=\pm1/2$, and we can simplify the transition amplitude by summing the coupling constants of the constituents \cite{Lewis:1979mu,Sehgal:1986gn}. We have
\begin{align}
\label{Transition}
\mathcal{M}_\mathrm{atom}=&\frac{G_F}{2\sqrt{2}}\left[\overline{u}(p^\prime_\nu)\gamma_\mu\left(1-\gamma_5\right)u(p_\nu)\right]\notag\\&\bigg[\overline{u}(p^\prime_{a})\sum_t\left(C_L+C_R\right)_t\gamma^\mu\,u(p_{a})-\overline{u}(p^\prime_{a})\sum_t\left(C_L-C_R\right)_t\gamma^\mu\gamma^5u(p_{a})\bigg],
\end{align}
where the $u(p_\nu)$, $u(p^\prime_\nu)$ are the initial and final neutrino states and $u(p_a)$, $u_(p^\prime_a)$ are the initial and final states of the target atom. 
The coupling coefficients $C_L$ and $C_V$ are defined as
\begin{align}
&C_L=c_V+c_A,\,\,\,\,\,C_R=c_V-c_A,
\end{align}
where the coupling constants for neutrino scattering with proton, neutron, and electron are given by \rt{Table_coupling}. The coupling constants for $\nu_{\mu,\tau}$ are the same as for the $\nu_e$, excepting the absence of a charged current in neutrino-electron scattering.

%%%%%%%%%%%%%%%%%%%%%%%%%%%%%%%%%%%%%%%%%%%%%%%%%%%%%%
\begin{table}
\begin{tabular}[c]{c|c|c|c|c}
\hline\hline
& Electron ($Z^0$ boson) & Electron ($W^\pm$ boson) & Proton (uud) & Neutron (udd)\\
\hline
$C_L$ & $-1+2\sin^2\theta_W$ & $2$ & $1-2\sin^2\theta_W$ & $-1$ \\
\hline
$C_R$ & $2\sin^2\theta_W$ & $0$ &$-2\sin^2\theta_W$ & $0$ \\
\hline\hline
\end{tabular}
\caption{The coupling constants for neutrino scattering with proton, neutron, and electron.}
\label{Table_coupling} 
\end{table}
%%%%%%%%%%%%%%%%%%%%%%%%%%%%%%%%%%%%%%%%%%%%%%%%%%%%%

Given the neutrino-atom coherent scattering amplitude \req{Transition}, the transition matrix element can be written as
\begin{align}
\label{scattering_matrix}
|\mathcal{M}_{\mathrm{atom}}|^2=\frac{G^2_F}{8}L_{\alpha\beta}^{\mathrm{neutrino}}\,\Gamma^{\alpha\beta}_{\mathrm{atom}},
\end{align}
where the neutrino tensor $L_{\alpha\beta}^{\mathrm{neutrino}}$ is given by
\begin{align}
\label{neutrino_tensor}
L_{\alpha\beta}^{\mathrm{neutrino}}
&=\mathrm{Tr}\left[\gamma_\alpha\left(1-\gamma_5\right)(\slashed{p}_\nu+m_\nu)\gamma_\beta\left(1-\gamma_5\right)(\slashed{p}^\prime_\nu+m_\nu)\right]\notag\\
&=8\left[(p_\nu)_\alpha\,(p^\prime_{\nu})_\beta+(p_\nu)^\prime_\alpha\,(p_\nu)_\beta-g_{\alpha\beta}(p_\nu\cdot\,p_\nu^\prime)+i\epsilon_{\alpha\sigma\beta\lambda}(p_\nu)^\sigma(p^\prime_\nu)^\lambda\right],
\end{align}
and the atomic tensor $\Gamma^{\alpha\beta}_\mathrm{atom}$ can be written as
\begin{align}
\label{atomic_tensor}
\Gamma^{\alpha\beta}_\mathrm{atom}
&=\mathrm{Tr}\bigg[(C_{LR}\gamma^\alpha-C^\prime_{LR}\gamma^\alpha\gamma^5)(\slashed{p}_a+M_a)(C_{LR}\gamma^\beta-C^\prime_{LR}\gamma^\beta\gamma^5)(\slashed{p}^\prime_a+M_a)\bigg]\notag\\
&=4\bigg\{(C^2_{LR}+C^{\prime2}_{LR})\left[(p_a)^\alpha\,(p^\prime_a)^\beta+(p_a)^{\prime\alpha}\,(p_a)^\beta\right]\notag\\
&\qquad-g^{\alpha\beta}\bigg[(C^2_{LR}-C^{\prime2}_{LR})(p_a\cdot\,p_a^\prime)-(C^2_{LR}-C^{\prime2}_{LR})M^2_a\bigg]\notag\\&\qquad\qquad+2iC_{LR}C^\prime_{LR}\epsilon^{\alpha\sigma^\prime\beta\lambda^\prime}(p_a)_{\sigma^\prime}(p^\prime_a)^{\lambda^\prime}\bigg\},
\end{align}
where $M_a$ is the target atom's mass $(M_a = AM_\mathrm{nucleon}, A=Z+N)$, and the coupling constants $C_{LR}$ and $C^\prime_{LR}$ are defined by
\begin{align}
C_{LR}=\sum_t(C_L+C_R)_t,\,\,\,\,\,\,C^\prime_{LR}=\sum_t(C_L-C_R)_t.
\end{align}
Substituting \req{neutrino_tensor} and \req{atomic_tensor} into \req{scattering_matrix}, the transition matrix element for coherent elastic neutrino atom scattering is given by:\index{neutrino!scattering matrix element}
\begin{align}
|\mathcal{M}_{\mathrm{atom}}|^2&=\frac{G^2_F}{8}L_{\alpha\beta}^{\mathrm{neutrino}}\,\Gamma^{\alpha\beta}_{\mathrm{atom}}\notag\\
&=8G^2_F\bigg[(C_{LR}+C^\prime_{LR})^2\,(p_\nu\cdot\,p_a)(p^\prime_\nu\cdot\,p^\prime_a)+(C_{LR}-C^\prime_{LR})^2\,(p_\nu\cdot\,p^\prime_a)(p^\prime_\nu\cdot\,p_a)\notag\\&
\,\,\,\,\,\,-(C^2_{LR}-C^{\prime2}_{LR})M^2_a(p_\nu\cdot\,p_\nu^\prime)\bigg].
\end{align}
Taking the atom at rest in the laboratory frame, and considering small momentum transfer to an atom from a neutrino, i.e., $q^2=(p_\nu-p^\prime_\nu)^2=(p_a^\prime-p_a)^2\ll\,M^2_a$, we have
\begin{align}
&p_\nu\cdot\,p_a=E_\nu\,M_a,\\
&p_\nu^\prime\cdot\,p_a=E_\nu^\prime\,M_a\approx\,E_\nu\,M_a,\\
&p^\prime_\nu\cdot\,p^\prime_a=p^\prime_\nu\cdot(p_a+q)=E^\prime_\nu\,M_a\left[\left(1+\frac{q_0}{M_a}\right)-\frac{|p^\prime_\nu||q|}{M_a}\cos\theta\right]\approx\,E_\nu\,M_a,\\
&p_\nu\cdot\,p^\prime_a=p_\nu\cdot(p_a+q)=E_\nu\,M_a\left[\left(1+\frac{q_0}{M_a}\right)-\frac{|p^\prime_\nu||q|}{M_a}\cos\theta\right]\approx\,E_\nu\,M_a.
\end{align}
Then the transition matrix element for neutrino coherent elastic scattering off a rest atom can be written as
\begin{align}\label{M_general}
|\mathcal{M}_{\mathrm{atom}}|^2&=8\,G^2_F\,M_a\,E_\nu^2\left[C^2_{LR}\left(1+\frac{|p_\nu|^2}{E^2_\nu}\cos\theta\right)+3C^{\prime2}_{LR}\left(1-\frac{|p_\nu|^2}{3E_\nu^2}\cos\theta\right)\right],
\end{align}
which is consistent with the results in papers \cite{PhysRevD.38.32,Lewis:1979mu,Papavassiliou:2005cs,Smith:1984gym}.
From the above formula we found that the scattering matrix neatly divides into two distinct components: a vector-like component (first term) and an axial-vector like component (second term). They have different angular dependencies: the vector part has a $\left({|p_\nu|^2}/{E^2_\nu}\cos\theta\right)$ dependence, while the axial part has a $\left(-{|p_\nu|^2}/{3E_\nu^2}\cos\theta\right)$ behavior. However, in the case of the nonrelativistic neutrino, both angular dependencies can be neglected because of the limit $p_\nu\ll\,m_\nu$. 

Next, we consider the nonrelativistic electron neutrino $\nu_e$ scattering off an general atom with $Z$ protons, $N$ neutrons and $Z$ electrons. Then from Eq.~(\ref{M_general}), the matrix element can be written as
\begin{align}
\label{Probability_e}
|\mathcal{M}_{\mathrm{atom}}|^2&=8\,G^2_F\,M_a\,E_\nu^2\left[\left(3Z-A\right)^2\left(1+\frac{|p_\nu|^2}{E^2_\nu}\cos\theta\right)+3\left(3Z-A\right)^2\left(1-\frac{|p_\nu|^2}{3E_\nu^2}\cos\theta\right)\right]\notag\\
&\approx32\,G^2_F\,M_a\,E_\nu^2\left(3Z-A\right)^2,
\end{align}
where we neglect the angular dependence because of the nonrelativistic limit, and the coefficient $\left(3Z-A\right)^2$ for different target atoms are given in \rt{Table001}. 

%%%%%%%%%%%%%%%%%%%%%%%%%%%%%%%%%%%%%%%%%%%%%%%%%%%%%%%%%%%%%%
\begin{table}[ht]
\centering
\begin{tabular}{c|c|c}
\hline\hline
 Neutrino Flavor:&$\nu_e$ &$\nu_{\mu,\tau}$\\
\hline\hline
Target Atom & $(3Z-A)^2$ & $(Z-A)^2$\\
\hline
$H_2(A=2, Z=2)$ & $16$ & $0$\\
\hline
${}^{3}H_e(A=3, Z=2)$ & $9$ & $1$\\
\hline
$HD(A=3, Z=2)$ & $9$ & $1$\\
\hline
${}^{4}_2H_e(A=4, Z=2)$ &$4$ & $4$\\
\hline
$DD(A=4, Z=2)$ & $4$ & $4$\\
\hline
${}^{12}_{{}6}C(A=12, Z=6)$ & $36$& $36$\\
\hline\hline
\end{tabular}
\caption{The coefficients for transition amplitude and scattering probability of $\nu_e$ and $\nu_{\mu,\tau}$ coherent elastic scattering off different target atoms. The definition of atomic mass is $A=Z+N$, where $Z$ and $N$ are the number of protons and neutron respectively.}
\label{Table001} 
\end{table}%
%%%%%%%%%%%%%%%%%%%%%%%%%%%%%%%%%%%%%%%%%%%%%%%%%%%%%%%%%%%%%%

For nonrelativistic $\nu_{\mu,\tau}$, the scattering matrix is given by
\begin{align}
\label{Probability_mt}
|\mathcal{M}_{\mathrm{atom}}|^2&=8\,G^2_F\,M_a\,E_\nu^2\left[\left(A-Z\right)^2\left(1+\frac{|p_\nu|^2}{E^2_\nu}\cos\theta\right)+3\left(A-Z\right)^2\left(1-\frac{|p_\nu|^2}{3E_\nu^2}\cos\theta\right)\right]\notag\\
&\approx32\,G^2_F\,M_a\,E_\nu^2\left(Z-A\right)^2,
\end{align}
where the coefficient $\left(Z-A\right)^2$ different target atoms are given in \rt{Table001}. The transition matrix for $\nu_e$ differs from that of $\nu_{\mu,\tau}$; this is due to the charged current reaction with the atomic electrons. Furthermore, the neutral current interaction for the electron and proton will cancel each other because of the opposite weak isospin $I_3$ and charge $\mathcal{Q}$. As a result, the coherent neutrino scattering from an atom is sensitive to the method of the neutrino-electron coupling.
%%%%%%%%%%%%%%%%%%%%%%%%%%%%%%%%%%%%%%%%%%

%%%%%%%%%%%%%%%%%%%%%%%%%%%%%%%%%%%%%%%%%%%%%%%%%%%%%%%%%%%%%%%%%%%%%%%%
\para{Mean field potential for neutrino coherent scattering}
When neutrinos are propagating in matter and interacting with the background particles, they can be described by the picture of free neutrinos traveling in an effective potential~\cite{Wolfenstein:1977ue}. In the following we describe the effective potential between neutrinos and the target atom, and generalize the potential to the case of neutrino coherent scattering with a multi-atom system.


Let us consider a neutrino elastic scattering off an atom which is composed of Z protons, N neutrons and Z electrons. For the elastic neutrino atom scattering, the low-energy neutrinos are scattering off both atomic electrons and the nucleus. Considering the effective low-energy CC and NC interactions, the effective Hamiltonian in current-current interaction form can be written as 
\begin{align}
\label{H_atom}
\mathcal{H}_I^{\mathrm{atom}}&=\mathcal{H}^\mathrm{electron}_I+\mathcal{H}^\mathrm{nucleon}_I=\frac{G_F}{\sqrt{2}}\,\left(j_\mu\,\mathcal{J}^\mu_{\mathrm{electron}}+j_\mu\,\mathcal{J}^\mu_\mathrm{nucleon}\right),
\end{align}
where $\mathcal{J}^\mu_{\mathrm{nucleon}}$ denotes the hadronic current for the nucleus, and $j^\mu$ and $\mathcal{J}^\mu_{\mathrm{electron}}$ are the lepton\index{lepton} currents for neutrino and electron, respectively. According to the weak interaction theory, the lepton current for the neutrino and electron can be written as
\begin{align}
&j_\mu=\overline{\psi_{\nu}}\,\gamma_\mu\,\left(1-\gamma_5\right)\,\psi_\nu,\\
\label{Current_e}
&\mathcal{J}^\mu_{\mathrm{electron}}=\overline{\psi_{e}}\,\gamma_\mu\,\left(1-\gamma_5\right)\,\psi_e\,\,\,\,\,(\mathrm{W^\pm\,exchange}),\\
&\mathcal{J}^\mu_{\mathrm{electron}}=\overline{\psi_{e}}\,\gamma_\mu\,\left(c_V^e-c_A^e\gamma_5\right)\,\psi_e\,\,\,\,\,(\mathrm{Z^0\,exchange}),
\end{align}
where $\psi_\nu$ and $\psi_e$ represent the spinor for the neutrino and electron, respectively. From Eq.~(\ref{NC_int}) the coupling coefficient for electrons are $c^e_V=-1/2+2\sin^2\theta_W$ and $c^e_A=-1/2$. The hadronic current is given by the expression~\cite{Giunti:2007ry}
\begin{align}
\label{Current_h}
\mathcal{J}^\mu_\mathrm{nucleon}\equiv\overline{\psi_t}\,\gamma^\mu\left(c^t_V-c^t_A\gamma^5\right)\psi_t,
\end{align}
where subscript $t$ means the target constituents (protons and neutrons). From Eq.~(\ref{NC_int}) the coupling constants for proton(uud) and neutron(udd) are given by
\begin{align}
&c^p_V=\frac{1}{2}-2\sin^2\theta_W,\,\,\,\,c^p_A=\frac{1}{2},\,\,\,\,\,\mathrm{proton}\,,\\
&c^n_V=-\frac{1}{2}\,\,\,\,c^n_A=-\frac{1}{2},\,\,\,\,\,\mathrm{neutron}\,.
\end{align}

To obtain the effective potential for an atom, we need to average the effective Hamiltonian over the electron and nucleon background. For the neutrino-nucleon (proton, neutron) interaction, we only have the neutral current interaction via $Z^0$ boson. However, for the neutrino-electron interaction, we can have charged-current or neutral current interaction depending on the flavor or neutrino. In following, we consider interaction between $\nu_e$ and electrons first which includes both charged and neutral-currents interaction for general discussion.

Considering atomic electrons as a gas of unpolarized electrons with a statistical distribution function $f(E_e)$, the effective potential for neutrino-electron interaction can be obtained by averaging the effective Hamiltonian over the electron background~\cite{Giunti:2007ry}
\begin{align}
\langle{\mathcal{H}^\mathrm{electron}_{I}}\rangle&=\frac{G_F}{\sqrt{2}}\int\,\frac{d^3p_e}{(2\pi)^32E_e}\,f(E_e,T)\left[\overline{\psi_\nu}(x)\,\gamma_\mu\left(1-\gamma_5\right)\,\psi_\nu(x)\right]\notag\\&\times\frac{1}{2}\!\sum_{h_e=\pm1}\!\!\langle\,e^-(p_e,h_e)|\overline{\psi_e}\,\gamma^\mu\big((1+c^e_V)\!-\!(1+c^e_A)\gamma_5\big)\,\psi_e|e^-(p_e,h_e)\rangle,
\end{align}
where $h_e$ denotes the helicity of the electron. The average over helicity of the electron matrix element can be calculated with Dirac spinor and gamma matrix traces ~\cite{Giunti:2007ry}. Then the average effective Lagrangian can be written as
\begin{align}
\langle{\mathcal{H}^\mathrm{electron}_{I}}\rangle&=\frac{G_F}{\sqrt{2}}(1+c^e_V)\int\,\frac{d^3p_e}{(2\pi)^3}f(E_e)\left[\overline{\psi_\nu}(x)\,\frac{\gamma^\mu{p_e}_\mu}{E_e}\left(1-\gamma_5\right)\,\psi_\nu(x)\right]\notag\\
&=\frac{G_F}{\sqrt{2}}\,(1+c^e_V)\,\left[\int\,\frac{d^3p_e}{(2\pi)^3}f(E_e)\left(\gamma^0-\frac{\vec{\gamma}\cdot\vec{{p}}_e}{E_e}\right)\right]\overline{\psi_\nu}(x)\left(1-\gamma_5\right)\psi_\nu(x)\notag\\
&=\left[\frac{G_F}{\sqrt{2}}\left(1+c^e_V\right)n_{e}\right]\,\overline{\psi_\nu}(x)\gamma^0\left(1-\gamma_5\right)\psi_\nu(x),
\end{align}
where $n_e$ is the number density of the electron. In this case, the effective potential for the case of neutrino-atomic electron interaction can be written as\index{neutrino!effective potential}
\begin{align}
V^{\mathrm{electron}}_{I}=\frac{G_F}{\sqrt{2}}\left(1+c^e_V\right)n_{e}=\frac{G_F}{\sqrt{2}}\left(4\sin^2\theta_W+1\right)n_{e}.
\end{align}
The same method can be applied to the neutrino-nuclear interactions. Following the same approach and averaging the effective neutrino-nuclear Hamiltonian over the nuclear background, the effective potential experienced by a neutrino in a background of neutron/proton is given by~\cite{Giunti:2007ry} 
\begin{align}
&V_{I}^{\mathrm{proton}}=\frac{G_F}{\sqrt{2}}\left(1-4\sin^2\theta_W\right)n_{p},\qquad V_{I}^{\mathrm{neutron}}=-\frac{G_F}{\sqrt{2}}\,n_{n},
\end{align}
where $n_p$ and $n_n$ represent the number density of proton and neutron.
Combining the neutron and proton potential together, we define the effective nucleon potential experienced by a neutrino as 
\begin{align}
V_I^{\mathrm{nucleon}}\equiv-\frac{G_F}{\sqrt{2}}\bigg[1-\left(1-4\sin^2\theta_W\right)\xi\bigg]n_{n},\qquad\xi=n_{p}/n_{n},
\end{align}
where $\xi$ is the ratio between proton and neutron number density.

In our study, we generalize the effective potential to the case of neutrino coherent scattering with multi-atom system, we consider a neutrino coherent forward scatters from a spherical symmetric system which is composed by atoms. In this case, the neutrino scatters off every atom, and it is impossible to identify which scatterer the neutrino interacts with and thus it is necessary to sum over all possible contributions from each atom. In such circumstances, it is appropriate to assume that the number density of electrons and neutrons can be written as
\begin{align}
&n_e=Z_e\,\left(\frac{N_\mathrm{atom}}{V}\right),\,\,\,\,\mathrm{and}\,\,\,\,n_n=N\,\,\left(\frac{N_\mathrm{atom}}{V}\right),
\end{align}
where $N_\mathrm{atom}$ is the number of atoms inside the system, $V$ is the volume of system, $Z$ is the number of electrons, and $N$ is the number of neutrons.
%%%%%%%%%%%%%%%%%%%%%%%%%%%%%%%%%%%%%%%%%%%%%%%%%%%%%%%%%%%%%%%%%%%%%
Then the effective potential is given by
\begin{align}
\label{Potential}
V_{I}&=V_I^{\mathrm{electron}}+V_I^{\mathrm{nucleon}}\notag\\&=\frac{G_F}{\sqrt{2}}\left(\frac{N_\mathrm{atom}}{V}\right)\bigg\{\left(4\sin^2\theta_W\pm1\right)\,Z_e-\bigg[1-\left(1-4\sin^2\theta_W\right)\xi\bigg]\,N\bigg\},
\end{align}
where the $+$ sign is for electron neutrinos $\nu_e$ and the $-$ sign is for muon(tau)\index{muon} neutrinos $\nu_{\mu,\tau}$, separately. 
Equation~(\ref{Potential})  shows that the effective potential depends on the number density of electrons and nucleons contained within the wavelength. 
Thus by increasing the atoms contained in the wavelength or selecting different atoms as targets, we can enhance the effective potential and may be able to provide a sensitive way to detect the cosmic neutrino background. Beside the detection of the cosmic neutrino background, the effective potential for multi-atom can also provide new approaches for studying other aspects of neutrino physics in the future.

%%%%%%%%%%%%%%%%%%%%
%%%%%%%%%%%%%%%%%%%%%%%%%%%%%%%%%%%%%%%%%%%%%%%%%%%%%%%%%%%%%%%%%%%%%%%%
\para{Matrix elements of incoherent neutrino scattering}
To determine the freeze-out temperature (chemical/kinetic freeze-out) for a given flavor of neutrinos, we need to know all the elastic and inelastic interaction processes in the early Universe and compare their interaction rates with the Hubble expansion rate. In this section we summarize the matrix elements for the neutrino annihilation/production processes and elastic scattering processes that are relevant for investigating neutrino freeze-out. These matrix elements serve as one of the fundamental ingredients for solving the Boltzmann equation ~\cite{Birrell:2014uka}.

Considering the Universe with temperature $T\approx\mathcal{O}$(MeV), the particle species in cosmic plasma are given by:
\begin{align}
\mathrm{Particle\,\,species\,\, in \,\,plasma:}
\left\{\gamma,\, l^-,\,l^+,\, \nu_e,\, \nu_\mu,\, \nu_\tau,\, \bar{\nu}_e,\, \bar{\nu}_\mu,\, \bar{\nu}_\tau\right\},
\end{align}
where $l^\pm$ represents the charged leptons. In this case, neutrinos can interact with all these particles via weak interactions and remain in equilibrium. In~\rt{T005} and \rt{T006} we present the matrix elements $|M|^2$ for different weak interaction processes in the early Universe.

%%%%%%%%%%%%%%%%%%%%%%%%%%%%%%%%%%%%%%%%%%%%%%%%%%%%%%%%%%%%%
\begin{table} 
\centering
\begin{tabular}{p{0.21\textwidth}p{0.79\textwidth}}
\hline\hline
Annihilation \\
\& Production & Transition Amplitude $|M|^2$ \\
\hline
$l^-+l^+\longrightarrow\nu_l+\bar{\nu}_l$&$ 32G^2_F\bigg[\left(1+2\sin^2\theta_W\right)^2\left(p_1\cdot p_4\right)\left(p_2\cdot p_3\right)+\left(2\sin^2\theta_W\right)^2\left(p_1\cdot p_3\right)\left(p_2\cdot p_4\right)$
%
$+2\sin^2\theta_W\left(1+2\sin^2\theta_W\right)m^2_l\left(p_3\cdot p_4\right)\bigg]$ \\
\hline
$l^{\prime-}+l^{\prime+}\longrightarrow\nu_l+\bar{\nu}_l$ & $32G^2_F\bigg[\left(1-2\sin^2\theta_W\right)^2\left(p_1\cdot p_4\right)\left(p_2\cdot p_3\right)$
%
$+\left(2\sin^2\theta_W\right)^2\left(p_1\cdot p_3\right)\left(p_2\cdot p_4\right)$
%
$-2\sin^2\theta_W\left(1-2\sin^2\theta_W\right)m^2_{l^\prime}\left(p_3\cdot p_4\right)\bigg]$ \\
\hline
$\nu_l+\bar{\nu}_l\longrightarrow\nu_l+\bar{\nu}_l$ &
$32G^2_F\bigg[\left(p_1\cdot p_4\right)\left(p_2\cdot p_3\right)\bigg]$ \\
\hline
$\nu_{l^\prime}+\bar{\nu}_{l^\prime}\longrightarrow\nu_l+\bar{\nu}_l$ &
$32G^2_F\bigg[\left(p_1\cdot p_4\right)\left(p_2\cdot p_3\right)\bigg]$ \\
\hline\hline
\end{tabular}
\caption{The transition amplitude for different annihilation and production processes. The definition of particle number is given by $1+2\leftrightarrow3+4$, where $l,\,l^\prime=e,\,\mu,\,\tau\,(l\neq\,l^\prime)$.}
\label{T005}
\end{table}
%%%%%%%%%%%%%%%%%%%%%%%%%%%%%%%%%%%%%%%%%%%%%%%%%%%

In the calculation of transition amplitude, we use the low energy approximation for $W^\pm$ and $Z^0$ massive propagators, i.e.
\begin{align}
&\mbox{$Z^0$ boson}:\frac{-i\left[g_{\mu\nu}-\frac{q_\mu q_\nu}{M^2_z}\right]}{q^2-M^2_z}\approx\frac{ig_{\mu\nu}}{M^2_z},\quad
&\mbox{$W^\pm$ boson}:\frac{-i\left[g_{\mu\nu}-\frac{q_\mu q_\nu}{M^2_W}\right]}{q^2-M^2_W}\approx\frac{ig_{\mu\nu}}{M^2_W},
\end{align}
and consider the tree-level Feynman diagram contributions only. Then, following the Feynman rules of weak interaction~\cite{Griffiths:2008zz}, we obtain the matrix elements $|M|^2$ for different interaction processes.\index{neutrino!incoherent scattering} For a detail discussion please \rsec{ch:coll:simp} in Appendix.

%%%%%%%%%%%%%%%%%%%%%%%%%%%%%%%%%%%%%%%%%%%%%%%%%%%%%%%%%%%%%
\begin{table} 
\centering
\begin{tabular}{p{0.21\textwidth}p{0.79\textwidth}}
\hline\hline
Elastic ($\nu_e$) \\
Scattering Process & Transition Amplitude $|M|^2$\\
\hline
$\nu_l+l^-\longrightarrow\nu_l+l^-$ & 
$ 32G^2_F\bigg[
 \left(1+2\sin^2\theta_W\right)^2\left(p_1\cdot p_2\right)\left(p_3\cdot p_4\right)+\left(2\sin^2\theta_W\right)^2\left(p_1\cdot p_4\right)\left(p_2\cdot p_3\right)$ 
%
 $-2\sin^2\theta_W\left(1+2\sin^2\theta_W\right)m^2_l\left(p_1\cdot p_3\right)\bigg]$ \\
\hline
$\nu_l+l^+\longrightarrow\nu_l+l^+$ &
$ 32G^2_F\bigg[
 \left(1+2\sin^2\theta_W\right)^2\left(p_1\cdot p_4\right)\left(p_2\cdot p_3\right)+\left(2\sin^2\theta_W\right)^2\left(p_1\cdot p_2\right)\left(p_3\cdot p_4\right)$ 
% 
 $-2\sin^2\theta_W\left(1+2\sin^2\theta_W\right)m^2_l\left(p_1\cdot p_3\right)\bigg]$ \\
\hline
$\nu_l+l^{\prime-}\longrightarrow\nu_l+l^{\prime-}$ &
$ 32G^2_F\bigg[
 \left(1-2\sin^2\theta_W\right)^2\left(p_1\cdot p_2\right)\left(p_3\cdot p_4\right)+\left(2\sin^2\theta_W\right)^2\left(p_1\cdot p_4\right)\left(p_2\cdot p_3\right)$\hfill
% 
 \hspace{1cm}$+2\sin^2\theta_W\left(1-2\sin^2\theta_W\right)m^2_{l^\prime}\left(p_1\cdot p_3\right)\bigg]$ \\
\hline
$\nu_l+l^{\prime+}\longrightarrow\nu_l+l^{\prime+}$ &
$ 32G^2_F\bigg[
 \left(1-2\sin^2\theta_W\right)^2\left(p_1\cdot p_4\right)\left(p_2\cdot p_3\right)+\left(2\sin^2\theta_W\right)^2\left(p_1\cdot p_2\right)\left(p_3\cdot p_4\right)$ 
% 
 $+2\sin^2\theta_W\left(1-2\sin^2\theta_W\right)m^2_{l^\prime}\left(p_1\cdot p_3\right)\bigg]$ \\
\hline
$\nu_l+\nu_l\longrightarrow\nu_l+\nu_l$ &
$\frac{1}{2!}\frac{1}{2!}\times32G^2_F\bigg[4\left(p_1\cdot p_2\right)\left(p_3\cdot p_4\right)\bigg]$ \\
\hline
$\nu_l+\bar{\nu}_l\longrightarrow\nu_l+\bar{\nu}_l$ &
$32G^2_F\bigg[4\left(p_1\cdot p_4\right)\left(p_2\cdot p_3\right)\bigg]$ \\
\hline
$\nu_l+\nu_{l^\prime}\longrightarrow\nu_l+\nu_{l^\prime}$ &
$32G^2_F\bigg[\left(p_1\cdot p_2\right)\left(p_3\cdot p_4\right)\bigg]$ \\
\hline
$\nu_l+\bar{\nu}_{l^\prime}\longrightarrow\nu_l+\bar{\nu}_{l^\prime}$ &
$32G^2_F\bigg[\left(p_1\cdot p_4\right)\left(p_2\cdot p_3\right)\bigg]$ \\
\hline\hline
\end{tabular}
\caption{The transition amplitude for different elastic scattering processes. The definition of particle number is given by $1+2\leftrightarrow3+4$, where $l,\,l^\prime=e,\,\mu,\,\tau\,(l\neq\,l^\prime)$.}
\label{T006}
\end{table}
\clearpage