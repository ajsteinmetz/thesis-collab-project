\markboth{2. Strangeness  as an Observable of QGP}{Strangeness in QGP: Diaries}
\section{Strangeness in  Quark-Gluon Plasma}
\begin{itemize}
\item
I open the review of the theoretical developments describing the work carried out in the period  1980-1990. For this I will use the first half of the 1992 Il Ciocco Summer School introduction to the topic of strangeness signature of QGP.  
\item
This is followed by the elaboration of strange antibaryon signature, with emphasis on strange antibaryons at \lq high\rq\ momentum dating from 1987. 
\item
A short proposed lecture abstract from 1992 describes the lack of success in presenting this signature of QGP to a wider Nuclear Physics community.  
\item
A report presented  at the 1990  \lq 4th Workshop on  Experiments and Detectors for RHIC\rq\ discusses how these insights influence the experimental landscape under development. 
\item
A few pages from  the 1996 progress report provided to the US funding agency about strangeness production process show refinement of strangness production using QCD properties and show how strangness production and strange antibaryon enhancement depends on collision energy in the SPS energy range. 
\item 
I describe my effort to join the BNL STAR collaboration in 1997-9 where I hoped my theoretical and practical insights gained with the CERN SPS data analysis could be of use, showing additional strangeness production results.
\end{itemize}
Several of these documents are shown in part only, in order to focus on the topic considered and avoid duplications. This is inicated as appropriate, and  in the following section, where methods of experimental result analysis will be introduced, we continue to reprint the contents.

 
%%%%%%%%%%%%%%%%%%%%%%%%%%%%%%%%
\subsection[Strangeness production
\hfill\textbf{reprinted sections are not numbered}]{Strangeness production}
%%%%%%%%%%%%%%%%%%%%%%%%%%%%%%%%%%%%%%%%%%%%%%%%%%%%%%%%%%%%
%\subsubsection{Kinetic theory}
%
\noindent\textit{The following is part I of the IlCiocco July 12-24, 1992 Summer School~\cite{Gutbrod:1993rp} \lq\lq Particle Production in Highly Excited Matter\rq\rq\ presentation~\cite{Rafelski:1992td} (for part II see page \pageref{Trail1992p2}):}\\[-0.7cm]
%
%%%%%%%%%%%%%%%%%%%%%%%%%%%%%%%%%%%%%%%%%%%%%%%%%%%%%%%%%%%%
\begin{mdframed}[linecolor=gray,roundcorner=12pt,backgroundcolor=Dandelion!15,linewidth=1pt,leftmargin=0cm,rightmargin=0cm,topline=true,bottomline=true,skipabove=12pt]\relax%
%
\centerline{\includegraphics[width=1.0\textwidth]{./AllFigs/92CCIlCioccoS.jpg}}
%\caption
\noindent{\small IlCiocco Summer School July 12-24, 1992 Cern Courier Report December 1992: the here shown picture is the color original -- CERN Courier was printed in B/W}\\[0.4cm]
%
%
{\Large \bf On the Trail of Quark-Gluon-Plasma:\\
Strange Antibaryons in Nuclear Collisions}
%
%%%%%%%%%%%%%%%%%%%%%%%%%%%%%%%%%%%%%%%%%%%%%%%%%%%%%%%%%%%%
\addcontentsline{toc}{subsubsection}{Strangeness as an observable}
%%%%%%%%%%%%%%%%%%%%%%%%%%%%%%%%%%%%%%%%%%%%%%%%%%%%%%%%%%%%
\section*{Strangeness as an observable}\label{Trail1992}
%%%%%%%%%%%%%%%%%%%%%%%%%%%%%%%%%%%%%%%%%%%%%%%%%%%%%%%%%%%%
There may be no easy laboratory observable of quark-gluon plasma (QGP).
But I hope to convince you that much can be learned about this new form
of nuclear matter studying diligently the third (and still rather light)
strange \lq $s$\rq\ quark flavor. I will in particular address the predicted and observed abundant emission of strange antibaryons in relativistic nuclear collisions which in my opinion constitutes a rather clear indication of new physics$^{1,2,3}$. \ldots
%As it is hard to be selfcontained or to give justice in such a short presentation to all the effort in this active field, it seems to me more prudent to stress the relation of this report to numerous other lectures presented in this volume, and in particular the preceding experimental survey of E. Quercigh. 
For further theoretical details and numerous references the reader should consult the more comprehensive presentations$^{4,5,6}$ (status 1992).
 
I will address primarily the relative production of strange and
multi-strange antibaryons. As I will mostly address ratios of particle
multiplicities, an important point will escape further attention: there
is a considerable enhancement of the production cross section of these
particles above and beyond the expectations based on hadronic cascading
reactions of the type $p$--$p$. This is in agreement with the naive
expectations based on a scenario involving production of an intermediate
drop of quark-gluon plasma. But why should such an enhancement be
expected in QGP? I will begin by recalling the simple, historic and
somewhat correlated arguments why strange particles in general, and
(multi)strange  (anti)baryons in particular possess a priori a distinct
diagnostic function of the behavior of highly excited nuclear matter:\par
\indent 1: abundance symmetry of $\bar s, \bar u, \bar d$ in statistical
description;\par
\indent 2: strongly differing production rates in different phases of
nuclear matter;\par
\indent 3: high  $s\bar s$--quark pair density in QGP phase;\par
\indent 4: small strange antibaryon background from $p-p$ reactions in
the central region.\par
\noindent I now elaborate on the origin and importance of each of these
points.
 
%%%%%%%%%%%%%%%%%%%%%%%%%%%%%%%%
%\addcontentsline{toc}{subsubsection}{Anti-flavor symmetry}
\subsubsection*{Anti-flavor symmetry} 
Recent CERN-SPS experiments indicate that up to currently available
energies the central rapidity interaction region has a sizable baryon
number and therefore a relatively large baryo-chemical potential
$\mu_{\rm B}$, in CERN experiments with S$\to$W at 200 GeV\,A interaction
we will see that it is about 340 MeV. Statistical models permit
straightforward evaluation of the quark densities in the QGP phase once
$\mu_{\rm B}$ is known. 
 
One easily finds that the heavy quark flavor has a comparable abundance
to the light quarks because of the finite baryon density in the
interaction region. For a central QGP-fireball in chemical equilibrium,
the number of light antiquarks is governed by the factor $e^{-\mu_{\rm
B}/3T}$, while the deconfined strange and antistrange quarks, are not
affected in the QGP by $\mu_{\rm B}$ and are governed in comparison to
$\bar u, \bar d$ quarks only by their non zero mass described by the
factor $e^{m_{\rm s}/T}$ (with $m_{\rm s}\simeq 150-180$ MeV). Both
factors are not very small and also as mentioned, quite similar in
magnitude. Consequently, provided that strangeness production has
saturated the available phase space, the abundance of antiquarks $\bar u, \bar d, \bar s$ will be very similar.
 
%%%%%%%%%%%%%%%%%%%%%%%%%%%%%%%%
\subsubsection*{Production rates} 
%\addcontentsline{toc}{subsubsection}{Production Rates} 
Rates for production of $s\bar s$ pairs in the QGP phase were often calculated$^{4,5,7}$ and the relaxation time constant which characterizes the scale of time needed to saturate the phase space is of the order of $10^{-23}$~s results, while in hadronic gas phase it is 10 to 30 times slower$^{4}$ at the same temperature and baryo-chemical potential. This difference is mainly due to the presence of gluons in QGP and to the difference in reaction thresholds. On the other hand the typical time scale for the creation and decay of a central fireball can be estimated as the time to traverse at light velocity, the fireball diameter $2R$~fm i.e. $\simeq 2-4\times 10^{-23}$~s. If the fireball is made of QGP, and is sufficiently large, \textit{e.g.} formed in Pb--Pb collisions, strangeness abundance can reach statistical equilibrium values, in a thermal hadronic gas this is not expected because of the long relaxation time. Should there be some anomalous production mechanisms involving multi-particle scattering, then we have to turn to the next point.
 
%%%%%%%%%%%%%%%%%%%%%%%%%%%%%%%%
\subsubsection*{$s\bar s$--density}
%\addcontentsline{toc}{subsubsection}{$s\bar s$--density}
Even in a slow hadronization of an expanding QGP, $s\bar s$ density (now half as high as at its peak due to dilution in expansion) is$^{4}$ about 0.4 strange particle pairs per fm$^3$. In the explosive disintegration \lq\lq Particle Production in Highly Excited Matter\rq\rq\ the discussion of strangness saturation in the QGP fireball. These introductory remarks appeared under title \lq\lq On the Trail of Quark-Gluon-Plasma: Strange Antibaryons in Nuclear Collisions.\rq\rq scenario, the density of strangeness is that of a QGP, thus for a fully saturated phase space up to 0.8 strange particle pairs per fm$^3$. Such a high local strangeness density should favor the formation of multi-strange objects, and particularly multi-strange antibaryons: $\overline{\Omega}=\bar s\bar s\bar s$, $\overline{\Xi}=\bar q\bar s\bar s$. In the hadronic gas phase we lack the strangeness density and therefore we should not expect to form these particles abundantly, as a series of unlikely reactions has to occur in their formation, while their destruction is easily possible in collisions with the hadronic gas constituents
 
%%%%%%%%%%%%%%%%%%%%%%%%%%%%%%%%
\subsubsection*{Direct reaction \lq background\rq} 
%\addcontentsline{toc}{subsubsection}{Direct reaction \lq background\rq} 
It is helpful to recall the magnitude of backgrounds expected for the
production of the multi strange (anti) baryons. The $\overline{\Xi}/
\overline{Y}|_{p_\bot}$ (where $Y=\Lambda,\Sigma$ are the $qqs$ hyperons) ratio seen at ISR at $\sqrt{s}$ = 63 GeV is only 0.06$\pm$0.02 in the central rapidity region$^{8}$. The expected quark-gluon matter result with saturated phase space is up to ten times greater and greatly
enhanced yields have been recently reported by the WA85
collaboration$^9$.
 
The predicted huge strangeness pair density in QGP phase is in my opinion the main point of interest and should be relentlessly pursued by further measurement of the diverse strange particle signatures. It is hard to imagine another scenario in which in particular the strange antibaryons would be abundantly produced: I note that, for example, in a chirally symmetric phase in which the kaon mass could be considerably reduced and hence strangeness could possibly be produced abundantly, there is no  particular reason to expect unusual coagulation of (anti)strangeness into multi-strange antibaryons.


%%%%%%%%%%%%%%%%%%%%%%%%%%%%%%%%%%%%%%%%%%%%%%%%%%%%%%%%%%%%%%%%
\section*{Kinetics of strangeness production and evolution} 
\addcontentsline{toc}{subsubsection}{Kinetics of strangeness production and evolution} 
%%%%%%%%%%%%%%%%%%%%%%%%%%%%%%%%%%%%%%%%%%%%%%%%%%%%%%%%%%%%%%%%
\subsection*{Elements of kinetic scattering theory}
%%%%%%%%%%%%%%%%%%%%%%%%%%%%%%%%%%%%%%%%%%%%%%%%%%%%%%%%%%%%%%%%
Since the time scale in a typical nuclear collision is very short, the
strangeness content of either the QGP phase or the HG phase cannot
without further study be assumed to be saturated (be in `absolute
chemical equilibrium') and it is necessary to determine in a kinetic
approach the rate of strangeness production in alternative scenarios of
fireball evolution. In order to proceed we have to first compute the
typical time required to produce strange quark flavor in the abundance
corresponding to fully saturated phase space. This chemical relaxation
time constant is strongly state dependent, as quite different processes
are contributing in the QGP or HG states of hadronic matter.
 
The strangeness relaxation time constant for the quark-gluon phase is
believed to be in competition with the typical time scale for the
creation and decay of a fireball. Sophisticated calculations which I will
introduce below only in qualitative terms show that we can expect that
strangeness will nearly saturate the available phase space should a
quark-gluon deconfined phase be formed. It is evident that the
coincidental similarity of the computed time constant of strangeness
production with the computed life-span of the QGP has the effect of
making strangeness a quantity very appropriate to the study of the
dynamics of nuclear collisions and QGP. There is furthermore a
considerable impact of the hadronization scenario on strange antibaryon
yield. I like to assume rapid disintegration of the putative QGP phase 
and therefore the strange antibaryon particle yields are representative
of the QGP conditions. An alternative  scenario would be to assume a
relatively slow hadronization which leads to particle yields
characteristic of a (nearly) chemically saturate  hadronic gas$^{10}$.
 
We now briefly consider the theoretical method for the computation of the
rate of the strange quark pair production. Whichever the microscopic
mechanism one adopts for computation of the strange flavor production in
the yet unknown form of high density nuclear matter that has been
generated in these collisions, one can identify the different factors
controlling the yield of the strangeness production processes in a rather
model independent way. Consider two as yet unidentified constituent parts
of centrally interacting nuclei, $A$ and $B$ producing strangeness in
individual collisions. The total number of pairs produced (neglecting
possible strangeness annihilation), leading either to deconfined {\it or}
bound (confined) strange quarks within individual hadrons, is given by
\begin{align*}\tag{1}
N_{\rm s}= V\cdot t\cdot \left(\frac{dN_{\rm s}}{dVdt}\right)\;.
\end{align*}
Here $V$ and $t$ describe the 3+1 dimensional volume in which the
reactions have taken place. The (invariant) rate of production per unit
of time and volume is given by
\begin{align*}\tag{2}
\left(\frac{dN_{\rm s}}{dVdt}\right)\equiv {\cal A}=
\langle\sigma^{\rm s}_{\rm AB}v_{\rm AB} \rangle 
\rho_{\rm A}\rho_{\rm B}\;. \label{Arateis}
\end{align*}
Since  $\rho=N/V$, the specific strangeness yield is:
\begin{align*}\tag{3}
\frac{N_{\rm s}}{n_\pi}=\frac{N_{\rm A}}{n_\pi}\cdot 
\frac{N_{\rm B}}{V} \cdot t\, \cdot 
\langle \sigma^s_{\rm AB} v_{\rm AB}\rangle\;. 
\label{NS}
\end{align*}
The first factor $N_{\rm A}N_{\rm B}/n_\pi$ is rather independent of the
form of {\it proto}matter: the number of components in $A$ or $B$, be
they gluons and quarks or be they pions, will always remain of the same
magnitude as the final multiplicity. This is dictated by the entropy
conservation believed to hold during the evolution of the thermalized
central fireball. Enhancement of strangeness production (as reported by
many CERN and BNL experiments) relative to normal hadronic interactions
(e.g. p+A interactions) is in view of Eq.(\ref{NS}) due to:\\
\indent 1: smaller effective volume $V$ per particle and/or\\
\indent 2: longer interaction time $t$ and/or\\
\indent 3: enhanced microscopic cross section.\\
All these conditions are satisfied in the QGP fireball. Note that the
cross section for formation of particles has the general behavior (note
that $s$ here stands for $(\sqrt{s})^2$, not strangeness):
\begin{align*}\tag{4}
\sigma\simeq k\frac{\alpha^2}{s}\ \sqrt{1-s_{\rm th}/s}\;,
\end{align*}
where the constant $k$ is generally $O(1)$, the threshold $s_{\rm th}$
controls the low energy behavior and the high energy behavior is governed
by the usual $1/s$ form, and $\alpha$ is the strength of the interaction.
These (angle averaged) cross sections for strangeness production have
been studied for many processes involving light quarks, gluons, pions and
so on. They can be parameterized successfully by taking $k\alpha^2\simeq
1$, leading to values of about 0.5~mb for processes at
$\sqrt{s}=2.5T+2.5T\sim1$~GeV. The threshold $s_{\rm th}$ differentiates
to some degree the differing possible processes - in the QGP we expect
$s_{\rm th}=2m_{\rm s}\simeq 350$ MeV, while, in hadronic interactions,
this value is considerably greater on the scale of relevance here: 700
MeV in $\pi+\pi\rightarrow K+K$ reactions. Also, in a confined phase, one
cannot invoke summation over color quantum numbers in the final state,
reducing cross sections still further.
 
%%%%%%%%%%%%%%%%%%%%%%%%%%%%%%%%%%%%%%%%%%%%%%%%%%%%%%%%
\subsection*{Gluons in plasma} 
\addcontentsline{toc}{subsubsection}{Gluons in plasma} 
Including a first-order perturbative effect$^{11}$ the equilibrium gluon
number density in QGP can be estimated using Eq.\,(\ref{fg}) to be:
\begin{equation}\tag{5}
\rho_g \mbox{(fm}^{-3}) = 1.04 \left( {T\over 160\mbox{MeV}} \right)^3
\left( 1- {{15\alpha_{\rm s}}\over{4\pi}} \right)
\end{equation}
giving for a typical temperature of 200 MeV a value of 0.55 fm$^{-3}$ for
$\alpha_{\rm s} = 0.6$ and 0.8 fm$^{-3}$ for $\alpha_{\rm s} = 0.5$. For
a quark-gluon phase volume with radius 4--5 fm, we therefore have
200--300 gluons. Note also that this density rises as the cube of the
temperature. Because gluons can be created and annihilated easily in
interactions with other gluons and light quarks, the gluon density
closely follows the evolution of temperature in the course of the
quark-gluon phase evolution. The equilibrium gluon energy density is
\begin{equation}\tag{6}
\epsilon_g = {8\pi^2\over 15} T^4 \left( 1 - {15\alpha_{\rm s}\over 4\pi}
\right)\;.
\label{fg}
\end{equation}
 
%%%%%%%%%%%%%%%%%%%%%%%%
%\begin{figure}[t] \sidecaption[t]
\centerline{\includegraphics[width=0.6\columnwidth]{./AllFigs/92ICFig1.png} }
%\caption
\noindent{\small Fig. 1 Lowest-order production of $s\bar s$ by gluons and light quarks
%\label{F1}
}
%\end{figure}
%%%%%%%%%%%%%%%%%%%%%%%%%%%%%%%%

Gluons thus play a major role in the dynamics of the quark-gluon
phase-hadronic gas phase transition, also because they carry much of the 
QGP entropy. It is therefore interesting to note that in an indirect way,
strangeness enhancement demonstrates the dynamical presence of glue
degrees of freedom. In QGP strangeness can be formed by processes shown
in Fig.\,1, and higher order corrections of the same basic type.
Calculations show that it is predominantly formed by reactions of gluons,
rather than quarks, despite the fact that the QCD cross sections, shown
in Fig.\,2, are similar for both processes. However the
statistical
factors entering the thermal average will strongly favor the gluon
induced processes: there are simply more glue-glue than quark-antiquark
collisions of suitable quantum number in the plasma. In a scenario of QGP
based fireball practically all of the $s{\bar s}$-pair production results
from collisions of the central gluons, which in a first approximation can
be assumed to be in a thermal distribution. Because of glue dominance of
the production process, the time evolution of strangeness during the
production process is a function of temperature, which solely controls
the glue abundance, but not of the baryo-chemical potential, which
determines the quark densities. Consequently, the detailed baryon number
retained in the plasma (baryon stopping) is of no importance for strange
particle yield; the actual plasma lifetime, volume and temperature (i.e.
gluon density) are the critical parameters  determining the absolute
yield in baryon free and baryon rich environments.
 
%%%%%%%%%%%%%%%%%%%%%%%%
%\begin{figure}[t] \sidecaption[t]
\centerline{\includegraphics[width=0.6\columnwidth]{./AllFigs/92ICFig2.png}}
%\caption
\noindent{\small Fig. 2 Strangeness production cross sections in QGP, $\alpha_{\rm s}=0.6$ and $m_{\rm s}=170$ MeV
%\label{F2}
}
%\end{figure}
%%%%%%%%%%%%%%%%%%%%%%%%%%%%%%%%

Gluons not only produce strangeness flavor dominantly but they provide
the key distinction between the QGP phase and the HG. The high gluon
abundance and density in the plasma impacts the entire history of the
plasma state, in particular also the process of hadronization at the end
of the quark-gluon phase lifetime, in which appreciable strangeness
production occurs again. Indeed, {\it abundant strangeness should be
viewed  as a signal for presence of gluons} or alternative color charged
objects which are not quarks (strings, ropes etc). 
%
%%%%%%%%%%%%%%%%%%%%%%%%%%%%%%%%%%%%%%%%%%%%%%%%%%%%%%%%%%%%%%%%%% 
\section*{Approach to absolute chemical equilibrium}
\addcontentsline{toc}{subsubsection}{Approach to absolute chemical equilibrium}
%%%%%%%%%%%%%%%%%%%%%%%%%%%%%%%%%%%%%%%%%%%%%%%%%%%%%%%%%%%%%%%%%% 
In order to quantify the strangeness production in the dynamical
situation of the rapidly evolving heavy ion collision, it is convenient
to introduce the  quantity:
\begin{equation}\tag{7}
\gamma_{\rm s}\equiv {\int d^3\!p/ 2\pi^3\,
     n_{\rm s}(\vec p,\vec x;t)\over N_0/V}\;,
\end{equation}
which characterizes approach to saturation of the phase-space of strange
particles. $N_0/V$ is the equilibrium particle density. The integration
over the momenta is appropriate if the thermal (kinetic) equilibration 
occurs at a considerably shorter time scale than the (absolute) chemical
equilibration. This observation implies that the factor $\gamma_{\rm s}$
effectively enters the momentum distribution as a multiplicative factor:
\begin{equation}\tag{8}
n_{\rm s}(\vec p,\vec x;t)=\gamma_{\rm s}(t)n_0(\vec p;T(\vec x),\mu_{\rm
s}(\vec x))\;,
\end{equation}
where the $\vec x$ dependence is contained in the statistical parameters.
The absolute chemical equilibrium corresponds to $\gamma_{\rm s}=1$ found
for a fully saturated phase space of QGP or HG. In HG the absolute
density $N_0^{\rm HG}/V$ is smaller by a factor 2-5 (in dependance on the
physical conditions in which the phases are compared) primarily due to
the larger degeneracy following from the liberation of the color degrees
of freedom, and the lower masses of strangeness carrying particles. I
will show below that it is indeed quite easy to measure the value of
$\gamma_{\rm s}$, and thus this discussion is a very important practical
element in understanding the behavior of hadronic phases: $\gamma_{\rm
s}$ can be studied varying a number of parameters of the collision, such
as the volume occupied by the fireball (varying size of the colliding
nuclei and impact parameter), the trigger condition (e.g. the
inelasticity), the energy of colliding nuclei, searching for the
threshold energy of abundant strangeness formation.
 
The theoretical dynamical model to study $\gamma_{\rm s}(t)$ has been
developed to considerable detail. It arises from a standard population
evolution equation. Detailed balance assures that the production dn
annihilation processes are balancing each other as $\gamma_{\rm s}\to 1$:
\begin{align*}\tag{9}
2\tau_{\rm s}\left( {d\gamma_{\rm s}\over dt}+\gamma_{\rm s}
          {dV\over Vdt}\right)=1-\gamma_{\rm s}^2(t)\;,
\label{gamt}
\end{align*}
where 
\begin{align*}\tag{10}
\tau_{\rm s}=0.5n_0/{\cal A}\;,
\label{taus}
\end{align*}
with $\cal A$ as defined by Eq.\,(\ref{Arateis}). The factor $0.5$ is
introduced in this  definition Eq.\,(\ref{taus}) of the strangeness
relaxation time constant to allow for the relation at large times to
assume the standard form: $1-\gamma_{\rm s}\propto e^{-t/\tau_{\rm s}}$.
The second term on the left hand side of Eq.\,(\ref{gamt}) is the
dilution term arising from the possible expansion of the volume occupied
by the system. Ignoring the dilution and assuming that there is no
appreciable change in $\tau_{\rm s}$ with time, a well known solution of
Eq.\,(\ref{gamt}) is:
\begin{equation}\tag{11}
\gamma_{\rm s}(t)=\tanh(t/2\tau_{\rm s})\;.
\end{equation}
 
The first calculation$^{7}$ of $\tau_{\rm s}$ in QGP in which glue
processes were considered without dilution, see Fig.\,IC-3, has shown
that
strangeness can be produced rapidly and abundantly; subsequent
study$^{12}$ has obtained $\gamma_{\rm s}(t\to\infty)$ including the
dilution effect for both QGP and HG phases. It is evident that a
quantitative calculation of the value of $\gamma_{\rm s}$ reaches in the
actual collision requires as input the relaxation time constant
$\tau_{\rm s}$ and the logarithmic derivative of the local volume, i.e.
the dilution of the local density as function of time due to the dynamics
of the collision. Calculations so far performed use perturbative QCD to
obtain $\tau_{\rm s}$ and model the dilution term using dilution $d(\ln
V)/dt=n/t$ with $n=3$ for a spherically expanding fireball and $n=1$ for
longitudinal expansion. The latter case leads to considerably greater
saturation, also because the temperature parameter which enters the
relaxation time decreases as $T\propto t^{-n/3}$ and there is more time
for strangeness production. However, the spherical expansion is probably
a more appropriate model for the situation encountered in S--W or Pb
collisions, and certainly more applicable to the case of forthcoming Pb--Pb experiments.
 
%%%%%%%%%%%%%%%%%%%%%%%%
%\begin{figure}[t] \sidecaption 
\centerline{\includegraphics[width=0.75\columnwidth]{./AllFigs/92ICFig3.png}}
%\caption
\noindent{\small Fig. 3 Relaxation time constants for strangeness production in QGP: total, gluons only ($GG\to s\bar s$) and light quarks only ($q\bar q\to s\bar s$) with $\mu_{\rm B}=400$ MeV,  computed for $\alpha_{\rm s}=0.6$ and $m_{\rm s}=170$ MeV.
%\label{F3}
}
%\end{figure}
%%%%%%%%%%%%%%%%%%%%%%%%%%%%%%%%


I will not review in this brief and qualitative talk the detailed results
about the asymptotic value $\gamma_{\rm s}$ assumes in different
scenarios. Suffice here to say that for a $3$ fm radius initial plasma
drop at initial temperature of 250 MeV one estimates $\gamma_{\rm
s}(t\to\infty)\simeq0.5$, with an error as large as 50\% due to the
assumed values of the QCD parameters such as the coupling constant
$\alpha_{\rm s}$ and the strange quark mass $m_{\rm s}$, not to mention
the systematic uncertainty associated with use of perturbative expansion.
Given the current remarkable results on $\gamma_{\rm s}$ it appears
imperative that the models be improved to the level of the experimental
precision which is presently at about 15\%.

%
\footnotetext{\vspace*{-0.5cm}\begin{itemize} 
\item[1]
J. Rafelski,  Phys. Rep.  C \textbf{88} 331 (1982)
\item[2]
J.~Rafelski and M.~Danos,  Phys. Lett.  B \textbf{192} 432 (1987)
\item[3]
J. Rafelski {\it Phys. Lett.} B262:333 (1991);  Nucl. Phys. A \textbf{544} 279c (1992)
\item[4]
P.~Koch, B.~M\"uller and J.~Rafelski,  Phys. Rep.  C \textbf{142} 167 (1986)
\item[5]
H.C.~Eggers and J.~Rafelski, Int. Journal of Mod. Phys.  A \textbf{6} 1067 
(1991) 
\item[6]
J.~Letessier, A.~Tounsi, U.~Heinz, J.~Sollfrank and J.~Rafelski, {\it
Strangeness Conservation  in Hot Fireballs} Preprint
Paris PAR/LPTHE/92-27, Regensburg TPR-92-28, Arizona 
AZPH-TH/92-23, 1992 (published: Phys. Rev. D \textbf{51} 3408  (1995))
\item[7]
J.~Rafelski and B.~M\"uller,  Phys. Rev. Lett.  \textbf{48} 1066 (1982); and \textbf{56} 2334(E) (1986) 
\item[8]
T.~\AA kesson et al. [ISR-Axial Field Spect. Collab.], Nucl. Phys. B \textbf{246} 1 (1984)
\item[9]
E. Quercigh, this volume; S. Abatzis {\it et~al}.,   Phys. Lett. 
B \textbf{270} 123 (1991)
\item[10]
J. Zimanyi, this volume~\cite{Gutbrod:1993rp}, pp.243-270 
\item[11]
S.A.~Chin,   Phys. Lett.  B \textbf{78} 552 (1978)  
\item[12]
P. Koch, B. M\"uller and J. Rafelski,  Z. Physik  A \textbf{324} 3642 (1986)
\end{itemize}
}
\end{mdframed}
%%%%%%%%%%%%%%%%%%%%%%%%%%%%%%%%%%%%%%%%%%%%%%%%%%%%%%%%%%%%%%%%%%



 
%%%%%%%%%%%%%%%%%%%%%%%%%%%%%%%%%%%%%%%%%%%%%%%%%%%%%%%%%%%%%%%%%%
\subsection{Strange antibaryon production}
\subsubsection{High $p_\bot$ recombinant enhancement}\label{sec:highPT}

In the March 1987  lecture~\cite{Rafelski:1987bx} \lq\lq Strange Signals of Quark-Gluon Plasma\rq\rq\ at the Rencontres des Moriond in Les Arcs, France~\cite{TranThanhVan:1987tm}, I extended  the science case for strange antibaryon signatures of quark-gluon plamsa employing the recombinant mechanism~\cite{Koch:1986ud,Rafelski:1987un}. The key result is the enhancement of particle production in  the relatively high $p_\bot$ domain.

These results were confirmed by the RHIC experiments and the data analyis along the line of the recombination model has found general acceptance~\cite{Fries:2003vb,Fries:2003kq} -- the pioneering contributions~\cite{Rafelski:1987un,Rafelski:1987bx}, are not well known. The March 1987 workshop  contents in the pre-web period was seen only by participants -- and few from the RHI community were present at this particle physics meeting. \\ 

\noindent\textit{%
CERN-TH/4716, May 1987 report on high $p_\bot$ strange antibaryons~\cite{Rafelski:1987bx} produced in quark combinant processes:}\\[-0.7cm]
%
\begin{mdframed}[linecolor=gray,roundcorner=12pt,backgroundcolor=Dandelion!15,linewidth=1pt,leftmargin=0cm,rightmargin=0cm,topline=true,bottomline=true,skipabove=12pt]\relax%
%
{\Large {\bf Strange Signals of Quark-Gluon Plasma}}\\[0.4cm]
%Prepring CERN-TH/4716
\textbf{Abstract:} It is shown that an overabundance of $\bar\Xi$ is a diagnostic observable of quark-gluon plasma phase of matter. The pertinent physical phenomena are briefly surveyed. New results on $\bar\Xi/\bar\Lambda$ ratios at medium to large transverse momenta are presented. Relevant experiments are discussed.
 
\section*{Introduction}

We address here the question of how the occurrence of otherwise rarely produced multiply strange hadrons can be used to study the formation of the new phase of matter, the quark-gluon plasma\footnotemark[1]\footnotetext{$^1$J. Rafelski, Nucl. Phys. A418 (1984) 215 and references therein} (QGP). At the outset it is important to recognize that the basic subprocess for strange quark production, namely the pair production process is, in principle, the same for both phases of hadronic matter, viz. QGP and Hadronic Gas (HG). But in the latter case of well separated individual hadrons with the nonperturbative (\lq true\rq) QCD vacuum in-between, strangeness production can only take place during the actual constraints in space and time. Furthermore, all initial and final state hadrons are color singlets and the effective number of available degrees of freedom is greatly reduced in comparison to the QGP, in which colored states are permitted.

In the plasma phase there is not only significantly more rapid strangeness production\footnotemark[1]  the higher possible equilibrium strange quark abundance per unit of volume facilitates, in particular, abundant formation of multiply strange antibaryons when the plasma state fragments and recombines to form individual hadrons. In the baryon rich regime of quark-gluon plasma the $\bar s$-quarks are more abundant than the $\bar u$- or $\bar d$-quarks with the consequence that formation of antibaryons with high strangeness content is particularly facilitated during the conversion to the hadronic gas (HG) of the plasma phase\footnotemark[2]\footnotetext{$^2$J. Rafelski and R. Hagedorn, \lq\lq From Hadron Gas to Quark Matter II\rq\rq. CERN-TH 2969 (1980) which 	appeared in \textit{Thermodynamics of Quarks and Hadrons,} (North Holland, 1981) H. Satz, ed.}. One of the highly relevant insights not discussed here is the fact that during the conceivable hadronic reaction time of less than 10$^{22}$ sec, the strangeness produced in HG will not saturate the available (small) phase space. However, any strange particles present will be nonetheless efficiently distributed among various individual hadronic states. These remarks about chemical reequilibration of strange particles apply also to the debris of the QGP after its hadronization. But particles emitted from QGP early on will not be affected by these rescattering phenomena. Strange antibaryons emanating early from the expanding ball of QGP may be far off the relative equilibrium. Upon a brief survey of $s \bar s$-production mechanisms we turn to describe key results on strange particle production with the emphasis being laid on the rare multiply strange baryons.

%%%%%%%%%%%%%%%%%%%%%%%%%%%%%%%%%%%%%%%%%%%%%%%%%%%%%%%%%%%%%%%%%% 
\section*{Strangeness Production in Quark-Gluon Plasma}
\addcontentsline{toc}{subsubsection}{Strangeness production in quark-gluon plasma}
%%%%%%%%%%%%%%%%%%%%%%%%%%%%%%%%%%%%%%%%%%%%%%%%%%%%%%%%%%%%%%%%%% 
In QGP the gluonic production rate dominates strangeness production and leads to equilibration times comparable to the expected plasma lifetime\footnotemark[3]\footnotetext{$^3$J. Rafelski and B. M\"uller, Phys. Rev. Lett. \textbf{48} (1982) 1066; and \textbf{55} (1986) 2334(E)}. The averaged cross sections for quark-pair production in lowest order in the QCD coupling constant $\alpha_s$ are used to obtain the rates for strangeness production. In such calculations it is assumed that each perturbative quantum (light quark, gluon) will rescatter many times during the lifetime of the plasma. Hence the required momentum distribution functions are taken to be the statistical Bose, or respectively, Fermi distribution functions, where the temperature $T$ and chemical potential $\mu$ may be functions of $\vec x$, the location of a volume element within the fireball.

\textit{(Repetative Fig.1 showing rates of  strangeness production is omitted.)}\\
%In Fig.M-1a the rates for strangeness production for the parameter values $\alpha_s  = 0.6, \;m_s  = 150\;\mathrm{MeV}$ are depicted. 
The gluon contribution dominates the strange\-ness creation rate, while $q \bar{q} \to  s\bar{s}$ (dashed lines) contributes less than 20 percent to the total rate. The relaxation time $\tau$ is also dominated by the gluonic production mechanism and is falling rapidly with increasing temperature.% as shown in Fig.M-2.1b. As Fig.M-1b indicates, 
\ldots there is virtually no net strangeness annihilation $s \bar s\rightarrow g\bar g$ as the plasma expands, because in the expansion process the temperature and strangeness density both drop rapidly decoupling effectively the strangeness abundance from the statistical equilibrium. This is confirmed in detailed calculations including the dilution term\footnotemark[4]\footnotetext{$^4$\,P. Koch, B. M\"uller and J. Rafelski. Phys. Rep. \textbf{142} 	(1986) 167; P. Koch, Ph.D. Thesis, 1986; P. Koch, 	 contribution to this conference}. It is further found that the strangeness density at hadronization of QCD is in the interval 
$0.15/\mathrm{fm}^3 < \rho_s <  0.3/\mathrm{fm}^3$. This value indicates that clustering of two strange quarks in one hadronic volume $V_h \sim  5 \mathrm{fm}^3$ will be frequent. The high particle density of strange quarks in plasma virtually assures that the numerous $s$ and $\bar s$ quarks will facilitate production of otherwise rare particles such as $\bar \Xi, \bar \Omega$ and particularly important, their antiparticles, instead of being bound in kaons only. Consequently, we will emphasize below our expectations about production of these multistrange hadrons.


%%%%%%%%%%%%%%%%%%%%%%%%%%%%%%%%%%%%%%%%%%%%%%%%%%%%%%%%%%%%%%%%%% 
\section*{Strange Hadron Formation from Quark-Gluon Plasma}
\addcontentsline{toc}{subsubsection}{Strange hadron formation from quark-gluon plasma}
%%%%%%%%%%%%%%%%%%%%%%%%%%%%%%%%%%%%%%%%%%%%%%%%%%%%%%%%%%%%%%%%%% 
First we must appreciate that substantial fragmentation of gluons and quarks is required at the transition from QGP to HG. This is easily seen noting that if quarks and antiquarks would recombine into mesons, mainly pions while gluons would vanish in the vacuum, there would be only half as many pions afterwards as there were quarks and antiquarks before which, in turn, are only half of all particles in QGP. Hence the entropy ratio between quark-gluon gas and the hadronic (pion) gas would be 4. In order to conserve entropy during the hadronization process, every gluon and about one third of the quarks must fragment before coalescing into mesons. At finite baryochemical potential the necessity for quark fragmentation is somewhat reduced, since baryon formation accounts for a significant fraction of the total entropy of the hadronic gas. Another aspect of the fragmentation process is that it is producing, with relative strength of about 15\%, further strange quark pairs. This fragmentation value is known from color string breaking considerations.

The flavor composition of all the quarks and antiquarks that finally become constituents of the hadrons produced in the breakup of the plasma is now fully determined. At a given time there are for each flavor, the primary quarks or antiquarks and there are those generated by glue fragmentation. When we combine all these, we obtain the final number of quarks and antiquarks of each flavor, that effectively contribute to hadronization. A combinatoric breakup model to determine the flavor composition of hadrons at the beginning of the evolution of the final hadronic phase can then be used. A further element needed is a model of phase coexistence between HG and QCP. This has been carried through in some detail by P. Koch$^4$. For our purposes it will be sufficient to make a simple estimate. We are particularly interested in the relative abundance of the anticascades ($\bar{\Xi}=\bar{ssq}$) to anithyperon ($\bar Y = \bar{sqq}$). In order to establish this ratio of abundance of $\bar \Xi$ to $\bar Y$, we consider the probabilities $P_i$ of finding the constituent particle in a unit volume $V$. Incorporating gluons which have to fragment when plasma hadronizes into 
$q \bar q$-pairs (85\%) and $s \bar s$-pairs ($f \geq 15$\%), we have
\begin{equation}\tag{3.1}
%(3.1)
{{\bar \Xi} \over {\bar Y}}=
{
{(P_{\bar s} + f P_G)^2 (P_{\bar q} +(1-f)P_G)}
\over
{B(P_{\bar s} +f P_G) (P_{\bar q} + (1-f)P_G)^2}
}\;.
\end{equation}
Here B is the branching ratio reflecting the possibility that an $\bar s GG$ system may make not only the desired system $\bar s\bar q\bar q $ + $qq$, but also systems such as $\bar s q + \bar q G$ etc. Clearly  $B<1$. We have from Eq.(3.1)
\begin{equation}\tag{3.2}
%(3.2)
B{ {\bar \Xi} \over {\bar Y}}=\displaystyle\frac
{{f \over {1-f}}+ { 1 \over{1-f}} \frac{P_{\bar s}}{P_G}}
{1+{1\over{1-f}}\frac{P_{\bar q}}{P_G}}\;.
\end{equation}
Using the statistical weights for $P_{\bar s}/P_G=3/8$ we obtain 
$\bar \Xi/\bar Y >0.6$(!). 

Including the $\bar q$-density is easy, as 
$P_{\bar q}/P_G=e^{-\mu^s_b/3T} 6/8$, 
 with the last factor again being statistical. In detailed calculations in which various branchings for the different reactions have been allowed\footnotemark[5]\footnotetext{$^5$\,M. Jacob and J. Rafelski, "Longitudinal $\bar\Lambda$Polarization, $\bar\Xi$ Abundance and Quark-Gluon Plasma 	 Formation", CERNTH 4649/87, Phys. Lett. B in 	press (published: \textbf{190} (1987) 173)}, the approximate form Eq.(3.2) given the above parameters is well recovered:
\begin{equation}\tag{3.3}
%(3.3)
{{\bar \Xi} \over {\bar Y}} \equiv \frac 3 4
{ 1 \over
{1+0.8e^{-\mu_b /3T}}
}\;.
\end{equation} 

Dashed line in Fig. 2 shows the result Eq.(3.3). It is important to note that a number of these ($\bar \Xi , \bar Y$) particles are expected to be produced in each single nuclear reaction event leading to formation of quark gluon plasma with a volume of several hundred fm$^3$. At this point it is interesting to note that the UA5 collaboration  
(S$  p\bar p  $S) has indeed observed such an anticascade anomaly\footnotemark[6]\footnotetext{$^6$\,G.J. Alver et al (UA5 collaboration) Phys. Lett. B \textbf{151} (1985) 309} in nondiffractive interactions at 
$\sqrt{s}$ = 540 GeV, quoting $\Xi/Y$ ratio of 0.7 at the production point  (note that since baryon density is zero, there is particle-antiparticle symmetry). They further find about 0.1 $\Xi^-$ per event, in which the mean charged particle multiplicity is  $35\pm 4$. We also record that the observed UA5  $\bar \Xi/\bar Y$  ratio is nearly ten times that seen at ISR at $\sqrt{s}=63$\;GeV (central rapidity)\footnotemark[7]\footnotetext{$^7$\,T. Akesson et al (ISR Axial Field Spec. Coll.) 	Nucl. Phys. B \textbf{246} (1984) 1}. It is extremely tempting to conclude that there is quark gluon plasma in $p\bar p$ interactions at  $\sqrt{s}=540$\;GeV, the more so, since the total strangeness yield seems to show anomalous increase as function of 
$\sqrt{s}$ between 500 and 900 GeV.

The discussion presented here confirms the particular suitability of the global abundance of strange antibaryons for diagnosis and study of the quark-gluon plasma state. However, one can also consider the direct effect of the large $s, \bar s$ density in the early plasma before hadronization, and in particular the possibility of strange particle radiation from such a hot plasma state. Two channels of early particle emission may be considered. In the first a fast quark (or diquark) from the plasma impinges on the boundary between the plasma perturbative vacuum and the true vacuum. In the associated color string breaking process, at least one quark-antiquark pair is formed. We will refer to this process as a \lq microjet,\rq\ (not to be confused with the minijet processes). Second, in the prehadronization recombination approach, several constituents of the plasma, clustered into colorless objects, penetrate the surface and hadronize. This latter process is similar to the phase transformation of the plasma to the hadronic gas at the later stages of the plasma life which we have just considered. However, this recombination radiation is, in detail, different in that we may not allow gluon fragmentation. The equilibrium quark abundances in plasma contain the effect of continuous gluon and quark fragmentations and recombinations. Only at phase transition equilibrium is lost and additional microscopic processes such as gluon fragmentation must be explicitly considered.

In Fig. 2 (full lines) we show the predictions based on prehadronization models for the  $\bar \Xi/(\bar Y/2)$ ratio and compare them to global abundance ratios as discussed further above, as functions of the chemical potential (baryon density). The surprise is that in both early emission processes (microjet, recombination) populating medium to high $m_\perp$ portions of the spectra we are led to expect more anticascades than antilambdas. Since  $\bar \Sigma^0 \rightarrow \bar \Lambda +\gamma$ we included $\bar \Sigma^0$ into the \lq lambda\rq\ abundance hence the figure shows only half of the anithyperons abundance  the other half contained in $\bar \Sigma^+$ and $\bar \Sigma^-$ remains invisible in a typical experiment.

%%%%%%%%%%%%%%%%%%%%%%%%%%%%%%%%%%%%%%%%%%%%%%%%%%%%
%\begin{figure}[t]\sidecaption[t]
\centerline{\includegraphics[width=0.65\columnwidth]{./AllFigs/87Moriond31.png}}%3.1
\noindent\small{Fig. 2 $\bar \Xi /(\bar Y /2)$ particle ratio as a function of 
$\mu_b/3T$ for the microjet, recombination pictures of particle radiation at medium to high $E_\perp$. Dashed: global ratio}
%\label{fig:Moriond31}
%\end{figure}
%%%%%%%%%%%%%%%%%%%%%%

\section*{Summary}

The results presented here substantiate the expectation that abundances of strange particles, most notably of strange antibaryons, provide a powerful tool to probe the quark-gluon plasma phase of nuclear reactions at very high energy and perhaps even QGP is found in $p \bar p$ annihilation. Strange hadronic particles are expected to emerge from the quark-gluon plasma phase significantly more abundantly than this would be the case in a purely hadronic gas. It is important here to emphasize that strong enhancements of the strange antihyperon $\bar Y$ and anticascade $\bar \Xi$ production is particularly characteristic for the baryon rich plasma phase. In other words: $\bar Y, \bar \Xi$ and $\bar \Omega$ abundance anomalies are characteristic for plasma formation because they will exceed the size of the phase space of individual hadrons.

The source of all these surprising results about strange hadrons from QGP can be traced back to the fact that strange quark-pair production in the plasma phase proceeds at a sufficiently fast rate to permit statistical equilibrium abundance to be established in less than $10\;\mathrm{fm}/c$. This is due to the abundant presence of gluonic excitations, allowing for quark-pair production in gluon-gluon collisions, c.f. Sec. 2. In a way, therefore, abundant strange antibaryon production is indicative of an environment in which gluon collision processes are an essential element of the reaction picture. We have further argued that normally rare strange antibaryon particles with strangeness content provide a very promising experimental signal in the search of the quark-gluon plasma. In particular, abundant strangeness production is indicative of the presence of gluon excitations, a characteristic property of the deconfined QCD phase. Measurement of $\bar \Xi, \bar Y$ particle spectra at medium $p_\perp$ and central rapidity may reveal most notable anomalies.
\end{mdframed}
%\vskip 0.5cm
%%%%%%%%%%%%%%%%%%%%%%%%%%%%%%%%%%%%%%%%%%%%%%%

\subsubsection{Strange antibaryons; has anyone noticed?}
I made many efforts to present the strange antibaryon signature of QGP to the nuclear science community at large. Despite many topical and Summer/Winter school talks, and  contributed parallel sessions of the American Physical Society (APS) Spring, or Divisional Nuclear Physics Fall meetings, neither I, nor that I am aware,  anyone   presented a theoretical lecture  at a major conference on specifically this topic; I did present many times at small workshops, topical events, theoretical schools with 30-200 participants.

Given the importance of the strange antibaryon signature for the understanding of the QGP this is an unusual situation. There were many talks on all other, arguably less relevant, topics; all it takes is a quick look at the proceedings of Quark Matter conference series. To assure fairness of this remark, I note that the organizers of the Venice QM2018 were intending to invite me to present a plenary lecture. This was motivated  by  ALICE results described in Sec.~\ref{sec:AliceSys}. However, the topic was dropped from the program. I was told that two members of the organizing committee objected since I ask too many \lq direct\rq\ questions. A very good friend gave me the advice: \lq\lq You do not want to protest this year.\rq\rq 

The following abstract could be (changing the experimental code number)  submitted today. However it had been submitted to the attention of the 1992 International Nuclear Physics Conference (INPC 92) held 26 July - 1 August 1992 in Wiesbaden, Germany.\\

\noindent\textit{This talk was not selected by INPC 1992 for presentation:}\\[-0.7cm]
%
\begin{mdframed}[linecolor=gray,roundcorner=12pt,backgroundcolor=Dandelion!15,linewidth=1pt,leftmargin=0cm,rightmargin=0cm,topline=true,bottomline=true,skipabove=12pt]\relax%
%
\begin{center}
{\bf QGP and Strange Antibaryons} 
\end{center}

Substantial enhancement of production rates of {\it multi}strange {\it anti}baryons in nuclear collisions at central rapidity has been identified as an interesting observable and a potential signature for quark-gluon plasma (QGP) formation in relativistic nuclear collisions\footnote{J. Rafelski, Phys. Lett. {\bf B262} (1991) 333; and \lq\lq Strange and hot matter,\rq\rq\ to appear in Nucl. Phys. {\bf A}, 1992 (and references therein, \textit{published}: Volume \textbf{544}, July 1992, pp. 279-292}. A number of CERN experiments has studied this observable in 200~GeV~A Sulphur collisions with Sulphur and/or heavy nuclei (experiments WA85, NA35, NA36). In the past two years results have been presented which have eluded explanation in terms of models developed for $p$--$A$ scattering processes. These results suggest that the production of $\Lambda, \overline{\Lambda}$ is indeed occurring in a centrally formed fireball reaching temperatures of $T=210\pm10$ MeV\footnote{J. Rafelski, H. Rafelski and M. Danos, \lq\lq Strange Fireballs,\rq\rq\ Preprint AZPH-TH/92-7, \textit{published: Phys. Lett. B \textbf{294}, 5 November 1992, pp 131-138}}. It is possible to consider the properties of a hot hadronic matter fireball source for
$\Lambda, \overline{\Lambda}, \Xi, \overline{\Xi}$ without identifying the nature of the state, viz. if it is deconfined type QGP or normal $\pi,\,N$, etc hot nuclear gas (HG). The difference between these states is than seen in the \lq\lq measured\rq\rq\ values of the parameters:
\begin{itemize}
\item $0<\gamma<1:$ the degree of saturation of the strangeness phase space, with the expectation being $\gamma\simeq 0$ for HG and $\gamma\simeq 1$ for QGP, the enhancement arising from effective glue-based strangeness production processes;
\item $\mu_s:$ the strange quark chemical potential which distinguishes the strange from antistrange hadrons and in general vanishes for the case of QGP, and assumes a wide range of values for the evolving HG fireball;
\item $\mu_b/T:$ determines the density of baryons in the central fireball, viz the degree of their stopping and hence the fireball energy and entropy content.
\end{itemize}
When the data of the experiment WA85 for the abundance ratios of strange baryons and antibaryons are interpreted in such a way~[a] one finds the remarkable set of values: $\gamma=0.7,\ \mu_s=0,\ \mu_b/T=1.5$ with relatively small error bars. These results favor the interpretation of the central fireball in terms of a state which is abundantly producing the strangeness flavor, is symmetric with regard to the formation of both strange and anti-strange quarks, and has considerable baryon content and density; the possibilities that such a result could arise in the context of a QGP or HG fireball are considered\footnote{In preparation, \textit{Published: Jean Letessier, Ahmed Tounsi, and Johann Rafelski \lq\lq Hot hadronic matter and strange anti-baryons,\rq\rq\ Phys. Lett. B \textbf{292}, 15 October 1992, pp. 417-423}}. 
 
It is impossible to come to a definitive conclusion on the basis of a single experimental point. To assert a reaction mechanism we will have to find agreement between the systematic behavior of the measurements and theoretical expectations, while a number of available parameters (such as the mass of the projectile, impact parameter, and the energy per nucleon) are varied. Nevertheless, even today these results imply for a fully strangeness saturated (as would be expected of the larger Pb-Pb originating fireball if formed at similar physical conditions) one aught
to find:
${\overline \Xi}/{\bar \Lambda}={\bar \Lambda}/{\bar p}=1.55\pm0.13 \;,\ 
\Xi/\Lambda=\Lambda/p=0.64\pm0.05 \;. $ 
This result applies to the high transverse momentum sector of the spectrum and relates particle abundances considered at the same {\it transverse energy} in a narrow, central region of rapidity. It displays the anomaly that the more heavy and strange anti-baryon is more abundant. 
\end{mdframed}
%\vskip 0.5cm
%%%%%%%%%%%%%%%%%%%%%%%%%%%%%%%%%%%%%%%%%%%%%%%


%%%%%%%%%%%%%%%%%%%%%%%%%%%%%%%%%%%%%%%%%%%%
\subsection{Soft and strange hadronic observable of QGP at RHIC}
\textit{%
In a lecture~\cite{Rafelski:1990dk}  presented at the July 1990  RHIC-BNL-Workshop~\cite{Fatyga:1990ubn}  I describe strangness flavor related hadronic observables  and evaluate their significance for the observation and identification of QGP:}\\[-0.7cm]
%%%%%%%%%%%%%%%%%%%%%%%%%%%%%%%%%%%%%%%%%%%%%%%%%%%%%%%%%%%%
 \begin{mdframed}[linecolor=gray,roundcorner=12pt,backgroundcolor=Dandelion!15,linewidth=1pt,leftmargin=0cm,rightmargin=0cm,topline=true,bottomline=true,skipabove=12pt]\relax%
%
\centerline{\bf\Large Flavor Flow From Quark-Gluon Plasma}
\addcontentsline{toc}{subsubsection}{Flavor flow from quark-gluon plasma}
%%%%%%%%%%%%%%%%%%%%%%%%%%%%%%%%%%%%%%%%%%%%%%%%%%%%%%%%%%%%
\noindent{\bf Abstract:}
%In this lecture presented at the July 1990 RHIC-BNL-Workshop 
I discuss diverse hadronic observable of the reactions between relativistic heavy ions related to the production and flow of flavor, and its significance for the observation and identification of quark-gluon matter. This discussion in particular includes a brief survey of our current understanding of the strange particle signature of quark-gluon plasma.\\[0.5cm]
%%%%%%%%%%%%%%%%%%%%%%%%%%%%%%%%%%%%%%%%%%%%%%%%%%%
\noindent\textbf{\large Looking For Quark-Gluon Plasma}\\[0.1cm]
The inherent difficulty of the study of Quark-Gluon matter is its expected fleeting presence when two heavy nuclei collide. Therefore, an important element in theoretical investigation of relativistic heavy ion collisions has been the identification of an observable of this new state of matter. We must from outset realize that an observable can be either \lq characteristic\rq\ and/or\lq descriptive\rq. A characteristic measurement would tell us unequivocally that some time during the nuclear interaction quark-gluon matter has been formed. A descriptive observable will not necessarily be characteristic, but should allow us to study the properties of the quark-gluon matter phase, if we can with certainty assume its formation.  
 
First I note that we can in principle measure as function of rapidity and transverse mass the following simple hadronic observable:
\begin{itemize}
\item the yield of charge;
\item the yield of baryon number;
\item the yield of strange particles and in particular that of:
\begin{itemize}
\item single strange particles ($\bar s q, s\bar q, \bar s qq, s\bar q\bar q$),
\item multi strange baryons ($ssq, \bar s\bar s\bar q, sss, \bar s\bar s\bar s$),
\item $\phi$-meson yield ($\bar s s$),
\item HBT correlations of strange particles,
\item strange exotica.
\end{itemize}
\end{itemize}

In order to present a comprehensive and complete description of the diverse processes occurring, a theoretical interpretation of the data must necessarily account for details of the collision dynamics. This information is at present not available for the energies accessible at RHIC and theoretical models are by necessity dependant on a number of assumptions, in absence of a truly fundamental approach to the collision dynamics. Furthermore, there are additional uncertainties related to carrying through a simulation of the collision dynamics involving a possible phase transformation. Thus it is of essence for the discussion here presented that initial RHIC experiments determine:
\begin{itemize}
\item the \lq\lq stopping power\rq\rq\ of the constituent quarks in the colliding nuclei, as measured by the rapidity distribution of the electrical charge; \item the baryon number stopping power of the nuclear medium, as measured conveniently by rapidity distribution of (strange) baryons; \item the entropy produced in the collision, as measured e.g.~by the particle multiplicity, in particular pion to baryon ratio as function of rapidity; 
\item the characteristic \lq\lq temperature\rq\rq, as measured e.g.~by the slopes of transverse mass spectra;
\end{itemize}
The primary observable we address here is the strange quark flavor and in addition to the above I would like to see a measurement of:
\begin{itemize}
\item the high density, above-equilibrium nature of the over saturated strangeness phase space density, which is noted for by the abundance of multistrange baryons and in particular their anomalous abundance enhancement as compared to singly strange antibaryons, which in turn are enhanced as compared to antiprotons produced; and 
\item the overabundance of strangeness flavor as measured by overabundance of strange particles produced in $A$--$A$ collisions compared to $p$--$p$ and $p$--$A$ reactions;
\item kaon HBT correlations, which should show a smaller source than pionic HBT size of the fireball.
\end{itemize}
The remainder of this lecture is organized as follows: Next, I explain why strangeness flow is viewed as an observable of quark-gluon matter. This is followed by a brief consideration of lessons from the present strangeness data.\\[0.5cm]
%%%%%%%%%%%%%%%%%%%%%%%%%%%%%%%%%%%%%%%%%%%%%%%%%%%%%%%%
\textbf{\large   Why flavor--strangeness?}\\[0.1cm]
%%%%%%%%%
\addcontentsline{toc}{subsubsection}{Why flavor--strangeness?}
%%%%%%%%%%%%%%%%%%%%%%%%%%%%%%%%%%%%%%%%%%%%%%%%%%%%%%%%
I proposed about ten years ago$^{1,2}$  strangeness as an
observable of quark-gluon matter. Following on early equilibrium
considerations it became
soon apparent that strangeness production must be treated in a kinetic
approach\footnotemark[3]. Furthermore, in a review prepared for QM\,1982\footnotemark[4] 
\begin{quote}
\ldots measurement of production cross section of strange antibaryons could be already quite helpful in the observation of the phase transition \ldots\\ Measurement of the relative $K^+/K^-$ yield, while indicative for the value of the chemical potential (in hadronic gas phase) may carry less specific information about the plasma. The $K/\pi$ ratio may indeed also contain relevant information - however it will be more difficult to decipher the message ...it appears that otherwise quite rare multistrange hadrons will be enhanced ... hence we should search for the rise of the abundance of particles like $\Xi, {\bar \Xi}, \Omega, {\bar \Omega}, \phi$ and perhaps highly strange pieces of baryonic matter (strangeletts), rather than in the K-channels. It seems that such experiments would uniquely determine the existence of the phase transition to quark gluon plasma \ldots.
\end{quote}
This is in a shell nut my position today, though in the elapsed decade the initial simple ideas have undergone a substantial evolution$^{5,6}$ and have come under intense scrutiny, see Ref.\,[7] and references therein.

I think that those who have been critical of \lq\lq strangeness\rq\rq\ have never taken time to study the detailed ideas related to flavor (strangeness) flow, of which the simplest point of view I have quoted myself above. It seems indeed that we have just gone more than 8 years back, as in the strangeness review at QM\rq\ 90 we can read (see Ref.\, [7]) 
\begin{quote}
Strangeness has been proposed as a signal for quark- gluon plasma formation in RHI collisions. Subsequent to the original proposal several papers appeared which considerably weakened (hic) the early claims (which???) made for strangeness production in heavy ion collisions (references follow from 1985,1986,1988 addressing the question what Kaons can tell us or not.).  I  quote from Ref.[7]  \lq \ldots we conclude that there is no natural large difference in flavor composition between the \ldots QGP and an {\it equilibrium} hadron gas\rq. 
\end{quote} 
The experimentalists working presently in the field investigate the key point which eludes some theorists\footnotemark[8]. The question is not only how much strangeness there is, but {\it what happens to the strange and antistrange quarks}, and how this compares with control data e.g. from $p$--$A$ collisions.

Clearly, the interest to measure strangeness is there  discounting   theoretical controversy, as every experimentalist hopes to see a spectacular phenomenon, a \lq smoking gun\rq\ of the phase transition. Interest in observing strange particles also derives from the second objective of experiments involving relativistic nuclear collisions, the study of equations of state of highly excited nuclear matter. Namely, even without the formation of quark-gluon phase, that is in case that the collision proceeds via the intermediate stage of a fireball consisting only of highly excited hadron gas, the strange particle flow provides essential information about the properties of matter under extreme conditions. However, the relation between observable particle spectra and the equation of state presents many difficulties of detail, and much theoretical modeling will be required; for quark-gluon phase these difficulties are compounded as the observable of a quark-gluon state can be seen only after undergoing a phase transition back into a hadronic form. The phase transition in turn depends on the equations of state. Hence the study of strange particles emanating from collisions at conditions believed not to lead to quark-gluon phase is extremely important as it helps us understand the backgrounds to the quark-gluon phase signatures, at the same time as we learn about {\em confined} nuclear matter.  

A comprehensive survey of the status of the theory of strange particle production and evolution in hadronic collisions before 1985 can be found in Ref.\,[5]. The progress of experiments and theory has been recorded at the Tucson HMIC meeting\footnotemark[5]. An update has been recently prepared by Eggers et al.\footnotemark[9]. \\[0.5cm]
%
%%%%%%%%%%%%%%%%%%%%%%%%%%%%%%%%%%%%%%%%%%%%%%%%%%%%%%%%
\addcontentsline{toc}{subsubsection}{Strange signatures of quark-gluon plasma}
\textbf{\large  Strange Signatures of Quark-Gluon plasma}\\[0.1cm]
%%%%%%%%%%%%%%%%%
Let us consider the situation in some more detail: as is apparent several experimental options for the study of the flavor--strangeness signal of QGP in heavy ion collisions are available. The most obvious measurement is the determination of the multiplicity of various strange hadrons, often represented as ratios to reduce the influence of the experimental bias (trigger). In this class of measurements, however, components originating from all the different production processes are included; for example, strange hadrons may be formed in
\begin{itemize}
\item initial high energy hadronic collisions, 
\item inside the QGP, 
\item during QGP hadronization,
\item in the final expanding hadron gas,
\item rescattering from spectator nuclear matter,
\end{itemize}
or, if the QGP is not formed at all, during the various (equilibrium and non-equilibrium) stages of a hadron gas fireball. This means that the QGP strangeness signal must be evaluated in relation to proton nucleus reactions and detailed conventional wisdom cascade calculations.

Somewhat more specific approach to identify strangeness signal of QGP is to measure strange particle rapidity and transverse energy or momentum spectra. The above mentioned distinct physical processes normally emit particles into different windows of rapidity or transverse energy, making it possible to select particles from a specific process by introducing appropriate cuts in the differential cross section data. Transverse energy spectra are often divided into separate, although overlapping, regions in which a specific physical process dominates\footnotemark[9]. This conjecture is supported by the fact that ratios of different particle species vary strongly with $m_\perp$. At low $m_\perp$, one finds particles formed in the rescattering of the spectator nucleons. At slightly higher $m_\perp$, particles produced in the hadron gas, which decoupled at the freeze-out temperature of the fireball are dominant. Particles emitted with moderately high $m_\perp$ originate from hot and dense form of matter, conceivably the early QGP. A number of mechanisms can be responsible for this sector of the particle abundance. For example in Ref.\,[11] two processes were considered: in the first a quark or diquark from the high-momentum tail of the QGP strikes the phase boundary. It than may create a $q\bar q$ pair e.g. via string-breaking and so a high $m_\perp$ meson or baryon is emitted in such a micro-jet process. Alternatively, a baryon or meson like cluster in the QGP leaves the QGP in unison. In particular it follows from this consideration and the high $\bar s$ density that the differential measurement of multistrange antibaryons should have a good (QGP) signal to (HG) noise ratio. If such multistrange antibaryon yields can be analyzed in terms of their transverse and longitudinal flow, the signature for new phenomena will be clear.  

From this discussion it is clear that the most interesting part of the particle spectrum involves central rapidity, median (e.g. 1-5\,GeV/c) transverse momenta. To sum up the different ways of measuring strangeness, a schematic diagram is shown in Fig.~1.\\

%\begin{figure}
\begin{center}
%\vspace{0.1cm}
%\hspace{-0.8cm}
\setlength{\unitlength}{0.92mm}
\begin{picture}(120,73)(8,2)
\put(38,69){\framebox(60,10){QGP Flavor Signal}}
%
\put(68,67){\line(0,1){2}}
\put(32,55){\line( 3,1){36}}
\put(104,55){\line(-3,1){36}}
\put(32,53){\line(0,1){2}}
\put(104,53){\line(0,1){2}}
%
\put(11,43){\shortstack{$s,\bar s$ abundance\\ descriptive QGP signal}}
\put(82,43){\shortstack{$s,\bar s$ density\\ characteristic QGP signal}}
\put(08,41){\framebox(48,12)}
\put(80,41){\framebox(48,12)}
%
\put(32,39){\line(0,1){2}}
\put(104,39){\line(0,1){2}}
%
\put(14,33){\line( 3,1){18}}
\put(50,33){\line(-3,1){18}}
\put(86,33){\line( 3,1){18}}
\put(122,33){\line(-3,1){18}}
%
\put(14,31){\line(0,1){2}}
\put(50,31){\line(0,1){2}}
\put(86,31){\line(0,1){2}}
\put(122,31){\line(0,1){2}}
%
%\put(00,20){\framebox(28,8){all $y, m_\perp$}}
\put(00,20){\framebox(28,11){all $y, m_\perp$}}
\put(36,20){\framebox(28,11)}
\put(42,21){\shortstack{high $m_\perp$\\ central $y$}}
\put(72,20){\framebox(28,11){all $y, m_\perp$}}
\put(108,20){\framebox(28,11)}
\put(114,21){\shortstack{high $m_\perp$\\ central $y$}}
%
\put(14,16){\line(0,1){4}}
\put(50,16){\line(0,1){4}}
\put(86,16){\line(0,1){4}}
\put(122,16){\line(0,1){4}}
%
\put(00,00){\framebox(28,16)}
\put(36,00){\framebox(28,16)}
\put(72,00){\framebox(28,16)}
\put(108,00){\framebox(28,16)}
%
\put(02,1.5){\shortstack{$sqq$ and $\bar s\bar q\bar q$\\global, central\\
$m_\perp$ spectra}}
%
\put(38,1.5){\shortstack{$ssq$ and $\bar s\bar s\bar q$\\global, central\\
$m_\perp$ spectra}}
\put(75,1.3){\shortstack{multistrange\\correlations\\$K^+K^+$}}
%
\put(112,02){\shortstack{$\bar s\bar s\bar q$ ratio to\\$\bar s\bar q\bar q$ and
$\bar q\bar q\bar q$}}
%
\end{picture}

\vspace{5mm}
%\caption
%\centerline
{\textbf{Fig. 1.} Strange particle quantities for diagnosis of QGP} 
%\label{fig:classif}
%\end{figure}
\end{center}
%
%%%%%%%%%%%%%%%%%%%%%%%%%%%%%%%%%%%%%%%%%%%%%%%%%%%%%%%%%
\subsubsection*{\bf Arguments for strangeness as a QGP observable}
The correlated factors why strange particles possess a priori a distinct diagnostic function of the behavior of highly excited nuclear matter and are well suited as a signal distinguishing quark-gluon phase from the hadron gas are as follows:
\begin{enumerate}
\item near flavor symmetry for antiquarks $\bar s, \bar u, \bar d$ in all conditions (baryon rich and baryon poor), 
\item strongly differing production rates in different phases and strangeness mass thresholds which are of the same magnitude as temperature;
\item extremely high $s\bar s$--quark pair density in the quark-gluon phase.
\item the predicted strange antibaryon abundance is greater than background $p$--$p$ ISR results.
\end{enumerate}
We now discuss in more detail each of these points. 

{\bf 1. Anti-flavor symmetry:} Recent BNL and CERN experiments indicate that up to currently available energies the fireball usually has a sizable baryon number and therefore a relatively large baryo-chemical potential $\mu_\mathrm{B}$. This means that, for quark-gluon phase in chemical equilibrium, the number of light antiquarks is suppressed. Deconfined strange and antistrange quarks, on the other hand, are not affected by $\mu_\mathrm{B}$ and so are suppressed in quark-gluon phase only by their non zero mass. Consequently, but provided that strangeness production has saturated the available phase space, the abundance of antiquarks $\bar u, \bar d, \bar s$ will be nearly equal. In baryon free region, as possibly established at RHIC, this flavor symmetry of hadronic particles is also in part a result of the fragmentation of the numerous gluons.  

{\bf 2. Production rates and thresholds:} Rates for production of $s\bar s$ pairs in the quark-gluon phase were often calculated, the latest reference being\footnotemark[8]. The strangeness production time constant in the quark-gluon phase is of the order of $10^{-23}$~s, while in hadronic gas phase it is 10 to 30 times slower\footnotemark[12] at the same temperature and baryo-chemical potential. This difference is mainly due to the presence of gluons in QGP and different reaction thresholds. The typical time scale for the creation and decay of a fireball can be estimated as the time to traverse, say, a distance of 15~fm i.e. $\simeq 5\times 10^{-23}$~s, and so strangeness in a thermal hadronic gas will not likely reach equilibrium values, contrary to quark-gluon phase expectations. Thus we expect that any kinetic description of strangeness production involving the usual hadronic particles will give a total strange particle yield significantly below the limits obtained from an equilibrium picture of hadronic gas fireballs. The most accessible reaction (if allowed) is usually the creation of a $\Lambda K$ or $K\bar K$ pair and requires at least 700 MeV. In the quark- gluon phase, on the other hand, the threshold is given by the rest mass of the strange-antistrange quark pair, i.e. only $2 m_s \simeq 350$\,MeV. This difference between the two thresholds though insignificant at the initial high energies, is noticeably impacting the time scale of strangeness production in a \lq\lq thermalized \rq\rq\ glob of hadronic matter. It is anticipated that at RHIC temperatures of $250\pm 50$\,MeV will be reached. Here I note the trivial, though important point that in general strangeness production occurs in the numerous rescattering processes, not in the highly energetic initial parton-parton collisions. From this we expect in particular substantial enhancement of strangeness in Nucleus-Nucleus collisions, as compared to scaled p-Nucleus yield, (this subject to the validity of the hypothesis of formation of a hadronic fireball of any\lq texture\rq\ ). I recall here, however, the discussion of Koch and Rafelski\footnotemark[13] concerning the abundance of strangeness in regular hadronic interactions. It was found so close to the expected equilibrium abundance, that it seems as if quark-gluon plasma like phase were formed, permitting to saturate the available strangeness phase space in most hadronic collisions. However, Wr\\rq\ oblewski\footnotemark[14] determined that regular hadronic interactions are about three times less effective in making strange flavor as compared to light flavors. Since QGP based estimates lead me to expect flavor symmetry in QGP, some strangeness enhancement must be expected in comparison to $p$--$A$ scaled result. 

{\bf 3. $s\bar s$--density:} Even at the time of hadronization, $s\bar s$ density (now half as high as at it peak) is about 0.4 strange particle pairs per fm$^3$. As consequence, most of baryons and antibaryons emerging is strange, and non- strange nucleons are expected to be only 20\% of the total baryon--antibaryon abundance\footnotemark[5]. In the hadronic gas phase, by contrast, all antibaryons are suppressed, particularly those with high (anti)strangeness content\footnotemark[5], leading to the expectation that quark-gluon phase be distinguishable from hadronic gas phase by relatively enhanced numbers of anti-strange hadrons$^{1,4}$. This argument, initially developed for baryon rich quark-gluon matter remains valid without change at RHIC energies at central rapidity region, i.e. in the central fireball. As detailed calculations\footnotemark[5]  have shown. there is an abundance anomaly expected for strange antibaryons arising primarily from the enormous strange pair density in thequark-gluon matter.  

{\bf 4. Expected direct reaction\lq background\rq\ } It is helpful to consider the magnitude of backgrounds expected for the multi strange (anti) baryons. The $\bar \Xi / \bar Y$ ratio seen at ISR at $\sqrt{s}=63$\,GeV is only 0.06$\pm$0.02 in the central rapidity region\footnotemark[15]. The expected quark-gluon matter result at RHIC is predicted to be ten times greater\footnotemark[5], or even up to 50 times greater\footnotemark[11], at relatively high $m_\perp$. The parallel ratio $\bar Y/\bar N$ is 0.27$\pm$0.02 as measured in the same experiment at ISR, my expectation is that $\bar Y/\bar N\vert_{plasma} \sim 2 \pm 0.5$. We thus see that both $\bar \Xi/\bar Y$ and the $\bar Y/\bar N$ ratios a interesting, with the former being characteristic of the new form of matter, as it is more difficult to imagine how an enhancement along the theoretical QGP prediction could be made otherwise.  

{\bf In conclusion:} The enormous strangeness pair density to be expected in RHIC--QGP is in my opinion the main experimental objective of flavor based RHIC experiments. This property of the QGP state is particularly interesting, since the primary production mechanism of strangeness is by gluons present in the deconfined phase. Measurement of strangeness density removes interpretational ambiguities, related to our present ignorance of reaction dynamics, in attempting a comparison of the respective {\em total} strangeness content of quark-gluon phase and hadronic gas phase, as enhancement of quark-gluon phase strangeness may be diluted by the geometry of the ensemble of collisions and can be argued away on the basis of the perpetual ignorance of the lifetime of the hypothetical hadronic gas phase fireball. Thus strange particle abundance per se, though perhaps most interesting\lq barometer\rq\ and\lq thermometer\rq\ of the quark-gluon matter phase, is to be employed to study QGP properties only once the high strangeness density has been established.  

%%%%%%%%%%%%%%%%%%%%%%%%%%%%%%%%%%%%%%%%%%%%%%%%%%%%%%%%%%%%%%%
\subsubsection*{\bf Paths to observe multistrange (anti) baryons}
\addcontentsline{toc}{subsubsection}{Paths to observe multistrange (anti) baryons}
%%%%%%%%%%%%%%%%%%%%%%%%%%%%%%%%%%%%%%%%%%%%%%%%%%%%%%%%%%%%%%%
Even though at RHIC the \lq\lq common knowledge\rq\rq\ is that the central rapidity region is baryon free, I will not assume here this prejudice and hence refer to the (strange) anti-baryons, which are characteristic for QGP irrespective of the degree of stopping of the baryon number. However, practically every point discussed applies both to baryon flow in baryon free region, and it is of preference if both strange baryons and antibaryons are measured. I will assume that any detector aiming at the measurement of baryon flow will permit the observation and measurement of the charged decay\lq V\rq\ of the neutral $\bar \Lambda$ particles. The decaying $\bar \Lambda$ particles originate in part in the (rapid) electromagnetic decays of the $\bar \Sigma^0$ particles. All anticascades ultimately become $\bar \Lambda$, while only half of all anti-hyperons $\bar Y$ will be in the $\bar \Lambda$-decay chain, of which 64.2\% are giving they typical\lq V\rq\ decay pattern. Assuming full acceptance for the\lq visual\rq\ detector for all V\rq s, the total sample of all seen V-events is 
\begin{equation}\tag{1}
N_{\bar V}=0.642 \bar Y\left({1 \over 2}+{ {\bar \Xi} \over {\bar Y}}\right)\;,
\end{equation} 
and, should the abundance ratio ${\bar \Xi}/{\bar Y} \sim 1/2$, we see that half of the observed V\rq s would be associated with the primordial $\bar \Xi$ abundance.  

The difficulty is that the observable $\Xi^-, {\bar \Xi^+}$ decay over a significantly shorter path (c$\tau = 4.92$ cm)than $\Lambda$ (c$\tau = 7.89$ cm), making necessary a novel detector directly outside the beam pipe. This poses particular instrumental problems, related both to the interface between the two detectors, but more significantly, to the need for extremely high resolution in view of the enormous multiplicity of charged particles, in which the occasional cascade\lq kink\rq\ has to be searched for. Probably this path to the measurement of multi strange (ant) baryons will be ultimately attempted. However, I would like to draw attention to an alternate approach\footnotemark[16]: in order to find out how many $\bar \Lambda$ descend from the cascade decay all that is needed is the measurement of the longitudinal $\bar \Lambda$ polarization.

There is a significant difference in this polarization of the $\bar \Lambda$ descending from the weak $\bar \Xi$ decays. The weak decay polarizes the $\bar \Lambda$-spin longitudinally, the mean value of its helicity being given by the decay asymmetry parameter $\alpha_\Xi$. In the subsequent weak $\bar \Lambda$ decay this polarization is effectively\lq analyzed\rq. The practical approach is to consider the so-called up-down asymmetry of the $\bar \Lambda$ decay with reference to the plane normal to the $\bar \Lambda$-momentum, i.e., to measure how often in the $\bar \Lambda$ rest frame the antiproton appears `above\rq\ as compared to\lq below\rq\;, with respect to a plane normal to the direction of $\bar \Lambda$-momentum.  

The simple criterion which determines the up-down asymmetry is identified boosting the antiproton momentum to the $\bar \Lambda$ rest frame and considering $S$, the vector product between $\bar \Lambda$-momentum and $\bar p$-momentum. I obtain: 
\begin{equation} \tag{2}
S:= \displaystyle\frac{\vec P_{\bar \Lambda} \cdot \vec P_{\bar p}}{P_{\bar \Lambda}^2} -\displaystyle\frac{E_{\bar p}}{E_{\bar \Lambda}}= 
\begin{cases}  \mbox{positive for up}\cr 
\mbox{negative for down}\cr
\end{cases}. 
\end{equation} 
Here we have, as usual, $\vec P_{\bar \Lambda} = \vec P_p + \vec P_\pi$ for the respective particle momenta and similarly for their energies $E = \sqrt{m_i^2 + P_i^2}$. At this point, I note that the longitudinal polarization considered here is of entirely different origin and nature than the transverse polarization of $\bar \Lambda$ associated with hadronic formation processes of these particles. Multiple scattering in the hadronic gas cannot create longitudinally polarized $\bar \Lambda$ out of primordial transverse polarization. However, the longitudinal polarization will be influenced by spin rotation in a magnetic field.  

This up/down asymmetry is given by\footnotemark[16]: 
\begin{equation} \tag{3}
{{N_u-N_d}\over{N_u+N_d}}= {1 \over 2} \alpha_{\bar \Lambda} \wp_{\bar \Lambda}\;, 
\end{equation} 
where $\wp_{\bar \Lambda}$ is the $\bar \Lambda$ polarization and is equal to the $\alpha_\Xi$ decay parameter. This polarization is analyzed by the $\alpha_{\bar \Lambda}$ decay parameter. The different values of the parameters found in the data tables are: $\alpha_\Lambda = - \alpha_{\bar \Lambda} = 0.642 \pm 0.013$; $\alpha_{{\bar \Xi}^0}=-\alpha_{\Xi^0}= 0.413 \pm 0.022$; and $\alpha_{{\bar \Xi}^+}=-\alpha_{\Xi^-}= 0.455 \pm 0.015$. The total up-down asymmetry of all V-events is 
\begin{equation} \tag{4}
{{N_u-N_d} \over {N_u+N_d}}= {{N_{\bar \Xi}} \over {N_{\bar V}}} {1 \over 2} \alpha_\Lambda \alpha_\Xi\;, 
\end{equation} 
where we have included the relative abundance of all polarized $\bar \Lambda$ to the total abundance of V\rq s: $N_{\bar \Xi} / N_{\bar V} = (2 {\bar \Xi}/{\bar Y})/ (1+2 {\bar \Xi}/{\bar Y}).$ With $\bar \Xi/\bar Y$ in the range 1/2( resp. 1/3) we expect a negative up-down asymmetry of 14\% (resp. 11\%). For the\lq normal\rq\ value $\bar \Xi/\bar Y \sim 0.06$ there is the hardly observable asymmetry of only 1.6\%. Hence observation of the longitudinal polarization is QGP specific!  I further note that $\bar \Omega$ weak decays have a negligible influence over the particle abundances and, in particular, their polarizations, since $\bar \Omega$, $\Omega$ are at least five times less abundant\footnotemark[5] than $\bar \Xi$, $\Xi$   and their decay asymmetry parameter ("polarizer" capability) is 5-20 times weaker (depending on the decay channel). The fact that some $\bar Y$, $\bar \Xi$ are descendants of strong decays of $\bar Y$(1385), $\bar \Xi$(1530), etc. is also of no consequence, as abundances of these particles has been considered part of $\bar Y$ resp. $\bar \Xi$ abundance. 

%%%%%%%%%%%%%%%%%%%%%%%%%%%%%%%%%%%%%%%%%%%%%%%%%%%%%%%%%%%%%%%
\subsubsection*{\bf Gluons in plasma}
The key role played by gluons in making high strangeness density an important observable is self-evident. Not only do gluons produce strangeness flavor dominantly (see below) but more importantly they provide the key distinction between the quark-gluon phase and the hadron gas. The high gluon abundance and density in the plasma impacts the entire history of the plasma state, in particular also the process of hadronization at the end of the quark-gluon phase lifetime, in which appreciable strangeness production occurs again. Indeed, strangeness can be considered a signal for gluons in the quark-gluon phase. We will briefly summarize here the expectations about the gluonic component in the plasma. We note that since gluons do not carry electrical charge, but only the strong charge, they can be observed (indirectly of course) only by suitable measurement of strongly interacting particles.  

Including a first-order perturbative effect\footnotemark[17] the gluon number density can be estimated from the equilibrium density as 
\begin{equation} \tag{5}
\rho_g [\mbox{fm}^{-3}] = 1.04 \left( {T\over 160\mbox{MeV} } \right)^{3} \left( 1- {15\alpha_s\over 4\pi} \right) 
\end{equation} 
giving for a typical temperature of 200 MeV a value of 0.55 fm$^{-3}$ for $\alpha_s = 0.6$ and 0.8 fm$^{-3}$ for $\alpha_s = 0.5$. For a quark-gluon phase volume with radius 4--5 fm, we therefore have 200--300 gluons. Note also that this density rises as the cube of the temperature. Because gluons can be created and annihilated easily in interactions with other gluons and light quarks, the gluon density closely follows the evolution of temperature in the course of the quark-gluon phase lifetime. The equilibrium gluon energy density is 
\begin{equation}\label{fg} \tag{6}
\varepsilon_g = {8\pi^2\over 15} T^4 \left( 1 - {15\alpha_s\over 4\pi} \right) \;,
\end{equation} 
and the gluon partial pressure is 
\begin{equation}\label{fh} \tag{7}
P_g (\mathrm{ GeV\; fm}^{-3} )  
 = {{1\over 3}} \varepsilon_g 
 =  0.15 \left( {T\over 160 \mathrm{MeV}} \right)^{4} \left( 1 - {15\alpha_s\over 4\pi} \right) 
\end{equation} 
which for $T = 200$\,MeV and $\alpha_s = 0.6$ yields 100MeV,$\,\mbox{fm}^{-3}$  and forms the major component of the quark-gluon phase pressure. (The total quark-gluon phase pressure must, of course, be larger than both the vacuum pressure ${\cal B}^{1/4}$ and the pressure of the hadron gas surrounding it.) 

Gluons also play a major role in the dynamics of the quark-gluon phase-hadronic gas phase transition: they carry much of the quark-gluon phase entropy, contributing an entropy density of about 
\begin{align*}\label{eq:fi} 
\tag{8}\sigma_g \;\mbox{fm}^{-3} 
&=  {32\pi^2\over 45} T^4
   \left( 1 - {15\alpha_s\over 4\pi} \right)
\ms{5}
\\
&=  3.76 \left( {T\over 160 \mbox{MeV}} \right)^{3} 
    \left( 1 -{15\alpha_s\over 4\pi} \right)
\notag
\end{align*}
which for $T = 200$\,MeV and $\alpha_s = 0.6$ is 2 units per $\;\mbox{fm}^{-3}$ (3.6 units per gluon). This large amount of entropy plays a major role in the hadronization phase transition, forcing gluons to fragment into quarks.

%%%%%%%%%%%%%%%%%%%%%%%%%%%%%%%%%%%%%%%%%%%%%%%%%%%%%%%%%%%%%%%%
\subsection*{Strangeness production in the quark-gluon phase}

Since the time scale in a typical nucleus-nucleus collision is very short, the strangeness content of both quark-gluon phase and hadronic gas phase cannot {\em a priori} be assumed to be in equilibrium: it is necessary to determine explicitly the rate of strangeness production in both phases. The key result was obtained in the work of Rafelski and M\"uller\footnotemark[3]. The plasma initially contains very few, if any, strange quarks as those produced in pre-quark-gluon phase direct hadron-hadron reactions will generally be at higher rapidity than the fireball. Essentially all the $s\bar s$ production is therefore dominated by collisions of the central gluons, which in a first approximation can be assumed to be in a practically thermal distribution; light quark-antiquark collisions, it turns out, play only a minor role. Therefore the time evolution of strangeness density during the production process is only a function of temperature and not of the baryo-chemical potential. I will give here a brief sketch of the theory of strangeness production and show how strangeness density grows with time.  


%%%%%%%%%%%%%%%%%%%%%%%%%%%%%%%%%%%%%%%%%%%%%%%%%%%%
%\begin{figure}[t]\sidecaption[t]
%\centerline{\includegraphics[width=0.7\columnwidth]{./AllFigs/90Fig2.png}}
%\caption 
%\noindent\small{Fig. 2 $t$-averaged strangeness production cross sections for $\alpha_s = 0.6$, $m = 170$\,MeV}
%\label{fig:crost}
%\end{figure}

%The $t$-averaged cross sections for strangeness production are shown  in Fig.~2. At this point it seems that 
\textit{(Fig 2. presented earlier is omitted here)} \ldots both glue and quark induced processes are of comparable magnitude\footnotemark[9]. However, as we will just see the statistical factors entering the thermal average will strongly favor the gluon induced processes: there are simply more glue-glue than quark-antiquark collisions of suitable quantum number in plasma. In order to identify the energy range contributing to the production of strangeness, it is useful to write the production rate as an integral over the differential rate\footnotemark[9] $dA/ds$ 
\begin{equation}\label{gf}\tag{9}
A_i = \int_{4m^2}^\infty\,ds\,(dA_i/ds) =
\int_{4m^2}^\infty\,ds\,\bar\sigma_i(s)\, P_i(s)\;, \qquad i = g,q\;.
\end{equation}
The weight function $P_g(s)ds$ is the number of (gluon) collisions within the interval ($s,s+\,ds$) per unit time per unit volume, with a similar interpretation for $P_q(s)$. In a thermal system
\begin{equation}\label{gma}\tag{10}
P_g(s) = \int\,{d^3p_a\over (2\pi)^3E_a}\,
{d^3p_b\over (2\pi)^3E_b}\, {s\over 2} \,
\delta[s-(p_a+p_b)^2]\,{1\over 2}g_g^2 f_g(p_a)f_g(p_b)\;.
\end{equation}
In principle, non-equilibrium momentum distribution functions should be used for $f_g$, presumably evolving from the structure functions of the incoming reacting hadrons towards their equilibrium forms. However, because of the high gluon-gluon cross sections, this should happen very quickly\footnotemark[18,19]. In first approximation, one can therefore use the (thermal and chemical) Fermi and Bose equilibrium distributions. In Fig.~3, the product of the weight functions $P_g(s)$ and $P_q(s)$ with the respective cross sections is plotted for $T= 250$\,MeV and $m = 170$\,MeV. In one case, $\alpha_s = 0.6$, in another, the running coupling constant was used with $\Lambda = 200$\,MeV. Note that most $s\bar s$ pairs are made at $\sqrt{s} \simeq 0.5 \mathrm{GeV}$, giving at least some credence to the use of perturbative QCD, and in particular the value $\alpha_s = 0.6$ selected.


%%%%%%%%%%%%%%%%%%%%%%
\centerline{
\includegraphics[width=0.7\columnwidth]{./AllFigs/90Fig3.png}
} 
\noindent{\small Fig. 3 Differential production rate $dA/ds = P(s)\bar\sigma(s)$, with $T$ =250 MeV and $m$ = 170 MeV, for gluons and $q\bar q$ pairs, with $\mu_\mathrm{B}$ = 400 MeV. Solid lines are for running $\alpha_s$ with $\Lambda = 200$\,MeV, dotted lines for $\alpha_s = 0.6$}
%\label{fig:dads}
%\end{figure}

In Fig.~4, the time evolution of the density of strange quarks in quark-gluon phase is shown ($\alpha_s = 0.6, m_s = 170$\,MeV). As expected, there is a strong threshold effect at temperatures around 150 MeV. A similar calculation which included an expansion model of the fireball\footnotemark[5] showed that the strong dependence of $s\bar s$ production on the temperature also implies that the strangeness abundance freezes out with a value characteristic of the highest temperatures reached during the collision. No significant strangeness annihilation occurs during the fireball expansion. 
 


%%%%%%%%%%%%%%%%%%%%%%%%%%%%%%%%%%%%%%%%%%%%%%%%%%%%%
\centerline{
\includegraphics[width=0.7\columnwidth]{./AllFigs/90Fig4.png}
}
\noindent{\small Fig. 4 Time evolution of the strange quark density in quark-gluon phase for different values of the temperature. Dashed lines: no $s\bar s$ annihilation.}
 

%%%%%%%%%%%%%%%%%%%%%%%%%%%%%%%%%%%%%%%%%%%%%%%%%%%%%%%%%%%%
\section*{Lessons From Present Experimental Results on Strangeness}
\addcontentsline{toc}{subsubsection}{Lessons from early experimental results}
%%%%%%%%%%%%%%%%%%%%%%%%%%%%%%%%%%%%%%%%%%%%%%%%%%%%%%%%%%%%
I will focus here on the aspects of current experimental work instructive to the described developments, giving only a schematic interpretation. The experimental method employed to determine the enhancement is to compare the yield of strange particles as a function of the inelasticity of the interaction. In order to demonstrate the kind of analysis we will have to implement for RHIC experiments let me now consider a hypothetical quark-gluon phase fireballs as being at the origin of the latest results in strangeness production. I consider the {\it experimental} situation as it presents itself in May 1990, following on the Quark Matter \rq\ 90 meeting. In all BNL and CERN experiments reported so far strangeness enhancement by a factor $2\pm 0.5$ has indeed been seen, but can not be taken without prejudice to be a signal of quark-gluon plasma.  

In this discussion I will use experimental results to estimate the value of the temperature and chemical potential at which the strange particles are likely to have been born and will try to determine if there is any glaring inconsistency of the present data with such a hypothesis. Alas, as we will see, total strangeness data of from BNL can not point to a particular phase of matter, much as expected. Nevertheless, in order to test the consistency, rather than two parameters $(T,\mu_\mathrm{B})$ we must consider three quantities characterizing the average thermodynamical properties of the fireball, e.g.:
\begin{enumerate}
\item The temperature $T$, as obtained directly from particle transverse energy spectra. Here we must take care to distinguish the projectile and target rapidity regions from the central region which of greatest interest to us here. For most particles with large cross sections such as pions, the observed slopes of transverse mass particle spectra provide us with the temperature $T^{H}$ at freeze-out of the particular particle species in the hadron gas. However, strange particles can exhibit higher temperatures as their interaction length is larger. For the thermal picture to be applicable, a similar temperature should be found in the corresponding rapidity spectrum.
%
\item As a direct measure of the baryo-chemical potential, we can consider the entropy per baryon $S/B$, which we assume here to be mostly produced during the initial stages of the nuclear collision. During the subsequent, in particular hydrodynamical evolution of the quark-gluon phase or hadronic gas phase, no entropy is produced and the baryon number is also constant. For the case of the perturbative QCD equation of state, constant-$S/B$ implies $T \propto \mu$ with a determined constant\footnotemark[20,21]. The value of entropy per baryon reached in the reaction is obtained under the assumption that a gas of nucleons and pions dominates all central fireball secondaries (counting all mesons as pions, all baryons as nucleons)\footnotemark[22]:
\begin{equation}\label{el}\tag{11}
{S\over B} \simeq {S_N\over B} + {S_\pi \over \langle{n_{\pi}}\rangle}\;{\langle{n_{\pi}}\rangle \over B}\;,
\end{equation}
where the entropy per pion is about 4.05 and the entropy per nucleon outside of the degeneracy region is $S/B = 2.5 + (m_N-\mu_\mathrm{B})/T$.
%
\item The baryo-chemical potential $\mu_\mathrm{B}^H$ in the (final stage) hadron gas phase can be determined conveniently from the $K^+/K^-$ or $K_s/\Lambda$ ratios, which are sensitive to $\mu_\mathrm{B}^H$\footnotemark[23], because of large strangeness exchange cross sections which rapidly establish the so-called relative chemical equilibrium between different species of strange particles. This is true even if absolute chemical equilibrium is not attained for strangeness in the hadronic gas phase\footnotemark[5]. If the values of $T$ and $\mu_\mathrm{B}^H$ do not disagree to much with the entropy based QGP constraint (see above), this can be taken as a first indication that we are possibly close to the quark-gluon phase. 
\end{enumerate}

%%%%%%%%%%%%%%%%%%%%%%%%%%%%%%%%%%%%%%%%%%%%%%%%%%%%%%%
\subsection*{BNL--RHI results}
%\label{ssec:BNL}
There are two experiments at BNL measuring strange particle spectra, of which the more ambitious TPC-based E810 has just begun to collect data\footnotemark[24], while the magnetic spectrometer experiment E802 has essentially completed its data taking\footnotemark[25]. Both experiments see an appreciable strangeness signal in 14.6 A\,GeV/c Si--Au collisions (the beam rapidity is 3.44), with a central collision trigger. The common result of both experiments is that strange particles have a rather \lq\lq thermal\rq\rq\ shape in the central rapidity region, and that the temperature is in the vicinity of 150 MeV, but with a statistical error which is presently 15 MeV. While E810 expects to measure the abundances of diverse multi strange baryons and antibaryons in the near future, experiment E802 provides already today precise data on ratio of meson abundances\footnotemark[25]. Additional data including in particular the antiproton spectra has recently been presented at the BNL-HIPAGS workshop\footnotemark[26]. It therefore seems justified to assess the results of E802 with a simple fireball model in mind. We will need just the most naive of the pictures for further discussion: the tube model of the nucleus-nucleus collision leads to the formula for the number of participating target nucleons $A_\mathrm{t} = 1.5 A_{\rm projectile}^{2/3}A_{\rm target}^{1/3}$ predicts $A_\mathrm{t} \simeq 80$ for the Si--Au case and hence a total baryon content of the fireball $B \simeq 108$. This corresponds to a theoretical rapidity $y_{lab} \simeq 1.23$ for a fireball made out of $(A_p + A_\mathrm{t})$ nucleons, closely corresponding to the experimentally inferred central rapidity $y = 1.2$. This assumes complete stopping so that the accessible CM energy $\sqrt{s} = 261$\,GeV is mostly transferred into the internal excitation of the fireball,suggesting an energy content of 2.42 Gev per baryon, less energy in excitation of spectator matter. 

From rapidity particle densities we can now derive the central pion to baryon ratio: for $1.1 < y < 1.6$, the proton rapidity density $dN/dy$ is $16.2 \pm 0.3$, implying a baryon rapidity density $38 \pm 0.7$ (given the baryon to proton ratio of 2.35 in the tube model for Si-Au collisions). Both the $\pi^+$ and $\pi^-$ rapidity density is quoted at $16 \pm 1$. Allowing for an equal number of neutral pions the pion central rapidity density is $48 \pm 1.8$. This results in a pion to baryon ratio $1.25 \pm 0.05.$ For $T \simeq 125$\,MeV (observed pion temperature), the pion gas entropy per pion is $\simeq 4.3$, and hence the pion entropy per baryon is 5.4 units of entropy. This implies that we are still in a rather degenerate nuclear gas phase, and hence the entropy contribution of baryon gas is, relatively speaking, small. For $\mu_\mathrm{B} = 840, T = 125$\,MeV, the baryon gas contributes about 3.7 units of entropy, while at $\mu_\mathrm{B} = 500, T= 150$\,MeV we have 5.4 units additional entropy. Hence we are at $S/B \simeq 8 - 10 $, the lower value for the higher range of baryo-chemical potential.  

The observed relative abundance $K^+/\pi^+ = 0.203 \pm 0.019$ is obtained by ignoring the possible distortions of the low energy spectra due to\lq low energy\rq\ phenomena; both $K$ and $\pi$ spectra are extrapolated assuming the Boltzmann form controlled by the fireball properties. Similarly, the K-ratio $K^-/K^+ = 0.19 \pm 0.03 $ is found - with same limitations as described above. The question now is if these results on particle ratios, temperature, and other inferred fireball properties are consistent with the assumption of a particular phase of hadronic matter and the above constraints. Before beginning this discussion we note that the first of the particle ratios is indeed a lower limit, in the sense that whatever the reaction mechanism, we do not expect to saturate the strangeness phase space fully, and hence the preliminary {\it equilibrium} picture we develop should predict a larger value than is actually observed.  

Let us first make the hypothesis that hadronic gas was made\footnotemark[27]. I fix the $K^-/K^+$-ratio at 0.2. There are two options: 
\begin{itemize} 
\item I take T = 125 MeV as the freeze out temperature. I infer following Koch et. al\footnotemark[23] a value of baryo-chemical potential of 520 MeV; the expected $K^+/\pi^+$ ratio is about 0.26, allowing for pions from $\Delta$-decays, and assuming that the strangeness phase space has been saturated. 
\item Taking instead as basis the strange particle temperature T= 150 MeV (under the tacit assumption that pion spectra are distorted by $\Delta$ decays and rescattering on spectator matter) the K - ratio implies a slightly lower baryo-chemical potential of just below 500 MeV and the $K^+/\pi^+$ ratio is slightly higher at 0.334. 
\end{itemize}
Thus with the proviso that \lq only\rq\ 80, resp. 60\% of the strangeness phase space is saturated both temperature hadronic gas scenario seems fully consistent with the data, with the exception that we do not understand how so much strangeness could be made by hadronic gas processes. At this point I note that this discussion disagrees in its detail with Ref.\footnotemark[8], which assumes fully saturated strangeness phase space. Therefore a rather low temperature is found, incompatible with the transverse spectra, or said differently (allowing for flow effects), with the mean energy per particle. Interestingly, the difference between our (and L\'evai\rq s\footnotemark[27]) analysis and Ref.[8] is the predicted d/p ratio which is highly sensitive to the entropy per baryon: assuming to much strangeness, additional pions are needed in order to `dilute\rq\ the strange particle abundance, an effect which I estimate at about 3 units of entropy per baryon. By implication the expected value for Ref.[8] of the $d/p = 0.05$, our discussion suggests 2--3 times larger value.  

For both above considered choices $T,\mu_\mathrm{B}$, the energy per baryon, which in this region of parameters is mainly controlled by the $K^-/K^+$-ratio, turns out to be below 1.9\,GeV. To get a slightly higher value, as it may seem required within the simple fireball model presented above, we should have set the Kaon ratio to a larger value, allowing for an unseen low energy fraction of $K^-$. Taking a value 0.25 at $T = 150$\,MeV would lead to energy per baryon somewhat above 2\,GeV and at the same time a $\pi^+/p$ ratio near 1.3, in agreement with the value reported at central rapidity. 

Next, let us see how the data fare under the assumption of quark-gluon plasma phase. Naturally, the advantage of this assumption is that we have little difficulty swallowing the saturation of strangeness phase space, thanks to the described rapid strangeness production. Furthermore it turns out that in region of $(\mu_\mathrm{B},T) = (850,130)$ MeV there would be a similar amount of entropy in the quark-gluon phase, as in the hadronic gas phase at $(\mu_\mathrm{B},T) = (500,150)$\,MeV. The supposition is that in the phase transition of the isolated glob some reheating from about 130 to 150 MeV takes place, and there is corresponding reduction of the chemical potential. As any pre--transition emission from the plasma would in such environment be covered by the soft component of the hadronic gas phase, we should not expect any visible quark matter effects in kaon spectra. Thus solely from the observation of singly strange particles we can not make a definitive statement about the presence of QGP in nuclear collisions at BNL. However, it is interesting to note that the BNL conditions are near to the baryon-rich quark-gluon phase domain. This conjecture is supported by the recent finding of antiproton multiplicity\footnotemark[25], which in the central rapidity region is less than one part in thousand of the proton multiplicity.  

But presently the only argument one could make in favor of QGP at Brookhaven is that the values of the parameters estimated above imply that even at BNL energies strangeness production in the quark-gluon phase will be rapid and will nearly saturate the available phase space. It is therefore most interesting to look at BNL for strange antibaryons, which without quark-gluon phase formation should hardly be produced at these energies. Given the suppression of antiprotons, which is expected for a baryon-rich fireball consisting of either hadronic gas phase or quark-gluon phase, observation of a {\em greater} strange antibaryon yields would strongly suggest that already at BNL energies this state of matter may be formed. It is to be hoped that the results from experiment E810 will allow us to conclude this issue.  

%%%%%%%%%%%%%%%%%%%%%%%%%%%%%%%%%%%%%%%%%%%%%%%%%%%%%%%%%
\subsection*{CERN--RHI results}

At CERN the available energy is much greater and ranges from 60 up to 200 GeV per nucleon. However, the laboratory has not taken full advantage of the available machine resources as yet, by limiting its main experimental runs to the highest available energy. In the asymmetric reactions such as the S--W collisions studied by\footnotemark[28,29,30]  there is the advantage over the S--S collisions studied by\footnotemark[31,32] NA35 of the much greater baryon number stopping. But there are difficulties in interpreting the data, which are associated with overlap of the different kinematic regions (target, central and projectile). In this regard, one has here in principle less of a problem than at BNL since the rapidity window is almost twice as large as at BNL: the projectile rapidity at 200 A\,GeV/c is 6, compared to 3.4 for BNL. The particular advantage of the S--S data is the symmetry of the kinematics, permitting a much better understanding of particle flows. The disadvantage is the likely presence of significant transparency at 200\,GeV per nucleon. However, the central rapidity region is 3 (for symmetric collisions), making a particle in the laboratory very fast. Consequently an experiment similar to E802 is impossible, as the time of flight does not permit particle identification. Thus the small aperture spectrometer experiment at CERN, NA34, is concentrating on the target fragmentation region. In view of the currently available results and this discussion it would seem that it would be of considerable advantage to study the symmetric S--S collisions at lowest available CERN energy, viz. 60 A\,GeV, in expectation of the lead beam run initially at a similar energy.  

Points of importance to our work in the most recent results of NA35 are:
\begin{itemize}
\item The $\Lambda$--$\bar\Lambda$ rapidity distribution, which shows two
pronounced peaks within the projectile and target rapidity regions, an
indication of a severe depletion in the central region. This shows that
much of the $\Lambda$ signal derives from re-scattering in the baryon rich
projectile and target fragmentation region.
%
\item The ${\bar\Lambda}$ multiplicity is sharply confined to the central
region $y = 3 \pm 0.5$. The rate of ${\bar\Lambda}$ production in S--S
collisions is about 120 times greater than in $p$--$p$ collisions (the error
quoted is large). The per trigger event multiplicity of ${\bar\Lambda}$ is
given to be 1.5! This (120-fold) enhancement has to be confronted with the
36-fold enhancement of the negatively charged tracks (i.e. pions). This
truly surprising result cannot even remotely be explained by cascading in
hadronic gas, as the probability of ${\bar\Lambda}$ formation is
decreasing during the moderation of the beam energy.
%
\item The general strangeness flavor production is up by a factor 2.5 on
top of the factor 36 for negatives: the $K/\pi$-ratio at mid-rapidity
$y=3$ is 0.15, to be compared to 0.06 for similar energy $p$--$p$ system. 
\end{itemize}

All these results remind us of the quark-gluon phase. Unfortunately, we do not have comparable data on production of $\bar p$ or ${\bar\Xi}$ and thus cannot conclude that the expected systematic signal of quark-gluon phase has been found. The lack of data on the essential $\bar p$ and ${\bar \Xi}$ production is being filled by the large aperture $\Omega\rq\ $-spectrometer WA85 experiment, which has presented the first results from the study of S--W collisions at 200 A GeV. Because of complex Monte Carlo studies required to understand the relative sensitivity of the experiment to ${\bar\Lambda}$ and ${\bar\Xi}$, this ratio is not known as yet, though WA85 has already reported first observation of $\bar \Xi$. The following has now been reported\footnotemark[28,29,30] by WA85 
\begin{itemize}
%
\item The temperature (inverse slope) of negatives, $\Lambda$ and ${\bar\Lambda}$ is the same and is 227 MeV, i.e.~higher than the temperature seen in S--S collisions. Because of the greater stopping expected, this result can be taken as a confirmation that the highest energy and baryon densities were reached in this experiment. Unfortunately, we cannot determine the baryo- chemical potential for this experiment as yet, nor can we determine the entropy per baryon. To this end we would need data on kaon $(K_s)$, pion (negatives) and also positive particle (protons and positive kaons) spectra in relation to the strange baryons and antibaryons.
%
\item The yield of both $\Lambda$ and ${\bar\Lambda}$ per negative track in the central rapidity region $2.4 < y < 2.65 $ is enhanced by a factor 1.7 in comparison to the control $p$--W data. Both enhancements are similar and the ratio of $\Lambda$ to ${\bar\Lambda} \simeq 0.2$ does not change. 
%
\item There seems to be an enhancement in the anticascade to cascade ratio in S--W collisions $(\sim 0.43 \pm 0.07$) as compared to the control $p$--W run $(\sim 0.27 \pm 0.06)$. Clearly, more statistics are needed to reconfirm this result. Also, it is important to know by how much the $\bar\Xi/\bar\Lambda$ ratio is enhanced in S--W reactions with reference to $p$--W reactions.
\end{itemize}

In CERN data, we hence once again see a clear strangeness enhancement, accompanied now by a highly significant enhancement of strange antibaryon yield. We cannot imagine how to interpret this data other than in terms of quark-gluon plasma. However, the data are still fragile at the level of only a few standard deviations, and require some improvement in the statistics. Also, we need a more complete evaluation of all available data in order to be able to give more detailed characterization of the conditions reached in the S--W and S--S collisions.  


%%%%%%%%%%%%%%%%%%%%%%%%%%%%%%%%%%%%%%%%%%%%%%%%%%%%%%%
\section*{Concluding Comments}

Without a substantial interaction between experiment and theory, the most spectacular measurements remain, especially in this subject matter without much concrete insight. The situation is further complicated by numerous superficial if not wrong publications (as exemplified above) relating to the subject, as well as the process of \lq\lq reinventing the wheel\rq\rq, which so often leads not only to the repetition of the old mistakes. The particular reason why flavor flow experiments are very attractive in the beginning of any nuclear collider operation is the fact that the high expected strangeness production allows event by event analysis. Even if event rate should initially be small, strangeness will be clearly visible. The experiments suggested here are based on the following key observations:
\begin{enumerate}
%
\item At sufficiently high energy densities, heavy ion collisions may lead to formation of a deconfined phase of strongly interacting nuclear matter, the quark-gluon phase, in which flavor symmetry is partially restored and strangeness becomes abundant. The full event characterization is needed to fix the thermodynamic variables of essence for the basic understanding of reaction kinematics needed in understanding (strange) particle flows.
%
\item Compared to a hadron gas, in quark-gluon phase strangeness is produced faster and strangeness density is greatly higher. Also, strangeness is produced in quark-gluon phase almost totally by glue-glue processes. Uncertainty about the hadronization process makes global strangeness measurements less attractive as a signal of quark-gluon phase than observation of specific (multi)strange particles.
%
\item Anomalous (large) strange and multistrange antibaryon multiplicities can be viewed as the clearest signal that something unusual is happening in central collisions, particularly when viewed in specific windows of $(p_\perp,y)$. 
%
\item Multistrange antibaryons can provide crucial information as they are predominantly formed in phase space regions characterized by a very high strangeness density. 
%
\item As the theory of strangeness production and hadronization relies on key parameters of QCD, these will become accessible to measurement in heavy-ion collision induced reactions, through the measurement of diverse flavor and particle flows and detailed comparison of experiment with theory.
%
\end{enumerate}
{\it Acknowledgement} I would like to thank G. Odyniec, S. Lindenbaum and L. Madanski for their interest in this work and pertinent comments about important experimental aspects. I thank T. Ludlam for his kind hospitality at BNL.
\footnotetext{\vspace*{-0.5cm}\begin{enumerate}
\item J.~Rafelski and R.~Hagedorn, in:  \textit{Statistical Mechanics of
Quarks and Hadrons,} ed. H.~Satz, ed. North Holland, Amsterdam (1981)

\item J.~Rafelski, in:  \textit{Workshop on Future Relativistic Heavy Ion
Experiments,} eds. R. Bock and  R. Stock, GSI 81-6, Darmstadt 1981

\item J.~Rafelski and B.~M\" uller, Phys.~Revs.~Lett. {\bf 48}  1066 (1982); ibid. {\bf 56}, 2334(E) (1986) 

\item J.~Rafelski, ``Strangeness in Quark-Gluon Plasma'',
S. ~African~J.~Phys. {\bf 6} 37, (1983)  

\item P.~Koch, B.~M\" uller and J.~Rafelski, Phys.~Rep. {\bf 142} 167, (1986)
 
\item  P.~Carruthers and J.~Rafelski,  eds. \textit{Hadronic Matter in Collision 1988,} proceedings of a meeting held in Tucson, Arizona, October 6-12, 1988; (World Scientific, 1989)

\item  J.~Cleymans, ``Strangeness Production in Relativistic Ion
Collisions - Theoretical Review'', UCT-TP 142/90, to appear in proceedings
of QM'90 meeting, Menton, France, May 1990.

\item J.~Cleymans, H.~Satz, E.~Suhonen and D.W.~von Oertzen,
Phys.\ Lett.\ B {\bf 242} 111 , (1990)

\item  H.C.~Eggers and J.~Rafelski, ``Strangeness and Quark-Gluon
Plasma: Aspects of Theory and Experiments'', Preprint AZPH-TH/90-28,
submitted to J. Mod. Phys.  A (published: \textbf{6}  1067 (1991))

\item  E. Shuryak, in this volume

\item  J.~Rafelski and M.~Danos, Phys.\ Lett.\ B {\bf 192} 432  (1987)

\item  P.~Koch and J.~Rafelski, Nucl.\ Phys.\ A {\bf 444} 678  (1985) 

\item  P.~Koch and J.~Rafelski, S. ~African~J.~Phys. {\bf 9} 8  (1986)

\item  A.~Wr\'oblewski, Act.\ Phys.\ Pol. B {\bf 16} 379  (1985)

\item  T.~Akesson et al. [ISR-Axial Field Spectrometer Collaboration],
Nucl.\ Phys.\ B {\bf 246} 1, (1984)

\item  M.~Jacob and J.~Rafelski Phys.\ Lett.\ B {\bf 190} 173  (1987)

\item S.A.~Chin, Phys.\ Lett.\ B {\bf 78} 552  (1978)

\item S.~Mr\'owczy\'nski and J.~Rafelski, Phys.\ Rev.\ C {\bf 40} 1077  (1989)

\item  B. M\"uller, in this volume

\item J.~Rafelski and A.L. Schnabel, in: Conference on Intersections
between Particle and Nuclear Physics, Rockport 1988, AIP\# 176 

\item J.~Rafelski and A.L. Schnabel, Phys.\ Lett.\ B {\bf 207} 6, (1988)

\item N.K.~Glendenning and J.~Rafelski, Phys.\ Rev.\ C {\bf 31} 823, (1985)

\item P.~Koch, J.~Rafelski and W.~Greiner, Phys.\ Lett.\ B {\bf 123} 151  (1983)

\item  S.E. Eiseman et al. [E810 collaboration], ``Neutral $V$
production with 14.6 $x$ A GeV/c Silicon Beams'', BNL-44716, submitted to
Phys. Lett. B

\item  T.~Abbott et al. [E802 collaboration], Phys.~Revs.~Lett. {\bf 54} 847  (1990)
 
\item  J.B.~Costales [E802 Collaboration], HIPAGS Workshop,
Brookhaven, March 1990

\item  P.~L\'evai, B.~Luk\'acs and J.~Zim\'anyi, J.~Phys.~G {\bf 16} , 1019  (1990)

\item  N.J.~Narjoux et al. [WA85 Collaboration], Lecture at Quark
Matter '90, Menton, France, May 7--11, 1990 

\item D.~Evans et al. [WA85 Collaboration], Lecture at Quark Matter '90, Menton, France, May 7--11, 1990 

\item E.~Quercigh, in: ``Hadronic Matter in Collision 1988'', World Scientific, 1989, eds. P.~Carruthers and J.~Rafelski
 
\item R.~Stock et al. [NA35 Collaboration], Lecture at Quark Matter
'90, Menton, France, May 7--11, 1990

\item  H.~Str\"obele et al. [NA35 Collaboration], Lecture at Quark Matter '90, Menton, France, May 7--11, 1990

\end{enumerate}
}
 
\end{mdframed}


%%%%%%%%%%%%%%%%%%%%%%%%%%%%%%%%%%%%%%%%%%%%%%%%%%%%%%%%%%%%%%
\subsection{Strangeness production with running QCD parameters}\label{QCDrunning}

Many things were happening in 1995, advancing the strangness signature of QGP:
\begin{enumerate}
\item
In January 1995 a first Strangeness in Quark Matter meeting took place in Tucson, and the proceedings were published rapidly~\cite{Rafelski:1995zq}. This meeting grew into a conference series which gathers today several hundred participants, see listing at \url{https://sqm2019.ba.infn.it/index.php/previous-editions/}. As remarked maybe QGP discovery should have been announced at this venue, see page \pageref{SQM95an}.
\item
The Hagedorn conference of Summer 1994  was readied for publication~\cite{Letessier:1995ic}.
\item 
I completed a review addressing the theoretical developments advancing QGP formation in heavy ion collisionsj and strangness~\cite{Rafelski:1996hf}, including many SHM fit results.
\item
Refinement of the production rates of strangness was achieved considering the QCD-running of the strange quark mass and coupling constant, and allowed more detailed comparison with experiment~\cite{Letessier:1996ad}.  
\end{enumerate}
 

\noindent\textit{The March 1996 research progress report to the  US-DoE--Office of Science includes:}\\[-0.7cm]
%
\begin{mdframed}[linecolor=gray,roundcorner=12pt,backgroundcolor=Dandelion!15,linewidth=1pt,leftmargin=0cm,rightmargin=0cm,topline=true,bottomline=true,skipabove=12pt]\relax%
%
%%%%%%%%%%%%%%%%%%%%%%%%%%%%%%%%%%%%%%%%%%%%%%%%%%%%%%%%
{\large \textbf{From:} March 1996 progress report prepared for:}\\ 
\indent\indent\indent  The U.S.-D. of E.--Office of Science\\
\textbf{Abstract:} \ldots we were primarily  engaged in an effort to substantiate our suggestion concerning the  formation of deconfined and nearly statistically equilibrated QGP phase in 160-200 A GeV  ($\sqrt{s_{\rm NN}}\simeq 8.6 + 8.6$ GeV) interactions. We continued the exploration of the energy dependence of the observables, refining the understanding of the nonequilibrium parameters and hadronization models, developing strangeness production descriptions free of ad hoc parameters and assumptions, and comprising more adequate  description of the fireball development. 
% 
%%%%%%%%%%%%%%%%%%%%%%%%%%%%%%%%%%%%%%%%%%%%%%%%%%%%%%%%
\section*{Strangeness in Dense Hadronic Matter} 
%\addcontentsline{toc}{subsubsection}{Strangeness in dense hadronic matter} 
%%%%%%%%%%%%%%%%%%%%%%%%%%%%%%%%%%%%%%%%%%%%%%%%%%%%%%%%
Strange particle production is  recognized as one of the interesting hadronic observables$^1$ of dense, strongly interacting matter and much of the current theoretical and experimental effort in study of relativistic nuclear collisions is devoted to  this topic$^2$. Our great interest in the subject arises from the realization that the experimentally observed anomalous production of (strange) antibaryons$^{3,4,5.6}$ cannot be interpreted without introduction of some new physical phenomena. 

Quantum-Chromodynamics (QCD) is accepted as the theoretical foundation of strong interactions and we can expect that large regions of strongly interacting highly excited matter would obey the laws of perturbative thermal QCD, as is seen in lattice gauge calculations$^7$. There is little doubt about the existence of the deconfined phase (QGP) at high temperature, say  $T=1$ GeV in which physical processes are governed by perturbatively interacting quarks and gluons. The practical issue is, how extreme are the conditions required to form this new phase of matter? There are also quite intricate issues related to the short lifespan of the relativistic nuclear collision, which put in question the use of statistical physics methods, and suggest that elements of relativistic (quantum) transport theory must also be incorporated into the description of the physical phenomena occurring. For a recent survey of the physics of quark-gluon plasma we refer to review of Harris and M\"uller$^{8}$  and the many references cited there.
 
While near to the phase boundary of QGP with the confined hadronic gas (HG) phase quite complex phenomena may occur, involving in principle still other forms of hadronic matter, in a first approximation our effort  concentrates  on finding a suitable  extrapolation of the properties of perturbative QGP phase to this domain in order to be able to understand hadronic particle spectra  and abundances emerging in relativistic heavy ion collisions. From such a full description than emerges a diagnostic element of our work, as we seek to  correlate the properties of the source with  anomalies of particle abundances. We have proposed long ago$^{9}$ that strange antibaryons are the best tools in such a study of the QGP phase, and we are finding today$^{10,11,12}$ that the experimental results we mentioned above can be  interpreted using the properties of QGP computed for $T=250$ MeV, but that it is very difficult and outright impossible$^{13}$   to do so within completely conventional  pictures  of  nuclear reactions considering experimental data collected at 160--200 A GeV.

Aside of the study of strange antibaryon abundances, which in their production yield comprise knowledge of both the initial state and the freeze-out conditions, we also study the total final state strangeness yield, which is mostly dependent on the conditions prevailing in the dense hadronic matter in its most extreme initial moments. 

%%%%%%%%%%%%%%%%%%%%%%%%%%%%%%%%%%%%%%%%%
\subsection*{Production of Strangeness}
\addcontentsline{toc}{subsubsection}{Production of strangeness}
%%%%%%%%%%%%%%%%%%%%%%%%%%%%%%%%%%%%%%%%%%%%%%%%%%%%%
The production of strangeness flavor in deconfined QGP arises in its dominant fraction in gluon fusion processes$^{14,15}$ $gg\to s\bar s$,  and to a lesser extend in light quark fusion$^{16}$  $ q\bar q\to s\bar s$. While the first order free space (perturbative vacuum) strangeness production processes at fixed values of $\alpha_{\rm s}=0.6$ and $m_{\rm s}= 160$--200 MeV, have been considered for some time, non-perturbative effects were more recently explored. The thermal production rates in medium, incorporating temperature dependent non-perturbative particle masses$^{17}$  have lead to the total strangeness production rate which was found little changed compared to the free space rate. This finding was challenged$^{18}$, but a more recent revaluation of this work$^{19}$ confirmed that the rates obtained with perturbative glue-fusion processes when compared with thermal perturbative results are describing adequately the strangeness production rates in QGP. For further discussion of the current situation regarding thermal rates we refer  to reference$^{20}$.

%%%%%%%%%%%%%%%%%%%%%%%%%%%%%%%%%%%%%%%%%%%%%%%%%%%%%
Uncertainty in the value of  of the strong interaction coupling $\alpha_{\rm s}^2$ introduces considerable systematic error into the computed thermal rates. Recent experimental and theoretical studies of $\alpha_{\rm s}$ have allowed us to eliminate this as an ad-hoc parameter from our description of strange quark production$^{21,22}$. Using nonperturbative techniques of the QCD renormalization group we were able to obtain the strangeness production cross sections and thermal production rates in QGP using $\alpha_{\rm s}(M_Z)$ as input. Specifically, running QCD renormalization group is employed$^{21,22}$ to resum even-$\alpha_{\rm s}$  Feynman-diagrams involving two particles in initial and final states.  We used these results to extend our study of the two generic strangeness observables as function of the impact parameter (baryon content) and collision energy:
\begin{itemize}
\item Specific (with respect to baryon number $B$) strangeness yield  $\langle \bar s\rangle/B$\\ {\it Once produced strangeness escapes, bound in diverse hadrons, from the evolving fireball and hence  the total abundance observed is characteristic for the initial extreme conditions;}
\item Phase space occupancy $\gamma_{\rm s}$\\ {\it Strangeness freeze-out conditions at particle hadronization time $t_{\rm f}$, given  the initially produced abundance, determine the final state observable phase space occupancy of strangeness $\gamma_{\rm s}(t_{\rm f})$.} 
\end{itemize}
\centerline{
\includegraphics[width=0.6\columnwidth]{./AllFigs/96RepFig1.png} 
} %\label{fig-a1} 
\noindent Fig.\;1: $\alpha_{\rm s}(\mu)$, the $\Lambda$-parameter $\Lambda_0$ and $m_{\rm r}(\mu)=m(\mu)/m(M_Z)$ as function of energy scale $\mu$. Thick lines correspond to initial value $\alpha_{\rm s}(M_Z)=0.102$, thin lines are for the initial value $\alpha_{\rm s}(M_Z)=0.115$. Dotted lines are results obtained using the perturbative expansion for the renormalization group functions, full lines are obtained using  Pad\'e approximant of the $\beta$ function. Experimental results for $\alpha_{\rm s}$ selected from recent experimental work and Ref.\,[23]. In bottom portion the dots indicate the pair production thresholds for $m_{\rm s}(M_Z)=$~90~MeV (from Ref.[22]).\\
%%%%%%%%%%%%%%%%%%%%%%%%%%%%%%%%%%%%%%%%%%%%%%%%%%%%%%

The QCD renormalization group equations for the running coupling constant $\alpha_{\rm s}$ and quark mass are:
\begin{equation}%\label{dmuda}
\tag{1}
\mu \frac{\partial\alpha_{\rm s}}{\partial\mu}
=\beta(\alpha_{\rm s}(\mu))=-\alpha_{\rm s}^2\left[\ b_0
   +b_1\alpha_{\rm s} +b_2\alpha_{\rm s}^2 +\ldots\ \right] 
\,,
\end{equation}
\begin{equation}%\label{dmuda}
%\label{dmdmu}
\tag{2}
\mu {\frac{\partial m}{\partial\mu}} =-m\,
\gamma_{\rm m}(\alpha_{\rm s}(\mu))=-m\alpha_{\rm s}\left[\ c_0
+c_1\alpha_{\rm s} + \ldots\ \right]
\,,
\end{equation} 
and the coefficients $b_i,\,c_i$ are given in Ref.[23]. We introduce a Pade approximant of Eq.\,(1) and integrate these equations, using the precise determination of $\alpha_{\rm s}$ at the scale $M_Z=91.2$ GeV. The solutions in Fig.\,1 are obtained for $\alpha_{\rm s}(M_Z)=0.102$\, (thick lines) and $\alpha_{\rm s}(M_Z)=0.115$\,, (thin lines). In the top section of Fig.\,1 we show the variation of $\alpha_{\rm s}$, which is in here relevant 1GeV energy range not well characterized by the first order result often used. This is shown in the middle section of Fig.\,1 where the value $\Lambda_0$ based on a first order result, defined by the implicit equation:
\begin{equation} 
%\label{Lambdarun}
\tag{3}
\alpha_{\rm s}(\mu)\equiv\frac{2b_0^{-1}
(n_{\rm f})}{\ln(\mu/\Lambda_0(\mu))^2}\,,
\end{equation} 
is shown. We see that $\Lambda_0(1\mbox{\,GeV})=240\pm100$ MeV, assuming that the solid lines provide a valid upper and lower limits on $\alpha_{\rm s}$. However, the variation of $\Lambda_0(\mu)$ is significant for $\mu<3$~GeV, questioning the use of first order expressions. 
 
Because  Eq.\,(2) is linear in $m$, we consider the universal multiplicative quark mass scale factor
\begin{equation}  
\tag{4}
\hfill m_{\rm r}=m(\mu)/m(\mu_0)\,.
\end{equation}
Since we refer to  $\alpha_{\rm s}$ at the scale of $\mu_0= M_Z$  we use this reference point also for quark masses. As seen in the bottom portion of Fig.\,1, the change in the quark mass factor is highly relevant, since it is driven by the rapidly changing $\alpha_{\rm s}$ near to $\mu\simeq 1$~GeV. For each of the two different functional dependences $\alpha_{\rm s}(\mu)$ we obtain a different function $m_{\rm r}$. Note that the difference between Pade approximant result (solid lines) and perturbative expansion (dotted lines) in Fig.\,1 amounts to a slight `horizontal' shift of $\alpha_{\rm s}$ and $m_{\rm r}$ as function of~$\mu$.
 
%%%%%%%%%%%%%%%%%%%%%%%%%
\centerline{
\includegraphics[width=0.8\columnwidth]{./AllFigs/96RepFig2.png} 
} %\label{figsigsrun}
\noindent Fig.\;2: QCD strangeness production cross sections  obtained for running $\alpha_{\rm s}(\protect\sqrt{\rm s})$ and $m_{\rm s}(\protect\sqrt{\rm s})$.  Thick lines correspond to initial value $\alpha_{\rm s}(M_Z)=0.102$, thin lines are for the initial value $\alpha_{\rm s}(M_Z)=0.115$. Dotted: results for fixed  $\alpha_{\rm s}=0.6$ and $m_{\rm s}=200$ MeV. solid lines $gg\to  s\bar s$; dashed lines  $q\bar q\to s\bar s$ (adapted from Ref.\,[21])\\ 

%%%%%%%%%%%%%%%%%%%%%%%%%%%%%%%%%%%%%%%%%%%%%%%%%%%%%%
In Fig.\,2, the strangeness production cross sections are shown with $m_{\rm s}(M_Z)=90$~MeV. For the two choices of the running coupling constant considered in Fig.\,1 we depict the cross sections for the processes $gg\to s\bar s$ (solid lines, upper dotted line) and $q\bar q\to s\bar s$ (dashed lines, lower dotted line). Dotted are cross sections computed with fixed $\alpha_{\rm s}=0.6$ and $m_{\rm s}=200$ MeV cross sections shown here for comparison. We note that the glue based flavor production dominates at high $\sqrt{s}$, while near threshold the cross sections due to light quark heavy flavor production dominate. We note the different thresholds for the two values of $\alpha_{\rm s}(\mu)$ used. It is apparent that the cross sections are `squeezed' away from small $\sqrt{s}$ as we increase the value of $\alpha_{\rm s}(M_Z)$, but that the energy integrated cross sections ($\simeq$ rates) are relatively little changed.

%%%%%%%%%%%%%%%%%%%%%%%%%
\centerline{
\includegraphics[width=0.6\columnwidth]{./AllFigs/96RepFig3.png}
} %\label{figTaussrun}
\noindent Fig.\;3: QGP strangeness relaxation time obtained using running $\alpha_{\rm s}$ cross sections shown in Fig.\,2.  Thick lines correspond to initial value $\alpha_{\rm s}(M_Z)=0.102$, thin lines are for the initial value $\alpha_{\rm s}(M_Z)=0.115$. Dotted: results for fixed  $\alpha_{\rm s}=0.6$ and $m_{\rm s}=200$ MeV (adapted from Ref.\,[22])\\


%%%%%%%%%%%%%%%%%%%%%%%%%%%%%%%%%%%%%%%%%%%%%%%%%%%%%%
Using the QCD cross sections we have established, we compute the invariant strangeness production rate $A_{\rm s}$:
\begin{equation}
\notag 
A_{\rm s}= A_{gg}+A_{q\bar q}=\int_{4m_{\rm s}^2}^{\infty}ds
2s\delta (s-(p_{\rm A}+p_{\rm B})^2)
\int{d^3p_{\rm A}\over(2\pi)^32E_{\rm A}}\int{d^3p_{\rm
B}\over(2\pi)^32E_{\rm B}} 
\end{equation}
\begin{equation}
%\label{qgpA}
\tag{5}
\times\left[{1\over 2} g_g^2f_g(p_{\rm A})f_g(p_{\rm B})
{\sigma_{gg}}(s) + n_{\rm f}g_q^2 f_q(p_{\rm A})
f_{\bar q}(p_{\rm B}){\sigma_{q\bar q}}(s)\right]\,,
\end{equation}
and we obtain the relaxation time constants $\tau_{\rm s}$ shown in
Fig.\,3:
\begin{equation}%\label{tauss}
\tag{6}
\tau_{\rm s}\equiv
{1\over 2}{\rho_{\rm s}^\infty\over{(A_{gg}+A_{qq}+\ldots)}}\,,
\end{equation}
where the dots indicate that other mechanisms may contribute to strangeness production, reducing the relaxation time, obtained here considering the processes of gluon and quark fusion. Solid lines correspond to the two cases $\alpha_{\rm s}(M_Z)$ considered here, dotted line shows for comparison  the result of earlier studies with fixed   $\alpha_{\rm s}=0.6$ and $m_{\rm s}=200$ MeV.   

In Fig.\,4 both strangeness observables of interest here, are shown as function of the laboratory energy of the beam for central collisions.  To the right we see the observable most related to the initial conditions$^{12}$  
of the fireball: the ratio of the total strangeness produced $\langle s \rangle$ to the number of baryon participants $B$ in the fireball. By taking the ratio, we eliminate the explicit dependence of the QGP fireball volume. The result is a sensitive measure of the initial conditions. The experimental result we note is $\langle s \rangle/B=0.86\pm0.14$, reported in Ref.\,[24] for the collisions of S--Ag at 200 A GeV. This is in remarkable agreement with the results we obtained. Since the yield of strangeness per baryon is primarily determined  by the initial thermal properties of the QGP fireball and the early fireball evolution, we must presume that we have an appropriate description not only of strangeness formation rate, but also of the initial conditions (temperature) and the early evolution of the fireball \ldots 

%%%%%%%%%%%%%%%%%%%%%%%%%%%%%%%%%%%%%%%
\centerline{
\includegraphics[width=0.75\columnwidth]{./AllFigs/96RepFig4.png}
} %\label{figvarelab}
\noindent Fig. 4: $\gamma_{\rm s}(t_{\rm f})$ and $\langle \bar s\rangle/B$ as function of beam energy for central  S--W/Pb collisions (solid lines) and  Pb--Pb collisions (dashed lines) assuming $m_{\rm s}(M_Z)=90$ MeV, three dimensional expansion of the fireball with $v=c/\protect\sqrt(3)$, and stopping 50\% (S--W/Pb), 100\% (Pb--Pb). For $\gamma_{\rm s}$ we  take freeze-out at $T_{\rm f}=140$ MeV --- the vertical bar corresponds to the value of  $\gamma_{\rm s}$ found in S--W data analysis$^{11}$ (adpted from Ref.\,[1])\\

%%%%%%%%%%%%%%%%%%%%%%%%%%%%%%%%%%%%%%%%%%%%%%%%%%%%%

In a wide energy range we find that the specific strangeness yield rises linearly with the (kinetic) fireball energy content, reaching $\langle \bar s\rangle/B=0.8\pm0.15$ for S--W/Pb collisions at 200A GeV. It would be quite surprising to us, if other reaction models without QGP would find this linear behavior with similar coefficients, which we can determine in our case as function of the properties of QGP and its dynamical evolution. We therefore believe that this result is an interesting characteristic feature of our QGP thermal fireball model.

The phase space occupancy $\gamma_{\rm s}(t_{\rm f})$, shown to left in Fig.\,4 is influenced by the initial condition, the fireball evolution and the freeze-out conditions. Since the initially produced strangeness abundance does not reannihilate, strangeness can even overpopulate the final available phase space at plasma disintegration, so for large and long lived fireball scenarios, strange antibaryon abundances in Pb--Pb collisions could show $\gamma_{\rm s}>1$, and thus lead to spectacular enhancement of some particle ratios such as $\overline{\Xi}/\overline{\Lambda}\propto \gamma_{\rm s}$. However, results shown here, suggest that we just reach $\gamma_{\rm s}=1$ in Pb--Pb collisions up to 300 A GeV.


%%%%%%%%%%%%%%%%%%%%%%%%%%%%%%%%%%%%%%%%%%%%%%%%%%%%%%%%%%
\subsection*{Fireball Dynamics and Initial Conditions}
\addcontentsline{toc}{subsubsection}{Fireball dynamics and initial conditions}
%%%%%%%%%%%%%%%%%%%%%%%%%%%%%%%%%%%%%%%%%%%%%%%%%%%%%
A key input in the discussion has been the initial conditions and evolution$^{12}$ of  the fireball which in our approach are fully described in a dynamical reaction picture, and nothing is left to arbitrary assumption. However, the picture of the reaction is based on the conventional wisdom and comprises barely proven assumptions, such as that the fireball expansion is adiabatic, the freeze-out occurs at 140 MeV, in addition to a courageous extrapolation of the QGP equations of state to the temperature range of importance here. The initial conditions are determined from the requirement that the dynamic pressure imparted on the fireball in the collision should be balanced by the {\it internal parton pressure}, which we assume to be given in terms of the initial temperature. The energy per baryon content determines the baryon density in the fireball and thus also the chemical potentials, once the degree of chemical equilibration is  known. The energy per baryon is derived from the stopping fractions that can be extracted form global features of the experimental results. \ldots
 

%%%%%%%%%%%%%%%%%%%%%%%%%%%%%%%%%%%%%%%%%%%%%%%%%%%%% 
\centerline{
\includegraphics[width=0.6\columnwidth]{./AllFigs/96RepFig5.png} 
}%\label{fig1S95} 
\noindent Fig.\;5:  Temperature $T_0$, light quark fugacity $\lambda_{\rm q}$ and entropy per baryon $S/B$ at the time of full chemical equilibration  as function of the QGP-fireball energy content $E/B$. Results for momentum stopping $\eta=1$ (solid line), 0.5 (dot-dashed line) and 0.25 (dashed line) are shown. Experimental `data' points are derived from our interpretation of experimental data$^{11,25}$ (adapted from Ref.\,[12])\\
%%%%%%%%%%%%%%%%%%%%%%%%%%%%%

In  Fig.\,5 we show, as function of the specific energy content $E/B$,  the expected behavior of temperature $T_0$,  the light quark fugacity $\lambda_{\rm q}$ and the entropy per baryon $S/B$ at the time of full chemical equilibration in the QGP fireball.  The range of the possible values as function of $\eta$ is indicated by showing  results, for $\eta=1$ (solid line), 0.5 (dot-dashed line) and 0.25 (dashed line). The experimental bars on the right hand side  of the  Fig.\,5 show for high (8.8 GeV) energy the result of analysis$^{11}$ of the WA85 data$^{3}$. The experimental bars  on the left hand side of the  Fig.\,5 (2.6 GeV) are taken from our analysis of the BNL-AGS data$^{25}$, but note that in this case we had found $\lambda_{\rm s}=1.7$ and not $\lambda_{\rm s}=1$ as would be needed for the QGP interpretation at this low energy. 

Among the key features in the  Fig.\,5, we note that, in qualitative terms, the drop in temperature with decreasing energy and stopping is intuitively as expected. At low (BNL-AGS) energies there is relatively rapid variation in $\lambda_{\rm q}$ which drives much of the variation in $T_0$. The value of $\lambda_{\rm q}$ is relatively insensitive to the stopping power.  This implies that even when different trigger conditions lead to different stopping fractions $\eta_i$, the resulting value of $\lambda_{\rm q}$ which is determining  the strange particle (baryon/antibaryon) ratios, is rather independent of different trigger conditions. 
 
Since we have now obtained as function of collision energy both the particle fugacities and strangeness phase space occupancy, we can combine these to derive as function of energy the  antibaryon ratios $\overline{\Lambda}/\overline{p}$ and $\overline{\Xi^-}/\overline{\Lambda}$.  The result, shown in Fig.\,6, (we show here the ratio of integrated yields with $m_\bot\ge 1.7$ GeV.) suggests that for a QGP fireball these ratios essentially stay constant, as the energy is lowered, since the increase associated with an increase in baryochemical potential is just compensated by the decrease in $\gamma_{\rm s}$ arising from lower gluon  collision frequency. We know that these ratios are much smaller, essentially zero,  in transport models involving conventional hadronic matter. Consequently, a way to explore the possibility that deconfinement has been present at high energies is  to seek a  substantial decrease in in this ratio, and thus a change in the reaction mechanism, as the collision energy is lowered. We thus believe that in order to ascertain the possibility that indeed the QGP phase is formed at energies available today (up to 9 GeV per nucleon in the CM frame) a more systematic exploration as function of collision energy of two above discussed strange particle observables $\gamma_{\rm s}(t_{\rm f})$ and $\langle \bar s \rangle /B$, and in particular the strange antibaryon ratios, is needed. \ldots

%%%%%%%%%%%%%%%%%%%%%%%%%%
\centerline{
\includegraphics[width=0.56\columnwidth]{./AllFigs/96RepFig6.png}
} %\label{stratiogam}
\noindent Fig.\;6: Fixed $m_\bot$ $\overline{\Lambda}/\overline{p}$ and $\overline{\Xi^-}/\overline{\Lambda}$ as function of $E/B$ in the Pb--Pb fireball, taking into account variation of $\gamma_{\rm s}$ shown in Fig.\,4. Thick lines correspond to initial value $\alpha_{\rm s}(M_Z)=0.102$, thin lines are for the initial value $\alpha_{\rm s}(M_Z)=0.115$ (from Ref.1)\\ %%%%%%%%%%%%%%%%%%%%%%%%%%%%%%%%%%%%%%%%%%%%%%%%%%%%%  

\footnotetext{\vspace*{-0.5cm}\small
\begin{enumerate}
%
\item%1{actab}
J. Rafelski, J. Letessier, and A. Tounsi, {\it Strange Particles from 
Dense Hadronic Matter}, survey prepared for {\it Acta Phys. Pol. B} June (1996) (published Acta Phys.\ Polon.\ B {\bf 27}, 1037 (1996))

\item%2{AIP}
{\it Strangeness in Hadronic Matter: S'95} Proceedings of Tucson
workshop, January 1995, American Institute of Physics Proceedings Series
Vol.\,340, (New York 1995), J. Rafelski, editor 
 
\item%3{WA85}  %{WA85AIP}
WA85 collaboration presentations in  Ref.\,[2];
see also:\\
D. Evans {et~al.} (WA85 collab.), {Nucl. Phys.} A
{\bf 566}, 225c (1994);\\
S. Abatzis {et~al.} (WA85 collab.), {Phys. Lett.} B
{\bf 259}, 508 (1991);\\
S. Abatzis {et~al.} (WA85 collab.), {Phys. Lett.} B {\bf
270}, 123 (1991) 
  
\item%4{Omega} 
F. Antinori, in  Ref.\,[2];\\
S. Abatzis {et al.} (WA85 collab.), {Phys. Lett.} B
{\bf 316}, 615 (1993);  {Phys. Lett.} B {\bf 347}, 158 (1995) 
 
\item%5{NA35}   %{Ga95}
% \item{Al94} \item{NA35AIP} 
Th. Alber et al. (NA35 collab.), Z. Phys. C{\bf 64}, 195 (1994), 
and references therein;\\
M. Ga\'zdzicki et al. (NA35 collab.), Nucl. Phys. A{\bf 590},
 197c (1995);\\
P. Foka and the NA35 collaboration and
M. Ga\'zdzicki and the NA35 collaboration, in Ref.\,[2] 
 
\item%6{NA35pbar} 
J. G\"unther for the NA35
collaboration, to appear in proceedings of QM'95, Monterey, January
1995 (edited by A. Poskanzer et al.); and private communication.
 
\item%7{lattice}
C. DeTar, {\it QGP in Numerical Simulations of Lattice
QCD}, in {\it Quark-Gluon Plasma 2}, p.1,  R.C. Hwa, Editor, World
Scientific, (Singapore 1995);\\
T. Blum, L. K\"arkk\"ainen, D. Toussaint, S. Gottlieb, 
Phys. Rev. D {\bf 51}, 5153 (1995)

\item%8{HM96}
J. W. Harris and B. M\"uller, {\it The search for the Quark-Gluon Plasma},
to appear in {\it Ann. Rev. Nuc. Science}, preprint: hep-ph/9602235 (1996) 

\item%9{Raf82} 
J. Rafelski, {\it Phys. Rep.} {\bf 88} (1982) 331;\\
J. Rafelski and R. Hagedorn, in: {\it Statistical
Mechanics of Quarks and Hadrons}, edited by H. Satz, North Holland, 
Amsterdam 1981, pp. 253--272;\\ 
J. Rafelski, in: {\it Workshop on Future Relativistic
Heavy Ion Experiments}, edited by  R. Bock and R. Stock, GSI-Orange 
Report 81-6, Darmstadt 1981, pp. 282--324 

\item%10{Raf91}   
J. Rafelski, {\it Phys. Lett. B} {\bf 262}, 333 (1991); \\
J. Rafelski,{\it Nucl. Phys. A} {\bf 544}, 279c, (1992).
 
\item%11{analyze} 
J. Letessier, A. Tounsi, U. Heinz, J. Sollfrank and J. Rafelski, 
{Phys.\ Rev.} D {\bf 51}, 3408 (1995);\\
J. Letessier, J. Rafelski and  A. Tounsi,  {Phys. Lett.} B {\bf 321}, 394
(1994);\\
J. Sollfrank, M. Ga\'zdzicki, U. Heinz and J. Rafelski,  {Z. Physik} C
{\bf 61}, (1994) 
 
\item%12{dynamic}
J. Letessier, J. Rafelski and A. Tounsi, 
{\it Energy dependence of strange particle yields from
a quark-gluon plasma fireball} Preprint AZPH-TH/95-13
and PAR/LPTHE/95-24, submitted to {\it Phys. Rev. C};\\
J. Letessier, J. Rafelski,  and A. Tounsi,
{\it Quark-gluon plasma formation and strange antibaryons}, 
Preprint AZPH-TH/95-14R and PAR/LPTHE/95-36R,
submitted to {\it Phys. Lett. B} ;\\
J. Letessier, J. Rafelski,  and A. Tounsi, A., 
{\it Phys. Lett. B} {\bf 333}, 484, (1994) 

\item%13{Csernai}
L. Csernai, private communication;\\
N.S. Amelin, L.V. Bravina, L. P. Csernai, V.D. Toneev, K.K.
Gudima, S.Yu. Sivoklokov, {Phys. Rev.} C {\bf
47} (1993) 2299 
 
%%%%%%%%%%%%%%%%%%%%%%%%%%%%%%%%%%%%%%%%%%%%%%%%%%%%%%%

\item%14{RM82}
J. Rafelski and B. M\" uller, Phys. Rev. Lett.  {\bf 
48}, 1066, (1982); {\bf 56}, 2334E, (1986) 

\item%15{MSM86}
T. Matsui,B. Svetitsky and L.D. McLerran, 
{\it Phys. Rev. D} {\bf 34}, 783, (1986) 

\item%16{BZ82}
T. Bir\'o and J. Zim\'anyi, {Phys. Lett. B} {\bf 113}, 6,
(1982); {\it Nucl. Phys. A} {\bf 395}, 525, (1983) 
 
 
\item%17{BLM90} 
T.S. Bir\'o, P. L\'evai and B. M\"uller, 
{\it Phys. Rev. D} {\bf 42}, 3078, (1990) 

\item%18{AS93} 
T. Altherr and D. Seibert, {\it Phys. Lett. B}
{\bf 313}, 149, (1993), and {\it Phys. Rev. C} {\bf 49}, 1684, (1994) 
 
\item%19{BCDH95} 
N. Bili\'c, J. Cleymans, I. Dadi\'c and D. Hislop,
{\it Phys. Rev. C}  {\bf 52}, 401, (1995).  

\item%20{SH95} 
J. Sollfrank and U. Heinz {\it The Role of
Strangeness in Ultrarelativistic Nuclear Collisions}, Helsinki
preprint HU-TFT-95-27, in {\it Quark-Gluon Plasma 2},
R.C. Hwa (Eds.), World Scientific, Singapore (1996).

\item%21{impact}
J. Letessier,  J. Rafelski,  and A. Tounsi,
{\it Impact of QCD and QGP properties on strangeness production},
{\it submitted to  Phys. Lett. B}, Preprint 
AZPH-TH/96-08, PAR/LPTHE/96-10,  (1996),
 
\item%22{snow}
J. Rafelski, J. Letessier,  and A. Tounsi,
{\it $\alpha_{\rm s}(M_Z)$ and Strangeness Production}
to appear in proceedings of Snowbird Nuclear 
Dynamics Workshop, February 1996, eds. W. Bauer 
and G. Westphal,  Plenum Press, New York (1996)


%%%%%%%%%%%%%%%%%%%%%%%%%%%%%%%%%%%%%%%%%%%%%%%%%%
 
\item%23{QCD95} 
I. Hinchliffe, {Quantum Chromodynamics}, 
update (URL:http://pdg.lbl.gov/) August 1995, in  
L. Montanet {et al.}, 
{\it Phys. Rev. D} {\bf 50}, 1173, (1994);\\
T. Muta, ``Foundations of Quantum Chromodynamics'', 
World Scientific, Singapore (1987).

\item%24{GR96}
M. Ga\'zdzicki and D. R\"ohrich, 
{\it Strangeness in Nuclear Collisions}
preprint IKF-HENPG/8-95, {\it Z. Physik C} (1996).

\item%25{BNL-AGSthermal} 
J. Rafelski and M. Danos {Phys. Rev.} C {\bf 50}, 1684 (1994);\\  
J. Letessier, J. Rafelski and A. Tounsi, {Phys. Lett.} B {\bf 328}, 499
(1994).
\end{enumerate}
}
\end{mdframed}
\vskip 0.5cm
%%%%%%%%%%%%%%%%%%%%%%%%%%%%%%%%%%%%%%%%%

%%%%%%%%%%%%%%%%%%%%%%%%%%%%%%%%%%%%%%%%%
\subsection{A picture with  STAR at RHIC}
% 
One of the outgrowths of the Hagedorn 1995 meeting was further strengthening of my good relationship with Hans Gutbrod: he became a SUBATECH  lab director and I became engaged in a collaboration  with the RHI research group at SUBATECH in Nantes, France. This in turn  resulted in  my search for a more active role in the research program of the STAR collaboration, taking with me the SUBATECH group. In  1997 I prepared  an individual theoretical proposal how I could contribute to the STAR collaboration work. I show in the following a few pages only from this lengthy document.

At the February 1999  STAR collaboration meeting  held at BNL, I was  not admitted to the membership. The  STAR collaboration saw my effort that paralleled SUBATECH entry as a maneuver, probably an effort to gain early insight into the forthcoming STAR experimental data -- which were delayed  by technical RHIC problems to mid-2000. I was told to consider creating an experimental group at Arizona that would have to apply to join STAR, a long-term diplomatic delay, if feasible.  

This decision  was  of disadvantage for STAR, since they did not have my data analysis support. On the other hand I could  present, beginning with 2004/5, my own analyzis of their available data~\cite{Rafelski:2004dp,Letessier:2005kc}. These results were demonstrating the universality of RHIC and SPS bulk fireball properties. This insight was complemented after LHC-ALICE date became available by demonstration that the more precise lower energy results from STAR at RHIC and the very high energy results from ALICE at LHC also originate in the same bulk fireball properties~\cite{Petran:2013qla,Rafelski:2014cqa}. 

There is another possible consequence of this fateful decision. I  was wired into the CERN-SPS context. I could have helped to coordinate between the CERN work on the QGP announcement that followed exactly the February  events at BNL. On the other hand I was now perhaps mistrusted at CERN and, at the same time I could not help the flow of information between the labs. As consequence, CERN coasted to the new phase of matter announcement without much of communication with STAR.  

I believe the STAR decision  was the event that severed any chance of possible joint QGP announcement between RHIC and SPS groups. It should be remembered that the STAR  collaboration  made the decision to go alone, arguably in the belief they were holding in February 1999 the key to the QGP discovery.\\

\noindent\textit{The February, 1999 RHIC-STAR collaboration photo is followed by a few paragraphs from the 1998 proposal I presented, when attempting to join STAR, this short segment from a very long document introduces SHM analysis method and updates kinetic theory of strangeness production, more details of SHM presenteed in this document  are found on  page~\pageref{SHM-STAR} below:}%\\[-0.7cm]  REStORE IF SAME PAGE
%
%%%%%%%%%%%%%%%%%%%%%%%%%%%%%%%%%%%%%%%%%%%%%%

  
\begin{mdframed}[linecolor=gray,roundcorner=12pt,backgroundcolor=Dandelion!15,linewidth=1pt,leftmargin=0cm,rightmargin=0cm,topline=true,bottomline=true,skipabove=12pt]\relax%
%%%%%%%%%%%%%%%%%%%%%%%%%
\centerline{
\includegraphics[width=1.0\columnwidth]{./AllFigs/Star1999cs.jpg} 
} %\label{figsigsrun}
\noindent February 1999 STAR collaboration picture. Front raw center: John Harris, spoksperson; the author is third person on his right also fourth person in picture from left


\section*{Hadronic Probes of QGP}
\addcontentsline{toc}{subsubsection}{Hadronic probes of QGP}
%%%%%%%%%%%%%%%%%%%%%%%%%%%%%%%%%
\subsection*{Strangeness as observable of QGP}
\ldots  We would like to refine the capability of the STAR detector such that certain longer lived strange particles can be detected with greater efficiency and precision, creating a more effective diagnostic tool of the dense state created in nuclear collisions at 100 A GeV center of mass energy.

Strangeness has been predicted already 15 years ago to be abundantly produced should the deconfined QGP phase be formed. Further study has confirmed that it is not reannihilated in rapid decomposition of the dense matter state and that the pattern of strange particle production is specific for the state of matter formed, its evolution and hadronization process.  Because there are many different strange particles, we have a very rich field of observables with which it is possible to explore diverse properties of the particle source. \ldots  

Today, it has been seen  in SPS experiments up to 200 A GeV that overall particles containing  strangeness are indeed produced more abundantly in relativistic nuclear collisions. Through the diligent work of the NA35/NA49 collaboration, which has developed a complex `$4\pi$'-detectors strangeness production excess of about factor two over  expected yields based on simple scaling of $N$--$N$ reactions has been today established$^{1}$. 

Equally significantly, the work of WA85/94/97 collaborations$^{2,3,4}$  as well as that of NA35/49$^{5}$ for the ratio $\overline\Lambda/\bar p$, shows that the abundance of strange antibaryons is rather unusual in that production pattern of these particles appears to arise in manner expected from the evaporation from the deconfined QGP phase. For example, the remarkable observation that the yield of strange antibaryons $\overline{\Lambda}+\overline{\Sigma}$ exceeds the yield of antipartons$^{5}$ is quite unexpected for a reaction picture involving confined particles. 

On the other hand it is a rather natural consequence for particles evaporated from deconfined QGP phase in which strange and non-strange quarks are have reached chemical equilibrium. Unfortunately, this spectacular result of the NA35 collaboration, is marred by lack of precision, one can still argue that the surprise is comprised in a few standard deviations, and thus could still go away. It is  our hope that the fully developed STAR tracking will allow to see such anomalous  effects at the level of precision which  will allow to perform detailed comparison with theoretical models, establishing the production mechanisms and thus the formation of deconfined phase.

%%%%%%%%%%%%%%%%%%%%%%%%%%%%%
\subsection*{Strangeness signatures of deconfinement}
\addcontentsline{toc}{subsubsection}{Strangeness signatures of deconfinement}
%%%%%%%%%%%%%%%%%%%%%%%%%%%%%
As implied in above qualitative discussion, there are two generic strangeness  observables which allow further diagnosis of the physical state produced in relativistic heavy ion interactions:
\begin{itemize}
\item 
{\bf absolute yield} of strangeness: Once produced in hot and dense hadronic matter, {\it e.g.},  the QGP phase, strangeness/charm is not reannihilated in the evolution of the deconfined state towards freeze-out, because in the expansion and/or cooling process the rate of  production/annihilation rapidly diminishes and becomes negligible.  Therefore the massive flavor yield is characteristic of the initial, most extreme conditions, including the approach to chemical equilibrium of gluons in the deconfined phase.
\item 
{\bf phase space occupancy$^{6}$  $\gamma_{i}$}: $\gamma_{i}$ describes how close the flavor  yield per unit of volume ($i={\rm s,c}$) comes to the chemical equilibrium expected; $\gamma_i$  impacts strongly the distribution of flavor among final state  hadronic particles.
\end{itemize}

This rise of $\gamma_s\to 1$, (in QGP phase) which in the collision occurs rapidly as function of time, and leads to a large freeze-out value seen in experiment, as observed in nuclear collisions at 160--200 A GeV, is believed to be at the origin of the significantly enhanced abundance of multistrange particles.

\ldots\ldots


\ldots the invariant rate the strangeness relaxation time
$\tau_{\rm s}$ shown in Fig.\,1, as function of
temperature is obtained$^{9,10}$:
\begin{equation}\tag{4}\label{tauss}
\tau_{\rm s}\equiv
{1\over 2}{\rho_{\rm s}^\infty(\tilde m_{\rm
s})\over{(A_{gg}+A_{qq}+\ldots)}}\,. 
\end{equation}
Note that any so far unaccounted strangeness production processes would add to the production rate incoherently, since they can be distinguished by the presence of incoming/outgoing gluons. Thus the current calculation offers an upper limit on the actual relaxation time, which may still be smaller. In any case,  the result shown in Fig.\,1   suffices to confirm that strangeness will at the end of QGP evolution at RHIC be very near to chemical equilibrium, assuming that the lifespan of QGP is at least given by the size of the colliding system.

We  see in Fig.\,1 also the impact of a 20\% uncertainty in $m_{\rm s}(M_{{Z}})$, indicated by the hatched areas. The calculations made$^{11,12}$ at fixed values $\alpha_{\rm s}=0.5$ and $m_{\rm s}=200$~MeV (dotted line in Fig.\,1) are well within the band of values related to the uncertainty in the strange quark mass. \\
 
%%%%%%%%%%%%%%%%%%%%%%%%%
%\begin{figure}\sidecaption[t]
\centerline{
\includegraphics[width=0.7\columnwidth]{./AllFigs/1997Fig2.png} 
}
\noindent{\small Fig. 1: QGP strangeness relaxation time, for $\alpha_{\rm s}(M_{Z})=0.118$, (thick line) and = 0.115 (thin line); $m_{\rm s}(M_{{Z}})=90$~MeV. Hatched areas: effect of variation of strange quark mass by 20\%. Dotted: comparison results for fixed  $\alpha_{\rm s}=0.5$ and $m_{\rm s}=200$ MeV (adapted from Refs.[9,10])\\
}
%\label{figTaussrun}
%\end{figure}
%%%%%%%%%%%%%%%%%%%%%%%%%%%

Allowing for dilution of the phase space density 
in expansion, the dynamical equation describing 
the change in $\gamma_{s}(t)$ is:
\begin{equation}\tag{5}\label{dgdtf}
\hspace*{-0.2cm}{{d\gamma_{\rm s}}\over{dt}}\!=\!
\left(\!\gamma_{\rm s}{{\dot T m_{\rm s}}\over T^2}
     {d\over{dx}}\ln x^2K_2(x)\!+\!
{1\over 2\tau_{\rm s}}\left[1-\gamma_{\rm s}^2\right]\!\right).
\end{equation}
Here K$_2$ is a Bessel function and $x=m_{\rm s}/T$. With the relaxation constant $\tau_{\rm s}(T(t))$, these equations can be integrated numerically, leading to the values of the two observables, $\gamma_{\rm s}$ and $N_{\rm s}$ that control the yields of strange particles$^{13}$ evaporated from the expanding and hadronizing QGP blob, \ldots
 
{\it The proposed analysis methods using particle ratios   appears in the following section of these diaries.}

%%%%%%%%%%%%%%%%%%%%%%%%%%%%%%%%%%%%%%%%%
\subsection*{Concluding remarks}
The relative total abundance of strangeness is most related to the initial condition, the \lq hotter\rq\ the initial state is, the greater the production rate, and thus the final state relative yield, to be measured with respect to  baryon number or global particle multiplicity (entropy). The phase space occupancy of strangeness $\gamma_{s}$ depends aside of the initial production rate, on the final state dilution characterized by dynamics of the expansion and the freeze-out temperature. We believe that we will be able to use observed  features of strange mesons,  baryon and antibaryon production to see the formation of the deconfined state and to study some QCD properties and parameters. Experience with the analysis at SPS energies confirms that validity of this method to the study of the deconfined phase, though the precision of the SPS results so far has not sufficed to convince everyone that indeed a QGP phase has been formed. However, the strange particle production results obtained at 160--200 A GeV are found to be well consistent with the QGP formation hypothesis. 

\footnotetext{\vspace*{-0.5cm}
\begin{enumerate}

\item%1{Ody97}
G. Odyniec, invited lecture at the QM'97-Tsukuba, to be published in
proceedings {\it Nucl. Phys. } {\bf A}, (1998).

\item%2{WA85} 
D. Evans, (WA85 Collaboration), p.\,79 in \cite{S96}. %\\
M. Venables, (WA94 Collaboration), p.\,91 in \cite{S96}.

\item%3{WA85Casc} 
S. Abatzis et al. (WA94 Collaboration)
Phys. Lett. B {\bf 354}, 178 (1995); D. Evans {et~al.} (WA85 Collaboration),
{Nucl. Phys.} A {\bf 566}, 225c (1994).
  
\item%4{WA85Om}
S. Abatzis {et al.} (WA85 Collaboration), 
{Phys. Lett.} B {\bf 347}, 158 (1995); 
{Phys. Lett.} B {\bf 316}, 615 (1993).  
 
\item%4{WA97} 
I. Kr\'alik, (WA97 Collaboration), 
{\it Hyperon and Antihyperon Production in 
Pb--Pb Collisions at 158A GeV/c}, in \cite{SQM97};\\
QM'97presentation, to be published in
proceedings {\it Nucl. Phys. } {\bf A}, (1998).

\item%5{NA35Lpbar}
T. Alber et al, (NA35 Collaboration)
{\it Phys. Lett} {\bf B 366}, 56 (1996).

\item%6{Raf91} 
J. Rafelski, {\it Phys. Lett.} {\bf B 262}, 333 (1991);
{\it Nucl. Phys.} {\bf A544}, 279c (1992).

\item%7{LRT96}  
J. Letessier, J. Rafelski, and A. Tounsi,
 {\it Phys. Lett.} {\bf B389}, 586 (1996).
 
\item%8{Sch96} 
M. Schmelling, 
{\it Status of the Strong Coupling Constant}, p91, in
proceedings of: 28th International Conference on High
Energy  Physics, Warsaw, July 1996, Z. Ajduk and 
A.K. Wr\'oblewski , Eds.,  (World Scientific, Singapore 1997)

\item%9{ICHEP96} 
J. Rafelski, J. Letessier and A. Tounsi,
{\it Thermal Flavor Production and Signatures of Deconfinement},
p.971 in proceedings of: 28th International Conference on High
Energy  Physics, Warsaw, July 1996, Z. Ajduk and 
A.K. Wr\'oblewski , Eds.,  (World Scientific, Singapore 1997)

\item%10{acta98} 
J. Rafelski, J. Letessier and A. Tounsi,
{\it Acta Phys. Pol.} {\bf B},  (1998) (in press); hep-ph/9710340.

 
\item%11{RM82}
J. Rafelski and B. M\"uller, {\it Phys. Rev. Lett}
{\bf 48}, 1066 (1982); {\bf 56}, 2334E (1986).

\item%12{KMR86}
P.~Koch, B.~M\"uller, and J.~Rafelski, {\it Phys. Rep.} {\bf 142},
167 (1986); Z. Phys. {\bf A324}, 453 (1986).

\item%13{acta96} 
J. Rafelski, J. Letessier and A. Tounsi,
{\it Acta Phys. Pol.} {\bf B27}, 1035 (1996).

\end{enumerate}
}
\end{mdframed}
%%%%%%%%%%%%%%%%%%%%%%%%%%%%%%%%%%%%%%%%%%%



 
 
