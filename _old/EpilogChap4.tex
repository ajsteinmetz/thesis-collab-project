\markboth{4. Epilogue: Using QGP}{Strangeness in QGP: Diaries}
\section{Epilogue: Using QGP}
One natural use of the QGP understanding is the connection between the QGP phase in the early Universe, and the matter surrounding us, see page \pageref{UniverseR1}. In a first step one must understand the particle content in the Universe once quark-gluon  radiation turns into hadron matter. I submitted a manuscript~\cite{Fromerth:2002wb} for publication in 2002. The initial reviews were  outright helpful. 

The problem  was, they conflicted with editorial procedures. As we struggled to include the requested additional material at the cost of deleting something else, in order to meet the constraints of the four page limit, the publication process  derailed. I show the referee requests from the first round of review of~\cite{Fromerth:2002wb}, and reprint the manuscript below in response format v3. The final arXiv\rq ed v4  presents yet  a different remix of results. In the end I through up my arms. Our valuable set of insights remained unpublished -- a decade later some of the material was embedded into a longer conference report~\cite{Fromerth:2012fe}. \\

\subsection{Quark-hadron Universe}
\subsubsection{Good or bad advice?}
\noindent\textit{First round proposals of referees of the  Physical Review Letters manuscript LY8289 \lq Hadronization of the Quark Universe\rq\  by \textbf{Michael J. Fromerth and Johann Rafelski} received  on December 23,  2002.}\\[-0.7cm]
%
\begin{mdframed}[linecolor=gray,roundcorner=12pt,backgroundcolor=GreenYellow!15,linewidth=1pt,leftmargin=0cm,rightmargin=0cm,topline=true,bottomline=true,skipabove=12pt]\relax%
% 
\textbf{Referee A}\\
\indent \ldots need to revise their manuscript substantially to make clear their goals and accomplishments and why it is that their readers should be interested and care about their results. \ldots  Why should I (the reader) care about the d-quark chemical potential \ldots Is it the chemical potential, or the RATIO of it to the temperature that is the physically interesting quantity? \ldots  Much of the early universe cosmology described by the authors may not really be needed in their text. \ldots The authors seem confused by the differences between annihilation in a symmetric and in an asymmetric universe. \ldots PRODUCTS of cross sections and velocity become constant. This is true for exothermic processes, such as s-wave annihilation. However, none of these numbers really play a crucial role in what - I think - the authors are interested in. \ldots The use of \ldots  an unrefereed book, to justify details or assumptions seems inappropriate to me. \ldots There are schemes for the creation of baryon number in which the baryon number and the lepton number can be dissimilar. \ldots  I am confused by the issue of charge distillation.\\

\noindent\textbf{Referee B:}\\
\indent The authors trace back chemical potentials in the early Universe. They find that during the quark-hadron phase transition the two phases have differing and non-zero charge densities on the average. I think that the phenomenon of charge distillation is interesting, though quantitatively the effect is not so large. In my opinion the manuscript in general meets the criteria of Physical Review Letters. I recommend that it would be published after \ldots  There is some entropy generated. But presumably only a tiny amount, because of the weakness of the transition. I think that the effect for the present results should be in practice negligible. \ldots  the authors require that B-L = 0. I agree that this makes a lot of sense for practical purposes. However, it is not unconditionally dictated by baryogenesis. \ldots It would be interesting to understand how the these results would incorporate into the dynamics of the (first-order) phase transition \ldots
\end{mdframed}
\vskip 0.5cm 
%%%%%%%%%%%%%%%%%%%%%%%%%%%%%%%%%%%%%%%%%%%%%%%%%%%%%%%%%


\subsubsection{Hadronization of the quark universe}
\noindent\textit{This is v3 of the arXiv\rq ed manuscript  astro-ph/0211346 of December 31, 2002, which contains changes made to accomodate the first round of  referee comments.}\\[-0.7cm]
%
%%%%%%%%%%%%%%%%%%%%%%%%%%%%%%%%%%%%%%%%%%%%%%
% 
\begin{mdframed}[linecolor=gray,roundcorner=12pt,backgroundcolor=Dandelion!15,linewidth=1pt,leftmargin=0cm,rightmargin=0cm,topline=true,bottomline=true,skipabove=12pt]\relax%
%%%%%%%%%%%%%%%%%%%%%%%%%
\textbf{\Large Hadronization of the Quark Universe}\\[0.2cm]
\textbf{Abstract}
{\small Recent advances in the understanding of equations of state of strongly interacting matter allow exploration of conditions in which matter (protons, neutrons) formed. Using the recently solidified knowledge about photon to baryon ratio, and neutrino oscillations, we are able to trace out the evolution of particle chemical potentials, beginning in quark-gluon phase (QGP) when the Universe was 1\,$\mu$s old, through hadronization and matter-antimatter annihilation toward onset of nucleosynthesis.  In  the mixed hadron-quark phase a significant hadron sector  electric  charge distillation is found given non-zero chemical potentials.}\\


%\section{Introduction}
%%%%%%%%%%%%%%%%%%%%%%%%
In the standard big-bang model, the large primordial baryon abundance formed at hadronization of the deconfined quark-gluon plasma (QGP) disappears due to mutual annihilation, exposing a slight net baryon number observed today~[1]. The annihilation period began after the phase transformation from the QGP to a hot hadronic gas (HG), roughly $20$--$30\,\mu$s after the big bang when the Universe was at a temperature of $\sim 160$~MeV. In the ensueing evolution the energy fraction in baryons and antibaryons in the  Universe dropped from $\sim 10\%$ when Universe was about 40\,$\mu$s old to less than $10^{-8}$ when it was one second old. 

Tracing the evolution  of particle chemical potentials with temperature in the hadronic domain allows us to connect this picture to ongoing laboratory relativistic heavy ion collision experimental work, and to verify our understanding of the hadronic matter behavior in the early Universe. We will show how the study of the Universe chemical composition opens up the possibility of amplification of a much smaller and preexisting matter-antimatter asymmetry in a  matter-antimatter distillation separation process. 

The observational evidence about the antimatter non-abundance in the Universe is supported by the highly homogeneous cosmic microwave background derived from the period of photon decoupling~[2]. This has been used  to argue that the  matter-antimatter domains on a scale smaller than the observable Universe are unlikely~[3]; others see need for further experimental study to confirm this result~[4]. The current small value of the baryon-to-photon ratio is the result of  annihilation of the large matter-antimatter abundance. Considering several observables, the range $\eta \equiv n_{\rm B}/n_\gamma = 5.5 \pm 1.5 \times 10^{-10}$ is established~[5]. The  matter-antimatter asymmetry is expressed by non-zero values of the chemical potentials. Our objective is to  quantify the values of  chemical potentials  required to generate the observed value of $\eta$, and to use this to quantify the electrical charge distillation occurring during hadronization. 

%%%%%%%%%%%%%%%%%%%%%%%%%%%%
%{\bf Equations of State:} 
To compute the thermodynamic properties of the QGP and HG phases, we study the partition functions $\ln{Z_{\rm QGP}}$ and $\ln{Z_{\rm HG}}$ as described in Ref.~[6,7,8]: we use a phenomenological description of QGP equations of state developed in~[7], which agrees well with properties of quantum chromodynamics (QCD) at finite temperature obtained in lattice QCD  for the limit of vanishing particle density~[9], and at finite baryon density~[10]. This approach  involves quantum gases of quarks and gluons with perturbative QCD corrections applied, and a confining vacuum energy-pressure component ${\mathcal{B}}= 0.19$~GeV~fm$^{-3}$. In the HG partition function, we sum partial gas contributions including all hadrons  from Ref.~[11] having mass less than 2~GeV, and apply finite volume corrections~[8].

Our use of partition functions assumes that local thermodynamic equilibrium (LTE) exists. Considering the particle  spectra and yields measured at  the Relativistic Heavy Ion Collider at Brookhaven National Laboratory (RHIC-BNL), it is observed that a thermalization timescale on the order of $\tau_{\rm th}\lesssim 10^{-23}$~s is present in the QGP at hadronization~[6]. The microscopic mechanisms for such rapid thermalization are at present under intense study. We expect this result to be valid qualitatively in the primordial  QGP phase of matter. This then assures us of LTE being present in the evolving Universe. The local chemical equilibrium (LCE) is also approached at RHIC, indicating that this condition also prevails in the early Universe. 

To  apply these experimental results we recall that the size scale $R$  of the radiative Universe  evolve as $R \propto t^{1/2}$. Furthermore, if the expansion is adiabatic and energy conserving, then: $R \propto T^{-1}$. The thermalization timescale is roughly: $\tau_{\rm th}\approx   {1}/{n\, \sigma\, v}$, with $n$ the particle number density, $\sigma$ the cross section for (energy-exchanging) interactions, and $v$ the mean velocity. For a roughly constant value of $\sigma$, we can expect an increase in $\tau_{\rm th}\ $ as we cross from the relativistic QGP ($v \simeq c = 1$) to the HG  phase having strong nonrelativistic components ($v \propto T^{1/2}$). Allowing for the change in relative velocity and  a reduction in density, we expect $\tau_{\rm th}\lesssim 10^{-22}$~s for the HG at $T = 160$~MeV. Since the thermalization time scales as $T^{-7/2}$  in the cooling HG phase, considering $R(T)$ and $v(T)$, its value increases to $\tau_{\rm th}\lesssim 10^{-14}$~s at $T=1$~MeV. At this point, the Universe is already one second old, so our assumption of LTE (and also of LCE) has a large margin of error and is in our opinion fully justified throughout the period of interest.

Chemical equilibration timescales are longer, due to  significantly smaller cross sections, than thermalization timescales. When chemical equilibrium cannot be  maintained in an expanding Universe, we find particle yield freeze-out. Near the phase transformation from HG to QGP, chemical equilibrium for hadrons made of $u,d,s$ quarks is established. Hadron abundance evolution in the early Universe and deviations from the local equilibrium at lower temperature have not yet been studied in great detail. In a baryon symmetric Universe there is a freeze-out of nucleon and antinucleon densities. On the other hand, in a locally asymmetric Universe  baryon annihilation reactions essentially cease at a temperature near 35 MeV, and baryon density at lower temperature is  determined by Universe expansion. However, the antinucleons keep annihilating  until there are none left. 

In order to further the  understanding of all particle and in particular hadron abundances, we obtain the values of chemical potentials describing particle abundances beginning near to QGP hadronization through the nucleosynthesis period. In a system of non-interacting particles, the chemical potential $\mu_i$ of each species $i$ is independent of the chemical potentials of other species, resulting in a large number of free parameters. The many chemical particle interactions occurring in the QGP and HG phases, however, greatly reduce this number.

First, in thermal equilibrium, photons assume the Planck distribution, implying a zero photon chemical potential; i.e., $\mu_\gamma = 0$. Next, for any reaction $\nu_i A_i = 0$, where $\nu_i$ are the reaction equation coefficients of the chemical species $A_i$, chemical equilibrium occurs when $\nu_i \mu_i = 0$, which follows from a minimization of the Gibbs free energy. Because reactions such as $f + \bar{f} \rightleftharpoons 2 \gamma$ are allowed, where $f$ and $\bar{f}$ are a fermion -- antifermion pair, we immediately see that $\mu_f = -\mu_{\bar{f}}$ whenever chemical and thermal equilibrium have been attained.

Furthermore, when the system is chemically equilibrated with respect to weak interactions, we can write down the following relationships~[12]:
\begin{equation}\tag{1}\label{dmul}
\mu_e - \mu_{\nu_e}=\mu_\mu - \mu_{\nu_{\mu}}=\mu_{\tau} - \mu_{\nu_{\tau}}\ 
\equiv\ \Delta \mu_l,
\end{equation}
along with $\mu_u=\mu_d - \Delta \mu_l$, and $\mu_s=\mu_d$, with the chemical equilibrium of hadrons being equal to the sum of the chemical potentials of their constituent quarks; {\it e.g.}, $\, \Sigma^0\,$({\it uds}) has chemical potential $\mu_{\Sigma^0}=\mu_u + \mu_d + \mu_s=3\, \mu_d - \Delta \mu_l$, and the baryochemical potential is:
\begin{equation}\tag{2}
\mu_b=\frac32(\mu_d +\mu_u)=3 \mu_d -\frac32 \Delta \mu_l.
\end{equation}
  
We will use $\mu_d$ in both QGP and HG phases to characterize hadron abundances and note that $\mu_b\simeq 3\mu_d$ in the HG phase. Finally, if the experimentally-favored ``large mixing angle'' solution~[13] is correct, the neutrino oscillations $\nu_e \rightleftharpoons \nu_\mu \rightleftharpoons \nu_\tau$ imply that~[14]: $\mu_{\nu_e} = \mu_{\nu_{\mu}} =\mu_{\nu_{\tau}} \equiv \mu_\nu$, which reduces the number of independent chemical potentials to three, where we assume that in dense matter oscillations occur rapidly. We choose these potentials  to be $\mu_d$, $\mu_e$, and $\mu_\nu$. To determine these in a homogeneous (i.e., single phase) Universe, we seek to satisfy the following three criteria:
\begin{itemize}
\item[i.] {\it Charge neutrality} ($Q = 0$) is required to eliminate  Coulomb energy.  This implies that:
\begin{equation}\tag{3}\label{Q0}
n_Q\equiv \sum_i\, Q_i\, n_i (\mu_i, T)=0, 
\end{equation}
where $Q_i$ and $n_i$ are the charge and number density of species $i$, and the summation is carried out over all species present in the phase. \item[ii.] {\it Net lepton number equals net baryon number} ($L = B$) is required in baryogenesis.  This implies that:
\begin{equation}\tag{4}\label{LB0}
n_L - n_B\equiv \sum_i\, (L_i - B_i)\, n_i (\mu_i, T)=0 ,
\end{equation}
where $L_i$ and $B_i$ are the lepton and baryon numbers of species $i$. This expression can be generalized  for schemes in which $L\ne B$. A modified lepton density would require an increase in lepton chemical potential which is not essential to the understanding of the hadron behavior, given the results we obtain.  \item[iii.] {\it Constant entropy-per-baryon} ($S/B$).  This is the statement that the Universe evolves adiabatically, and is equivalent to:
\begin{equation}\tag{5}\label{SperB}
\frac{s}{n_B}\equiv\frac{\sum_i\, s_i(\mu_i, T)}{\sum_i\, B_i\, n_i(\mu_i, T)}
  ={\rm constant,}
\end{equation}
where $s_i$ is the entropy density of species $i$.
\end{itemize}

The value of $S/B$ can be estimated from the value of $\eta$. In the low temperature era ($T \ll 1$~MeV), the entropy is dominated by photons and nearly massless neutrinos. It is straightforward to compute the entropy densities of these species from the partition function, and then convert $\eta$ to $S/B$ using the known photon number density. In doing so, we obtain a value of $S/B = 4.5^{+1.4}_{-1.1} \times 10^{10}$.

With $S/B = s/n_B$ fixed, Eqs.~(\ref{Q0})--(\ref{SperB}) constitute a system of three coupled, non-linear equations of three unknowns ($\mu_d$, $\mu_e$, and $\mu_\nu$) for a given temperature. These equations were solved numerically using the Levenberg-Marquardt method~[15] to obtain Fig.~1, which shows the values of $\mu_d$, $\mu_e$, and $\mu_\nu$.

The bottom axis of Fig.~1 shows the age of the Universe, while the top axis shows the corresponding temperature. The error bars arise from ``experimental'' uncertainty in the value of $\eta$. Note that the value of the chemical potentials required to generate the current matter-antimatter asymmetry are significantly smaller than 1~eV (horizontal line in Fig.~1) at the time of hadronization.

As the temperature decreases, the value of $\mu_d$ asymptotically approaches weighted one-third of the nucleon mass ($(2m_n-m_p)/3\simeq 313.6$~MeV). This follows because the baryon partition functions are in the classical Boltzmann regime at these temperatures, and the residual baryon number is dominated by the proton and neutron degrees of freedom.

%%%%%%%%%%%%%%%%%%%%%%%%%%%%%%%
\centerline{\includegraphics[width=0.8\columnwidth]{./AllFigs/MuComb.png}}
\noindent {\small Fig. 1: Chemical potentials $\mu_d$, $\mu_e$, and $\mu_\nu$ around the time of the QGP-HG phase transformation.  The error bars arise from the uncertainty in $\eta\,$.  Insert --- expanded view around the phase transformation. Horizontal and vertical lines inserted to guide the eye.} 
%\label{chem_pot} % 1
%%%%%%%%%%%%%%%%%%%%%%%%%%%%%%%%%%%%%%%%%%%%%%%%%%%%%%%


Figure~2 shows the hadronic energy content in the Universe as a function of temperature. The fraction of energy in baryons and antibaryons is roughly 10\% at the QGP-HG phase transformation, but rapidly vanishes, becoming significant again only when the Universe has cooled and enters its atomic era. In today's matter-dominated Universe, the large nucleon rest mass overwhelms completely the background radiation.


During the QGP to HG phase transformation, when both phases co-exist, the macroscopic conditions i.~--~iii.~above must be satisfied for the system as a whole, but may be violated locally. This means that Eqs.~(\ref{Q0})--(\ref{SperB}) are no longer valid within either the QGP or HG phases individually, and that the correct expressions must contain combinations of the two phases.

We therefore parameterize the partition function during the phase transformation as $\ln{Z_{\rm tot}}=f_{\rm HG} \ln{Z_{\rm HG}}+(1 - f_{\rm HG}) \ln{Z_{\rm QGP}}$, where the factor $f_{\rm HG}$ represents the fraction of total phase space occupied by the HG phase. The correct expression analogous to Eq.~(\ref{Q0}) is:
\begin{align}
Q  = & n_Q^{\rm QGP}\, V_{\rm QGP}+n_Q^{\rm HG}\, V_{\rm HG} 
\notag \\ 
\tag{6}
   = & V_{\rm tot} \left[ (1-f_{\rm HG})\, n_Q^{\rm QGP}+f_{\rm HG}\, n_Q^{\rm HG} \right]=0,
\end{align}
where the total volume $V_{\rm tot}$ is irrelevant to the solution. Analogous expressions can be derived for Eqs.~(\ref{LB0}) and (\ref{SperB}). These expressions were used to obtain Fig.~3, which shows the fraction of the total baryon number in the QGP and HG phases as a function of $f_{\rm HG}$.
 
%%%%%%%%%%%%%%%%%%%%%%%%%%%%%%%
\centerline{\includegraphics[width=0.65\columnwidth]{./AllFigs/EUni.png}}
\noindent {\small Fig. 2:  The hadronic energy content of the luminous matter in the Universe as a function of temperature assuming a constant entropy-per-baryon number of $4.5 \times 10^{10}$.} 
%\label{e_frac} 2
%%%%%%%%%%%%%%%%%%%%%%%%%%%%%%%%%%%%%%%%%%%%%%%%%%%%%%%

In Fig.~1, it was assumed that the value of $f_{\rm HG}$ evolved linearly in time and that the duration of the phase transformation was $\tau_h=10\,\mu$s. In reality, these quantities are sensitive to the properties of the equations of state and the dynamics of the phase transformation. Our value of $\tau_h$ is an estimate discussed in~[6], and is consistent with the phase dynamics considered in~[16].

Figure~4 shows the net charge per baryon in each phase as a function of $f_{\rm HG}$. At its onset the small region of HG phase takes on an initial positive charge density, which can be attributed to the proton-neutron bias toward positive charge. As a result, the QGP domain takes on a (initially tiny) negative charge density. The charge asymmetry cannot be avoided, since in general it is impossible to obtain at given values $T,\mu_i$ the vanishing of both $n_Q^{\rm QGP}$ and $n_Q^{\rm HG}$. Such distilled dynamical asymmetry in particle yields was previously explored for strangeness separation and associated strangelet formation~[17,18].

%%%%%%%%%%%%%%%%%%%%%%%%%%%%%%%
\centerline{\includegraphics[width=0.65\columnwidth]{./AllFigs/Mixed.png}}
\noindent {\small Fig. 4: The fraction of baryons in the HG and QGP during phase transformation. The parameter $f_{\rm HG}$ is the fraction of total phase space occupied by the hadronic gas phase.} 
%\label{bary_frac}  3
%%%%%%%%%%%%%%%%%%%%%%%%%%%%%%%%%%%%%%%%%%%%%%%%%%%%%%%

Since the sign of the effect seen in Fig.~4 is the same across the entire hadronization region, the total charge of the remaining QGP domains is ever-increasingly negative and one would expect development of electromagnetic potential, which effectively alters the values of chemical potentials for charged species. It is evident  that the process of charge distillation will have a feed-back effect on the QGP-HG transformation, and that flows of particles will occur that will alter the uniformly small net baryon density~[19]. This can affect (during the phase transformation) any local initial baryon-antibaryon asymmetry, and may also serve as a mechanism for generating magnetic fields in the primordial Universe~[20]. Evaluation of this baryon asymmetry enhancement effect in greater precision  requires  methods of advanced transport theory beyond the scope of this work. We note that a  separation of baryons and antibaryons into domains could maintain a homogeneous zero charge density Universe, a phenomenon which could, {\it e.g.}, play a significant role in amplifying a pre-existent, much smaller net baryon yield.   

%%%%%%%%%%%%%%%%%%%%%%%%%%%%%%%%%%%
\centerline{\includegraphics[width=0.65\columnwidth]{./AllFigs/Chargee.png}}
\noindent {\small Fig. 3: Net charge (including leptons) per net baryon number in the HG and QGP during phase transformation.  Horizontal line at zero inserted to guide the eye.}
%\label{QperB} 4 
%%%%%%%%%%%%%%%%%%%%%%%%%%%%%%%%%%%

In summary, we have determined the chemical potentials required to generate the matter-antimatter asymmetry in the Early Universe. The baryochemical potential reaches its minimal value at the begin of  hadronization of the quark Universe, where $\mu_b=0.33^{+0.11}_{-0.08}$\,eV. Our quantitative results derive from the known entropy content per baryon in the Universe. Other than a  small and most probably negligible increase of entropy in the phase transition of QGP to HG, during nucleosynthesis, and electron-ion recombination, the entropy to baryon ratio is a constant of motion, allowing us to trace out the chemical potentials in the early Universe. There remains some theoretical uncertainty in the behavior of the equations of state at the QGP/HG phase boundary which, along with the time dependence of the mixing fraction $f_HG$ and the dynamics of phase transition or transformation (e.g., expanding HG bubbles, shrinking QGP droplets, and the here uncovered distillation process), will need to be addressed in future work.
%\vskip 0.3cm
%\noindent{\bf Acknowledgments:} Work supported in part by a grant from the U.S. Department of Energy,  DE-FG03-95ER40937. 
 
\footnotetext{\vspace*{-0.5cm}
\begin{enumerate}

\item %1 {Kolb90}
 E. W. Kolb and M. S. Turner, \textit{The Early Universe}
(Addison-Wesley, 1990).

\item % 2 {Fixsen96} 3 arX
 D. J. Fixsen, E. S. Cheng, J. M. Gales, J. C. Mather,
R. A. Shafer, and E. L. Wright,  Astrophys. J. \textbf{473}, 576
(1996).
\item % 3 {Cohen98} 4 arX
A. G. Cohen, A. de Rujula, and S. L. Glashow, Astro-
phys. J. \textbf{495}, 539 (1998).

\item % 4 {Kinney97} 5 arX
 W. H. Kinney, E. W. Kolb, and M. S. Turner, Phys. Rev. Lett. \textbf{79}, 2620 (1997).

\item % 5 {Steigman02}
G. Steigman, Fortsch. Phys. \textbf{50}, 562 (2002).

\item % 6 {Letessier02} 2 arX 
J. Letessier and J. Rafelski, \textit{Hadrons and Quark-Gluon
Plasma} (Cambridge, 2002). 

%\item % 6 arX
%[6] D. Spergel et al., astro-ph/0302209 (2003).


\item  % 7 {Hamieh00} 7 arX
S. Hamieh, J. Letessier, and J. Rafelski, Phys. Rev. C \textbf{62}, 064901 (2000).


\item % 8 {Hagedorn80}  9 arX
R. Hagedorn and J. Rafelski, Phys. Lett. B \textbf{97}, 136 (1980).

%\item % 10 arX
%[10] E. Battaner, Astrophysical Fluid Dynamics (Cambridge, 1996).

\item % 9 {Karsch00}
F. Karsch, E. Laermann, and A. Peikert, Phys. Lett. B \textbf{478}, 447 (2000);\\
Nucl. Phys. B \textbf{605}, 579 (2001).

\item % 10 {Fodor02}
Z. Fodor, S.D. Katz, and K.K. Szabó, hep-lat/0208078,
Phys. Lett. B, in press (2003).

\item %11 {PDG98} % 8 arX
C. Caso et al., Eur. Phys. J. C \textbf{3}, 1 (1998).

\item %12 {Glendenning00}  11 arX
N. K. Glendenning, \textit{Compact Stars: Nuclear Physics,
Particle Physics, and General Relativity} (Springer, 2000).
 
\item % 13 {Ahmad02} 12 arX
 Q. R. Ahmad et al., Phys. Rev. Lett. \textbf{89}, 11302 (2002).
 
\item % 14 {Raffelt02} 13 arX
G. G. Raffelt, hep-ph/0208024 (2002); and: \\
A.D. Dolgov, S.H. Hansen, S. Pastor, S.T. Petcov, G.G. Raffelt, and
D.V. Semikoz, Nucl. Phys. B632, 363 (2002).

\item % 15 {Press92}  14 arX
W. H. Press, S. A. Teukolsky, W. T. Vetterling, and B. P.
Flannery, \textit{Numerical Recipes in C. The Art of Scientific
Computing} (Cambridge, 1992).

\item %16 {Ign01}
J. Ignatius, and D.J. Schwarz, Phys. Rev. Lett. \textbf{86}, 2216
(2001).

\item %17 {Greiner87}  arX 15
C. Greiner, P. Koch, and H. Stocker, Phys. Rev. Lett.
\textbf{58}, 1825 (1987).

\item % 18 {Greiner91} arX 16
C. Greiner and H. Stocker, Phys. Rev. D \textbf{44}, 3517 (1991).

\item % 19 {Witten84} arX 17
E. Witten, Phys. Rev. D \textbf{30}, 272 (1984);

\item %20{Cheng94} arX 18
 B. Cheng and A. V. Olinto, Phys. Rev. D 50, 2421 (1994); and:\\
 G. Sigl, A. V. Olinto, and K. Jedamzik, Phys. Rev. D \textbf{55}, 4582 (1997).

 
\end{enumerate}
}
\end{mdframed}
%%%%%%%%%%%%%%%%%%%%%%%%%%%%%%%%%%%%%%%%%%%
\vskip 0.5cm 

%\vskip 0.3cm
\noindent{\bf Acknowledgements:} Some of the here reprinted arXiv\rq ed work was  supported in part by the U.S. Department of Energy, Office of Science, Office of Nuclear Physics grant DE-FG03-95ER40937.  
I thank Victoria Grossack for her support and editorial help. 

