\markboth{3. Soft Hadron Data Analysis and Interpretation}{Strangeness in QGP: Diaries}
\section{Soft Hadron Data Analysis and Interpretation}
%
\subsection{Statistical hadronization model (SHM)}
We return to the \lq\lq On the Trail of Quark-Gluon-Plasma: Strange Antibaryons in Nuclear Collisions\rq\rq discussion~\cite{Rafelski:1992td} presented at the IlCiocco July 12-24, 1992 Summer School. In this part, the just invented  statistical hadronization model~\cite{Rafelski:1991rh} (SHM) was introduced  in more detail for the first time. Note that SHM, as the name of the model, only appeared several years later.\\

\noindent\textit{Part II of Ref.\cite{Rafelski:1992td} \lq\lq On the Trail of Quark-Gluon-Plasma: Strange Antibaryons in Nuclear Collisions.\rq\rq, for Part I see page~\pageref{Trail1992}:}\\[-0.7cm]
%
\begin{mdframed}[linecolor=gray,roundcorner=12pt,backgroundcolor=Dandelion!15,linewidth=1pt,leftmargin=0cm,rightmargin=0cm,topline=true,bottomline=true,skipabove=12pt]\relax%
%
\centerline{\includegraphics[width=1.0\textwidth]{./AllFigs/StrangeThing.jpg}}
%\caption
\noindent{\small Singing students at the Il Ciocco  Summer School July 12-24, 1992   \lq celebrating\rq\ strangeness and QGP}
%
%%%%%%%%%%%%%%%%%%%%%%%%%%%%%%%%%%%%%%%%%%%%%%%%%%%%%%%%
%
\section*{The central fireball}
\subsection*{Particle spectra}
\label{Trail1992p2}
\addcontentsline{toc}{subsubsection}{Particle spectra}
%%%%%%%%%%%%%%%%%%%%%%%%%%%%%%%%%%%%%%%%%%%%%%%%%%%%%%%%
The favorite scenario as suggested above looks like this: very rapid thermalization of the fireball energy in a central high energy nuclear collisions in which numerous radiation quanta, gluons are formed, followed by glue based formation of strange quark pairs. Next step is the formation of final state hadrons, either in the process of general QGP decomposition or in radiative emission from QGP. It is in this step that particles are formed that are ultimately observed in the experiment. I will now describe how we can use the observations to obtain information about the proto-phase of the reaction. In this I will develop a method which is equally suitable for the case that no QGP has been formed and that the reaction has proceeded by the way of usual hadronic interactions (HG). However, my approach rests on the presence in the reaction of a locally equilibrated fireball of dense nuclear matter, and the question arises if there is evidence today for such a reaction mechanism of strange particle production in relativistic nuclear collisions.
 
How can we argue that the strangeness enhancement originates in a central fireball or another similar high density state? First note that the relative probability to find a composite particle per unit of phase space volume $d^3\vec x \, d^3\vec p/(2\pi)^3$ becomes 
\begin{equation} 
\tag{12} \frac{d^6N}{d^3\vec x \, d^3\vec p /(2\pi)^3}= \prod_{\rm i}
~g_{\rm i}\ \lambda_{\rm i} \ \gamma_{\rm i} \ 
{\rm e}^{-E_{\rm i}/T_{\rm f}} \;.
%\label{eq77}
\end{equation}
The overall normalization of the yield is not easily accomplished, but the relative yields should be well described by Eq.\,(12). We will return to discuss in detail the important pre-exponential factors in next section. We will solely concentrate on the exponential (Boltzmann) term: for a composite particle at energy $E=\sum_i E_i$, Eq.(12) becomes simply a phase space factor times the Boltzmann exponential $e^{-E/T}$ factor.
 
In order to arrive at measured rapidity spectra in the fireball model, an integration of the Boltzmann spectrum, Eq.~(12) over $m_\bot$ is required. With $p_\parallel = m_\bot \sinh (y-y_{\rm f})$, $E=m_\bot \cosh (y-y_{\rm f})$ we have:
\begin{equation}
\tag{13} \frac{dN}{dy} ~=~ C
\int_{m_\bot^{\rm min}}^{\infty}m_\bot\;dm_\bot\;\; m_\bot
\cosh(y-y_{\rm f})~{\rm e}^{-m_\bot\cosh(y-y_{\rm f})/T_{\rm f}}
 \;,
%\label{dNdy}
\end{equation}
with the normalization constant $C$ being dependent on the volume and
other intrinsic properties of the fireball.
%%%ADDED COMPARED TO IL CIOCCO
These spectra are shown in Fig.\,4. %\ref{{FICextra}}

%%%%%%%%%%%%%%%%%%%%%%%%
%\begin{figure}[t] \sidecaption
\centerline{\includegraphics[width=0.75\columnwidth]{./AllFigs/RapiditySpec.png}}
%\caption
\noindent{\small Fig.\;4 The rapidity spectra of particles according to Eq.\,(13) 
%\label{FICextra}
}
%\end{figure}
%%%%%%%%%%%%%%%%%%%%%%%%%%%%%%%%

For the rapidity of the
central fireball $y_{\rm f}$, using simple relativistic kinematics I
obtain:
\begin{equation}
\tag{14} y_{\rm f} ~=~ {\textstyle 1\over2} ~y_{\rm P}~-~{\textstyle
1\over2} ~\ln(A_t/A_{\rm P})
%\label{yf}
\end{equation}
where $y_{\rm P}=5.99$ is the sulphur projectile $(A_{\rm P}=32)$
rapidity at the CERN sulphur beam with energy 200~GeV\,A, and $A_t$ is
the number of participating target nucleons. In a simple geometric model
one finds that the number of participants from the target nucleus is
$A_t=1.32~A_{\rm T}^{1/3} ~A_{\rm P}^{2/3}$ for the asymmetric case such
as the S-Pb collision. This would imply that the central fireball is
shifted from the symmetric position at $y_{\rm f}=3$ to $y_{\rm f}=2.54$. 
Clearly, the assumption of complete stopping can not be made with
certainty, nor can we assume that all matter in the path of the
projectile participates fully in the inelastic reaction forming the
fireball; such effects tend to make the collision system more symmetric,
resulting in a fireball rapidity being closer to the symmetric value
$y_{\rm f}^{\rm s}=3$. 
 
We can easily obtain explicitly the shape of Eq.\,(13) for
$m_\bot=m_0$ or any lower cut off suitable to the experimental conditions
of an experiment - the limit $m_0<<T$ can be done analytically. I wish to
note here the analysis$^{13}$ of the NA36 experiment$^{14}$ which
strongly supports the hypothesis of the central fireball: the rise of the
spectra seen$^{14}$ near to the expected central rapidity region, as well
as indications that the width of the peak in rapidity is in general
agreement with the form Eq.\,(13), which for the heavy $\Lambda$
is of the magnitude 1. Similarly, the $m_\bot$ spectral distribution is
in substantial agreement with the thermal model if a temperature of the
magnitude 205 Mev is used. This last observation is true not only for the
NA36 spectra, it results from a study$^{13}$ of (strange) particle
transverse mass spectra reported for S -- A collisions, with A$\sim 200$
and which suggests $T=215\pm10$ MeV.
 
I wish to stress that relatively narrow strange particle rapidity
distributions are not in contradiction to the relatively wide pion
rapidity spectra: while pions can be produced in many processes and are
easily rescattered on heavier particles, shifting their rapidity, the
strange particles, in particular strange anti-baryons are predominantly
produced in the central region only and are destroyed in rescattering.
Another important issue is the longitudinal flow stemming either from
remembrance of the entrance momentum, or from the expansion of the
compressed matter which than contributes in hadronization and strongly
widens the already relatively wide pion distribution. I thus conclude
that the observation of the central rapidity production of strange
particles and in particular of $\overline{\Lambda}$ by the experiment
NA36 in S -- Pb collisions at 200 GeV\,A collisions strongly supports a
fireball as the source of these particles. Similar behavior was already
reported for the S -- S collisions at 200 GeV\,A by the NA35 
collaborations, but it is hard to ignore the relatively strong 
longitudinal flow visible in these relatively small size collisions. It
will be interesting to see the result of the Pb -- Pb collisions.
 
%%%%%%%%%%%%%%%%%%%%%%%%%%%%%%%%%%%%%%%%%%%%%%%%%%%%%%%%%%%%%%%%%% 
\section*{Counting (strange) particles in hot matter}
\addcontentsline{toc}{subsubsection}{Counting (strange) particles in hot matter}
%%%%%%%%%%%%%%%%%%%%%%%%%%%%%%%%%%%%%%%%%%%%%%%%%%%%%%%%%%%%%%%%%% 
As in an experiment strange particles consisting of a few constituents,
we have to understand how the abundance of composite particles is
governed by the thermal parameters of the fireball. I will now show that
statistical counting rules allow to describe the relative abundances of
strange baryons and antibaryons at fixed $m_\bot$.
 
%%%%%%%%%%%%%%%%%%%%%%%%%%%%%%%%%%%%%%%%%%%%%%%%%%%%%%%%%%%%%%%%%% 
\subsection*{Chemical potentials} 
The statistical variables of the fireball system are aside of the
temperature $T$ the chemical potentials $\mu_i$ of the different
conserved quark flavors $i=u,d,s$. Chemical potentials, as we shall see
in detail, are introduced to allow to set a prescribed abundance of
particles of the kind `$i$'. Akin to temperature which characterizes the
mean energy per particle, the chemical potential is generally related to
the energy expanded in the change of the number of particles. For
example, the cost in energy to replace the particle of kind $i$ by the
particle $j$ is $\mu_j-\mu_i$. The chemical potentials are related to the
so called fugacities $\lambda_i$ in the usual way:
\begin{equation}
\tag{15} \lambda_i=e^{\mu_i/T}
%\label{lammu}
\end{equation} 
Thus the factors in Eq.\,(12) which control the formation of
composite particles in dense matter are: the Boltzmann exponential,
statistical multiplicity factors $g_i$, referring to the degeneracy of
the $i(=u,d,s)$ component, and characterizing also the likelihood of
finding among randomly assembled quarks, the suitable spin-isospin of the
particle; chemical fugacities see Eq.\,(15) which define the
relative abundance of quarks and anti-quarks with $(\lambda_{\bar q} =
\lambda_q^{-1})$. The factors $\gamma_i$ allow for the absence of
chemical equilibrium $(0 \le \gamma_i \le 1)$ for each quark flavor. The
difference between $\gamma$ and $\lambda$ is that $\gamma$ is the same
for {\it both} quarks and antiquarks of the same flavor:
$\gamma_q=\gamma_{\bar q}$. I will assume that, for light flavors, the
$\gamma$-factor is effectively unity, and will consider only the
possibility that strange quarks are not in absolute chemical equilibrium:
$0\le\gamma_{\rm s}\le1$, as is suggested by the dynamical models of
strange flavor production, discussed above in Eq.\,(9).
 
The fugacity of a composite particle is the product of the fugacities of
the components, that is of the fugacities for the `valance' quarks, since
the contributions of the see quark pairs cancel and eventual glue content
has no chemical attribute. Thus we will use as the chemical variables the
strange quark chemical potential $\mu_{\rm s}$ and the light quark
chemical potential $\mu_q$ with:
\begin{align*}
\tag{16}\mu_q&=(\mu_d+\mu_u)/2 \;,\quad\mu_{\rm B}=3\mu_q \;,\\
\tag{17}\delta\mu&=\mu_d-\mu_u \;;
\end{align*}
$\mu_{\rm B}$ is the baryo-chemical potential and $\delta\mu$ describes
the (small) asymmetry in the number of up and down quarks due to the
neutron excess in heavy ion collisions. The magnitude of $\delta\mu$
depends on the $u,d$ asymmetry, that is neutron - proton asymmetry in the
fireball. It has been obtained for both the QGP and HG models of the
fireball. In QGP a simple analytical relation between the ratio of $u$
and $d$ content arises from the perturbative expressions for the quark
density:
\begin{equation}
\tag{18} R_{\rm f}^{\rm QGP}\equiv {{\langle d \rangle- \langle{\bar d} \rangle}
\over{ \langle u\rangle - \langle{\bar u} \rangle}}
={{2-{Z_{\rm f}/A_{\rm f}}}\over {1+{Z_{\rm f}/ A_{\rm f}}}}
={{\mu_d/T\Big(1+\Big({\mu_d\over{\pi T}}\Big)^2\Big)}\over
{\mu_u/T\Big(1+\Big({\mu_u\over{\pi T}}\Big)^2\Big)}}
\simeq{\mu_d\over\mu_u}, 
%\label{delmu}
\end{equation}
where the last equality arises because $(\mu_q/\pi T)^2<<1$. In the tube
model, in which all nucleons in the target in the path of the isospin
symmetric projectile participate in the fireball, $R_{\rm f}$ is 1.08 for
the Sulphur-Tungsten collision and 1.15 for Pb-Pb collisions. From
Eq.\,(18) arises the simple relationship:
\begin{equation}
\tag{19} {\delta\mu\over T} \approx {\mu_{\rm q}\over T}(R_{\rm f}-1)\ .
% \label{deltamu}
\end{equation}
I have not included above the superscript \lq QGP\rq\ since detailed
calculations show$^6$ that in a standard model of HG this results also
approximately holds in the domain of $T,\mu_{\rm B}$ associated with the
source of strange antibaryons. In view of the expected smallness of the
effect of the neutron - proton asymmetry I will mostly ignore $\delta\mu$
here, or employ the theoretical model, Eq.\,(19) to fix its
value.
 
Returning to the discussion of the strange quark chemical potential I
first note that despite the fact that $\mu_{\rm s}$ was introduced into
the nomenclature in the manner described here, a certain confusion is
possible with some recent work using instead the Kaon chemical potential
to characterize strangeness. Since the quark content of Kaons is $q\bar
s$ the chemical potential of Kaons denoted $\mu_{\rm S}=\mu_{\rm
q}-\mu_{\rm s}$. There is considerable advantage in the use of the
strange quark chemical potential, as one can directly compare the
properties of the QGP phase with the HG phase. For example since the
production of strange pairs is not influenced by presence of $u,d,\bar
u,\bar d$ quarks in the QGP phase, independent of the baryon number
content we always have (as long as $\langle s \rangle =\langle \bar s
\rangle$) $\mu_{\rm s}^{\rm QGP}=0$. On the other hand the HG, when
constrained to zero strangeness, implies in general a non vanishing value
of $\mu_{\rm s}$. In consequence, if we demand $\mu_{\rm s}^{\rm HG}=0$,
this establishes a constrain in the $\mu_{\rm q}$ -- $T$ plane, which
turns out to be a simple line akin in its form to the expected phase
transition boundary between HG and QGP. However, the values are very
different$^{15}$: at temperatures of the order 150 MeV the baryochemical
potential is $\mu_{\rm B}\sim900$ MeV, and $\mu_{\rm B}=0$ arises at
temperature $T\sim230$ MeV. The letter value is somewhat dependent on the
number of strange hadronic resonances included, which may still be
undiscovered, or their statistical factors (spin etc) which are either
unknown, or assumed with some degree of confidence in the structure
models of hadrons. A thorough discussion of the values of $\mu_{\rm s}$
is contained in Ref.\,[6] and we refrain here from entering into a more
detailed discussion which requires a rather thorough study of the
properties of HG -- the presented details will fully suffice to
understand the points addressed presently.
 
%%%%%%%%%%%%%%%%%%%%%%%%%%%%%%%%%%%%%%%%%%%%%%%%%%%%%%%%%%%%%%%%%% 
\subsection*{Measuring chemical potentials} 
\addcontentsline{toc}{subsubsection}{Measuring chemical potentials} 
%%%%%%%%%%%%%%%%%%%%%%%%%%%%%%%%%%%%%%%%%%%%%%%%%%%%%%%%%%%%%%%%%% 
All baryons considered have spin $1/2$, but they include spin $3/2$
resonances which become spin $1/2$ states through hadronic decays. This
is implicitly contained in the counting of the particles by taking the
product of the quark spin degeneracies; since in all ratios to be
considered this factor is the same, I shall ignore it, even though a
slight change results$^6$. As the method of measurement distinguishes the
flavor content, I keep explicit the product of $\lambda$-factors;
$\gamma_{\rm s}$ will enter when one compares particles with different
number of strange quarks and antiquarks.
 
When considering hyperons two different charge zero states of different
isospin must be counted: the experimental abundances of $\Lambda$ and
$\overline{\Lambda}$ (I=0) implicitly include, respectively, the
abundance of $\Sigma^0$ and ${\overline{\Sigma^0}}$ (I=1, I$_3$=0),
arising from the decay $\Sigma^0\rightarrow\Lambda^0+\gamma(74$ MeV),
and similarly for ${\overline{\Sigma^0}}$. Thus the true abundances must
be corrected by nearly a factor 2 (exactly 2 when the small difference in
mass $m_\Sigma - m_\Lambda=77$ MeV is neglected). 
 
The method of approach is very simple: I compare {\it spectra} of
particles within overlapping regions of $m_\bot$ and find that in
suitable ratios most statistical and spectral factors cancel allowing to
determine the conditions prevailing in the source. For example the
ratios:
\begin{align*}
\tag{20} R_\Xi&={\overline{\Xi^-}\over {\Xi^-}} =
{{\lambda_d^{-1}\lambda_{\rm s}^{-2}}\over 
{\lambda_d\lambda_{\rm s}^2}}\;,\\
%\label{R1a}
\tag{21} R_\Lambda&={{{{\bar \Lambda}}\over{\Lambda}}} =
{{\lambda_d^{-1}\lambda_u^{-1}\lambda_{\rm s}^{-1}}\over 
{\lambda_d\lambda_u\lambda_{\rm s}}} .
%\label{R1b}
\end{align*}
determine the quark fugacities. Indeed, the cascade and lambda ratios can
easily be related to each other, in a way which shows explicitly the
respective chemical asymmetry factors and strangeness fugacity
dependance. Eqs.(20,21) imply, in view of the definition
Eq.\,(15):
\begin{align*}%\label{R2a}
\tag{22} R_\Lambda&=R_\Xi^2 \cdot e^{2\delta\mu/T}e^{6\mu_{\rm s}/T}\;,\\
\tag{23} R_\Xi&=R_\Lambda^2 \cdot e^{-\delta\mu/T}e^{6\mu_q/T}\;.
%\label{R2b}
\end{align*} 
Eqs.(22,23) are generally valid, irrespective of the state
of the system (HG or QGP). They fix the value of the chemical potentials,
subject to the tacit assumption that the particles considered are emitted
from the central fireball.
 
%%%%%%%%%%%%%%%%%%%%%%%%%%%%%%%%%%%%%%%%%%%%%%%%%%%%%%%%%%%%%%%%%%
\subsection*{Experimental particle ratios} 
\addcontentsline{toc}{subsubsection}{Experimental particle ratios} 
%%%%%%%%%%%%%%%%%%%%%%%%%%%%%%%%%%%%%%%%%%%%%%%%%%%%%%%%%%%%%%%%%%
In order to determine the values of the chemical potentials as enter
Eqs.\,(22,23) we recall that the $\overline{\Xi^-}/ \Xi^-$
ratio has been reported as$^9$:
\begin{equation}
\tag{24} R_\Xi:={\overline{\Xi^-}}/ \Xi^-|_{m_\bot} 
 = 0.39\pm 0.07\ \quad \mbox{ for }
 y\in(2.3,3.0) \mbox{ and } m_\bot>1.72\ \mbox{GeV}.
% \label{cascade}
\end{equation}
Note that, in p--W reactions in the same $(p_\bot,y)$ region, a smaller
value for the $R_\Xi$ ratio, namely $0.27\pm 0.06$, is found. The ${\bar
\Lambda}/\Lambda $ ratio is:
\begin{equation}
\tag{25} R_\Lambda:={{\overline \Lambda}/\Lambda}|_{m_\bot} 
 = 0.13\pm 0.03\ \quad \mbox{ for } 
 y\in(2.3,3.0) \mbox{ and } m_\bot>1.72\ \mbox{GeV}.
%\label{lambda}
\end{equation} 
In Eq.\,(25), corrections were applied to eliminate hyperons
originating from cascade decays, but not those originating from decays of
$\Omega \to \Lambda + \overline{K}$ or $\overline{\Omega} \to
\overline{\Lambda} + K$ which are of little signification for the high 
$m_\bot$ considered here. The ratio $R_\Lambda$ for S--W collisions is
slightly smaller than for p--W collisions in the same kinematic range.
 
From these two results, together with
Eqs.\,(22,23,19) and the value of $u$ -- $d$
asymmetry I obtain the following values of the chemical potentials for
S--W central collisions at 200 GeV A:
\begin{align*}%\label{muq}
\tag{26} {\mu_{\rm q}\over T} & = {\ln R_\Xi/R_\Lambda^2 \over 5.92}
    = 0.53 \pm 0.1\;;\\ 
\tag{27} {\delta\mu\over T} & = {\mu_{\rm q}\over T}(R_{\rm f}-1)
    = 0.042 \pm 0.008\;;\\
%\label{dmu}\\
\tag{28}{\mu_{\rm s}\over T} & = {\ln R_\Lambda/R_\Xi^2\ -0.084\over 6}
    = -0.018 \pm 0.05\;.
%\label{mus}
 \end{align*}
The last result translates into the value $\lambda_{\rm s} = 0.98 \pm
0.05$ for the strange quark fugacity. It turns out that many physical
properties of the fireball (such as e.g. entropy per baryon in the QGP
phase) depend only on the dimensionless values given above, and hence do
not depend on the determination of temperature. Under the assumption that
the transverse mass slope of the produced particles is entirely due to
the thermal motion leads to the temperature $T=210\pm10$ MeV, and
therefore:
\begin{equation} 
\tag{29} \mu_{\rm B} = 340 \pm 20 \mbox{ MeV}\ ,\quad 
 \delta\mu = 9 \pm 2 \mbox{ MeV}\ ,\quad 
 \mu_{\rm s} = -3.8 \pm 10 \mbox{ MeV}\ .
% \label{chempot} 
\end{equation} 
To considerable surprise we see that the strange particle ratios imply
that the strange chemical potential is very small and perfectly
compatible with zero. Another way to note this surprising result is to
observe the square of $R_\Xi$ is nearly equal to $R_\Lambda$. Is this
behavior characteristic for collisions involving the large nuclei at
these high energies, or is this a chancy coincidence? This is in this
field a big question which will have considerable impact on how we
understand the physical processes involving strangeness production in
relativistic heavy ion collisions.
 
%%%%%%%%%%%%%%%%%%%%%%%%%%%%%%%%%%%%%%%%%%%%%%%%%%%%%%%%%%%%%%%%%%
\subsection*{Phase space saturation} 
\addcontentsline{toc}{subsubsection}{Phase space saturation} 
%%%%%%%%%%%%%%%%%%%%%%%%%%%%%%%%%%%%%%%%%%%%%%%%%%%%%%%%%%%%%%%%%%
We turn our attention now to the determination of $\gamma_{\rm s}$. A
complete cancellation of the fugacity factors occurs when I consider the
product of the abundances of baryons and anti-baryons. Furthermore I can
take advantage of the cancellation of the Boltzmann factors by comparing
this product for two different particle kinds, e.g. consider:
\begin{equation}
\tag{30} \Gamma_{\rm s} \equiv \left. {\Xi^-\over\Lambda}
 \cdot {\overline{\Xi^-}\over \overline{\Lambda} 
 }\right\vert_{m_\bot>m_\bot^{\rm cut}} \, .
 % \label{gam1}
\end{equation} 
If the phase space of strangeness, like that of the light flavors, were
fully saturated, the fireball model would imply $\Gamma_{\rm s}=1$.
However, any deviation from absolute chemical equilibrium as expressed by
the factor $\gamma_{\rm s}$ will change the value of $\Gamma_{\rm s}$. 
\begin{equation}
\tag{31} \Gamma_{\rm s}=\gamma_{\rm s}^2\, .
 %\quad \mbox{ no feed-down from hadronic resonances.}
% \label{gam2}
\end{equation}
In principle, the measurement of $\gamma_{\rm s}$ can be done with other
particle ratios, in the absence of resonance feed-down we have
\begin{equation}
 \tag{32} \gamma_{\rm s}^2 = \left. 
 {\Lambda\over p} \cdot {\overline{\Lambda}\over \overline p} 
   \right\vert_{m_\bot>m_\bot^{\rm cut}}
   = \left. 
 {\Xi^-\over\Lambda} \cdot {\overline{\Xi^-} \over
\overline{\Lambda}}
   \right\vert_{m_\perp>m_\perp^{\rm cut}} 
   = \left. 
 {\Omega^-\over 2\Xi^-} \cdot {\overline{\Omega^-} \over 
      2\overline{\Xi^-}}
   \right\vert_{m_\perp>m_\perp^{\rm cut}} \, ,
 %\label{gam3}
\end{equation}
where in the last relation the factors 2 in the denominator correct 
for the spin-3/2 nature of the $\Omega$.
 
In the kinematic domain of Eqs.\,(24,25) the
experimental results reported by the WA85 collaboration are:
\begin{equation}
\tag{33} \frac{\overline{\Xi^-}}{\overline{\Lambda}+\overline{\Sigma^0}}
 = 0.6 \pm 0.2\, , \quad
 \frac{\Xi^-}{\Lambda+\Sigma^0} = 0.20 \pm 0.04\, .
% \label{newa}
\end{equation}
If the mass difference between $\Lambda$ and $\Sigma^0$ is neglected,
this implies in the framework of the thermal model that an equal number
of $\Lambda$'s and $\Sigma^0$'s are produced, such that
\begin{equation}
\tag{34} \frac{\overline{\Xi^-}}{\overline \Lambda}
 = 1.2 \pm 0.4\, , \hspace{1.15cm} 
 \frac{\Xi^-}{\Lambda} = 0.40 \pm 0.08\, .
% \label{new1}
\end{equation}
The fact that the more massive and stranger anti-cascade exceeds at fixed
$m_\bot$ the abundance of the anti-lambda is most striking. These
results are inexplicable in terms of cascade models for the heavy-ion
collision$^{16}$. The relative yield of $\overline{\Xi^-}$ appears to be
5 times greater than seen in the $p$--$p$ ISR experiment$^{8}$ and all
other values reported in the literature$^9$.
 
Combining the experimental result Eq.\,(34) with 
Eqs.\,(30,31), we find the value
\begin{equation}
\tag{35} \gamma_{\rm s}=0.7 \pm 0.1\, .
% \label{gams}
\end{equation} 
A more detailed discussion$^6$ including the resonance decays leaves
this result intact, and only if significant flow component is assumed
such that the fireball temperature drops to zero, there is an increase in
$\gamma_{\rm s}$ to a value near 0.9. However such a flow model is
somewhat inconsistent with the current understanding of the
hadronization, as the observed value $\mu_{\rm s}=0$ and the requirement
for strangeness conservation.

%%%%%%%%%%%%%%%%%%%%%%%%%%%%%%%%%%%%%%%%%%%%%%%%%%%%%%%%%%%%%%%%%%
\section*{Entropy of the fireball} 
\addcontentsline{toc}{subsubsection}{Entropy of the fireball} 
%%%%%%%%%%%%%%%%%%%%%%%%%%%%%%%%%%%%%%%%%%%%%%%%%%%%%%%%%%%%%%%%%%
The properties of the HG and QGP fireballs are considerably different in
particular with regard to the entropy content. Both states are easily
distinguishable in the regime of values $\mu_{\rm B},T$ of interest here.
the specific entropy per baryon in the hadronic gas is ${\cal S}^{\rm
HG}/{\cal B} =21.5\pm1.5$ for $T=215$ MeV and $\mu_{\rm B}=340$ MeV. 
This is less than half of the QGP based expectations ${\cal S}^{\rm
QGP}/{\cal B} =50\pm5$ and which are as shown somewhat dependent on the
value of the QCD parameters. Clearly the difference is considerable in
terms of experimental sensitivity, as it implies different final state
multiplicity. Note that charged particle multiplicity {\it above 600} in
the central region has been seen$^{17}$ in heavy ion collisions
corresponding possibly to a total particle multiplicity of about 1,000,
as required in the QGP scenario for the central fireball we described
above.
 
In order to study the relationship between the specific entropy and
particle multiplicity it is best to consider the quantity:
\begin{equation}
\tag{36} D_{\rm Q}= {{N^+-N^-}\over{N^++N^-}} \ ,
\end{equation}
since on the experimental side it is straightforward to measure it, while
on theoretically it is closely related to the yield of baryon number per
pion. Indeed, if only pion number $N_\pi$ and nucleon number $N$ is
considered:
\begin{equation}
\tag{37} D_{\rm Q}^{\pi,N} = 0.75 {N\over N_\pi} {1\over 1+0.75 N/N_\pi}
\end{equation}
where $N=2p$ is the total number of nucleons in the source, twice as
large as the proton number $p$, and $N_\pi=3N_{\pi^-}$ is the total
number of pions, which includes the three different charge components.

I note that in the central region of 200 GeV A S-Ag interactions$^{17}$
with the \lq central\rq\ trigger being the requirement for the total charged
multiplicity $> 300$ all up to date scanned (15) events yield $D_{\rm
Q}(\eta=2.5\pm0.5)=0.088\pm0.007$. It is rather simple to find in a
theoretical model that the specific entropy per baryon ${\cal S/B}\propto
D_{\rm Q}^{-1}$ for here interesting conditions, with the proportionality
constant being rather $T,\mu_{\rm B}$ independent. This result is rather
model independent as long as there is only limited production of entropy
in the hadronization. ${\cal S/B}\cdot D_{\rm Q}$ is essentially the
nearly constant entropy content per pion, with strange and heavy
resonance effects largely balancing out. The importance of this
observation is that for the observed value of $D_{\rm Q}=0.88\pm0.007$ we
find an entropy content${\cal S/B}=50\pm5$ as would be expected from the
QGP fireball. 

On the other hand in the conventional model of HG the relationship
between $D_{\rm Q}$ and $\mu_{\rm B}$ is found to be$^{6}$:
\begin{equation}
\tag{38}\hfil D_{\rm Q}= {\mu_{\rm B}\over\mbox{1.3 GeV}}\ \mbox{for }
\mu_{\rm B}<0.6 \mbox{ GeV}\ .
\end{equation}
where the temperature for each $\mu_{\rm B}$ is selected to assure
strangeness conservation. We see that the experimentally compatible value
$\mu_{\rm B}=340$ MeV implies a multiplicity ratio $D_{\rm Q}=0.26$,
which is incompatible with the data$^{17}$ of the experiment EMU05.
 
The source of the strange antibaryons is not a simple hadronic gas. The
source has entropy per baryon enhancement by the factor two expected from
the QGP equations of state.
 
%%%%%%%%%%%%%%%%%%%%%%%%%%%%%%%%%%%%%%%%%%%%%%%%%%%%%%%%%%%%%%%%%%%% 
\section*{Final Remarks}
I have shown that in studying the formation of rare strange particles,
one can obtain very precise and detailed information about the highly
excited nuclear matter formed in relativistic heavy ion collisions. Full
event characterization with considerable precision is needed to fix the
parameters of the system essential to a basic understanding of the state
of matter formed. Measurement of excitation functions for quantities such
as $\gamma_{\rm s}$ and possibly $\mu_{\rm s}$ would lead to a definitive
understanding of the high density source of these particles. Noteworthy
is the fact that the entropy content of the central interaction region
seems to favor a high entropy phase with properties as expected of a QGP
source of the observed antibaryons.
 
The observed enhancement of (relative) production rates of {\it
multi-}strange {\it anti-}baryons $\overline \Xi$ in nuclear collisions,
in particular at central rapidity and at highest transverse masses,
cannot be obtained so far in microscopic reaction models. After some
considerable effort to the contrary$^6$ I still cannot imagine how to
interpret these data other than in terms of a explosively evaporating
drop of quark-gluon plasma, in particular considering the substantial
hadronic multiplicity seen. Thus my tentative conclusion first put
forward a year ago$^3$ still holds: the source of the high $m_\bot$
centrally produced anti-cascades is the primordial and/or explosive QGP
state of matter with $T\simeq 215\pm10$ MeV and $\mu_{\rm B}\simeq
340\pm20$ MeV. 
%
\footnotetext{\vspace*{-0.5cm}\begin{itemize} 
\item[1]
J. Rafelski, Phys. Rep. C \textbf{88} 331 (1982) (not cited in this text fragment)
\item[2]
J.~Rafelski and M.~Danos, Phys. Lett. B \textbf{192} 432 (1987) (not cited in this text fragment)
\item[3]
J. Rafelski {\it Phys. Lett.} B262:333 (1991); Nucl. Phys. A \textbf{544} 279c (1992)
\item[4]
P.~Koch, B.~M\"uller and J.~Rafelski, Phys. Rep. C \textbf{142} 167 (1986) (not cited in this text fragment)
\item[5]
H.C.~Eggers and J.~Rafelski, Int. Journal of Mod. Phys. A \textbf{6} 1067 
(1991) (not cited in this text fragment)
\item[6]
J.~Letessier, A.~Tounsi, U.~Heinz, J.~Sollfrank and J.~Rafelski, {\it
Strangeness Conservation in Hot Fireballs} Preprint
Paris PAR/LPTHE/92-27, Regensburg TPR-92-28, Arizona 
AZPH-TH/92-23, 1992 (published: Phys. Rev. D \textbf{51} 3408 (1995))
\item[7]
J.~Rafelski and B.~M\"uller, Phys. Rev. Lett. \textbf{48} 1066 (1982); and \textbf{56} 2334(E) (1986) (not cited in this text fragment)
\item[8]
T.~\AA kesson et al. [ISR-Axial Field Spect. Collab.], Nucl. Phys. B \textbf{246} 1 (1984)
\item[9]
E. Quercigh, this volume; S. Abatzis {\it et~al}., Phys. Lett. 
B \textbf{270} 123 (1991)
\item[10]
J. Zimanyi, this volume~\cite{Gutbrod:1993rp}, pp.243-270 (not cited in this text fragment)
\item[11]
S.A.~Chin, Phys. Lett. B \textbf{78} 552 (1978) (not cited in this text fragment)
\item[12]
P. Koch, B. M\"uller and J. Rafelski, Z. Physik A \textbf{324} 3642 (1986) (not cited in this text fragment)
%%%%%%%%%%%%%%%%%%%%%%%%%%%%ABOVE ALREADY PRESENTED IN EARLIER QUOTATION
\item[13]
J. Rafelski, H. Rafelski and M. Danos, Phys. Lett. B: in press
(1992) (published: \textbf{294} 131 (1992))
\item[14]
R. Zybert and E. Judd, this volume, p.545-553; E. Andersen {\it et~al}. Phys. Lett B in press (1992) (published: \textbf{316} 603 (1993) 
\item[15]
J. Letessier, A. Tounsi and J. Rafelski, Phys. Lett. B: in press
(1992) (published: \textbf{292} 417 (1992))
\item[16]
L. Csernai, N.S. Amelin, E.F. Staubo and D. Strottman, Bergen University
Report 1991-14, table 4 and private communication
\item[17]
Y. Takahashi et al, CERN-EMU 05 collaboration, private communication (added for this republication: the results here quoted appear in\href{https://cds.cern.ch/record/295506/files/SC00000469.pdf}{CERN/SPSLC 93-18, Fig. 28, p.39} \lq\lq A Research Proposal submitted to the CERN SPSC For the Lead-Beam Experiments,Isospin Correlations in High Energy Pb+Pb Interactions,\rq\rq\ Submitted by the EMUO5 Collaboration, Y. Takahashi et al.)
\end{itemize}
}
\end{mdframed}
\vskip 0.5cm
%%%%%%%%%%%%%%%%%%%%%%%%%%%%%%%%%%%%%%%%%%%%%%%%%%%%%%%%%%%%%%%

\subsubsection{Proposal to STAR collaboration (continued)}
\noindent\textit{In the following we return to the final part of the report I prepared in 1998 to represent my potential input into the RHIC-STAR collaboration work. This segement addresses the particle ratio method for determining the SHM model parameters:} \\[-0.7cm]
%
\begin{mdframed}[linecolor=gray,roundcorner=12pt,backgroundcolor=Dandelion!15,linewidth=1pt,leftmargin=0cm,rightmargin=0cm,topline=true,bottomline=true,skipabove=12pt]\relax%
%%%%%%%%%%%%%%%%%%%%%%%%%
%%%%%%%%%%%%%%%%%%%%%%%%%%%%%%%%
\label{SHM-STAR}
\textbf{Relative particle yields}\\
\addcontentsline{toc}{subsubsection}{Relative particle yields and SHM parameters}
%%%%%%%%%%%%%%%%%%%%%%%%%%%%%%%%
Precise measurement of the multistrange particle
production is being used to determine the chemical properties of
the source$^{1,2,3}$, and we expect to be able to
also perform a similar analysis at RHIC. 

The relative number of particles of same type 
emitted at a given instance by a locally equilibrated, 
thermal hot source is obtained by
noting that the probability to find all the $j$-components
contained within the $i$-th emitted particle is
\begin{equation}\tag{1}\label{abund}
N_i\propto \gamma_{\rm s}^k\prod_{j\in i}\lambda_je^{-E_j/T}\,,
\end{equation}
and we note that the total energy and fugacity of the particle is:
\begin{equation}\tag{2}
E_i=\sum_{j\in i}E_j,\qquad \lambda_i=\prod_{j\in i}\lambda_j\,.
\end{equation}
The strangeness occupancy $\gamma_{\rm s}$ enters
Eq.\,(\ref{abund}) with power $k$, which equals the number of
strange and antistrange quarks in the hadron $i$. 
With $E_i=\sqrt{m_i^2+p^2}=\sqrt{m_i^2+p_\bot^2}\cosh y $ 
the transverse momentum range 
as constrained in the experiment (here $p_\bot>0.6 $ GeV)
and taking central rapidity region $y\simeq 0$, is integrated 
over to obtain the relative strengths of particles produced. All
hadronic resonances are allowed to disintegrate 
in order to obtain the final relative multiplicity of `stable'
particles required to form the observed particle ratios. 

As we can see, the relative abundance of particles emerging from
the thermal fireball is controlled the chemical (particle
abundance) 
parameters, the particle fugacities which allow to
conserve flavor quantum numbers. The fugacity of each hadronic 
particle species is the product of the valence quark fugacities, 
thus, for example, the hyperons have the fugacity 
$\lambda_{\rm Y}=\lambda_{\rm u}\lambda_{\rm d}\lambda_{\rm s}$.
Fugacities are related to the chemical potentials $\mu_i$ by:
\begin{equation}\tag{3}
\lambda_{i} =e^{\mu_{i}/T}\,,\quad 
\lambda_{\bar{\imath}}=\lambda_i^{-1}\qquad 
i={u,\,d,\,s}\, . \label{lam}
 \end{equation}
Therefore, the chemical potentials for particles and 
antiparticles are opposite to each other, provided that there is
complete chemical equilibrium, and if not, that the deviation from
the full phase space occupancy is accounted for by introducing a 
 non-equilibrium chemical parameter $\gamma$.
 
In many applications it is sufficient to combine the
light quarks into one fugacity 
\begin{equation}\tag{4}
\lambda_{\rm q}^2\equiv\lambda_{\rm d}\lambda_{\rm u}\,,\quad 
\mu_{\rm q}=({\mu_{\rm u}+\mu_{\rm d}})/2\,. \end{equation}

Since a wealth of experimental data can be described with just a
few model parameters, this leaves within this approach a
considerable predictive power and a strong check of the internal
consistency. In the directly hadronizing 
off-equilibrium QGP-fireball there are 5
particle multiplicity parameters characterizing all particle
yields. aside of the usual temperature $T$ and
$\lambda_q,\lambda_s$ (we expect $\lambda_{\rm
s}=1$ because of strangeness conservation in the QGP phase) it is
advisable to introduce two
particle abundance non-equilibrium parameters: the strangeness 
occupancy $\gamma_{s}$ and the ratio $R^{\rm s}_{\rm
C}$, of meson to baryon phase space abundance$^{2}$. The
last of these parameters is related to the mechanism 
governing the final state hadronization process. It does not appear
in any if we only consider baryon yields.

The ratios of strange antibaryons to strange baryons {\it
of same particle type\/}: 
\begin{equation}\tag{5}R_\Lambda=\overline{\Lambda}/\Lambda\,,\quad 
R_\Xi=\overline{\Xi}/\Xi\quad \mbox{and}\quad
R_\Omega=\overline{\Omega}/\Omega\,,\end{equation}
are in our approach simple functions of the quark 
fugacities. For example one has specifically
 \begin{align}\tag{6}
 R_\Xi &= {{\overline{\Xi^-}}\over {\Xi^-}} =
 {{\lambda_{\rm d}^{-1} \lambda_{\rm s}^{-2}} \over
 {\lambda_{\rm d} \lambda_{\rm s}^2}} \, ,
\qquad \label{ratioL}\\
\tag{7} R_\Lambda &= {\overline{\Lambda}\over \Lambda} =
{{\lambda_{\rm d}^{-1} \lambda_{\rm u}^{-1}
    \lambda_{\rm s}^{-1}} \over
 {\lambda_{\rm d} \lambda_{\rm u} \lambda_{\rm s}}} \, . 
\label{ratio}
 \end{align}

Only the ratios between antibaryons with
different strange quark content are dependent on
the strangeness saturation factor $\gamma_s$. 
At fixed $m_\bot$ and up to cascading corrections 
a complete cancelation of the fugacity and Boltzmann 
factors occurs when we form the product of the abundances 
of baryons and antibaryons, comparing this product 
for two different particle kinds$^{1}$, { e.g.}:
 \begin{equation}\tag{8}
 \left. {\Xi^-\over\Lambda}
 \cdot {\overline{\Xi^-}\over \overline{\Lambda}
 }\right\vert_{m_\perp>m_\perp^{\rm cut}} =\gamma_{\rm s}^2 \,, 
 \label{gam1}
 \end{equation}
where we neglected resonance feed-down contribution in first 
approximation, which are of course considered in numerical studies$^{2}$. Similarly we have 
\begin{equation}\tag{9}
 \gamma_{\rm s}^2 = \left.
 {\Lambda\over p} \cdot {\overline{\Lambda}\over \overline p} 
 \right\vert_{m_\perp>m_\perp^{\rm cut}}
%   = \left.
% {\Xi^-\over\Lambda} \cdot {\overline{\Xi^-} \over
% \overline{\Lambda}}
% \right\vert_{m_\perp>m_\perp^{\rm cut}}
   = \left.
 {\Omega^-\over 2\Xi^-} \cdot {\overline{\Omega^-} \over 
     2\overline{\Xi^-}}  
  \right\vert_{m_\perp>m_\perp^{\rm cut}} \, ,
\label{gam3}
 \end{equation}
where in the last relation the factors 2 in the denominator correct
for the spin-3/2 nature of the $\Omega$.


The evaluation of the resonance 
decay effect is actually not simple, since resonances at
different momenta and rapidities contribute to a given daughter 
particle $m_\bot$. The measurements sum 
the $m_\bot$ distributions with $m_\perp \geq m_\perp^{\rm cut}$
and it is convenient to consider this integrated abundance for
particle `i' at a given (central) rapidity $y$:
 \begin{equation}\tag{10}
 \left. {dN_i \over dy} \right\vert_{m_\perp \geq
     m_\perp^{\rm cut}}
 = \int_{m^{\rm cut}_\perp}^\infty dm_\perp^2
 \left\{ {dN_i^{0}(T) \over dy\, dm_\perp^2} +
 \sum_R b_{R\to i} {dN_i^R(T) \over dy\, dm_\perp^2} \right\}\, ,
\label{reso}
 \end{equation}
showing the direct `0' contribution and the daughter
contribution from decays into the observed channel $i$) 
of resonances $R\to i$\,, with branching ratio $b_{R\to i}$\,, see
Ref.[2]. Extracting the degeneracy factors
and fugacities of the decaying resonances, we write shortly 
\begin{equation}\tag{11}
 N^R_i \equiv \gamma_R \lambda_R \tilde N^R_i\,,
 \label{ntilde}
 \end{equation}
and imply that particles of same quantum numbers are comprised in
each $N^R_i$. Here $\gamma_R $ is the complete non-equilibrium
factor of hadron (family) $R$. Between particles and anti-
particles 
we have the relation
 \begin{equation}\tag{12}
N_{\bar i}^{\bar R} = \gamma_R\, \lambda_R^{-1}\, {\tilde N}_i^R 
  = \lambda_R^{-2}\, N_i^R \, .
 \label{ntildea}
 \end{equation}
Thus the above considered particle ratios now become:
 \begin{align}\tag{13}
 R_\Xi &= 
\left. {\overline{\Xi^-} \over \Xi^-}
\right\vert_{m_\perp\geq m_\perp^{\rm cut}}
  &= {\gamma_{\rm s}^2 \lambda_{\rm q}^{-1} 
\lambda_{\rm s}^{-2} 
 {\tilde N}_\Xi^{\Xi^*} +
  \gamma_{\rm s}^3 \lambda_{\rm s}^{-3}
  {\tilde N}_\Xi^{\Omega^*} \over
  \gamma_{\rm s}^2 \lambda_{\rm q} \lambda_{\rm s}^2 
 {\tilde N}_\Xi^{\Xi^*} +
  \gamma_{\rm s}^3 \lambda_{\rm s}^3
  {\tilde N}_\Xi^{\Omega^*} }\ ,
 \label{rxi}\\
\tag{14} R_\Lambda &= 
\left. \hphantom{^-}
  {\overline{\Lambda} \over \Lambda}
  \right\vert_{m_\perp\geq m_\perp^{\rm cut}}
&=
{  \lambda_{\rm q}^{-3}{\tilde N}_\Lambda^{N^*}+ 
\gamma_{\rm s} \lambda_{\rm q}^{-2} \lambda_{\rm s}^{-1}  
 {\tilde N}_\Lambda^{Y^*} +
 \gamma_{\rm s}^2 \lambda_{\rm q}^{-1} \lambda_{\rm s}^{-2} 
 {\tilde N}_\Lambda^{\Xi^*}
\over
  \lambda_{\rm q}^3 {\tilde N}_\Lambda^{N^*} +
  \gamma_{\rm s} \lambda_{\rm q}^2 \lambda_{\rm s} 
 {\tilde N}_\Lambda^{Y^*} +
  \gamma_{\rm s}^2 \lambda_{\rm q} \lambda_{\rm s}^2 
 {\tilde N}_\Lambda^{\Xi^*}
}\, , \label{rlam}\\
\tag{15} R_{\rm s} &= 
\left. {\Xi^- \over \Lambda}
  \right\vert_{m_\perp\geq m_\perp^{\rm cut}}
&=
{  \gamma_{\rm s}^2 \lambda_{\rm q} \lambda_{\rm s}^2 
   {\tilde N}_\Xi^{\Xi^*} +
  \gamma_{\rm s}^3 \lambda_{\rm s}^3
  {\tilde N}_\Xi^{\Omega^*}
\over
  \lambda_{\rm q}^3{\tilde N}_\Lambda^{N^*}+
  \gamma_{\rm s} \lambda_{\rm q}^2 \lambda_{\rm s} 
 {\tilde N}_\Lambda^{Y^*} +
  \gamma_{\rm s}^2 \lambda_{\rm q} \lambda_{\rm s}^2 
 {\tilde N}_\Lambda^{\Xi^*}
}\, .\label{rs}
 \end{align}
$\tilde N_\Lambda^{Y^*}$ contains also (in fact as its most
important contribution) the electromagnetic decay $\Sigma^0 \to
\Lambda + \gamma$.

This approach allows to compute the relative
strengths of strange (anti)baryons both in case of 
surface emission and equilibrium disintegration of a particle gas
since the phase space occupancies are in both cases properly
accounted for by Eq.\,(\ref{abund}). The transverse flow phenomena
enter in a similar fashion into particles of comparable mass and
are not expected to
influence particle ratios. Therefore detailed information about the
chemical and thermal composition of the particle source is derived,
provided that precise input particle abundances are used in the 
analysis. Presence of longitudinal flow in the dense matter from
which observed particles are emitted has no impact on the relative
particle ratios considered here, but it will need to be considered
for full evaluation of the dynamics of hadronic particle
production.

\footnotetext{\vspace*{-0.5cm}
\begin{enumerate}
\item%1{Raf91} 
J. Rafelski, {\it Phys. Lett.} {\bf B 262}, 333 (1991);
{\it Nucl. Phys.} {\bf A544}, 279c (1992).
 
\item%2{analyze1}
J. Letessier, A. Tounsi, U. Heinz, J. Sollfrank and J. Rafelski,
{\it Phys.\ Rev.} {\bf D51}, 3408 (1995);\\
J. Letessier, J. Rafelski and A. Tounsi, {\it Phys. Lett.} {\bf
B321}, 394 (1994); {\bf B323}, 393 (1994); {\bf B333}, 484 (1994);
{\bf B390}, 363 (1997); {\bf B 410}, (1997) 315. 
 
\item%3{acta96} 
J. Rafelski, J. Letessier and A. Tounsi,
{\it Acta Phys. Pol.} {\bf B27}, 1035 (1996).

\end{enumerate}
}
\end{mdframed}
\vskip 0.5cm


\subsection{Fireball of QGP in Pb-Pb collisions at CERN-SPS}\label{PRL2000}

\subsubsection{Fit to data and bulk fireball properties}

In the Winter 1998/9 I completed in collaboration with Jean Letessier an analysis of CERN-SPS Pb--Pb 158$A$\;GeV particle production data. On 8 March 1999 the PRL editorial office acknowledged the submission of our manuscript LC7284. The manuscript in v1, v2 and final v3 format is today available on arXiv~\cite{Rafelski:1999xv} as manuscript nucl-th/9903018. To best of my knowledge there was/is nothing wrong with this unpublished work in every version. 

Below, after this work is presented in its v3 format, I will show the pertinent correspondence with PRL. The reader should remember that had our paper been published  in the Summer 1999, this would have been a strong support of the CERN QGP announcement. Thus rejection of the publication of our work was of essence for those at CERN who were in opposition to QGP CERN announcement, see page \pageref{Heinz2000}. We recall that one of the coauthors of the  QGP discovery at CERN document was just that person, and that this document contains quite flawed theoretical phrases, compare Sec.\ref{Berk2000}. At the time in 1999/2000 this individual was also the divisional associate editor of Physical Review Letters (PRL): he very likely he was consulted by editors with regard to choices of referees, and he was the judge who terminated the publication process.\\
 
\noindent \textit{The manuscript LC7284 was received 8 March 1999 by editors of PRL. The process terminated on 13 January 2000, 4 weeks before CERN announces QGP discovery. The following shows the unpublished PRL manuscript~\cite{Rafelski:1999xv} {\bf LC7284} in final version v3, the other two versions can be found on arXiv:}\\[-0.7cm]
%
\begin{mdframed}[linecolor=gray,roundcorner=12pt,backgroundcolor=Dandelion!15,linewidth=1pt,leftmargin=0cm,rightmargin=0cm,topline=true,bottomline=true,skipabove=12pt]\relax%
%
\begin{center}
{\large {\bf On hadron production in\\[0.2cm] Pb-Pb collisions at 158$A$\;GeV\\[0.2cm]}}
\end{center}

{\bf Abstract:} \textit{A Fermi statistical model analysis of hadron abundances and spectra obtained in several relativistic heavy ion collision experiments is utilized to characterize a particle source. Properties consistent with a disintegrating, hadron evaporating, deconfined quark-gluon plasma phase fireball are obtained, with a baryochemical potential $\mu_{B}=200$--210\,MeV, and a temperature $T_f\simeq 140$--150\,MeV, significantly below previous expectations.}

Discovery and study of quark-gluon plasma (QGP), a state consisting of mobile, color charged quarks and gluons, is the objective of the relativistic heavy ion research program$^1$ underway at Brookhaven National Laboratory, New York and at CERN, Geneva. Thermalization of the constituents of the deconfined phase created in high energy large nuclei collisions is a well working hypothesis, as we shall see. The multi-particle production processes in 158$A$ GeV Pb--Pb collisions carried out at CERN-SPS will be analyzed in this paper, using the principles of the statistical Fermi model$^2$: strongly interacting particles are produced with a probability commensurate with the size of accessible phase space. Since the last comprehensive review of such analysis has appeared$^3$, the Pb-beam experimental results became available, and model improvements have occurred: we implement refinements in the phase space weights that allow a full characterization of the chemical non-equilibria with respect to strange and light quark flavor abundances$^{4,5}$. Consideration of the light quark chemical non-equilibrium is necessary in order to arrive at a consistent interpretation of the experimental results of both the wide acceptance NA49-experiment$^{6,7,8,9,10}$ and central rapidity (multi)strange (anti)baryon WA97-experiment$^{11,12,13}$.

We further consider here the influence of collective matter flow on $m_\bot$-particle spectra and particle multiplicities obtained in a limited phase space domain. The different flow schemes have been described before$^{14}$. We adopt a radial expansion model and consider the causally disconnected domains of the dense matter fireball to be synchronized by the instant of collision. We subsume that the particle (chemical) freeze-out occurs at the surface of the fireball, simultaneously in the CM frame, but not necessarily within a short instant of CM-time. Properties of the dense fireball as determined in this approach offer clear evidence that a QGP disintegrates at $T_f\simeq$\,144\,MeV, corresponding to energy density$^{15}$ $\varepsilon=\cal O$(0.5) GeV/fm$^3$. Our initial chemical non-equilibrium results without flow have been suggestive that this is the case$^{16}$, showing a reduction of the chemical freeze-out temperature from $T_f=180$\,MeV$^{17}$; an earlier analysis could not exclude yet higher hadron formation temperature of 270\,MeV$^{18}$. 

The here developed model offers a natural understanding of the systematic behavior of the $m_\bot$-slopes which differs from other interpretations. The near equality of (inverse) slopes of nearly all strange baryons and antibaryons arises here by means of the sudden hadronization at the surface of an exploding QGP fireball. In the hadron based microscopic simulations this behavior of $m_\bot$-slopes can also arise allowing for particle-dependent freeze-out times$^{19}$.

In the analysis of hadron spectra we employ methods developed in analysis of the lighter 200$A$ GeV S--Au/W/Pb system$^{5}$, where the description of the phase space accessible to a hadronic particle in terms of the parameters we employ is given. Even though we use six parameters to characterize the hadron phase space at chemical freeze-out, there are only two truly unknown properties: the chemical freeze-out temperature $T_{f}$ and light quark fugacity $\lambda_q$\, (or equivalently, the baryochemical potential $\mu_\mathrm{B}=3\,T_{f}\ln \lambda_q$) -- we recall that the parameters $\gamma_i,\,i=q,s$ controls overall abundance of quark pairs, while $\lambda_i$ controls the difference between quarks and anti-quarks of given flavor. The four other parameters are not arbitrary, and we could have used their tacit and/or computed values:\\ 1) the strange quark fugacity $\lambda_s$ is usually fixed by the requirement that strangeness balances$^{4}$ $\langle s-{\bar s}\rangle=0$. The Coulomb distortion of the strange quark phase space plays an important role in the understanding of this constraint for Pb--Pb collisions$^{16}$, see Eq.\,(\ref{lamQ});\\ 2) strange quark phase space occupancy $\gamma_s$ can be computed within the established kinetic theory framework for strangeness production$^{20,21}$;\\ 3) the tacitly assumed equilibrium phase space occupancy of light quarks $\gamma_q=1$\,; and \\ 4) assumed collective expansion to proceed at the relativistic sound velocity$^{21}$, $v_c=1/\sqrt{3}$.\\ However, the rich particle data basis allows us to find from experiment the actual values of these four parameters, allowing to confront the theoretical results and/or hypothesis with experiment. 

The value of $\lambda_s$ we obtain from the strangeness conservation condition $\langle s-{\bar s}\rangle=0$\ in QGP is, to a very good approximation$^{16}$:
\begin{equation}\label{lamQ}
\tag{1}\tilde\lambda_s\equiv \lambda_s \lambda_{\rm Q}^{1/3}=1\,,\qquad
\lambda_{\rm Q}\equiv
\frac{\int_{R_{\rm f}} d^3r e^{\frac V{T}} } {\int_{R_{\rm f}} d^3r}\,.
\end{equation}
 $\lambda_{\rm Q}<1$ expresses the Coulomb deformation of strange quark phase space. This effect is relevant in central Pb--Pb interactions, but not in S--Au/W/Pb reactions. $\lambda_{\rm Q}$ is not a fugacity that can be adjusted to satisfy a chemical condition, since consideration of $\lambda_i,\ i=u,d,s$ exhausts all available chemical balance conditions for the abundances of hadronic particles. The subscript ${R_{f}}$ in Eq.\,(\ref{lamQ}) reminds us that the classically allowed region within the dense matter fireball is included in the integration over the level density. Choosing $R_{\rm f}=8$\,fm, $T=140$\,MeV, $m_s=200$\,MeV (value of $\gamma_s$ is practically irrelevant), for $Z_{\rm f}=150$ the value is $\lambda_s=1.10$\,.

In order to interpret particle abundances measured in a restricted phase space domain, we study abundance ratios involving what we call compatible hadrons: these are particles likely to be impacted in a similar fashion by the not well understood collective flow dynamics in the fireball. The available particle yields are listed in table~1, top section from the experiment WA97 for $p_\bot>0.7$ GeV within a narrow $\Delta y=0.5$ central rapidity window. Further below are shown results from the large acceptance experiment NA49, extrapolated to full $4\pi$ phase space coverage. There are 15 experimental results. The total error $\chi^2_{\rm T}\equiv\sum_j({R_{\rm th}^j-R_{\rm exp}^j})^2/ ({{\Delta R _{\rm exp}^j}})^2$ for the four theoretical columns is shown at the bottom of table~1 along with the number of data points `$N$', parameters `$p$' used and (algebraic) redundancies `$r$' connecting the experimental results. For $r\ne 0$ it is more appropriate to quote the total $\chi^2_{\rm T}$, with a initial qualitative statistical relevance condition $\chi^2_{\rm T}/(N-p)<1$. The four theoretical columns refer to results with collective velocity $v_c$ (subscript $v$) or without ($v_c=0$). We consider data including \lq All\rq\ data points, and also analyze data excluding from analysis four $\Omega,\,\overline\Omega$ particle ratios, see columns marked \lq No-$\Omega$\rq. Only in letter case we obtain a highly relevant data description. Thus to describe the $\Omega,\,\overline{\Omega}$ yields we need an additional particle production mechanism beyond the statistical Fermi model. We noted the special role of these particles, despite bad statistics, already in the analysis of the S-induced reactions$^5$.\\

%%%%%%%% TABLE 1
%\begin{table}[tb]
%\caption{\label{resultpb2}
\noindent{\small Table 1: WA97 (top) and NA49 (bottom) Pb--Pb 158$A$ GeV particle ratios and our theoretical results, see text for explanation.}
\begin{center}
\begin{tabular}{|lcl|ll|ll|}
\hline
 Ratios & $\!\!\!\!$Ref. & Exp.Data & All & All$|_v$& No-$\Omega$ & No-$\Omega|_v$ \\
\hline
${\Xi}/{\Lambda}$ & [12] &0.099 $\pm$ 0.008 & 0.107 & 0.110 & 0.095 & 0.102 \\
${\overline{\Xi}}/{\bar\Lambda}$ & [12] &0.203 $\pm$ 0.024 & 0.216 & 0.195 & 0.206 & 0.210 \\
${\bar\Lambda}/{\Lambda}$ & [12] &0.124 $\pm$ 0.013 & 0.121 & 0.128 & 0.120 & 0.123 \\
${\overline{\Xi}}/{\Xi}$ & [12] &0.255 $\pm$ 0.025 & 0.246 & 0.225 & 0.260 & 0.252 \\
%\hline
${\Omega}/{\Xi}$ & [12] &0.192 $\pm$ 0.024 & 0.192 & 0.190 &0.078$^*$&0.077$^*$\\
${\overline{\Omega}}/{\overline{\Xi}}$ & [11] &0.27 $\pm$ 0.06 & 0.40 & 0.40 &0.17$^*$ &0.18$^*$ \\
${\overline{\Omega}}/{\Omega}$ & [12] &0.38 $\pm$ 0.10 & 0.51 & 0.47 &0.57$^*$ &0.60$^*$ \\
$(\Omega+\overline{\Omega})\over(\Xi+\bar{\Xi})$& [11] &0.20 $\pm$ 0.03
  & 0.23 & 0.23 &0.10$^*$ &0.10$^*$ \\
\hline
$(\Xi+\bar{\Xi})\over(\Lambda+\bar{\Lambda})$& [6] &0.13 $\pm$ 0.03
  & 0.109 & 0.111 & 0.107 & 0.114 \\
${K^0_{\rm s}}/\phi$ & [7] & 11.9 $\pm$ 1.5\ \ & 16.1 & 15.1 & 9.89 & 12.9 \\
${K^+}/{K^-}$ & [8] & 1.80$\pm$ 0.10 & 1.62 & 1.56 & 1.76 & 1.87 \\
$p/{\bar p}$ & [6] &18.1 $\pm$4.\ \ \ \ & 16.7 & 15.3 & 17.3 & 17.4 \\
${\bar\Lambda}/{\bar p}$ &[24] & 3. $\pm$ 1. & 0.65 & 1.29 & 2.68 & 2.02 \\
${K^0_{\rm s}}$/B & [23] & 0.183 $\pm$ 0.027 & 0.242 & 0.281 & 0.194 & 0.201 \\
${h^-}$/B & [23] & 1.83 $\pm $ 0.2\ \ & 1.27 & 1.55 & 1.80 & 1.83 \\
\hline
 & & $\chi^2_{\rm T}$ & 19 & 18 & 2.1 & 1.8\\
 & & $ N;p;r$ &15;5;4 & 16;6;4 & 11;5;2 & 12;6;2\\
\hline
\end{tabular}
\end{center}
%\vskip -0.8cm
%\end{table}
%%%%%%%%%%%%%%%%%%%%%%%%%%%%%%%%%%%%%%%%%

Considering results obtained with and without flow reveals that the presence of the parameter $v_c$ already when dealing only with particle abundances improves our ability to describe the data. However, $m_\bot$ spectra offer another independent measure of the collective flow $v_c$: for a given pair of values $T_{f}$ and $v_{\rm c}$ we evaluate the resulting $m_\bot$ particle spectrum and analyze it using the spectral shape and kinematic cuts employed by the experimental groups. To find the best values we consider just one `mean' strange baryon experimental value ${\bar T}_{\bot}^{\rm Pb}=260\pm10$, since within the error the high $m_\bot$ strange (anti)baryon inverse slopes are overlapping. Thus when considering $v_c$ along with ${\bar T}_{\bot}$ we have one parameter and one data point more. Once we find best values of $T_{\rm f}$ and $v_{\rm c}$, we study again the inverse slopes of individual particle spectra. We obtain an acceptable agreement with the experimental $T_{\bot}^j$ as shown in left section of table~2\,. For comparison, we have also considered in the same framework the S-induced reactions, and the right section of table~2 show a good agreement with the WA85 experimental data$^{25}$. We used here as the `mean' experimental slope data point ${\bar T}_{\bot}^{\rm S}=235\pm10$. We can see that within a significantly smaller error bar, we obtained an accurate description of the $m_\bot^{\rm S}$-slope data. This analysis implies that the kinetic freeze-out, where elastic particle-particle collisions cease, cannot be occurring at a condition very different from the chemical freeze-out. However, one pion HBT analysis at $p_\bot<0.5$ GeV suggests kinetic pion freeze-out at about$^{26}$ $T_k\simeq120$ MeV. A possible explanation of why here considered $p_\bot>0.7$ GeV particles are not subject to a greater spectral deformation after chemical freeze-out, is that they escape before the bulk of softer hadronic particles is formed.\\

%%%%%%%%%%%%%%% TABLE 2
%\begin{table}[bt]
%\caption{\label{Tetrange2}
\noindent{\small Table 2: Experimental and theoretical $m_\bot$ spectra inverse slopes $T_{\rm th}$. Left Pb--Pb results from experiment$^{10}$ NA49 for kaons and from experiment$^{13}$ WA97 for baryons; right S--W results from$^{25}$ WA85}
\begin{center}
\begin{tabular}{|l|cc|cc|}
\hline
 & $T_{\bot}^{\rm Pb}$\,[MeV]&$T_{\rm th}^{\rm Pb}$\,[MeV]&$T_{\bot}^{\rm S}$\ [MeV]&
  $T_{\rm th}^{\rm S}$\ [MeV]\\
\hline
$T^{{\rm K}^0}$ & 223 $\pm$ 13& 241& 219 $\pm$ \phantom{1}5 & 215\\
$T^\Lambda$ & 291 $\pm$ 18& 280& 233 $\pm$ \phantom{1}3 & 236\\
$T^{\overline\Lambda}$ & 280 $\pm$ 20& 280& 232 $\pm$ \phantom{1}7 & 236\\
$T^\Xi$ & 289 $\pm$ 12& 298& 244 $\pm$ 12& 246\\
$T^{\overline\Xi}$ & 269 $\pm$ 22& 298& 238 $\pm$ 16& 246\\
%$T^{\Omega+\overline\Omega}$& 237 $\pm$ 24& 335& --- & 260\\
\hline
\end{tabular}
\end{center}
%\vskip -0.8cm
%\end{table}
%%%%%%%%%%%%%%%%%%%%%%%%%%%%%%%%%%%%%%%%%

The six statistical parameters describing the particle abundances are shown in the top section of table~3, for both Pb- and S-induced reactions$^{5}$. The errors shown are one standard deviation errors arising from the propagation of the experimental measurement error, but apply only when the theoretical model describes the data well, as is the case here, see the header of each column --- note that for the S-induced reactions (see Ref.[5]) the number of redundancies is large since same data comprising different kinematic cuts is included in the analysis. We note the interesting result that within error the freeze-out temperature $T_{\rm f}$ seen in table~3, is the same for both the S- and Pb-induced reactions. The collective velocity rises from $v_c^{\rm S}=0.5c$ to $v_c^{\rm Pb}\simeq c/\sqrt{3}=0.58$. We then show the light quark fugacity $\lambda_{q}$, and note $\mu_\mathrm{B}^{\rm Pb}=203\pm5 >\mu_\mathrm{B}^{\rm S}=178\pm5$\,MeV. As in S-induced reactions where $\lambda_{s}=1$, now in Pb-induced reactions, a value $\lambda_{s}^{\rm Pb}\simeq 1.1$ characteristic for a source of freely movable strange quarks with balancing strangeness, {\it i.e.,} $\tilde\lambda_{s}=1$ is obtained, see Eq.\,(\ref{lamQ}).\\

%%%%%%%%%%%%%%%%%%%%%% TABLE 3
%\begin{table}[b!]
%\caption{\label{fitqpbs}
\noindent{\small Table 3: Top section: statistical parameters, and their $\chi^2_{\rm T}$, which best describe the experimental results for Pb--Pb data, and in last column for S--Au/W/Pb data presented in Ref.[5]. Bottom section: specific energy, entropy, anti-strangeness, net strangeness of the full hadron phase space characterized by these statistical parameters. In the middle column we fix $\lambda_s$ by requirement of strangeness conservation and choose $\gamma_q=\gamma_q^c$, the pion condensation point.}
\vspace{-0.2cm}\begin{center}
\begin{tabular}{|l|cc|c|}
\hline
 & Pb--No-$\Omega|_v$& Pb--No-$\Omega|_v^*$ & S--No-$\Omega|_v$ \\
$\chi^2_{\rm T};\ N;p;r$&1.8;\ 12;\,6;\,2 & 4.2;\ 12;\,4;\,2 & 6.2;\ 16;\,6;\,6 \\
\hline
$T_{f}$ [MeV] & 144 $\pm$ 2 & 145 $\pm$ 2 & 144 $\pm$ 2 \\
$v_c$ & 0.58 $\pm$ 0.04 & 0.54 $\pm$ 0.025 & 0.49 $\pm$ 0.02\\
$\lambda_{q}$ & 1.60 $\pm$ 0.02 & 1.605 $\pm$ 0.025 & 1.51 $\pm$ 0.02 \\
$\lambda_{s}$ & 1.10 $\pm$ 0.02 & 1.11$^*$ & 1.00 $\pm$ 0.02 \\
$\gamma_{q}$ & 1.7 $\pm$ 0.5 & $\gamma_q^c=e^{m_\pi/2T_f}$ & 1.41 $\pm$ 0.08 \\
$\gamma_{s}/\gamma_{q}$& 0.86 $\pm$ 0.05 & 0.78 $\pm$ 0.05 & 0.69 $\pm$ 0.03 \\
\hline
$E_{f}/B$ & 7.0 $\pm$ 0.5 & 7.4 $\pm$ 0.5 & 8.2 $\pm$ 0.5 \\
$S_{f}/B$ & 38 $\pm$ 3 & 40 $\pm$ 3 & 44 $\pm$ 3 \\
${s}_{f}/B$ & 0.78 $\pm$ 0.04 & 0.70 $\pm$ 0.05 & 0.73 $\pm$ 0.05 \\
$({\bar s}_f-s_f)/B$ & 0.01 $\pm$ 0.01 & 0$^*$ & 0.17 $\pm$ 0.02\\
\hline
\end{tabular}
\end{center}
%\vskip -0.8cm
%\end{table}
%%%%%%%%%%%%%%%%%%%%%%%%%%%%%%%%%%%%%%%%

$\gamma_q>1$ seen in table~3 implies that there is phase space over-abundance of light quarks, to which, {\it e.g.,} gluon fragmentation at QGP breakup {\it prior} to hadron formation contributes. $\gamma_q$ assumes in our data analysis a value near to where pions could begin to condense$^{27}$, $\gamma_q=\gamma_q^c\equiv e^{m_\pi/2T_f}$\,. We found studying the ratio $h^-/B$ separately from other experimental results that the value of $\gamma_q\simeq\gamma_q^c$ is fixed consistently and independently both, by the negative hadron ($h^-$), and the strange hadron yields. The unphysical range $\gamma_q>\gamma_q^c$ can arise, since up to this point we use only a first quantum (Bose/Fermi) correction. However, when Bose distribution for pions is implemented, which requires the constraint $\gamma_q\le\gamma_q^c$, we obtain practically the same results, as shown in second column of table~3. Here we allowed only 4 free parameters, {\it i.e.} we set $\gamma_q=\gamma_q^c$\,, and the strangeness conservation constraint fixes $\lambda_s$\,. We then show in table~3 the ratio $\gamma_s/\gamma_q$, which corresponds (approximately) to the parameter $\gamma_s$ when $\gamma_q=1$ had been assumed. We note that $\gamma_s^{\rm Pb}>1$. This strangeness over-saturation effect could arise from the effect of gluon fragmentation combined with early chemical equilibration in QGP, $\gamma_s(t<t_f)\simeq 1$. The ensuing rapid expansion preserves this high strangeness yield, and thus we find the result $\gamma_s>1$\,, as is shown in figure 33 in Ref.[21].

We show in the bottom section of table~3 the energy and entropy content per baryon, and specific anti-strangeness content, along with specific strangeness asymmetry of the hadronic particles emitted. The energy per baryon seen in the emitted hadrons is nearly equal to the available specific energy of the collision (8.6 GeV for Pb--Pb, 8.8--9 GeV for S--Au/W/Pb). This implies that the fraction of energy deposited in the central fireball must be nearly the same as the fraction of baryon number. The small reduction of the specific entropy in Pb--Pb compared to the lighter S--Au/W/Pb system maybe driven by the greater baryon stopping in the larger system, also seen in the smaller energy per baryon content. Both collision systems freeze out at energy per unit of entropy $E/S=0.185$ GeV. There is a loose relation of this universality in the chemical freeze-out condition with the suggestion made recently that particle freeze-out occurs at a fixed energy per baryon for all physical systems$^{28}$, since the entropy content is related to particle multiplicity. The overall high specific entropy content we find agrees well with the entropy content evaluation made earlier$^{29}$ for the S--W case.

Inspecting figure 38 in Ref.[21] we see that the specific yield of strangeness we expect from the kinetic theory in QGP is at the level of 0.75 per baryon, in agreement with the results of present analysis shown in table~3. This high strangeness yield leads to the enhancement of multi-strange (anti)baryons, which are viewed as important hadronic signals of QGP phenomena$^{30}$, and a series of recent experimental analysis has carefully demonstrated comparing p--A with A--A results that there is quite significant enhancement$^{13,31}$, as has also been noted before by the experiment$^{32}$, NA35.

The strangeness imbalance seen in the asymmetrical S--Au/W/Pb system (bottom of table~3) could be a real effect arising from hadron phase space properties. However, this result also reminds us that though the statistical errors are very small, there could be a considerable systematic error due to presence of other contributing particle production mechanisms. Indeed, we do not offer here a consistent understanding of the $\Omega,\,\overline\Omega$ yields which are higher than we can describe. We have explored additional microscopic mechanisms. Since the missing $\Omega,\,\overline\Omega$ yields are proportional (13\%) to the $\Xi,\overline\Xi$ yield, we have tested the hypothesis of string fragmentation contribution in the {\it final state}, which introduces just the needed `shadow' of the $\Xi,\overline\Xi$ in the $\Omega,\overline\Omega$ abundances. While this works for $\Omega,\,\overline\Omega$, we find that this mechanism is not compatible with the other particle abundances. We have also explored the possibility that unknown $\Omega^*,\,\overline{\Omega^*}$ resonances contribute to the $\Omega,\,\overline\Omega$ yield, but this hypothesis is ruled out since the missing yield is clearly baryon--antibaryon asymmetric. Thus though we reached here a very good understanding of other hadronic particle yields and spectra, the rarely produced but greatly enhanced $\Omega,\,\overline\Omega$ must arise in a more complex hadronization pattern. 
We have presented a comprehensive analysis of hadron abundances and $m_\bot$-spectra observed in Pb--Pb 158$A$ GeV interactions within the statistical Fermi model with chemical non-equilibrium of strange and non-strange hadronic particles. The key results we obtained are: $\tilde \lambda_s=1$ for S and Pb collisions\,; $\gamma_s^{\rm Pb}>1, \ \gamma_q>1$\,; $S/B\simeq 40$\,; $ s/B\simeq 0.75$\,; all in a remarkable agreement with the properties of a deconfined QGP source hadronizing without chemical re-equilibration, and expanding not faster than the sound velocity of quark matter. The universality of the physical properties at chemical freeze-out for S- and Pb-induced reactions points to a common nature of the primordial source of hadronic particles in both systems. The difference in spectra between the two systems arises in our analysis from the difference in the collective surface explosion velocity, which for larger system is higher, having more time to develop. Among other interesting results which also verify the consistency of our approach we recall: good balancing of strangeness $\langle \bar s-s\rangle=0$ in the Pb--Pb case; increase of the baryochemical potential as the collision system grows; energy per baryon near to the value expected if energy and baryon number deposition in the fireball are similar. We note that given the magnitude of $\gamma_q$ and low chemical freeze-out temperature, most (75\%) of all final state pions are directly produced, and not resonance decay products. Our results differ significantly from an earlier analysis regarding the temperature at which hadron formation occurs. Reduction to $T_f=140$--$145$\,MeV becomes possible since we allow for the chemical non-equilibrium and collective flow, and only with these improvements in analysis our description acquires convincing statistical significance, which e.g. a hadronic gas scenario does not offer$^{33}$. Because we consider flow effects, we can address the central rapidity data of the WA97 experiment at the required level of precision, showing the consistency in these results with the NA49 data discussed earlier$^{17}$. 

In our opinion, the only consistent interpretation of the experimental results analyzed here is that hadronic particles seen at 158$A$ GeV nuclear collisions at CERN-SPS are formed directly in hadronization of an exploding deconfined phase of hadronic matter, and that these particles do not undergo a chemical re-equilibration after they have been produced. 


\footnotetext{\vspace*{-0.5cm}
\begin{enumerate}

\item%1{QGP}
%THE SEARCH FOR THE QUARK - GLUON PLASMA.
J. Harris and B. M\"uller, {\it Ann. Rev. Nucl. Part. Sci.}
{\bf 46}, pp71-107, (1996); and references therein.

\item%2{Fer50}
E. Fermi, {\it Progr. Theor. Phys.} {\bf 5} 570 (1950);
{\it Phys. Rev.} {\bf 81}, 115 (1950);
{\it Phys. Rev.} {\bf 92}, 452 (1953).


\item%3{Sol97}
%``Chemical equilibration of strangeness'',
J. Sollfrank, {\it J. Phys. } G {\bf 23}, 1903 (1997).

\item%4{Raf91}
%``Strange Antibaryons from Quark-Gluon Plasma'',
J. Rafelski, {\it Phys. Lett. }B {\bf 262}, 333 (1991);
%``Hot and Strange Matter''
{\it Nucl. Phys.} A {\bf 544}, 279c (1992).


\item%5{LRa99}
%`` Chemical non-equilibrium and deconfinement in 200 A GeV Sulphur
% induced reactions'',
J. Letessier and J. Rafelski, {\it Phys. Rev.} C {\bf 59}, 947 (1999).
%;
% we use recomputed numerical results which
%are at times slightly different, for example 
%the published value of freeze-out
%temperature $T_f^{\rm S}=143\pm3$\,MeV is 
%now $T_f^{\rm S}=144\pm2$\,MeV\,.

\item%6{Ody98}
G.J.\,Odyniec, {\it Nucl. Phys.} A {\bf 638}, 135, (1998).

\item%7{Puh98}
F.\,P\"uhlhofer, NA49, 
{\it Nucl. Phys.} A {\bf 638}, 431,(1998).

\item%8{Bor97}
C.\,Bormann, NA49, 
{\it J. Phys.} G {\bf 23}, 1817 (1997).

\item%9{App98}
%Xi and anti-Xi production at 158 GeV/nucleon Pb+Pb collisions
H. Appelsh\"auser {\it et al.}, NA49,
{\it Phys. Lett.} B {\bf 444}, 523, (1998).

\item%10{Mar99}
S. Margetis, NA49,
{\it J. Physics} G {\bf 25}, 189 (1999).

\item%11{Hol97}
A.K.\,Holme, WA97, {\it J. Phys.} G {\bf 23}, 1851 (1997).

\item%12{Kra98}
I.\,Kr\'alik, WA97, 
{\it Nucl. Phys.} A {\bf 638},115, (1998).

\item%13{WA97}
%ENHANCEMENT OF CENTRAL LAMBDA, XI AND OMEGA YIELDS IN PB - PB
%COLLISIONS AT 158 A-GEV/C.
E. Andersen {\it et al.}, WA97,
{\it Phys. Lett.} B {\bf 433}, 209, (1998);
%{\it Strangeness enhancement at mid-rapidity in Pb-Pb collisions
%at 158 A GeV/c}, 
%E. Andersen {\it et al.}, WA97, {\it Phys. Lett.} B 
{\bf 449}, 401 (1999).


\item%14{Hei92}
%``Search for collective transverse flow using particle transverse
%momentum spectra in relativistic heavy-ion collisions''
K. S. Lee, U. Heinz and E. Schnedermann, 
{\it Z. Phys.} C {\bf 48}, 525 (1990). %;
%``Fireball Spectra'',
%E. Schnedermann, J. Sollfrank and U. Heinz, pp175--206 
%in {\it Particle Production in Highly Excited Matter},
%NATO-ASI B303, Plenum Press, (New York 1993), 
%H.H. Gutbrod and J. Rafelski, editors.

\item%15{Kar98}
%Deconfinement and Chiral Symmetry Restoration in an SU(3) 
%Gauge Theory with Adjoint Fermions Bielefeld preprint BI-TP 98/40,
%[hep-lat/9812023], to be published. 
F. Karsch, and M. L\"utgemeier, 
{\it Nucl. Phys.} B {\bf 550}, 449 (1999).


\item%16{LRPb98}
%``QGP in Pb--Pb 158 A GeV collisions: \\ Evidence
%from strange particle abundances and the Coulomb effect''
%%`` Chemical non-equilibrium in High Energy Nuclear Collisions''
J. Letessier and J. Rafelski, {\it Acta Phys. Pol.};
{\bf B30}, 153 (1999);
{\it J. Phys.} Part. Nuc. {\bf G25}, 295, (1999)\,.

\item%17{Bec98}
% ON CHEMICAL EQUILIBRIUM IN NUCLEAR COLLISIONS.
 F. Becattini, M. Gazdzicki and J. Sollfrank,
{\it Eur. Phys. J.} C {\bf 5}, 143-15, (1998). 

\item%18{Let97}
J. Letessier, J. Rafelski, and A. Tounsi, 
{\it Phys. Lett.} B {\bf 410}, 315 (1997);
%``Hadronic Signatures of Deconfinement in 
%Relativistic Nuclear Collisions''\\
{\it Acta Phys. Pol.} B {\bf 28}, 2841 (1997).

\item%19{HSX98} 
H. van\,Hecke, H. Sorge and N. Xu,
{\it Phys. Rev. Lett.} {\bf 81}, 5764 (1998).

\item%20{RM82}
%``Strangeness production in the quark-gluon plasma",
{J. Rafelski and B. M\"uller}, {\it Phys. Rev. Lett}
{\bf 48}, 1066 (1982); {\bf 56}, 2334E (1986);
%``Strange quarks in relativistic nuclear collisions",
{P.~Koch, B.~M\"uller and J.~Rafelski},
{\it Phys. Rep.} {\bf 142}, 167 (1986).

\item%21{acta96}
%``Strange particles from dense hadronic matter'',
{J. Rafelski, J. Letessier and A. Tounsi},
{\it Acta Phys. Pol.} B {\bf 27}, 1035 (1996), and references therein.

\item%22{Ody97}
G.J.\,Odyniec, NA49, 
{\it J. Phys.} G {\bf 23}, 1827 (1997).

\item%23{Jon96}
P.G.\,Jones, NA49, 
{\it Nucl. Phys.} A {\bf 610}, 188c (1996).

\item%24{Roh97}
D. R\"ohrig, NA49,
``Recent results from NA49 experiment on Pb--Pb collisions at 158 A GeV'',
see Fig. 4, in proc. of EPS-HEP Conference, Jerusalem, Aug. 19-26, 1997.


\item%25{WA85slopes}
D. Evans, WA85,
{\it Heavy Ion Physics} {\bf 4}, 79 (1996).

\item%26{Fer99}
D. Ferenc, U. Heinz, B. Tomasik, U.A. Wiedemann, and J.G. Cramer,
%{\it Universal pion freeze-out phase-space density}
%CERN-TH/99-14, [hep-ph/9902342] of 15 Feb. 1999.
{\it Phys. Lett.} B {\bf 457}, 347 (1999).

\item%27{Heinz99} 
U. Heinz, private communication.

\item%28{CR98}
%``Unified Description of Freeze-Out Parameters
%in Relativistic Heavy Ion Collisions''
J. Cleymans and K. Redlich, {\it Phys. Rev. Lett.} {\bf 81}, 5284 (1998);
and references therein.

\item%29{Let93} %\item%{Let95}
%``Evidence for a high entropy phase in nuclear collisions'',
%``Strangeness conservation in hot nuclear fireballs'',
J. Letessier, A. Tounsi, U. Heinz, J. Sollfrank and J. Rafelski
{\it Phys. Rev. Lett.} {\bf 70}, 3530 (1993); 
{\it Phys.\ Rev.} D {\bf 51}, 3408 (1995).

\item%30{Raf80}
%``Extreme States of Nuclear Matter,"
J. Rafelski, pp 282--324, in 
{\it Future Relativistic Heavy Ion Experiments}, 
R. Bock and R. Stock, Eds., GSI Report 1981-6; 
in {\it New Flavor and Hadron Spectroscopy},
 J. Tran Thanh Van, Ed. p 619, Editions Frontiers (Paris 1981);
and in {\it Nucl. Physics} A {\bf 374}, 489c (1982).

\item%31{WA85}
F. Antinori {\it et al.}, WA85,
{\it Phys. Lett.} B {\bf 447}, 178 (1999).

\item%32{Alb94}
Th. Alber {\it et al.}, NA35,
{\it Z. Phys.} C {\bf 64}, 195 (1994).

\item%33{BHS99}
P. Braun-Munzinger, I. Heppe, and J. Stachel, {\it Chemical
Equilibration in Pb+Pb collisions at the SPS}, [nucl-th/9903010],
submitted to {\it Phys. Lett. B}, March 1999. 

\end{enumerate}
}
\end{mdframed}
\vskip 0.5cm 
%%%%%%%%%%%%%%%%%%%%%%%%%%%%%%%%%%%%%%%%%%%%%%%%%%%% 

\subsubsection{Echos of forthcoming new state of matter CERN announcement}
The publication effort of the above manuscript terminated  on 13 January 2000, just 4 weeks before CERN announced its new phase of matter discovery. In my letter below the reader sees the context of this announcement mentioned which included some scientific arguments surrounding the CERN preparations for release of the QGP announcement: I refer to disputes between NYC Columbia University based Dr. Miklos Gyulassy and Maurice Jacob regarding the CERN announcement scheduled for early February.\\

\noindent \textit{My following letter to PRL editor in chief Jack Sandweiss of January 13, 2000 with whom I had a personal and freindly relation reads:}\\[-0.7cm]
%
%%%%%%%%%%%%%%%%%%%%%%%%%%%%%%%%%%%%%%%%%%%%%%%%%%%%
\begin{mdframed}[linecolor=gray,roundcorner=12pt,backgroundcolor=Dandelion!15,linewidth=1pt,leftmargin=0cm,rightmargin=0cm,topline=true,bottomline=true,skipabove=12pt]\relax%
January 13, 2000\\
Dear Jack,\\
%I really do not wish to move endlessly on with arguments and it would not cross my mind to carry the matter of LC7284 beyond your PRL desk. I went that far, since you once were stressing to me how important it is to follow the process.

\ldots If in your judgment this work is not PRL suitable, so be it, with all the ensuing consequences -- these have just begun. The endless delay of our work has muffled a scientific discussion (for others were standing by and watching what happens) and what you thus see today are draft (CERN) press releases fought off by Guylassy. It would have been nicer to have PRL papers arguing the matter of QGP at SPS energies. Somewhere things went bad.\\

\small{CLARIFICATION ABOUT CONTENTS:}\\
Permit me to notice that it is impossible to add to the paper within the prescribed length the items that the referees are asking for (predictions for example).

I wish also to set straight one impression: The key results and methods were at the time of submission to PRL completely original and who claims otherwise may be in need of an alibi.

Best wishes JAN
\end{mdframed}
%\vskip 0.5cm
%%%%%%%%%%%%%%%%%%%%%%%%%%%%%%%%%%%%%%%%%%%%%%%%%%%%%%%%%%%%%%%%%%%%%

\subsubsection{Conversation with referees about QGP fireball at CERN SPS}
Returning to the timeline: The manuscript LC7284 was sent out to referees on 24 March 1999. In my publication file I have only  two relevant reports, with second iterations, as well as follow-up correspondance.\\

%%%%%%%%%%%%%%%%%%%%%%%%%%%%%%%%%%%%%%%%%%%%%%%%%%%%%%%%%%%%%%%%%%%%%%%%
\noindent \textit{On 21 May 1999 I received the following letter from PRL with one referee report; a second referee report follows by fax on 26 May 1999:}\\[-0.7cm]
%
\begin{mdframed}[linecolor=gray,roundcorner=12pt,backgroundcolor=GreenYellow!15,linewidth=1pt,leftmargin=0cm,rightmargin=0cm,topline=true,bottomline=true,skipabove=12pt]
\relax%
%
%\ldots reports include a critique which is sufficiently adverse that we cannot accept your paper on the basis of material now at hand. We enclose pertinent comments.

%If you feel that you can overcome or refute the criticism, you may resubmit to Physical Review Letters. Please accompany any resubmittal by a summary of the changes made, and a brief response to all recommendations and criticisms.\\

{\bf Referee A:}\\
This work discusses very interesting measurements from both the NA49 and the WA97 collaborations at the CERN-SPS. However, this variant of the fireball model is based on methods already well developed (see ref. 16, refs. 3,4,5 and ref. 14) with only incremental refinements as noted in the first paragraph. In addition, the point of this paper proported by the authors does not bring any new physical insight with regard to these measurements (see their paper, ref. 16, for example). Furthermore,  their claim of a QGP at the SPS is at best controversial.

Interestingly, this article completely skirts the intriguing observations made by WA97 of the small inverse $m_t$ slopes and large abundances of the multi-strange baryons. The $m_t$ dependent assumptions of this fireball model for the baryons which result in larger inverse $m_t$ slopes with more strangeness are inconsistent with the measured smallar inverse $m_t$ slopes with more strangeness. This unique dependence has been discussed in light of the early freezeout of the multi-strange baryons by H. Van Hecke, H. Sorge and N. Xu, Phys.Rev.Lett.81:5764-5767,1998, (a reference that should be included in this work).

Furthermore, understanding particle production with the Omegas is critical since the enhanced production of multi-strange baryons is predicted as one of the signals of the QGP (see Ref. 21). The poor description by this fireball model of particle ratios when including the Omegas are in contradiction of the authors\rq\ initial claim of a QGP scenario. In fact, as emphasized in the conclusion, only by relaxing chemical equilibration and introducing dynamical flow parameters can the \lq fit\rq\ be improved. However, the theoretical origin of such non-equilibrium features requires transport dynamical approaches which are totally ignored here.

In light of the above discrepancies and the lack of new methods or insights, I can not recommend this work for publiciation in Phys. Rev. Lett.
\end{mdframed}
\vskip 0.5cm 
 

\noindent {\it This prompted on 14/15 June, 1999 the response along with arXiv-ing of v2 of the manuscript \url{https://arxiv.org/pdf/nucl-th/9903018v2 dated 25 June 1999}:}\\[-0.7cm]
%
\begin{mdframed}[linecolor=gray,roundcorner=12pt,backgroundcolor=Dandelion!15,linewidth=1pt,leftmargin=0cm,rightmargin=0cm,topline=true,bottomline=true,skipabove=12pt]
 \relax%
%
Dear \ldots (PRL)\\
%thank you for making the reports of referees A and B available to us by E-mail/Fax.

%The following is a complete and full response to both referees, and the attached manuscript version of June 14 1999 comprises all the changes we felt needed to be made. \ldots .\\

\noindent a) \small{GENERAL REMARKS OF REFEREE B:}\\
In response to the general concern if the manuscript can be understood by the wider audience we did work hard to word the paper better making minute but frequent changes in the English. We believe that this short paper now meets the stated criterion.

To stress the theoretical nature of our study, pursuant to next remark of referee B in second paragraph of the review we have in particular\\
 i) replaced in the title \lq Hadrons from..\rq\ by \lq On hadron production in...\rq\ and \\
 ii) In the abstract, we added in the first phrase \lq ...obtained in several relativistic heavy ion experiments is utilized..\rq\\

\noindent b) \small{GENERAL REMARKS OF REFEREE A:}\\
We agree with the referee that the methods we have been developing since 1991 [4] are today well established and are widely used. Indeed this is the strength of our work. The refinement we here introduce are not trivial, even if these can be mentioned in one simple phrase: we have in full incorporated the radial flow, and its impact on the $m_t$ spectra, see old second paragraph. 

We have now refined that (part of our) discussion and as suggested by referee A, we include in the short third paragraph the new reference to the work of vanHecke, Sorge, Xu (new reference 19) on transverse slopes. The paragraph reads: \ldots The here developed model offers a natural understanding of the systematic behavior of the $m_\bot$-slopes which differs from other interpretations. The near equality of (inverse) slopes of nearly all strange baryons and antibaryons arises here by means of the sudden hadronization at the surface of an exploding QGP fireball. In the hadron based microscopic simulations this behavior of $m_\bot$-slopes can also arise allowing for particle-dependent freeze-out times [19]. \ldots


%\noindent c) We fixed the two first detailed points of referee B as recommended.\\

%\noindent d) we thank referee B for pointing out (third detail remark) a slight misunderstanding we had about this NA49 data-point. We removed the comment made in text, and have made the associated change in table I, top line of second (NA49) portion.\\

%\noindent e) In response to the fourth detail of referee B about the $\xi^2$ we added a pertinent sentence in second paragraph below equation 1, which reminds the reader that redundancies limit the validity of the usual tests of statistical significance, though indeed as suggested by the referee, this is a first verification of statistical validity to be made.\\

%\noindent f) In response to the fifth detail of referee B, we reword the pertinent remarks, that now read \lq \ldots kinetic freeze-out \ldots cannot be occurring at a condition very different from chemical freeze-out.\rq\\

%\noindent g) In response to the sixth detail of referee B we slightly reworded the phrase eliminating the usage of the word over-saturation. We also made the definition of $\gamma$ more clear reordering the 4th paragraph of the paper where this quantity had been introduced.\\

\noindent h) In response to referee A we separate and extend the discussion of the Omega particles into a separate paragraph (4th from the end):\\
ldots we do not offer here a consistent understanding of the $\Omega,\,\overline\Omega$ yields which are higher than we can describe. We have explored additional microscopic mechanisms. Since the missing $\Omega,\,\overline\Omega$ yields are proportional (13\%) to the $\Xi,\overline\Xi$ yield, we have tested the hypothesis of string fragmentation contribution in the {\it final state}, which introduces just the needed `shadow' of the $\Xi,\overline\Xi$ in the $\Omega,\overline\Omega$ abundances. While this works for $\Omega,\,\overline\Omega$, we find that this mechanism is not compatible with the other particle abundances. We have also explored the possibility that unknown $\Omega^*,\,\overline{\Omega^*}$ resonances contribute to the $\Omega,\,\overline\Omega$ yield, but this hypothesis is ruled out since the missing yield is clearly baryon--antibaryon asymmetric. Thus though we reached here a very good understanding of other hadronic particle yields and spectra, the rarely produced but greatly enhanced $\Omega,\,\overline\Omega$ must arise in a more complex hadronization pattern. \ldots \\

%\noindent i) We update, and streamline acknowledgments, the reference text and last entry in table II to make space for the additions mentioned above, after which the manuscript prints on 4 PRL pages.\\

Overall there are no major changes in the contents of our paper, though some improvement in presentation has been reached, due to thorough review and constructive comments of the referees.

We hope that the attached manuscript will be accepted for publication in PRL.
\end{mdframed}
\vskip 0.5cm
 
%%%%%%%%%%%%%%%%%%%%%%%%%%%%%%%%%%%%%%%%%%%%%%%%%%%%%%%%%%%%%%%%%%%
\noindent \textit{This was not to be. Jumping forward to 27 August 1999, PRL writes:}\\[-0.7cm]
%
\begin{mdframed}[linecolor=gray,roundcorner=12pt,backgroundcolor=GreenYellow!15,linewidth=1pt,leftmargin=0cm,rightmargin=0cm,topline=true,bottomline=true,skipabove=12pt]\relax%
%
\ldots I had sent your manuscript do a Divisional Associate Editor of Physical Review Letters for advice on your appeal. However, we recently received an updated report from the second referee B. In view of the attached additional remarks I would like to give you an opportunity to consider the updated report before I continue the appeals process.

Please let us know how you wish to process. We are holding your manuscript in this office awaiting your response.\\

Second report of Referee B: Updated report (meaning that there was a 2nd report we have not seen? JR)

Re: Manuscript LC7284

The ammendments made by J.Rafelski and J.Letessier in response to my comments on their manuscript \lq\lq On Hadron Production in Pb-Pb Collisions at 158 A GeV\rq\rq\ are satisfactory.

Not withstanding the opinion of the other referee, I believe the modifications to the fireball model introduced by Rafelski and Latessier are significant (e.g. the treatment of radial flow). Also I should point out that the measurements of Omega yields are relatively more uncertain than all the other ratios used in the R-L model analysis. Therefore, the two sets of calculations, with and without the omega ratio, are very informative.

However (and, unfortunately), I did overlook it when writing my original report) this model has a technical flaw in its treatment of the pion Bose statistics, resulting in the oversaturation of light quarks and the divergence (pion lasers) at $p_t=0$ ($\mu_\pi=156$ MeV was obtaind, inspite of Bose condensate at $\mu_\pi=$pion mass). This point clearly needs clarification before the paper can be recommended for publication.
\end{mdframed}
\vskip 0.5cm
%%%%%%%%%%%%%%%%%%%%%%%%%%%%%%%%%%%%%%%%%%%%%%%%%%%%%%%%%%%%%%%%%%%%%%%%%%%%

Here a word of explanation regarding the last paragraph is needed: I had in good faith discussed this work as I believe by phone with Ulrich Heinz of later QGP rejection fame; in this discussion the question of the relevance of the Bose statistic\label{HeinzBose} for pions came up. This was  not addressed in v1 and v2 of the manuscript.  We  know by means of precise HBT analysis that the pion emission lasts about 2\;fm/c. This means that even if the total pion abundance is large, there is no need for Bose statistics as this characteristic time for pion emission suffices to cumulate overabundance observed in experiment by means of sequential emission. 

However, in 1999 one could argue that the creation of the pion yield could be truly sudden. I was asked to investigate this for the purpose of completeness of our work and that we did for v3 of the manuscript. Consistency of the sudden hadronization model demands  that we show that Bose statistics describing the pion yield will work, so that perfectly sudden hadronization can be demonstrated -- even if it has little if any true physics meaning.

As above correspondance quote shows, somehow referee B learned about my related conversation.  This allowed   the PRL divisional associate editor (same person), to find a reason to reject our work. Without this reason the situation would be that we answered objections of referee A, implemented comments and received a nod from referee B; he would need to accept the paper that claims QGP was found at SPS. Clearly, by doing this the associate editor would have contradicted his objections to the CERN announcement of QGP. This is what he in essence told me and I still hear his words in my mind 20 years later.\\

\noindent \textit{I made a further effort with PRL without considering what was inevitable: That hostile  associate editor   would be the final judge and not the referee B. I respond to PRL on Sept 8, 1999:}\\[-0.7cm]
%
\begin{mdframed}[linecolor=gray,roundcorner=12pt,backgroundcolor=Dandelion!15,linewidth=1pt,leftmargin=0cm,rightmargin=0cm,topline=true,bottomline=true,skipabove=12pt]\relax%
%
In response to the request for a clarification made by referee B, we have recomputed our results using the Bose statistics for pions. We are sure that the astute referee will be convinced by the stability of our results subject to this technical refinement.

In order to properly explain the calculational contents, we have expanded as follows a phrase in the old manuscript to a longer comment now located at the end of page 2 of the current PRL style printout. The full paragraph reads:

\lq\lq $\gamma_q>1$ seen in table~3 implies that there is phase space over-abundance of light quarks, to which, {\it e.g.,} gluon fragmentation at QGP breakup {\it prior} to hadron formation contributes. $\gamma_q$ assumes in our data analysis a value near to where pions could begin to condense [27], $\gamma_q=\gamma_q^c\equiv e^{m_\pi/2T_f}$\,. We found studying the ratio $h^-/B$ separately from other experimental results that the value of $\gamma_q\simeq\gamma_q^c$ is fixed consistently and independently both, by the negative hadron ($h^-$), and the strange hadron yields. The unphysical range $\gamma_q>\gamma_q^c$ can arise, since up to this point we use only a first quantum (Bose/Fermi) correction. However, when Bose distribution for pions is implemented, which requires the constraint $\gamma_q\le\gamma_q^c$, we obtain practically the same results, as shown in second column of table~3. Here we allowed only 4 free parameters, {\it i.e.} we set $\gamma_q=\gamma_q^c$\,, and the strangeness conservation constraint fixes $\lambda_s$\,. We then show in table~3 the ratio $\gamma_s/\gamma_q$, which corresponds (approximately) to the parameter $\gamma_s$ when $\gamma_q=1$ had been assumed. We note that $\gamma_s^{\rm Pb}>1$. This strangeness over-saturation effect could arise from the effect of gluon fragmentation combined with early chemical equilibration in QGP, $\gamma_s(t<t_f)\simeq 1$. The ensuing rapid expansion preserves this high strangeness yield, and thus we find the result $\gamma_s>1$\,, as is shown in figure 33 in Ref. [21].\rq\rq\ \ldots 
 
To make space for this explanation of the procedure we conclude our paper now with the abreviated conclusions which contain just one remark as follows:

\lq\lq In our opinion, the only consistent interpretation of the experimental results analyzed here is that hadronic particles seen at 158$A$ GeV nuclear collisions at CERN-SPS are formed directly in hadronization of an exploding deconfined phase of hadronic matter, and that these particles do not undergo a chemical re-equilibration after they have been produced. \rq\rq\
 
 
We have also added in penultimate paragraph the phrase: \lq\lq We note that given the magnitude of $\gamma_q$ and low chemical freeze-out temperature, most (75\%) of all final state pions are directly produced, and not resonance decay products.\rq\rq\\
b) updated references 15, 26, shortened references 14 and 29 (now 30) 
c) smoothed English\\
 i) in paragraph below equation 1,\\
 ii) in the paragraph above the new paragraph and \\
iii) the second last paragraph.\\


We sincerely hope that you will consider our manuscript 
now suitable to be published in PRL.
\end{mdframed}
\vskip 0.5cm
%%%%%%%%%%%%%%%%%%%%%%%%%%%%

The editors of PRL consulted  the  divisional associate editor, who did not recuse himself despite his well-known public position against QGP at CERN. Through PRL channels, he rejected our work on September 28, 1999. His letter was full of inaccuracies as he mixed up two manuscript files that he was both rejecting for different reasons, thus a further delay applied before the manuscript was rejected as noted previously.  

%%%%%%%%%%%%%%%%%%%%%%%%%%%%%%%%%%%%%%%%%%%%%%%%%%%%%%%%%%%%%%
\noindent \textit{The scientific argument presented  by the PRL  divisional associate editor on September 28, 1999 were:}\\[-0.7cm]
%
\begin{mdframed}[linecolor=gray,roundcorner=12pt,backgroundcolor=GreenYellow!15,linewidth=1pt,leftmargin=0cm,rightmargin=0cm,topline=true,bottomline=true,skipabove=12pt]\relax%
%
RE: LC7284\\
\ldots  I am thus sure that I understand very well what the authors have done and achieved in their paper.

My own assessment of the paper is as follows: By restricting their attention to the abundance ratios between only a fraction of the measured hadrons, leaving out the Omega and Anti-Omega baryons, and by allowing the pions to develop a Bose condensate in order to absorb the measured large pion abundance at the suggested low freeze-out temperature, the authors leave more questions open than they solve. Since the methods used in the paper are not new, its importance must be judged by its results and the implied physical picture. The claim for fame of the paper is based on the very low chemical freeze-out temperature found by the authors (contradicting all other published values) and the high statistical significance of their fit. But the resulting physical picture is not convincing: of 15(-4) measured particle abundance ratios, 11(-2) can be fit very nicely with 5 parameters, but only if 4(-2) others are excluded from the fit, by postulating (but not successfully identifying) a different creation mechanism. (In brackets I counted the redundant ratios.) The authors emphasize that the slopes of the kaon, Lambda and Xi transverse mass spectra are fit well if strong transverse flow is allowed for, but their model is unable to explain the steeper Omega spectra, and the expected strong effects of the Bose condensation of pions (as implied by their fit and mentioned in the latest version of the MS) on the pion spectrum are not discussed. Finally, even if they don't stress this in words, the authors make the dramatic prediction of a pion condensate in heavy ion collisions,  without discussing the many other observable effects which should result from such a phenomenon, nor the fact that it contradicts the findings in Ref. [26]. 
  
\end{mdframed}
\vskip 0.5cm
%%%%%%%%%%%%%%%%%%%%%%%%%%%%%%%%%%%%%%%%%%%%%%

Regarding the argument seen above: 
\begin{enumerate}
\item We did not predict pion condensates, see page \pageref{HeinzBose}. We demonstrated answering referee after-thaught demand that Bose statistics of emitted pion density does not alter our results, earlier obtained without. 
\item
Seen from twenty years historical perspective: 
Everything the associate editor said and used was in essence a personal opinion,  a mix-up with another opinion he was preparing in parallel for another PRL paper.
\item
What we presented  was a result of a model that withstood the test of time as this long article has demonstrated.  
\end{enumerate}
To conclude: associate editor of PRL has now admitted in the 2019 interview, see page~\pageref{Heinz2019}, to have been mistaken in his rejection of the possibility of QGP at CERN-SPS. I extend these remarks to include this evaluation of our work.
 
