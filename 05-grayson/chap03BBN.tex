%%%%%%%%%%%%%%%%%%%%%%
\subsection{Effective inter-nuclear potential}\label{sec:potential}
We calculate the potential of light nuclei in the early Universe electron-positron plasma by Fourier transforming the screened scalar potential $\phi$ of a single traveling nuclei \req{eq:phi}
\begin{equation}\label{eq:potent}
 \phi(t,\boldsymbol{x}) = \int \frac{d^4k}{(2\pi)^4} e^{-i\omega t+ i\boldsymbol{k}\cdot\boldsymbol{x}} \frac{\widetilde{\rho}_\text{ext}(\omega,\boldsymbol{k})}{\varepsilon_\parallel(\omega,\boldsymbol{k})(\boldsymbol{k}^2-\omega^2) }\,,
\end{equation}
where $\widetilde{\rho}_{\text{ext}}(\omega,\boldsymbol{k})$ is the Fourier-transformed charge distribution of nuclei traveling at a constant velocity, and $\varepsilon_\parallel(\omega,\boldsymbol{k})$ is the longitudinal relative permittivity. The relative permittivity can be written in terms of the polarization tensor as
\begin{equation}\label{eq:epsilon}
 \varepsilon_\parallel(\omega,\boldsymbol{k})= \left(\frac{\Pi_{\parallel}(\omega,\boldsymbol{k})}{ \omega^2}+1\right)\,.
\end{equation}

In the linear response framework \req{eq:ohm}, the electromagnetic field still obeys the principle of superposition so the potential between two nuclei can be inferred simply from the potential of a single nucleus. 

We can perform the $\omega$ integration in \req{eq:potent} using the delta function in the definition of the external charge distribution, whose effect is to set $\omega = \boldsymbol{\beta_{\text{N}}}\cdot \boldsymbol{k}$ where $\boldsymbol{\beta}_N = \boldsymbol{v}_N/c$ is the nuclei velocity. Then we have
\begin{equation}\label{eq:potentk}
 \phi(t,\boldsymbol{x}) = Ze\int \frac{d^3\boldsymbol{k}}{(2\pi)^3} e^{ i\boldsymbol{k}\cdot(\boldsymbol{x}-\boldsymbol{\beta_{\text{N}}} t)} \frac{ e^{-\boldsymbol{k}^2\frac{R^2}{4}}}{\boldsymbol{k}^2\varepsilon_\parallel(-\boldsymbol{\beta_{\text{N}}} \cdot \boldsymbol{k},\boldsymbol{k}) }\,,
\end{equation}
where $R$ is the Gaussian radius parameter.
In nonrelativistic approximation the Lorentz factor $\gamma \approx 1$ and we use the convention $\varepsilon_\parallel(-\boldsymbol{\beta_{\text{N}}} \cdot \boldsymbol{k},\boldsymbol{k})$ used in~\cite{Montgomery:1970jpp,Stenflo:1973,Shukla:2002ppcf,Shukla:1996ccc} which gives the correct causality for the potential. This ensures that, without damping, the wakefield occurs behind the moving nucleus.

%%%%%%%%%%%%%%%%%%%%%%%%%%%%%%%%%%%%%%%%%%%%%%%%
\para{Reaction rate enhancement}
We use the same argument as \cite{Salpeter:1954nc} to find the enhancement factor due to damped-dynamic screening. The enhancement of a nuclear reaction process by screening is related to the WKB probability of tunneling through the Coulomb barrier
\begin{equation} \label{eq:penprob}
    P(E) = \exp{\left( - \frac{2\sqrt{2 \mu_r}}{\hbar c}\int_{R}^{r_c}dr \sqrt{U(r)-E}\right)}\,,
\end{equation}
often referred to as the penetration factor. $U(r)$ is the potential energy of the two colliding nuclei, $\mu_r$ is their reduced mass, $E$ is the relative energy of the collision, $R$ is the radius of the nucleus, and $r_c$ is the classical turning point. In the weak screening limit, the screening charge density varies on the scale of $\lambda_D$, which is here on the order of \AA ngstrom. The distance scales relevant for tunneling are between $R$ and $r_c$, which is on the order of $10\,$fm. This allows us to approximate the contribution to the potential energy from screening, $H(r)$ defined as
\begin{equation}
    H(r) \equiv U(r) - U_\text{vac}(r)\,,
\end{equation}
as constant over the integral in \req{eq:penprob} taking the value of \req{eq:pos_point_DDS} at the origin,
\begin{equation}
     H(0) = Z_1\phi_2(0) = Z_1 Z_2 \alpha \left(m_D - \frac{\beta_N m_D^2}{2 \kappa}\right)\,.
\end{equation}
Then, the screening effect reduces to a constant shift in the relative energy $E \rightarrow E+H(0)$. In this approximation, the enhancement to reaction rates can be represented by a single factor \cite{Salpeter:1954nc,Kravchuk:2014sps}
\begin{equation}\label{eq:DDSenhance}
   \mathcal{F} = \exp\left[\frac{H(0)}{T} \right]=\exp\left[\frac{Z_1 Z_2 \alpha}{T} \left(m_D - \frac{\beta_N m_D^2}{2 \kappa}\right)\right]\,.
\end{equation}
This result is only valid in the weak damping limit $\omega_p<\kappa$. The first term is the normal weak field screening result, and the second is the contribution of damped-dynamic screening. Due to the large damping rate in comparison to the Debye mass and the small velocities of nuclei \req{eq:vel} during BBN\index{Big-Bang!BBN}, the correction due to damped dynamic screening is small, changing $H(0)$ by $10^{-5}$. 
